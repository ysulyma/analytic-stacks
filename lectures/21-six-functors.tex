% !TeX root = ../AnalyticStacks.tex

\section{\ufs 6 functors for analytic stacks (Scholze)}

\url{https://www.youtube.com/watch?v=R5JNomeHjtI&list=PLx5f8IelFRgGmu6gmL-Kf_Rl_6Mm7juZO}
\renewcommand{\yt}[2]{\href{https://www.youtube.com/watch?v=R5JNomeHjtI&list=PLx5f8IelFRgGmu6gmL-Kf_Rl_6Mm7juZO&t=#1}{#2}}
\vspace{1em}

\begin{unfinished}{0:00}
The main thing I want to talk about today is the form of six funs for step. But before I get there, let me try to say something that Dustin kind of said last time and that I kind of also said last time, but we both didn't quite really do - to do the construction.

First, the construction of the norm on the SK fa. Conveniently, Dustin kind of already explained that we can assume we actually have all the roots of Q as well. And so then we want to define the algebra that corresponds to the locus where the absolute value of T is at most R and, or like the over convergent functions on this disc. As Dustin explained, the ex of Norm some tell you that this has to be like there's an if guess for what this might be. So this might be like this free algebra on the top element t or scale suitably so that the norm is equal to R, but that's not quite right, but it's almost right. But when you pass the over convergence thing, then it actually becomes exactly correct.

I will construct the norm for which the absolute value of Q will be a half. I want to take this here A and so I want to take a p a module on a top l element where I rescale T by some power of cube. And so this should be such that this definitely converges here. We want that we should secretly think that maybe the norm of t r, and then we want the norm of Q to the T * T this should be bigger than one. But okay, so the nor of Q is half, so a half T * R should be bigger than one. And now you can do the algebra. This means that 2 to T must be less than R, so you take the co liit of all t for which two to T is less than R.

This is a condition that if R is very large, say, then we should multiply by a large power of Q to make it somewhat small. Okay, was there a disagreement or just commiseration? Okay, thanks. So this has is the only possible choice is what you explained last time.

And so what we have to see is that these unimportant have all the required properties to Define well, first of all, you have to see this Al item and they plus the anog for G inverse and inin Define map. Okay, now you can replace T by T inverse and then the T supp Infinity.

Okay, so here are some ways to and I mean all of these are some kind of simp computations. Let me just check that the locus where it's at least R and the locus where it's bigger than equal to to S where R is less than S this would be empty, and for this you have to compute something like you take a and then join to T * T head and say noers. And then you can draw in T and the S time T head inverse. You want to show that such a thing is zero for if say, and one situation says R is less than one after rescaling, you can assume that some one is squeezed in between them. And so then you can assume that t is negative and S is also negative.

But in particular here, you have a map from two toping elements which are X goes to Q to the T3 and Y goes to Q F and say satisfies relation that if you multiply the two you get something the negative power of s, so Yus Q to the some axy is equal to zero where a is some positive number. But and in turn if you just remember the product, so then you also have this is an algebra over a Jo C Aus a * C, but this is precisely I mean a join Z head, this is precisely this projective generator that we always use, and then here you're doing oneus Q.
To the a * shift, and if a was equal to one, then we precisely declared this to be an isomorphism on the projective generator in the gas is ring structure. I also explained somewhere that the power doesn't really matter when you do this, so this implies the pr fractional power. So, this is actually zero, and this is really by the definition of the right. So, the minimal thing we really need to get this map is that we know this, which shouldn't intersect, and this really exactly comes down to the thing we enforce and the guesses. Then, there are some other things you have to check, and none of the computations is hard.

So, those, and then you have to check they cover that the guys like the they need to check the cover, but again for this, you just have to show that. Actually, I will probably come to this later in the course again, but I mean basically, you just have to, if you say we SCT to the Fine Line, then you just have to show that. Then, you expect that there is some they form a cover, and so they expect that there is some St like sequence like functions on the fine line just pooms going to functions on a dis and functions on functions on the opposite dis and then functions of the overlap, and for all of these things, you can just write down what they are and then just check that you get an exact sequence, and this implies the covering condition because it implies that the polinomial are generated by in a finary way by modules over over the ah, okay. So, you just have to check a cyclicity like the analog of data cyclicity, you just have to check theity.

I mean, this is enough to check the covering conditions. By the way, the reason that we really want to use this growth topology that we chose, I think there's another perspective here that makes some of these things easier, which is like instead of using this free thing on a topologically nil poent, it's like the free think where you make t topologically nil potent, which is not really like a subset. You could use the subset where te becomes gaseous like an analytic ring that's not induced, but they're all sandwiched between each other, so for the over convergent thing, it doesn't matter. Then, at least you have honest subsets, and then it's like kind of easier to check which subset. Oh, like I'm saying, there's a instead of using this thing I was calling D as your basic dis, you could instead use an analytic ring which is the analytic ring gotten by forcing the coordinate T to be gaseous in the sense of like one minus t * shift. What was D? Oh, D was just the spec of the pr this array which is this U, let me so I keep my notation keeps this notation, which to the line and then projective line. So, that's one thing you can consider. On the other hand, you can also Define the following funny analytic ring. So, you can take the polinomial and then make t guessers in the sense that you enforce 1 minus t * shift as usual on Project generator to be Aus. What is sh here? I mean, on this t-basis change to this ring, and so this is also an analytic R, and you can also take the aspect of this, and this is actually a Subspace of of P1 over a, and it's again a Subspace that will behave like like dis behave, so it will be training Norm, it will be contained in the locus, so it is contained in Long inverse of 01 and it contains Norm inverse of the op of open inter relation don't each other. Well, what one you can't say contain, and you could even consider a third version which would be the intersection, so you can also intersect these two. Then, something that Dustin mentioned last time is that actually, if you intersect these two, this map becomes an embedding, and also, this one becomes an embedding. You can also consider a Jo T head guess, and this is also true here. So, all of these three things, there are some slightly different versions of what a dis might be, and there also sandwich between like the over converging close unit disc and the open unit disc.

In particular, one way to verify the item potency without really sort of without doing any calculations is to see that the this da and this npec with the T gases are are sandwiched in between each other in this co-limit, and then it formally, since for one of the terms, you always have a monomorphism, it follows for the other term, you mean, there one thing you really have to check which is this that these two do not intersect. I think
Clever way to arrange the argument. You can avoid them, but yeah. Yeah, so if the radii are the same, they always intersect. Any disc around zero and one around infinity will intersect. Okay, for all versions of the story. Okay, yeah. I mean, it well, it depends on which base you're working over, okay, but over the universal base. Yeah. Okay, so all right. Um, yeah, so let me start by talking about six functors.

Let me first recall that on an analytic space, we had a functor that was taking an object $a$ to $d(a)$ and we also considered a different functor which was to look at the representable stable infinity line over $d$. This was a functor from the category of rings towards symmetric monoidal infinity categories. In particular, the commutative algebra structure here has insisted on some $T$ functor, and actually, there's also always in both cases they're closed, so there's an internal Hom, and it's also functorial in the ring. This gives you, in particular, for any map of rings, a base change functor, which geometrically is always $f^*$ and $f_*$.

These are four of the six functors, and then we defined this class of free maps, and for these, we had also a functor of lower shriek and upper shriek. Why does he move discretely? Because of the, the, yeah, something, a one-picture instead, yeah, another high quality. Camera usually, like, you don't see, right, exactly. So, we, right, um, so one convenient way to package all this data is in the notion of an abstract six-functor formalism, which I call the following.

Whenever you see any category $\mathcal{C}$ and $\mathcal{E}$ some class of morphisms, let me always assume that my infinity category is all finite limits and colimits, and here is a class of morphisms stable under composition and base change, where composition includes the empty composition. Then, we can define a new infinity category $\mathcal{C}_\mathcal{E}$, where the objects are these spans, and the composition is given by taking two such objects and following the fiber product. It's a symmetric monoidal structure, and there's an obvious way of taking a product of two morphisms.

We have this definition that, well, first of all, a three-functor formalism, where the three functors here that I'm talking about are $f_!$, $f_*$, and $f^!$ when $f$ is an $\mathcal{E}$ morphism. There's a correspondence between this category with the end of product and infinity categories with certain properties. In particular, I would like to discuss this example where these categories themselves are not presentable.

There's also a six-functor formalism, what is the just-Stein product, and the means in which direction there are two ways, like $d(x \times y)$, right? So, yeah, so "lights" here means that you only roughly speaking, you only have maps from $d(x) \times d(y)$ to $T(x)$, but they need not be equal; they just have maps. And if you apply this to the empty product, this also means that $d$ of the final object has a distinguished object.

Like, from the empty product here, so from the final category, so the point, there's a map to the drive category of the empty product, so $P$ of the empty set, and from there, from the empty set, you can go further to anything else by a pullback, so some particular, like all the, all the you have access here are pointed, which is giving you the unit objects, really.

All for point, I think it's of the empty, it's sorry, it's, yes, think it's than you. All right, and so, a this is, very difficult amount of data that's, it's not so clear how to construct it, but conveniently, like Le Jen proved some very difficult Comorbi recipe. Well, the recipe is extremely easy, the proof that it works is rather difficult, um, but that makes it extremely easy in practice to construct these things.

In particular, like for $C$ brings up and $E$ the streetable $NS$, um, sending a either to $D$ of $a$ or to $crl$ of $a$, um, which I'll probably discuss in a second, because, okay, let me just dove this one for now, um, uh, is an example.

Um, and so, uh, but now this and then some to the category of analytics text, and we would like to extend, uh, this, maybe to also some larger class of fre maps, well, not just I mean the fre maps of aine, certainly you want also base changes, but maybe much more than that, um.

And so, uh, yeah, so here's the, here, um, there is a minimal cost of morphism, um, unique, um, morms, the following property. You mean there is the smallest, because any class satisfying the condition contains this, yeah, any other class that satisfies the following properties will contain this class, yeah. So unique minimal is slightly different in some conventions, because it means that the pos it as a, as a unique minimal thing, it doesn't mean that it is the smallest, because so it's, but this depends on your notions, but it's slightly, according to some definition, better to say smallest than unique minimal, but because minimal doesn't imply that it is smallest in for parti other sets, unique minimal doesn't imply it smallest.

Um, uh, properties, so, uh, first of all, the the class, uh, a class of morphisms of maps of analytic rings define maps, that's something we want. Um, also we want that if we have just a disint union of fre m recable, so um, stable under unions in the following sense, so if, uh, you have some $f$ from some $XI$ to some common $y$, uh, all of them, then $F$ the dist Union of all the $F$ from the over $y$, so from the of all $x$ yint this able $i$, it's clear how you would define a l stre function in this case, I mean, the drive car of the dist a product, and then you just take the direct sum of all the streaks.

Um, it's this class is local on the target, uh, in the following sense, so if you have a morphism of analytics text, such that the exist some cover, sorry, I wrote $X$ before, so stick to this, um, that exist some $y$ Prime, subjecting onto $y$, for which, uh, $F$ Prime, which is a base change of of $f$, uh, which fime isn't class, uh, the class, say, uh, I'm not quite sure whether I need to explicitly say it, so I definitely want this class in the end to be stable under compeition and base change, and I'm not sure whether I should, whether this follows at some point, but always assumption.

Um, all right, so, uh, right, so this says that to check whether something is triable, you can, uh, localize on the Target, and by the way, I mean, if you wanted to find the lower streak funter for such a guy, it's actually quite clear you would do it, because you already have this ler streak fun for this one, and you know that streak fun they always come with base change, so to define lower streak on $y$ Prim on $y$, you can do it after this base change to $Y$ Prime, but then all the further
X. G here. The following properties that G already lies in each other and is of universal strength in the sense that I will explain in a second.

And the composite and the composite. MTH, so you should think that this is rather stacky, but now you've covered it by something more reasonable for which you already know that the composite map is is is needer then the original map. Maybe I should make some remark now.

Did you say what Universal strength descent here? I mean, strength descent after pullback to any fine, so because this map already has strength maps, you can always ask whether the like the r of X meaning strength maps to the r ex and all fiber products an equivalence, and I ask this not only over X but after pullback to any.

But Peter, doesn't it follow from the assumption that it's a cover in the gr topology? How do you know is very very true. Thanks, thank you Dustin.

Yeah, what is this? So, how do we know that this, how do we know that there is a, for those kind of, oh, it's the assumption that G is in this, well, G is an. No, if you mean no, no, no, but EA, we don't know that there is, the theorem say there is a minimal class, and wait, wait, maybe maybe I need the next condition to to justify it.

If you have an $m: X \to Y$ is an eer, and why is that fine, then I actually ask that you can always find a cover of $X$ by things which are over $Y$ which I, and fin, then all the fiber products are also. Because and all the, also, $X_i$ is covering over $X$, and then you need the full simplification. So, if I if I if my base is fine, then I make a strong restriction on anything that's possibly allowed, then I always ask that I can always further find such a shable, I mean such a yeah, shable $C$, where all the $X_i$s themselves, they lie in this original class of coverable maps.

Peter, don't you also need a variant of four with disjoint unions, like, or would this be strictly more general? Like, doesn't two say that I could always, no, but two doesn't say that $X_i$ to the disjoint union is in $E$. In other words, if you want to localize by using a family instead of one object, then you need some argument to reduce to the one object, and it seems convenient instead of two to have also the fact, in addition, I guess I suppose in some places that $X_i \to X \sqcup Y$ probably it should be in your class because it's just, yeah, but this is just because it's locally in the class $X \to X \sqcup Y$. Why is it in the class? Because it's locally in the class. Okay, okay, okay, okay, okay, okay. Just sorry, and then the empty to the something is always in the class because it's why, empty to anything is in the class, may you have to add it. No, this might be the empty collection. Ah, okay, okay. No, the empty to something $X$ is empty and then going to $Y$, this would be in the class, yeah, you can reduce to $Y$ being $\emptyset$ by, by, yes, and then it is in $E$. Okay, okay, okay, okay, okay.

Okay, all right, I hope I'm not screwing up, but, right, so to, you see this Universal strength descent after pullback to fine for such a map, then, after pullback to fine, you're in the situation of this one, but then you always know that you can refine by cover by strength, find for which you know strength. And there's one last condition which is really one we're interested in, that the six fun or the three fun formalism than you, Neely, from CE, I mean, from analytic rings and one we have to analytic STs uniquely in the space. This you have to put it in ER to the prev, yeah, so for some of these Surs I'm implicitly using that, I mean, okay, let say that. Yeah, so this follows from results, from exra, I mean, this has nothing to do with a specific six dealm, this just an is all about six formalisms, and I've given some account of this in my notes on six funes using Min different setup, but I think the same argument works here. All right, so that's an extremely

Because it gives us this extremely structured formalism of six operations. Whether I say three or six, it doesn't really matter. Here, let me say six, because for this, these are all presentable categories, and the fun $f$ is all co-limit. Then, as Dustin explained, we always have the right joints anyway.

So, you have these six funs. Now, for some classes of stacks, and if you want to check whether - and it's in practice extremely easy to check whether any map - um, has or, in practice, basically all morphisms have three funs. And using some of these criteria that you can localize on $S$ and $Target$, it's also usually rather easy to check that this is the case. So, you can first simplify the target as much as you want by localizing, and then you just have to find some kind of presentation of your guy which makes it simpler.

All right, let me say a few words about how it's done. It's kind of hard to explicitly describe, but it's easy in practice. You can check that the given map belongs to this class. So, let me just give an example to illustrate this a little bit. What happens? This must be covered by, but then is it enough? No, then you iterate, you do that again and again and again. You iterate, working locally on the source and the target, and you have to iterate and transform this, because once you get new shakable maps, then working locally, you get new things that can be covered by those.

Let's assume you're in a situation where you have some $EX, Y$, where maybe the $Y$ is something simple, just a point, and the $X$ is something spey. But then you can find the cover where, for this, you already have street maps. Then, the $D$ of $X$ because of the cover, it's a limit of the street maps of $EX$ and $Er$, but this is also the same thing as again the co-limit along the lower street funs. Now, taking in $PRL$, these are the upper street fun, but you have lower street fun in the opposite direction, and for all of these things, you already have lower street funs, and they are compatible with the transition maps. Thus, because it's a co-limit and this guy is presentable, all the $M$ preserving, you get the unique $L$ street fun.

Let me actually give an even more specific example. Let's consider the interval mapping to a point. This is not something which happens in condensed sets, but as we said, condensed sets can be embedded into analytic stacks. As $s$ explained last time, there is this descendible cover of this by a contour set, and this map here is stable, because when you face change to some cover here, so you can check it locally. And if I base change this to the contour set, then the fiber product here becomes itself some live profile contour, and this map here is just a flat map, I mean, it corresponds to a faithfully flat map of fine things, and so this here is an $EA$. And also, this map here is already in $E$. And so, up prior in our world, we've only defined street maps here, and we didn't define what the street map would be for the general compact Hausdorff space. But then, it turns out that in fact, at $L^{\star}$, one can observe that actually $F^{\infty}$ happens to also be an $L^{\star}$, as you would actually expect, because this looks like a proper map.

Let me do yet another example. This way, you can automatically extend from profile sets to compact Hausdorff spaces, at least the finite dimensional ones. It also gives an extension to all compact Hausdorff, no, let me not say that, to the finite dimensional ones. But then, you also have locally compact Hausdorff guys, so, for example, you have the reals, and now maybe this is a point where Dustin said I should have taken this $J$ somewhere. Oh, there it's okay.

This is covered by the disjoint union over all $n$ of the interval from $-n$ to $n$, and because, why is it subjective? Because, from the fun points, as condensed said, this is a union of these things. So, this is definitely subjective, it's also, and one

This argument will generally work for any kind of locally compact, supportable, and five-dimensional thing. We get some FL streak, and you can actually check that it is a usual streak. You can observe the comp, and why? I mean, you can look at this definition here. The drive category of R can be written as a Coit of the graph carries on these minus n NS along the lower street maps. To describe the low streak, it just means the col presing fun, so it's enough to describe what it does on all these things, but here it is just chology which is compact for chology. This very abstract procedure actually recovers a lot of like upior geometric intuition about what a low-frequency function should be of.

When I discuss the t a look curve, I was mentioning that some should be proper and that it should be smooth. Let me say a few words about smoothness. There is an inductive definition, a morphism, and why? I actually wanted to be in the, and then let me say proper. Maybe weekly proper, it's not, it doesn't precisely match one something I call proper in the context of an abstract. Let me just say proper, and maybe Dustin will complain that he wants this for something more specific.

There's a question in the chat from Peter asking if we have any explicit description of f uper shriek of the unit object for FX to a point in the category analytics tax or at least in some special cases. This is a very good question. Let me refer to the discussion of Smooths coming up in just a second. If you look at like f low streak in these examples here, it was just usual f low streak, and so then the f uper streak is also usual dualizing complex. In virtually any situation where you already have a dualizing complex, our up streak will agree with what you say is dualizing complexes. Except, one thing you have to be careful about is that if you embed schemes into analytic stats in this way, in this kind of with a tri analytic ring structure and all maps become proper as we said, the f l streak is always f lower star, and the upper streak is then this funny r joint of a lower star that is sometimes considered but is not so well behaved in some ways. If you want the usual dualizing complex on a scheme of finite type or something like this, you should rather use this other embedding using relative s with this caveat.

Let me try to describe find prop morphisms. One case that I definitely want to be proper is that it's relatively representable on f lines, so meaning that after you pull back to any fine becomes fine, uh, repres FES. I'm sorry, the following and proper you want toor for all Prim y, y um X Prime, which is the final product F and X Prime to Y Prime is proper in the sense that we used before for F, so that it has structure. Sorry, Peter, we were just discussing a remark that in the setting of aine analytics, Stacks the prop, you know, this the proper Maps satisfied descent for the gro and de topology. Someone pointed out that whether I say, well, they don't, right, because they have some connectivity issue. That's true, but I think I pass once more to diag or something I can avoid this. So, let me make this, so therefore, it's not the same, not the same as as so as being locally no. Yeah, we should, so we really shouldn't ask for all why Prime mapping to. Yeah, yeah, they will be a second part of this definition. I would hope that, was the second part, it wouldn't matter what I said here, but okay, I didn't carefully check it, so let me maybe not. However, one thing that this implies is that if this so happens in case one, then you already know that f streak has a canonical equivalence with f star, that's given to you because in for these Maps, we Define the f streak to be the f star, then because you can Constructors ISM locally d canonically, get it. So basically, the class of program that's be the one for which a floor speak is a floor.

But, but, say it like this: it's not a condition, but I want to say something which is really just a condition. And one way to do this is to, oh yeah, like for these guys here have such an identification. And in general, you can ask a diagonal $\Delta F$ from $X$ to $X^*$. This is why I'm saying it's an inductive definition.

So, this might be proper in the sense of one, or it might also be proper in the sense of two, in which case it passes the diagonal again, then hope those are proper. And once the diagonal is proper, you get an identification between $\Delta FL^{str}$ and $\Delta FL^{star}$. And this allows you to construct a canonical comparison map from a floor $F^{str}$ to a floor $F^{star}$. Maybe recalling the second where it comes from and induced, does it follow after? And actually, you might think you want to also ask this after any base change, it actually follows. And actually, something even weaker of $F$ is $F$, speak of the unit, and then solve it after the new. So, does this lead to a transfinite process? Namely, suppose you have, you can use $\mathcal{F}$ many times, and then you can say, no, I want us to determinately find many things. Yeah, there's no way to get it to go, let us say, suppose that you have a cover $Y'$ to $Y$ by half. Ah, in this case, one is a terminating condition, and the other is just you can apply $\mathcal{F}$ many times, but at one point, you should run into the one condition.

Okay, so the joint union of countably many guys $X_i$ to $Y_i$, which are proper in the $\mathcal{Y}$ sense, will not be proper if when the sense goes to infinity. No, no, sorry again. When I said there is a cover by such, I was allowing that it's a cover by a collection of such. No, no, no, but I'm asking, suppose $X_i$ to $Y_i$ is proper, then the disjoint union of $X_i$ to the disjoint union of $Y_i$ is not proper. And I think it follows, no, because if you have a sense one because of, ah, because. Okay, I see what you're saying, but maybe you could just take this as the terminating thing as a definition, but, okay, so maybe it's not proper in my sense right now.

Yes, okay, what's the problem? It's not the problem. There's a variant where you, because this is just simple induction, yeah, without going to Omega, yeah. So, suppose you have $X$. So, yeah, so maybe I should really make this, I should say the proper is first defined proper, both of these things are like $\mathcal{QQS}$, because I think this should be the case. And then, in general, if it's true after pullback $\mathcal{J}Y F$ or something. No, no, because suppose you have $\mathcal{QCQ}$ things $X_i$ to $Y_i$ which are proper in the, but in higher and higher senses, you need higher and higher, this will not. All right, but I'm not exactly sure in this theory how much you can reduce to $\mathcal{QCQ}$, because anything is a cover, but this. Yeah, the the equivalence $\mathcal{R}$ itself is not $\mathcal{QQS}$ in general, so it doesn't, you don't have good reduction, because I, the proper case, these things I think should be $\mathcal{QQS}$ because here the basic ones are $\mathcal{I CPS}$, and then the diagonal is. And I think from this you can deduce that $F$ must be $\mathcal{QU}$ compa, at least because it implies that the floor is talking to $\mathcal{SCHORE}$ limits. So, these things should be $\mathcal{QQS}$ in a very strong sense, these maps, all the maps of 

Itself, then it's Asis funs and it's true of Base change, and again this is something you can find in my notes on six fun EX for. And so, in particular with this definition of what proper is, then, yes, so then this map here is actually proper in this non-generalized sense to any careful. However, that if you have some kind of H Cube mapping to a point, then I'm not sure whether it's actually in. DUS, do you know? In any case, it's not proper because there is some the lower star fun doesn't actually con with col because of some infinite ises.

Right, so you can Define this things, and you can also Define smooth. So, for smooth, for smooth system, I want that the upper streak is basically the same as the paper star, but in this case, you don't actually usually expect isomorph on the nose. You rather expect that there's a same some twist, and also for smooth Maps, you would never assume that this is a condition that's stable under passing to diagonals.

So, the definition of SMU is, for this reason, is a bit different. And here, I us call the things loic, and this now really matches the abstract thing defined for any six fun form is if.

And now, I really want to ask, after there any base, sorry, this CL maps on Bas change? So, there's the natural transformation from pullback Stander with f upper streak of the unit, what upper streak, and of unit which will be some object of X, is actually an inversal object tender and commute to space change. And so, I mean, you can form this F upper spe to dualizing complex for x or y, but you can also do it for a base change, but it should also be an invertible object, and I ask that the formation actually commute to space change. Again, you can somewhat reduce the amount of data you actually have to check.

Let me rather just briefly say where the center comes from. And so, again, I'm mapping to write a joint fun. So, what you try should try to do is in produce a m from the corresponding L stre, but then this is precisely a projection form your situation here, pull back of something. And so, then this thing, but then, it's just by a junction. So, this class is St on the base change, uh, composition. I mean, also problem St in the base change, uh, composition. It's in some sense local on the source if you smooth cover which smooth, and okay.

So, there are some ways like that to be able to talk about some Notions of proper, smooth maps, and I mean, they some have the expected consequences on the rough categories, and so you can, yeah, execute many arguments as usual. So, I just want to briefly now come back to ttic curve.

All right, so, a is now again this guest displacer, and so now we have the norm. And so, we can def find the analytic over that's a of. So, we have multiplication by two, all right. Now, we can make some assertions.

First of all, this St here, this is proper. Well, it's not representable in Fes, clearly, but the diagonal is, because it's just a diagonal P1 f, and you can check that p stre and p star the unit just agree, and this is actually just the I don't know, you can check that in algebraic geometry, proper, and it's also smooth.

And for this one, should I mean, I you can just really just check that directly, it's not hard. You can also do the following argument that, whenever you have like the morphism of six fun formalisms, so, for example, like schemes mapping to analytic Stacks, then any such morm of six fun fors will always preserve comically smooth Maps, because there's some kind of diagrammatic way of encoding it in the six fun formalism. And so, the smoothness that you somehow know in algebraic geometry automatically implies it, the algebraic version implies anal question.

This J here, it's actually also, I mean, it's also shable, and it's actually somehow, I mean, I didn't Define it, but I could have defined also in an open imersion. So, this just means that, see identity, and again, this is something you can basically reduce to just zero infinity and like the open guy.
The close guy, very some everything produces the usual six funs on locally compact $\mathcal{A}$ spaces. These are, in particular, also like $\mathcal{J}$ lower streak, so $\mathcal{J}$ upper streak in this case. $\mathcal{A}$ jerar particular is also, so this guy is not proper, just like this, I mean it's not $\mathcal{F}$ compa. Okay, but now again, we can take since $\mathcal{T}$ the curve, which is quotient. Ah, but one thing that is proper right, so the proper Maps they satisfy the two out of spe property because they were Som defined by passing, and this map here is actually proper. And so it also follows that if you pass to this and $\mathcal{G}M$, then it's still proper over zero.

Infinity, there's a question in the chat, Peter. Yeah, are open immersions of analytics Stacks representable? Absolutely not representable by $\mathcal{F}$. I mean if they were represent by $\mathcal{F}$, in particular, would have to be qua complex, but this $\mathcal{M}$ is very much not quic complex, so we get much more General open Emer.

Right, but now the $\mathcal{S}$ tatic curve, it's a of analytic $\mathcal{Z}$, and so this to $\mathcal{Z}$ Infinity mod rescaling by po of two, which is a circle. And so like the base change to $\mathcal{Z}$ infin you see that this map is actually proper, but if you project as one to the point, this is also proper. And so you see that the $\mathcal{T}$tic curve is proper or the point, the point is now like everything.

So you see that this constru gives you an $\mathcal{A}$ntic $\mathcal{S}$t that's proper, but also, I mean this map here is locally split, and so this means also this is here Lo local is the same as this $\mathcal{A}$nalytic $\mathcal{G}M$ which was smoo, so also the sky here isical smoo.

I find it a bit remarkable how you can like combine intuition of about the properness of compa spaces with some really algebraic geometry kind of properness and so on, and it all works really well.

Right, so, yeah, also it comes with a section as a unit section $\mathcal{I}$ given by one and $\mathcal{G}M$.

So, there some of the claim that this $\mathcal{E}$qa curve, which is $\mathcal{A}$nalytic $\mathcal{S}$t over a limit $\mathcal{A}$lun from $\mathcal{S}$ches schemes over the underlying $\mathcal{N}$ice $\mathcal{N}$ essential image of the fully faceful embeding from schemes over the underlying ring and ring ration.

So, to execute this argument in a really nice way, I should prove a little bit more about this ring $\mathcal{A}$ here. In particular, classify like the dualizable objects in $\mathcal{D}$ of $\mathcal{A}$ that they are just perfect complex, which I think is true. Let me do it a little bit more by hand because it's fun.

Modle and I asked that it SL on the essential image of H. Me what, sorry, I because it might be derived. In my situation, I'm in everything, there's no derived structure, so it doesn't really matter.

I definitely want that there's noology and, um, assum have this situation, um, and it's not clear the difference between Poli. I mean, this, it's just, I'm just trying to say you have an an line bundle in some approximate sense. So, for example, I want that it's Global, it or repl by some tens power, it's globally generated, uh, and there's no higher chology of tens powers which you can arrange. And I also want that the global sections are not something, some just to the SC module, because I want an algebraic schem in the end. So then we can define something algebraic to be just the project of the. Is this AR GMA? Is it just H zero, or it have a positive? Yeah, I mean, if some X is actually derived, it could have positive stuff, let's not worry so much about it. So you can Define this thing here is a direct scheme over, and you automatically get a comparison that, uh, from this algebraic guy, analytic fied, and, because, uh, globally generate, you can actually check that this is really a qu comp guy.

Okay, this is the stand, but it's not enough to, it's not L, to I hope. Um, so something one can say in the situation is that, uh, so anything comes from algebraic geometry, as we said, it's automatically improper, so M actually automatically improper. Meod, but also, uh, a simple observation is that in this abstract setting, automatically, it will be the case that aest of the unit is the unit structure, she is a structure, she, um, why, because you can check this, uh, we can check after the pullback, uh, to f, f, x of x, uh, but there on global sections, because I'm the fine guy, like of CH is just of the global section, because it's e, um, uh, but, uh, but this inclusion here is really just given by algebraically inverting something, and so then this a star here is then just some some filter po Lim of inverting some function f, so this actually just reduces to, uh, to to the isor on sections that we know. I mean, so this is like the plus of some f, where f is some fun, some Global section of, and then, uh, you're really just getting Co liit over multiplication by F of global sections, sorry, the Al, I mean, either on the algebraic guy or on the on X, but some the AL guy was ripped, so that like the go section of the line bundle are the same on X and the albra what K, the co liit over K transition modication by some section of l, f section of L, the K or n, maybe I don't know. Uh, F was an unfortunate Choice, actually, because that was your map from X, oh, sorry, yeah, thanks, thanks, sorry, G, did I use G? Let, all right.

So, uh, right, so you have a proper map of analytics F, where FL St unit is unit, uh, this actually implies that it's a s active net, because, like, yeah, this this is a SE true after any pullback, because proper St stre any pullback, and so this means that after pullback to any Fon, you can get the algebra, uh, in terms of the image, and so this actually means that f is subjective. All right, and so what does it now take to show that F isomorphism, uh, see, I mean, this generality it won't automatic the case that it's an isomorphism, but, uh, to check that isomorphism, uh, it now suffices to show that the diagonal of f is an isomorphism, right, because if you have any subject M of sheets, it's Isis only diagonal Isis, uh, but this, the diagonal of X here, he want just this a pull back here, uh, actually this also implies something else I should have said, also implies that the pullback map from the garage category of this guy is actually inly Faithfully into X, because, um, because if I look at low star upper star, then by the proection formula, this is just

So, if it so happens that there is a map $\mathcal{D}$ of $X$ is $\mathcal{F}$ on, and $\Delta x$ of one is in the image of $f * \mathcal{F}$, we discuss to show that to give an example of a proper map, explain the $\mathcal{C}ur$ algebra.

Okay, so I claim that I only need to check that this diagram gives a diagonal commutes. But in practice, like the diagonal, I don't know, some kind of complex or something, this, this, some risky $\mathcal{C}los$ immersion. In particular, some kind of $f$ math, and the $\mathcal{X}$ $\mathcal{S}$ of one is really just some coherent $\mathcal{S}he$.

I claim that it's actually surprising to see that this lies on the image $\mathcal{F}$ time. Then we win because of it $\mathcal{L}IC$ image of times upper star, and it's actually the same thing as applying $f$ $f$ lower star and then upper star again. But if you apply the low stars and you run through this diagram, you see that it's actually the same thing as a pullback of, which actually implies that it's equal to $\mathcal{T}imes$ of $\mathcal{D}el$ algebraic.

But then this means that, yeah, but then this means the some of $\mathcal{P}$ diagram. All right, so the upshot is, if you want to prove that some such analytic space, proper analytic space is algebraic, you have to produce an ample line bundle in that sense over there. Then you automatically get this kind of comparison $\mathcal{M}$ to some algebraic something algebraic.

The only thing that remains to check is to see that the structure sheet of the diagonal lies in the subcategory that comes from this algebraic.

My time is up, but these are things that are simple to check in the case of static $\mathcal{C}$. So, you can, for this $\mathcal{L}$, you just take the $\mathcal{A}$ mod given by the zero section, and then you can actually just compute what all the global sections are just inductively. The global sections of the structure sheets you can just compute the $\mathcal{A}R$ easily, and then the rest you just acquire copies of just the zero section in addition, which are just your base ring. So, this is okay, and well, and for the diagonal, you can actually use that this is a group, so you can move the diagonal to the $\mathcal{Z}$ zero section, and the zero section is easy.

All right, my time's up, so let me stop here. I want to clarify just from the Viewpoint of General scheme Theory some point here. You consider $\mathcal{P}ro$ of rings which are not assumed to be finally generated, that is, suppose that we don't have a derived ring, we just have the usual ring. For example, the graded ring of $\mathcal{H}$ zero on some scheme of powers of a line bundle. So, we just interpret what you said for normal schemes for Simplicity, because I'm not sure about.

Now, when you look at the notion of proper, like $\mathcal{R}otend$ defines finite type, separated, universally close, but of course, you people thought about the case of non-finitely generated. So, you can have a universally closed, maybe separated, but then this turns out to be a projective limit in the $\mathcal{Q}C$ $\mathcal{Q}S$ case, a projective limit of usual proper things. So, my question is, suppose you have a $\mathcal{P}$ of the type that you consider. What we know is that it's covered by Finly many opens, a standard basic principle opens, whatever. But is it the case that it is proper well in this universally closed sense?

The problem is that, of course, in the $\mathcal{T}$ litic $\mathcal{C}$, I think the algebra will be finitely generated, but I allow the case that the graded

Let's specialize the notion of properness I defined here. In that setting, you get the usual notion of properness. Okay, because it implies COMP. Okay, all right.

\end{unfinished}