% !TeX root = ../AnalyticStacks.tex

\section{\ufs !-descent continued (Clausen)}

\url{https://www.youtube.com/watch?v=rN_iM7Z8vdE&list=PLx5f8IelFRgGmu6gmL-Kf_Rl_6Mm7juZO}
\renewcommand{\yt}[2]{\href{https://www.youtube.com/watch?v=rN_iM7Z8vdE&list=PLx5f8IelFRgGmu6gmL-Kf_Rl_6Mm7juZO&t=#1}{#2}}
\vspace{1em}

\begin{unfinished}{0:00}
  Okay, let's get started. I will continue the discussion that Peter started last time on !-descent.

Let me recall the setup. We have analytic rings, and we are going to build some geometric objects on them, somewhat in the model of scheme theory. The first thing we do is define what the affine things are - we formally define them to be the opposite category of analytic rings. An object in this category, let's call it $X$, corresponds to an analytic ring, which we can denote as $\mathcal{O}_X, \mathcal{D}_X$. An analytic ring consists of an animated condensed ring and a certain full subcategory of modules over that, in the full derived category of modules over that ring.

I will denote the first coordinate as $\mathcal{O}$ of $X$, and the second coordinate as $\mathcal{D}$ of $X$ in this notation. Here, $X$ is really just a formal symbol.

In the derived context, there is a way to define the full derived category. For some purposes, it's easier to just restrict to the non-negative part, as the two formalisms are equivalent - you can go back and forth between them. But for the discussion today, it's actually much better to consider the full derived category as the primary object. The reason has to do with a descent phenomenon.

It turns out that the full derived category will have this !-descent property, but the non-negative part will not. In other words, for an analytic ring, we have a nice $t$-structure on this category, but for a general analytic stack, there won't be a $t$-structure on the derived category of the global object, which is obtained by gluing these local derived categories.

For an analytic ring, the category $\mathcal{D}_X$ is by definition a full subcategory of the derived category of $\mathcal{O}_X$-modules. This has a natural $t$-structure, and the claim is that this $t$-structure is detected all the way down in the condensed Abelian groups, and it induces a $t$-structure here as well.

We singled out the notion of a \'shable\' map, which means it can be factored as an open immersion followed by a proper map. For a proper map $f: X' \to X$, this means that an element $m$ in $\mathcal{D}_{X'}$ lies in $\mathcal{D}_X$ if and only if its image in $\mathcal{D}_Y$ lies in $\mathcal{D}_Y$, where $Y$ is the target of the map. In other words, you just take the class of complete modules on the analytic ring $X'$ and inherit the notion of completeness up to $X$.

We claimed that the composition of such maps is also \'shable\', but I forgot the details now.

Peter claimed it last time, I believe. But he didn't give the argument. It's quite straightforward, though. 

Let me continue the discussion. The open immersion means that the functor $J_!$ always exists. This should be a localization, and the kernel should be just modules over some idempotent algebra. It will necessarily be a compact object because, by the definition of a map of analytic rings, the right adjoint of this commutes with colimits.

This is just in the pure category theory sense. There is a notion of localizing by a multiplicative set in usual categories. But these are presentable categories, and the good definition is what Peter said: the right adjoint should be a fully faithful functor.

We already know, as part of the discussion of analytic rings, that the right adjoint exists. So this implies that there is a left adjoint.

Let me continue the discussion. We had a claim or a theorem. The remark is that for a proper map $\pi$, the right adjoint $\pi_*$ is nice: it commutes with colimits, satisfies the projection formula (i.e., it's $\mathcal{D}(Y)$-linear), and commutes with base change.

For an open immersion $J$, there exists a left adjoint $J^!$ or $J^\natural$, which has similar nice properties.

The theorem discussed last time was that there exists a six-functor formalism on $\mathcal{D}$ such that the class of "shabable" maps $f: X \to Y$ is characterized by having the property that for $f$ proper, $f_! = f_*$, and for $J$ an open immersion, $J^!$ is the left adjoint.

This class of shabable maps has good closure properties: it is closed under composition and base change.


If $F_1: X_1 \to Y$, $F_2: X_2 \to Y$, ..., $F_n: X_n \to Y$ are maps with the same target, then $f_i$ is $\mathsf{shtuka}$-able for all $i$ if and only if the map $F$ from the disjoint union over $i$ of $X_i$ to $Y$ is $\mathsf{shtuka}$-able. 

And these disjoint unions in this category, those finite disjoint unions in this category, they correspond to finite products in this category of analytic rings, and they're kind of just naively defined coordinate-wise.

So for finite finite coproducts, maybe I'll do just a reminder of some example of this six functor formalism. Let's take $Y$ to be $\mathsf{Spec}$ of the solid $\mathbb{Z}$-theory, and let's take $X$ to be $\mathsf{Spec}$ of the solid $\mathbb{Z}[T]$-theory, with the natural map from $X$ to $Y$. Then we can take $X'$ to be $\mathsf{Spec}$ of the solid $\mathbb{Z}[T, T^{-1}]$-theory. 

So this is a proper map, and this is an open immersion. And the complementary algebra is equal to this $\mathbb{Z}[[T]]^{-1}$. And then, for example, $j_! (\mathcal{O}_X)$ is this two-term complex. And then that's also the formula for $\pi_! \pi_*$, which is just the forgetful functor, forgetting that you have a module structure over $\mathbb{Z}[T]$. Intuitively speaking, this is functions on the affine line, and this is functions localized near infinity, localized near the missing point. And so this is like functions on the affine line which vanish near infinity, so to speak, because you're taking a fiber. So that's kind of compactly supported cohomology of the structure sheaf on the affine line, so to speak.

And this is what six functor formalisms are supposed to be doing in general: they're supposed to be specifying some notion of compactly supported cohomology, relative compactly supported cohomology, which behaves well in families. The key property of this $j_!$ is that it commutes with base change in complete generality.

So the algebraic objects are proper, even if you're dealing with something like the affine line, which is not proper in traditional algebraic geometry, but it's kind of compensated by the existence of this solid theory, where you have a new version of the affine line, the solid affine line, which is not proper anymore.

Okay, so a $\mathsf{shtuka}$-able map satisfies Čech descent. If you take $\mathcal{D}_Y$, $\mathcal{D}_X$, and $\mathcal{D}_{X \times_Y X}$, and then continue like this, using the $f_!$ functors to define this diagram, this induces the $\check{\mathcal{C}}$ limit of the $\mathcal{D}_X$ to the $\mathcal{D}_{X/Y}$. The naive thing would be to consider the star descent, you
This condition, what about the composition of Shri? Does it involve interchanging proper and open, or is it quite straightforward? The essential reason it's straightforward is that there's a canonical candidate for the $X'$ in the factorization. Your $X$ contains this algebra $\mathcal{O}_X$, the structure sheaf, and then you can just take $X'$ to be $\mathcal{O}_X$ and then $\mathcal{D}(X')$ should just be the full subcategory of $\mathcal{D}(\mathcal{O}_X)$ consisting of things whose image in $\mathcal{D}(\mathcal{O}_Y)$ lands in $\mathcal{D}(Y)$. 

So you have an open immersion and then a proper map, and then you claim that this, in étale spaces, the... Okay, so you use the other... Yes, okay, and you check that this is canonical.

The second goal for today is to be able to give examples of Čech covers, so that you can then apply these descent results. I don't remember if it was defined in the previous talk I saw, but what the six functor formalism is exactly. I think he referred to Lurie, so I'm going to actually discuss it today, because the precise way it's encoded will help give some arguments. So it's in fact the next topic that I'm going to turn to.

The existence of this six functor formalism on $\mathbf{S}$ follows rather immediately from work of Lurie, as reinterpreted by Gaitsgory and Rozenblyum. Following Gaitsgory, Rozenblyum, and Lurie, Gaitsgory encodes the six functor formalism in terms of span categories. Let me now set this up.

Suppose we are given an $\infty$-category $C$ with all pullbacks, and a class of maps $S$ in $C$ stable under composition and pullback. Then we get a span category $\operatorname{Span}_{C,S}$. The idea is that this is going to encode the six functor formalism, where you have shriek maps defined for maps that lie in $S$, and you have star maps like upper star defined for any map whatsoever.

This has objects the same as in $C$, but maps $X \to Y$ given by diagrams $X \leftarrow M \to Y$, where the right-hand map lies in $S$. The composition is given by pullback. A two-functor formalism on $C$ with respect to $S$ is just a functor from $\operatorname{Span}_{C,S}$ to some category of categories.


Suppose you are given a functor $F_!$ from the derived category $\mathcal{D}(X)$ to the derived category $\mathcal{D}(Y)$. Here, this functor $F_!$ is giving you a functor $F^*$ from $\mathcal{D}(X)$ to $\mathcal{D}(Y)$. These functors are stable under composition and include having the identity.

Yes, it does. This is what happens when you compose an empty set's worth of composable maps. If you have a pair of composable maps, the functoriality amounts to the base change formula for the $F_!$ functors. The fact that $F_!$ commutes with $F^*$ when you have a Cartesian square in your category.

So, a two-functor formalism versus a six-functor formalism, but two is the essence of six in this case. The other functors are: a base change formula, compatibility of $F^*$ under composition, and it also encodes compatibilities between the base change formulas and the compositions that you have on these things, so there's higher-order data implicit in this.

Provided this admits a right adjoint, then you get the right adjoint as well. And the remaining functors are some tensor product operation you have defined on each of these, and some adjoint to it, some internal Hom. If you have this and it satisfies a certain property, then it has some adjoint.

To encode this tensor product and the expected interaction of it with the functors here, namely projection formulas, we use a symmetric monoidal structure on this span category, induced by the Cartesian product in $C$. We need to assume that $C$ has a terminal object and pullbacks, so that we have products.

This symmetric monoidal structure on the span category is not the Cartesian product in the category anymore, it's just some symmetric monoidal structure. We then request that your functor $\mathcal{D}$ from this span category be lax symmetric monoidal with respect to that tensor product we just defined and the symmetric structure here given by the Cartesian product of categories.

The basic data is that you have $\mathcal{D}(X) \otimes \mathcal{D}(Y)$ should map to $\mathcal{D}(X \times Y)$, plus some compatibilities which are conveniently encoded in this being a lax symmetric monoidal functor. In particular, $\mathcal{D}(X) \to \mathcal{D}(X)$ using the diagonal, and then you can probably reconstruct this by pulling back over $Y$. But then one can wonder why you need the product in the category---it's not enough to specify $\mathcal{D}(X) \to \mathcal{D}(X)$ for every $X$ in some coherent way, you need to encode the compatibility with the $F_!$ functors somehow.


Object, oh boy, yeah. I'm sure it comes for free from, so it lacks a unit. I mean, there should be some unit.

Yeah, yeah. Okay, so I have to advance this story a bit. I said that if you just have these three functors satisfying certain properties, then you automatically get all six functors. And there's a very convenient way to organize the passage from 3 to 6, and that's a bit of higher categorical magic defined by Lurie. This is Lurie's magic category called PRL.

I have to say a little bit about this and how it works. Here, the objects are the presentable $\infty$-categories. "Presentable" means you have all small colimits and you're in some sense controlled by a small subcategory, in the sense that there's a small subcategory such that the whole thing is gotten by formally adjoining some sufficiently filtered colimits. So you should be $\kappa$-compactly generated for some $\kappa$. The morphisms are the colimit-preserving functors, and that's equivalent to admitting a right adjoint.

So let me tell you a bit more about this magic category. PRL has all small limits, and the forgetful functor PRL to the $\infty$-category of categories preserves them. That's nice--limits in this category exist and are completely naive. But also, PRL has all small colimits, and this is the really remarkable thing: you can also access these colimits in a completely naive way. There is a constant version of the forgetful functor where you take a category C and send it to C, but then you take a functor from C to D and send it to its right adjoint. You can make this into an honest functor, and this functor preserves colimits, which translates to limits in the $\infty$-category of categories.

So also the colimits in Lurie's magic category are just calculated as naive limits of the underlying categories, but with respect to passing to the right adjoints of the functors in your diagram. This is quite magical, because in the $\infty$-category of categories, colimits can be very difficult to calculate. But this makes it easy.

In particular, the condition of shriek descent for maps of fiberable maps of abelian groups is actually the same thing as cocartesian descent for the lower shriek functors in PRL, which is a much more convenient way of thinking about it.

There's more magic in PRL--there's also a symmetric monoidal structure, which is maybe the main theorem in Lurie's book. It's a tensor product characterized by a universal property, just like you would expect of a tensor product. Maps in PRL from C tensor D to E are the same thing as functors from the product that commute with colimits in each variable separately. And this tensor product on PRL also commutes with colimits in each variable separately, so you get a very nicely behaved tensor product on this category.

In fact, the internal Hom from C to D is just the $\infty$-category of colimit-preserving functors from C to D.

Of course, we only remember isomorphisms. So you can somehow recover the full mapping category by using this tensor structure. That's one way of recovering it at least.

In principle, PRL should be considered as an Infinity 2-category because there's a whole category of maps between two objects. But you don't really need to remember the Infinity 2-categorical structure because it's just the internal equivalence or isomorphism in this theory.

There was a reference to a paper on Infinity 2 in some places. So do they define Infinity categories and all of this in general? I haven't kept up with that literature; so far, I've been fine with just Infinity 1.

Okay, I want to continue a little bit more. If you have a PRL, it is now a tensor category, a category with a symmetric monoidal structure. You can ask what is a commutative algebra in this tensor category, meaning an object equipped with some symmetric multiplication and higher coherences about it being sufficiently commutative. What this means is that C is a symmetric monoidal presentable Infinity category, and the tensor product on C commutes with colimits in each variable. Some people call this a presentably symmetric monoidal Infinity category.

A basic example for us would be D of X for an algebraic variety X. You can then consider modules over C in PRL, which are certain presentable Infinity categories tensored over C. This allows you to say they are enriched over C in some sense. You have a C-internal hom from M to N, which is characterized by a universal property relating maps from C tensor M to N.

The category of modules over an algebra R1 tensor D R2 should be the same as the D of the tensor product, unless you're in characteristic 0. In that case, you get something bigger. But let me continue the story a bit more.

Mod C PRL also has all limits and colimits, and the forgetful functor Mod C PRL to PRL preserves them. It also has a tensor product which is a tensor product over C, a standard relative tensor product. 

If you have C in PRL and then commutative algebra objects R and S in C, you can consider left modules over R or S, which will be in Mod C PRL. Then Mod R C tensor over C Mod S C is just Mod R tensor S C.

Course, you say, "C" tends to him, but of course this comes with, yeah, there's more coherencies and there is a good way to formulate this system of coherencies. In this, we have to read some higher algebra. You have to read higher algebra, but you know, yeah, you, yeah, it's not easy to read higher algebra, I know, but it's possible. People have done it, and someone even wrote it, which is even more amazing, yeah.

Okay, so Tser products. Okay, so now I want to connect this to analytic rings. So, note, there's a functor or, well, maybe, so, yeah, so now I can say the good way to encode, or a good way at least to encode a six functor formalism, is a lax symmetric monoidal functor from Span(C's) to PRL with this tensor product here. So, we're no longer using the Cartesian product on Cat(Infinity), but this tensor product on PRL. But it really just amounts to a condition on the kind of formalism we had in the other sense. It's just the condition that when you look at this D of X cross D of Y going to D of X cross Y, that that should commute with colimits in both D and X and D and Y separately.

So, it's just a condition on this formalism here, but it's best to think of it as being a formalism with values in PRL, okay. Now, I want to connect with our specific example. So, note, analytic rings map to PRL by sending our triangle, D of R, to D of R. Well, in fact, as I already said, it maps to C alge(PRL), but, in fact, it maps to C alge(PRL) over a certain other commutative algebra object, namely, the derived category of condensed Abelian groups. So, everything by its nature lives over this derived category of condensed Abelian groups. So, we have a presentably symmetric monoidal category with a functor from this presentably symmetric monoidal category, and then I claim this functor commutes with colimits and detects isomorphisms under, yes, under, thank you. We're in the world of algebra, so I should say "under".

Yeah, we have this category here, and then we have an object in this category, and this notation means that you consider an object in this category together with a map from this object to that object. So, it's this slice category or "under" or "over" category or something. Condensed, every condensed abil, co, gives you, by pullback, some guy in of, okay. I mean, it's just, yeah, I mean, the initial object here, I want to say, it always SS to see over over IM of the addition, ah, okay. That's an analytic ring to what's the initial object, Z. Okay, so for Z, you get D of, condensed, so you don't need them, ah, okay, CLA this F, just, yeah, just a sec, just a sec.

In particular, so if you want to, the derived category of a pushout, so, let's say, a, so this is pushout in analytic rings, which was this kind of slightly subtle operation in this perspective on analytic rings because you had to complete some pre-analytic ring structure and so on, but actually on the level of the categories, it's quite naive. So, it is just this Lurie tensor product, this relative tensor product in PRL. So, the proof is not so difficult. Let me indicate what's going on in the proof, just in this special case, which is really all we need. In the case where A and B are both proper over R, I'm not assuming any Schreier ability, but still, let me use this language of proper maps. In the case when A and B are proper over R, it's just an instance of this general fact here. Did you say that commutative and now got confused, all this? So, somehow you get, you said the modules as limits and colimits, and then did you say that symmetric? Oh, I didn't say that C alge has colimits. I should have maybe discussed it, but it does. And, as usual, pushouts are calculated by relative tensor products. Pushouts in C alge(PRL) are calculated by relative tensor products in PRL. Okay, and it has limits also. Those


You can still factor any map of analytic rings as a proper map followed by a localization. For localizations, it's quite easy to check the universal properties to compare the two sides of this. Granting the case of proper maps, you then just have two different quotients of your category that you're trying to identify, and you can identify them by looking at the descriptions.

For a proper map, the algebra over the bigger ring is the algebra of the source. However, for general analytic maps, the algebra over the localization is not necessarily the algebra of the source. But you can still use this reduction to the proper case. Every map of analytic rings still factors as a proper map followed by a localization, though this localization won't have a left adjoint.

This functor is not fully faithful for a technical reason. To be fully faithful, you'd have to be able to recover the triangle just from the data of the D of R with its tensoring over condensed abelian groups. You can certainly recover the underlying condensed abelian group of the triangle, but there's not quite enough structure here in the animated ring context to recover the full animated ring.

As a corollary, if you take Y in f and consider f-shriek over Y, the category of X mapping to Y via a schable map, then the six functor formalism applies. We can look at the span category of this, but now we don't need to restrict to schable maps anymore, because every map in this category is schable. This span category is a priori lax symmetric monoidal, and in fact, it is symmetric monoidal.

For X to Y schable, the object D of X in modules over D of Y in PRL is dualizable with respect to the relative tensor product over D of Y, and in fact, it is canonically self-dual. With respect to this self-duality, the dual of the pullback map from X to X' over Y should be a map from the dual of X to the dual of the dual of D of X to the dual of D of X'.


So, you just get a map in the other direction from $D(X)$ to $D(X')$. This map is none other than the lower shriek functor. The proof is that you can check this universally in any span category. We have a symmetric monoidal functor, so if you want to, the image of a dualizable object will be dualizable, and if you exhibit a duality pairing here, you get one there and so on.

So, why is every object dualizable in this Span category self-dual? You have a unit, which is a counit, that is the same thing going in the reverse direction. Then you just check the triangle identities; it's completely straightforward. And then you can see that with respect to this self-duality of every object, passing to the dual is just the same thing as transposing the span.

So, we have a proposition: if $f: X \to Y$ is a schematic map, then the following are equivalent. Peter said this last time but didn't give the proof, and now we're going to give the proof. One is that $f$ satisfies shriek descent. Two is that for all $M$ in $D(Y)$, you have descent for $M$. In other words, what Peter equivalent---oh, yes, yes, yes. By definition, shriek descent means that you have descent with respect to this weird pullback, these real weird shriek pullback functors on the $D$'s. But the conclusion here is that you get it also automatically for the star pullback instead.

And then the third condition is kind of a categorified version of star descent, studied by Grothendieck and Rosenberg, which is a remarkable thing. If you take this whole category, then that assignment, assigning to $Y$ this category, also satisfies descent in the sense of $\infty$-1, but if you have---I'll explain more or less why it's the same, requiring this for $\infty$-1 and for $\infty$-2, but I'm not going to try to say I know what an $\infty$-2 means.

So, do you have also a shriek version of two? Yes, you have a shriek version of two, but that's completely formal from one. Let's get into the proof, and then it'll become more clear. Or maybe let me state a corollary first. A corollary is that a pullback $f$ satisfies shriek descent, implies any pullback also does. You can see that from two and using this symmetric monoidal that we discussed, but also, I think in the course of the proof, a simpler explanation will arise for why shriek descent is closed under pullbacks, which is important for using this to define a Grothendieck topology in the first place.

Okay, why the $P$? Because the $D$ of the---yeah, because the pullback in analytic rings is already calculated by a relative tensor product. So, if you took, for example, $M$ to be $D(Y')$, then this would give you star descent for the pullback, but you could take $M$ more generally to be any $D(Y')$-module. And then you see that condition two is stable under pullback, under base change.

So, we have star shriek descent, but I explained that the best way to think of that is that you have this colimit in $\mathrm{PRL}$, but I also explained that colimits in module categories are the same as on the underlying---so that's the same thing as modules over

Now, let's take the internal Hom out to $M$ for any $M$ in $\mathrm{Mod}_D(Y)$. Then, on the right-hand side, we're taking the internal Hom from the unit to $M$, so we just get $M$, and it's being identified isomorphically with some limit. So, the colimit pulls out of the internal Hom limit of the internal Hom in $D(Y)$ with $M$.

But I then explained over here that this thing is self-dual, it's dualizable, and in fact, self-dual. So then, you can actually move it over here with a tensor product. So, this is the internal Hom in $D(Y)$ with $M$, and I said that the manner in which it's self-dual is such that it converts shriek functors into lower shriek functors into upper star functors. So, then this is exactly a proof that one is equivalent to two.

Of course, you have to verify that the construction you get is really the expected one in some higher sense. You have to produce the data making this coherent identification of the dual of this self-duality and the dual of this, identifying with this, universally in the span category, and then map it into modules over $D(Y)$. I haven't actually done that, but there's also another independent argument for this which doesn't require such things. I just thought I would present this because it feels the clearest to me, even if it's slightly difficult to make technically work.

Okay, and then two implies three. We have to check this statement here. The map has a right adjoint, which is basically you take a system of modules here and you forget them all to $D(Y)$, and then you take the limit in this category here. So, it's like taking a system of modules $M_n$ and then you take the limit of the $M_n$. And you want to check that the unit and the counit of this adjunction are isomorphisms.

For the counit being an isomorphism, it's true after you base change up to $X$. This is because $D(X)$ is dualizable over $D(Y)$, which implies that you can pull the limit out. And then, the covert is split on pullback to $D(Y)$ or $D(X)$, and there's an obvious base change property.


We are completely okay then. Let's see, because it's dualizable, any limit commutes with it. But this is actually not correct from the start. Oh, I should change the whole system from D of Y to D of X, yeah, that's good. So then, thanks.

After base change to D of X, in other words, you take this whole collection here and you base change to D of X, the limit can commute. So you can do it in the system after tensoring, and after tensoring you get the system of the Adic categories for the fiber product space, a fine space, so to speak. Because we said that the D of the fiber product is D of the tensor product of the D for fine things, yes or no?

Okay, and then the covering is split, and so by the usual argument, this means that you can play with some Amitsur cohomology. Okay, then what is point three, the base change property? Oh, that's so that when you tensor the thing up to D of X, you can identify that system with the corresponding system for the pullback of the cover to D of X. So it's purely formal.

Well, actually, I think this argument is in Kirill Mathews' paper, and maybe it's also in Gates-Goresky-Rosen-Bloom, and so on. I mean, which you referred to as the original reference for descend ability, it's like the reference I wrote down last time.

Okay, so then this collection, and then the image of this, and then this, those two things are the same, isomorphic, after you tensor to D of X. But then two also implies that tensoring D of X detects isomorphisms. Done. Okay, it's a bit of magic, but you also get now a strong form of three with the two infinity.

Let me quickly explain the argument for three implies two, which is kind of more or less explaining why you get a strong form of three. If you take mapping spaces, if you have a what's that, Peter said. Right, you're right, that this argument shows that two is exactly the claim that the unit is an isomorphism. So clearly, three implies two, but I was going to give a different argument which maybe also explains why you get an infinity 2 categorical three implies two, because three means that this functor is an equivalence, but this functor has a right adjoint. So being an equivalence is the same as the unit and the counit being isomorphisms, but when you unwind what it means for the unit to be an isomorphism, you get exactly two. So clearly, three is stronger than two.

The idea is to go from $\Delta^n$ to this, and then if you let $N$ vary, you recover the whole well-completed simplicial space associated to this category, just in terms of mapping spaces.

I'm saying that what you get a priori is the space of objects in this $\infty$-category that you're interested in, but if you then vary the source by tensoring with presheaves on $\Delta^n$, and that tensoring kind of commutes with everything, then you get not just the set of objects, but the set of morphisms, the set of composable pairs of morphisms, the set of, etc., etc.

Yes, I was just giving it a little more explicitly. $\Delta^n$ is a finite category, and I have to put it in this big world of $\operatorname{Prl}$, so I just take presheaves of whatever spaces. If I want this to be a tensor over $\mathcal{D}$ of $Y$, then I should do presheaves with values in $\mathcal{D}$ of $Y$, and $\Delta^n$ is the simpli[cial] finite category or opposite.

Okay, so this was roughly half of what I wanted to do today, so let's go for another two hours, shall we? No, I'm kidding. We're taking a little break. I guess the next lecture is scheduled for the 10th of January, which is a Wednesday, so I'll be picking up and actually finishing what I intended to do today.

This was some consequence of \v Cech descent. Maybe just mention one more corollary of this, which is that the topology of \v Cech descent is subcanonical, so the \v Cech $\infty$-category fully faithfully embeds into the sheaf category. So those are some consequences of \v Cech descent---you get some very strong descent results. And next time, I'll talk about how you produce examples of \v Cech covers.


\end{unfinished}