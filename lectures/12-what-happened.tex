% !TeX root = ../AnalyticStacks.tex

\section{\ufs What happened so far? (Scholze)}

\url{https://www.youtube.com/watch?v=VMgZSP9sRdo&list=PLx5f8IelFRgGmu6gmL-Kf_Rl_6Mm7juZO}
\renewcommand{\yt}[2]{\href{https://www.youtube.com/watch?v=VMgZSP9sRdo&list=PLx5f8IelFRgGmu6gmL-Kf_Rl_6Mm7juZO&t=#1}{#2}}
\vspace{1em}

\begin{unfinished}{0:00}
e
I  guess  it's  time  to  start  uh  good
morning  um  for  the  people  oneone  you  can
fill  out  the  survey  there  uh  now  or
anytime  this  week  or  next  week
um  all  right  so  so  where  are  we  in  the
course  uh  so  let's  try  to  uh  take  a
little  bit  of  a  broader  view  so
um  so  what  happened  so
far
um  so  we  started  with  this  idea  like  I
know  condens  set  condens  the  being  cured
and  so  on
so  cond  mathematics  and  we  try  to  make
the  point  that  this  is  a  good
framework  uh
for  for  combining  homological
algebra  uh  and  functional
analysis
and  the  goal  is  to  use  this  framework  to
develop  a  general  good  notion  of
geometry
um  but  then  at  some  point  we  Su  made  the
wanted  a  notion  of  completeness  so
uh
and
um  and  at  that  point  like  we  we  went
into  a  certain  direction  namely  with
know  uh  were  concentrating  on  the  solid
Ser  so  we  had  the  solid
modules
um  first  over  the  integers  but  then  over
our  let's  say  in  finite
type  the  algebra  D  discussed  last  time
particular  um  or  in
any
and  uh  well  this  this  is  closely  related
to
this  it's  good  for  nonan
geometry  um  and  a
specific  uh  framework  we  discussed  is
really  extremely  closely  related  to
Hoover  theory  of  edic
spaces  uh  and  I  mean  there  were
certainly  uh  some  conceptual  idea  that
emerged  from  uh  from  this  namely  that  uh
this  notion  of  completeness  shouldn't  be
something  that  you  define  once  and  for
all  uh  but  rather  is  is  an  extra  datum
that  someh  part  of  the  datm  of  like  what
a  communative  ring  is  in  the  setup
um  so
completeness  is  a  more  relative
not  some  sense  so
it's  not  AB  no  concept  for  all  but  it's
part  of  dat  of
antic  which  modules  are  con  as  part  of
the  data  of  what
we  then  defin  as  an
analytic
and
so
uh
so  what  we  are  we  trying  to  first
through  in  the
course
uh  um  so  we  have  the  St  analytic  G  and
maybe  um  for  reason  that  we  already  saw
a  little  bit  that  when  you
localize  some  of  the  structur  might
become  not  7  de  zero  anymore  um  we  might
change  the  preise  definition  here  a
little  bit  to  some  of  our  deriv  rings
from  the  start  uh  but  this  is  something
I  don't  want  to  go  into  today
um  but  so  we  want  to  start  from  this  C
of  analytic  rings  and  G  say  all  basic
building  blocks  to  then
form  like  as  usual  algebraic  geometry
you  start  with  Community  rings  and  then
you  build  schemes  by  gluing  the  Spectra
of  commun  rings  and  similar  there  should
be  some  procedure  where  you  start  with
this  C  of  analytic  rings  and  then  do
some  some  kind  of  gluing  to  produce  some
kind  of  notic  spaces  or  maybe  we  will
directly  actually  go  to  some  C  of
specs
um  some  kind  of
procedure
um
so  there  will  be
uh  some  very  general  such  class  that  we
will  introduce  at  some  point
um  and  but  then  within  within  this  world
we  want  to  find  all  sorts  of  uh  to
find  this  and  find  within
the
where  is
CL
no
okay  um  but  certainly  want  our  Theory  to
not  just  be  a  nonen  we  definitely  want
uh  to  accommodate  also  geometry  over  the
anal  geometry  over  the  real  or  complex
numbers  and
so  uh  so  but
first  uh  we  would  like  to  see  more
examples
um
so  some  key  key  questions  maybe
are
um
um  real
numbers  uh  have  an
natur  you  know  that's  Som  suitable  for
doing  complex  or  realic
geometry  and  another  important  question
is  uh  if  you  can  put  some  natural  search
on  the  real  numbers
um  and  so  can  do  some  kind  ofi  geometry
but  the  other  also  have  non  geometry
using  the  solid  modules  um  maybe  and
then  there's  a  question  what  a
meaningful  way  to  combine  uh  combine  the
two  settings  so  is  there  a  good
way  um  some  combine
um  and  there  might  be  several  different
ways  of  how  one  might  try  to  start  to
thinking  about  this  um  I  want  to  develop
one  way  uh
uh  um  with  the  following  example  really
uh  uh  in
mind  so  maybe  the  most  prime  example
really  of  of  one  one  should  be  able  to
do  in  some  world  of  geometry  so  here
yeah  key  example  that  we  want  to
accommodate  um  is  the  famous  table  look
the  um  so  let  me  spend  a  little  bit  of
time  just  talking  about  this
uh  what  is
this
which
um  right  so
uh  so  T  elliptic  curve  say  is  some
elliptic  curve
e  e  or  EQ  um  that  can  be  defined  for
example  over  the  ring  D  wrong
series  and  uh  here's  one  way  to  define
it
that's  basically  uh  using  uh  the
framework  of
Ed  it's  actually  uh  if  you  look  at
Uber's  paper  I  think  it's  the  one  called
the  generalization  of  formal  schemes  and
something  like  that  um  there  really  is
also  one  of  the  first  applications  of  a
Ser  he's  again  discussing  like  the  ti  of
De  curveent  that's  a  generalization  to
high  dimensional  BM
varieties  um
so  you  have  the  edic  spectrum  of  like  WR
series
Al  series
Al  and
then  over  there  you  can  look  at  some  of
the  gem  as  an  space
over  so  inally  speaking  that's  all
the  like  there's  some  coordinate  T  here
and  T  should  have  the  proper  the
absolute  value  of  T
um  is  bound  with  between
some  uh  some
BO  so  Q  is  some  on  top  element  in  there
and  then  you  look  at  the  part  uh  of
multiplicative  group  where  uh  the
absolute  value  of  T  is  abounded
by  by  power  um  and  then  acting  on  here
you  have  multiplication  by
Che
and  uh  this  is  actually  a  totally
discontinuous  operation  because  it
multiplies  the  absolute  value  of  T  by
the  absolute  value  of  Q  which  is  like
something  between  zero  and  one  um  so
it's  actually  uh  free  and  totally
discontinuous  action  uh  and  so  in  the
world  of  X  basis  you  can  really  pass  to
the  quotient  and  the  taking  this  quo  is
nicest  kind  of  quo  the  other  way  uh  just
get  where  thetion  is  like  locally  split
and  locally  locally  the  space  just  looks
like
this  and  so  you  can  def  find  EQ  to
be  uh  this  analytic  DM  and  then  you
potion
by  Q  to
Z  and
and  I  mean  how  it  look  like  I  me  you
started  with  with  I  don't
knowm  and  then  modifcation  by  Q  it's
moving  everything  towards  the  origin  but
then  when  you  take  a  qution  this  someh
the  same  thing  as  taking  some  Ang  of
radius  uh  one  one  of  of  Q  and  then
identifying  boundary
anul  and  so  using  this  you  can  actually
see  that  this  is
some  uh  yeah  it's  actually  proper  so
take  far  the  Compact  and  all  the
separated  um  it's  also  smooth  because
it's  locally  just  a
gem  it's  proper  smooth  connected  curve
analytic  curve
over  and  actually  also  I  mean  there  a
group  structure  on
gmed  by  sub  groups  it  actually  has  a
group  structure
second
and  then  and  basically  in  any  world  of
analytic  geometry  that  should  be  a  theum
that  something  purple  smooth  one
dimensional  is  always  algebraic  because
you  can  always  find  an  ample  devisor  by
just  taking  any  close  point  and  then
taking  the  corresponding  inverse  ideal
sheet  um  this  gives  an  EMP  line  bundle
and  then  I  if  you  have  any  kind  of
version  of  RAR  nor  your  theory  which
should  always  say  show  that  they  not
function  to  Pro  algebraicity  of  the
thing  and  so  it's  definitely  true  here
so  there
actually  so  here  there  implicitly  there
is  some  Gaga  for  over  a  nice  narian  H
basic  Uber  pairs  but  also  the  statement
about  Relative  Dimension  One  it  works
over  a  non  archimedian  field  that  the
proper
curve  prop  is  a  projective  but  if  you
have  a  more  General  basis  like  Spa  of
something  like  here  then  of  course
locally  for  the  analytic  topology  you
can  use  formal  model  some  device  to  get
formal  model  with  relative  dimensional
it  is  easy  to  construct  bundle  but  to
globalize  it  here  you  have  a  section
actually  which  gives  an  ample  device  so
the  identity  section  but  in  general  I'm
not  sure  if  it  is  clear  that  a  proper
Relative  Dimension  one  over  a  nice  base
of  course  locally  analytically  it  is
projected  by  using  some  formal  models
but  I'm  not  sure  about  what  happens
globally  over  Spa  of  nice  pairs  so  but
it's
not  I  have  a  section  okay  which
I  right  so  but  anyway  so  in  this  case
lot  hard  to  this  automatic  algeb
right  and  from  the  end  it  just  defines
an  algebraic  tic  curve  some  defines
antic
curve  over  just  in  the  end  it's  just  Cur
of  the  abstract  policy  I  mean  in  between
we  Som
use  use  some  kind  of  topological
condensed  structure  on  this  series  Q  but
in  the  end  we  just  get  an  litic  P  of  the
abstract
um  and  okay  so  it's  a  litic  curve  so  you
can  write  down  in  R  press  equation  first
um
and
uh  uh  so  if  you  write  down  the  wi  press
equation  I  mean  for  I  guess  if  you
remove  the  zero  section  as  usual  um  then
you  can  write  in  the  following
form
um  I  copy  this  from  Wikipedia  so  I  hope
it's
correct
uh  okay  so  to  write  our  integral  and
usually  can  omit  this  term
but  usually  have  to  Inver  two  to  do  this
so  if  you  don't  you  should  put  it  Mo
form
um  and  then  these  are  some  power  Series
in  Q  and  what  are  they  uh
my  notes  tell  me  I  should  write  the
phone
okay
so  I  know  I  mean  this  definition  looks
rather  simple  to  me  these  formulas  look
rather  complicated  to
me
but  when  I  first  saw  those  formulas  I
remember  that  I  was  struck  that  if  you
actually  look  at  the  coefficients  okay
so  you  don't  actually  see  the
coefficients  of  Q  to  the  n  in  here  but  I
mean  okay  so  it's  easy  to  invert  oneus  Q
to  the  N  right  it's  1  plus  Q  Plus  One  Q
to  2  N  S  and  so  if  you  imagine  doing
that
then  I  mean  here's  the  coefficients  they
are  just  polinomial  and  N  right  and  I
mean  when  you  do  this  this  geometric
series  here  I  mean  the  new  coefficient
still  stay  poal  on  end  uh  so  some  kind
of  funny  observation  is
that
uh
uh  uh  the  for  coefficient  the
coefficients
of  Q  to  the
end  uh  po
gross  um  this  implies  in
particular
um  like  after  we  just  to  find  something
form  series  room  but  then  when  you
actually  write  down  this  equation  you
realize  that  this  is  something  that  you
can  specialize  uh  to
any
uh  also  not  any  Arian  value  uh  between
zero  and  one  and  well  that  would  have
also  been  clear  from  this  kind  of
analytic  description  I  mean  you  can
certainly  take  like  C  Star  as  a  uh  as  a
complex  analytic  space  as  a  complex
manifold  and  for  any  complex  number
between  zero  and  one  I  mean  now  this
picture  makes  literal  sense  I  mean  you
can  literally  take  that  potion  and  you
get  an  actual  Taurus  right  and
so  so  you  can  do  that  but  also  uh  the
analytic
description
SDM  keep  V
uh  makes  perfect  sense
then
so  so  we  would  really  like  to  have
a
uh  yeah  to  be  able  to  do  geometry  so
that  we  can  Som  perform  this  kind  of
construction  of  like  an6  GM  Q2  Z  not
just  over  the  full  wrong  series  algebra
but  really  over  some  sub  algebra  but  we
put  some  growth  condition  on  the
coefficient  so  that  we  can  also  later  on
specialize  to  median  part  um  note
however  that  the  precise  grow  condition
that  you  get  here  is  quite  a  bit
stricter  than  just  the  observation  that
you  can  plug  that  it  converges  when  the
have  to  VAR  CU  less  than  one
uh
uh  so
convergence
um  this  is  just  a  subexponential  growth
of  the
coefficients
um  but  I  mean  I  even  you  think  about
maybe
if  you  think  about
yeah  so  it's  pretty  clear  that  there
there  should  be  a  geometric  reason  that
geometrically  should  be  pretty  clear
that  this  cor  series  should  converge
when  Q  is  less  than
one  uh  so  there  there's  some  kind  of
direct  geometric  way  of  seeing  that  the
whatever  coefficients  are  getting  here
that  their  coefficients  much  have  at
most  uh  sub  exponential  growth  uh  but
it's  not  so  clear  how  you  would  see  that
they  actually  have  most  poomi  goes  but
there's  a  geometric  reason  for
that
um  all  right
um
so  uh  let  me  actually  talk  before  um
really  getting  again  to  this  Mark  median
Theory  let  me  actually  talk  a  little  bit
about  uh  the  geometry  of  the  space
space  so  this  The  Continuous  valuations
on  on  on  Z  WR  series
Q
so  so  I  don't  know  so  if  you  have
some  they  call  variations  but  I  still
want  to  denote  some  as  absolute  value  so
it's  this  to  G
zero
and  so  in  particular  the  absolute  value
of  Q  will  be  a  certain  element  in  here
will  actually  not  be  zero  so  it's
certainly  actually  G  part
um  and  it  turns  out  that  like  you  have
this  to  new  poent  unit  and  this  is
actually  an  extremely  convenient
structure  to  have  because  it  allows  you
to  compare  ABS  values  of  all  other
functions  against  the  absolute  value  of
of  Q  just  like  we  did  here  when  we  had
some  other  function  like  like  the
corrent  function  of  GM  we  can  somehow
gauge  how  large  It  Is  by  comparing  it  to
the  absolute  value  of
Q  and  this  is  something  you  can  always
do  once  you  have  such  a  top  LOD  new
poent  unit  um  and  so  this  in  particular
this  means  that  there's  actually  unique
M  here  to  the  real  Great  equal  to  zero
um  which  sends  the  absolute  value  of  Q
to  some  pre-specified  element  between
zero  and  one  and  let's  just  choose  a
half  um
and  for  most  of  the  valuations  that  we
care  about  here  um  this  will  actually  be
injective  but  there's  some  rank  two
valuations  where  there's  a  little  bit  of
extra  information  in  the  GMA  that's  not
remembered  by  this  qution  but  but  in
first  approximation  you  can  really  um
think  uh  right  so  this  gives  a
map  from  this  Ed
Spectrum
um  towards  what's  known  as  the  burage
Spectrum  maybe  there  are  several  any  11
call  it
nber
um
um  here
for
f  recall  that
um  uh  uh  the  is  an  example  of  such  a  t
ho  ring
um  and  said  last  time  that  this  can
really  be  endowed  with  a  bout  Norm  where
this  this  thing  becomes  a  the  unit  ball
and  then  again  if  you  spe  uh  specify
what  the  absolute  value  of  Q  should  be
you  can  decide  what  the  absolute  value
of  anything  is  looking
at  the  power  of  Q  that  you  need  to  make
an  integral  and  then  taking  the
corresponding  power  of  a  half  so  for
Bing  R  um  could  you  write  slightly
larger  yes  I  think  sorry  maybe  just  use
next
St
perpose  here  a  set  of
all  now  again  called  absolute  values  but
now  they  really
take  Y  is  and  RoR  equal  to
zero
and  now  you  ask  the  following  things
uh  where  first  of  all  I  don't  know  there
are  some  stupid  things  like  normal  Z  is
z  normal  one  is  one  um  it's
multiplicative  nor  X  nor  X  nor  y  B  which
I  mean  here
valuations  which  in  particular  satisfies
this  strong  Triangle
inity  C  the  is  not  restrictive  to  non
setting  so  it  considers  the  usual
triangle
inquality  but  you  also  ask  that  they  all
of  them  should  be  bounded  by  the
norm  um  you  have  specified  on  on
r
um  exactly  how  he  sets  it  up  say  less
equal  to  a  constant  time  but  I  think  in
this  case  it
doesn't  and  so  this  naturally  match  to
the  product  of  copies  of  R  greater  Expos
to  zero  enumerated  by  all  elements  X  and
R  and  uh  he  end  this  with  a  Subspace
topology
maybe  so
actually  uh  because  of  this  Foundation
you  can  actually  replace  R  greater  equal
to  Z  here  always  by  the  interval  from  Z
to  the  absolute  value  of  x
and  so  then  these  are  all  compact
intervals  and  then  an  arbitrary  comp
product  of  compact  h  space  is  still
compact  H  so  there  still  a  compact
space  and  all  the  other  conditions  that
you  see  here  they  are  all  closed
subspaces  this  is  actually  a  closed
Subspace  and  so  particular  the  verage
space  is  actually  a  nice  compact  h  of
space
yes
and
then  actually  effective  of  this
situation  that  actually  the  only
difference  between  these  two  spaces  here
is  the  possibility  that  in  the  TIC
Spectrum  you  can  have  higher  rank  things
but  these  higher  rank  things  they  give
rise  just  to
some  uh  very  inmal  changes  in  the  space
actually  uh  this  potion  that's  or  this
NE  here  from  the  Spectrum  the  C  Spectrum
this  actually  maximum  House  of
potion  so  the  two  are  extremely
similar  note  here  that  implicitly  I'm
endowing  the  integers  here  with  a  norm
where  zero  the  nor  of  Zer  is  zero  and
the  nor  of  all  elements  is  one  because
they  all  containers  unit  for  the  ring
I'm
considering  okay
so  he  has  this  tic  space  for
this  and  so  I  still  want  to  understand  a
little  bit  about  this
geometry  it's  more  or  less  the  same
thing  as  anyways  as  the  b  space
um  but  that  actually  Alat  down  to  the
Bur  space  of  the
integers  and  I  mean  usually  if  you  have
any  kind  of  spectrum  of  the
integers  it  doesn't  have  a  rich  geometry
maybe  I  mean  they  have  a  prime  and  maybe
one  generic  point  or  something  I  guess
but  the  bir  space  of  the  integers  is
much  more
interesting
um  here's  here  the
proposition  uh  maybe  I  should  stress
here
yes  Norm  on  Z
implicit  B  Norm  on  Z  it  mean  yeah  it's
the  one  where  the  normal
zero  is  zero  and  but  the  norm  of  any  n
which  is  not  zero  is  one
because  all  of  them  are  contained
in  um  but  there  is  a  sometimes  more
natural  uh  Norm  you  can  put  on  Z  just
the  absolute
value  right  so  let  me  draw
a
for  the  norm  which  is  z  or
one  um  which  to  noted
um  um  this  is  the  following  thing
um  so  actually  this  very  Norm  it  is
satisfies  all  the  properties  the  nor  of
Z  is  Zer  nor  of  one  is
one  uh  so  actually  defin  point  it's  a
more  less  a  generic
point  but  of  course  you  know  many  other
natural  absolute  values  on  the  integers
namely  for  any  prime  number  P  you  have
the  pic  absolute  value  and  it  also
satisfies  all  those  properties
there  but  whenever  you  take  the  pic
absolute  value  you  actually  have  a
choice  what  is  the  absolute  value  of  p
and  so  for  each  prime  number  P  you
actually  have  a  full  line  uh  of  possible
like  full  interval  like  going  from  zero
to
one
I'm  not  running  here  I  guess  um
see  and  then  there's  also  a  close  point
at  the  end  of  this  Ray  uh  where  you  go
to  FB  um  so  you  can  also  get  tou  an
absolute  value  if  you  first  project  from
Z  to  FP  and  then  do  this  kind  of  thing
that  zero  goes  to  zero  and  everything
that's  non  zero  goes  to
one
okay  you  can  do  this  corresponds  to  two
this  corresponds  to  three  so  on
corresponds  to  five  and  then  for  for
each  prime  number  uh  you  have  such  aray
and  then  some  kind  of  inverse  limit  of
joining  one  Ray  at  a
time  uh  right  so  F
points
so  the  points  and  all  correspond  to  maps
to
to  complete  B  fields  um  and  so  there's
one  that  corresponds  to  Q  with  the
value  so  first  go  from  Z  to  q  and  then
do  this  thing  which  we  just  recover  this
abute
Val  here
here
then  you  can  go  to  QP  for  any  p  and  the
absolute
value  with  the  norm  of  P  being  any
specified  element  in  the  interval  from  0
to
one  and
then  FP  again  with  a
tri
um  all  right  uh
let  me  also  already  discuss  now  um  the
side  variation  of  this  so  you  can  also
take  the  b  space  of  the  integers  but  the
usual  absolute
value  I  mean
so  Norm  of  n
is  plusus  n
um  positive  version  of  it
um  so  this  the  same  pictures  uh
the  difference  between  these  two  things
I  means  the  same  data  satisfying  these
three  same  conditions  the  only  condition
that's  different  is  this  boundedness
condition  that  you  have  here  so
previously  someone  asked  that  it's
uniformly  bounded  by  one  so  in
particular  the  absolute  value  of  two  can
never  be  two  like  it  would  be  for  the
real  numbers  uh  but  now  I  kind  of  fix
that  so  now  the  absolute  value  of  two
can  be  two  uh  and  so  this  is  actually
some  of  the  same  picture  so  there's
again  this  thing  like  for  each  FR  number
part
before  but  now  there  something
extra  then  there  is  some  kind  of  Heth
interval
um  um  so  this  corresponds
to  the  map  from  Z  to  R  and  then  you  take
the  usual  absolute  value  of  r  to  any
power
P  where
p
uh  go  from  zero  up  to
one  and  now  it  looks  like  this  was  also
an  interval  from  zero  to  one  but
actually  uh  it's  better  to  think  of
this  if  you  take  the  usual  absolute
value  Peter  yes  can  I  vote  that  you  use
Alpha  instead  of  P  when  talking  about
raising  an  absolute  value  to  the  P  power
thank  you
yeah
um  just  in
time  uh  to  the  alpha  so  if  I  if  I
parameterize  this  in  terms  of  some  fixed
absolute  P  absolute  Val  say  the  one
where  absolute  p  is  one  over  p  uh  then
this  line  here  there  corresponds  to  this
to  the  alpha  where  now  Alpha  can  be
anything  from  zero  to
Infinity  uh  but  for  the  real  you  fix  the
usual  absolute
value  then  you  can  raise  that  you  can
try  to  raise  that  to  any  real  power  and
it  will  definitely  satisfy  this
condition  and  this  condition  but
actually  if  you  want  the  triangle
inequality  to  be  satisfied  you  realize
that  this  happens  only  if  Alpha  is  at
most  one  uh  so  in  this  sense  this  this
line  for  the  real
numbers  is  actually  some  stops  in  the
middle  compared  to  the
others
okay  uh  right
so  where  are
we  uh  right  so  we  are  trying  to
understand  uh  a  little  bit  about  gometry
of  what  this  add  spectrum  of  Z  long
series  Q  actually  looks
like  and  it's  basically  the  same  as  the
ver  spectrum  and  this  is  fire  over  the  B
spectum  of  the
integers  and  now  let's  actually  try  to
understand  uh  what  are  all  the  fibers
right  of  this
map
and  so  for  example  if  if  you  take  uh  the
fiber  at  a  prime  P  of  like  just  FP
um  well  then  is  just  one  point  then  as
usual  some  of  the  formation  of  dark
space  commut  with  some  kind  of  fiber
products  and  so  Rings  FP  series
q  and  I  mean  this  is  just  a
point
um  because  there
already  non  field  and  you  fix  the  value
of  Q  to  be  the  half  like
okay  so  in  characteristic  P  the  thing
has  just  uh  one  fiber  which  is  this  kind
of  the  wrong  series
r
maybe  I  should  have  said  I  mean  this
proposition  is  basically  justr  zero
right  because  find  the  absolute
values
um  um  but  then  uh  you  can
also  okay  you  have  this  dark
space  of  the
integers  and  then  for  each  P  you  have
this  off  line
here  line  of
FS
so  if  you  how  Bas  change  this  whole  half
line  of  QPS  um  someone  looking  at
the  you're  in  some  sense
taking  now  this  is  some  kind
of  punctured  open  unit  dis  and  now
you're  this  bace  change  to  QP  will
actually  be  some  kind  of  punctured  open
unit  QP  so  this  will  actually  be  a
punctured
and  uh  this  map  here  to  this  to  this  uh
line  zero
Infinity  depending  on  how  you
parametrize  at  01  or  whatever
um  uh  this  should  be  an  incarnation  of
the  radius
matter  uh  but
actually  in  a  slightly  funny  way  it  say
one  over  the  L  of  the  radius  or
something  like  this  I  won't  get  it
straight
um  so  there  there  is  a  whole  function  of
unit  here  of  which  has  Like  An
Origin  um  and  a  boundary  so  to  say  there
not  point  in
there  and  whenever  you  fix  a  specific
point  on  here  then  this  fiber  will  be
some  specific  anual  in
here  of  where  the  absolute  value  is
fixed  um  and  now  you  can  wonder  what
happens  as  you  move
towards  is  invisible  anyways  but  uh  I'm
always  tempted  to  try
um  see  the  image  is  this  circle  here  and
then  when  you  move  towards  uh  towards
this  characteristic  P  point  right  uh  so
at  the  end  of  this  r  at  the  end  here
that  would  be  q  and  at  the  end  here  that
would  be
FP  um  and  as  you  move  towards  FP
actually  this  anular  here  it  goes
towards  the  outside  of  funed  open
unities  and  some  sense  the  F  wrong
series  t
sits  or  nor  Q  it  sits  near  the  boundary
of  this  of  this
Ang  um  whereas  when  you  move  upwards  on
this  Ray  go  towards  Q  uh  then  you  will
get  other
Ani  and  they  will  get  closer  to  the
origin  okay  so  what  what  does  this  thing
actually  look
like  um
so  there  is  one  lary  point  which  is  qong
series
q  and  then  uh  there  are  special  points
which  are  wrong
series  and  on  the  way  there  you  have  the
punct  open
un  the  region  in  the  middle  the  punct
open  and  the  slightly  mindbending  thing
is
that  yeah  how  how  the  different  parts
are  GL  to  each  other  I  mean  the  whole
thing  is  a  compact  h  space  so  it  makes
sense  like  to  ask  when  you  go  in  this
direction  where  you  end  up  um
and  so  yeah  so  if  you  move  towards  the
puncture  of  this  open  unit  disc  you  move
towards  this  point  and  you  move  towards
the  boundary  we  end  up  towards  this
point  so  this  whole  Space  is  space  that
has  peric  regions  for  each  P  each  peric
region  is  a  function  open  QP  and  then
they  glue  to  each  other  in  this  funny
way  yeah  where  for  each  one  if  you  go
towards  the  center  you  will  end  up  at
the  common  point  which  is  is  a  kind  of
generic  point  and  for  each  one  when  you
go  towards  the  boundary  you  end  up  the
characteristic
B  so  you  mentioned  this  fact  that  this
map  is
ma  is  this  special  for  this  r  no  no  this
for  like  T  U  ring  t  Okay
than  yeah  so  for  these  three  things
actually  the  two  series  are  extremely
different  right  because  we  saw  that  like
if  you  take  just  the  integers  here  then
the  addic  spectrum  of  that  would  just  be
like  non  very  nonous  thing  like  the
spectrum  of  ring  is  like  very  non
um  uh  but  once  you  introduce  this  extra
top  variable  allows  you  to  fix  absolute
values  and  then  it's  actually  the  two
series  you're  closing  your
leg
oh
all  right  but
uh  but  I  need  already
suggest
to  look  at  a
varant  uh  of  Z  WR  series
Q  as  b
r  where  I  we  would  like  to  now  like
we've  already  seen  that  for  the  B  of  SP
of  the  integers  there's  a  natural  way  to
enlarge  this  picture
by  instead  of  doing  this  kind  of  Noni
Norm  we  can  put  the  usual  Norm  then
there  is  some  aian  part  that's  naturally
contained  in  there  um  and  we  could  of
course  try  to  do  the  similar  thing  for
zero  wrong  series  Q
um  where  we  put  the  usual
Norm  on  Z  and
well  like  today  from  the  start  I  kind  of
made  the  choice  that  the  abute  Q  was  a
half  and  so  far  it  didn't  really  matter
now  actually  suddenly  starts  to  m  a
little  bit  and  it's  a  bit  weird  okay  but
you  simply  declare  the  abute  value  of  Q
to  be  a
h  uh  so
concretely  and  like
Z  wrong
corals  was  an
nor
the  normal
sum  and  Z  almost  all  zero  to  the  N  is
the  sum  of  the  ne  Norms  of  the
ANS  times  one  two  to
the  and  then  we  can  just  complete  this
range
like  operations  unar  doing  their  kind  of
Tor  to  match  literally  B  which  is
Original  Series  where  which  is  about  B
frames  when  we  actually  do  this  in  our
the  we  will  do  something  something
slight  different  operation  but  it's
extremely  closely  related  um  complete
as
thing
to  get  some  thing  call
the
a
um  which
are
those  half
2
uh  so  these  are  concretely
um
uh  nasty  but
um  these  are  I  mean  this
convert
for  values  Q  in  the  complex
numbers  that
are  sorry  yeah  yeah  yes  they  do
converge  and  um  they  Define  a
holomorphic  function  of  Q  in  the  region
where  Q  is  less  than  a  half  and  this  hph
function  send  to  continuous  function  on
the  whole
uh  and
now  right  so  uh  I  apologize  for  doing
this  people  taking  notes  um  but  now  what
happens  if  I  look  instead  of  the  verage
space  now  of  this  ring  I  put  this  extra
aredian  convergence
condition  so  now  uh  there  the  reason
maps  to  the  nonent  part  anymore  so  it's
now  just  snaps
to  uh  this  SP  space  was  a  fual  absolute
value  which  has  this  extra  real  part  and
now  we  can  try  to
understand
uh
instead  what  happens  when  you  pull
back
uh  this  half  interval  of
wheels  and  that  PB  will  be
precisely
uh  uh  this  opuni
disc  uh  actually  mod  compex  conation
because  two  complex  conjugate  things
they  give  the  same
Val  and  again
uh  right  uh  now  it's  liter  like  this
basically  this  projection  here  is  some
kind  of  radius  of  function  where  the
points  of  radius  of
that's  to  a  half  to  what  this  boundary
Point  here  and  when  you  go  towards  zero
then  you  go  down  this  line  and  so  then
again  you  end  up  end  up
here  so  so  what  does  the  space  look  like
so  now  it  has  some  kind  of  function  open
uni  this  for  each  place  of  the  rational
numbers  so  pic  place  and  the  acum  place
um  and  everywhere  when  you  go  towards
the  center  uh  you  always  end  up  with  the
center  point  of  picture
the  sly  awkward  feature  of  this  picture
is  that  I  had  to  now  specify  the  ABS
value  of  Q  in  advance  and  then  as  the
acum  part  of  this  picture  this  this
punctured  op  just  stopped  that  the  BR
new  although  like  from  the  perspective
of  the  ttic  curve  there  was  no  reason
for  stopping  at  qal  to
half
the
I
right  and  so  then  there
uh  right  so  then  one  but  like  this  is
precisely  the  kind  of  picture  in  which
we  would  like  to  combine  non  median
geometry  I  mean  so  this  this  space  his
has  Parts  which  are  literally  complex
analytic  or  real  ptic  analytic
things  but  they  sit  together  in  one  joh
same  um  and  so  one  kind  of  PR  version  of
the
question
uh  about  existence  of  antic  R  structures
um  is  now  uh  like
K1  and  this  guy  or  more  any  albra  for
theator  um
was  a
natural  and  uh  this  has  been  a  question
that  was  very  much  in  our  minds  uh  back
when  we  first  found  out  about  the  solid
Theory  and  then  tried  to  really  go
further  and  um  and  which  we  eventually
answered  with  a
serum  uh  yes  need  liquid
construction  I  mean  it  actually  turns
out  that  to  Define  this  one  it's
slightly
better
uh  to  work
with  not  just  precise  the  health
convergence  condition  but  functions
which  Converge  on  some  rate  some
slightly  larger  dis  so  there  is  a
union  over  all  radi  which  are  bigger
than  a  half  of  the  things  that
converge
that's  more  techn
point  and  some  well  generally  speaking
like  for  any  kind  of  B  offing  which  has
a  top  Lo  unit  in  the  usual  way  uh  you
can  produce  this  kind  of  liquid
analytic  and  the  resulting  the  will  be
extremely  close  uh  to  B  which  Theory  and
this  is  something  that  we  want  to
discuss  at  some  point  um  today
however  uh  I  want  to  also  talk  about
some  something
else  something  that  we  only  found  out  a
few  weeks
ago  I
maybe  so  once  you  you  have  this  with
liquid  and  structure  over  Z  long  series
Q  maybe  greater  than  a  half  you  can
literally  repeat  the  construction  of  the
ttic  curve  that  I  did  in  the  beginning
over  this  ring  and  now  they  all  frame
here  I  guess  you  could  also  just  do  it
in  B  here  um  and  then  someone  show  that
the  ttic  curve  is  definitely  defined
over  this  ring  and  then  okay  in  the  end
you  could  also  make  a  half  larger  and
larger  and  would  get  get  that  it's
defined  over  the  whole  open  unit  dis  but
you  would  not  get  this  way
this  strange  poomi  draws  bound  on  the
coefficients  so
uh  yeah  let  me  make  a  kind  of  somewhat
philosophical  or  personal  digression  um
so  the  said  back  when  we  did  this  so
back  in  2019  or  back  very
recently
um  was  that  if  you  want  to  Define  aning
structure  what  you  should  do  and  what
we've  somehow  done  in  all  the  examples
when  we  produce  them  uh  uh  is  to  Define
is  to  describe  the  pre  complete
mod  and  okay  maybe
actually  have  enough  time  to  actually
specify  a  few
examples  so  when  you  take
so  in  all  the  examples  I  will  take  D  to
be  an  inage  limit  of  Finance  ACC  end  or
let  me  actually  use  I
think
so  let's  say  live
uh  and  so  something  we've  seen  is  that
if  you  want  to  describes  the  three  solid
modules  on  S  and  they  have  a  very  very
simple
description  um
this  is  actually  just  the  limit  of  all
the  three  modules
on  and  this  also  has  a  meaning  in  terms
of  uh  measures  in  the  sense  so  it's
also  like  integer  values  measures  from
this  profile
access  so  to  define  a  measure  you  just
have  to  assign  an  integer  to  each  glal
subset  in  which  way  a  way  which  is  Finly
additive  um
uh  actually  let  me  discuss  a  different
example  I  think  I  already  said  it  at  one
point  but  let  me  Str  it  again  you  can
actually  also  describe  the  abstractly
free  modules  without  any
completion  they  actually  turn  out  to  uh
sit  inside  there  um  and  it's  some  the  L
zero  bounded
card
uh  in  which  which  sense
uh  so  in  this  case  actually  more  natural
from  um  it's  a  union  over  all
integers  of  the  imers
limit  I  of  the  part  of  the  free  module
on  SI  so  this  is  just  a  finite  free
group  with  a  specified  basis
and  so  then  you  can  define  an  L  zero
nor  where  you
just
uh  yeah  okay  I  mean
maybe  well  I  guess  in  this  case  same  as
one  but  that's  still  like  to  call  L  zero
um  so  all  to  Summit  most  and  Sumit  most
and
what  the
difference  so  on  each  of  these  finite
three  guys  you  have  the  basis  vectors
and  then  you're  low  to  take  some  final
bounded  Summit  by  most  n  of  them
so  this  gives  you  in  each  of  these
things  it  just  gives  you  a  finite  subset
and  it  takes  a  limit  of  all  these  guys
gives  you  some  Prof  finite  set  it  takes
a  unit  of  all  these  gives  you  a  certain
small
subset
um  so  there  is  actually  a
description  uh
of
um  of  the  liquid  structure  of  the  three
modules
um
right
um  so
yes  let
until
um
S  as  I  times  to  the
end  I
uh  to
be  the  sum  of  Val  of  the  n  as  usual  then
in  this  R  thing  we  take  have  of  Q  to  be
R  and  uh  right  and  this  s
i
um
then  the
three  liquid
modules  um  on  S  over  this  ring  were  kind
of  defined  to  be  and  then  the  hard  the
is  to  proes  that  this  actually  antic
ring  comes  from  antic  ring  structure  def
find  to  be  the  following
thing  bit  of  space  here  so  I  mean  just
like  the  r  is  this  Union  of  all  bigger
than  a  half  also  the  fre  modules  are
this  Union  over  reg  bigger  than  a
half  and
then  but  then  also  as  in  like  the  free
discreet  guy  um  we  have  this  Union
overall  size  vals  and  this  time  I  maybe
want  to  index  them  by  some  real  numbers
C  greater  than
zero  but  then  once  you  fix  the  radius
and  size  you're  just  taking  an  inverse
limit  of  such  free
modules
of  the  part  where  this  Norm  that  I've
just  defined  there  is  at  most
C  and  so  similar  situation  where  you  can
show  that  if  you
bounce  uh  the  norm  here  and  you  give
this  a  certain  kind  of  topologies  and
this  actually  becomes  the  compact  H
thing  the  limit  is  still  compact
take  describing  it  as  condensed
set  and  I  mean  this  is  really  the  thing
naturally  suggests  itself  when  you  try
to  Define  some  notion  of  complete
modules  over  the  screen  because  I  mean
what  the  thing  in  place  is  like  the  B
Norm  on  this  ring  that  you  have  which  is
defined  just  this  way  for  one  element
and  then  of  course  if  you  have  a  a  five
free  module  you're  just  summing  uh
summing  the  absolute
values  and  so  this  proposal  for  what  to
free  complete  modules  in  some  sense  wres
itself  when  you're  trying  to  find  antic
instruction  on  this
one
y
um  let  me  actually  just  the  course
example  um  that  like  like  in  the
beginning  of  my  lecture  today  I  was
asking  two  questions  first  is  there
natural  stru  on  the  reals  second  is
there  one  which  allows  you  to  like
combine  the  things  now  I'm  starting  to
answer  them  in  the  opposite  order  so  I
first  I  said  that  there's  something  that
combines  them  uh  let  me  now  also  give
the  answer  to  the  first  question  what
are  the  antic  ring  structures  of  the
reals
so  uh  you  can
specialize  from  Z  Ser  CU  greater  than  a
half  to
R  by  sending  Q  to  some  number  T  here
less  than  two  a
half  and
well  this
defines  the  point  of  the  SP
space  now  which  maps  to  the  b
space  of
Z  and  so  must  actually  to  some  power  of
the  usual  absolute  value  of  the  reals  so
it  actually  maps  to  like  inside  there
you  have  this  half  interval  for  the
wheels
um  where  Alpha  and  01  corresponds  to  the
absolute  value  on  R  to  the
alpha  and
so  so  in
here  in  here  you  also  have  to  interval
of  T  which  me  from  from  zero  to  to  a
half
and  now  this  map  is  actually  realizing
some  isomorphism  between  these  two
things  we  writing  the  value  of  T  to  some
number  R  and  this  can  also  be  made
explicit  namely  not  screwing  up  T  is  a  h
to  the
off  no  teach  the
off  I  think  that's  right
okay  so  so  the  value  of  T  determines
some  Alpha  and  I  could  have  just  told
you  the  formula  let  Alpha  be  the  D  such
T  to  Alpha  as  a  half  but  this  would  seem
slightly  curious  what  does  it  mean  I
mean  the  meaning  is  that  you  have  a
point  in  Sp  space  maps  to  b  space  of  the
integers  and  you  get  some  point  there
this  is  the
AL  okay  so  we  have  structure  here  so  we
get  one  here  but  the  complete  modules
are  just  those  that  are  complete  when
you  just  goet  SC
what  are  the  three
modules  so  this  means  that  for  zero  less
than  Alpha  at  no
one  gets  a
socaled  alpha  liquid  ring  structure  on
r
um  okay  let  me  use  next  point  to
describe
Al
liquidations  uh  so  as  before  here  it  was
a  c  Union  of  a  radi  bigger  than  a  half
here  will  be  a  certain  Union  over  let
call  betas  maybe  which  are  less  than
Alpha
and  again  Union  over
constants  that
um  and  then  you  do  same  thing  so  here
you
have  the  three  modules  on  SI  and
then  like  recall  that  some  of  the  alpha
corresponded  to  the  the  absolute  value
on  the  re  which  was  the  absolute  value
to  the  alpha  now  I'm  using  this  Union
here  so  I  take  some  of
it  the  Ala  nor
uh  is  at
most  so  here's
um  of  x  i  s
i  the  Ala  nor  is  a  sum  of  of  X
beta  um  note  here  that  zero  is  less  than
beta  is  less  than  one  so  this  is  not  one
of  the  usual  kinds  of  norms  that  you
would  usually  consider  in  real
functional  analysis  this  is  not  locally
convex  usually  when  you  put  l  p  Norms
the  P  Li  between  one  and  infinity  here
we  us  going  to  the  left  and  use  the  non
lole
s  so  you  mean  it's  not  Ultra  it's  not
doesn't  satisfy  the  triangle  inequality
this  this  this  is  not  a  norm  in  the
standard  sense  there  is  no  triangle
inequality  no  it's  better  than  the
triangle  or  is  it
better  it's  not  real  it's  not  over  is
it  don't  take  one
of  so  this  this  nor  satisfies  the  trying
inquality  but  it  doesn't  satisfy  usual
scaling  XM  so  you
it  to  the  beta  power  of  the  norm  but
usually  would  ask  it  satisfies  the
scaling  with  respect  to  the  usual  real
numbers  but  this  this  doesn't
do  but  I  mean  most  conely  like  if  you  if
you  just  I  know  for  two  dimensional
Vector  space  you  look  at  what  the  unit
balls  so  to  say  look  like  then  they  they
have  this  very  SECU  shape  that  they  look
something  like
that  and  uh
yeah  so  maybe  when  you  see  this  you're
tempted  to  think  that  this  is  a  stupid
uh  way  to  put  a  ring  R  structure  on  the
real  and  you  should  rather  uh  do
something
else  but  this  do  work  so  when  I  E
guess
um  that  the  threee  complete  modules  on  R
uh  should  be  the  Rong  measures  on  r
measures  one  way  to  define  them  is  we
just  the  Dual  for  continuous
functions  um  does  not
work  does  not  find
instent
uh  so  the  way  to  describe  the  S  by  the
way  would  be  to  do  the  similar
construction  here  but  just  U  one
more  um  and  this  would  be  really  closely
related  to  the  usual  the  of  complete  loc
compx  Vector  spaces  in  about  a  clean  as
way  as  the  usual  theory  of  solid  I  mean
the  Ser  of  solid  modules  Rel  to  the
usual  series  of  like  linearly  complete
modules  um
so  we  were  when  we  developed  the  series
we  were  really  hoping  that  this  one  here
is  an  PR  structure  on  the  reals  and  was
really
uh  yeah  somewhat  impressing  and  you
realized  it
isn't  but  it's  Rel  to  some  fun  stuff  uh
uh  In  classical  function  analysis  um  so
one  thing  we  do  always  want  um  in
structure  is  that  the  classic  complete
modules  stable  under  extensions  but  this
is  simply  false  for  Loc  convex  Vector
spaces  so  you  can  have  natural  extension
of  locally  convex  Vector  spaces  even
just  of  an  L1  Space  by  the  real  numbers
where  the  extension  is  not  locally
convex  this  was  realized  around  1918  by
Reba  and  somebody
else  and  it  comes  from  the  entropy
function  so  it's  it  is  not  some
completely  crazy  thing  it  has  some  kind
of  natural  meaning  but  it  destroys  the
option  but  then  once  you  allow  some
local  compx  Vector  space  into  the
picture  the  Ser  works
again  but  you're  really  forced  to  uh  use
stly
Conex
and  I  guess  I  should  also  discuss  one
other  thing  very  quickly
um  uh  so  in  for  we  specialized  uh
this  this  uh  the
uh  to  the  real  numbers  uh  we  could  also
specialize  it  to  the  pic
numbers  um  you  would  expect  that  like  in
terms  of  verage  space  we  were  really
just  trying  to  extend  over  this  missing
part  at  the  real  numbers  so  you  would
expect  that  maybe  at  the  non  part  of
this  picture  we  didn't  really  change
anything  and  like  also  the  convert  extra
convergence  condition  we  put  on  our  on
our  on  our  function  was  just  an  exra
convergence  Condition  it's  a  real
numberss  uh  so  naturally  you  would
expect  that  when  you  specialize  liquid
structure
uh  on  on  this  convergent  Artic  drone
Series  ring  that  back  to  a  piging
then  uh  then  you  would  just  get  the
solid  structure  but  that's  not  true  uh
actually  you  get  new
ones  I  can  specialize  as
TP
um
and
uh  if  you  do  some  kind  of  similar
analysis  and  now  okay  there's  a  slighty
tricky  issue  that  I  can  dis  send  Q  to
any  fractional  power  of  P  here  uh  this
can  be  rectified  by  long  the  half  to  VAR
instead
um  and  what  you  see  from  this  is  that
for  every  Alpha  that's  in  the  non  meting
case  allowed  to  be  anything  between  zero
and
infinity  so  in  other  words  again  for  any
point  the  bage  Spectrum  really  of  the
integers
um  we  get  an  alpha  liquid
restructure  on  QP
um  again  the  pr  modules  can  be  described
and  the  formula  is  just  the
same
Q  the  op  liqu  one  and  it's  again  Union
over  less  Al  in  the  unit  of  all
constants  foring  a
norm  of  the  three  modules  on  the
si  uh  which  you  now  endow  with  the
exactly
same  as  in
the  uh  and  this  should  be  compared  to
uh  the
solidification  this  is  just  described
similar  way  as  limit  Union  of  all
constant  of  limit  of  joint  s  i  but  now
on  this  guys  we  put  the  infin  Infinity
the
Supreme  so  these  are  all  sub  space  in
here
so  suddenly  you  get  a  whole  new  world  of
and  liqu  structures  even  over  periodic
numbers  way  maybe  you  didn't  feel  like
you  need  them  because  the  Sol  Theory
already  works  with
you
so  I  want  to  ask  a  small  question  just
to  clarify  possible  confusion  so  you
have  the  the  L  beta  for  different  beta
which  are  again  I  believe  defined  in  the
same  way  just  by  some  of  the  beta's
power  without  taking  one  over  beta  just
this  in  the  same  way  a  slight  mismatch
what  I  know  if  I  literally  specialize
the  definition  OFA  to  Infinity  I  would
be  summing  the  supremum  Norms  which  is
not  what  I'm  doing  when  I  do  this
Infinity  here  the
Supreme  here  it  is  the  the
actual  there's  a  little  bit  of  a
mismatch  notation  between  here  and  here
but  I  think  okay  but  what  is  the  when
you  take  Union  over  beta  less  than  Alpha
I  think  usually  it  was  a  filtered  Union
that  is  is  it  the  case  because  here  you
have  to  be  slightly  careful  that  the  way
you  IND  index  the  constants  some  should
change  when  you  increase  ah  so  yeah  so
so  so  why  why  the  the  the  limit  the
union  over  beta  what  are  the
inequalities  between  the  L  beta  here  the
Tak  unit  over  all  C  yes  and  then  and
then  there  it's  literally  the  case  that
these  spes  get  larger  and  larger  as  you
increase
beta
ah  okay  braet  it  like  this  in  some
sense
okay  uh
so  something  you  can  think  of  a  whole
series  of  series  that  go  from  zero  all
the  way  to  Infinity  where  you  put  some
put  the  corresponding  Norm  here  where  at
this  end  you  have  follow  modules  and
then  at  each  point  Alpha  here  you  have
the  alpha  liquid  ones
and  uh  I  in  terms  of  the  class  of
modules  uh  I  mean  being  solid  is  a  much
stricter  condition  than  being  Alpha
liquid  so
uh  um  the  class  of  modules  that  you
allow  here  it  becomes  larger  and  larger
as  you  uh  make  alpha  closer  to  zero  and
some  things  as  you  go  to  what  zero  here
you  have  all  condensed
object  because  if  you  kind  of  naively
put  some  kind  of  L  zero  Norm  there
meaning  that  there  there's  only  bounded
number  of  coefficients  that's  nonzero
and  then  you  put  some  bound  um  then  this
actually  recovers  a  free  condensed
module  without  any  competion  by  varant
of  what  I  said  at  the  beginning  about
the
integers
okay  all  right  uh  I  guess  it's  time  for
the
uh  for  the  uh  for  the  new  thing
um  right  so  back  in  back  then  we  were
trying  to  look  uh  find  natural
candidates  for  how  what  the  fre  compete
modules  could
be  and  then  it  was  some  hard  matter  of
proving  that  they  actually  Define  the
EXs  of  being  an  antic
brain
uh  uh  but
but  this  in  this
course  we  have  a  different
mindset  is  that  to  the
F  and  the
structure  uh  L  find  something  natural
the  morphisms  of  this  projective  object
P  that  you  want  to  be
morphisms
B  the  fre  out  no  sequence  uh
that
should  come  after
comption
and  because  in  life  being  groups  this  P
has  these  very  strong  properties  like
being  Compact  and  internally
projective  this  will  always  Define  drink
structure  so  has  become  extremely  easy
to  produce  analytical
instrues
and  here's  here's  an
example  I  mean  so  it  turns  out  that  to
Define  uh  the  taptic
Curve
you  only  really  need  two
things
um  the  first  is  that  P  should  be
topology
um  right  it'sin  new  and  should  be  unit
clear  and  both  of  these  have  clear
meaning  in  terms  of  the  underline  cont
ring  right  I  mean  so  toing  newent  means
that  you  have  a  map  from  like  basically
this  guy  and  up  with  the
natural  natural  ring  structure
sending  one  t  to  C
um
but  then  you  need  some  you  need  some
completeness  for  your  modules  you  need
to  be  able  to  some  certain  sequences  and
really  the  only  thing  you  need  is  that
one  minus  Q  *
shift  acting  on  this  projective
p
and  it's  clear  that  there  is  a  universal
example  of  such  an  antic  ring  I  mean  you
just  take  the  free  guide  generate  by
unit  that's  just  cond  ring  and  then  this
condition  put  some  ring  structure  on
this
ring  also  sign  which  is
good  to
much
h
Ser  is  that  there  is  an  initial  such  ex
okay  Ian  existence  is  easy  the  hard  part
a
description  hey  so  it's  a  pair  of  a
triangle
and  you  can  actually  describe  a
triangle  and  well  let  me  describe  the
underlying  ring  um  this  is  precisely
those
those  those
sums  such  that  uh
the
PO  and  thus  if
you  like  the  claim  is  that  we  can  do
analytic  geometry  as  usual  over  such
rings
and  so  you  can  just  repeat  the
construction  to  T  the  curve  and  then  you
will  see  for  like  geometric  reasons  in
some  sense  that  it  must  live  or
precisely  this  ring  where  the
most  and  uh  I  can  affect  this  SC  of
three  the  free  modules  so  a  s  now
already  means  the  free  complete  guy
right  a  triangle  Jo  s  would  be  a  free
noncomplete  guy  but  a  triangle  is  free
complete
guy  it  is
is  the  union  overall  a  greater  than  zero
sorry  U  yeah  so  as  usual  as  is  a  limit
of
s
s
um
of  yeah  something
also
um  oh
limit
I
of  those
um  of  like  the  part  of  the
free  module  let's  say  was  long  sering  on
SI  but  where
uh  the
coefficient  of  Q  to  the
N  has  a  z
Norm  at
most  I  don't  know  n  plus
n
um  yeah  let  me  maybe  in  the  future
lecture  give  a  more  precise  description
but  basically  you're  asking  doing  a
similar  description  as  before  but  now
this  time  you're  asking  for  some  po  nor
gr  conditions  on  the
coefficients  oh  all  right  and
uh  so  what  you  have  written  on  the
Blackboard  is  barely  visible  to  in  the
all  the  indices  and  so  on  are  almost
invisible  so  you  have  Union  over  K
bigger  bigger  than  zero  or  bigger  than
or  equal  to  zero  I  mean  k  gets  large
okay  and  then  and  also  gets  large  um
basically  I  I  want  to  put  say  that  the
condition  here  is  uh  is  of  size  at
like  the  coent  of  Q  to  the  N  can  grow  at
most  like  a
polinomial  in  N  of  degree  K  so  most  n  to
the  K  but  then  I  also  have  to  allow  the
presence  of  negative  coefficient  so
that's  why  there's  some  kind  of  index  n
here  that  allows  me  to  put  negative  for
coefficients  so  anyway  then  the  liit  N
has  zero  Norm  at  n  plus  n  to  the  K  which
is
just
what  you  have  limit  of  over  M  or  I  or
what  I  in  the  I
okay  and  then  you  had  the  coefficient  of
Q  to  the  n  as  l  z  Norm  at  most  n  plus  n
to  the  k  n  plus  n  plus  n  to  the  K  what
is  n  plus  n  2
N  plus  n  to  the
k  n  n
the  but  n  plus  n  is  two
m  is  the  exponent  of
Q  ah  Q  to  the  m  ah  Q  to  the  m  okay  I
didn't  I  didn't  okay  I  to  the  okay
okay  thank
you  um  basically  if  if  you  want  your
free  you  want  the  ring  to  be  the  thing
where  you  have  Co  of  at  most  polom  grows
and  you  think  about  a  way  to  encode  that
on  the  modules  then  this  is  what  you
would  you  would  be  doing  um  Peter  do  you
really  mean  l  zero  Norm  or  L1  Norm  yeah
because  it's  it's  I  mean  these  are  again
it's  about  a  free  module  on  DSi  so  yeah
but  I  mean  if  you  want  the  correct
answer  for  if  you  want  the  correct
answer  for  S  is  a  point  I  mean  it
seems  yeah  me  yeah  I  don't  know  let's
say  yeah  one  yeah  I  think  so  I  mean  for
me  like  on  theet  guy  all  L  zero  and  L1
of  some
saying  but  yes  let's  say  I
one
um  right
uh  right  so  yeah  so  within  like  doing
analytic  geometry  over  this  space  you
will  get  a  like  geometric  way  to
construct  the  tape  curve  really  over  the
kind  of  smallest  Ringle  which  is
actually  the
sun
um
uh  this  actually  also  means  that  if  you
specialize  back  uh  to  liopic  numbers
then  it  turns  out  that  the  Ganges  series
it's  somewhere  ah  I  didn't  yes  is  yeah
so
this  and  so  the  the  gas  research  it's
very  close  but  not  quite  at  zero
that  and  uh  it  is  a  fun  exercise  to  take
this  description  here  of  the  fre  modules
and  base  change  it  to  QP  let  me  not
write  down  the  formula  it's  somewhat
nasty  uh  but  anyways  okay  let  me
startop
\end{unfinished}
