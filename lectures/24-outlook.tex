% !TeX root = ../AnalyticStacks.tex

\section{\ufs Outlook (Scholze)}

\url{https://www.youtube.com/watch?v=YKw1XaueLJY&list=PLx5f8IelFRgGmu6gmL-Kf_Rl_6Mm7juZO}
\renewcommand{\yt}[2]{\href{https://www.youtube.com/watch?v=YKw1XaueLJY&list=PLx5f8IelFRgGmu6gmL-Kf_Rl_6Mm7juZO&t=#1}{#2}}
\vspace{1em}

\begin{unfinished}{0:00}
  e
okay  so  so  so  welcome  to  the  last
lecture  um  so  so  so  today  I  want  to  give
some  kind  of  Outlook
uh  right  I  mean  so  with  with  Dustin's
lecture  on  Wednesday  we  kind  of  we  kind
of  finished  what  we  promised  in  the
first  lecture
um  and  so  today  I  want  to  talk  about
some  directions  one  could  go  in  with
with  the  kind  of  Machinery  we  developed
um
uh  so  maybe  yeah
so  so  some  I  don't  know  five  or  six
years  ago
um
uh
like  I  mean  I  did  all  lot  of  pic
geometry  and  like  I  always  wanted  to
have  a  way  to  do  this  not  just
periodically  but  also  with  real  numbers
and  over  spy  but  it  was  always  clear  to
me  that  I  really  needed  a  completely  new
language  to  to  talk  about  these  things
um  and  so
this  is  I  mean  as  I  said  already  in  my
first  lecture  this  kind  of  the  reason
that  uh  I  was  really  putting  a  lot  of
effort  into  this  project  um  and  so  I
finally  have  the  feeling  that  we're
basically  have  now  the  language  that  we
we
um  we  always  wanted  and  then  now  is  a
sensible  question  to  just  try
to  uh  really  use  it  to  do  a  lot  of
things  um
and  and  then  so  it  appeared  to  me  that
actually  I  mean  there  was  this  original
goal  that  we  I  maybe  had  in  mind  for
what  the  series  should  do  um  but  on  the
other  it's  also  good  to  look  uh  in  other
areas  of  mathematics  what  how  the  theory
might  be  useful  and  so  I  and  there  I'm
not  really  really  competent  but  I  was
still  want  to  give  some  wake  ideas  that
I  think  might  be  worth  looking
at  okay  all  right  so  so  here  yeah  yes
some  possible  directions  and  I  will
start  somewh  from  the  most  welldeveloped
to  the  most
specul
um  all  so  something  then  something  this
is  not  just  a  possible  Direction  but
that  not  just  works  I  mean  so  we  do  have
now  a  general
theory  of  and
shes  not  just  no  finds  conditions
imposed  on  the  modules  uh  in  analytic
geometry  in  all  labers  of  analytic
geometry  and  all  of  these  different
flavors  of  analytic  geometry  are  also
unified  into  this  one  thing  which
just
um  and  what's  even  better  is  that  this
is  not  I  mean  this  is  really  a  full  six
fun  form  L  of
sixun  Street  uper  Street
T  and
uh  and  this  just  lets  you  play  a  lot  and
uh  in  particular  like  some  things  we
kind
of  looked  at  into  this  uh  using  this
formalism  is
that  well  actually  this  sounds  stupid
but  it's  actually  for  example  as  Dustin
explained  the  last  lecture  it's  actually
non-trivial  application  of  this  first  of
all  a  general  the  of  structure
sheet  without  any  Mysterion  or  otherwise
hypothesis  um  where  in  particular  like
for  B  spaces  even  those  kind  of  a  finite
type  over  the  integers  J  had  to  really
work  a  lot  to  Def  find  subject  just
saying
um  and  even  for  edit  spaces  there  was
this  kind  of  nonch  issues  which  are
resolv  things
so  you  have  to  Define  what  it
derived  well  but  in  our  I  mean  we  just
have  antic  St  they  come  with  the
structure  period
okay  um
uh
then  there  all  sorts  of  Gaga  theorems
that  you  can  reprove  but  you  can  also
prove  various  new  sorts  of  of  Gaga
serums  actually
um  uh  there  are  various  resolv  about
like  the  sand  of  vector
bundles  and  again  the
songs  B  silly  but  um  and  for  example
there's  this  famous  paper  of  greenfeld
about  like  infinite  dimensional  Vector
bundles  or  something  like  this  is  called
but  in  particular  he  proves  that  if  you
send  a  ring  R  to  the  like  vector  bundles
of  all  the  wrong  series  T  So  something
that  R  is  natural  if  you  think  about
Loop  groups  and  so  on  um  then  he  proves
some  theend  results  for  that  and  I  mean
it's  a  difficult  paper  and  uh  AR  Messi
was  recently  revisiting  this  uh  using
some  of  this  descendible
ideas  um  but  just  a  few  weeks  ago  chich
was  asking  me  about  yet  another  variant
of  of  of  that  kind  of  result  of  dril  and
whether  all  techniques  could  be  useful
for  proving  it  and  I  think  they  are  and
that's  just  a  very  simple  proof  using
this  kind  of  uh  General  formulism  that
we  have  that  it's  very  easy  to  prove
such
statements
um  uh  and  then  we  also  like  know  complex
course  on  complex  geometry  we
discussed
like  off  for  complex  man
and  and  such  things  are  kind
of  one  kind  approach  using  this
um  uh  one  thing  which  we  in  some  sense
still  haven't  quite  figured  out  but
we're  quite  optimistic  that  in  principle
could  be  done  is  also  to  prove
uh  uh  the  sing  index  theum  using  using
our
technology
um  related  to  these  last  points  on  I
mean  they  are  of  course  very  very
closely  related  to  the  notion  of  CAS
Theory  um
and  uh  right  and  so  one  thing  that  was
kind  of  missing  for  a  while  is  the
notion  of  the  cas  of  Analytics
spes
so  like  an  algebraic  geometry  this
defined  first  by  toal  and  then  for
General  schemes  by  Thomason  and  it
really  use  it  that  you  have  a  well
behaved  category  of  coherent  Chiefs  on
it  and  then  maybe  actually  stable
Infinity  category  of  per  complexes  and
then  you  can  Define  the  case  theory  of
that
but  going  to  analytic  geometry  there  was
the  issue  that  there  is  not  a  good
enough  category  of  modules  of  which  you
could  then  take  uh  apply  these
categorical  techniques  and  get  some  kind
of  case  Theory
um  and  using  our  technology  can  you  can
actually  do  it  you  actually  don't  use
exactly  that  those  CL  Shields  as  there
but  some  varant  of  nuclear  modules  uh
but  this  has  not  been  analyzed  to  some
extent  so  this  definitely
uses  uh  the  work  of  Sasha  eimo  to  Define
Cas  of  dualizable  categories  um  and  then
if  you  your  analytic  space  are  actually
Ed  space
tic  spaces  this  was  worked  out  in  the
PHD  Ser
of
andf  um  and
there  we  had  some  problems  at  first  to
the  find  the  complex  numbers  but  Al  some
ideas  how  to  do  that  um  the  gas  here
might  actually  also  help  for
this  there  actually  also  some  other
relations  to  to  to  the  work  of  sasim  of
so  he  has  these  very  strong  results
about  the  category  of  localizing  motives
so  proving  that  it's  a  rigid  category
particular  dualizable  itself  um  and
using  this  you  can  actually  Define
certain  refined  variant  of  Clic  hology
topological  cycology  and  so  on  that  um
that  are  actually  not  taking  values  in
some  kind  of  complete  category  as
usually  when  you  take  as  one6  points  you
get  modules  over  some  power  Series  ring
which  are  complete  but  instead  you  can
go  to  nuclear  modules  in  our  sense  again
and  so  this  ties  back  in  this  uh  that
the  again  to  all  kind  of  an
geometry  um  let  me  mention  maybe  that  if
it's  okay  Dustin  that  Dustin  has  a  joint
project  with  brund  well  they  use  some  of
this  nuclear  Cas  Theory  to  settle  some
like  questions  uh  like  froma  homotopy
Theory  oh  yeah  but  we  figured  out  how  to
avoid  it  actually  ah  okay  yeah  too
bad
yeah  I  was  I  was  a  bit  disappointed  too
but  um  all  right  so  that  maybe  where
where  I'm  kind  of  coming  from  and  where
there  had  a  lot  of  applications  and
where  really  a  lot  of  work  has  already
been  done  is  in  the  area  of  like
thetical  modation  and  so
on
um
so  um
and  there  there  was  a  spetic  hology  and
then  which  was  defined  from  former
schemes  and  one  question  that  people  had
was  how  to  really  Define  it  for  uh  rigid
generic  fibers  um  and  this  then  very
much  is  related  to  analytic  uh
stats  in  our  sense  so  in  particular
there  is  no  what's  called  the  analytic  R
st  uh  defin  the  work  of  what  gu
um  uh  so  which  gives  you  way  to  talk
about  uh  six  fun  formalism  on  what's
classically  known  as  like  dcat  module  so
there's  a  certain  completion  of  the  Ring
of  differential  operators  giving  some
kind  of  differential  operators  of
infinite  order  um  defined  by  aov  and  one
SL  and
then  they  suggested  that  when  if  you
Works  someone  getting  anal  geometry
should  really  look  at  these  uh  modes
over  these  dcap  modules  and  try  to  get  a
six  formalism  for  that  but  as  usual  I
mean  you  run  into  something  of  function
analysis  issues  doing  so
uh  yeah  but  but  using  our  analytic
geometry  can  really  handle  this  and  Ro  G
was  able  to  write  down  I  in  some  sense
by  specializing  the  six  fun  formul  to
specific  St  get  a  very  general  six  fun
formul  for  these  uh  DK
modules
um  uh  so  this  is  related  to  D  modules
there's  also  in  the  pic  world  the  analog
of  like  hi  hi  bundles  and  so  some  kind
of  Simpson  correspondence  uh  the  first
incarnation  of  which  goes  back  to  Dinger
and
fings  um  and  so  these  can  also  be
interpreted  as  like  in  terms  of  a  stack
and  so  that's  there  so  called  Analytics
St
St
um  so  this  you  can  find  in  particular
there  like  okay  maybe  not  yet  using  our
technology  just  uh  essentially  work  of
aner
and  and  H  in  particular  recently
obtained  really  strong  results  on  the  P
correspondence
and  uh  this  work  on  these  hotate  staks
is  also  very  closely  related  uh  to
what's  known
as  geometric  10
the  uh  so  something  that  originally
arose  from  Sensi  which  is  something
about  P  go
representations  um  and  which  has  been
used  to  very  great  effect  uh  by  Lou  pun
and  also
ralo
um  for  applications  to  P
point  so  also  last  Friday  Vin  Pon  gave  a
talk  about  his  joint  work  with
boxer  uh  Kary  and  G  I  believe  um  where
is  a
uh  um  prove  uh  modularity  of  Genus  2
curves  of  a  q  uh  in  many  cases  and  and
some  of  the  key  technical  part  of  this
proof  is  really  using  this  kind  of
Technology
um
and  what  also  is  pointing  to  is  really
that  there  should  be
some  uh  version  of  like
theal  for  loc
representations  yeah  in  terms  of  some
kind  of  geometric  Lance  of  the
center  so  some  version  of  what  I  did  in
my  paper  was  long
thought
mention  here  I  there  something  that's  in
some  sense  combining  or  just
different  the  and  St  St  so  is  also  from
antic
citiation  and  now  was  all  I  mean  all
these  people  that  I  mentioned  um  let's
hope  that  one  can  uh  formulate  uh  such
thing
um  uh  which  maybe  a  proposal  for
something  this  direction  I  this  proposal
by  Helman
by  um  and  by  uh  and
G  in  this
direction  and
so  I  mean  this  of  course  maybe  what  my
original  interest  is
that  there  was  this  work  with  spk  on
deration  of  local  Lance  and  I  would  like
to  formulate  all  inations  of  local  Lance
uh  in  such
terms  and  eventually  then  also  Global  LS
um  and  so  this  is  just  to  a  large  extent
I  mean  at  least  work  in  progress  and
this  is  probably  already  very
speculative  but  this  is  something  that's
very  actively
investigated  um
and
so  actually  a  lot  of  progress  on  this
was  made  during  this  uh  house  trimester
that  happened  last  summer  here  in  bone
and  in  particular  we  discussed  a  lot
about  these  things  and  then  at  some
point  we
realized  that  now  that  we  understand  the
pedix  story  really  well  and  that  we
understand  uh  the  kind  of  correct
geometric  language  to  phrase  these
things  in  we  can  just  basically
one-on-one  translate  all  the  ingredients
in  the  pedig  world  to  the  real  real
world  so  there  there's  a
real  analog  of  virtually
everything  that's  happening
three  so  so  there's  also  an
analytic
St  and  that's  actually  a  funny  version
that  is  actually  isomorphic  to  B
St  uh  as  it  so  happens  in  this  casee
where  this  is  some  ination  of  a  re  H
correspondence  uh  you  can  also  Define
an  an  analytic
P  spe  which  in  this  case  actually  maybe
there's  some  kind  of  here  it's  this
non-trivial  JP  I  think  it's  a  trial
J
okay  and  again  there's  some  this
analization  of  10  curve  also  here  you
can
find  I  don't
know  yeah  so  there's  also  some  some
analytics  stack
mization  in  some  sense
encodes  the  vector  BCE  on  it  encodes
some  kind  of  peric  variations  of  f
structures  and  so  there's  also  an
analytic  stack
including  variations  of  twist
structures
um  so  I  mean  of  course  this  ties  in  very
well  with  all  the  work  of  Simpson
in  this  world  and  then  mauki  developed
this  really  developed  the  Ser  variations
of  f
structures  um  which  are  generalization
of  variations  of
structures  there's  a  certain  action  of
new1  on  on  these  things  and  you  want
objects  um
and  and  then  it  seems  to  be  possible  so
synthesize  everything  and  get  also  get  a
formulation  of  real  local  Lang  lines
local  correspondence
for  real  Le
groups
um  uh  for  some  kind
of  locally  antic
representations
good  C  can  I  ask  a  question  yes  uh  can
you  produce  the  weight  filtration  too  in
the  analytics  stack  um  it's  a  very  good
question  let  me  comment  on  this  in  a
second  um  not
yet
but
okay  some  kind  of  geometric  Lang  on
twist1  which  is
uh  a  kind  of  real  analog
of  so  I  give  a  talk  in  Minster  three
months  ago  so  um  where  I  uh  was
outlining  the  general  form  that  we
should  take  um  and  I  will  give  some  uh
three  lectures  actually  in  Princeton  uh
a  months  from  now  or  might  say  a  bit
more  about  how's  this  supposed  to  work
um
actually  part  of  this  again  it's  maybe  a
small  thing  but  maybe  not  I  mean  um  so
usually  when  you  talk  about
representations  of  of  G  of
R  you  run  into  all  sorts  of  functional
analysis  issues  again  and  usually  you
replace  them  by  more  algebraic  notion  of
thek
modules  by  passing  to  the  uh  vectors
finite  vectors  and  some  compact  subgroup
but  then  the
theory  somewh  less  inv  variant  because
you  need  needed  to  choose  this  K  and
well  it  becomes  more  algebraic
but  um  in  all  worlds  there's  really  not
much  of  an  issue  of  really  uh
encoding  representations  of  a  real  group
Direction  but  we  don't  need  we  don't
need
those  but  you  have  you  have  many
representations  which  have  the  same
Chandra  modules  because  you  can  use
different  function
maybe
maybe  this  question  comes  up  so  so  what
we  actually  do  is  we  will  look  at  so  the
real  group  it's  a  real
analytic  it's  an  group  object  in  real
analytic  manifolds  and  so  you  can  do  the
kind  of  real  analytic
ination  antic  St  and  so  this  a  group
object  in  or  analytic  STS  and  you  can
take  the  classifying  space  of  this  so
let's  say  St
to  guess  this  complex  number  is  uh  not
that  um  oh  real
uh  and
so  I  mean  as  usual  like  she  on
classifying  spe  there  are  something  like
representations  of  of  the  group  uh  but
to  realize  them  representations  you  need
to  go  to  back
to  uh  the  point  where  and  there  two
funter  so  there's  a  projection  from  the
point  to  here  and  you  can  take  either  P
or  P  uper
streak  and  like  for  any  of  these  usual
GK  modules  there's  a  canonical  object  in
here
actually  and  then  the  P  upper  star
should  I  mean  this  is  something  we
haven't  checked  yet  but  the  P  should
produce  a  minimal  globalization  the  P
Street  should  produce  a  maximal
globalization  so  what  C  you  take  you
take  C  with  the  gas  yeah  you  can  take
the  gasas  it's  enough  and  so  you  have
the
the  of  course  this  is  closely  related  to
some  results  on  existence  of  analytic
vectors  in  in
representations  it  should  you  should
otherwise  you  would  not  do  you  do  do  you
use  such  I  mean  the  the  fact  there
enough  analytic
vectors  let  me  not  try  to  say  anything
precise  about  this  relation  here  because
something  I  still  need  to  uh  think  much
more  about
um
um  right  uh
so  so  there  was  this  question  about
waste  structures  um  and  I  believe  it's
related  to  the  following
so  so  now  we  run  into  the  speculative
Reon  really  uh  so  for  these
things  I'm  pretty  confident  that  some
version  of  this  will  work  out  uh  this
this  is  at  this  point  more  of  a
speculation  but
I  have  a  very  strong  belief  in  it  but
it's  a  speculation
um  uh  so  classically  in  all  sorts  of
questions  about  function  analysis  and  I
don't  know  compex  geometry  and
everything  uh  you  often  you  really  need
to  put  metrics  on  something  like  if  you
really  want  to  prove  the  H  decomposition
at  some  point  you  need  to  do  some  L2
stuff  put  metrics  on  stuff  and  so  on  um
this  is  something  that  we  cannot  yet
Incorporated  all  or  what  we  yeah  that  we
cannot  yet  translate  into  our  our  world
everything  that's  related  to  metrics  at
this  point  but  I  believe  I  it's  clear
way  to  look  um  namely  you  should  look  at
this  extended  ver  F  that  we
had  um  I  will  connect  this  in  a  second
to  this  question  about  weight  structures
okay
so  uh  so  weall  this  bur  space  and  maybe
the  way  do  was  WR  so  you  have  some  kind
of  Central  Point  related  to  like  Ral
numbers  and  then  there  was  a  r  let's  say
for  Q2  and  then  at  the  end  of  this  you
have  kind  of
F2  and  then  there  was  a  rate  for  g3  at
the  end  of  the  G
F3  all  the  other  primes  and  then  there
was  also  theum
Prim  corresponding  to  the
reals  um  and  now
usually  like  in  the  bir  space  this  kind
of  ends  in  the  middle  because  you  asked
for  a  triangle  inequality  but  we  don't
have  to  do  that  and  so  there  is  no  Point
add  infinity
here
and  there's  this  point  at  infinity  and  I
mean  our  of  geometry  kind  of  tells  you
that  this  point  must  be  related  to
metric
and  uh  to  some  extent  you  can  already
see  that  like  when
you  but  just  just  some  fragments  of  it
but  also  by  analogy  here  like  like  if
you  work  two  rically  then  like  extending
over  this  point  precisely  means  that  you
put  some  kind  of  like  a  vector  bundle
over  this  thing  like  a
Z2  mod  over  the  integral  toic  numbers
and  so  extending  over  bun  pic  places  is
definitely  putting  some  pic  metric  on
things  and  so  it's  very  sensible  to
think  that  whatever  exactly  happens  here
it  must  have  something  to  do  with
putting
metrics
uh  um  metrics  on  some  kind  of  real  stuff
um
and  uh  yeah  so  so  some  question  we  have
in  our  mind  is  uh  whether  there  is  a  way
to  like  prove  the  object  composition
like  for  complex  K  manifolds  or
something  like  this  using  some  geometry
that  will  involve  this  extra  point
here
um  in  this  abstract  language  this  kind
of  difficult  for  me  to  think  about  but
um  we  actually  have  a  very  good  analogy
uh  again  I  mean  we  can  again  use  this
anology  between  three  and  four  um  so
periodically  um  so  I  said  that  there's
something  about  Lo  representations  and
so  if  you  work  over  this  kind  of  per
part  of  this
picture
um  then  over  this  part  of  the  bur  space
um  like  over  the  opening  part  you  can
Define  comple  tic  space  our  sense  called
QP  locally  analytic  um  this  corresponds
to
the  what  union  over
Pusa  where  Z  is  really  just  the  analytic
spectrum  of  the  locally  ENC  tun  from  DP
to
GP  so  ones  that  are  locally  developable
you  can  locally  develop  into  Power
series
expansion
um  and  this  kind  of  comes  up  very
naturally  in  all  these  investigations
there  and  we  know  that  this  analytic
space  which  lives  over  the  open  part
because  naturally  has  K  coefficients
this  has  Canal
extension  um  what  was  the  F2
Point  uh  and  so  this  is  very  much
related  to  a  theory  of
locally
antic  long  series  two
representations  uh  I  mean  these  are
these  are  the  coefficients  of  the  groups
of  some
PP  um  some  at  least  some  fragments  of
which  you  can  find  someone  on  the
literature
um  so
periodically  there  is  some  kind  of  yeah
so  this  Canon  is  actually  a  little  bit
subtle  to  write  down  it  of  some  divided
Powers
um  uh  but  exist  is  very  important  for
this  uh  P
story
and  one  thing  this
suggests  and  which  I  don't  yet  know  how
to
do  um  is  that  if  you  look  at  the  the
real  part  of  the  picture
real  and  then  close  on  Infinity  um  then
over  here  again  you  also  have  the  real
numbers  um  like  as  a  real  antic  space  so
the  thing  that's  covered
by  yeah  so  there  r  as  a  real  antic
manifold  and  again  you
locally  uh  the  thing  with  the  functions
are  the  real  analytic  functions  so  the
ones  that  are  locally  developable  and
locally  be  developed  to  power  series
expansion  um  something  that  match  to  the
base  and  I  mean  it  match  to  the  to  the
open  part  of  the  base  um  but  like  each
each  fi  over  a  point  is  this  one  um  and
it  suggest  that  this  should  this  should
extend  canally  over  the
function  yeah  canally  over  close  Point
Infinity
um  in  a
way  I  don't  yet  know  how  to  really  think
about  um  but  this  also  suggest  that  if
you  have  some  like  I  mean  if  this  could
be  done  and  this  would  be  some  kind  of
ring  object  then  also  this  real  antic
group  would  flips  over  like  oh  yes  this
should  also  should  also
expend
so  so  here  you  put  you  use  the
G  at  all  points  you  use  the  Gaz
structure  or  use  Liquid  for  different  uh
no  we
don't  I  will  come  to  liquid  structure
later  today  for  now  it's  not  needed  for
anything  so  you  just  use  the  same  the
same  at  all  points  of  the  yeah  so  it
yeah  so  acting  on  this  again  you  have
the  rescaling  action  so  there's  a  lot  of
different  copies  of  the  real  numbers  now
sorry  for  that  um  you  a  raling  action  of
the  nor  every  single
V  but  but  and  it  maps  to  the
belage  in  in  your  sense  to  the  bovich
space  I  mean  to  the  analytic  stack  of
the  so  this  yeah  so  this  this  maps  to
the  space  of  norms  and  it's  an  open
Subspace  of  the  space  of  norms  and  over
there  you  have
this  ST  which
is  like  real  of  locally  antic  antic
thing  which  lives  up  over  the  open  part
over  the  open  Ray  but  then  there  should
be  a  way  to  canonically
extent  so  but  properly  speaking  you  mean
rla  cross  01  right  or
Z
and
so  I  expect  that  whatever  kind  of  group
that  is
in  maybe  representations  of  this  have
some  kind  of  metric  structure  attached
to  them  I  don't  really  know  like  that  if
you  want  extend  representation  over  the
open  part  of  the  punct  I  would  expect
this  something  to  putting  a  metric  on  it
but
I  don't
know  these  are  some  objects  that  also
exist
uh  that  I  don't  yet  know
um  all  right  maybe  let  me
mention  ah  I  mean  also  I  mean  this  some
like  and  you  can  also  just  just  try  to
understand  like  we  can  look  at  analytics
text  over  the  space  of  Longs  that  really
map  to  the  close  Point  infinity  and  try
to  understand  what  the  kind  of  geometry
is  there  and  this  is  some  very  peculiar
geometry  where  you  is  it's  still  about
some  kind  of  real  complex  manifolds  or
something  like  that  but  you  are  able  to
localize  much  much  more  you're  able  to
really  zoom  in  inally  into  your  space  um
and  so  I  think  it's  very  interesting  to
try  to  investigate  what  geometry  this
point  looks  like  I  can  all
agree  so  we  can  zoom
in  and  some
it's  a  very
different  so  so  um  so  I  mean  you  have  to
for  example  you  have  to  us  REM
sphere  and  then  the  bounded  part  of  the
REM  sphere  is
actually  just  what  seems  to  be  just  the
Clos  unit  disc  the  bounded  part  is  some
kind  of  weird
overconvergent  minimally  over  convergent
neighborhood  of  the  uh  of  the  close  unit
disc
is  essentially  just  the
CL  so  any  real  number  that's  bigger  than
one  is  an  unbounded  function  on  this
thing
so  I  think  it  will  take  a  while  to
figure  out  how  why  do  you  picture  the
sphere  sorry  Peter  why  do  you  picture
the
sphere  um  I  mean  actually  mean  we  were
always  looking  at  this  this  normal  like
T1  right  T1  Infinity  yeah  and  and  the  P1
over  like  this  whole
part  this  whole  part  still  lives  over
thetic  spectrum  of  our
gas  so  whatever  happens  over  this  point
it's
still  happening  also  over  the  gas  is
reals  but
then  can  do  some  kind  of  final
localization  there  so  particular  can  Tak
a  usual  keyb  over  gu  his  wheels  and  base
change  it  to  here  and  then  look  at  what
the  bounded  part  is  it's  prob
that  but  if  you  remove  the  color  part  is
it  confirm  equivalent  to  the
plane  does  come  from
any  sorry  I  didn't  catch  the  question  if
you  remove  the  color  part  is  it  confirm
equivalent  to  the  upper  half
plane  half  random  question
uh  it's  not  really  base  change  from
anything  over
here  okay  okay  okay
uh
um  all  right  so  uh  I  think  that's  a
extremely  fascinating  Prospect  to
investigate  this  type  of
geometry  and  yeah  as  I  expect  it  will  be
it  should  be  related  to  all  these
questions  where  usually  you  put  metrics
on
stuff  all  right  so  that  was  five  uh
six  um
so  like  one  key  thing  that  our  formalism
allows  you  to  do  is  really  mix
situations  where  you  have  to  do  both
function  analysis  and  use  like  fancy
category
Theory  and  but  at  this  point  you  can  go
completely  fancy  so  I  mean  so  there's  a
theory
of  of  PS  of  higher
categories
Antics
text  so  like  to  any  fine  guy  I  mean  we
first  associate  just  the  drive  category
but  then  we  could  also
associate  uh  module  categories  over  it
but  then  as  I  was  explaining  last  Friday
a  Ser  of  present  Infinity  n  SP  due  to
stanage  um  and  so  you  can  just  so  you
can  do  two
PRL  just  continue  forever  and  I  mean  so
you  can  Define  n  PRL  on  X  for  for  any
X  and  you  cannot  just  Define  it  but
again  there's  some  you  have  all  the  kind
of  six  fun  on  it  and
so  you  can  just  play  with
it  so  the  existence  of  this  is  not  a
speculation  but  what  comes  next  is
definitely  a
speculation
so  okay  so  I  have  a  strange  project  with
stars  and  Z  where  they're  Computing  some
Quantum  CH  Sim  Varian  and  I  don't  know
what  they  are  but  they  seem  to  get
certain  power  that  seem  to  be  very
related  to  the  kind  of  periodic
structures  I'm  seeing  and  for  a  long
time  I'm  trying  to  make  sense  of
whatever  they're  doing
and
uh  but  now  recently  I  realized  that  I'm
able  to  send  about  higher
categories  and  that  well  the  C
hypothesis  tells  you  that  this  kind  of
to  fi  series  I  should  really  just  be
certain  higher  categories  and  then  try
to  understand  which  one  they  should  be
and  you  can  essentially  say  it  but  it
doesn't  quite  work
so  so  here's  what's  called  Quantum
transus  or  something
this  I  you  must  be  aware  that  I'm  saying
words  uh  that  I  don't  understand
um
uh  okay  so  so  so  extent  that  I
understand  anything  um  you  have  maybe
start  with  a  comple
group  and  maybe  forly  simple  or
something  like
this  relevant
um  and  then  you
fit  what's  called
level  I  mean  so  there  a  funny
computation  that  if  you  look
at  like  maybe  I  don't  know  let's  see
Simply  Connected  so  then  the  first
interesting  formology  group  of  G  is  a
third  formology  Group  which  I  mean  a
simple  assumption  z  um  and  yeah  so  let's
fix  the  level  which  is  just  an
element  um
and  so  here  G  is  just  considered  a
topological  space  but  then  you  also  have
kind  specifying  space  of  G
um  I  mean  gmap  to  be  Cub  so  this  if  you
want  the  net  from
G  cling  space  of  z  um  this  actually  D
Loops  it's  map  of  groups  automatically
obstru  manag  so  M
from  BG  to  B4  to  the  z  z
um  and
then  but  you  can  also  restrict  that  to  G
as  a  real  antic
thing
right  because  in  general  like  I  had  to
think  that  whenever  you  have  a  real
analytic  manifold  it  maps  to  the
Incarnation  of  M  like  is  a  condensed  set
kind  of
incarnation  and  this  basically  what  I'm
using
here  uh  and
so  but  then  there  like  okay  so  then
there's
also  like  the  exponential
sequence  analytic
yeah
thanks  um  have  exponential  sequence  so
everything's  living  here  over  like  let's
say  guess  is  complex  numbers  um  and  then
a  composite  map  to  be  to  the  four
of  GA  venes  because  for  here  chology  has
trivial  and  so  this  actually  lifts  to  be
cubed  of  the  analytic
GM  and  of  course  the  analytic  gmf  us
so  now  this  sounds  really  like  the  enet
um
all  right  but  so  so  so  what  does  it  mean
to  give  a  map  to  be  cubed
here  well  this  is  too  hard  for  me  to
think  about  but  um  so  map  to  bgm  well
that's  just  the  line  model  right  so  bgm
that's  just  a  classifying  space  for  line
bundles  then  b  square  GM  uh  this  is  some
giving  you  somei  algebra  or  something  so
this  giving  you  a  Twist  of  the  category
of  modules  and  then  the  Cub  GM  this
means  that  you  give  a  Twist  of  the
category  of  categories  is  over  over  it
yeah
so  so  this  snap  called  Alpha  from
BG  a  real  grp
qm
specifies  a
Twist  of  like
prlg  I  some  let's  call  it  just  l  or
invertible  it's  invertible  object  is  a
2pr
this  all  right  okay  so  have  this  uh  and
you  have  the  projection  just  to  the
point  or  the  point  is  like  the  guess
complex
numbers  and  I  think  what  people  do  is
like  okay  so  they
they  they  want  to  look  at  like  some
family  of  G  tsers  like  bundles  with  a
flat  connection  or  non  flat  connection
whatever  um  and  but  sometimes  Twisted  by
this  Alpha  and  so  I  mean  this  is  somehow
more  or  less  governed  by  pulling  back
this  L  under
Alpha  but  then  they  want  to  integrate
over  the  space  of  all  G  ches  so
basically  they're  taking
this  and  actually  g  a  finite  group
that's  literally  what  they're  doing  as  I
learned  from  paper  of  three  topkins  l
T  um  and  so  I  mean  this  just  makes  sense
and  gives  you  an  object  in
two
uh  and  then  you  can
Wonder  like  so  this  this  this  guy  here
it's  some  symmetric
monid
ceg  very  and  categories  from  n  plus  one
category
um  and  I  mean  Quantum  trans  S  Series
should  be  a  threedimensional  to  Quant  F
whatever  that  is  so  it  should  be  a  three
by  the  cism  hypothesis  it  should  be  a
three  dualizable  object
somewhere  uh  in  a  three  category  so  now
we  have  one  so  the  question  is  is  this
streetable  and  uh  as  any  expert  would
immedately  tell  you  this  there's  no
chance  of  working
uh
um  uh
because  the  space  of  conformal  blocks  I
believe  it's  called  um  there  you  must
put  some  holomorphicity  constraint  and
I'm  kind  of  not  doing  that  here
um  but  I  mean  in  some  sense  it's  not  so
far  so
but  it  is  true
duable  I  think  that's
that's  you  don't  don't  need  need  much  uh
okay  so  I  didn't  carefully  check  it  but
I  believe  it  is  to  dualizable  and  um  and
to  check  that  it  would  be  S  realizable
and  for  suiz  Ability  you  would  need  the
following
thingss
uh  you  would  need
that  pie  is  kind  of
from  proper  and
smooth  so  diagonal  of
Pi  proper  Smo  and  the  diagonal  diag  of  P
proper  Smo  so  there  are  six  conditions
to  check  and  only  one  of  them  fails  okay
so  so  more  concretely  this  is  like  the
point  the
point  this  one
reduces  St  change  to  the  grou  being
proper  and  smooth  and  this  means  that
the  inclusion  of  the  point  into
this  so  for  all  of  these  you  would  need
that  they  proper  that  they  smooth  smoo  I
mean  yeah  Comm  smooth  uh  any  guesses  for
which  of  the  six
fails  actually  the  one  you  probably
would  expect  the  least  is  the  smoothness
of  this
map  the  smoothness  of  the  second  yeah
only
just  a  digression  the  map  to  be  you  said
that  it  lifts  to  to  a  map  to  from  B4  Z
to  B3  GM  analytic  but  you  want  to  BGR
locally  analytic  instead  of  BG  so  the
obstruction  is  something  having  to  do
with  chology  with  coefficient  in
something  like  continuous  chology  but  it
is  in  your  language  so  I'm  not  sure  what
it  corresponds  to  but  it's  just  aent
chology  of  the  stack  basically  what
basically  just  Theology  of  the
stack  but  it  is  not  true  for  BG  itself
it  lift  or  I  mean  here  would  be  the
chology  of  kind  of  condensed  set  ST
which  would  actually  be  amount  chology
of  this  funny  St  is  singular
chology  so  if  you  would  go  here  to  be
for  those  complex  numbers  you  would  just
T  the  this  is  a  singular  commod  with  a
complex  number  so  this  one  wouldn't  lift
you  need  to  go
to  you  you  the  obstruction  is  is  has  to
do  with  with  maps
to
to  right  yeah  so  it's  some  kind  of  I
mean  up  to  this  analytic  it's  basically
coherent  but  on  this  classifying  St  as  a
condens  set  the  coherent  chology  is  kind
of  singular  chology  ah  okay  okay
and  it's  because  it's  a  compact  group
that  it  vanishes  on
the  go  to  the  lotic  yeah  but  the  compact
is  much  more  crucial  for  for  these  kind
of  problem  sols  we  can  hear
um  okay  so  this  doesn't  word  but  in  some
sense  I  feel  like  it  was  so  close  to
working  so  maybe  you  just  need  to  tweak
this  this  here  a  little  bit  and  as  I
said  in  five  uh  there's  actually
canonical  candidate  where  you  kind  of  go
to  the  Point  Infinity  maybe  that  one
works  very  naive
question
uh  doesn't
word  how
replacing
by
fiber  fiber  of  the
extension  first
and  it  seems  weird  that  this  should  put
the  kind  of  correct  holomorphicity
constraint  into  the  picture  but  I  think
the  formalism  might  just  work  out  to  do
that  and  there's  also  some  structure
you're  not  using  with  the  like
like
um  like  I  think  this  map  to  B3  GM  that
you
build  it  should  really  have  a  connection
I  mean  like  I  think  that  there  you  know
could  kind  of  be  going  to  deline  chology
and  weight  too
uhuh  yeah  so  so  maybe  this  is  not  yet
completely  correct  I  just  want  to  point
out  that  I  mean  you  can  just  play  with
these  things  and  you  can  hope  that  using
if  you  would  actually  understand  what
you're  really  trying  supposed  to  do  uh
you  could  write  down  something  that
would  actually  produce  something  spe
Lal  so  I  mean  usually  what  you  would  try
to  do  this  you  would  run  to  all  sorts  of
issues  that  you  always  want  to  do
functional  analysis  and  high  categories
and  I  mean  people  manage  to  do  a  lot  of
things  but  here  you  can  just  very
naively
try
um  all  right  so  this  brings  me  to  the
last  thing  I  want  to  talk  about  a  little
and  this  one  I  actually  want  to  uh  go
into  a  little  bit  more  details  and
actually  I  mean  this  was  very  fancy  hyro
and  now  it  becomes  bit  more  concrete
again
um  so  this  is  about  a  theory  of
uh  where  fundamental
proces  now  it  is  we  don't  see  very  well
it's  condensed  yes  cond  it'll
resolve
um  so  yeah  so  let  me  ask  a  precise
question  and  then  I  will  actually  answer
um  I  so  this  is  a  question  that  comes  up
in  particular  like  in  sleed  Geometry  for
GR  of  vit  variant  and  things  and  that  if
you  are  in  some  more  algebraic
situation  there  is  a  known  solution
since  some  20  30  years  uh  but  if  you're
really  in  the  analytic  context  it's  much
more  subtle  um
and  one  reason  I  looked  into  this  is
because  it's  a  natural  I  mean  these  kind
of  things  they're  doing  syic  geometry
they're  naturally  some  kind  of  spaces
that  are  simultaneously  kind  of  analytic
derived  stacky  and  so  on  and  so  this  is
the  kind  of  theory
where
our  situation  where  our  series  should  be
useful  and  so  I  just  try  to  see  to  what
extent  it  is  useful  and  uh  yeah  I  think
it  can  actually  do  something
um  okay  so  let's
consider  fiber  like  a  model  situation
that  want  May  interested  in  the
following  consider  a  fiber
product  uh  of  manys  M1
M2  next
um
of  let's  say  compact
Mo  say  oriented
comp  M  M1
M2  feel  free  to  assume  that  these  Ms  are
closed  immersions  I  don't  think  that's
relevant  um
but  I  mean  so  yeah  uh  consider  but  which
is  of  expected  Dimension
zero  in  other  words  like  d  M1  plus  d
M2  so  then  if  the  transl  intersection
turned  out  to  be  transverse  this  x  would
just  be  a  finite  set  of  points  oriented
points  actually  and  so  you  could  count
the  number  of  intersection
points
so  this  intersection
transfers
Fin  and  actually  also
oriented  so  any  any  point  knows  whether
it  should  be  count  as  plus  one  or  minus
one  and  so  so  you  get  a  well
defined  uh
counted  sign  count  of  the  elements  of
X  this  is  a  variant  under
under  so  long  as  you  stay  as
intersection  stays  transverse  and  okay
so  then  that's  a  setup  so  for  always  a
question  the  question  is  can
you  uh  Define  this  sign  count  of  X  in
general  I  mean  without  transfers
intersection  purely  intrinsically  on
X  of  course  every  connected  component
will  we  have  a  well  defined  like  in
Fulton's  Theory  every  connected
component  will  have  a  well  defined
number  which  is  the  part  of  the
intersection
number  coming  from  this  connected
component  yeah
sure  but  but  even  these  local  ones  you
need  to  Define  right  um  right  so  so  the
good  situation  I  know  mean  you  have  two
circles  meeting  transversely  and  then
this  plus  one  this  minus  one
intersection  zero
um  but  now  this  might  generated  I  you
might  have  two
circles  that  I  don't
know  some  some  situation  like  this  where
this  might  be  I  know  this  this
might
sitation  it  can  be  infinitely  conic  and
Stu  I  don't  know  I  mean  this  might  be
tangent  to  infinite  order  somewhere
might  be  the  same  for
then  cross  I  don't  know  all  sorts  of
really  funny  behavior  um  and
then
right  so  if  instead  of  compa  move
manifolds  you  have  some  kind  of  smooth
projective  varities  there  then  in  this
case  it's  known  how  to  do
this  but  also  like  the  intersection
cannot  be  as  bad  as  for  smooth  manifold
where  intersection  basically  as  a
topological  space  has  basically  no
structure
whatsoever
okay  so  let  me  first  uh  State
Z  yes
compute  the  F
product  in  our
C  so  in
particular  X  is  some  kind  of
deriv  um  and  this  time  it's  actually
somewhat  critical  to  do  this  not  over
the  gases  real  numbers  but  over  the
liquid  real
numbers  for  some  choic
ofid  structure  depends  on  parameter  that
doesn't  actually  matter  for  this
application
yeah  so  for  everything  else  we  did  in
our  course  this  kind  of  liquid  structure
that  we  once  produced  was  very  much
effort  not  so  relevant  but  here  I  think
it  actually  really  is  relevant  for
reason  I  will  explain  in  a
second  so  what's  the  so  yeah  so  we  can
Define  kind  of  notion
of  we  can  look  at  analytics  stext  in  our
setups  that  just  have  the  property  that
locally  it  is  possible  to  write  them
such  an  intersection  and  then  uh  one  can
can  only  produce  a  virtual  fundamental
clause
on  okay  so  but  uh  so  there  are  some
symplectic  experts  present  at  NP  in
particular  Nate  botman  and  K  barheim  and
uh  I  kind  of  discussed  this  a  little  bit
with  them  already  and  I  want  to  continue
those  discussions  um  but  in  particular
they  made  me  aware  that  like  even  in
this  kind  of  simple  space  it's  kind  of
surprising  that  this  should  work  uh  so
so  so  let  me  say  why  this  is  surprising
so  let's
consider  consider  the
case  where  like  those  M1  and  M2  are
Cur  there's  my  picture  over  there
um
uh  and  then  just  assume  you're  in  local
situation  where  you  kind  of  have  two
things  meeting  at  just  one
point  locally
intersection  with  one
point  and  let  try  to  analyze  what
happens  in  the  neighborhood  so  in  the
best  possible  scenario  this  intersection
is  transverse  you  just  get  a  point
counts  I  don't  know  plus  or  minus  one
depending  on  your  orientations  um  I
don't  know  then  the  next  wor  scenario  is
if  it's  a  square  function  then  okay  this
to  be  zero  and  if  it's  like  a  cubic
function  there  should  be  again  first  one
and  so  on  um  and
so  this  tells  you  that  if  it's  Vanishing
to  finite
order
uh  then  basically  it's  clear  that  we're
okay  right  because  we  just  need  to
remember  the  vanishing  order  of  that
function  and  this  will  tell  you  what  the
this  function  crosses  the  line  or  it
doesn't  uh
but
no
uh  but  now  let's  consider  a  situation  as
you  can  have  SM  manifolds  but  there  more
tangent  to  infinite  order  something  like
x  -  one  X2  but  then  you  can  have  a  very
similar  function  which
is  uh  s  of  X  time
x
and  so  what  I'm  claiming  is  that  the
logus  where  this  function  is
zero  whatever  that  means  uh  determines
whether  this  function  crosses  Al  line  or
not  the  claim
is  Vanishing  locus  of  these
functions  just  the  Vishing  Locus  uh
determines
I  mean  distinguishes  these  two  cases  in
particular  determines  whether  the
function  crosses  or
line  in  which
category  in  our  category
okay  so  let  me  make  it  slightly  more
explicit  that  me  like  like  in
classical  geometric  picture  is  quite
unclear  what  kind  of  structure  you  to
give  the  anything  I  mean  the
intersection  point  which  knows  it  it
must  in  some  sense  must  know  more  than
all  the  derivative  of  this  function
because  it  will  never  be  able  to  double
from  AO
derivatives  uh  but  it  knows  much  much
less  than  the  germ  of  this  function  it
really  only  knows  eventually  so
classically  would  maybe  try  to  use  the
whole  germ  of  this  function  in  order  to
remember  whether  proc  or
not
um
uh  right
so  I  so  what  is
locus  of  say  some  Infinity  function  from
R  to
R
uh  let's  assume  it  really  just  has  an
isolated
zero
Z  so  this  Vanishing  Locus  general  form
this  just  the  analytic
Spectrum  alra  Infinity  functions  from
modul  generated  by  f
where  is
this  this  thing
here  I
mean  it  is  a
liquid  special  type  of  condensed
are
uh  um  with  the  analytic  ring
structure  antic  ring  structure  doesn't
matter  um
it  is  a  very  funny  one  though  I  mean  if
F  would  only  manage  to  find  out  order
then  this  would  just  be  TR  poal
algebra  finish
in  and  so  this  is  some  nice  algebra  but
um  if  vanishes  to  infinite
order  then  this  is  some  funny
non-separated
thing  and  so  technically  you  would  have
trouble  and  this  with  a  topology  but  in
the  condens  world  I  mean  has  a  natural
cond  and
so
so  here's  a  proposition  that  tells  you
that  the  Spanish  loc  someone  knows  about
this  so  if  you  have  two  functions  at
above  and  the  exist  in
isomorphism  of  liquid
algebras  between  their
CS
you  mean  analytic  rings  with  the
liquid  well  I  mean  I  don't  need  to  put
the  anal  structure  on  them  because  it's
just  induced  one  so  I  can  really  just
look  at  them  as  liquid  albas  also  I  mean
f  is  still  non  zero  divisor  here  under
this  assumption  I  made  so  this  really
still  concentrating  degree  zero  just  not
house  T  so  I  don't  have  to
say
um  so  assume  that  these  are
isic
so  the  condens  set  of  C  Infinity  RR
means  that  you  you  have  to  Define  smooth
functions  from  R  cross  a  profite  set  to
R  you  do  it  in  the  usual  way  yes
IND  construction  just  completely
internally  in  the  condensed  world  and  it
will  produce  the  right  condensed
structure  or  just  remember  that  it's  a
FR  a  space  topological  thing  just  pass
to  condens  after  ver  those  give  you
right  and  the  liquid  thing  does  it
doesn't  change  the
condensed  liquid  is  just  a  condition
right  I  mean  any
of  so  it  satisfies  the  liquid  condition
I
mean  just  say
condensed
be
okay
um  so  assume
this  then  uh  actually  f  is  G  times  U  for
us  is  an
function  and  so  in  particular  being  in
veral  function  from  R  to  R  must  either
be  everywhere  positive  or  everywhere
negative  and  so  multiplying  by  such  a
unit  cannot  change  whether  the  function
crosses  the  line  or
not
no  no  no  but  what  happens  if  you  rescale
the  function  you  take  the  exponential  e
to  the  minus  one  X  squ  and  then  you  take
there  say  instead  of  x  constant  times  x
so  those  things  have  different  grow  rate
it  cannot  be  that  there
are  there  are  you  use  different
coordinates  on  R  it's  it  it  doesn't  seem
to  be  it  seems  fishy  because  the
isomorphism  is  abstract  of  condens  AR  is
abstract  right  right  but  still  it's
true  but  e  to  the  minus1  /x  S  and  E  to
Theus  one
over  let  us  say  maybe  it's  true  in  this
example  but  I  I  I  I  I  is  true  the  offer
it  is  true  so  let  me  give  a  pro
um  so  first  of
all  why  why  do  I  stress  the  liquid  so
much
um  uh  the  liquid  means  that  if  you  take
CN  functions  on  M1  and  tender  them  in
the  liquid
sense  with  liquid  real  numbers  with  CNP
functions  on  some  other  real  manifold
them  too  and  this  definitely  maps  to  C
functions  of  the
product  but  by  computation  that  we
spelled  out  uh  in  the  complexometry  not
I  believe  like  on
nuclear  nuclear  fet  spasis  um  the  liquid
tensor  product  is  doing  the  expected
thing  and  so  it  actually  is  producing
the  C  functions  on  the  product  um  but
it's  not
true  uh  for  the  overall
G  there  would  be  some  smaller
thing  but  concretely  what  does  it  mean  a
a  liquid  is  a  is  a  r
right
h  a  set  set  I  mean  r  is  just  the  usual
condensed  ring  yeah  real  numbers  but
then  you  put  different  ring  structure
there  which
is  much  closer  to  asking  that  like  the
building  blocks  are  some  kind  locally
convex  things  they're  not  quite  locally
convex  they're  just  Alpha  locally  conx
versus  Alpha  might  be  slightly  less  than
one  but  you're  very  close  to  locally
conx  setting  and  so  when  you  form  this  T
products  it  basically  allowing  all
locally  comx  combinations  and  at  least
if  you  kind  of  have  nuclear  transition
Maps  then  the
precise  kind  of  complexity  you  need
actually  it  doesn't  matter  one  limited
works  okay
so  this  is  the  deriv  T  of  product  it
would
be  works  so  this  is  a  non  computation
that  comes  out  right  in  the  liquid
World  um  and  so  from  this  you  can  deduce
that  if  you  take  C  Infinity  functions
from  R  to  r  that  this  is  actually  item
poent  over  RT
uh
is  a  usual  coordinate
yeah  so  maybe  theomorphism  should  be
condensed  our  algebra  with  the
coordinate  in  the  proposition  otherwise
I  don't  believe  if  you  don't  have  the
coordinate  so  Ian  you  can  just  no  I  mean
I  I  said  that  both  of  them  have  the
isolated  zero  at  zero  Peri  okay  both  of
them  just  have  0  8  Z  that's  what  I  use
but  I  don't  use  anything  more  than
that  let's  see  what  he's  doing  but  he's
not
all  right  so  let's  analyzation  so  let's
assume  we  just  have  an  abstract
Asm  uh  so  for  both  of  both  of  these
admit  unique  map  to  the  real
numbers  because  you  can  classify  all  the
maps  from  C  Infinity  R  to  R  they  just
given  by  real  numbers  uh  a  varation
that's  some  real  number  and  only  one  of
them  one  of  them  f  is  equal  to  zero  so
so  what  is  definitely  true  is  that  uh
they  have  this  unique  evalation  to  the
real  numbers  and  this  must  commute
because  in  those
cases  um  all  right  and
now  uh  I  have  R  joint  key  uh  equal  to
zero  nothing  to  here  why  is  the  usual
coordinate
um  right  I  me  maybe  I  should  Infinity  r
r
here
and  so  uh  I  get  there  and  so  where  does
this  T  go
to  well  I  mean  let's  assume  G  is  not
just  also  uh  Vanishing  first  order
then  like  I  mean  you  have  this  unique
map  and  you  also  have  to  Unique  first
interal  neighborhood  so  this  must  be  a
function  that  it's  not  managing  to  first
order  here  do  you  actually  care  about
this  maybe  I  don't  care  um
all  right  so  I  have  this  map  uh  but  okay
I  also  have  this  compos
map  and  of  course  it  has  a  quotient
here
and  if  I  want  I  can  also  lift  to  here
right  because  I  just  need  to  lift  my
element  T  from  here  to  here  and  I  can  do
that
um  yeah  so  you  get  it  f  is  equal  to  G  *
unit  up  to  a  change  of
coordinate  so  it's  it's  G  times  the
change  of  coordinate  compos  a  coordinate
change  times  the  unit
change
okay  let  me  to  this  algebra  here  just  I
mean  let  call  this  here  A  and  this  here
B  um  so  one  I  want  to  CLA  that  this  met
from  R  joint
ke
be
um
so  this  is  a  usual  um  there  exists  the
unique  extension  which  is  just  a
condition
okay
and  uh  so  is  a  module  same
and  there  is  a  unique  extension  in  which
sense  in  the  sense  abstract  Rings  or  or
I
mean  this  sense  here  right  so  it  can  be
at  most  one  extension  okay  saying  um  and
I  claim  it  exists  and  for  this  I  can
it's  enough  to  check  that  it  exists  here
but  I  can  classify  all  M  from  joint  T  to
here  and  they  all  all  of  them  do  expand
right  okay  they  do  extend  by  just  comp
by  comp  okay  all
right  and  there  it's  okay  and  you  really
just  need  the  extension  is  necessary  you
need  to  just  need  to  prove  existence  but
this  map  is  given  by  some  C  Infinity
function  from  R  to  R  but  then  any  other
C  Infinity  function  from  R  to  R  can  just
be  evaluated  there  so  that
works
okay  so
uh  so  this  means  that
this  map  from  here  to  here  it's  factors
uniquely  to  here  so  you  actually  get
this  map  such  a  map
here  where  this
commutes  but  this  is  to  this  as  this  is
also  exposure
right
um  okay  so  so  maybe  this  a  real
parameterization  that  g  wants  me  uh  to
incorporate  all  this
time  uh
t
sorry
um
and  at  this  point  basically  like  both  of
these  are  compatibly  equion  of  these
function  from  R  to  R  so  and  then  the
ideal  must  so  this  presentation  must  be
the  same
uh
uh
anyway
I  so  I  think  from  there  you  can  conclude
at  least  that  this  property  about  F
crossing  the  line  is  only  deep  crossing
the  line  because  uh  so  there  is  n's
presentation  not  saying  it  right  but  um
let  make  a  m  point  here  so  there  are
some  existing  theorems  series  of  um  like
derived  manifolds  uh  but  they  proceed
very  differently  and  particular  they
consider  some  kind  of  alra  of  functions
where  by  the  definition  whenever  you
have  a  function  then  uh  you  can  kind  of
compose  with  any  c  Infinity  function
from  R  to  R  this  is  some  kind  of  data  I
put  on  the  Rings  but  this  observation  is
telling  you  that  these  kind  of  funny
derived  potions  that  we're  getting  they
have  just  have
happen  to  have  the  property  that
whenever  you  have  a  function  you  can
always  canonically  extend  uh  to  cinity
function
so  they  acquire  the  structure  that  for
any  function  you  can  compose  with  a
smooth
function  um  but  you  don't  need  to  put  it
have  to  put  it  in  in  the
beginning
try  to  see  if  can  Solage  was  trying  to
save
um
this  is  the
canonical  put  that
um  that  you  use  that  probably  and  I  mean
those  things  can  be  classified  these  are
just  given  by  usual  new  function  from  R
to
R  um  and  so  then  okay  so  has  a  ization
and  so  maybe  you  have  to  put  a
reparameterization  in  uh  then  then  just
makes  a
similar  sorry  I  want  to  say  the  same
thing  it's  faithful
embedding
H  so  this  just  pullback  by  right  so  yeah
okay  maybe
up
okay
so  right  but  so  some  of  the  thing  that
makes  this  work  is  precisely  this  thing
that  you  do  get  these  unique  extensions
to  cinity  function  because
otherwise  uh  you  cannot  control  the  kind
of  cin  from  here  to  the  cin  section  from
here  you  wouldn't  be  able  to  compare
all  right
um  all  right  so  so  so  let  me  know  State
some  more  objects
here
um  let's  say  as  locally  compact  store
space
and  let's  assume  also  it's  a  five
dimensional  which  is  always  satisfied
when  it's  intersection  like  that  um
uh  plus
acheve  of  animated  all
together
ohos  so  that  is  local  e
intersection  slly  isomorphic  but  not
with  a  given  data  so  this  is  just  a
condition  uh
intersection
um  then  one
canine  so  this  is  a  family  depending  on
on  S  of  smooth  manifer  or  just  constant
intersection  of  smooth
manifolds  the  intersection  of  smooth
manifolds  is  constant  locally  or  is  it
varying
continuously  no  I  mean  locally  you  can
write  open  substance  of  s  can  be  written
as  the
fin  where  the  that's  so  coming  from  the
point  yeah
yeah  ah  okay  s  itself  is  okay  s  s  this
is
okay  then  one  can  define  define
uh  an  expected
Dimension  um  there  some  C  uh  it's  a
locally  constant  function  this  integer
coefficient  um  and
orientation  Omega  s  so  this  is  a  zero
persistent
on
um  right
and  uh  okay  so  let  me  do  not  by  pi  from
s  to  Star  just  the  projection  of  locally
contact  po
spaces  and  virtual  fundamental  T  of  like
s  equ  structure  sheet  Al  also  this  this
depends  on
S  all  all  of  these  depend  on  S
um  which  is  a  global  section  on  S  of  the
dualizing
complex
T  this  this  makes  sense
yes  this  makes  a  lot  of  sense  and  is
because  in
particular  if  the  expected  diine  is
zero
um  as  is
Compact  and  uh  and  it  choose  an
orientation  oh  my  God
fix  I  mean  assum  exist  and  fixed  one  um
then  you  can
Define  the  Fundamental
class  Global  fundamental  class  as  some
of  the  integral  of  the  virtual
fundamental
class
Global  I  know  the  count  signed  count  of
SOS  as  as  the  integral  of  the  virtual
fundamental
class  over
s
okay  so  I  mean  under  G  equal  to  zero  in
this  orientation  this  goes  away  and  then
you  just  and  as  compact  you  have  a  trace
map  that  goes  back  to  the
integers
SC
all  right  um  so  let  me  sketch  the
construction  um  and  so  basically  this  is
just  the  direct
analog  uh  of  the  algebra  geometric
construction
um  so  I  think  this  was  pioneered  in  the
work  of  uh
and
um  uh  then  many  many  people  worked  on
this  and  I'm  I'm  not  an  expert  on  this
at  all  um  the  paper  where  I  learned  it
from  some  kind  of  algebra  I  mean
lination  I  know  by  a  um  but  the  ideas  I
think  will  maybe
um  so  so  what  they  Define  first  of  all
is  this  uh  what  they  I  think  called  the
intrinsic  normal
cone  um  and  so  this  realiz
uh  so  there  some  kind  of  intrinsic
normal  con  over  s
um  uh  that  describes  so  the  fibers  some
of  the
fibers  some  of  this  corresponds  to  the
cotangent
complex
um  of
like  over
R  um  and
uh  so  in  our  world  I  mean  first  of  all
you  can  somehow  Define  such  a  derived
intersection  of  like  smooth  things  but
also  if  you  compute  what  this  kind  of
cotangent  complex  does  that  can  like
abstractly  defined  for  C  Infinity
functions  basically  for  the  same  reason
that  the  T  products  came  out  right  um
also  the  Cent  complex  comes  out  right
for  C  Infinity  functions  on  a  manifold
and  then  for  such  dve  T  part  of  course
you  get  S  so  this  is  actually  a  perfect
complex  so  what  what's  the  superscript
on  the
L  uh
liquid  and  so  uh  here  some
kind  so  actually  I  just  want  to  Def  find
this  here  as  a  locally  comp  c
space  uh  sorry
St  let's  say  it's  a  condensa  so  some
kind  of  stacky
version  um  where  uh  some  tangent
directions  give  you  st
directions
I  mean  it's  not  really  don't  need  to  go
all  the
way  only  group  you  don't  need
High  I  don't  need  High
yeah  so
um  so  four  for  ah  because  the  L  itself
is  concentrated  in  the  okay
okay  yeah  sorry  perfect  amplitude  uh
amplitude
01  so  the  already  did  it
okay  uh  for  comp  also  space  or  profile
set  if  you  want  um  T
um  well  first  of  all  from  T  to
S
uh  you  can  just  regard  this  as  a  as  a
map  from
aspect  of  the  continuous  functions  from
T  to
R  to
some  let  me  call  it  unexpect  SOS  um  so
that  you  can  by  locally  the  spectrum  of
this  this  algebra  you  can  build  an
analytic  stack  in  all  sense  and  then  if
you  evaluate  this  on  just  the  continuous
functions  um  you  just  get  um  Ms  from  T
to  the  underlying  set  um  but  now  I  want
to  Define  what  what  are  the  maps  from  T2
this  intrinsic
pH  and  this  can  actually  be  defined  as  a
map  from  a  funny  algebra  it's  you  Tak
the  this  guy  and  you  model
than  by
zero
um  in  other  words  this  is  a  TENS  product
The  Continuous  function  from  T  to  r  t
over
R  kind  of  R  join  Epsilon  where  the
degree  of  Epsilon  is  equal
toic
one  and  some  particular  Square  to
zero  but  where  the  structure  shift  comes
in  ah  okay  it's  s  o  is
there  so  here  it  doesn't  really  matter
because  always  Factor  over  Pi  Z
continuous  functions  but  here's  the  WR
structure  that  is  be  cusing  that  to
Something  There  is  ah  okay  so  in  the
first  thing  you  can  replace  it
by  the  first  thing  yeah  you  can  replace
by  the  kind  zero  and  even  by  the  house
of  because  see  here  here  not  and  just  by
De  the  you  can  analyze  what  does  it  take
to  lift  the  map  from  here  to  here  and
it's  just  some  thing  in  terms  of  the
cotangent  complex  and  so  you  can  realize
that
uh  the  fiveways  here  have  some  kind
of
uh  some  kind  of  geometric
Incarnation  you  also  cotangent
complex  so  all  the  tangent  directions  uh
they  give  you  kind  of  stacky  directions
and
uh  yeah  and  if  there  some  kind  of  actual
obstruction  uh  this  is  non  smooth  part
and  there's  actually  Al  some  Vector
bundle
directions
yeah  some  particular  some  kind  of  smooth
map  um  or  can  understand  it  and
so  uh
and  yeah  right  so  there  a  cent  complex
and  like  the  the  perfect  complex  so  it
has  a  kind  of  Dimension  which  is  the
difference  between  the  vector  B
dimensions  and  so  this  already  gives  you
the  expected  dimension
some  look  at  the  rank  of  this
guy  um  and  this  kind  of  situation  here
this  geometry  here  will  also  produce  the
orientation  feet  for  you  think  about  it
so  actually  what  you  really  need  to
produce
is  need  the  glob
section  so
element  in  this  intrinsic  normal
cor  uh  with  values  in
the  that's  dat  you  really  data  you
really  need  to  produce
um  and  now  there's  this  really  funny  uh
thing
that  let's  take  r  as  a
cond  set  here  down  here  and  then  one  can
write  down  something  down  which
everywhere  except  at  zero  it's  just  a
point  but  in  this  F  at  zero  you  get  this
funny
fullness
there  some  kind  of  con
as  the
form  which  lives  over  the  real  numbers
and  which  everywhere  except  at  zero  is
just  point  but  then  as  you  degenerate  to
the  point  suddenly  there  some  very
sticky  thing  and  you  see
this  Con
here
fine  can  I  ask  question
yes  is  it  the  same  phenomenon  as  you
described  before  in  the  analytic  stack
of  P1  where  you  compactify  over  the
fixed  point  at
Infinity
I  don't  understand  the  question  let  me
just  continue  with  this  discussion  yeah
sure  a  so  so  let's  define  what  what  is
this  formed
con  uh  so  mapping  T2  there  this
corresponds  to  but  first  of
all  uh  let  me  use  G  for  safety  because
I'm  not  sure  if  I've  used  F  before  uh
corresponds  to  a  map  from  like  this
should  map  to  R  so  first  of  all
correspond  to  a  continuous  map  from  T  to
R
but  once  you  have  that  you  can  look  at
maps
from  uh  The  Continuous  function  from  T2
r  modul  g  right  uh
to
um  where  where  of  course  if  G  is  the
zero  Maps  then  this  is  just  recovering
what  I  told  you  before  if  G  is  non  zero
well  then  if  equ  equ  by  an  invertible
function  this  is  just  zero  so  you're  not
getting  this  is  no  data  um  so  you  guess
just  just  get  a  point  but  then  if  G  is  a
function  that's  somewhere  invertible
somewhere  zero  then  suddenly  you  have
this  kind  of  thing  where  is  producing
some  kind  of  doing  these  things  together
okay
uh
and  now  it's  just  some  simple  game  with
six  fun  just  to  produce  a  class  because
basically  generically  you  just  have  a
canonical  section  here  it's  just  a  point
right  and  then  you  just  degenerate  that
uh
section  uh  let  me  try  to  do  it
so  so  consider  the
sheath  F  which  is
form  so  so  this  guy  here  is
streetable  and  that's  the  only  way  I
currently  know  how  to  check  it  is  to  use
it  I  assume  that  a  der  section  of  smth
manifolds  and  for  manifolds  it's  true
and  then  it's  disable  on  fiber
products  in  principle  you  could  probably
produce  this  canonic  virtual  fundamental
class  with  much  small  assumptions  the
only  thing  you  really  need  to  that  this
map  here  isable
um  but  then  you  can  Define  this  thing
which  is  a  sheet
on
followe
and
uh  and  then  okay  so  you  have  an  open
subset  J  which  is  the
cone  where  from  zero  and  then  you  have  I
which  is  exclusion  from  the
cone  and  then  you  have  some  excision
sequence  for  f
uh
F  and  here  you
have  which  is  just  I  the  of  the  p
stre
and  then  in  particular  you  get  a
boundary  map  here  and  the  boundary  map
just  gives  you  the  CLA  you
want
and  that's  always  a  degree  shift  I'm
getting  confused  about  but  it  did  work
out
um  because  the  ididentification  the  i  f
when
you
yeah's  a  shift  by  one  appearing  here
because  I'm  pulled  back
from  uh  yeah  so  there  there's  shift  by
plus  or  minus  one
here  minus  one  because  uh  I  take  Z  here
on  the  real  line  and  not  not  pulled  back
from  from  the
point  and  so  then  you  get  a  m  from  the  X
zero  of  r  z
z  towards  the
H  and  okay  this  and  then  you  take  a
class  here  which  is  zero  on  one
connected  component  and  one  on  the  other
and  the  image  is  what  you're  looking
for  by  the  way  is  it  is  it  true  that
this  pie  cone  def  is  not  just  triable
but
smooth
no  I
I  don't  think
so  I  would  have  guessed  okay  I  don't
think
so  I  so  this  cone  of  s  this  is  some
crazy  thing  right  I  mean  it's  some  kind
of  fiberwise  it's  some  kind  of  vector
bule  like  thingy  but  over  this  crazy  s
mean  this  this  s  is  completely  nasty  in
general  so  if  this  map  was  smooth  and
also  like  the  F  was  smooth  but  this  has
just  no  way  of  being  smooth  I  don't
think
okay  so  I  don't  think  this  this  thing
that  appears  here  is  in  vertical  at  all
but  you  I  mean  you  can  still  can  still
degenerate  the  topological  fundamental
class  on  a  point  towards  this  cone  using
a  little  bit  of  six  fun  and  then  yeah
this  Con  difference  from  the  original
space  just  by  some  kind  of  fiber  BS
which  you  can  analyze  and  which  give  you
both  the  dimension  shift  and  the
orientation  I'm  over
time  I  want  I  don't  know  if
it  obstruction
the  well  the  TR  obstruction  series  isn't
that  just  that  there  this  quention
complex  that  behaves
right  okay  that's  just  completely  infil
this
six  any  further
questions  the  the  what  was  the  the
exponential  shift  sequence  in  the
previous  Point  uh  for
where  you  have  GA  local  analytic  and  GM
local  analytic  in  which  and  Z  so  in
which  sense  so  I  believe  that  those
exist  Stacks  maybe  with  symmetric
monoidal  structure  so  this  is  like  an
exact  sequence  of  symmetric  mono
analytic  stack  so  what  in  which  sense  is
it  something  like  this  or  or  is  it  yeah
a  group  figurative  groups
right  this  St  with  the  community
structure
and  yes  I'm  not  quite  sure  what  the
question  is  I  mean  it's  like  I  mean  all
STS  they
are  the  definition  sheets  and  then  all
these  are  sheeps  of  a  being  grps
or  the  shifts  in  the  higher  sense  okay
okay  exact  sequence
of  which  probably  stly  speaking  would
again  be  a  triangle  because
okay  language  so  it's  just  but  this  is
an
unstable  being  group  Str  kind  of  puts  it
into  a  stable  context
right
okay  all  connective  and  is  an  exact
sequence  in  there  well  but  it's  also  the
right  so  it
stays  triangle  on
St  okay  okay
okay
okay  thanks  Peter  yeah  all
right
\end{unfinished}