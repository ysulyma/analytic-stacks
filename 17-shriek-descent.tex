% !TeX root = AnalyticStacks.tex

\section{\ufs !-descent continued (Clausen)}

\url{https://www.youtube.com/watch?v=rN_iM7Z8vdE&list=PLx5f8IelFRgGmu6gmL-Kf_Rl_6Mm7juZO}
\renewcommand{\yt}[2]{\href{https://www.youtube.com/watch?v=rN_iM7Z8vdE&list=PLx5f8IelFRgGmu6gmL-Kf_Rl_6Mm7juZO&t=#1}{#2}}
\vspace{1em}

\begin{unfinished}{0:00}
e
okay  so  let's  um  let's  get  started  so
I'm  going  to  uh  so  well  I'm  going  to
continue  the  discussion  that  Peter
started  last  time  on  shriek
descent
um  so  well  let  me  recall  the
setup  um  so  we're  kind  of  well  what  are
we  trying  to  do  we  have  analytic  Rings
um
and  then
well  we're  going  to  build  some  geometric
objects  on  them  in  somewhat  in  the  model
of  scheme  Theory  where  you  have  start
with  commutative  Rings  um  so  the  first
thing  you  do  is  you  define  what  are  the
apine  things  um  and  might  as  well  just
formally  Define  them  to  be  the  opposite
category  of  analytic
Rings  um  and  then  so  an  object  in  here
maybe  you  call  X  and  it  corresponds  to
uh  let's  say  an  analytic  ring  which
maybe  one  can  denote  Ox  comma  DX  so
weall  that  an  analytic  ring  consists  of
well  an  animated  condensed  ring  and  then
a  certain  full  subcategory  of  modules
over  that  of  the  say  full  derived
category  of  modules  over  that  so  I'll
denote  the  first  coordinate  by  o  of  X
and  the  second  coordinate  by  D  of  X  in
this  notation  so  here  I  mean  X  is  really
just  a  formal  symbol  here  but  okay  also
in  the  in  the  derived  context  you
somehow  use
is  just  this  yeah  and  I  suppose  there  is
a  way  to  define  the  full  dve  category
also  yes  so  but  I  don't  I'm  not  sure  how
I  mean  but  is  it  important  to  just
restrict  to  the  I
mean
for  for  some  purposes  it's  easier  when
you  for  so  the  the  two  formalisms  are
equivalent  you  can  either  think  of  this
or  you  can  think  of  the  full  thing  um
and  you  can  get  back  and  forth  between
them  for  some  discussions  it's  more
convenient  to  consider  the  greater  than
or  equal  to  zero  part  to  be  the  primary
object  but  for  the  discussion  today  it's
actually  um  much  better  to  consider  this
thing  as  the  primary  object  and  the
reason  has  to  do  with  a  descent  so  it
turns  out  that  the  um  the  full  derived
category  will  have  this  shriek  descent
but  the  greater  than  or  equal  to  zero
will  will  not  so  in  other  words  another
way  of  saying  it  is  for  an  for  an
analytic  ring  we  have  a  nice  T  structure
on  this  category  but  for  a  general
analytic  stack
um  there's  not  going  to  be  a  a  t
structure  so  um  on  which  on  the  on  the
on  the  derived  category  of  the  global
thing  so  which  is  obtained  by  gluing
these  local  derived
categories  the  for  an  analytic  ring
there's  a  t  structure  which  is  so  this
thing  is  by  definition  a  full
subcategory  of  you  know  derived  category
of  ox  modules  okay  this  has  a  natural  t-
structure  and  the  claim  is  that  you  mean
just
the  yeah  you  just  forget  down  to  derived
the  condensed  ailan  groups  so  the  t-
structure  is  detected  all  the  way  down
here  and  the  claim  is  it  passes  to  it
induces  a  t  structure
here
and  what  is  official  of
the  the  full  one  yeah  so  me  when  I  talk
about  it  what  I'm  thinking  of  is  the
following  there  is  a  functor  from
simplicial  rings  to
say  E1  rings  so  meaning  uh  uh  just
algebra  objects  in  this
category  um  and  then  I  just  take  modules
over  that  algebra  object  in  this
category  here  so  I  take  I  use  some
Infinity  category  formalism  of  algebras
and
modules  yeah  so  it's  not  defined  by  but
it  can  be  defined  I  I  suppose  you  can
take  like  chain  complexes  of
mod  yeah  no  doubt  but  I  prefer  to  think
of  it  like  this  I  mean  you  can
understand  perfectly  well  like  how  to
calculate  with  with  this  definition  I
mean  the  free  you  know  the  free  objects
are  like  things  in  here  tens  are  with  a
and  then  you  know  you  can  make
resolutions  and  you
know  ring  is  the  same  as  E1  Rings  yeah
simpli  Rings  up  to  equivalent  of  course
yeah
yeah
okay
okay  okay  so  uh  right  and  um  and  we
singled  out
uh
um  there's  a  there  was  this
definition
so
uh
so  uh  a  map  is  shable  if  it  can  be
Factor  factored  as  an  open
immersion  followed  by  a  proper
map
um  and  uh  what  was  the  definition  of
these  Concepts  well  for  proper  map  um  so
let's  say  let's  call  this  uh  F  and  let's
say  J  and  then  Pi  um  so  for  a  proper  map
we  have  uh  this  means  that  uh  an  element
m
in  uh  D
of  o  x
Prime  uh
lies  in  D  of  X  Prime  uh  if  and  only  if
uh  it's
image  in  D  of  a  y  I  mean  image  under  the
forgetful  functor  uh
lies  in  D  oh  D  oh  sorry  yeah  lies  in
Dy  um  so  in  other  words  you  just  take
the  class  of  complete  modules  you  have
on  this  analytic  ring  um  and  then
inherit  the  notion  of  the  notion  of
completeness  up  here  so  just  given  the
data  of  this  ring  you  can  always  do  such
a  thing  you  can  Define  this  D  of  X  Prime
to  be  ones  where  you  forget  down  and  you
ask  it  to  be  complete  there  and  that
always  defines  an  analytic  ring
structure  and  those  are  exactly  the  ones
corresponding  to  these  proper
Maps
um
yes
exactly  yes  and  did  you  claim  things
like  the  composition  of  I  forgot  now
uh  yes  we  claim  that  I  think  I  mean
Peter  claimed  it  last  time  I  believe
didn't  didn't  give  the  argument  but  it's
um  it's  quite  uh  it's  quite
straightforward  so  one  of  the  one  of  the
one  just  but  let  me  let  me  continue  the
discussion  um  and  so  open  immersion
meant  that  uh  well  first  of  all  this  uh
J  upper  star  functor  which  you  always
have
uh  this  should  be  a
localization
and  uh  the
kernel  should  be  just  modules  over  some
uh  some
algebra  yeah  it  will  necessarily  be  item
potent  it  will  also  necessarily  be  a
compact  object  because
the  um  because  by  definition  of  a  map  of
analytic  Rings  the  right  ad  joint  of
this  commutes  with  Co  limits  and  then
the  this  is  not
in  yeah  it's  yeah  just  in  the  pure
category  Theory
sense  there  is  a  notion  of  like  usual
categories  you  can  localize  by
multiplicative  set
for
example  no  but  these  are  these  are  like
big  categories  presentable  categories
and  the  good  definition  is  just  what
Peter  said  that  the  right  ad  joint
should  be  a  fully  faithful
functor  right  joint  is  fully  well  yes  so
but  this  this  fun  well  we  we  I  mean  we
already  know  as  part  of  the  discussion
of  analytic  rings  that  the  righted  joint
exists
so  yeah
yeah
um  want  a  left  joint  uh  yeah  this  uh
this  implies  that  there's  a  left  joint
yeah  um  so  let  me  yeah  so  I'll  I'm  going
to  continue  the
discussion
um  right  um  and  then  so  we  had  uh  we  had
this  claim  or  this
theorem  uh  oh  yeah  so  then  the  remark  is
that  for  a  proper  map
um  for  a  proper  map  Pi  uh  the  right
adjoint  Pi  lower  star  is
good  uh  and  what  does  that  mean  it  it
commutes  with
co-limit  uh
satisfies  uh  projection
formula  uh  I  e  it's  it's  D  of  Y
linear  so  if  you  take  an  object  in  D  of
Y  and  you  pull  it  back  and  then  you
tensor  with  something  and  push  it
forward  that's  the  same  well  yeah  so  if
you  take  an  object  in  D  of  X  Prime  you
can  either  and  an  object  in  D  of  Y  you
can  either  pull  it  back  tensor  and  then
push  forward  or  you  can  push  forward  and
then  tensor  and  those  give  you  the  same
thing  there's  a  natural  map  and  you  ask
it  to  be  an
isomorphism  um  and  uh  and  commutes  with
base  change
or  so  in  other  words  a  pullback  of  a
proper  map  first  of  all  is  proper  and
second  of  all  um  the  the  formation  of
this  lower  star  this  R  joint  commutes
with  that  base  change  that  you  can  write
down  um  and  then  uh  for  an  open
immersion  uh  J  there  exists  a  lefta
joint
uh  I  don't  know  JLo  or  whatever  people
use  uh  uh  which  is
good  in  the  same
sense  J  I  don't  know  I've  seen  I've  seen
this  J  lower  sharp  J  lower  natural  yes
yes  yes  it  will  be  JLo  or  shriek  yeah  I
mean  okay  I'll  just
yeah  yeah  no  there  is  there  is  the  other
thing  which  is  an  adint  like  for
sometimes
for  in  Duality  like  an  agent  of  f  star
no  no  okay  it  is  not  the  the  bizar  it
is
usual  yeah  it  is  lower
shriek  okay
um
uh  right  um  and  then
the  the  theorem  that  was  discussed  last
time  was  uh
so  so  there  exists  six  functor  formulism
on
F  uh  such
that
uh  the  collection  of  maps
FX  to  Y  for
which  this  extra  funter  you  have  F  lower
shriek  from  D  of  x  to  D  of  Y  is
specified  uh  is  is  the  class  of  shable
maps  uh  such  that  for  F
proper  F  lower  shriek  equals  F  lower
star  and  for  f  for  J  an  open
immersion  well  F  now  now  here's  why  I
wanted  to  yeah  you  know  what  I'm  going
to  I'm  going  to  say  left  ad  joints  or
Jor  or  natural  and  yeah  just  so  that  I
can  make  a  non-  tological  claim  here
yeah
um
um  oh  thank  you  yeah  J  thank
you  all
right
um  so  so
ahm  contains  the  notion  of  proper  the
class  of
proper  it  does  not  no  so  I'll  explain  in
just  a  second  exactly  the  manner  and
what  like  what  the  precise  definition  of
this  and  and  how  it's  encoded  and
everything  um  in  that  encoding  it's  not
specified  what  is  a  proper  map  and  what
is  an  open  immersion  there's  it's  just
specified  what  is  the  class  of  shable
maps
um  uh  oh  yeah  and  maybe  I  remark  I
perhaps  should  have  made  before  is  that
there's  a  the  class  of  shable
maps
has  has  good  closure
properties  so  it's  uh  if  so  if  we  have  X
to  Y  to  Z  say
FG  um  then  f  and  g
shable  implies  f  g  g  composed  with  f
shable
uh  G  and  G  composed  with  f
shable  implies  F
shakable  that's  nice  so  that  a  map
between  shable  Maps  is  shable  that's  the
way  to  think  about  this  um  and  uh  uh
shable  maps  are  closed  under  base
change
oh  and  maybe  I'll  add  one
more  uh  so
if
um
uh  if  like  F1  X1  to  Y  F2  X2  to  Y  uh  FN
xn  to  Y  are  maps  with  the  same  Target  uh
then  uh  uh  fi  is  oops  fi  is
shable  for  all  I  if  and  only  if  uh  the
map  F  from  the  disjoint  Union  over  i  x  i
to  Y  is
shakable
um  and  this  these  disjoint  unions  in
this  category  those  those  finite
disjoint  unions  in  this  category  they
correspond  to  finite  products  in  this
category  of  analytic  rings  and  they're
kind  of  just  naively  defined  coordinate
wise  so
um  so  for  finite  finite  Co
products
um  so  maybe  I'll  do  just  uh  I  think
Peter  discussed  it  last  time  but  just  a
let  me  do  a  reminder  some  example  of
this  six  funter  formalism
so  so  let's  take
um  y  to  be  sort  of  spec
of  the  solid  Z
Theory
um  then  we  could  and  then  let's  take  X
to  be  spec  of  the  solid  ZT
Theory
um  uh  with  the  natural  map  from  X  to  Y
which  is  just  a
yeah  um  then  we  can
take  X  Prime  to  be  spec  of  the  solid  ZT
z  uh
Theory
um  so  this  is  pro  now  this  is  a  so  that
we  have  X  goes  to  X  Prime  goes  to  Y  now
this  is
proper  and  this  is  an  open
immersion
um  and  the  the  complimentary  algebra
a
uh  is  equal  to  this  uh  Z  lant  series  T
inverse
um  and  then  for  example  J  lower  shriek
of  the  structure  chath  of  x  uh  is  uh  is
this  two  term  complex
um
so
um  and  then  that's  also  the  formula  for
pi  lower  star  Pi  lower  star  is  just  the
forgetful  functor  just  forgetting  that
you  have  a  module  structure  over  ZT  um
and  so  intuitively  speaking  this  is
functions  on  the  Aline
line  uh  and  this  is  functions  localized
uh  near  Infinity  localized  near  the
missing  point  and  so  this  is  like
functions  on  the  Aline  line  which  vanish
near  Infinity  so  to  speak  because  you're
taking  a  fiber  um  so  that's  kind  of  uh
compactly  supported  chology  of  the
structure  sheath  on  the  Aline
line  so  to
speak  and  this  is  what  uh  six  funter
formalisms  are  supposed  to  be  doing  in
general  they're  supposed  to  be
specifying  some  notion  of  compactly
supported  chology  relative  compactly
supported  chology  which  behaves  well  in
families
um  so  the  the  key  property  of  this  eor
shriek  is  that  it  commutes  with  base
change  in  in  complete
generality  so  so  this  is  so  this  yeah  so
the  algebraic
um  algebraic  objects  are  proper  even  if
you're  dealing  with  something  like  the
Aline  line  which  is  not  proper  in
traditional  algebraic  geometry  but  it's
kind  of  compensated  by  the  existence  of
this  solid  Theory  where  you  have  a  new
version  of  the  Aline  line  the  solid
afine  line  which  is  not  proper  anymore
um
yeah
okay  so
um  uh  and  then  there  was  the  definition
so
um  uh  a  shable  map
uh  satisfies  Street  descent
um
if  um  so  you  can  take  D  of  Y
um  D  of  x  and  then  D  of  uh  X  cross  YX
and  then  you  continue  like  this  um  and
then  here  you  use  the  upper  shriek
functors  to  Define  this  diagram  so  upper
shriek  being  the  the  right  ad
joint  uh  to  lower  shriek  in  every
case  um  uh  so  this
induces
so  to  the  shriek  limit  uh  of  the  D  of  uh
x  to  the  NS  over
y  so  the  naive  thing  would  be  to
consider  the  star  descent  you'd  want  to
know  that  a  an  object  in  here  is  the
same  thing  via  pullback  as  a  compatible
collection  we  actually  asked  for
something  different  uh  it  turns  out  to
be  stronger  which  we'll  show  in  the
lecture  today  that  when  if  you  take  an
object  here  you  ask  that  it's  you  know
the  same  as  giving  compatibly  it's
shriek  pullbacks  along  this  this  same
diagram
um  so  now  I  can  that  was  the  recap  of
last  time  now  I  can  explain  the  goal  for
today
goal  so  well  first  goal  would  be  to
um
uh  uh  prove
some  streak  descent
results  I.E  uh
Identify  some  natural  she  uh  natural
pre-
shees  that  have  have  shriek
descent  so  in  other  words  that
um  some  pre-  on  aphids  such  that
whenever  you  have  a  a  cover  satisfying
shriek  descent  which  by  definition  means
that  D  of  Y  satisfies  streek  descent  for
upper  shriek  um  we  wanted  to  find  some
check  what  other  kinds  of  descent
properties  that
implies
um  or  well  maybe  I  should  actually  give
the  definition  sorry  sorry  the  the  Groen
de
topology  uh  of  shriek
descent  uh  so
covers  are  uh  given
by  uh  finite
collections  uh
fi  x  i  to  Y  uh  sorry  um  sorry
of  shable
maps  uh  such  that  um  f  from  the  disjoint
Union  of  the  XI  to  Y  uh  has  streak
descenty  on  the  underlying
yes
so  yeah  so  I  mean  or  yes  so  I  could  say
a  seeve  is  covering  if  it  contains
finitely  many  shable  Maps  satisfying
this  condition  yeah  what  about  the
composition  of  Shri  it's  involve
interchanging  proper  and  openion  or  is
it  it's  quite  straightforward  and  the
reason  the  essential  reason  it's
straightforward  is  there's  a  canonical
candidate  for  the  X  Prime
the
factoring  because  your  X  contains  this
algebra  o  of  X  the  structure  sheath  and
then  you  can  just  take  X  Prime  to  be  o  o
have  the  same  o  of  X  and  then  D  of  X
Prime  should  just  be  the  full
subcategory  of  D  of  O  of  X  consisting  of
things  whose  image  in  D  of  O  of  Y  land
in  D  of
Y
so  so  you  you  you  have  a  like  an  open
immersion  and  then  a  proper  yeah  and
then  you  say  that  this  so  this  is  like
in  adct  spaces
the  okay  so  you
take
uh  the  ones  okay  you  use  the
other
uh
uh  yes  okay  so  you  claim  that  thinking
about  it  you  have  a  canal  and  you  check
it  yes  exactly
yeah  okay  and  the  second  uh  goal  for
today  is  a  well  then  you  want  to  be  able
to  give  examples  of  Street  covers
so  so  that  you  then  can  apply  these
descent
results  I  don't  remember  if  it  was
defined  this  was  a  talk  I  saw  in  the  way
but  whether
it  sh  def  what  six  fun  formalism  is
exactly  I  think  he  refer  to  Lucas  man  so
I'm  going  to  actually  discuss  it  today
because  the  precise  way  it's  encoded
will  help  uh  give  some  arguments  so  it's
in  fact  the  next  topic  that  I'm  going  to
turn  to
um  uh
okay
um
so  yeah
so  so  the  the  this  this  theorem
so  the  existence  of  this  six  funter
formalism  on
apids  um  follows  rather  immediately
uh  from  work
of  Li
Jen  as
reinterpreted  by  man
and  so
following  Gates  soryy  Rosen
Bloom  uh
man  encodes  uh  six  funter
formalisms  uh  in  terms  of  span
categories  so  let  me  um  Let  Me  Now  set
this  up  um
so  so
suppose
given  uh  well  let's  say  could  be  an
Infinity
category  uh
C  uh  with  all
pullbacks
um
yeah
um  and  a  class  of
maps  s  and  C  stable
under
composition  and
pullback
um  then  so  I'm  not  going  to  give  the
precise  uh
um  the  precise  description  of  the
infinity  category  of
spans  I'll  give  a  kind  of  handwavy
description  the  precise  description  has
been  given  by  Clark  Barwick  um  so  as  a
complete  seagull  space  I  mean  there's
there's  well  or  no  he  did  it  as  an
anyway  doesn't  matter  there's  there's
different  ways  of  making  it  precise  and
it  has  been  done  um  so  then  we  get  a
span  category  uh  span  C  let's  say  s  all
is  the  notation  so  the  idea  is  this  is
going  to  encode  let  you  encode  six  fun
formalisms  where  you  have  shriek  Maps
defined  for  uh  maps  that  lie  in  s  and
you  have  uh  star  Maps  like  upper  star
defined  for  any  map  whatsoever  um  so
this  has
objects  the  same  as  in
C  uh  but
Maps  uh  X  to  Y  given
by  uh
diagrams  so  X
Y  and  then  you  put  something  in  between
uh  M  say  like
this  yes  thank  you  yeah  or  this  map  um
let's  say  this  map  is  an
s  or  do  I  would  I  rather  say  the  other
map  lies  in  S  I  don't  know
you  have  to  make  a
choice  and  then  the  uh
composition  uh  is  given  by  pullback  so
I'll  just  write  it  down  like
this  so  you  have  a  span  from  X  to  Y  and
a  span  from  y  to  Z  you  want  to  make  a
span  from  X  to  Z  you  just  do  the
pullback
here
um  and  then  a  I  don't  know  a  two  functor
formalism  on
C  uh  with  respect  to  s  uh  is  just  a
functor  um  from  span
CS  all
to  some  some  category  of  categories
so  so  I  to  every  object  X  you  assign  a
category  Infinity
category  uh  D  of  X  say  we  can  call  it  D
um  um  and  then  to  every
span
uh  you  assign  a
functor
so  well  what  is  this  doing  so  there's
two  special  classes  of
maps  there's  the  ones  where  you  only
have  uh  something  interesting  going  on
in  the  S  part  and  there's  the  ones  where
you  only  have  some  something  interesting
going  on  in  the  Y
part
um  and  every  map  is  a  composition  of  is
canonically  a  composition  of  uh  two  such
Maps  like  this  so  you  can  break  this  up
into  well  it's  kind  of  yeah  x  m  mm  Y  and
you  can  write  this  then  as  a  composition
of  two  of  these  so  if  you  kind  of  at
least  intuitively  want  to  understand
what  your  functor  is  doing  on  morphisms
it's  enough  to  know  what  it's  doing  here
and  what  it's  doing  here  um  and  here
this  is  giving  you  a  functor  that  you
call  uh  let's  say  this  is  map  is  F  this
is  giving  you  a  functor  you  call  F  lower
shriek  from  D  of  x  to  D  of  Y  and  here
it's  if  this  is  F  this  is  giving  you  a
functor  uh  you  call  F  upper  star  from  D
of  x  to  D  of
Y
stable  under  composition  includes  having
the  identity  yes  yes  it  does  it's  what
happens  when  you  compose  an  empty  sets
worth  of  composable
maps
um
um  and  then  if  you  and  then  it's  a
functor  so  you  get  some  relation  when
you  have  a  pair  of  composable  maps  and
the
functoriality  um  amounts  to  the  base
change
formula  uh  for  the  lower  shriek  functors
so  the  fact  that  the  lower  shriek
commutes  with  upper  star  when  you  have  a
cartisian  square  in  your  category
so  two  is  a  smaller  number  than  six  a
two  functor  formalism  versus  six  functor
formalism  but  two  is  the  essence  of  six
in  this  case  because  the  what  are  the
other  functors  um  so  then  that's  two  of
the  six  functors  but  then  there's
a  so  if  uh
well  base  change  formula  and  also
compatibility  uh
of  and  upper  star  under
composition  and  it  also  encodes  kind  of
compatibilities  between  the  base  change
formulas  and  the  compositions  that  you
have  on  these  things  so  there's  kind  of
higher  order  data  implicit  in
here  um  right  so  that's  two  of  the  of
the  six  functors  you'd  like  to  have  but
then  so  then  that  that  gives  you  these
funs  F  lower  shriek  and  F  upper  star  but
then  for  free  provided  this  admits  a
right  a  joint  then  you  get  the  right
joint  as  well  and  provided  this  yeah  and
provided  this  admits  a  ride  ad  joint
then  you  get  the  ride  ad  joint  as  well
um  and  that  gives  you  four  functors  you
don't  need  to  encode  them  explicitly
they  just  kind  of  it  just  kind  of
follows  um  and  then  the  remaining
functors  are  some  tensor  product
operation  you  have  defined  on  each  of
these  um  and  then  some  adjoint  to  it
some  internal  arom
and  again  that  if  you  have  this  and  it
satisfies  a  certain  property  then  it  has
some  some  adjoint  so  you  so  the  only
remaining  question  is  how  to  encode  this
tensor  product  and  the  uh  expected
interaction  of  it  with  uh  these  functors
here  namely  projection
formulas  um
so
so
uh
uh  plus
compatibility  with  lower  freaking  upper
star  and  for
this  we  use  a  symmet  the  symmetric
monoidal
structure  on  this  span
category  uh
induced  uh
by  by
product  uh  in  C  so  now  I  have  to  assume
I  guess  maybe  I  I  assume  C  has  a
terminal  object  I  mean  I  assume  it  has
pullbacks  I  need  I  need  products
um  yeah  so  let's  say  that  c  has  a
terminal  object
um
see  um
whoops
uh  so  yeah  that's  cartisian  product  in  C
but  it's  not  the  cartisian  product  in
this  band  category  anymore  it's  just
some  symmetrical  monoidal
structure
um  and  then  and  then
request  that  your  functor  D  from
this  could  you  say  that  again
Peter  prze  one  U
produ  break
symmetry
of  because  at  least  if  you  take  for  all
maps  then  so
farc
yeah  notmc  anymore  for  strcture  because
only
the
yes
uh
yeah  Peter  was  making  a  claim  that  the
um  well  I  I  didn't  finish  the  sentence
maybe  yeah
uh  so  we  request  that  this  be  LAX
symmetric
monoidal  uh  with  respect  to  that  some
that  uh  tensor  product  we  defined  just
defined  here  and  the  uh  symmetrical
structure  here  are  given  just  by
cartisian  product  of
categories  um
so
uh
um  so  the
um  uh  so  what  is  this  amount  to  so  the
basic  data  is  that  you  have
DX  uh  cross
Dy  should  map  to  D  of  X  cross  y  so
that's  the  direction  of  the  laxness  if
you  were  wondering  um  plus  uh  you
know  some  compatibilities  which  are
conveniently  encoded  in  this  being  a  LAX
symmetric  monoidal
functor  so  in  particular  DX  DX  BS  to  DX
using  the  diagonal  yes  and  then  you  can
probably  reconstruct  this  by  pulling
back
over  y  but  then  one  can  wonder  why  you
need  the  product  in  the  category  it's
not  enough  to  specify  DX  DX  to  DX  for
every  X  in  some  coherent  way  well  you
need  to  encode  the  compatibility  with
the  lower  shriek  functors  somehow
so  that's  extra
data  yeah  but  yeah  indeed  if  it  were
just  if  if  s  were  like  trivial  class  um
then  then  giving  this  would  be  the  same
as  giving  a  symmetric  monoidal  structure
on  each  of  the  values  DX  so  yeah  this  so
part  of  this  uh  is  yeah  symmetric
monoidal  structure  on  D
ofx  and  then  the  f  ofar  functors  are
also  symmetric
monoidal  as  as  of  said  by  if  you  take
xals  y  you  can  do  this  and  then  you  can
pull  back  along  the
diagonal  what  about  object  one  does  it
come  for  free  one  in  the  of  the  terminal
object  oh  boy  yeah  I'm  sure  it  comes  for
free  from  so  it  lacks  unital  I  mean
there  should  be  some  unit
yeah
yeah
um  okay
um
so  uh  yeah  I  have  to  advance  this  story
a  bit  so  I
um  yeah  so  I  said  that  if  you  just  have
these  well  really  these  three  functors
ding  ding  ding  satisfying  certain
properties  then  you  automatically  get
all  six  functors  and  there's  a  very
convenient  way
to
um  to
organize  uh  the  passage  from  3  to  six
and  that's  uh  it's  it's  a  a  bit  of
higher  categorical  magic  defined  by  lury
so  this  is  L's  magic
category  uh  called
PRL  um  so  I  have  to  say  a  little  bit
about  this  and  how  it  works  so  here  the
uh
objects  are  the
presentable  Infinity
categories  presentable  means  you  have
all  small  co-  limits  and  um  you're  in
some  sense  controlled  by  a  small
subcategory  so  in  the  sense  that  there's
a  small  subcategory  such  that  the  whole
thing  is  gotten  by  formally  adjoining
some  sufficiently  filtered  Co  limits  so
you  should  be  Kappa  compactly  generated
for  some
Kappa  um
uh  and  the
morphisms  are  the  cimit  preserving
functors  and  then  that's  the  equivalent
to  admitting  a  r
joint
um
so
um  so  let  me  uh  tell  you  a  bit  more
about  this  magic
category
um  so  so  PRL  has  all  small
limits  and  um  the  forgetful
funter
uh  PRL  to  cat  Infinity  uh  preserves
them
um  so  that's
um  that's  nice  so  limits  in  this
category  exist  and  are  completely  naive
so  um  but  also  PRL  has  all  small  Co
limits
and  this  is  the  really  remarkable  thing
is  that  you  can  also  access  these  co-
limits  in  a  completely  naive  way
um  uh  so  so  there  is  besides  the  naive
forgetful  functor  there's  also  a  contant
version  of  the  forgetful  functor  where
you  take  C  and  send  it  to  C  um  but  then
you  take  a  funter  from  C  to  D  and  you
send  it  to  its  right  ad  joint
um  yeah  and  it  and  you  can  make  this
into  an  honest  functor  so  that's  part  of
what  lri  does  um  and  this  functor
preserves  co-
limits  which  uh  colimits  I  mean  colimits
in  the  opposite  category  so  really  it's
limits
um  which  translates  to  limits  uh  in  cat
Infinity  so  also  the  co  limits  in  L's
magic  category  are  just  calculated
naively  limits  of  underlying  categories
but  with  respect  to  passing  to  the
righted  joints  of  the  functors  in  your
diagram  this  is  quite  magical  because  in
like  in  cat  infinity  or  category  of
categories  or  something  like  co-  limits
can  be  very  difficult  to  calculate  like
for  example  you  know  Amalgamated
products  of  groups  are  examples  you  know
of  Co  limits  in  categories  and  it's  kind
of  a  uh  not  so  simple  construction  but
this  uh  this  is  sort  of
easy
um  uh
right  um  so  in  particular  let  me  let  me
so  in  particular  uh  just  to  connect  with
what  we  had  earlier  uh  this  weird
looking  condition  of  shriek
descent  uh  that  we  had  for  maps  of  for
shable  maps  of  AIDS  it's  actually  the
same  thing  as  um  well  Cod  descent
so  so  Cod
descent  for  the  lower  stre
funs  in
PRL  and  that's  actually  a  much  more
convenient  way  of  thinking  about
it  maybe  I  should  put  this  down  here
now  and  then  this  is  the
limit  but  the  limit  is  to  be  interpreted
in  the  the  Co  liit  in  the  sense  of  PRL
exactly  which  is  not  this  which
is  okay  which  is
a
okay  um
now  uh  there's  more  magic  in  PRL  there's
also  a  symmetric  monoidal
structure  so  which  is  maybe  the  main
theorem  in  in
Lor's  book  or  something
uh  H  it's  either  I  don't  remember  which
book  it's  in  I  would  claim  the  whichever
book  it's  in  it's  the  main  theem  in  that
book  my  opinion  I  mean  it's  really  yeah
so  um  so  it's  a  tensor  product  and  it's
it's  it's  characterized  by  Universal
Property  just  like  you  would  expect  that
of  a  tensor  product  so  maps  in  PRL  from
C  tensor  D  to  E
uh  that  is  the  same  thing  as
functors  it's  from  the
product  uh  which  commute  with  co-
limits  in  each  variable
separately  um
and
uh  so  this  all  so  this  uh  and  then  this
this  tensor  product  on
PRL  also  actually  commutes  with  co-
limits
uh  in  each  variable
separately  so  you  get  a  a  very  nicely
behaved  um  tensor  product  on  this
category
um  so  in  fact  uh  it's
so  there's  a  there's  a  there's  a  there's
also  a  corresponding  internal
H  is  given
by  internal  H  from  C  to  D  is  just  the
infinity  category  of  co-limit  preserving
functors  from  C  to  D
so
note  so  note  uh  the  the  mapping  space  in
this  category  from  C  to
D  is  what  you  get  from  this  category
here  uh  sort
of  by  uh  only  remember  isomorphisms
so  but  you  can  get  you  can  somehow
recover  the  full  mapping  category  uh  by
using  this  uh  tensor  structure  that's
that's  one  way  of  recovering  it  at
least
um  so  in  principle  PRL  should  be
considered  as  an  Infinity  2  category
because  there's  a  whole  category  of  map
between  two  objects  but  you  don't  really
need  to  remember  the  Infinity  2
categorical  structure  because  it's  it's
just  also  just  the  internal
hm
um  you  mean
equival  isomorphism  in  this  Theory  I
mean  after  to  almost
okay  this  is  Infinity
two  yeah  there  was  a  reference  to  a
paper  on  Infinity  to  in  some  places  but
so  do  they  Define  infinity  and
categories  and  all  of  this  in  general  or
I  haven't  kept  up  with  with  that  I  don't
know  yeah  yeah  SE  some  recent  paper  or
someone  I  know  about  this  infinity
infinity  or  yeah  I  uh  I  haven't  kept  up
with  the  literature  on  infinity  infinity
so  far  I've  been  fine  with  just  Infinity
one  so  when  the  when  the  need  arises
oh
yeah  um
okay  okay  I  have  to  or  I  want  to
continue  a  little  bit  more
uh  uh
so
um  so  there's
more  so  if  you  have  a  so  now  you
can  this  is  PRL  is  now  a  um  a  tensor
category  it's  a  category  with  a
symmetric  monoidal  structure  and  you  can
ask  what  is  a  commutative  algebra  in
this  tensor  category  so  meaning  just  a
it's  an  object  in  here  equipped  with
some  kind  of  symmetric  multiplication  I
mean  the  basic  data  is  this  but  then
there's  some  higher  coherences  and  so  on
about  it  being  sufficiently
commutative  um  and  what  it  what  it  means
is  that  is  this  is  the  same  thing  as  C
is  a  symmetric
monoidal  uh
presentable  Infinity  category  and  the
tensor  product  on
C  commutes  with  Co
limits  uh  in  each  variable
um  some  people  often  call  this  because
this  is  a  a  mouthful  people  often  call
this  a  presentably  symmetric  monoidal
Infinity  category
so  um  and  uh  well  basic  example  for  us
would  be  a  d  of  X  for  an
apoid
um  uh  and  then  you  can  consider  so  can
consider  modules  over  C  in
PRL  uh  these  are  sort  of  a
cinear  uh
presentable
Infinity
categories  so  they're  presentable
Infinity  categories  which  are  tensored
over  a
c
um  but  that  also  lets  you  say  that
they're  enriched  over  C  in  some  sense  so
um  so  what  yeah  what  do  you  have  you
have  like  C  TENS  for  M  goes  to  M
um
uh  yeah  so  that's  the  basic  structure
that  you  have  here  um  but  then  if  you
have  any  two  objects  like  M  and  N  in
here  you  can  make
a  you  can  make  sort  of  a  c  internal  home
from  M  to  n  uh  as  some  adjoint  to  a  fun
you  can  easily  write  down  here  so  so
maps  from  an  object  in  C  to  this  should
be  the  same  as  maps  from  C  tensor  M  to  n
in  m  so  it's  a  nice  it's  a  convenient
way  of  saying  uh  you  have  a  presentable
Infinity  category  where  the  mapping
spaces  are  enriched  to  objects  in  C
yeah  can  you  say  again  if  you  have  a  c
to  M  to  how  do  you  en  the  mapping  space
Oh  you  want  to  so  you  you  characterize
this  by  a  universal  property  and  then
you  prove  there's  a  a  representing
object  so  the  Universal  Property  is  that
if  you  have  a  giving  a  map  like  this  is
the  same  thing  as  giving  a  map  from  C
tensor  M  to
n  um  I'm  almost  done  with  the  4  a  into
into  PRL  um  I  promise  it's  going  to  be
helpful  so  when  you  T  like  category  of
modules  over  AR  like  Dr  R1  t  d  R2  should
be  so  as  far  as  I  know  if  you  T  not  over
D  of  Z  you  get  D  of  the  the  T  you  T  like
this  you  get  something  big  yes  okay  yeah
so  let  me  yeah  that's  uh  yeah  you  what
you  get  something  bigger  um  well  unless
you're  in  characteristic  zero  um
but  uh  yeah  so  maybe  let  me  give  an
example  of  uh  no  no  sorry  so  that  now
yeah  just  a  I  I  want  to  continue  the
story  just  a  little  bit  so  mod
cprl  also  has  all  limits  and
colimits  and  mod  C
PRL  to  PRL  preserves  them  the  forgetful
functor
so
completely  uh  completely  naive  uh  very
simple
Theory
um  uh  it  also
has  has  a  tensor
product  which  is  a  tensor  product  over
c
um  so  so  this  is  kind  of  a  standard
thing  with  some  geometric  realization  of
of
uh
yeah  some  some  some  standard  relative
tensor
product
um
uh  and  uh  let  me  give  an  example  of  a
calculation
so
so  if  um  so  if  you  have  c  c
PRL  and  then  you  have
R  uh  and  S  which
are  commut  algebra  objects  in
C  um  then  you  can  make  mod  r  c  and  oh
let's  just  say  algebra  objects  yeah  and
so  you  can  consider  left  modules  over  r
or  S  and  these  will  be  in  uh
mod
cprl  um  so  just  R  modules  in  C  S  modules
in  C  and  then  mod  r  c  tensor  over  C  Mod
s
c  uh  is  just  mod  R  tensor  s  c
so  so  in  the  definition  of  c  linear  s  of
course  you  say  C  tens  to  him  but  of
course  this  comes  with  yeah  yeah  there's
more  coherencies  and  there  is  a  good  way
to  formulate  this  system  of  coherencies
in  this  so  we  have  to  read  some  higher
algebra  you  have  to  read  higher
algebra  but  you  know  yeah  you  you  yeah
it's  it's  not  easy  to  read  higher
Algebra  I  know  but  it's  possible  people
have  done
it  and  someone  even  wrote  it  which  is
even  more
amazing  yeah
okay
uh
so  tser
products  okay  so  now  I  want  to  connect
this
to  analytic  rings  so  note  uh
so  there's  a
functor  or  well  maybe  so  yeah  so  now  now
I  can  say  the  the  good  way  to
encode  or  a  good  way  at  least  to  encode
a  six  funter
formalism  is  a  is  a  LAX  symmetric
monoidal
functor  uh  from  span  C
s  to
PRL  with  this  tensor  product
here  so  we're  no  longer  using  the
cartisian  product  on  cat  Infinity  but
the  this  tensor  product  on  PRL  but  it
really  just  amounts  to  a  condition
on
uh  on  the  the  kind  of  formalism  we  had
in  the  other  sense  it's  just  the
condition  that  when  you  look  at  the  this
D  ofx  cross  D  of  Y  going  to  D  of  x  x
cross  y  that  that  should  commute  with  Co
limits  in  both  D  and  x  and  d  and  y  a  d
of  Y  separately
so  um  it's  just  a  condition  on  this
formalism  here  but  it's  best  to  think  of
it  as  being  a  formalism  with  values  in
PRL  okay
um  um  now  I  want  to  connect  with  our
specific  example  so
note  um
so  analytic
Rings  uh  maps  to
PRL  uh  by  sending
um  our  triangle  D  of
R  to  uh  D  of  R  well  in  fact  as  I  already
said  it  maps  to  C  alge
PRL
um  but  uh  in  fact
uh
it  it  it  maps  to  cge  PRL  uh  over  a
certain  other  commutative  algebra  object
namely  a  derived  category  of  condensed
to  bilon  groups  so  everything  by  by  its
nature  lives  over  this  derived  category
of  condensed  to  bilion  groups  so  we  have
a  presentably  symmetric  monoidal
category  with  a  functor
from  this  uh  presentably  symmetric
monoidal
category  um  and  then  I  claim
uh  this
functor  commutes  with
colimits  uh  and  and  and  detects
isomorphisms  under  over  under  thank  you
under  yes  yeah  we're  in  the  world  of
algebra  so  I  should  say  under  yeah  thank
all  so  we  have  this  we  have  this
category  here  and  then  we  have  an  object
in  this  category  and  this  notation  means
that  you  consider  an  object  in  this
category  together  with  a  map  from  this
object  to  to  that  object  so  it's  this
some  slice  category  or  under  under  or
over  category  or
something  condense  every  condens  every
complex  of  condensed  abil
Co  gives  you  by  pull  back  some  guy  in  of
okay  yeah  I  mean  it's
just  yeah  I  mean  the  so
the  the  initial  object
here  I  want  to
say  it  always  SS
to  see  over  over  IM  of  the  addition  ah
okay  that's  that's  a  analytic
ring
to
uh  what's  the  initial  object  Z  okay  so
for  Z  you  get  D  of
condens  so  you  don't  need  them  ah  okay
CLA  this
F  just  yeah  just  a  sec  just  a  sec  so
uh  in
particular  um  so  if  you  want  to  so  the
derived  category  of  a  pushout  so  uh
let's  say
a  so  this  is  push  out
in  in  analytic
Rings  which  was  this  kind  of  um  slightly
subtle  operation  in  this  perspective  on
analytic  Rings  because  you  had  to
complete  some  pre-analytic  ring
structure  and  so  on  but  actually  on  the
level  of  the  categories  it's  quite  naive
so  it  is  just  this  lury  tensor
product
um  this  is  the  relative  tensor
product  uh  in
PRL
um  so  the  proof  is  not  um  so  difficult
so  let  me  indicate  what's  going  on  um  in
the  proof  uh  just  in  this  special  case
which  is  really  all  we  need
um  so  in  the  case  where  A  and  B  are  both
proper  over  R  I'm  not  assuming  any
shriek  ability  but  still  let  me  use  this
language  of  proper  Maps  so  in  the  case
When  A  and  B  are  proper  over  R  um  it's
just  an  instance  of  this  general  fact
here  so  did  you  did  we  say  that
commutative  and  now  got  confus  all  this
so  somehow  you  get  you  said  the  modules
as  limits  and  Co
limits  and  then  did  you  say
that
symmetric  oh  I  didn't  say  that  c  alge
has  co-  limits  I  should  have  maybe
discussed  it  but  it  does  so  and  um  as
usual  pushouts  are  are  calculated  by
relative  tensor  products  pushouts  in  cge
PRL  are  calculated  by  relative  tensor
products  in
PRL  okay  and  and  and  it  has  limits  also
it  has  limits  also  those  are  yeah  those
are  naive  those  are  calculated  in  PRL  it
has  Co  limits  I  would  it  has  arbitrary
Co  limits  yeah  and  filtered  Co  limits  um
or  more  generally  sifted  co-  limits  are
calculated  again  in
PRL  so  yeah  um  right  so  in  the  case  case
where  A  and  B  are  proper  over  R  um  it's
an  instance  of  this  general  fact  in  the
general  case  um  well  it's  not  shable  but
you  can  still  Factor  by  basically  by
definition  any  map  of  analytic  rings  as
a  a  proper  map  followed  by  a
localization  um  and  then  for
localizations  it's  quite  easy  to  um  just
check  Universal  properties  uh  to  compare
the  two  sides  of  this  so  granting  the
case  of  proper  Maps  um  then  you  just
have  two  different  quotients  of  your
category  that  you're  trying  to  identify
and  you  can  just  identify  them  by  uh
looking  at  the
descriptions
um  so  for  proper  you  say  that  it
is  because  in  that
case
the  uh  in  this  in  that  case  for  proper
map  this  D  of  is  is  the  the
algebra  yeah  are  algebra  over  the  bigger
ones  and  you  said  that  algebra  are
yeah  no  for  any  analytic  GRE  it's  not  an
category  the  D  of  the  source  is  not
algebra  over  because  in  the  calization
you  have  the  K  algebra  something  but  not
yeah  okay  okay  and  that  was  only  for
shable  maps  also  um
so  right  so  let  me  make  a  corollary  H  uh
so
so  no  no  no  here  no  shable  it's  not
necessary  I  got  but  you  said  something
about  D  of  a  10  B  Ral
D  this  is  for  general  not  necessarily
right
right  but  you  can  still  use  this
reduction  to  the  proper  case  so  you  you
still  Factor  as  a  every  map  of  analytic
ring  still  factors  is  a  proper  map
followed  by  a  localization  but  this
localization  won't  have  a  left  ad  joint
I  mean  it  won't  but  that  doesn't  matter
so  much  for  this  argument  okay  so  you
don't  need  the  i  s  okay  so  so  for
localization  in  general  you  can  somehow
check  it  from  gring  all  your  stuff  and
then  you
yeah  okay  um  oh  yeah  I  wanted  to
say  uh  this  functor  is  not  fully
faithful  for  a  technical  reason  so  to  be
fully  faithful  you'd  have  to  be  able  to
recover  our  triangle  just  from  uh  the
data  of  the  of  of  D  of  r  with  its
tensoring  over  condensed  ailan  groups  so
you  can  certainly  recover  the  underlying
condensed  ailan
group  of  our  triangle  because  you  can
take  you  have  the  unit  object  in  here
and  then  you're  tensored  over  this  so
you  can  consider  endomorphisms  kind  of
or  the  internal  uh  the  internal  the
internal  endomorphisms  of  this  and  this
will  give  you  our  triangle  so  over  this
linearity  as  a  condens  as  a  an  object  in
the  derived  category  of  condensed  ailan
groups  in  fact  you  can  do  more  because
this  is  the  unit  you  actually  get  this
as  a  uh  in  fact  you  get  it  as  a
commutative  algebra  in  this  category  um
condens  stab
um  but  uh  so  so  you  recover  the
underlying  infinity
ring  but  if  you're  working  in  this
animated  ring  context  that's  not  enough
to  recover  the  the  animated  ring  so
there's  there's  not  quite  enough
structure  here  in  the  animated  ring
context  to  to  recover  the  r
triangle  um  there're  certainly  enough  to
detect  isomorphisms  because  uh  yeah  you
get  the  underline
object  um  if  you  wanted  to  recover  the
animated  ring  you'd  have  to  do  something
like  remember  derived  functors  of
symmetric  Powers  incode  them  somehow  uh
say  what  kind  of  extra  structure  you
have  on  this  coming
from  uh  non  ailan  derived  functors  of
symmetric  Powers  which  is  the  kind  of
extra  structure  you  get  on  modules  over
an  animated  ring  compared  to  modules
over  an  in  infinity  ring  which  only  has
the  symmetric  monoidal
structure  okay  um  yeah  uh  any  anyway
there's  a  corlar  so
let's  let's  say  you  take  a  I  don't  know
Y  in
f  um  and  then
consider  F  shriek  over  y  this  is
the  uh
category  of  X  mapping  to  Y  via  a  shable
map
um
then  the  six  funter
formalism  so  we  can  we  again  look  at  the
span  category  of  this  but  now  we  don't
need  to  restrict  to  shable  Maps  anymore
because  every  map  in  this  category  is
shable  by  this  kind  of  two  of  three
property  of  shable  maps  so  I  can  just
put  all
all
um  um  which  is  a  prior  lack  symmetric
monoidal  um  it's  actually  symmetric
monoidal  uh  wait  should  I  uh  no  I  want
to  put  a  sorry  I  want  to  uh  ah  sorry  I
should  do  sorry  I  want  to
yeah  yeah  I  have  to  remember  the  the
coefficient  category  linear  structure
over  PRL  yeah
did  um  because  of  what's  going  to  come
next  okay  but  I  mean  the  fact  that
actually  yeah  that's  true  that's  true
yeah  but  uh  yeah  I  wanted  to
yeah
okay
um  uh  right
so  man  I  had  no  idea  it  was  going  to
take  this  long  to  get  through  this
formal  stuff  uh
so  so  a
corollary  is  that  uh  for
X  to  Y
shable  uh  then  this  object  D  of  X  in
modules  over  D  of  Y  in  PRL  uh  is
dualizable  uh  with  respect  to  the  tensor
product  the  relative  tensor  product  over
D  of  Y  uh  and  in
fact  it's  canonically
selfdual
and
um  with  respect  to
this
self-duality  uh  the
Dual  of  uh  so  if  so  so  if  you  have  X
going  to  X  Prime  over
y  uh  then  the
Dual  of  uh  let  say  pull
back  um  so  it  should  be  a  map  the  Dual
of  this  map  should  be  a  map  from  the
Dual  of  x  to  the  Dual  of  the  Dual  of  DX
to  the  Dual  of  DX  Prime  but  they  self
duel  so  you  just  get  a  map  in  the  other
direction  from  D  of  x  to  D  of  X
Prime
uh  this  map  is  uh  none  other  than  the
lower  streek
functor
so  the  proof  proof
is  you  can  check
universally  uh  in  any  span
category  kind  of  span  C  all
all
um  so  we  have  a  symmetric  monoidal
functor  so  if  you  want
to  so  the  image  of  a  dualizable  object
will  be  dualizable  and  if  you  exhibit  a
duality  pairing  in  co-  pairing  here  you
get  one  there  and  so
on  um  so  so  for  examp  so  how  so  why  is
every  object  dualizable  in  this  Span
selfdual  in  this  span  category  well  you
have  a
unit  which
is
uh
um  right
so
uh  and  then  you  have  a
co-unit  um  which  is  the  same  thing  going
in  the
reverse  and  then  you  just  check  the
triangle  identities  it's
completely  completely
straightforward  and  then  you  can  see
that  the  with  respect  to  this  this
self-duality  of  every  object  that
passing  to  the  duel  is  just  the  same
thing  as  transposing  the  span
so  to  define  the  C  you  need  the  pairing
from  BX  and
BX  either  up
what's  DX  TS  of  DX  to  the  unit  which  is
dy  so  what  uh  so  yeah  so  if  you  unwind
this  then  what  you  get  is  so  you  have  DX
tensor  DX  implicitly  point  is  y  here  uh
so  the  C  in  this  example  will  be  uh  this
fek  over  y  so  then  you  get  you  have  DX
tensor  over  Dy  DX  you  pull  back  to  DX
and  then  you  do  lower  shriek  to
Dy
to
um  so  then  we  have  a
proposition  so  that  if  F  from  X  to  Y  is
a  shable  map
uh  then  the  following  are  equivalent  so
Peter  uh  uh  said  this  last  time  but  he
didn't  give  the  proof  and  now  we're
going  to  give  the  proof  so  one  is  that
um  f  satisfies  streak
descent  um  two  is  that  for  all
uh  for  all  categories  linear  or  over
D  of  Y
um  uh  so  you  have  descent  for  for  m
so  um
so
so  if  you  take  m  equals  D  of  Y  uh  you
get  star
descent  or  in  other
words  what  what's  what's  that  what's
that
Peter  equival  oh  thank  you  yeah  yeah
yeah  um  so  that  that  so  that  so  by
definition  shriek  descent  mean  that  you
have  descent  with  respect  to  this  weird
pullback  these  real  weird  shriek
pullback  funs  on  on  these  D's  but  then
the  conclusion  here  is  that  you  get  it
also  automatically  for  the  star  pullback
instead
um
uh  and  then  the  third  condition  is  kind
of  a  a  categorified  version  of  star
descent  so  star  s  it's  called  two
descent  uh  studied  by  Gates  gory
uh  uh  and  Rosen
bloom  is  kind  of  a  remarkable  thing  so
that
um
so  if  you  take  this  whole  category  uh
then  that's  the  same  thing  then  that
that  assignment  assigning  to  why  this
category  also  satisfies  descent
so
in  the  sense  of  infinity  one  but  if  you
have  I  I'll  explain  more  or  less  why
it's  the  same  requiring  this  for
Infinity  one  and  for  Infinity  2  but  I'm
not  going  to  try  to  say  I  know  what  an
Infinity  2  means  so  but  you  well  in  in
the  proof  that  three  implies  two  I  mean
well
so  do  you  have  also  a  Shri  version  of
two  yes  you  have  a  shriek  version  of  two
yeah  but  that's  uh  that's  completely
formal  from  one  so  let's  get  into  the
proof  and  then  it'll  become  more
clear
or  maybe  let  me  State  a  corollary  first
um  or
yeah  so  a  corollary  is  that
uh  a
pullback  uh  F  satisfies  streak
descent  implies  any
pullback  also
does
um  you  can  see  that  from  Two  and  using
this  symmetric
monoidal  uh  that  we  discussed  but  also  I
I  think  in  the  course  of  the  proof  a
more  simp  a  simpler  explanation  will
arise  for  why  streek  descent  is  uh
closed  under  pullbacks  so  that's
important  for  using  this  to  define  a  Gro
deque  topology  in  the  first
place
um  okay  what  why  the  P  because  the  D  of
the  yeah  because  the  pullback  in
analytic  Rings  is  already  calculated  by
a  relative  tensor  product  so  if  you  took
if  you  took  for  example  M  to  be  D  of  Y
Prime  then  this  would  give  be  give  you
star  descent  for  the  pullback  but  you
could  take  M  more  generally  to  be  any  D
of  Y  Prime  module
and  yeah  and  then  you  get  so  You'  see
that  condition  two  is  stable  under  under
pullback  under  base
change
um
so
so  um  so  what  do  we  have  we  have  star
streek  descent  but  I  explained  that  the
best  way  to  think  of  that  is  that
um  is  that  you  have  this  co-  limit  in
PRL
uh  in
PRL  but  I  also  explained  that  co-  limits
in  module  categories  are  the  same  uh  as
on  the  underlying  so  that's  the  same
thing  as  modules  over  D  of  Y  in
PRL
um  uh  now  take  internal  hom
out  uh  to  M  for  for  any  m  in  mod  D  of  Y
PRL  then  this
uh  so  um  well  on  the  right  hand  side
we're  taking  internal  hum  from  the  unit
to  M  so  we  just  get  M  and  it's  being
identified  isomorphically  with  some
limit  so  the  the  co-  limit  pulls  out  of
the  internal  H  uh  limit  of  uh  uh
internal
H  uh  from  D
of  uh
whoops  but  uh  sorry  what  did  I  what  did
I  just  write
D  but  I  then  I  explained  over  here  that
this  thing  is  selfdual  it's  dualizable
and  in  fact
selfdual  uh  so  then  you  can  actually
move  it  over  to  here  with  a  tensor
product
so  so  this  is  H  internal  H  in  D  of
Y  uh  M  and  I  said  that  the  manner  in
which  it's  self  duel  is  is  such  that  it
converts  shriek  functors  into  lower
shriek  functors  into  upper  star  functors
so  then  this  is  exactly
a
two  I  guess  in  fact  this  is  a  proof  that
one  is  equivalent  to  two
course  you  have  to  verify  that  the  the
con  that  you  get  is  really  the  expected
one  in  some  higher  yeah  yeah  you  do  it
universally  in  the  span  category  kind  of
the  this
identification  the  self  Duality  turns
this  system  to  the  other  yeah  to  the
expected  one  yeah  but  there  is  some
delicate  Point  absolutely  okay  it's  not
because  of  the  higher  nature  of  the  SC
not  so  sure  what  you  you  have  to  check
so  yeah  yeah  you  have  to  produce  the
data  coherently  ident  you  know  you  have
to  produce  the  data  making  this  coherent
identification  of  the  Dual  of  this  of
this  self-duality  and  and  the  Dual  of
this  identifying  with  this  you  produce
it  universally  in  the  span  category  and
then  you  map  it  into  modules  over  D
ofy  and
um  I  haven't  actually  done  that  but
there's  also  another  argument  for  this
so  that  doesn't  require  this  stuff  I
just  wanted  um  I  mean  it  can  be  done  I'm
100%  sure  but  yeah  I  haven't  haven't
actually  done  it
but  there's  also  another  independent
argument  for  this  which  doesn't  require
such  things  I  just  thought  I  would
present  this  because  it  feels  the
clearest  to  me  even  if  it's  slightly
difficult  to  make  technically
work
um
okay  uh
right
um  okay
okay  and  then
uh
so  2  implies
three
um  so  yeah  we  have  to  check  this
statement  here
um  so  there  is  a  so  the  well
this
the
map  has  a  right
joint  uh  which  is  basically  you  take  a
system  of  modules  here  and  you  forget
them  all  to  D  of  Y  um  and  then  you  take
the  limit  in  this  category  here  so  it's
like  taking  a  system  of  modules  MN  and
then  you  take  go  to  the  limit  of  the  MN
um  and  you  want  to  check  that  the  unit
and  the  co-unit  uh  of  this  adjunction
are
isomorphisms  so  the  the  unit  being  an
isomorphism  uh  this  is  exactly
two
um
for  the  co-unit  being  an
isomorphism  um  so  it's  true
after
uh  if  you  base  change  up  to  x  uh  then
the  co-unit  is  an
isomorphism
um  uh
because  so  one  so  d  X  is
dualizable  over  D  of  Y  implies  that  can
you  can  pull  the  limit
out  um  and  then  two  the  cover  is
split  uh  on
pullback  uh  to  D  of  Y  or  D  of
X  and  then  three
uh  there's  an  obvious  base  change
property
um
so
yeah
um  B  change  probity  you  mean  yeah  I  mean
that  this  the  write  a  joint  say  from  Mod
so  you  have  a  mod  D  of  Y  PRL  you  have  a
pullback  funter  to  mod  D  of  X  PRL  and
then  you  have  the  right  ad  joint  which
is  the  forgetful  funter  and  that
forgetful  funter  commutes  with  base
change  or  where  base  change  means  push
out  in  cge  PRL  so  it's  just  like  purely
algebraic  base  change  like  no  I  don't  uh
yeah  so  first  of  all  you  say  the  co  unit
means  the  map  from  from  something  to  the
IND  limit  of  the  PB  yes
yeah
yeah  right  so  you  have
um  yeah  so  you  have  a  a  collection  of
things  here  you  take  its  limit  and  you
base  change  it  back  I  guess  I  wasn't  um
yeah  I  yeah  I  I  think  you're  right  I
didn't  set  this  up  quite  correctly  um  so
then  you  have  a  a  map  of  systems  here
and  you  want  to  check  it's  an
isomorphism  for  all  n  um  and  then  maybe
I  was
um  yeah  there  is  a  confusion  of
unit  and  think  your  unit  and  unch  are
they  really
oh  you're  wait  no  are  they  no  no  no  no  a
unit  is  an
isorm  sorry  no  no  no  no  no  no  no  no  the
unit  is  the  right  one  so  the  map  from  a
guy  to  okay  the  unit  is  the  right  one
the  co  me  you  have  a  system  and  you  want
to  compare  it  to  the  full  of  limit  which
is  okay  and  then  you  say  so  I  was  I  was
giving  the  argument  when  n  equals  1  and
I  should  have  yeah  sorry  I  think  okay  so
let's  go  again  the  cor  being  in  is
is
yeah  okay  so  we  we  we  don't  say  after
going  to  dxx  because  this  is  confusion
with  the  co  and  we  are  we  are  completely
we  are  completely  okay  then  let's  see
because  one  because  it's
dualizable  any  limit  commutes  with  no
but  this
is  initially  should  from  t  t
x  oh  I  should  yeah  I  should  Bas  change
the  whole  system  from  D  of  Y  to  D  of  X
yeah  yeah  that's  that's  that's  good  yeah
yeah  yeah  no  no  yeah  yeah  so  than  thanks
thanks  thanks
yeah
yeah  true  After  Base
change  to  D  ofx  yeah  yeah  in  other  words
you  take  this  whole  collection  here  and
you  base  change  to  D  of
X  right  um  yeah
so  uh
um  change  you  tensing  with  x  in  your  yes
tensor  over  D  of  Y  with  d  of
X  and  then
the  the  limit  can  be  commutes  commutes
so  you  can  do  it  in  the  system  after
tensoring  and  after  tensoring  you  get
the
system  uh
of  the  are  categories  for  the  fibro
product  space  a  fine  so  to  speak  spaces
because  we
said  the
d  d
is  some  D  of  the  F  product  is  D  of  the  T
of  product  of  the  D  for  Fine  Things  yes
or  not
okay  and  then  the  covering  is
plit  and  so  by  as  as  in  usual  the  same
the  this  means  that  you  can  play  with
some  amot  Toopy
to  okay  then  you  you  but  and  so  what  is
point  three  the  B  change  property  oh
that's  so  that  you  can  identify  the  when
you
tensor  the  thing  up  to  D  ofx  that  you
can  identify  that  system  with  the
corresponding  system  for  the  pullback  of
the  cover  to  D  ofx  so  it's  it's  some
it's  it's  it's  nothing  I  mean  it's  it's
it's  purely  formal  um  okay  so  here  there
is
no  okay
but  kind  of  it's  hard
to  how  to  be  sensual  about  those  things
well  actually  yeah  I  mean  I  think
actually  this  argument  I  believe  is  in
Kil  Matthew's  paper
yeah  and  maybe  it's  also  in  Gates  Gore
Rosen  bloom  and  so  on  I  mean  which  M  you
galwa
galwa
gwa  the  Tre  refer  to  as  some  in  the  last
it's  the  original  reference  for  descend
ability  it's  like  yeah  the  reference  I
wrote  down  last  time
that
yeah  um  there  was  something  in  back  or
they  yeah  then  they  they  they
characterized  the  descendible  maps  so  in
some  type  context  um
yeah  um  okay  I  man  okay  it's  okay  it's
okay  we'll  continue  again  after
Christmas  um  so  uh  right  okay  so  then
the  so  then  this  collection  and  then  the
image  of  this  and  then  this  those  two
things  are  the  same  after  isomorphic
after  you  tensor  to  DX  but  then  two  also
implies  that
uh
tensoring  uh  DX  detects
isomorphisms  so
done
dets
two
also  ah  all  right  right  all  right
okay  uh  yes  yes
okay
yeah  it's  a  bit  of  magic  but
yeah  um  and  you  get  also  now  a  strong
form  of  three  with  the  two  Infinity
right  so  let  me  quickly  explain  the
argument  for  three  implies  two  which  is
kind  of  more  or  less  explaining  why  you
get  a  strong  form  of  three  so  if  you
take  mapping  spaces  so  yeah  if  you  have
let's  say  you  have  a  what's  that
Peter
said
right  you  cies  oh  that's  true  yeah  that
yeah  so  I  don't  need  to  yeah  you're
right  yeah  unit  being  yeah  you're  right
oh  yeah  Peter  I  was  going  to  give  a
different  argument  Peter  is  right  that
this  argument  shows  that  um  two  is
exactly  the  claim  that  the  unit  is  an
isomorphism  so  uh  so  clearly  three
implies  two  um  but  I  was  going  to  give  a
different  argument  which  maybe  also
explains  why  you  get  an  Infinity  2
categorical
uh
three  two
because  so  three  means  that  this  functor
is  an  equivalence  but  this  functor  has  a
right  ad  joint  so  being  an  equivalence
is  the  same  as  the  unit  and  the  co-unit
being  isomorphisms  but  when  you  unwind
what  it  means  for  the  unit  to  be  an
isomorphism  you  get  exactly  two  so
clearly  three  is  stronger  than  two  three
as  a  right  the  right  joint  being  ah  okay
the
the  uh
being
what  the  in  ah
okay  so  and  then  the  being  the
unit  ah  okay  so  this  actually  shows  that
you  have  two  yeah  going  giving  another
proof  yeah  okay  you  said  you  have
another  yeah  there's  another  way  of
seeing  it  which  is  I  mean  one  way  of
okay  so  if  you  have  just  three  in  the
infinity  one  categorical  sense  what  do
you  get  you  get  that  if  you  have  M  and  N
in
mod  uh  D  of  Y
PRL  uh  then  you  get  the  the  mapping
space  in  mod  y  d  of  Y  uh  PRL  from  M  to  n
uh  is  the  inverse  limit  of  the
corresponding  mapping  spaces  after  you
do  the  base  changes  and  this  is  um
this  is  the  set  of  objects  in  some
internal  H
so
uh
uh  so  it's  a  priori  weaker  than  the
claim  that  for  the  internal  HS  you  get  a
corresponding  limit  statement  which
would  be  kind  of  the  Infinity  2  uh
version  of  this  um  but  if  you  just  you
can  just  tensor  this  with  like  prees  on
Delta  n  for  any  n  and  then  you'll  get
the  um  you  know
fund  from  Delta  n  to  this  and  then  if
you  let  N  vary  um  then  you  recover  the
whole  well  complete  seal  space
Associated  to  this  category  just  in
terms  of  mapping  spaces
and  yeah  so  I'm  saying  what  you  get  a
priori  is  the  under  the  space  of  objects
in  this  Infinity  category  that  you're
interested  in  but  if  you  then  vary  The
Source  by  tensoring  with  prees
on  on  Delta  n  and  that  tensoring  kind  of
commutes  with  everything  uh  then  you  get
not  just  the  set  of  objects  but  the  set
of  morphisms  the  set  of  composable  pairs
of  morphisms  the  set  of  etc  etc
etc  elaborate  saying  again  that  theu
these  things
deter  yes
yes  yeah  yeah  I  was  just  giving  it  a
little  more
explicitly  yeah  and  so  D  Delta  n  is  is
bres  on
on  what  is
it
sense  uh  oh  I  mean  I've  Delta  n  is  like
a  finite  category  and  I  have  to  put  it
in  this  big  world  of  PRL  so  I  just  take
pre-  sheeps  of  whatever  spaces  I  mean  if
or  if  I  want  this  to  be  a  tensor  over  D
of  Y  then  I  should  do  pre-s  with  values
in  D  of
Y  and  Delta  n  is  the  simpli
finite  yeah  category  or  opposite
yeah  good  question  actually  because  I
want  maybe  I  want  to  but  they're  they're
isomorphic  so  maybe
[Laughter]
I
yeah
um
okay  so  this  was  uh  roughly  half  of  what
I  wanted  to  do  today  so  let's  go  for
another  two  hours  shall  we  no  I'm
kidding  yeah  um  so  yeah  we're  taking  a
little  break  I  I  think  the  next  lecture
we  scheduled  for  the  10th  of  January
which  is  a  Wednesday  so  I'll  be  I  guess
picking  up  and  actually  finishing  what  I
intended  to  do  today  giving  so  this  was
some  consequence  of  shriek
descent  um  H  Maybe  just  mention  one  more
Coral  are  of  this  which  is  that  uh
topology
of  of  shriek
descent  is  subconical
so  so  that  aphids  will  embed  fully
Faithfully  into  the  sheath  category
so  um  yeah  so  those  are  some
consequences  of  shriek  descent  you  get
some  very  strong  star  descent  results
and  next  time  I'll  talk  about
uh  examples  how  you  how  you  produce
examples  of  Street
covers  so  logically  so  just  to  get  this
so  you  you  you  said  that  in
three  you
have  St  the  world  Infinity  one  then  you
tasing  with  d  of  Delta  and  you  get  it  in
the  sense  of  of  infinity  so  at  least  the
mapping
spaces  the  mapping  c  not  not  the  mapping
spaces  the  mapping
categories  are  the  limits  of  the
corresponding  ones  group  are  the  mapping
spaces  yes  this  is  the  mapping
categories  are  equivalent  yes  in  in
three  and  from  this  you  deduce
two  how  do  you  deduce  from  c  2  oh
because  you  could  just  take  the  mapping
Infinity  category  from  the  unit  object  D
of  Y  to  M  and
then  ah  okay  so  once  you  know  the
mapping  spaces  you  get  two  another  way
so  to  speak
and  and  Peter  said  that  the  if  you  have
equivalence  in  Infinity  one  and  Infinity
2  follows  by  it  said  something  different
than  your  argument  or  uh  I  don't  know
but  yeah  one  thing  to  note  is  that  you
know  there's  there's  two  parts  yeah  if
this  you  want  this  to  be  an  equivalence
of  Infinity  2  categories  you  need  it  to
be  fully  faithful  and  essentially
subjective  but  now  fully  faithful  means
isomorphism  on  mapping  category  is  that
we  just  explained  but  in  essential
surjectivity
um  actually  follows  from  just  the
infinity  one  statement  because  here  you
just  have  a  bunch  of  yeah  yeah  so  then
fully
faceful  in  the  infinity  one  sense  is
weaker  than  fully  faceful  exactly  a
priori  PRI  but  in  context  like  this  it's
actually
equivalent  uh  you  mean  when  you  have
a  a
T  when  you  have  some  kind  of
a  yeah  when  you  can  perform  this
trick  well  okay  thank  you  everybody  see
you  uh  see  you  after  the
break
\end{unfinished}