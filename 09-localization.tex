% !TeX root = AnalyticStacks.tex

\section{\ufs Localization of solid analytic rings (Clausen)}

\url{https://www.youtube.com/watch?v=lJTLj8gYAtg&list=PLx5f8IelFRgGmu6gmL-Kf_Rl_6Mm7juZO}
\renewcommand{\yt}[2]{\href{https://www.youtube.com/watch?v=lJTLj8gYAtg&list=PLx5f8IelFRgGmu6gmL-Kf_Rl_6Mm7juZO&t=#1}{#2}}
\vspace{1em}

\begin{unfinished}{0:00}
e
so  um  welcome  back  everyone  um  so  today
um  we're  going  to  be  doing  a  little  bit
of  localization  in  a  in  a  certain
framework  in  the  setting  of  solid
analytic
Rings  um  but  first  I  want  to  start  with
a  uh  a  recap  of  last
time  um  so  last  time  we  defined  the
notion
of  analytic
ring  um  and  that's  a
pair  r  with  funny  triangle  uh  and  then
well  the  whole  parel  being  denoted  just
plain  old  R  and  it  consists  of  an  r  with
a  funny  triangle  and  what  we  would  want
to  call  the  category  of  R  modules
where  um
uh  so  this  triangle  R  is  a  condensed
string
commutative  uh  and  this  mod
R  is  a  full
subcategory  of  mod  R  triangle  by  which  I
mean  condensed  R  triangle
modules  um  satisfying  some  axioms  some
closure  axioms  and  the  first  one  is  that
R  is  closed
under  all  limits
colimits  and
extensions  in  particular  it  should  be  an
aelon
subcategory  um  but  even  more  because  I'm
allowing  infinite  colimits  and  limits
here  um  and  the  second  is  a  bit  funny  so
if  m  is  in  uh  mod  R  triangle  and  N  is  in
mod
r  then  we  want  to  require  that  the
internal
XS  uh  in  this  category  uh  from  M  to  n
should  also  lie  in  the  smaller
category  uh  for  all
I  and  this  um  condition  was  necessary  to
get  a
good  tensor  product  uh  on  this  category
and  on  the  derived  analog  as  well  I'll
make  a  couple  more  remarks  about  that  in
a  second  um  and  then  the  the  third
condition  is  that  our  triangle  itself
the  unit  in  this  category  mod  R  triangle
should  lie  in  mod
R  what  about  the  existence  of  our  joints
which  said  one  so  there  were  a  couple  of
technical  points  that  came  up  last  time
that  were  not  uh  fully  addressed  that
I'll  take  the  time  to  address  now  um  so
there's  a  and  one  of  them  was  indeed
about  the  whether  the  existence  of  a
left  adjoint  is  automatic  or  or  not  and
there  was  a  claim  that  it  was  and  it's
justified  by  this  theorem  of
uh
uh
ademic
rosit  they  call  called  reflection
principle  so  if  C  is  a  presentable
category  uh  which  used  to  be  called
locally  presentable  um  but
and  if  D  is  a  full
subcategory  so  it's  a  it's  a  version  of
an  adjoint  functor
theorem  uh  closed  under  all
limits  now  that's  the  condition  under
which  you  might  reasonably  expect  a  left
ad  joint  to  the  inclusion  but  um  there
are  set  theoretic  technicalities  that
get  in  the  way  in  general  um  but  if  you
also  uh  Plus
uh  there  exists  a  regular
Cardinal  Kappa  such  that  D  is  closed
under  all  Kappa  filtered  Co
limits  um  so  if  Kappa  is  the  first
infinite  Cardinal  so  Al  if  not  then  this
is  the  just  the  just  Kappa  filtered  Co
limits  is  just  filtered  Co  limits  but  if
you  make  Kappa  bigger  it's  a  it's  a
weaker  condition  that  it  be  closed  under
Kappa  filtered  Co  limits  um  so  but  uh
then  then  uh  D  is
presentable  uh  and  there  exists  a  left
ad
joint  okay  so  this  is  nice  inclusion
okay  the  was  I'm  not  about  the  Der  the
categ
and  the  infinity  category
version  uh  is  proved
by  by
ronov
schlunk  right  uh  let  me  make  sure  I  got
the  name  completely  correct  uh  yeah
ragimov  I'm  so  sorry  ragimov  I'm  so
sorry  ragimov  schlank  yeah  but  you  have
to  there  was  some  circular  thing  I  see
in  one  of  the  discussions  I  don't
remember  exactly  that  if  you  want  the
reservation  under  limit  that  is  let  say
infinite  let  me  make  another  remark
concerning  a  technical  point  that  came
up  in  the  last  talk  so  recall  that  we  we
were  discussing  the  derived  analog  of
this  and  in  this  context  of  this
definition  we  made  the  definition  uh
that  we  Define  the  derived  category  of  r
as  a  full  subcategory  of  the  derived
category  of  you  know  mod  R  triangle
let's  see  let's  just  write  derive
category  of  our
triangle  um  consists  of
those  m  in  here  such  that  uh  on  homology
you  lie  in  this  ailan  category  mod  R  for
all  I  and  we  wanted  to  show  that  this
derived  category  satisfies  the  analoges
of  all  of  these  conditions  in  particular
these  closure  properties  under  limits
and  colimits
and  for  proving  limits  the  the  thing  to
prove  is  to  prove  closure  under  products
and  there  was  a  sticking  point  that  came
up  that  we  needed  uh  that  if  uh  say  m
alpha  alpha  and  a  is  a  collection  of
elements  of
objects  in  mod
R  we  needed
that  um  if  you  take  if  you  view  them  now
in  the  derived  category  concentrated  in
degree  Z  and  you  take  the  product  in  the
derived  category  then  it  should  satisfy
this  condition  and  the  subtlety  is  that
infinite  products  countable  infinite
products  are  exact  in  our  setting  of
light  condensed  ailan  groups  but
arbitrary  infinite  products  are  not  so
the  product  functor  has  right  derived
functors  and  you  need  that  this  lies  in
mod  R  for  all
I
um  okay  but  but  this  can  be  proved  as
follows  uh  from  the  axioms  but  consider
but  note  uh  so  but  Set  uh  M  to  be  the
direct  sum  over  Alpha  and  a  of  m
alpha  then  this  guy  is  a
retract
of  r  i  product  Alpha  in  a
m  because  termwise  it's  obviously  you
know  this  this  system  is  a  retract  of  of
this  system  here  um  but  this  is  just  the
same  thing  as  X  internal  X
I  from  direct  sum  of
copies  of  our  triangle  to
M  ah  okay
okay  and  what  about  product  of  infinite
complexes  unbounded  below  complexes  yeah
the  argument  given  in  the  last  lecture  I
mean  this  this  was  treated  in  the  last
lecture  ah  because  there  are  I  mean  kind
of  it's  POS  liit  so  that  you  can  reduce
to  countable  do  this  one
exactly  okay  so  that's  that's  and  I
think  also  that  I  don't  know  maybe  you
know  the  refence  so  if  you  have  in
general  Gren  the  category  and  Gren  the
subcategories  that  closed
under  called  close  under  coal  and  filter
and  direct  direct  but  then  in  the  derive
category  when  you  consider  object
chology  orology  your  lies  in  the
subcategories  and  this  is  also
present  this  is  this  I  think  can  be  sh  I
don't  I  don't  know  the  the  reference  for
this  I  don't  know  yeah  okay  but  this  is
just  an  elementary  thing  and  you  can  can
do  it  the  of  complexes  and  I  think  this
can  be  used  to  to  give  a  less  I  mean
once  you  know  this  which  actually  one
can  formulate  it  in  terms  of  every
complex  kind  of  a  limit  of  small  on  with
some
bound  then  then  this  can  slide  this  well
yeah  I  don't  think  yeah  to  to  this
without  using  the  I  don't  think  it's
necessary  to  use  this  I  mean  actually
you  can  kind  of  explicitly  construct  the
left  ad  joint  on  the  drive  level  just  by
taking  left  Drive  functors  of  the  um  the
left  joint  you  have  on  the  ailion  level
well  maybe  it  doesn't  quite  work  like
that  but  you  can  well  yeah  I  I  I'm
willing  to  I'm  very  much  willing  to
believe  that  the  infinity  categor
version  can
be  uh  can  be  um  avoided  anyway
let's  you  can  first  the  actually
presentable  because  you're  just  that  all
the
hi  right  right  right  right  right  yeah
category  yeah  that's  that's  a  good
argument  yeah  because  this  is  not
necessarily  the  DED  category  of  of  mod
are  but  what  Peter  said  was  that  um
there's  a  general  principle  about  like
presentable  categories  being  closed
under  limits  and  the  category  of
categories  as  long  as  you  have  functors
that  commute  with  co-  limits  and  the
homology  functors  commute  with  co-
limits  so  the  then  you  can  see  that  this
thing  has  to  be  presentable  and  then  the
infinity  categorical  adjoint  function
theorem  the  more  naive  version  proved  by
lri  um  would  give  you  the  left  ad  joint
okay  anyway  enough  of  that  right  uh
let's  move  on  to  some  real  math  um
okay
so  and  when  you  did  before  without  light
you  did  everything  so  is  there  a  close
relation  between  the  Notions  in  the
light  and  the  general  set  that  is  so  a
condensed  ring  in  the  light  gives  one  in
the  general  and  subcategory  subcategory
and  so  on  maybe  or  it  is  more  Twisted
well  the  the  first  statement  is  is
completely  accurate  so  light  condensed
Rings  embed  fully  Faithfully  into  all
condensed  rings  but  when  it  comes  to  the
analytic  ring  structure  it's  a  little
bit  more  it's  a  little  bit  more
subtle
um  yeah  because  you  a  prior  if  you  have
a  light  an  analytic  ring  in  the  light
sense  youve  aior  a  prior  only  decided
what  the  free  module  should  be  on  light
condensed  sets
um  okay  so  uh  we  also  in  the  previous
lecture  we  also  um  discussed  some
examples  of  analytic  ring  structures  in
the  solid  context  so  let's  say  we  have  a
a  solid  uh  solid  ring  so  I  use  this
notation
uh  so  it's  a  condensed  ring  whose
underlying  um  condensed  aan  group
happens  to  be  solid  um  and  then  we  had
this  uh  subset  of  power  bounded
elements  um  and  then  for
any  subset  r+  contained  in  the  power
bounded  elements  and  maybe  I'll  say  that
one  lies  in  the  subset  um  we  get  an
analytic  ring
structure
um  on
R  uh
denoted
uh  like  this  our  triangle  R  plus  solid
and  um  well  as  the  as  the  notation
indicates  the  underlying  condensed  ring
is  just  our  ring  our  triangle  and  then
the  category  of
modules
uh
um  uh  such  that  if  you  look  at  the  this
space  of  null  sequences  in  M  um  then  you
have  an  endomorphism  from  this  to  itself
given  by  1  minus  shift  time  F  this
should  be  an  isomorphism  for  all  uh  F  in
R  uh
Plus
and  let  me  make  a  remark  uh
so  so  the  the  condition  that  one  is  an
r+  ensures
that  uh  every  every  m  in  mod  R  is
actually  solid  is  in  solid
Z  that  was  actually  so  if  you  plug  in
FAL  1  that  was  our  definition  of  solid  z
um  and  then  uh  this  condition  that  r+  is
contained  in  r0o  that's  what  gives  that
uh  the  ring  itself  is  complete  so  to
speak  so  I  recall  that  the
interpretation  of  this  uh  second
coordinate  in  the  analytic  ring
structure  was  that  it's  specifying  some
notion  of  complete  modules  adapted  to
whatever  situation  and  you  kind  of  want
the  ring  itself  to  be  complete  otherwise
you  would  complete  it  um  and  that's  a  so
that's  the  condition  on  power  bounded
Elements  by  definition  just  translates
to  saying  that  the  the  underlying  ring
itself  should  be  complete  which  was  one
of  our  required
axioms  um  and  another  remark  is  that
there's  no  loss  of
generality
uh  that  uh
r+  uh  is  an  integrally  closed
subring
uh  with  uh  containing  all  the
topologically  nil  potent
elements
um  because  we  saw  that
um  well  that  if  you  have  a  a  solid
analytic  ring  structure  on  a  solid  ring
so  meaning  a  analytic  ring  structure
such  that  every  module  over  it  is  solid
uh  as  an  underlying  aelan  group  then  the
collection  of  f  for  which  this  is  an
isomorphism  for  all  M  uh  is  actually  an
integrally  closed  subing  containing  all
the  topologically  nil  potent  elements  so
you  could  always  just  throw  in  this
these  guys  take  the  subing  generated  by
them  and  take  the  integral  closure  and
that  will  not  change  the
theory
um
okay  and  then  then  there  was  an  example
uh  so  if
uh  if  this  is  a  Huber
pair  so  recall  that  means  this  is  a  a
Huber
ring  uh  or  let  sorry  s  excuse  me  sorry
I'm  going  to  rephrase  um
if
so  so  it's  a  Huber  ring  um  when  those
things  can  always  be  considered  as  solid
condensed  Rings  uh  just  by  taking  the
associated  condensed  ring  um
then  uh  so  then  the  integrally
closed  sub
Rings
uh  are
r+  uh  like  this  are  the  same  thing
as  as
open  uh  integrally  closed  sub
Rings  uh  like  this  and  these  are
exactly  uh  the  a  the  r  pluses  in  in
hub's
Theory
so  the  general  setup  we  have  for  an
arbitrary  solid  ring  of  the  possible
choices  of  r+  when  you  specialize  to  the
case  of  a  Huber  ring  it  recovers  exactly
the  the  the  choices  of  r+  you  have  in
Huber's  Theory  complete  hu  ring  yes
thank  you  yes  I  let  me  say  that  from  now
on  whenever  I  say  Huber  ring  I'll  mean
complete  Huber  ring  unless  otherwise
specified  so  so  there  is  also  sometimes
people  use  SoCal  derived  comp  complete
things  and  if  you  have  a  derived
complete  thing  it  also  seems  to  give  a  a
condensed  thing  and  then  it's  also  is  it
possible  to  say  that  this  is  also  a
solid  yes  yes  Yes  actually  I  think  I
might  I  was  I  was  considering  talking
about  that  there's  a  fun  story  with  that
that  maybe  I'll  talk  about  it  later  in
the  lecture  yeah  yeah  but  yes  the  answer
is  yes
um  uh  right  uh  and
moreover  in  this  case  the  r+  is  actually
recovered
from  recovered  by  the  analytic
ring  is  it  well  it's  basically
equivalent  to  the  you  know  yeah  so  the
so  hubber  is  r  r+  or  I  should  say  R
triangle  r+  is  exactly  the  same  thing  as
this  solid  analytic  ring
structure  so  in  the  general  case  I
didn't  quite  claim
that  uh  if  you  start  with  an  I  didn't
claim  and  it's  probably  probably  not
true  that  if  you  start  with  an
integrally  closed  subg  satisfying  these
conditions  and  you  form  that  theory
there  might  for  some  other  reasons  also
be  other  things  that  F  that  satisfy  this
property  for  all  your  M  possibly  but  in
the  Huber  case  you  can  show  that  that  no
there  aren't  not  we  did  not  do  it  or  we
did  do  well  maybe  we  didn't  do  it  last
lecture  but  it  was  done  in  a  previous
lecture  so  the  the  argument  is  in  a  is
in  the  the  analytic  PDF
okay  yeah  maybe  the  second  to  last
lecture  I
think
yeah
okay
questions
so  now  we're  going  to  discuss  uh
localization
so  um  so  let  me  make  a  remark  so
localization  so  I'll  start  with  a  remark
which  might  be  a  bit  uh  shocking  at
first  glance  but  it's  actually  trivial
um  so  if  R  is
a  r  is  a  solid  ring
so  and  then  we  have  this
r+  satisfying  these  conditions  and  again
you  can  feel  free  to  assume  it's  an
integrally  closed  subring  containing  the
topologically  no  potent
elements  um  note  that  this
condition  defining  the  analytic  ring
structure  is  just  a  a  condition  that
you're  imposing  for  all  F  in  r+  and  R
plus  was  by  definition  a  subset  of  the
underlying  discret  ring
R  so
so  um  all  the  the  data  that  you're  using
to  define  the  analytic  ring  structure
actually  already  appears  at  the  discret
level  so  uh  then  get  another
pair
uh  just  with  the  same  r+  and  the  power
bounded  elements  in  the  discret  case  are
just  uh  are  just  all  the  elements  so
certainly  it's  still  going  to  be  power
bounded  inside  there  um  and  this
argument  shows  that  uh  or  this
observation  shows  that  if  you  you're
interested  in  solid  modules  over  your
original  RR  plus  um  uh  sorry  RR  plus
solid  well  you  can  take  the  ones  over
where  you  have  a  discret
ring
uh  and  then  that  already  has  all  of  the
uh  the  information  about  the  analytic
ring  structure  and  All  That  Remains  is
to  observe  that
R  will  be  a  commutative  algebra  object
in  here  and  you  just  kind  of  abstractly
take  our  modules  uh  in  this  ailion
category  so  it's  important  to  note  that
since  we're  doing  condensed  modules  even
when  you  have  a  discrete  ring  you  have  a
huge  amount  of  new  modules  besides  the
discret  modules  you  can  consider  and  in
particular  over  this  discrete  condensed
ring  you  have  R  the  honest  condensed
ring  and  the  theory  for  an  arbitrary  R
is  actually  base  changed  from  the
discret  case  in  this  completely  naive
way
okay  um
so  we're  going  to  discuss  localization
so  how  kind  of  uh  how  these  categories
glue  but  it's  actually  going  to  be
sufficient  to  treat  the  discret  case
because  if  you  understand  how  this
category  glues  then  you  could  just  put
the  r  module  structure  on  top  of  that
and  you'll  understand  how  this  category
glues  um  and  I  want  to  stress  from  the
beginning  that  I'm  talking  just  about
one  kind  of  example  of  gluing  I'm  not
claiming  this  is  the  most  General  um  but
it  is  a  it  is  nonetheless  quite  General
um  but  it's  just  a  a  certain  framework
for  gluing  you  can  call  it
um  so  now  uh  let  me  make  an  analogy
uh  so  well  so  we're  going  to  be  in  the
world  of  discret  rings  now  but  uh  so  if
R  is  a  commutative
ring  uh  then  we  have  its  usual  derived
category  of  R  modules  So  This  Is
Us
of  and  I'll  for  EMP  emphasis  I'll  put
discrete  R
modules  um  this  localizes
on  over  uh  the  Z  risky  spectrum  of
R  and  I'll  I'll  say  more  precisely  what
that  means  in  a  second  but  just
uh  and
um  uh  and  what  is  this  Spec  R  well  it's
the  yeah  so  Spec  R
are  the  set  of  prime
ideals  um  and  the  there's  a  a  basis  of
Quasi  compact
opens  closed  under  finite
intersection  these  are  the  uh  the
so-called  distinguished
opens
something  called  principal  principal  uh
principal  opens  well  it  depends  on  I
don't  know  I've  seen  the  principal  in
some  references  but  it  is  not  I'm  not
sure  what  it  what  do  you  prefer  I  don't
know  because  uh  actually  myself  I  didn't
remember  I  remember  sometimes  I  said
basic  open  because  I  did  not  know  but
then  I  saw  in  some  text  books  it  was
principle  but  I  don't  know  how  standard
it  is  are  you  okay  with  distinguished
open
okay  okay  uh  distinguished  opens  so
let's  say
UF  those  are  well  set  of  prime  ideals  P
such  that  f  is  not  in  P  so  F  maps  to
something  nonzero  in  the  residue  field
um  and  um  there's  a  structure
Chief  so  and  it's  its  value  on  this
distinguished  open  uh  is  so  this  ring  R1
over  F  localize  at  F  and  it's  kind  of
important  to  note  that  the  neither  UF
nor  R1  over  F  kind  of  determine  F  right
so  but  they  they  do  determine  each  other
um  uh  and  and  in  fact  uh  this  UF
actually  gets  identified  with
spec  uh  of  R1  over  F  and  this  matches
up
uh  up  distinguished
opens
um  and  then  um  so  if  if  this  is  an
inclusion  of  distinguished  opens  uh  Sor
well  sorry  sorry  sorry  sorry  um  so  if  U
distinguished
open  uh  to  this  we  can  assign  uh  the
derived  category  of  the  value  of  the
structure  sheath  on
U  um  and  if  U  is  if  V  is  contained  in  U
then  you  get  a  base  change
functor  and  then  the  theorem  uh
completely  classical  I  guess  uh  is  that
um  this  she  she  or  this  pre-
sheie  uh  is  a
sheath  no  that's  true
yeah  so  it's  a  sheath  of  infinity
categories
um  so  well  in  in  this  setting  you  also
have  a  sheath  of  a  billion  categories  I
could  have  told  the  same  story  with  the
mod  the
usual  yeah  yeah  yeah  yeah  I  mean  yeah
okay  this  is  but  also  there  is  a
certainty  of  a  sh
of  hyper  cover  versus  check  true  in  both
senses  but  yes  that's  right  so  this  is
proved  in  in  lri  somewhere  or  where  is
it  proved  I  assume  it's  proved  in  L
somewhere  yeah  uh  but  but  people
sometimes  don't  know  EXA  okay  it's  kind
of  in  this  language  I'm  sure  it's  uh
it's  lury  yeah  so  um  I  I'll  explain  the
argument  in  a  second  but  um
right  so  let  me  so  in  yeah  in  this  case
we  also  have  a  sheep  on  the  level  of  a
bilan  categories  and  even  more  as  gabra
says  a  hypers  chief  but  uh  never  mind
that  um  I  just  want  to  warn  you  that  in
the  setting  I'm  about  discuss  just  to
discuss  here  we  we  will  only  get  a
sheath  on  the  level  of  derived
categories  we  won't  get  a  sheath  on  the
level  of  a  billion  categories  and  the
basic  problem  uh  is  one  that  uh  came  up
in  my  my  previous  lecture  so
uh  in  this  situation  the  localization
maps  are  flat  so  O  of  U  going  to  O  of  V
uh  is  a  flat  map  well  it's  just  a
localization  and  so  if  if  you  have  a
derived  statement  then  uh  using  the
flatness  of  localization  it's  not  too
difficult  to  deduce  an  aelan  statement
but  in  the  setting  I'm  about  to  discuss
um  these  localization  Maps  will  not  in
general  be  flat  and  I  kind  of  uh  so  one
example  of  such  a  localization  map  is
going  to  be  this  localization  from  the
apine  line  to  the  closed  unit  disc  which
was  this  t  solidification  or  ZT
solidification  and  I  already  mentioned
that  it's  not  exactly  T  exact  there's  a
discrepancy  by  one  and  that  AB  actually
obstructs  uh  the  um  uh  the  chief
condition  on  the  aelan  level  this
closely  related  statements  so  one
statement  is  that  the  dve  category  of  a
ring  is  the  same  as  the  Quasi  ceran
derived  category  of  spec  of  the  ring  and
I  think  in  without  boundedness  is  proved
in  prob  in  the  tax  project  okay  and  then
there  is  another  statement  about  the
shift  property  which  I  I  also  don't  know
the  reference  but  I  think  that  people
that  it  is  this  line  of  L  see  that  you
have  if  you  have  any  ring  topos  then  you
can  associate  to  every  object  in  the
site  or  the  top  most  the  category  of  uh
of  this  uh  U  of  U  and  this  should  also
be  a  hypers  shet  but  is  it  what  do  is  it
so  this  this  is  in  L  also
or  so  first  of  all  what  I  said  should  be
correct  I  think  is  it  I'm  sorry  I'm  I'm
my  brain  is  a  little  not  working  very
well  right  now  I  actually  I  actually
zoned  out  while  you  were  talking  my
apologies  any  toos  and  you  haveo  any
object  VI  the  right  category  of  the
topos  restricted  to  you  Val  in  the  she
yes  then  this  is  a  hyers  she  hypers
Chief  no  Chief  yeah  hyp  and  also  hypers
well  why  why  would  it  automatically  a
hypers
chief  no  no  I  think  I  was  able  to  to  do
it  in  some  classical  more  classical
formulation  but  it's  it's
a  is  it  so  what  do  you  know  about  this
statement  uh  oh  yeah  the  usual  Drive
category  I  maybe  it  is  maybe  it  is  a
hypers  chief  yeah  okay  I  don't  know  I
don't  know  yeah  you  don't  know  the  ref  I
I  mean  I  don't  know  the  statement  it's  I
guess  now  that  I  think  about  it  it
sounds  plausible  but  I  mean  the  hypers
sheath  but  there's  a  certainly  lury
proves  it's  a  sheath  yeah  and  hypers
shief  maybe  I  I  don't  think  proves  it  in
one  of  his  books  yeah  I'm  sure  yeah
there  is  also  something  that  I  saw  that
you  men  I  mean  in  some  text  that  they
found  on  the  internet  instead  of  looking
at  the  drive  category  you  can  look  at
funs  Infinity  funs  or  shifts  this  virus
in  the  aan  group  on  the  yeah  and  you
claim  that  this  is  the  same  as  the  D
category  of  the  l  oh  hypers  sheaves
hypers  sheaves  yeah  if  you  do  hypers
sheaves  with  values  in  D  of  a  billion
groups  that's  the  same  thing  as  the
derived  category  of  the  category  of
sheaves  of  a  billion  groups
yeah  and  this  is  also  proved  in  in  I
assume  it's  somewhere  in  lurry  yeah  now
uh  I'd  like  to  move  on  so  maybe  we  have
this  discussion  at  another
time  um  okay  so  that's  uh  one  part  of
the  analogy  kind  of
uh
and  the  second  part  is  uh  so  now  we  have
uh  r
r+  a  discrete  Huber
pair  so  that  just  means  R  is  an  ordinary
commutative
ring  and  r+  is  an  integrally  closed
subring
um  well  then  we've  assigned  to  this  uh
this  Dr
rr+
solid  um  and  the  claim  is  that  this
localizes  on  something
else  on  the  valuative  spectrum  of  this
pair  R
r+  um
okay  so  what  is  this
so
so  that  was  the  set  of  prime  ideals  and
the  the  kind  of  purpose  of  a  prime  ideal
in  this  setting  is  to  let  you  know  where
functions  vanish  or  don't  vanish  so  kind
of  you  could  think  of  it  that  way  so
kind  of  a  binary  condition  of
whether  you're  you're  zero  or  nonzero
and  in  the  valuative  spec
you  are  allowed  some  more  refined
information  not  just  information  about
whether  a  given  function  vanishes  or
doesn't  but  given  two  functions  you  can
ask  whether  one  is  is  bigger  than  the
other  uh  um  and  the  the  way  you  can
measure  that  is  by  means  of  evaluation
so  so  it's  a  a  a  function  from  R  to
gamma  Union  zero  so  this  is  an  ailan
group  written  multiplicativity  the
atively  um  and  then  there  are  axioms  so
multiplicativity  vfg  equals  VF  VG  and
there's  the  obvious  rule  about  0  *
anything  equals  z  um  there  is  the  non-
archimedian  condition  VF  plus
G  uh  is  less  than  or  equal  to  maximum  of
VF
VG  ordered  thank  you  thank  you  ordered
but  the  order  is  ver  when  you  pass
multiplicative  yes  yes  yes  yes  yes  but
thank  thank  you  yes  um  uh  V  of0  equals  0
and  V  of  1  equals  1  and  then  we  involve
the  sub  ring  r+  so  we  ask  that  all  of
these  uh  kind  of  be  integral  with
respect  to  the  valuation  um  so  may  VF
for  all  F  in  r+  and  then  there's  an
equivalence  relation  on  valuations
because  you  could  always  for  example
enlarge
the  enlarge  the  order  to  be  in  group  in
some  arbitrary  way  without  really
changing  what's  going  on  and  one  way  to
describe  this  equivalence  relation  is  uh
that  so  uh  so  V  is  equivalent  to  W  if
and  only  if  for  all  f  and  g  and
R  we  have  V  of  f  less  than  or  equal  to  V
of  G  if  and  only  if  W  of  f  less  than  or
equal  to  W  of  G
so
um  so  in  other  words  that  what's  really
important  about  evaluation  is  this
binary  relation  on  functions  which  is
how  you  testing  whether  one  is  bigger
than  the  other  is  it  the  same  in  this
case  is  the  same  as  Spa  in  the  sense  oh
it's  also  Spa  yeah  yeah
yeah  um  yeah  because  the  continuity
condition  is  vacuous  when  the  the  ring
is  discreet
yeah  um  okay  that's  a  bit  of  a  mouthful
if  you've  never  seen  it  before  there's
another  perspective  on  these  things  that
maybe  explains  exactly  in  what  sense
it's  bigger  than  Spec  R  is  um  one  Val
this  is  also  the  same  thing  as  a  uh  just
a
pair  of  a  prime  ideal  and  then  um
for  uh  evaluation  sub  ring  so
evaluation  domain  in  the  residue
field  so  you  have  the  information  which
records  whether  your  function  vanishes
or  not  and  then  you  have  this  extra
information  saying  when
uh  a  function  or  a  fraction  of  functions
uh  where  the  denominator  doesn't  vanish
uh  should  be  less  than  or  equal  to  one
basically  and  that's  the  same  thing
as  uh  as
this  yeah  thank  you  yes  yes  yes  yes  yeah
yeah
uh  oh
you
yeah  um  containing  A+  thank  you  yes
containing  r+  the  IM  plus  yeah  thank  you
yes
plus
uh
right
okay  um  so  the  the  point  here  is  that  we
now  we  have  a  much  bigger  category  and
there's  more  flexibility  for  how  to
localize  and  um  it  connects  with  this
classical  discussion  of  valuations  so  if
you've  never  seen  this  before  then  uh
you  know  you  can  look  for  example  at  uh
I  don't  know  like  the
rational  numbers  or  something  then  maybe
it's  maybe  you  know  the  classification
of  valuations  there's  the  trivial
valuation  um  which  I  guess  corresponds
to  equality  here  for  every  Prime  ideal
you  have  the  trivial  valuation  where
it's  zero  if  your  element  is  zero  and
one  uh  zero  if  your  element  lies  in  the
prime  ideal  and  and  one  otherwise  but
then  also  for  every  prime  prime  P  you
have  a  petic  valuation  so  you  have
generic  point  of  spec  Z  you  have  the
special  points  of  spec  Z  but  then  you
have  you  have  these  things  in  between
which  kind  of  are  nearby  P  but  not  equal
to  P  these  ptic  valuations  um  uh  oh
sorry  I  was  talking  about  Z  not  Q  um  and
then  but  then  the  fact  that  you  can
classify  those  is  kind  of  um  is  a  little
bit  misleading  because  once  you  add  an
extra  variable  then  all  of  a  sudden
uh  things  explode  and  there's  many
different  kinds  of  valuations  basically
because  you  in  a  surface  you  can  have
lots  of  different  kinds  of  Curves
passing  through  a  given  point  and  uh  you
have  valuations  of  so-called  higher  rank
which  introdu  additional  complications
into  the  theory  um  so  I'm  not  going  to
go  too  much  into  into  this  but  yeah  so
I'll  I'll  stick  to  mostly  formal  aspects
for
now  um  okay  so  let's  continue  the  table
of  analogies
so  um  so  there  we  had  Spec  R  and  we  had
this  particularly  nice  basis  for  the
topology  quasi  compact  closed  under
finite  intersections  and  each  of  them
was  also  of  the  same  form  as  the  global
guy  just  for  a  different  input  datum  so
R1  over  F  and  we  have  the  same  thing
here  so  we  have  a
basis  of  Quasi  compact
opens  uh  closed
under  finite
intersection  and  these  are  called  the
rational  opens  in  this
case  and  they're
uh  they  dep  depend  on  the  choice  of  some
elements  in  your  ring  so  you  choose
arbitr
F1  FN  and  G  inside  your  ring  then  you
can  form  this  thing  and  what  is  it  it's
the  set  of  those  valuations  V  uh
satisfying  all  of  these  conditions  such
that  moreover  so  V  of  G  is  non  zero  and
V  of  fi  is  less  than  or  equal  to  V  of  G
for  all
I
so  in  some  sense  it  lives  inside  the
zisy  the  distinguished  open  the  zisy
open  given  by  just  deciding  G  should  be
non  zero  and  then  we  use  this  extra
flexibility  of  we  can  also  impose
inequalities  so  we're  shrinking  this
risky  open  a  little  bit  using  some
inequalities  and  we  still  get  an  open
subset  um  okay
continuing
um  so  uh  so  there's  a  struct  there  there
actually  there's  a  structure  sheath  but
actually  there  are  structure
sheaves
so  uh  on  this  F1  FN  over  G  uh  you  have
one  thing  which  just  takes  the  algebra
risky
localization  um  but  then  you  also  get  a
choice  of  integral
elements  um  and  that  you  get  by  it's
going  to  be  has  it's  going  to  be  a  sub
ring  of  here  and  you  get  it  by  taking  uh
the  integral  elements  you  had  before  or
rather  their  image  in  there  and  then
adjoint  and  then  looking  also  at  these
elements  F1  over  G  FN  over  G  and  then
that  might  not  be  integrally  closed  so
you  take  the  integral  closure
so  basically  you  just  uh  just  look  at
all  of  the  elements  all  of  the  elements
which  the  valuations  in  your  open  subset
think  should  be  less  than  or  equal  to
one  so  you've  kind  of  already  have  it
for  this  by  Fiat  and  you  forced  it  for
these  and  then  the  collection  of  those
things  is  an  integrally  closed  sub  ring
so  um  you  have  it  for  all  of  those
guys  um  and  then  again  you  have  this
nice  recursive  property  that  uh  U  of
fub1  FN  over  G  is  just  the  same  thing  as
the  valuative  spectrum  of  o  u  o  plus  u
and  this  matches  up  rational
opens  and  here  is  another  place  you  can
see  the  kind  of  necessity  of  including
the  data  of  this  r+  in  the  general
Theory  because  if
you  you  I  mean  you  could  have  said  okay
well  I  I  want  a  bigger  space  I'll  leave
out  this  condition  why  should  I  ask  for
it  but  then  you  define  these  rational
open  subsets  and  they  will  no  longer  be
of  the  form  SPV  R  for  some  ring  R
because  you've  forced  certain  elements
these  elements  fi  over  G  to  be  less  than
or  equal  to  one  and  the  abstract  ring  R1
overg  doesn't  remember  that  so  if  you
want  this  condition  that  the  r  opens
themselves  should  be  of  the  same  form  as
the  global  object  then  it's  absolutely
necessary  to  include  this  extra  data  of
r+  from  the
start  so  of  course
this  okay  implicitly  so  this  imp  that
when  Russ  Russ  including  the  other  you
say  that  this  Ms  this  but  what  what  the
statement  matches  up  Russi  what  do  you
mean  there  I  mean  that  the  so  there's  a
a  continuous  map
from  uh  this  to
this  and  uh  it  induces  a  bje  the
pullback  say  induces  a  bje  between
rational  opens  in  here  which  are
contained  in  this  open  subset  and
rational  opens  in
here  okay  no  this  open  Ral  open  rational
open  okay  remember  okay  yeah
okay
okay  um  and  then
uh  the  theorem  uh  ah  well  no  I  should
say  again  okay  so  u  a  rational
open  to  that  we  can  attach  uh  D  of  o  u  o
plus  u
solid  uh  let  me  I'll  put
this  um  and  then  if  U  is  contained  in  V
inclusion  uh  then  we  get  the  pullback
map  well  there's  a  in  fact  there's  a  it
comes  from  a  map  of  analytic
Rings  uh  from
uh  ouu  o  u  plus  o  plus  u  uh  solid  to  o  v
o  plus  v  solid  in  the  sense  of  the
previous  lecture  so  we  have  a  map  of
condensed  Rings  which  is  just  in  this
casee  a  map  of  discrete  Rings  such  that
if  you  have  a  complete  module  here  then
when  you  restrict  scalers  it's  also
complete  here  so  that's  the  kind  of
forgetful  functor  um  and  then  that
always  has  a  left  adjoint  which  is  this
base  change  functor  and  explicitly  you
get  it  by  taking  your  module  here
abstractly  tensoring  up  from  this  ring
to  this  ring  and  then  recomp  completing
in  this  Theory  here  that's  the  bace
change
funter
um
and  then  the  theorem  oh  I've  kind  of  run
out  of  space  but  maybe  I'll  put
it
so  uh  this
pre-  that  one  over  there  uh  is  a
sheath  of  infinity
categories  and  I'll  put  the  warning  that
this  is  not  true  on  a  bilan
level  in
contrast  to
classical  case  these  these  pullback
functors  our  functors  are  not  t
exact  in
general
because  the  pullback  involves  a
solidification  a  t  solidification  which
as  I  said  is  not  a  not  a  flat
operation  does  it  bound  theological
Dimension  uh  yes  it  does  yeah  so  it's  B
it'll  be  bounded  by  uh
n  the  solidification  is  bounded  by  I
mean  the  homology  is  zero  up  yeah
yeah  okay  so  I  think  we'll  take  a  5
minute  break  um  before  I  get  to  the
proofs
proof  is  it  I  complete  or  not  um
probably  not  in  general  but  there's  an
abstract  result  that  if  so  that  if
you're  space  has  finite  CRA  Dimension
then  hyper  completeness  is  automatic
this  is  sometimes
useful  so  if  I  start  with  uh  a  Sol  ring
I  can  take  it  underline  set  which  in  is
discret  to  and  we  mention  it's  the  same
as  modu  versus  thing  and  is  there  Cas  is
it's  actually  poent  of  this  to  algebra
and  um  there  uh  no  I  don't  think  so  so
the  things  these  tend  I  mean  so  for
example  like  I  don't  know  Z  so  we  we  we
showed  the  Z  power  series  T  is  item
potent  over  z  polinomial  t  but  I  mean
this  this  Des  this  discret  ring  is  is
going  to  be  way
too  big  I  think  yeah  so  I  mean  like  this
is  not  going  to  be  item  potent  there's
no  extra  reason  why  this  should  be  I
mean  I  didn't  think  about  it  carefully
but  I  would  assume  the  answer  is
no  what's
that  right  right  right  so  let's  um  so  so
I've  I've  stated  the  theorem  um  and  now
I  want  to  uh  explain  the  proof
um  but  uh  to  motivate  it  I'll  give  a  a
certain  proof  of  this  classical  theorem
here  and  there  so  there's  many  actually
in  the  this  classical  case  there's  many
different  possible  arguments  um  for  this
um  especially  because  these
localizations  are  flat  there's  lots  of
flexibility  in  how  you  set  things  up  um
but  I  want  to  describe  a  particular
argument  for  this  claim  here  which  will
uh  kind  of  translate  over  without  too
much  difficulty  to  to  this  case  here  so
but  maybe  I  I  erase  some  boards  um  I
don't
know
there  was  maybe  one  remark  that  one  can
make  in  both  settings  that  I  forgot  to
make  uh  so  I  said  so  I  defined  a  um  I
Define  this  prief  of  infinity  categories
on  this  rational  opens  okay  not  every
open  subset  is  rational  they're  just  a
basis  for  the  topology  but  there's  this
General  result  and  when  you  have  a  a
basis  for  the  topology  closed  under
finite
intersections  that  condition  being
actually  necessary  in  the  infinity
context  um  then  a  she  on  that  basis
uniquely  extends  to  a  sheath  on  the
whole  space  in  kind  of  the  naive  manner
of  have  an  arbitrary  open  you  take
limits
um  yeah  so  we're  only  describing  this
sheath  of  categories  on  the  rational
opens  but  after  the  fact  you  get  also  a
category  attached  to  an  arbitrary  open
whether  or  not  it's
rational
okay  um  so  proof
of  of  classical  theorem
so
meaning  uh  descent  zisy  descent  you
could  call
it  uh  for  D  of
R  um  so  you  can  start  with  the  so  we're
we're  interested  in  this  uh  you  could
say  this  this  site  of  uh  distinguished
opens  inside  Spec  R  so  we  have
this
all  with  the  open  cover
topology  and  there's  um  and  so  we  we
kind  of  have  an  understanding  in  of  what
it  means  for  to  have  a  cover  um  you  know
the  distinguished  opens  cover  a  ring  if
and  cover  a  spec  of  a  ring  if  and  only
if  the  you  know  if  you're  you're
inverting  some  elements  and  those
elements  should  generate  the  unit  ideal
um  but  you  can  do  a  series  of  reductions
actually  which  will  show  you  that  you
only  need  to  check  the  sheath  condition
for  very  specific  a  very  specific
example  of  such  a  a  situation  so  the
Dilemma  is
that  this  Gro  de
topology  is
generated  by  covers  of  the  following
form  so  you
take  u
specr  a  distinguished
open  you  take  an  element  of  the
structure  sheath  on  you  and  you  form  the
cover  uh  which
is  uh  U  of  F  and  U  of  1us
F  covering
U  so  this  is  a  very  simple  example  of
two  elements  which  generate  the  unit
ideal  uh  inside  this  ring  and  the  claim
is  if  you  want  to  check  something  as  a
sheath  you  only  need  to  check  the  sheath
condition  in  this  one  specific
situation  yeah  this  was  originally  in
quance  proof  of  the  okay  it  doesn't
anyway  it's  easy  but  it  was  there  was
something  of  qu  and  he  proed  the  cell
conjecture
okay  reduces  to  I  mean  you  want  to  prove
that  if  you  have  some  Vector  bundle  on  a
fine  space  over  a  ring  which  is  driven
over  local  ring  then  it  is  extended  from
the  ring  and  it  did  it  by  reducing  to
this  and  it  was  a  bit  trickier  yeah
kin's  a  clever  guy  so  let's  give  the
proof
um  so
well  I  said  you  know  we  know  you  can
describe  algebraically  the  covers  so  if
you  if  you  in  general  the  covers  would
be  described  like  this  you  take  F1  up  in
FN  in  in  O  of  generating  the  unit  ideal
so  such  that  there  exist  X1  xn  in  O  ofu
with  uh  X1  F1  plus  dot  dot  dot  plus  xn
FN  =
1  and  the  general  the  general
cover  is  the  U
of
fi  oh  darn  it  I  can't  believe  I  didn't
think  of  that  will  not  be  okay  because
you  cannot  generate  the  empty  the  empty
set  from  non  empty  things  damn  it  I  I
can't  believe  I  forgot  to  check  those
things  yeah  uh  thank
you  I  should  know  better  by  now  but
um  uh  plus  empty
cover  of  empty
set  okay  checking  the  sheath  condition
there  just  means  you  check  that  the
value  of  your  sheath  on  the  empty  set  is
the  terminal  object  in  the  category  that
is  the  target  of  your  sheath  okay  so
that  usually  can  be  done  without  much
difficulty
um
okay  any  is  there  a  question  or  comment
from
Bon
no  okay  um  but  note  that  uh  this  cover
here  is  uh  but  this  is  but  this
cover  is  refined  by  another
cover  uh  where  you  take  fi  *
XI  so  this  is  a  smaller  distinguished
open
um  and  those  still  generate  the  unit
ideal  because  because  of  the  same
expression  uh  so  so  then  we  can
assume  just  that  F1  plus  dot  dot  dot
plus  FN  is  equal  to
one  um  and  then  you  can  do  an  induction
on  N  so  you  can  then  induct  on
N  so  let's  say  for  example  you  had  uh  F1
+  F2  +  F3  =  1  and  you  want  to  check  the
chief  condition  uh  then  because  you're
assuming  the  sheath  condition  for  covers
of  this  form  you  can
localize  uh  to  uh  you  know  U  of  fub1
plus  F2  and  U  of  F3  and  it's  enough  to
check  the  sheath  condition  when  you  do
these
localizations  but  when  you  localize  to
here  then  fub1  plus  FS2  is  equal  to  a
unit  and  by  dividing  by  the  or
multiplying  by  the  inverse  of  the  unit
you  reduce  to  that  case  there  um  and
then  when  F3  is  a  unit
um  uh  well  there's  something  similar
right
uh  first  there  is  a  statement  about
generation  of  the  topology  ah  sorry  wait
topology  containing  this  and  then  there
is  another  statement  which  is  probably
in  L  that  under  some  conditions  that
it's  enough  to  check  the  shift  condition
on  generators  of  on  coverage  to  generate
topology  and  the  Bas  changes  yes  yes  yes
course  assuming  it  seems  like  the
intersections  exist  yes  exactly  but  the
yeah  I  chose  these  collection  to  be
compatible  under  base  change  so  that  um
yeah
so  I  this  this  collection  here  is  closed
under  base  change  so  that  that  is  an
important  technical  point  when  you  get
into  making  this  argument  completely
precise  but  I  set  it  up  so  that  it's
true  oh  yeah  and  sorry  when  you  restrict
your  cover  to  this  the  cover  is  split  I
mean  the  because  this  was  one  of  your
covering  elements  so  the  the  chief
condition  is  automatic  here  and  the
chief  condition  here  follows  by  by
induction  so  um  that's  that's  basically
the  the  argument  fairly
simple
um  okay  so  but  what  about  this  here
um  okay  so  what  are  we  trying  to  show  so
in  now  so  checking  the  sheath
condition  uh  for  well  I  can  it's  a
distinguished  open  and  an  element  F  but
there's  no  now  now  there's  no  loss  of
generality  in  assuming  U  equals  Spec  R
so
uh
uh
um  so  well  so  what  is  the  sheath
condition  in  this  case  so  in  this  case
we  have  uh  well  we  have  just  two
elements  and  then  we  have  their
intersection  so  the  sheath  condition
says  that  if  you  look  at  the  derived
category  of  R  and  then  the  derived
category  of  R1  over  F  the  derived
category  of  r  one  over  1  minus  f  um  and
then  the  deriv  category  on  their
intersection  which  is  you  invert
both  uh  this  should  be  a  pullback  of
infinity
categories  um  and  now  let  me  make  a
pause  because  I  didn't  spend  much  time
discussing  what  this  notion  means  like
sheath  of  infinity  categories  limit  of
infinity  categories  and  so  on  but  I  can
make  it  completely  so  to  speak
Elementary  in  this  case  of  these
pullbacks  so  you  get  a  feel  for  what
what  the  claim  is
so  yeah  so  claiming  that  this  is  a
pullback  of  infinity  categories  what
does  it  mean  concretely  well  you  have  a
it  means  the  funter  from  D  of  R  to  the
pullback  category
uh  1  over  1  minus  F  uh  should  be  an
equivalence  of  of  categories
so  now  how  how  to  think  of  this  Infinity
category  well  so  you  give  yourself  an
object  in  the  derived  category  here  and
an  object  in  the  DED  category  here  and
then  you  give  yourself  extra  data  of  an
isomorphism  between  them
here  um  and
that's  uh  but  it's  not  an  it's  not  it's
not  an  isomorphism  in  the  usual  derived
category  it's  an  isomorphism  in  some
infinity  version  so  you  can  imagine  for
example  if  this  is  represented  by  a
complex  of  projective  objects  objects
this  is  represented  by  a  complex  of
projective  objects  then  you'd  actually
want  to  give  a  chain  homotopia
equivalence  between  their  images  there
let's  I  let  me  say  they're  bounded  above
just  for  Simplicity  and  then  you  make  an
infinity  category  out  of  that  so  you  def
find  some  notion  of  chain  homotopy  there
and  so  on
um  uh  right  so  then  what  does  uh
exential  surjectivity  mean  it  means  you
can  GL  glue  in  the  derived  category  if
you  have  a  chain  complex  here  a  chain
complex  here  and  an  explicit
identification  between  them  so  maybe  you
choose  some  quasi  isomorphic  models  and
make  a  chain  homotopia  equivalence
between  them  then  that  collection  of
data  uniquely  comes  from  an  element  here
um  up  to  you  know  up  to  quasi
isomorphism  so  the  point  being  that  you
actually  have  to  specify  the  data  of  the
chain  homotopia  equivalence  here  in
order  to  get  the  well-  defined  object
there  that's  the  essential  surjectivity
the  fully  faithfulness  says  something
else  it  says  that  if  you  have  two
objects  here  and  you  want  to  know  the
the  homs  between  them  so  you  can  think
of  calculating  X  groups  for  example  so
the  r  homs  between  two  objects  here  you
can  get  it  by  you  base  change  here  and
you  take  R  homs  you  base  change  here  and
you  take  R  homs  here  and  you  take  R  homs
and  then  you  do  a  homotopy  pullback  of
those  complexes  for  aroms  you  have  there
uh  which  is  the  same  thing  as  like  a
shift  of  a  mapping  cone  of  some  direct
some  of  these  two  mapping  to  that  but  is
it  equivalent  to  like  the  D  of  the
diagram  that  you  can  say  in  the  a
billian  level  a  module  over  diagram  is
just  giving  module  over  every  ring  and
marks  and  then  not  necessarily  without
then  you  take  the  Dory  with  something
posing  some  conditional  mology  maybe  uh
it  sounds  sounds  reasonable  but  I'm  not
entirely  sure
yeah  yeah  that's  certainly  not  how  I
think  about  it  but  okay
um  okay  so  that's  kind  of  how  to  think
about  this  result  it's  it  lets  you  glue
objects  that  are  defined  locally  in  a
drived  sense  sense  but  it  also  lets  you
do  Global  if  you  can  do  Global  X
calculations  by  localizing
um
okay  uh  but  how  do  you  now  how  do  you
formally  prove  such  a  statement  so  note
that  so  the  proof  so
note  each  base  change
fun  has  a  a  right  a
joint
uh  which  is  just  the  forgetful  functor
so  from  dve  category  of  R1  over  F  to
derived  category  of  R  and  then  it  so
that's  for  each  of  the  individual  maps
in  this  diagram  but  then  it  actually
follows  formally  that  this  functor  uh
also  has  a  right  ad  joint
um
uh  doeses
two  and  you  can  explicitly  describe  what
this  right  ad  joint  is  so  it  sends  if
you  have  a  pair  so
MN  Alpha  so  Alpha  is  an  isomorphism  so  m
is  a  module  here  N  is  a  module  here  and
Alpha  is  an  isomorphism  between  their
common  base
changes
uh  then  you  just  take  you  apply  the
right  adjoints  to  each  of  these  objects
and  then  you  take  a  limit  so  you  just
take  M  cross  over  n  with  M1  over  F  which
is  the  same  thing  as  n  1  1us
F  okay  didn't  exactly  Define  those
things  so  it's
not  yeah  you  have  to  open  up  lurry  and
find  the  precise  version  and  but  but  it
works  yeah  um
okay
um  so  the  trick  to  get  used  to  conver  by
two  op  this  is  just  to  have  a  EAS  EAS
diam  yes  yes  yes  yes  yes  in  principle
you  could  you  could  also  do  this
argument  without  doing  the  reduction
that's  correct  but  it's  certainly  easier
to  talk
about  okay  because  it's  fin  many
intersection  just
inter  like
alter  important  fin  maybe  yeah  I  think
uh  I  think  in  the  um  once  you  get  to
this  the  the  statement  we're  trying  to
prove  with  the  valuative  Spectrum  then
you  really  probably  don't  want  to  um
well  I  don't  know  maybe  you  could
probably  organize  it  cleverly  but  I
think  doing  the  reductions  makes  it  much
easier  so
um  okay  right  oh  so  so  and  this  is  good
news  I  mean  this  is  the  great  thing
about  proving  something  as  a  Shea  of
categories  it's  like  uh  you  have  a  you
have  an  automatic  candidate  for  the
inverse  it's  some  right  ad  joint  so  you
have  a  functor  you  want  to  prove
his  uh  an  equivalence  you  have  a  right
ad  joint  that  means  you  have  a  unit  and
a  co-unit  you  need  to  check  our
isomorphisms  so  then  you  need  to
check  so  one  of  them  will  be  a  map  in
this  category  and  one  of  them  will  be  a
map  in  this
category
um
so  so  so  for  example  for  the  unit  uh  you
need  that  if  m  is  in  D  of
R  uh  you  need  that  M  uh  M1  over  F  uh  m  1
over  1  -  F  M  1  f  1  -  F  this  is  a
pullback  so  that  would  be  the  claim  that
the  unit  of  the  adjunction  is  a  an
isomorphism
um  and
um  so
this  so  This  follows
from  uh  the  statement  that  if  uh  if  you
have  any  element  let  me  call  it  n  in  D
of  R  uh  such  that  N1  /  f  equal  n  1  over
1  -  FAL  0  uh  then  n  equal
0  so  you  want  to  test  something  as  a
homotopy  pullback  you  can  measure  the
difference  between  this  and  the  homotopy
pullback  by  a  mapping  cone  and  that
object  you  can  check  with  this  um  will
satisfy  this  condition  and  then  you  want
to  conclude  that  the  object  itself  is
zero  um  and  then  well  this
is  well  this  is  for  for  example  now  you
could  reduce  to  the  aelion  level  because
an  object  is  zero  if  and  only  if  its
homology  is  zero  and  these  are
localizations  that  are  flat
so  it's  reduces  to  the  same  claim  on  the
aelan  level  which  is  kind  of  very  easy
to  check
um  and  then  the  that  was  the  that  was
for  the  unit  and  the  co-unit  uh  is
the  it  actually  reduces  to  the  exact
same  claim  again  um  but  what  so  um  so
that's  kind  of  the  proof  but  I  want  to
point  out  what's  uh  what's  formally  used
okay  so  we  use  it  localization  by  F
somehow  and  this  right  adint  commute
that  is  I'm  going  to  say  it  I'm  going  to
say  it
over
um  so
each  uh  base
change  is  a
localization  so  the  right  ad  joint  is
fully
faithful  um  the  localizations  commute
with  each  other
just  a  sec  I'll  say  what  it  mean  I'll
say  what  it  means  it  means  that  if  you
take  something  which  is  local  with
respect  to  say  this  operation  so
something  on  which  one  minus  FX  inverti
and  then  you  invert  F  on  that  it'll
oneus  F  will  still  act
inverti  um  these  are  things  which  are
all  just  obvious  in  this  context  but  I
want  to  point  out  exactly  what's  being
used  in  this  argument  um  and  the  third
condition  was  that  yeah  if  if  M  uh  maps
to  zero  on  each  each  element  of
cover  then  m  equals
z  so  if  you  have  a  pullback  diagram
satisfying  these  uh  three  properties
then  and  then  I  mean  yeah
and  then  um  then  you're  going  to  you're
going  to  automatically  get  this  uh  I
mean  a  limit  I  mean  a  square  like  this
satisfying  these  three  properties  then
you're  automatically  going  to  get  a
pullback
okay
H
okay  so  then  the  proof  of  the  solid
analog  you  use  the  Dr  when  you  invert  F1
minus  f  is  some  of  the  intersection  of
the  two  that  is  those  which
are  yes  yes  yes  yeah  you  want  to  know
that  when  you  yeah  exactly  you  want  to
know  that  when  you  pass  to  right  ad
joints  this  thing  is  just  the
intersection  of  those  two  things  as  well
yeah  maybe  I  should  have  added  that  to
the  list  um  yeah  thanks
yeah
well  well  we  have  to  understand  it  in  a
suable  way  I  guess  you're  right  and  of
course  we  have  to  know  the  language  of  L
to  make  it  precise  yes  so  it's  not
doesn't  wor  maybe
to  well  I  mean  it's  a  the  the  wrer
joints  are  fully  faithful  so  it  really
is  just  kind  of  an  object-wise  condition
you  could  say  just
yeah
yeah  um  Okay  so  so  what's  the  analog  so
so  again  we  have  uh  the  site  of  rational
opens  Uh  u  in  this  Valu  of
spectrum  um  and  the  gr
topology  open
covers  and  then  we  have  a
Lemma  that  topology  is  generated
by  uh  so  the  m  cover  of  the  empty
set  and  for  all  rational
opens  and  all  F  in  o
r  uh  now  we  have  to  take  care  of  two
different  kinds  of  covers  so  uh  we  have
U  uh
F1  uh  and
u1f
cover  cover  u  a  geometry  that  you  have
those  two  yeah  right  so  the  way  you  read
this  is  well  you  read  this  F  less  than
or  equal  to  one  so  F  should  be  integral
and  the  way  you  read  this  is  that  f  is
non  zero  but  even  more  it's  f  is  bigger
than  or  equal  to  one  so  it's  it's  well
away  from  zero  it's  complement  of  the
open  unit  dis  that  um
and  uh  you
so  then  uh  yeah  1  over
f  u  and  u  1  over  1  -
F
so  this  is  again  so  again  this  is  where
f  is  non  zero  and  you  require  F  to  be
bigger  than  or  equal  to  one  and
similarly  here  so  this  is  actually  a
refinement  of  the  zariski  cover  we  had
previously  um  where  you  where  you  just
inverted  F  and  1us  F  and  that  was  a
cover  this  is  a  smaller  a  refinement  of
that  which  still  covers  and  the  last
thing  I  think  one  it's  enough  to  do  it
when  f  is  in  R  plus
okay  I  don't  think  that  will  be  helpful
but  uh  it's  that's  nice  to  know  yeah
it's  like  in  read  geometry  in  Tate's
original  work  where  he  proved  a
Simplicity  theorem  yeah  he  reduces  to  he
didn't  have  rational  domains  but  he  has
this  uh  two  two  types  of  covers  for
which  you  can  prove  a
cyclicity  and  it  turns  out  that  it
generalizes  to  the  a  Cas  yeah  I  mean  I
don't  know  if  um  maybe  in  the  Tate
setting  that  uh  one  can  arrange  that  f
is  no  and  there  f  is  in  you  need  because
you  have  the  you  call  them  vas  and  La
vas  no  not  not  him  maybe  some  other
people  read  it  anyway  then  the  V  is  just
F  in  the  a  z  i  mean  in  that  case  a  z  is
A+  you  know  about  CL  but  I  think  that  in
Uber  that  okay  then  I  also  wrote  some
letters  to  other  people  some  to  explain
okay  um  so  I  will  not  give  the  argument
for  this  it's  it  uses  well  it's  it's
just  it's  a  bit  more  complicated  because
well  the  value  of  spectrum  is  more
complicated  than  spec  but  the  idea  is
basically  well  the  idea  is  somewhat
similar  you  could  say  but  it's  actually
a  somewhat  complicated  argument  so  but
it's  um  Uber  there  is  maybe  a  statement
that  it's  enough  to  have  a  rational  that
is  you  have  a  r  f1n  generating  the  unit
ideal  and  then  you  it's  enough  to  check
for  those  yes  yes  yes  you  can  do  some
little  bit  of  work  to  reduce  to  exactly
exactly  exactly  yes  yes  so  so  Huber  so
so  maybe  I'll  yeah  Sketch  So  based  on
what  gabber  said  so  Huber
shows  uh  every
cover  refined  is  refined  by  one  of  the
following  form  so  take  fub1  FN
generating  the  unit  ideal
so  and  then  you  look  at  U  fub1  up  to  FN
but  then  you  leave  out  fi  and  you  put  it
on  the  bottom  instead  so  and  the
collection  of
these  I  and  I  so  again  it's  a  refinement
of  the  usual  zisy  cover  you  get  when  F1
through  FN  generate  the  unit  ideal  but
you  can  check  just  on  valuations  that
it's  still  it  still  covers  is  the  the
valuative  Spectrum  um  and  then  you  do
some  kind  of  you  do  something  similar  to
what  we  did  previously  you  have  with
this  this  here  you  can  using  these
covers  which  refine  all  those  risky
covers  you  can  assume  one  of  the
elements  is  equal  to  one  anyway  you  keep
playing  and  playing  and  playing  and
eventually  you  get  you  get  the  desired
thing  um
okay  uh  right  so  now  what  are  we  reduced
to  analogous  to  there
so  so  if  R  r+  discrete  hu
pair  and  we  take  F  uh  in
R  um  then  we  need  uh
that
um
uh
oh  all  right  so  there's  two  there  was
two  different  kinds  of  covers  there  was
the  F  over  one  and  1  over  f  um  and  then
um
yeah  um
uh  so  I'll  do  that  one
first  um  sorry  R  and
then
um  okay  so  uh  solid  solid
solid  we  need  that  and  we  need  the  other
one
so
um
uh  do  we  need  that  as
well
okay  um  so  we're  going  to  basically  just
verify  that  all  of  these  conditions  hold
so
so  what  is  this  so  there's  only  there's
only  really  two  types  of  covers  here  if
you  look  at  it  there's  the  cover  UF  over
one  and  then  there's  the  cover  U1  over  F
this  is  of  the  form  U1  over  F  where  f  is
just  one  minus  f  um  so  so  if  F  in
R  so  what  is  this  Dr
r+  uh  goes  to  D  R1  over  uh  sorry  let's
say  r  R  plus  a  joint  F  integrally
closed  um  this  is
just  uh
solidification  uh  t
solidification  uh  for  ZT  goes  to  r  t
goes  to
f
um  so  by
definition  uh  so  these  are  both  these
are  both  analytic  ring  structures  on  the
same  ring  right  and  uh  the  only
difference  is  here  we've  enforced  extra
conditions  of  uh  it's  just  by  definition
so  uh  if  you  have  so  here  we  have  for
all  everything  in  r+  um  you're  solid
with  your  T  solid  with  respect  to  that
variable  um  and  here  uh  well  and  if  you
enforce  this  then  you  also  have  that
condition  not  just  for  r+  but  for  F  but
then  we  said  you  automatically  then  get
it  for  the  integrally  closed  sub  ring
generated  by  those  so  then  you  exactly
get  this  condition  here  so  the  the
category  that  you're  getting  here  is
exactly  this  condition  here  and  the  left
ad  joint  is  just  the  the  T
solidification  um  and  recall  that  this
was  given  by  some
arom
uh
um  okay  that  was  the  first  type  at  least
need  to  invert  F  first  sorry  do  at  least
to  invert  F  and  you  no  I'm  not  doing
this  sorry  I'm  doing  this  one  here  yeah
so  sorry  thank  you  so  this  is  U  of  f
over  one  and  then  there's  the  one  where
U  of  1  over
f  um  so  that  gives  d  r  r+  uh  goes  to  D
R1  /  f  and  then  R  Plus  plus  1  F  integral
closed  um  so  what  is  this  this
is  so  first  invert  t  or  first  invert
F  uh  then  uh  T
solidify  uh  for  uh
ZT  uh  T  goes  to  1  over  F  R1  over
F
so
the  modules  here  are  full  subcategory  of
R1  over  F  modules  which  is  a  full
subcategory  of  R  modules  and  the
condition  is  FX  in  vertily  and  you  have
this  extra  thing  that's  supposed  to  be
solid  um  but  now  let  me  make  a  remark
about  that  second
situation  so  I  claim  that  that  whole
process  inverting  F  and  then  solidifying
with  respect  to  1  /  f  um  so  that  uh  base
change
there
is  also  described  by  just  an
rhom  so  now  now  I'm  going  to  take
ZT  mapping  to  r  with  t  going  to  f  not
one  over  f  um  and  I  take  R  home  over
ZT  um  so  Z  power  series  T  modulo
ZT  shifted  by  minus  one
there
so  IE  so  and  based  on  the  formalism  from
two  lectures  ago  this  is  the
localization  which
kills  uh  the  item  potent  algebra  item
potent
object  uh  Z  power  series  t  uh  in
mod  in
DZ  uh  Z
solid  so  we  had  ZT  modules  and  solid  Z
modules  we  said  that  this  was  item
potent  and  I  also  said  that  when  you
have  an  item  potent  algebra  then  when
when  you  you  can  kill  it  by  just  taking
the  mapping  cone  of  or  hom  homotopy
fiber  of  the  the  unit  map  hitting  that
thing  and  doing  this  Aram
formula  d  z
z  this  is  what  you  call  d  uh  this  was
yeah  this  would  be  the  same  thing  as  D
of  modules  over  a  z  bracket  T  in  solid
z  um  that's  kind  of  this  is  kind  of  the
new  notation  fitting  it  in  the  general
framework  because  before  for  for  Uber
pair  you  wrote  the
a  Dr  R  plus  you  did  not  write  solid  no
no  you  had  a  integral  Clause  ah  okay  you
you
wrot  D  so  let  me  explain  what  the  point
is  here  so  this  localization  was
supposed  to  be  given  by  first  inverting
F  and  then  doing  the
solidification  but  again  this
solidification  yeah  but  the  first  claim
is  that  this  functor  already  inverts  F
so  if  you  have  something  F  tors  and  it's
going  to  be  killed  by  this  and  the
reason  is  you're  killing  this  whole  guy
and  therefore  in  particular  you're
killing  any  module  over  this  guy  but  um
everything  T  torsion  is  a  module  over
over  Z  power  series  T  So  this
automatically  uh  inverts
F  because  anything  T
torsion  is  a  a  Z  power  series  T
module  so  if  you  have  a  solid  ailan
Group  which  is  a  filtered  colmit  of
things  killed  by  powers  of  T  then  it  is
a  ZT  module  that's  just  a  condition  so
you  can  to  check  it  you  can  it's  a
condition  closed  under  limits  and
colimits  so  you  can  reduce  to  checking
for  something  which  is  uniformly  killed
by  some  power  of  T  but  then  it's
obviously  a  Z  power  series  T  module
because  it's  a  module  over  the  the
truncated  power  Series
ring  um  so  then  it  would  be  the  same
thing  to  write  this  formula  where  you
invert  T  but  then  you  if  you  that's  just
after  a  change  of  variables  it's  exactly
the  same  thing  as  t  solidification  as
described  by  the  or  t  inverse
solidification  as  described  by  the
previous  formula  so  inverting  F  and  then
solidifying  one  over  f  is  just  the  same
thing  as  as  doing  this
here
okay
um
so  basically  all  you  need  to  check  now
if  you  look  at  those  conditions  most  of
them  we  already  know  so  it's  a
localization  kind  of  by  construction  the
localizations  commute  that's  because
they're  both  given  by  R  homing  out  of
some  object  and  any  two  functors  R
homing  out  of  an  object  commute  with
each  other  just  because  the  tensor
product  by  a  junction  and  the  tensor
product  being
commutative  um
so  and  then  what  does  this  translate  to
in  terms  of  these  uh  item  potent
algebras  which  determine  these
localization  functors  so  three
translates
to  well  it's  a  different  condition  in  so
so
for  uh
so  um  it's  just  a  a  simp  translates  to  a
simple  condition  on  these  item  potent
algebras  here  so  there's  that  one  and
then  there's  that  one  it's  that  if  you
you  take  Z  power  series  T  and  tensor  it
in  solid  Z
modules  over  ZT  with  Z  laurant  series  T
inverse  uh  you  get
zero  so  if  you  have  something  that  dies
on  ROM  out  of  this  and  dies  on  our  home
out  of  that  then  by  messing  around  you
will  using  this  condition  conclude  that
it  just  has  to  be
zero  um  and  this  is
a
so  it's  kind
of
uh  yeah
so  you  can  use  the  geometric  series  to
to  see  that  this  is
zero  uh  yeah  so  what's  the
interpretation  here  by  the  way  remember
like  uh  so  this  so  this  you  can  think  of
this  as  localize
away  from  the  the  open  unit
dis  and  this  was
localized  uh
to  um  localized  to  the  closed  unit
disc  or  away  from  the  open  unit  disc
centered  at  infinity  and  the  reason
those  two  cover  intuitively  is  because
if  you  take  the  open  unit  disc  and  the
closed  unit  disc  then  uh  or  sorry  if  you
take  the  sorry  if  you  take  the  closed
unit  disc  centered  at  infinity  and  the
closed  unit  disc  centered  at  zero  then
those  Union  is  the  whole  space  but  in
terms  of  the  complements  that's  saying
if  you  take  the  open  unit  dis  and  the
open  unit  dis  at  Infinity  then  they
don't  intersect  and  that's  exactly  a  you
know  this  is  the  algebraic  translation
of  that
fact  um  and  then  similarly  so  for  the
second  kind  of  cover  uh  you  need  that  Z
power  series  t  uh  tensor  over
ZT  uh  Z  power  series  1  minus  t  you  need
this  to  be
zero
um  uh  but  again  this  is  just  Z  power
series  Tu  and  then  1  minus  U  plus
T  and  that's  also  zero  for  the  same
reason  and  again  the  interpretation  is
that  the  open  unit  dis  centered  at  zero
intersect  the  open  unit  disc  centered  at
one  they  don't  intersect  and  that  sounds
strange  until  you  remember  we're  doing
non-  archimedian  geometry  and  then
sounds  reasonable  again
um  yeah
okay  uh  all
right  so  I  want  to  finish  with  just  a
couple  of  remarks  so  so
remark
um  so  there's  the  the
Corollary  so  that  if  R  is  any  solid
ring  and  then  r+  paining  one  I  don't
know  sure  contain
in
um  so  the  uh  so  now  you  have  to  be  a
little  bit  careful  so  you  you  I  could
say  that  Dr  r+  solid
localizes  on  the  ative  spectrum  of  this
discret
ring
um  but  be
careful
namely  so  there  is  something  that  is
formal  which  is  what  I  said  that  so  on
the  global  sections  uh  that  this  thing
is  just  our  modules  in  uh  the  category
we  assigned  to  this  discret  Huber  pair
but  if  you  then  want  to  get  a  sheath  uh
so  then  you  send  you  a  rational  open
you  have  to  send  that  to  modules  our
modules  in  this  uh
D  O  of  U  uh  discret  uh  I  don't  maybe  I
want  to  say  uh  uh  so  o  for  the  discret
ring  okay
um  so  this  recover  the  topology  in  the
in  the  g  mode
nisr  okay  yeah
so  um  so  the  and  in  particular  so  what
is  the  unit  object  in  this
category  so  uh  you  take  you  take  R  um
and  then  you  invert  G  and  then  you
derive
solidify  uh  with  respect
to  uh  fi  all  the  FI  over
G's
so
um  necessarily  when  you  do  this  object
for  a  completely  General  solid  ring  um
you're  going  to  end  up  with  some  derived
phenomena  here
um  and  we'll  I  I  was  hoping  to  get  to  it
today  but  we'll  probably  discuss  exactly
how  that  happens  later  um  and  I  want  to
make  but  I  want  to  also  make  another
remark  uh  which  is  that  if  uh  so  in
fact  uh  this  this  this  this  sheif  so  Dr
r+  actually  lives  over  so  it  localizes
on  this  big  topological  space  for  the
discret  ring  but  it  actually  Lies  Over  a
much  smaller  subset  uh  so  the  closed
subset
uh  so  let's  say  SP  R  star  R  plus  r  you
if  you  remember  the  set  of  topologically
nil  potent  elements  then  here  you  add
the
condition  uh  that  if  F  here  is
topologically  nil
potent  uh  then  you  want  the  valuation  to
be  strictly  less  than
one  uh
so  for  an  individual  F  that's  a  closed
subset  and  then  it's  a  big  intersection
of  such  things  that's  a  closed  subset  of
this  topological  space  and  my  claim  is
just  that  if  you  take  this  sheath  of
categories  and  you  restrict  it  to  the
open  comp  ment  you  just  get  zero  so  it's
really  living  over  this  closed  subset
here
um
uh  on  the  other
hand  so  Huber  uh
studies  Huber
considers  uh
Spa  R
r+  um  which  is  the  continuous
valuations  um  which  are  less  than  or
equal  to  one  on
r+  and  that's  the  same  thing  as  uh  so
sub  the  subset  of  of  this  it's  a  a
potentially  smaller  subset  a  generally
smaller
subset  uh  such  that  satisfying  a
stronger  condition  says  that  if  you're
topologically  nil  potent  uh  then  the
uh  for  all  gamma  in  gamma  there  exists
an  N  in  N  such  that  the  valuation  of  f
to  the  N  is  less  than  gamma  okay  so
there's  some  subtlety  here
uh  that  uh  the  space  that  hubber
localizes  over  is  actually  smaller  than
the  space  that  we  localize  over  um  but
yeah  so  the  the  the  distinction  only  Cur
occurs  for  higher  rank  valuations  but  I
shouldn't  say  only  because  in  in  higher
Dimensions  higher  rank  valuations  are
just  everywhere  so  there's  actually  a
huge  difference
um  but
uh  but  Huber
shows  uh  and  and  and  I  and  I  should  say
that  this  um  yeah  continuing  the  remark
so  this  this  drr  plus  does
not  lie
over  uh  but  does  not  live  over  the
subset
uh  Spa  r  r
plus  r
r+  in  the  same  I  mean  I  I  claimed  here
that  the  sheath  here  becomes  zero  when
you  uh  restrict  to  the  complement  of
this  subset  it's  not  that's  not  what's
happening  here  but  there  does  exist  a
map  a
retraction  huh  what's  up
Peter  a
what  oh  thank  you  thank  you
well  yeah  does  not  live  over  the  subset
but
uh  there  is  a  retraction  like
this  um  which  is  actually  a
quotient  oh  sorry  R  cus  yeah  which  is  a
realizing  so  by  definition  it  was  a
Subspace  but  you  can  actually  realize  it
as  a
quotient
and  then  you  do  get  a  sheath  of
categories  on  spa  and  that's  the  correct
way  to  so  to  speak  get  a  a  sheath  of
categories  on  Huber's  topological  space
and  it's  the  retraction  that's  kind  of
the  good  map  in  the  sense  that  this  is
the  Quasi  compact  map  so  um  yeah  we'll
probably  so  in  in  in  general  you  get
more  flexibility  for  localization  using
this  picture  than  with  Huber's  picture
and  the  kinds  of  extra  things  you  get
are  something  that  we  already  for
example  the  things  we  already  discussed
something  like  this  so-called  functions
on  the  closed  unit  dis  will  arise  from
the  structure  sheath  in  this  general
setting  but  doesn't  arise  from  the
structure  sheath  in  hubber  setting  and
um  you  can  uh  you  can  analyze  these
things  but  I  think  I've  now  said  enough
thank  you  for  your  attention  so  as
farther  remember  in  this  retraction
context  yeah  so  actually  B  of  spectral
spaces  the  inclusion  is  not  spectral  in
general  the  retraction  is  spectal
exactly  over  it  is  such  that  when  you
have  a  shift  on  the  bigger  thing  then
the  direct  image  by  the  retraction  is
the  same  as  the  Restriction  to  the
Subspace  for  an  arbitrary  Chief  no  no  I
I  as  far  as  I
remember  uh  the  ah  yes  yeah  so  let  me
make  the  so  so  it  is  a  nice  situation
where  the  in  in  particular  the  higher
direct  images  so  the  retraction  are  zero
because  of  this  yeah  and  now  you're
dealing  with  shifts  in  this  Infinity  so
probably  the  same  thing  we  Lo  but  I'm
not  yeah  so
um  yeah  so  there's  a  the  basis  of
rational  opens  here  you  can  actually
parameterize  it  by  similar  data  but  with
an  extra  condition  that  these  things
generate  an  open
ideal  and  then  if  you  pull  those
rational  opens  back  here  you  get  exactly
the  corresponding  rational  opens  as
expected  but  if  you  take  a  general
rational  open  here  not  satisfying  that
condition  and  then  restrict  it  no  no  if
it  does  satisfy  that  condition  and  you
restrict  it  you  get  the  correct  thing
but  if  it  doesn't  satisfy  that  condition
and  you  restrict  it  you  get  something
new  which  is  not  necessarily  even  quasi
compact  not  not  a  rational  open  so  you
have  You'  have  to  write  it  as  a  union  of
rational  opens  and  this  is  kind  of  like
yeah  taking  the  open  unit  dis  and
writing  it  as  a  union  of  closed  unit
discs  is  a  typical  example  of  that
phenomenon
so  you  yes  so  you  said  something  about
getting  a  structural  shift  last  time
yeah  can  you  comment  on  this  or  or  will
it  come  later  uh  yeah  I  I  was  hoping  to
get  to  it  again  today  but  I  didn't  so  it
is  the  the  structure  Chief  would  just  be
you  take  R  which  is  living  globally  and
you  apply  the  localization  functor  to  to
get  something  living  in  in  here  instead
and  that's  that's  this  uh  it  gives  you
this  object  here  and  one  can  analyze  it
and  so  on  and  so  forth  yeah  and  it's
also  some  Center  Cate  some  category  is
there  a  way  to  think  of  it  as  a  center
of  Cate  Center  I  don't  know  yeah  oh  no
but  these  are  your  symmetric  monoidal
categories  so  it's  just  it's  just  the
unit  I  mean  uh  the  unit  of  the  symmetric
monoidal  derived  category  of  in  this
sense  yeah  okay  so
this  ah  so  you  claim  that  when  you  take
mode  R  of  this  this  is
actually
a  a  good  thing  which  so  it  is  associated
to  the  in  good  cases  it  is  associated  to
the  rubber  associate  to  the  rational  do
except  that  sometimes  you  have  to  do  yes
so  this  is  I  mean  this  so  this  will  also
correspond  to  an  analytic  ring  but  you
know  in  the  to  say  in  derived  sense  so
you  have  to  the  notion  of  analytic  ring
that  we've  discussed  so  far  you  had  an
ordinary  condensed  ring  in  a  full
subcategory  here  you  need  to  not  just
remember
that  ordinary  derived  you  need  to
remember  some  derived  enhancement  of  it
as  well  but  then  it  is  enough  to  just
remember  the  ordinary  aelan  category  of
modules  over  the  ordinary  thing  yeah  and
and  besides  the  derived  stuff  there's
also  a  quasy  separated  issue  where  the
value  of  the  structure  sheath  might  be
different  from  Hubers  even  if  it  lives
in  degree  zero  it  might  you  know  the
quotient  might  not  be  by  a  closed  idea
and  so  it  might  still  differ  from  Hubers
but  again  in  Practical  cases  that
doesn't  show
up  um  and  yeah  even  I  guess  even
inverting  G  can  introduce  non-  quasy
separated  behavior  in  general  yeah  we'll
I  think  we'll  discuss  this  in  coming
lectures  all  of  these  so  in  the  so
considering  those  two  spaces  in  U  Theory
where  you  have  a  retraction  which  I
think  he  maybe  use  a  slight  different
anotations  but  anyway  this  SPV  AI  so  you
have  this
Subspace  living  in  slightly  bigger  thing
is  a  retraction  which  is  spectral  so  you
have  shifts  you  can  consider  shifts  on
both  things  what  I  said  I  think  is
correct  that  the  direct  that  the
Restriction  to  the  Subspace  is  like  the
direct  image  okay  then  you  have  a  shift
like  in  this  shift  of  categories  and
some  sense  on  the  full  s  and  then  you
take  the  direct  image  yes  to  the
southpace  which  is  like  restriction
something  I  probably  but  the  question
can  also  be  asked
about  so  you  have  particular  shields  on
the  full  scene  which  are  direct  images
by  the  inclusion  of  shields  on  the
Subspace  so  the  the  so  the  question  is
like  in  this  context  so  you  have  your
let  us  say  you  have  a  a  rational  open  in
the  figure  C  and  you  consider  its
intersection  with  the  smaller  s  you  said
that  you  can  write  is  the  union  of  C  so
you  can  evaluate  your  shift  by  In  This
Way  by  inverse  limit  of  those  is  it
equivalent  to  the  shift  to  the  value  on
the  original  thing  in  the  big  in
SPV  so  like  whether  the  sh  of  categories
is  the  direct  image  of  its  restriction
to  the  Subspace  which
so  you  have  on  SPV  r  r  plus  r  z0  you
have  let  us  say  some  stock  let  us  say  of
categories  you  can  you  can  take  his
direct  image  to  which  I  think  is  the
same  is  a  restriction  to  that  okay  yes
and  then  it  is  AAL  Junction  map  for  to
our  lower  star  to  the  let's  say  is  the
inclusion
it  yeah  it's  not  the  same  because  for
example  on  global  sections  you  can  see
something  like  a  ah  well  no  it's  not  the
same  because  um  for  the  category  you
have  a  a  rational  subset  corresponding
to  the  closed  unit  dis  uh  giving  you
this  and  in  this  in  the  category  of
modules  over  this  guys  say  the  unit
is  compact  but  when  you  when  you  go
through  that  procedure  again  analytic
function  is  not  this  R  the  global
analytic  function  is  just  certain  where
you  don't  invert  a  fixed  power  you  just
have  convergence  on  each  yes  so  you  get
another  ring  of  function  this  is
something  else  this  is  another  the  so  it
is  not  so  the  answer  is  not  no  there's
really  more  data  in  the  in  the  in  the  in
the  structure  chief  on  this  guy  of  what
one  can  do  in  usual
exactly  exactly  which
is
okay  okay  so  uh  yeah  maybe  if  there  are
more  questions  I  can  handle  them  on  a
personal  basis  let's  release  people  from
the
room
\end{unfinished}