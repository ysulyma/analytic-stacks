\documentclass[reqno]{amsart}

\usepackage[dvipsnames]{xcolor}
\usepackage{comment}
\usepackage{fullpage}
\usepackage{todonotes}
\usepackage[all,2cell]{xy}
\usepackage{graphicx, amsmath, amssymb, amsthm}
\usepackage{newclude}
\usepackage{mathrsfs}
\usepackage[hidelinks,backref]{hyperref}
\usepackage{enumitem}

%!TEX root = AnalyticStacks.tex

% \prism
\usepackage{relsize}
\usepackage[bbgreekl]{mathbbol}
\usepackage{amsfonts}
\DeclareSymbolFontAlphabet{\mathbb}{AMSb} % to ensure that the meaning of \mathbb does not change
\DeclareSymbolFontAlphabet{\mathbbl}{bbold}
\newcommand{\prism}{{\mathlarger{\mathbbl{\Delta}}}} 
% random bits of notation

\newcommand{\<}{\langle}
\let\>=\undefined
\newcommand{\>}{\rangle}

\newcommand{\citeme}{\textcolor{red}{cite me}}
\newcommand{\lf}{\left\lfloor}
\newcommand{\rf}{\right\rfloor}
\newcommand{\lc}{\left\lceil}
\newcommand{\rc}{\right\rceil}
\newcommand{\psr}[1]{[\![#1]\!]}
\newcommand{\st}{\mid}

\newcommand{\ds}{\displaystyle}

\newcommand{\id}{\mathrm{id}}

\newcommand{\op}{\mathrm{op}}

\newcommand{\defeq}{\mathrel{:=}}
\newcommand{\eqdef}{\mathrel{=:}}
\newcommand{\tops}{\texorpdfstring}

\DeclareMathOperator{\ev}{ev}

\DeclareMathOperator{\res}{res}
\DeclareMathOperator{\tr}{tr}

% fancy diagram shortcuts
\newdir{ >}{{}*!/-10pt/@{>}}
\newcommand{\pullback}{\ar@{}[dr]|<<{\mbox{\Huge$\lrcorner$}}}
\newcommand{\pushout}{\ar@<2pt>@{}[ul]|<{\mbox{\Huge$\ulcorner$}}}

\newcommand{\Tamb}{\mathrm{Tamb}}
\newcommand{\m}{\underline}
\newcommand{\mpi}{\m\pi}

% symbols
\newcommand{\A}{\mathbf A}
\newcommand{\C}{\mathbf C}
\newcommand{\cF}{\mathcal F}
\newcommand{\sF}{\mathscr F}
\newcommand{\F}{\mathbf F}
\newcommand{\G}{\mathbf G}
\renewcommand{\H}{\mathrm H}
\DeclareMathOperator{\K}{K}
\renewcommand{\L}{\mathbf L}
\newcommand{\N}{\mathbf N}
\renewcommand{\O}{\mathcal O}
\newcommand{\cP}{\mathcal P}
\newcommand{\Q}{\mathbf Q}
\newcommand{\R}{\mathbf R}
\let\sec=\S
\renewcommand{\S}{\mathbf S}
\newcommand{\T}{\mathbf T}
\newcommand{\W}{\mathbf W}
\newcommand{\Z}{\mathbf Z}
\newcommand{\bdry}{\partial}

\newcommand{\THH}{\operatorname{THH}}
\newcommand{\TF}{\operatorname{TF}}
\newcommand{\TPhi}{\operatorname{T\Phi}}
\newcommand{\TCmin}{\operatorname{TC}^-}
\newcommand{\TP}{\operatorname{TP}}
\newcommand{\TR}{\operatorname{TR}}
\newcommand{\RO}{\operatorname{RO}}
\newcommand{\RU}{\operatorname{RU}}

\newcommand{\RP}{\mathbf{RP}}
\newcommand{\CP}{\mathbf{CP}}

\newcommand{\Fil}{\mathsf{F}}
\newcommand{\Nyg}{\mathcal{N}}

\DeclareMathOperator{\lcm}{lcm}
\DeclareMathOperator{\inj}{inj}
\newcommand{\regconn}[1]{\mathsf{P}_{#1}}
\newcommand{\regslice}[1]{\mathsf{P}^{#1}_{#1}}

% algebra
\DeclareMathOperator{\coker}{coker}

\newcommand{\tnsr}{\otimes}

\DeclareMathOperator{\Hom}{Hom}
\DeclareMathOperator{\End}{End}
\DeclareMathOperator{\Aut}{Aut}
\DeclareMathOperator{\Tor}{Tor}
\DeclareMathOperator{\Ext}{Ext}

\newcommand{\morph}{\mathop{\longrightarrow}\limits}
\newcommand{\xra}{\xrightarrow}

\newcommand{\from}{\mathop{\leftarrow}}

\renewcommand{\lim}{\mathop{\operatorname*{lim}\limits_{\longleftarrow}}\limits}
\newcommand{\colim}{\mathop{\operatorname*{lim}\limits_{\longrightarrow}}\limits}

% arrows
\newcommand{\epi}{\twoheadrightarrow}
\newcommand{\mono}{\hookrightarrow}
\newcommand{\monic}{\mathop{\rightarrowtail}\limits}
\newcommand{\iso}{\overset\sim\longrightarrow}
\newcommand{\eqv}{\simeq}
\newcommand{\isom}{\cong}

\DeclareMathOperator{\AnSpec}{AnSpec}
\DeclareMathOperator{\Spa}{Spa}
\DeclareMathOperator{\Spec}{Spec}
\DeclareMathOperator{\Spf}{Spf}

\DeclareMathOperator{\Fix}{Fix}
\DeclareMathOperator{\Orb}{Orb}

\newcommand{\Ainf}{{\mathbf{A}_{\mathrm{inf}}}}
\newcommand{\BK}{{\mathfrak S}}
\newcommand{\WCart}{\mathrm{WCart}}
\newcommand{\mup}[1]{\ar@/_1em/[u]_-{#1}}
\newcommand{\mdown}[1]{\ar@/_1em/[d]_-{#1}}
\newcommand{\mtamb}{\ar@{-}[d]}

%Theorem Styles
\newtheorem{theorem}{Theorem}[section]
\newtheorem*{theorem*}{Theorem}

\newtheorem{lemma}[theorem]{Lemma}
\newtheorem{corollary}[theorem]{Corollary}
\newtheorem{proposition}[theorem]{Proposition}

\theoremstyle{definition}
\newtheorem{definition}[theorem]{Definition}
\newtheorem{remark}[theorem]{Remark}
\newtheorem{example}[theorem]{Example}
\newtheorem{warning}[theorem]{Warning}

\renewcommand{\~}{\widetilde}
\renewcommand{\^}{\widehat}
\newcommand{\can}{\mathrm{can}}

\newcommand{\note}[1]{\textcolor{purple}{#1}}

\newcommand{\gasopen}{\Z[\hat q^\pm]^\gas}

% categories
\newcommand{\Ab}{\mathrm{Ab}}
\newcommand{\An}{\mathrm{An}}
\newcommand{\AnRing}{\mathrm{AnRing}}
\newcommand{\CRing}{\mathrm{CRing}}
\newcommand{\Cat}{\mathrm{Cat}}
\newcommand{\Cond}{\mathrm{Cond}}
\newcommand{\Set}{\mathrm{Set}}
\newcommand{\Solid}{\mathrm{Solid}}
\newcommand{\Sp}{\mathrm{Sp}}

% stacks
\DeclareMathOperator{\Mod}{Mod}
\DeclareMathOperator{\QCoh}{QCoh}
\newcommand{\Vect}{\text{Vect}}

\newcommand{\Aff}{\text{Aff}}
\newcommand{\Betti}{\text{Betti}}
\newcommand{\an}{\mathrm{an}}
\newcommand{\dR}{\mathrm{dR}}
\newcommand{\fpqc}{\mathrm{fpqc}}
\newcommand{\gas}{\mathrm{gas}}
\newcommand{\gq}{\mathbin{\!/\!\!/\!}}
\newcommand{\liq}{\mathrm{liq}}
\newcommand{\qgas}{\text{$q$-gas}}
\newcommand{\solid}{\blacksquare}
\newcommand{\tri}{\triangleright}

\newcommand{\yt}[2]{}

\newcommand{\Zar}{\text{Zar}}
\newcommand{\et}{\text{\'et}}


% \hypersetup{colorlinks=true}

% make todonotes play nicely with AMS toc
\makeatletter
\providecommand\@dotsep{5}
\renewcommand{\listoftodos}[1][\@todonotes@todolistname]{%
 \@starttoc{tdo}{#1}}
\makeatother

% xy configuration
\SelectTips{cm}{}
\UseAllTwocells
%\CompileMatrices
\NoCompileMatrices

% \includeonly{}

% uncomment to work on just one lecture
% \includeonly{03-light-condensed}

\title{Analytic Stacks}

\begin{document}

\maketitle
\begin{center}
  Lectures by Dustin Clausen and Peter Scholze
\end{center}

\tableofcontents

These are crowd-sourced lecture notes for the Clausen-Scholze course on Analytic Stacks,
\begin{center}
  \url{https://www.youtube.com/playlist?list=PLx5f8IelFRgGmu6gmL-Kf_Rl_6Mm7juZO}
\end{center}

The repository for these notes is
\begin{center}
  \url{https://github.com/ysulyma/analytic-stacks}
\end{center}

These notes are not endorsed by either Clausen or Scholze; any errors are solely the fault of the transcribers. Sections in red were auto-generated from the captions and have not yet been cleaned up by humans.

% !TeX root = ../AnalyticStacks.tex

\section{\ufs Introduction (Clausen)}

\url{https://www.youtube.com/watch?v=YxSZ1mTIpaA&list=PLx5f8IelFRgGmu6gmL-Kf_Rl_6Mm7juZO}
\renewcommand{\yt}[2]{\href{https://www.youtube.com/watch?v=YxSZ1mTIpaA&list=PLx5f8IelFRgGmu6gmL-Kf_Rl_6Mm7juZO&t=#1}{#2}}
\vspace{1em}



\subsection{Motivation}

Thank you everyone for your patience and welcome. This is going to be a course about analytic geometry. The title is "Analytic Stacks", and we're going to be trying to explain the foundations for analytic geometry that we've been trying to set up for the past few years. 

In this first lecture, I want to give an introduction to the first half of the course (because otherwise, I'd be talking about too many concepts in one single lecture). Let me set the stage by giving some \textbf{motivation}.

Classically, there are several different theories of analytic geometry. I'll just list the ones that I may be familiar with. 

\begin{enumerate}
    \item The gold standard is the usual theory of complex analytic spaces. In the smooth case, these are the complex manifolds. So, these are things you get by gluing together open subsets of $\C^n$ along biholomorphisms (maps that locally admit a power series expansion). There's the nonsmooth case as well, where you also locally allow the zero locus of some finite collection of analytic functions on those open subsets.


    \item There's a generalization of this which was presented in Serre's book on Lie groups and Lie algebras \citeme{}, which is the locally analytic manifolds (or generalization of the smooth case at least). Consider a complete normed field $(K, |\cdot|)$ which can be archimedean or non-archimedean. Then you glue open subsets of some $K^d$ along locally analytic isomorphisms. That means we obtain locally around every point you have a convergent power series expansion with coefficients in the field $K$. 
\end{enumerate}
 



In (2) when $K = \C$, it does just recover the smooth case of (1), and that's a nice reasonable theory. When $K = \R$, it's a version of the theory of real manifolds, which is also a nice reasonable theory. 

But when $K$ is non-archimedean, it's not particularly geometrically rich unless you have some extra structure like a group structure or something like this, which gives it more geometry. The reason for this is as follows 



In the non-archimedean case, for example, when $K=\Bbb{Q}_p$, there's the structure is not rich enough because the topology on $\Bbb{Q}_p$ (or $\Bbb{Q}_p^{d}$) is totally disconnected.

\begin{remark}
    Every unit ball will break up into $p$ many other unit balls, and those break up into $p$ many unit balls. For example, in Serre's book \citeme{}, you can find a discussion of the classification of compact $p$-adic manifolds, and it's just a very simple combinatorial classification. There's the dimension and another invariant. There's just really not much going on there.
\end{remark}
    
\begin{enumerate}
    \setcounter{enumi}{2} 
    
    \item  This and some examples coming from uniformization of elliptic curves led John Tate \citeme{} to introduce the \textbf{rigid analytic geometry} which is over a non-archimedean field. So, it's a geometrically rich theory which works in the non-archimedean case. 
\end{enumerate}



In contrast to the theories above, you don't kind of really think of it in terms of specifying some ways of gluing open subsets of $K^{d}$, or you don't even necessarily think about it in topological terms at all. You more actually think about it in algebraic terms. Instead of focusing on a local model, which in the classical case might be something like an open polydisk, you instead concentrate on a class of locally allowed functions. So, there's a turn here. 

\begin{itemize}
    \item Focus on local rings of functions instead of local topology and let the ring of functions tell you what the topology is supposed to be.

    \item The local model that Tate uses are not functions convergent on an open disk, but functions convergent on a closed disk, which is something that makes sense to use in the non-archimedean context. The local rings of functions are well quotients of functions convergent on a closed polydisk, and those are the so-called \textbf{Tate algebras}.
\end{itemize}

The manner in which you're allowed to glue these local models to get global rigid analytic spaces is halfway between algebraic geometry and usual analytic geometry.

%It's not clear if Tate algebra is just the functions convergent on the open disk or quotient of. Oh, okay, it's a point about terminology. Yeah, it's not. Sometimes, in some references, they, but I'm not sure if it's a mistake or another. I mean, because I'm not, because depends on the reference, right? Okay, that's in the two references they say, but all the other one avoid algebra. Okay, let's say it's an apoid algebra then. Yeah, I don't. Okay. 

\begin{enumerate}
    \setcounter{enumi}{3} 

    \item This was generalized by Huber to the theory of \textbf{adic spaces}, and this is a generalization of rigid analytic geometry. Note that rigid analytic geometry takes place over a base field, and a very loose analogy would be that adic spaces are to rigid analytic spaces as general schemes are to varieties over a field. So it is useful generalization where you don't have to have a fixed base field.
\end{enumerate}

\begin{center}
\begin{tabular}{ |c|c| } 
 \hline
 Varities over a field $k$ & Schemes \\ 
 Rigid analytic spaces & Adic Spaces \\ 

 \hline
\end{tabular}
\end{center}


Again, you don't think necessarily of your local models topologically, kind of think of them as being algebraically specified in terms of a ring of functions. But here also there's a new twist. 

\begin{itemize}
    \item  You still have local model as some ring of functions called $A$, but you also include extra data of a certain subring $A^+ \subset A $, and what $A^{+}$ does to $A$ is you attach some space of valuations and then $A^+$ will single out those valuations which view this subring $A^+$ as being consisting of integral elements.
\end{itemize}

It's actually a very nice extra flexibility you have in Huber's theory that you can consider different choices of $A^+$ on a given appropriate topological ring $A$. We'll see from a different perspective what this choice of $A^+$ is really doing, uh, later in this lecture.

%So, maybe I'll go over here. So, when you say local ring of functions, yeah, do you mean really a sheaf on that open set? I just, no, I just mean it's something like you have, you specify a certain kind of ring, and then you say that's my functions on my basic geometric model, you know, like for you have to have transition right when you go from one open set and intersect with other open, yeah, so I what I didn't discuss gluing, so that's that you need to say that kind of thing when you discuss gluing. We'll touch on it a little bit later, but I'm actually going to, for this lecture, I'm mostly going to stick to this sort of aoid context where you just look at a single A, but of course, that does need to be discussed and it will be discussed, yeah.


 Then we proceed to Berkovich's theory. The rings of functions you see both in the context of rigid analytic spaces and in the more general context of adic spaces. The basic examples are things like you have a ring and it's complete with respect to a finitely generated ideal and you give it the inverse limit topology where all the quotients are discrete and you are also allowed to invert something, provided that the inversion is inverting everything you completed along.

 So, the most basic example of these things is you can take $\Bbb{Z}[T]^{\wedge}_{p}[1/p]$, the ring of polynomials in one variable, and you can complete it with respect to the $p$-adic topology and then you can invert $p$ and that's the functions on the closed unit disc in the $\mathbb{A}^1_{\Bbb{Q}_p}$. So, we complete along something and then invert it, and these are also examples of $p$-adic Banach rings, and what Berkovich does is he says let's just work with arbitrary Banach rings to start with.
 
\begin{enumerate}
    \setcounter{enumi}{4} 
    \item \textbf{Berkovich Theory}: The local models are given by Banach rings $(R, |.|)$, note while the theories $(3)$ and $(4)$ are confined to the non-archimedean case, this theory over here actually allows archimedean phenomena as well because the real numbers and the complex numbers also count as Banach rings.
\end{enumerate}

The global theory here is not quite as smoothly functioning as in the case of $(3)$ and $(4)$. To the ring $(R, |.|)$, Berkovich attaches the space of multiplicative seminorms $\mathcal{M}(R, |.|)$, which is some compact Hausdorff space, and  the kind of gluing you're allowed to do is in some sense organized by this Berkovich space and I don't necessarily want to get into too much detail about that right now. But for example, you can see the complex analytic spaces as a special case of this, and you can also see rigid analytic geometry in some sense as a special case of this. So, this is my very, very brief review of classical theories of analytic geometry.


%and please feel free to ask questions if you have questions. Um, some certainties lost over like the shiness in the Uber Theory, correct in has this different gluing like in one you can glue along op get to get to get things like he has to work only over a field condition, then something quite artificial to glue it like you do in but in his language, in some sense it just transports yes to his language with some extra things which are difficult to remember, some conditions under which the theories are equivalent, yes, exactly, exactly like the notion allows you to understand, yes, exactly, exactly, yeah, so yeah, so the relation with original geometry is indeed artificial, it's kind of not the correct, okay, but anyway, um, so what do I want to say now? I want to exp... Okay, so

We have all these theories, that's great, but now what again was the motivation behind coming up with a new theory? Is  just going to be point $(6)$ on the list? Well, not exactly.

\begin{question}
    Why introduce a new theory?
\end{question}

Well, all from $(1) - (5)$ are theories of analytic geometry, and kind of the relationships between them are more or less well known and one can formulate comparisons and the web of these things is kind of well understood, in spite of the subtleties sometimes involved in the comparisons, it's fairly well understood.

But so far, there's no common framework in which you can put all of these examples, they all have their own flavors, and while you can formulate comparisons, it's not that those comparisons are taking place in some larger category that you consider, it's kind of done by hand in every situation, formulating the comparisons between these things.

\begin{answer}
    
        We want to accommodate all examples in a single theory, so that's one thing, um, you'd like a general theory of analytic spaces which can be specialized to whichever context you might uh be interested in.

        
        The second reason would be, out of all the above theories the only rich theory allowing both archimedean and non-archimedean geometry is Berkovich's but the gluing is not so well worked outand in particular, the gluing was investigated by Berkovich he basically restricted to the non-archimedean case almost from the start and then more general gluings were investigated by Piono \citeme{} for example but in that they always do the same thing they fix a Banach ring as the base ring and then they define an affine space of some dimension over that base ring and then they glue along some kind of subsets of a affine space that they that they pick out that's sort of how the gluing works in Berkovich's theory. The Banach rings they take as their base rings often have restrictive hypotheses on them\footnote{so its not really understanding how you can glue two Banach rings together to get some more global object and indeed the things you're gluing along in these a affine spaces are oftentimes not even controlled by Banach rings themselves they're just some other kind of objects so the nature of the gluing is constrained to a finite type situation and it's a little artificial, maybe not necessarily artificial but it doesn't quite fit the mold when you think of how you go from like affine schemes to general schemes for example just by gluing those local models in some naive way.} such as finite type.

        
         Even individually in their own context in which they're supposed to operate uh these theories are less flexible than for example the theory of schemes and one major reason has to do with issues of descent.

\end{answer}

\begin{unfinished}{19:54}

Uh, so for example, one of the main constructions when you have a scheme is the category of quasi-coherent sheaves. Um, and that's a big fancy name, but it's really something simple. When you have a commutative ring, you look at the category of modules over that ring, and then it just glues to a general scheme, and that's what a quasi-coherent sheaf is. Um, but you don't have that in analytic geometry in any of the classical theories, and the reason is, okay, I said you always have some local ring, um, which is describing the local geometry, so to speak, and of course, to that ring, you can certainly assign the category of all R-modules, but then it doesn't glue. So, it's just not the case that if you have, you know, in any of the allowed gluings that people choose, it's just not the case that an R-module on two open subsets or closed subsets or anything that have gluing data on the intersection globalizes to a general R-module. The view point was in analytic that coherent sheaves are the basic tool, yes, this is the classical person, yes, and I was going to say this, um, so you only can glue, uh, so maybe finite type or finally presented modules, and this does give rise to the theory of coherent sheaves, which is a beautiful and extremely useful theory. It's one of the main tools you have in analytic geometry, but it's still constrained by the finiteness hypotheses that come into it. So, for example, if you have a map of analytic spaces, you can't consider like the push-forward of the structure sheath as a coherent sheath, but that should be that's one of the main examples of quasi-coherent sheaths you like to play with in algebraic geometry. I mean, unless the map is finite or something, um, so the theory of coherent sheaths is really nice and it works well in analytic geometry and basically all of these contexts, but it's still not as general and flexible as we're used to from algebraic geometry with quasi-coherent she which have no inherent finesse conditions, um, so, so these are all kind of, you could say, theoretical reasons why we might want to new theory, um, but there's also potentially a practical reason, so this is much more speculative, um, so coming from the Langlands program, so, uh, far Fargan Schulza, they famously geometrized a local Lang lens and that led to kind of a clarification of the local Lang lens program, and what was this geometric geometrization based on, it was based on replacing QP, uh, by some more exotic object, uh, far Fanten curve or really you have to let the curve vary in families in some sense you could say far Fen curves, um, and this was produced in the language of attic spaces so it was attic space over QP not at all of finite type, um, so quite a somewhat exotic Beast which thankfully this theory of attic spaces existed to accommodate it, um, and um, again quite speculatively one might hope that not just the local uh Langlands program but the global Lang L program can also be geometrized, um, very okay Peter's always very optimistic but I'm always uh of global Lang length but this would involve replacing say Q or Z by some family of exotic analytic spaces, um, and whatever such a thing is it's going to have to have both archimedean and non-archimedean aspects, for example, there should also be a version over the real numbers which I believe Peter is is working out, um, and it also is not going to be finite type in any sense so there is simply no existing theory which could possibly give the language to describe such an object if such an object even exists, yeah, but it's good to have a a theory a precise theory to to guide exploration of of the possibility of such exotic things, um, so that's another motivation um, so that is the end of my um motivation section, uh, so now's a good time for questions if people have them, so you you are going to introduce this B SP I mean theory what what is are more general theory encompasses all of this yeah I'm going to we're going to introduce a new theory and explain the relation to the previous theories, yeah, yeah, so that yeah that's good to say yeah so our our goal is in this course is to introduce a new theory of analytic geometry and to explain the relation with the previous theories, yeah, so you will let not only the basic analytic ring but also some analytic spaces analytic yeah yeah but today I'm only going to give an introduction to the apoid situations kind of analytic Rings because it'll already be enough, uh, already be enough there, yes, why is the theory of BL spaces insufficient, oh, it doesn't have any archimedean, it doesn't, there are no, it's non-archimedean by Design, yeah, yeah, Al, some I remember forgot now how it was called some kind of spectrum that they combines and forgot the name of this someone consider but they didn't develop it so uhhuh yeah there could be other theories than the ones I

 listed I just listed the ones that I knew were well studied and yeah, okay, so then let me move on uh, so continuing the introduction um, I actually had a question, yes please, so um, in the previous uh, sort of theories like uh, I think like at the birk ofage setting there were like Rich uh, theories of chology itology and stuff that were Vari that yeah but I mean um yeah sure bur deeply studied atal kology in in setting of burkich spaces yeah okay and okay um so the next section is called condensed math so the I mean the cond condensed math yes uh the the F this this issue here the issue with the scent um it can be attributed to the fact that these in all these different theories these local rings that you have describing the local models they're not just abstract rings they're topological rings and for many purposes for example uh for uh well yeah so the local models classically are in fact topological rings and it's important to remember the topology so for example okay an algebraic geometry the uh polinomial ring in two variables which is functions on Aline and two space is the tensor product of polinomial ring in one variable and polinomial ring in one variable okay but in say rigid analytic geometry if you take the T algebra in dimension one and tensor it with a tate algebra in dimension one again you're going to get some crazy thing because you took an algebraic tensor product and you forgot the ptic topology but if you do a a Pally complete tensor product you get the ring of functions the correct geometric ring of functions the two variable case so you need to in performing constructions such as tensor products which are basically calculating fiber products geometrically you need to remember the topology that much is clear um so and that's at the basis for the reason why you don't have a naive theory of quasi-coherent sheaves and you have

Quasi-coherent sheaves and you have naive problems with gluing outside the finite type case. I mean, finally generated, uh, case, but topological rings, uh, and topological modules over them, which would be the kind of natural thing to do if you're thinking about quasi-coherent sheaves, are not suitable, uh, for a general theory, and the basic reason there's, there's many ways of saying it, but the basic reason is that, uh, you know, if you form this category, it's, it's not going to be a billion, the category of topological modules over a topological ring, and you can dress it up however you like, you can make it much more specific or whatever, it's just not going to be a billion, and the phenomenon is that if you have a dense inclusion of modules which happens all the time when you have infinite-dimensional things, uh, then it's going to be both an epimorphism and a monomorphism, uh, generally speaking, uh, in your reasonable categories, but it's not going to be an isomorphism, so you separated, yeah, yeah, I would have to say separated to make that literally a true claim, so you have a non-strict map, then the image is two topologies, and this is not yeah, CEG world, yes, yes, yes, yes, um, right, so, uh, so what we do is we kind of go very back to the start, um, so we, uh, go back to basics and Define a replacement for the category of topological spaces, and we do that in such a way that it's then very easy to pile algebraic structure on top of those things and talk about the analogues of topological rings and topological modules over them and that such that we will get an aelion category, uh, in the end, um, and the basic idea is one that is very old and I'm not sure, so certainly it was, uh, certainly, you know, Gro deck used this idea many times and I don't but I don't, I think it might even be older than Gr, topological space is kind of funny because you have second-order data, you have a set of points and then you have a set of subsets of that, and that's fundamentally what makes it difficult to mix with uh, algebraic structures, so instead you'd want to stick with kind of first-order data just points and the basic idea is you single out uh, a collection of nice let's say test spaces, uh, s, and then instead of uh encoding a topological Space X as traditionally so a set and some set of open subsets um, we just record the data uh of what should be continuous Maps uh, from your test space to that topological space, and kind of atati the structure and properties you see in that situation so we were only we we're going to choose a nice collection of test spaces and then we're going to say we only we only care about a topological space in so far as uh, the phenomena are seen by maps from these nice test objects, um, so you still have X set I put quotation marks around this so so I I'll become a little more formal in a second and then you can ask your question, no, but traditionally like they wanted to doop Theory with then so one way was to have ACC a set and to fix a collection of S to X some people some work with this yes, yeah, yeah, exactly substitute usual on you can Define homotopy and ology and do something yes EX exactly yes, yeah, um, so yeah, maybe spanner is a person and yeah, I don't know, yeah, um, so formally, uh, so formally, uh, our test spaces, uh, will be profinite sets so-called profinite sets, uh, which is which is the same thing as a totally disconnected uh, compact house door spaces, um, it's also just the same thing as inverse limits of finite sets where the finite sets have a discrete topology and the inverse limit has the inverse limit topology, um, and so this is what we we uh used in a previous iteration of this kind of course, um, so the first course Peter taught on condensed math was uh with this class of test spaces but it does cause some troubles because uh uh, it's a it's a large category so there's no if there's no Cardinal bound on the profinite sets then when you encode all of this data you're encoding more than a set's worth of data and it it does cause some technical troubles we more or less worked around them but so we're actually going to take a a slight variant of this for the purposes of this course um and we'll explain in more detail in the first few lectures I think um why we make this precise choice but so so a we'll say a light profinite set uh is a accountable inverse limit of finite sets it's also the same thing as requiring it to be metrizable um and then so uh a light condensed set uh is a sheath of sets on the category of light uh, profinite sets uh with respect to the groi topology uh, and now I'll explain the groi topology so so it covers our finite collections uh uh of jointly surjective Maps continuous Maps yeah um um so finite dint unions are covers and a surjection gives a cover so this is like theology because you could add you can add also okay a covering seeve is one which contains a finite collection of jointly surjective Maps Okay um so okay this is we're using the language of gr toies here uh to be sure and possibly not everyone is familiar with uh the language of Gro to and kind of the general how you play with categories of sheaves and so on let me make it more explicit uh so more explicitly uh a light condensed set is a functor okay I'm not going to avoid the language of categories and functors uh so light profinite set up to the category of sets uh such that so the first thing is is that well X of empty set equals Point second thing is that X of a disjoint Union of two things is the product uh and the third thing is that if you have a surjection t to S then uh XS is The Equalizer of uh XT and then the two different pullback Maps you have to the fiber product um and and the example example is any topological Space X gives a condensed set where the funter of points is just given by The Continuous maps from s to X so it's easy to see well the the functor

iality is just that you if you have a map from T to S and a map from s to X you get a map from T to X and it's easy to verify all of these properties uh for continuous Maps out to an arbitrary topological space so groi topology I mean you can just unwind it to this but um actually it's kind of a bit of a bit elusory the elementary nature of this definition.

Because um we are going to fairly seriously make use of the theory of Gro IND topology and sheaves in this course, it's just not worth it to avoid that theory, especially since it comes up both here in the definition of light condensed set and also later in the way in which you GL glue uh the apine case of analytic spaces to General analytic spaces. Um, it's not going to be we're just going to use the theory, so if you're not familiar with the theory of Gro deque topologies and sheaves I suggest and you want to follow this course I suggest you read up on it, um, okay, yes, question, so what is a morphism between two uh, two profile sets like Prof set, yeah, just a continuous map, require some filtration it's just a continuous map but it's also a map of pro systems if you're thinking of it as a accountable inverse limit it's equivalent proem it's yeah, yeah, it's the same same thing yes, do we know what are the points in the category of FL content sets points in the c ah oh oh in the topos thetic sense yeah, I think so yeah they gean um debatable so we we'll discuss more such things in the in the coming lectures um isance of topological space in is that skul no not not for General topologic IAL spes but for a large class of topological spaces it is yeah um right so so some there's one thing that's clear right I started with a general idea or you take some collection of test spaces and then you okay then you say what you axiomatize the properties of maps from test spaces to a given topological space and you arrive at this axiomatics but okay it's actually not so simple because there's many possible choices of test spaces and beyond that there's many possible choices of which properties you want to put in your act which properties you see mapping out from those test spaces to x that you put in the axioms uh for your general objects like condensed sets actually there are other properties Satisfied by X of s when X is a topological space that I didn't put in the axiomatics um so there's kind of a little bit of a delicate balance here and we will discuss more about how we arrived at precisely this this Choice both of the test category of light profite sets and this for now I just want to make a couple of remarks which will maybe give a a sense for why we make this precise definition so so first of all uh two maybe the two most important examples of light profinite sets are the point uh and then there's this uh the one point compactification of the natural numbers um and with this you kind of get the underlying set so if you take X of s then you think of that as the underlying set of your condensed set and with this you get kind of a notion of a convergent sequences uh in your your condens set you get a set of conver set of convergent sequences again it's abstract so it's not literally well in the case of a topological space it literally is the set of convergent sequences so also those things were considered in this topology of oh yes yes yes that's correct yeah and but they had this they use this projective things projective covers which are not light that's correct so this will be discussed this will be discussed in in in due time yeah so but yeah so yeah it's true bot and schultza had this definition but that doesn't mean that it was necessarily the correct thing to do for the purposes I'm about to discuss but yeah it turned out it was um okay and then but then um uh so then another remark is that allowing all surjections uh to count as covers gives a nice simplification of the structure of the category and in particular it gives some good homological algebra properties when you pass to light condensed ailan groups um this you have lot of flexibility of working locally when you allow arbitrary surjections to count as covers but on the other hand uh restricting the topology or requiring uh the topology the gro de topology to be finitary uh gives good categorical compactness properties for light profite sets uh sitting by the on meding inside all uh light condensed sets and and moreover and even and even for uh all metrizable compact house D spaces so for example uh the unit interval famously is a has a surjection from the canra set given by decimal expansions uh and this is a light profinite set and the fact that you have a finitary Gro deque topology and on the other hand this this guy is covered by this guy which is one of the basic test objects it means that the compactness of these compact hous door spaces which kind of we know from General topology actually translates into a nice categorical compactness property inside this larger category um and for this it's actually important to have a these larger light profinite sets than just the sets of convergent sequences okay um let's see okay questions Prof at light it's countable no no that's not countable I mean you you could ask the collection of clo and subsets to be countable so count yeah or yeah something like or it's the same as metrizable for sure so yeah no no they adjust every point of the accountable neighborhood B is weaker than the second count yeah some way yeah yeah you need you need a accountable basis for the topology in total yeah so anyway okay yeah it is the same thing as opposite category of the countable yes I guess yeah exactly yeah it's the same saying that the set of clo and subsets is is countable yeah okay further questions um so I remind you that this is this is just an introduction we will go into much more detail in the in the coming lectures okay so so what do we have now we have uh oh maybe I give myself a Blackboard um so what do you have so far when now I've explained the light Prof finite sets and so um we're going to move on to analytic rings and I'll start with a point which is that these okay from now on sorry so I'm going to drop the light okay so just so I don't have to write it and say it all the time from now on condensed set means light condensed set and profite set means light profite set I always have this cabil uh hypothesis floating around um so condensed sets uh gives rise to the notion of condensed ring and condensed module over condensed ring and it's really uh if you're familiar with the gr topologies and so on it's completely immediate so it's just a sheath of rings and then a sheath of module over that sheath of rings on this uh on this site here it's also just a ring object in this category

 and a module object over the ring that ring ring object in this category yeah a question Zoom to write a little bit bigger oh a question to write a little bit bigger that sounds like more of a comment or a request okay uh I will do my best and please hassle me again if I don't live up to it yeah

Um they didn't ask me to write more clearly. Well anyway, Tech, sorry, the formulas more clearly, thanks. Yeah, okay, um. And that's all well and good and you might think naively, okay, so now we have a category of condensed rings, for example. Why can't we just use that as our uh local models for our analytic geometry kind of by analogy with schemes, where schemes are based on discret H. By the way, ring means commutative ring. Um, schemes, you start with discrete rings and then you figure out a way to glue them and then you get schemes. Um, but it's not enough just condensed rings to get a good theory of analytic geometry.

But no, please, cont string is the same object where SE is replaced by a. Yes, that's correct. So, uh, but there's no additional structure on ring, just a abstract ring, right? But condensed ring, it's just an abstract ring with those properties. No, no, but no, no, no, it's a collection of abstract Rings, one for each s, yeah, yeah, okay, right, yeah, but it should satisfy all those properties and Xs cross XT, what does it become, the tensent product of the Rings? No, the cartesian product of the Rings, yes, I see, okay, um, yeah, U where was I H, but are not enough uh to give a good geometry. And the basic reason is as follows, so maybe I should say condens ringings alone. So, the category of condensed rings has pushouts given by relative tensor products, just like in classical commutative Rings, um, and those relative tensor products are what geometrically speaking should calculating fiber products for you, um, and they're the things that I said should correspond to completed tensor products, right, um, but if you have condensed strings, a and b, over a condensed string K, and you form this relative tensor product in this category, you can ask well, what is the underlying ring of this and it turns out it's not actually hard to see from the nature of the grot topology that this is the same as the abstract tensor product of the underlying ring rings of all the individual things. So, this condensed ring here is just, just gives a condensed structure just on the yeah, yeah, just gives a an well non-trivial to be sure but just gives a condensed structure on the on the abstract tensor product, so it's not in particular is not giving a completed tensor product the completion procedure does change the underlying set, right, so this is not not yet doing the correct thing, uh, okay, so to fix this uh, we put additional structure on a condensed string, um, so we record some class of modules, I mean condensed modules, uh, which are to be considered as complete in some sense, complete. So, the basic um, yeah, and that will uh, oh, I forgot to write larger, oh, that would give the notion of analytic ring.

So, an analytic ring will be a condensed ring together with some extra structure which will tell you which of the which of the condensed modules over that condensed ring you should consider as complete with respect to the theory that's being described by the analytic ring, um, but before I make the definition more precise uh, I have to scare more people away. I already said you should know Gro de apologies, get ready, um, so I kind of I have to say more precisely what I mean by a ring, so but I'm going to scare you but then I'm going to say well you shouldn't be too scared, uh, so why is this why is this a question I just I already told you ring means commutative ring right, but um, You didn't say no, I I did you need to pay better attention um, so what kind of ring um, okay so experience in algebraic geometry shows that the generally correct notion of a fiber product of schemes is actually the derived fiber product which on the apine level corresponds to derived relative tensor product of rings. Now the reason more people don't do it that way is because uh, it it's a technical hassle to talk about these things, these derived tensor products and derived rings and so on, but actually that that's not true anymore we have Jacob L's works it's not a technical hassle anymore you just have to you just have to do it and it's no problem so we're going to do it because it's the the correct uh it gives you the correct relative tensor products with giving a good theory in general.

Now that being said basically all of the basic examples that we discuss almost all of the basic examples that come up will not have any derived structure they'll just be ordinary rings so you can comfortably follow the course even if you're not very familiar with derived rings but you should bear in mind that for you know for the for the general claims that we're making it won't necessarily be true if you imagine everything to be an ordinary ring although in examples many things will indeed be Ord Ary Rings um, but now we go down a rabbit hole because once you decide to work with some notion of derived Rings there's actually several inequivalent choices of what you could mean by that um, so so should be derived uh, but which kind there's there's two basic options and that's kind of uh, e infinity algebras and what some people call animated commutative rings. I'm one of those people uh, which

Are the things that are presented by simplicial commutative rings, and then there's also the choice of whether you want it to want to allow negative homotopy. In both cases, um, we won't allow negative homotopy, and we're going to, for the purposes of this course, we'll choose this one here. It's more directly tied to classical algebraic geometry. When you start with ordinary schemes and you take derived tensor products, the things you get always have this extra structure, so it makes sense to remember that and not think about these more general things. But actually, the whole theory that we're developing works perfectly fine in any of the different variants, and in fact, it's even less technical to set up in this seemingly more complicated setting here, for reasons which I think we'll get into. Um, but so algebra in the sense of Spectra exact, there's also that choice. Yeah, you could do you mean infinity algebras over Z or over the spher Spectrum. Yeah, um, so okay, so formally then, so just to get it on the so formally, uh, a light animated ring, oh, sorry, condensed animated ring, is a hypersheaf of animated rings on the uh, site of light profinite sets, but again, I'm not saying light. Um, okay, and now I'm going to make another convention that I'm probably just going to say ring when I mean animated ring, and if I'm want to stress that it actually just lives in degree zero, I'll say, uh, classical maybe or static. Um, static being kind of the opposite of animated, um, right, and that should hope also help those of you who are not familiar with the theory to just pretend that everything is an ordinary ring because that's that's pretty much okay. Um, all right, um, and the basic invariant for us of such a condensed animated ring is its derived category, so the, oh, I need to write bigger of such, uh, such an R is its full derived category. Uh, uh, is this the in L in somewh, yeah, so yeah, you look at just at hypersheaves of modules over this, you know, up unbounded modules over this sheaf of rings, are yeah, justo just a sec, do simpli and coal or does it do it, I don't know, coal no if you want to, you have simplicial modules of simplicial ring, this is not enough to have unbounded in the, no yeah, you don't I mean yeah, you don't really set it up like that, you don't talk about simplicial modules over simplicial ring because then that would be the connective part, you could do that and then just say you kind of formally add in the negative things by by filtered coits or something, I mean by yeah, you do it by I mean the way he does it is he forgets the infinity algebras and then that's just a commutative algebra object in D of Z, and then that has a natural notion of module in the infinity category Theory, so so I mean the theory of modules factors over the underlying infinity algebra and that's just and this is in in which in the in in higher algebra maybe or or S AG probably discussed in more detail spectral algebraic geometry, so but there is no uh, I'm going to say something to help Orient yeah, so if R is static so again that means it's just an ordinary T string uh, then this is the usual or the infinity category enhancement of the usual uh, unbounded derived category uh, of the aelon category of condensed R modules again in the in the totally naive sense of you have a sheaf of rings and you take a sheaf of modules over that sheaf of rings yes, can I ask you to comment on specifically why you want hypers sheaves everywhere it's because we want things like convergence of posnov Tower and we know we can prove that in the world of hypers sheaves but we can't prove that in the world of sheaves so we don't know that sheaves and hypers sheaves are the same thing and also we can always prove everything is hypers Chief and we can always I mean hyper covers never give us more trouble in practice than ordinary covers okay so we're not losing anything by requiring that yes just have a questiony like yes like we can Define schemes in general like just Co limits of representable sheets uhhuh what happens if we do the same here like we take like we glue representable shav over condensed strings no with just condensed strings you're never going to get a good theory you need you need this extra structure yeah I mean if you take sheet over condens strink Sal doesn't I mean the same reason will hold yeah I mean this will be this will be your pullback in pre- sheeps and it's just not the right thing so yeah uh other questions what oh yes hi Matthew yes is not admitted a static animated ring is a condensed anim in the new sense seems like you need some Vanishing of chology because was she condition in just a one categorical s it works I'm not going to get into it right now we'll talk after yeah yeah this is not the time for such a technical question apologies but don't worry if it's correct yeah um okay so uh okay so now I can give the the formal definition um wait did you define are theyre uh yes yes I don't want to say discreet because we also have this condensed stuff and then you discreet could mean yeah so that's the reason for changing the terminology yeah okay so what M no for no no right exactly exactly um okay so the definition is so an analytic ring uh is a pair R uh and then I'm going to use funny notation uh uh where this triangle thing uh is a condensed ring and uh the derived category of the analytic ring is supposed to be a full subcategory of the derived category of this condensed ring which is sort of its envelope um is such that and then we're going to demand some rather strong closure properties remember the idea was this was supposed to be singling out a collection of complete modules um and the first condition is that uh this full subcategory is closed under so inside this ambient category here is closed under all uh limits and colimits uh the second property is that if let's say n Li in here and M lies in here uh then kind of the internal Ram uh from M to n still lies in the smaller thing um the third condition uh is kind of technical so I said we uh we wanted our rings to be connective so no negative homotopy in some sense we also want to require that our analytic Rings be connective and we say it like this so if uh if uh I'd say

This uh denotes the left ad joint to the inclusion. Um, so again that's some kind of completion functor. Then uh this completion uh sends uh the connective subcategory here to the connective subcategory again so it preserves connective objects. I should have said I'm sorry I meant to remind over here so I said if R is static this is the usual ual unbounded derived category of this ailan category. In particular, it has a t structure, has a notion of connective objects and anti-c connective objects but even in general for a general R you still have a t structure but it's not the derived category of its hard anymore. When your ring is not static, say which kind of structure you have in just in terms of the vanishing of of chology, yeah, exactly, and you use homological notation. I always use homological notation. Yes, yes. Do we need object uh H why do we require this? There's some statements uh for some statements it's convenient to have a kind of reduction from a general animated ring to the static case, namely, it's Pi 0 and for this kind of reduction, it's um, it's important to have this kind of control on connectivity. So once you assume the limits and C this is under usual categories or Infinity categories, the same that is well if it's a triangulated subcategory closed under products and direct sums then yeah then to have the int you need. Sometimes some Cal condition is it automatic yes yes so like being generated by a set yeah it's automatic it's automatic yeah Qui one question yes uh with condens ring you mean condens animated ring yeah I I made that convention maybe only in words but yes exactly uhhuh so from now on a ring is a static ring and an animated ring is a ring um okay uh over let's let another one have a chance first uh yes the analog of the ICL inside that indeed it is indeed it is yes uh over no just to make sure so one and two implies that the left agent exists right okay um ah right I forgot to say what a map is so a map of analytic rings sorry I I I I I need to hear what you said so the cond ring is a animated for always anim anim yeah uhh is just a map of condensed rings such that so it's just a condition uh if M lies in D of s uh then the Restriction of scalers of M uh should lie in D of R so so uh along R to S yeah I wrote it a little funny but I hope I hope the meaning is clear so if you have an object in D of s triangle uh which happens to lie in D of s then when you restrict to an our triang our triangle module it should line D of R um okay so I'll make some remarks so uh there's always a t structure on D of R and in fact it's quite naive so the connective part is just uh the intersection of the connective part for the enveloping ring um and same with the anti-c connected part so you can check everything in this potentially more familiar category here so it's actually zg graded not just positively graded the modules are z-graded yes the ring is positively graded that's what I thought okay but the modules you allowed to be zrad indeed indeed um and in particular you get an aelan category so D of R the heart of the T structure which again is just a d of R intersect this um and actually this ailan category also determines the the analytic ring structure so I can also so I can also say that uh D of R is giving an analytic ring structure on our triangle and um there's actually an equivalent axiomatics uh just at the aelan level so I could instead say that I give a a anim a condensed ring and an aelan subcategory of you know the heart of D of r or D of R triangle satisfying certain axioms yes sorry um just maybe kindy of slow it down but for the definition an analytic R yeah I'm just trying to like understand why like uh why in what sense is it analytic like like why yeah so that's something I can't answer right now it'll come when I when we discuss examples and so but the motivation was the simple thing I said that we want relative tensor products to be completed and okay uh yes Robert can I drop hard triangle if I want and just remember the category as a I forgot an axium sorry you just reminded me that because the answer to your question is no because of this axium um sorry I forgot to require that the the unit should be complete so yeah the ring itself should be complete then why can't I drop the Top our triangle entirely well you yeah and then you need some extra structure on D of r i remember as a condensed category yeah yeah that's still not enough because I'm doing the animated context and not the infinity context but if I remember yeah then that's enough yeah yeah yeah yep um okay so that's another perspective is you were just giving an Aon category of complete modules there's also a third perspective which is useful so so to to understand

What an analytic ring structure is, this is very abstract, and it's talking about big categories and stuff. But for a light profite set, I wasn't, I said I wasn't going to say light S. We can consider the free S, the free module. So it's denoted R bracket S. And what is it by definition? You take the free module over the condensed ring, and then you just complete it. So put in there via the left joint. And these generate D of R greater than or equal to zero under co-limits. So those are kind of your basic basic building blocks, your basic generating objects. 

And again, there's another equivalent axiomatics which takes as the second data in the pair, not the category full subcategory or Dr, but just the collection of free modules on profinite sets, objects in D of R triangle. So to gain intuition about what this thing can kind of look like, it's useful to think. So this is not in the heart in general. In general, it's not in the heart. In basically all examples, it is, but in the general theory, it's not. 

So think, so intuition, so this R triangle bracket S is kind of, well, it's just R linear combinations of points in S, kind of completely intuitively, finite R linear combinations of points in S. But you can think of it as a space of R linear combinations of derac measures. Hey Robert, I did something for you. And then, then this RS is some completion. So that's a bigger space of measures. 

So again, an analytic ring structure can also be thought in terms of as specifying some space of measures on a profinite set. And what is the role of this space of measures? So an M, let's say in the heart for simplicity, lies in Dr heart if and only if for all maps F from our triangle S to M of our triangle modules, there exists a unique extension along R bracket S. 

Or in other words, if F is kind of a function from your profinite set to your module, which is some kind of linear algebra object with a topology, and mu is one of these kinds of measures that you're allowing, then we get a well-defined integral of this function along the integral over S of the function. I don't know D mu or whatever you want. 

You can pair them to get a value in the target module. So this is explaining some sense in which this is this behaves like a completeness condition. It's complete enough that you can do non-trivial integrals against certain classes of measures, which you specify as part of the data. You could say yes, is your D the full subcategory? Yes. And is it compa... no, even this one isn't. They're all presentable, but not, yeah, yeah. 

And that's because we did this light restriction, yeah. So we'll have to get into that. Yes, how is the light restriction? Is it Omega one compactly generated? It is, it is Omega one compactly generated. Yes. Is it dualizable? No, I don't think so. Yeah, yes. Well, I wasn't claiming there exists a unique extension. I was claiming this condition is equivalent to this condition. 

So were you saying why if this lies here, does there exist this unique extension? Yeah, so that's because basically just by left ad jointness, but it comes from unraveling the definitions. What is the question? Was which one is is only one comply generated? The big one or the small one? Neither. Oh no, both of them, sorry, both. Both. I marred the Omega one, yeah. Both are, oh, both, yes, okay. 

So I don't think I'm going to have time to get to examples, which is rather unfortunate, but I don't know, maybe Peter will, I don't know, we'll see what Peter plans for Friday. Be nice to talk about some examples. Um, but instead, I think I'll probably finish, well, yeah, I'll probably finish with a discussion of co-limits in the category of analytic rings, and in particular, I want to talk about pushouts because this is the crucial thing which is supposed to give completed tensor products which correspond to geometrically good fiber products. 

So where am I? Here, so your anima structure on... sorry, anity, oh yeah. So one of the things we prove is that it comes for free, okay, yeah. I mean you have, I mean this is a, I mean no make no mistake in the maps. You have a map of animated rings from the r triangle to S triangle, but then it turns out whatever the linear algebra operations you might expect to like symmetric powers and so on will sort of automatically go through. 

Yeah, it's not obvious by any means. So your definition of you back left that, yeah, that's right, that. So that's the most important functor, is the left ad joint to the thing that was in the definition, that's a very good point, that's yeah. So there's some things I'm not mentioning like that D of R is actually symmetric monoidal and this completion functor is a symmetric monoidal functor and these pullbacks, the left ad joints we were talking about, are symmetric monoidal. 

So these are the things that um and preserve. Subat, well by definition it's defined to be the left adjoint to this restricted functor, yeah, or you take the but but is not true that if you do the left joint on the level of the these envelopes that it necessarily preserves the category, you have to complete at the end. 

So again it's kind of a completed tensor product in this base change here, um, thanks for the question, yeah. Okay so so co-limits in analytic rings, um

 so so so um filtered co-limits or more generally sifted co-limits. I'm sorry, oh, the usual notion. So based on finiteness, I mean yes, oh Alf not filtered, yeah. I thought it might be important because you have, I mean indeed one could imagine it might be important. But when I say this then there's no discrepancy, so it's a, I mean there's no ambiguity, so um so if you have filtered co-limit of RI, then the underlying animated R is just the filtered co-limit of the underlying things, um, and also the free modules are are similarly described is just the filtered cimit of the the free modules on the um so that's rather rather naive, um. 

And what's kind of left is pushouts and this is more interesting so pushouts so if again we have maps of analytic rings I'll call them K A and B um I'll write the push out as a relative tensor product just because um then the derived category of the pushout can be more or less immediately described so I'm not talking about the first point in the data but just the second Point um uh so abstractly this category will be the same thing will

Actually, be a full subcategory of. You take the push out in condensed rings, and then it's the full subcategory such that the underlying a triangle module lies in DA and the underlying B triangle module lies in DB.

But, but this caution. A triangle tensor K triangle B triangle D A Tor K B is not an analytic ring. So, it satisfies one through three but not four. So, it's almost an analytic ring. The only thing is that the unit object, the underlying ring, is not complete. But then you fix that by applying a completion procedure to fix this. You can still prove there's a left adjoint to DA Tor K B sitting inside DA triangle tensor K triangle B triangle.

So, I think you're using D in two ways here. Simp, you mean the category of A is a ring. This when you're writing it, it's also it's also part of the definition of anal. That's true, that's true. So, let me make a remark to reconcile this that there's a trivial example of an analytic ring structure on any condensed ring, which is that you take D of A to be equal to all D of A triangle. And with this interpretation, the notations are completely consistent. So, you could call that analytic ring. You could call that analytic ring a triangle. It's just the um. So, every condensed ring can be viewed as an analytic ring with kind of trivial analytic ring structure or maximal analytic ring structure. Everything is complete. And with that in mind, then there's actually no conflict in the a different thing you could have done here, which is take so you have a an antic R. So, it has a subcategory is part the data. You could take DA tensor DP as the categor DK. And that would be a different thing then doing this. Oh no, actually, it's the same. Yeah, it's the same.

Yeah, so um okay. What next? Um, yeah, question. Oh, question. Oh, I'm going to call on the person who raised his hand first. Sor yeah, that one just the same as the tensor product of the categories. Oh, that was already asked and answered. Yeah, over okay, is it, do you change the ring when you apply completion? You change, you change the, because the simpli R is D anyway, you apply the completion in the category D and then you must get another simpli ring. Yes, we have to prove that it's not obvious but yes. So, there's a so yeah, so then when you apply that completion procedure, the category stays the same but then this becomes completed. And also, I should make a remark that in complete generality, it can be rather difficult to understand this completion process. You kind of have to iterate applying the completion for A and the completion for B sandwiching them between each other take account do it countably many times take a co-limit like abstractly that's the formula for this completion procedure here. Now in practice, it turns out you can calculate it and this is one of the points uh in practice I don't think I have time to discuss examples today but this completion procedure which replaces this by the true underlying ring of uh the the the pushout in uh analytic Rings it produces the geometrically correct completed tensor products in analytic geometry.

Um okay so maybe um there's a question from Zoom. Yes, is the condition on the fenus still necessary or did you show that it is always satisfying it's always satisfied in this that's one of the reasons for choosing this light condensed set framework it's actually always satisfied it doesn't matter um so maybe okay maybe I'll start to talk about examples so I risk kind of getting cut off in the middle of an explanation but I feel like it's just too dry without well I won't get very far okay let's do let's try let's try um so I want to talk about uh maybe solid analytic Rings um and this will relate to attic spaces or hu pairs um so uh it's kind of going to be a non-archimedian addition so so I mentioned that if you if you have an analytic ring and we haven't talked about how to produce them yet but if you have one um then it's nice to look at these free modules and profinite sets to get an idea about what what's going on what what spaces of measures do you are you actually looking at here and I also said that the basic example of a profinite set well besides the point was this n Union Infinity classifying convergent sequences so but um in this linear case it's it's natural to consider the following so given an analytic ring R it's natural to consider what you could call I guess space of measures on the natural numbers and I don't mean the free module on this discret thing but what I mean is you take uh the free module on this sequence space and then you mod out by Infinity so this in some sense classifies null sequences in our modules oh I have a backboard up there too and it turns out it's not hard to show that um addition on n induces a ring structure on uh on this MRN um and as a ring it kind of sits in between two rather extreme options uh uh so maybe maybe you want to think of this as t to the N for the purpose of this kind of discussion um so it's sitting somewhere in between the polinomial algebra and the power series algebra over your ring R um as you could imagine for something like a space of null sequences right uh or sequences with some gross growth condition I mean it's really dual to null sequences it's maybe some kind of summability condition um so geometrically speaking we have the apine line and we have some version of the formal neighborhood of the origin and then we have something that sits somewhere in between right um and now I'm going to single out a condition which is kind of a non-archimedian condition that morally speaking will mean that this guy uh lies inside the open unit dis of radius one um but formally so let's say definition uh R is solid uh if um if you if you do this you get zero but since you qu by the constant R Infinity is just the ah okay this is the home yeah that's just R and then but you know and then it's mapping into r n Union Infinity by the inclusion of the point Infinity so R infinity is the free yeah in this the free R module on

 this finite yes view as a as a as a condensed object or no no it's not cond what what is that like solid over Z or uh just a sec I mean this is the i' so far I've just said this definition right so you I mean in this sense but

Just a second.

Okay, so there's an interpretation of this which is well if you have this and it's actually equivalent then you get a measure so to speak. So, $T - 1$ has to, you know, multiplication by $T - 1$ has to kill. So, if you have anything here there has to be a pre-image under multiplication by $T$ minus $1$. So, this means you get some measure here such that $T - 1 \times \mu$ is equal to the unit object in this ring $MRN$, which is kind of the sequence $1, 0, 0, \ldots$. And if you think about what this means, thinking about this measure space sitting between polynomial and power series, this corresponds to kind of sum over $n$, $t$ to the $N$, that's at the very least what it maps to in the formal power series ring. But on the other hand, this measure space, as I said, classifies null sequences.

And you know, so and this measure pairs with a null sequence to if you have a null sequence in an $R$ module $M$ and a measure, I said you can pair the two things to get a function. And the way it works is you take your null sequence, you put it as coefficients here. And yeah, you set $t$ equal to one. So, what this, the interpretation of this is that every null sequence, you kind of have to work it out but the interpretation is that every null sequence is summable, which is kind of classic non-archimedean condition. So, solid is kind of one way of saying non-archimedean in this context but it's kind of fun that geometrically you can think of it as constraining the location of something between zero and the whole real line.

Maybe I will state the theorem. Yes, no. Okay, so theorem, there's a question from chat, the multiplication closed in $MR$, multiplication closed, I mean, it's a ring. I don't know what may I have sort $S$ as you construct it. I guess it's ConEd as some cone it's not the best. Okay, okay, yeah, yeah, yeah, yeah, yeah, yeah, yeah, but in the Universal case $Z$ with the trivial analytic ring structure, it lives in degree zero, there's no, I mean and also that's a summand, it's really but it's still not it's not in the heart it is in the Universal case it is and to produce a ring structure I can work in it's a ring yeah in the Universal case it's a ring and then by base change it's a ring in whatever sense you want in whatever other context I okay, yeah, okay, um, right. 

So theorem, so there exists a solid analytic ring, so it's called "zolid" and well, the underlying condensed ring is just the usual integer $Z$ kind of discret topology. And then well, the derived category is something which I'll discuss in more detail such that an analytic ring is solid if and only if there exists necessarily unique map from zolid to $R$ and moreover, you can actually understand this analytic ring very very explicitly. So, there are some nice results on linear algebra in this basic category.

So, the first thing is that you, well, the first thing you want to ask is what are the free modules on profinite sets. So, let's say $S$ is some countable inverse limit of finite sets then this is just the inverse limit of the free module on the finite set which is just a finite direct sum of copies of $Z$. And also, this is abstractly isomorphic to some countable product of copies of $Z$ countably infinite unless of course $S$ is itself a finite set. What does that say next to the there exist is that say unique? It says necessarily unique, yeah, yes. 

So, the second thing is that, oops. All right, so these, so these. These are remember I said that these guys always generate the category so in this case these products generate the category but moreover these guys here are Compact and projective generators of the well let's say the of the heart um but they live in degree zero so another thing you have is that the derived category here is just the $D$ usual derived category of its heart so it's enough to talk about the bilan category um and also these are flat with respect to the tensor product the completed tensor product uh which I'm kind of mentioned exists um so here's another point where we use the lightness because Sasha fimale proved that this is not hold if you increase the cardinalities on the profinite sets um and moreover you can calculate tensor products rather easily so tensor product of this with this uh over $Z$ solid is just have the infinite distributive law so to speak um that makes for very easy calculation and uh sorry so just asking the the regular is not enough it is really yeah, it's really accountable really yeah, yeah um right so the and the the collection of finitely presented objects in $D$ of $Z$ heart uh which is it generates it under filtered Co limits um is a bilan and closed under extensions and every every finitely presented $M$ has a resolution a free resolution you could say by product of copies of $Z$'s of length at most two meaning a complex where there's three non-trivial terms and two non-trivial maps uh so this kind of gives you a very good hold on calculations in this category very very explicit um so note that you can interpret this sort of as saying that uh so zolid behaves like a regular ring of Dimension two uh so $Z$ is a regular ring of Dimension one we somehow picked up an extra dimmension and that can be attributed to the nonous dorf phenomena that you see in uh in solid ailon groups um but in in all things told you get a very good handle on this category so I think that's the the only example I have time to discuss and thank you for your attention last actually there's only isable there's only one uh right there's a single generator actually this is kind of a general phenomenon because the the free module on the canra set will always will always generate yeah, yes when when when we taking the pushup uh there's a completion fter AI gives you a DED category object why is it that it's a ring yeah that's something you have to prove and

 it's not obvious thank you yeah mhm yes Chad ask question if there will be lecture notes and video recordings video recordings yes lecture notes we're kind of trying to write a book at the same time as we give these lectures and it's not clear to what extent we'll be releasing things sequentially or all at once at some point so in
 Uh, uh, uh. I mean, keep me motivated to. No, I'm sorry. . What can I say? Yeah, I don't know. Or do you like Reman Rock? Of course, I like Reman Rock. Okay then, it proves the most general possible Reman Rock theorems in analytic geometry. Yeah, so he crucially uses these derived categories and the fluidity of the formalism and so on.

Yes, what's the notion of the right-hand triangle? Um, Peter told me Huber uses it. Uh, had to find something. Yes, is there technical advantage to using a light condens set instead of the general condens? I mean for this antic. Yes, advantage, but for just topology. Oh, for general topology, well, I don't, yeah, I mean, there are some more subtle properties that tend only to hold for general topology maybe, not, but once you start talking about topological groups and so on maybe, yeah, yeah, nice. The same question but backward, yes, go ahead.

I just strictly prefer the light setup to the other one, unboundedness and yeah. Is there any case where I would actually want to go back to that one? Well, it's, at the very least, psychologically comforting when you have a strong limit cardinal that you get like compactly generated derived categories and so on. Turns out it's not so important necessarily in practice, but it's kind of maybe quite important psychologically. But then okay, that's just a larger cardinality bound why you would go all the way, it's just to avoid choosing a cardinality bound and so you can say all compact Hausdorff spaces are profinite I mean or condensed sets but there's no real reason necessarily.

What is dist use Infinity algebras? What are those? The structure, can we compare? But I don't, I don't understand the question but what is this term infinite? Oh, e-Infinity, ah, okay, uh, uh, everything works the same except it's a bit easier if you use e-Infinity algebras, yeah, the people who are comfortable with the infinity algebras are not laughing, why, why not, why not even better? Why not E1? I what is the commutativity, oh yeah, so you could, you could do a version of this theory of course with E1 and E2 but but there's something very special about e infinity or animated commutative which is that co-products are the same as relative tensor products that's very nice and moving to realms where that's broken can be a real pain.

Okay, so that's it, thank you.
\end{unfinished}
% !TeX root = AnalyticStacks.tex

\section{\ufs Light condensed sets (Scholze)}

\url{https://www.youtube.com/watch?v=_4G582SIo28&list=PLx5f8IelFRgGmu6gmL-Kf_Rl_6Mm7juZO}
\renewcommand{\yt}[2]{\href{https://www.youtube.com/watch?v=_4G582SIo28&list=PLx5f8IelFRgGmu6gmL-Kf_Rl_6Mm7juZO&t=#1}{#2}}
\vspace{1em}

\begin{unfinished}{0:00}
All right, so welcome back to the second lecture. I guess I should make an announcement about the schedule for the next few weeks, which is a little bit different than usual. Usually, it should be the case that Dustin lectures Wednesdays at PS and Fridays I lecture here. But due to some traveling, I will give the next few lectures all at MPI, but note that there will be a gap of two lectures. So there are no lectures next Friday and the Wednesday after.

So the goal of today's lecture is to recall---partly this will be much repetition of a very similar lecture I gave four years ago, also five years ago. But something I want to stress throughout is why we switch to the Huber setting and which properties you gain when you make this restriction.

Before I go there, let me just say a few introductory words from my perspective about what was said last time. So what's our goal with all of this analytic geometry? For me personally, one goal since 10 years or so has always been to find---there's all this fancy geometry that has been developed $p$-adically, like perfectoid spaces, prismatic cohomology, $p$-adic shtukas, and the geometrization of local Langlands. This all works quite beautifully over the $p$-adic numbers. But it has to use some quite fancy $p$-adic geometry, like these highly non-noetherian perfectoid spaces and so on.

I mean, I'm always hoping that similar ideas or techniques could also be applied not just over $p$-adic local fields, but also real local fields. That's actually something that I think really is possible now. But then also over the whole integers, globally.

So in some sense, what I want is some kind of notion of a global $\Spec(\Z)$, something like that. And this guy, this should definitely be some kind of analytic space, some kind of adic space, some non-affinoid---I mean, some model by some rings with a non-archimedean norm. Well, it should definitely include those archimedean and non-archimedean parts, $\R$ and $\Z_p$. You know, hopefully a kind of uniform language to talk about these parts. But it should also be extremely non-noetherian---you can't hope for any of the usual finiteness properties often imposed.

So it's clear that you need some new language to talk about these things. And so basically, the goal of this course is to develop a language in which it is at least conceivable that such objects exist.

I mean, here's some kind of very vague idea that's possibly completely misguided. So when you do these perfectoid things, you have some kind of perfect field, maybe $\mathbb{C}_p$ or something like this. But then when you have an atlas over this, you're always doing the thing where you adjoin a variable, but with it also all its $p$-power roots. So I mean, this is a prototypical example.

Well, if you think about how you would do something similar, like not a fixed prime but kind of globally---well, then you say you definitely want to get rid of the choice of the fixed prime here. So okay, so maybe you have no idea what the base is, but at least you can try to understand what happens relatively. And then relatively, I think you at least certainly want to adjoin some variable with all of its rational powers. Because each prime $p$ is forced in some way.

But then you're only taking care of the finite primes, and then you maybe wonder, what could be the analog of this at an infinite prime? I mean, something that suggests itself---and again, maybe it's completely misguided, but something that you tend to wonder is whether you should try to form some ring where you adjoin all real powers of a variable $T$. But then you begin to wonder, what kind of object should this even be?

I mean, so you can definitely just treat the real numbers as a discrete thing and then adjoin here and all real powers, and then this would be some kind of algebra. But I mean, this feels somewhat artificial---that's the real, I mean, and sometimes then the real numbers are just some uncountable dimensional Cantor space, and you don't feel like you really made the situation any better.

So you definitely want to kind of keep track of the topology on the reals here. But then it's mixing the topology, like the profinite topology you usually have appearing, in a very strange way with the real

Correspondences, F-schemes and similar---this should be some algebra modeling for some local model of the space St I called this last time. But even I'm more unclear what kind of geometry that should be.

So maybe these are not at all the correct objects. But maybe we could at least, if these would turn out to be the correct objects, want to have a language we can talk about those. So some such things will be involved.

Okay, and so when Dustin came to Bonn in 2018, he somehow made the suggestion that one should really everywhere kind of replace topological spaces by condensed sets. Since 2018, we've been pursuing this path of replacing topological spaces by condensed sets.

Something that was definitely clear from the very start is that this switch does resolve a lot of the foundational issues you usually have when you work with some analytic geometry. For example, it makes it possible to remove Noetherian hypotheses that are quite pervasive, for example for adic spaces or formal schemes. I'll talk about abelian categories of complete modules and so on, and derived categories.

It was clear that a lot could be gained by doing the switch. Simultaneously, something that I always liked is that suddenly such a thing as $\Z[[T]]$ does make sense. This is something I will in some form also discuss today. Maybe this is not really the correct thing to consider, maybe some other thing, but I was quite happy that this kind of object at least exists in this framework.

So our attitude then was that maybe we have really no idea what we're really looking for, but let's just try to develop the foundations of some kind of analytic geometry from the perspective that you should start from some kind of condensed rings. Then try to build as natural as possible a framework so as to at least accommodate all the known examples and possibly extend them quite a bit further, for example allowing Schemes over $\mathbb{F}_1$ and $\mathbb{F}_{1^2}$.

Then just see where you're led to, and maybe someday one can hope to understand how such more exotic examples might be related to some kind of interesting geometry or number theory. Basically, try it out.

Also, in the known formalisms, there are sometimes things that are slightly beyond the traditional categories. For example, if you work with adic spaces, there's the usual things built on Banach algebras. But Grosse-Klönne for example has a theory where you have some overconvergent functions instead, and usually this is yet another category. You would also like to accommodate all these variants.

Peter, can you hear me? Great. There's a request to write a bit larger if possible. Oh, we cannot. Okay. Maybe we can increase the-- [Camera discussion]

Okay, so this course will not really touch on anything fancy like that. It will just try to really lay out what the formalism should be and how it accommodates the known examples.

Okay, let's start the course. Let me start by talking about condensed sets. The starting point for the whole theory is profinite sets, which will for the moment be restricted to these small things. First of all, I want to recall the following proposition. Sorry, I'm not writing bigger.

Always, the following categories are equivalent: The pro-category of finite sets. Sometimes this is a purely combinatorial kind of category, where the objects are just certain diagrams of finite sets, formal inverse limits, where the $S_i$ are just some finite sets and $I$ can be taken to be cofiltered.

Recall that this means that $I$ is not empty, and whenever you have two elements in $I$, you can find one that is strictly smaller.

The morphisms, well, this is an inverse limit, so you can pull that out first. But then there are two traditions about writing limits and colimits, either with an arrow and just a "lim" or writing "lim" and "colim". I'm kind of combining the two for maximum clarity.

I might at some point just write lim and colim, but for the moment I'll use the directional arrows. But then when you have a formal inverse limit, nothing just means one in TJ. This--

One can also think of this $S$ more concretely, but now using---in some sense this whole theory could be developed without mentioning topological spaces, but let's not try to do so. It's also equivalent to totally disconnected spaces, where here you send---let me write the functor.

A third possibility is to take the category of Boolean algebras. So recall what a Boolean algebra is: it's a commutative ring $R$ such that for all $x \in R$, $x^2 = x$. In particular, $(-1)^2 = -1$, meaning that $2 = 0$. So these are---let me write down some functors.

So if you have a formal inverse limit of $S_i$, you can map this to $S$, which is the limit of $S_i$ but with the inverse limit topology. And then this would map to the continuous functions on $S$ valued in $\{0,1\}$, which then also, if you compose the two things, is just the colimit of all $\{0,1\}$-valued functions.

And one can also go back. If you have a Boolean algebra $A$, this maps to $\mathrm{Spec}(A)$, which is also---you want a map from $A$ to---so actually all points of the spectrum are just $\mathrm{Hom}(A, \{0,1\})$. And actually, we can also write this $\mathrm{Hom}$ as a colimit, a limit over all finite subalgebras $A_i \subset A$ of $\mathrm{Hom}(A_i, \{0,1\})$. Alright, let me not say anything about the proof. I mean, one can actually quite easily check these functors.

So most often, I will actually think in terms of this presentation. Whenever I have a profinite set, I will usually just present it in some way as such a limit. I'll most often think about it that way.

And now that I want to come to the main things, I need to tell you two ways to measure how big---well, let's say $S$. When I write this, I implicitly mean that the $S_i$ are finite sets and the index category is filtered.

The size of $S$ is just the cardinality of the underlying set. And then there is something that's traditionally called the weight. Maybe this has a different name, let's just---so this is the cardinality of the corresponding Boolean algebra.

Let me make this definition. Here, $S$ is light if it has kind of the smallest possible weight, except it could be finite. We definitely want infinite things. $S$ is light if the weight is at most $| S |$. It's countable, yeah, so this is also equivalent to--- actually, if $\lambda$ is infinite, then it's also equal to the smallest possible cardinality for a small, possible---

In general, there are many ways to present a profinite set. For example, a point you could present as an index limit of a large profinite set of just always a one-element set, which is kind of silly. But there's always a minimal possible choice. And if it's infinite, then this is unique.

So in other words, a profinite set is light if and only if it's at most countable. Alright, so let me give some examples.

Just a quick question: "How does it compare to having countably many non-trivial closed subsets?" How do the properties compare to having countably many non-trivial closed subsets? If you take the maximum---I mean, the closed subsets are basically the ideals, right? How many of them---what does it mean?

Yeah, I think that's also the same thing. Because any such collection is given by countably many finite collections of such subsets. And I think this is still the same.

So let's do some examples of profinite sets and of their size and weight. Okay, so for finite sets, you can figure out what the size and weight is.

And then maybe the first, the smallest kind of infinite profinite set is the one-point compactification of the integers. And one way to think about this is that it's a limit of all maps $\N \to [n] \cup \{\infty\}$ counting up to 

Limit. So it's still light, but obviously it's much bigger than this one if you just think in terms of the number of points. Number of points is, in general, these are really different.

Let me give one more example that has some relevance in the theory, although excluded. So that one very important thing: any locally compact space has two compactifications---one compactification, and the Stone-Čech compactification. This one is small, this one is really huge. So let me not try to present this as a limit.

Well, let me instead of presenting it as a limit, let me really say what are the continuous functions from $S$ to $\F_2$. These are the continuous functions from the Stone-Čech compactification. But by the universal property of Stone-Čech compactification, this is really just a continuous map from the natural numbers to $\F_2$. But these are discrete, so this is really just the set of giving such $\N \to \F_2$. You just have to specify the image of zero, and this is just any subset of the integers. So this is just the set of all subsets of the numbers.

So we see that this actually has weight $2^{\aleph_0}$, and it's not super clear, but you can show that it has size $2^{2^{\aleph_0}}$. It's pretty large. So in all those examples, it turns out that, well, in the example that you gave, the size in the infinite case---the size is either equal to the weight, or it is bigger.

For some reason, I only hear you very softly. We need to get CL. Okay, okay. So I just... There is something that I don't remember the answer to. Do you hear me now? Yes, now I hear you better.

Okay, so in the examples, of course there are trivial estimates that the size is at most $2^{\text{weight}}$ and at most $2^{\text{size}}$. And so you can ask whether there is about inequalities between them. So in the case where you consider, like you have the countable set and the Stone-Čech, the size is $2^{\text{weight}}$. So the question is whether it can be the other way in the infinite case, that is, that you have actually... When I prepared, because I wanted the same question and I didn't, but I think I know an example, but anyway it's a bit complicated. But that's also what I suspected.

So let me tell you what I know is easy and sufficient for us to know. Here's a proposition. As Ofer mentioned, you have trivial estimates, and I will tell you they are trivial, that $\lambda$ is at most $2^{\kappa}$ and $\kappa$ is at most $2^{\lambda}$. And you see that this one can be attained, and actually in quite large generality you can make examples where this becomes as large as this. But it's actually quite hard to give examples where $\lambda$ becomes bigger than $\kappa$.

So let me just note that in the case that's most relevant to us, if you have a profinite set that's countable, then actually it can also be written as a countable limit of finite sets. So all the ones, these are like, I mean, this wouldn't follow from this inequality, right?

Why are these called "trivial"? $\lambda$, I said, was the continuous functions. Certainly bounded by all the maps from, this is obviously $2^{\kappa}$. And on the other $\kappa$ is the homomorphisms from the Boolean algebra, right? Because I said you can recover the profinite set as a map from Boolean algebra to $F_2$. And again, this is bounded by all the maps from $\lambda$.

So what's quite nice about this is that these estimates even hold true in the finite case. In the finite case, actually this one becomes $\kappa = \lambda$.

Let me do this one case where $\kappa$ is equal to $\aleph_0$. So then you can enumerate $S$, we can enumerate the elements. And then for each $n$, inductively choose a quotient $S_n$ compatible, so that first three elements $z_n$ inject any finite set. You can always find such a quotient.

Then you see that the map from $S$ to the inverse limit of $S_n$

As Scholze mentioned, you build a finite quotient where these elements are distinguished and then take the limit.

Alright, okay. So because this SL condensate will be so important for us, let me just rephrase the first proposition for light condensates. The following categories:

First, let's call it $\text{Pro}^N$, the sequential pro-category of finite sets. The objects here are not some fancy limit along some cofiltered poset, but really just some limit along the integers of some finite sets. And the morphisms are as before, but you are not allowed to change the $m$, so that's an important thing. So the $\text{Hom}$ from $\varprojlim F_m$ to $\varprojlim G_m$ is again, you can first pull out the limit and then take the other thing to be the colimit. Let me actually note that there's a different way to think about this. Basically, something you are always allowed to do in a pro-object is to pass to any cofinal subsequence. And then, if you want to give a morphism from here to here, you think that first you extract a subsequence of the $n$s, and then you really just give compatible maps. So one way to express it is, it's really some colimit over all possible strictly increasing functions $\varphi$ of compatible maps for all $m$ towards $G_{\varphi(m)}$, but from the three-scale version.

You can also phrase this pairability condition in terms of totally disconnected compact Hausdorff spaces, and it's precisely the condition that they are metizable. And the last one was Banach algebras, and for Banach algebras, we precisely made the condition so there we made it.

So let me just note one very simple proposition that will be used throughout. If you work in $\text{Prof}_{\text{fin}}$-sets, and this says you have all limits, but if you restrict to light $\text{Prof}_{\text{fin}}$-sets, then you still have all countable limits and sequential limits of surjections are surjective. Maybe I should say surjectivity is just meant in terms of the underlying sets, like taking the actual limit. So in other words, surjectivity on the point set of the component processes. And I mean, something similar is true, that any limits of surjections are surjective in all $\text{Prof}$-sets, but there it uses some rather high-powered compactness result, like the Tychonoff theorem. In the sequential case, it's really kind of stupid, you just successively lift.

Okay, so the first interesting thing I want to mention is that the natural numbers actually play a bit of a universal role within the light profinite sets. That is, if you have any light profinite set $S$, then there exists a surjection from $\widehat\N$ onto $S$. I mean, the proof is really simple induction. It just writes $S$ as a sequential limit and then at each point, just pick a large enough finite quotient of $\widehat\N$ to accommodate everything you already have.

Alright, so now I want to come to two properties that light profinite sets have and that fail in general, and that will play a technical role in what we're doing. These are some of the key reasons that we make the switch. Now, two properties that are special to light profinite sets, they may also work for some other $\text{Prof}$-sets, but that would be hard to single out there.

First, open subsets of light profinite sets can be too wild. If $U$ is an open subset of a light profinite set $S$, then $U$ is actually a countable disjoint union of profinite subsets of $S$. So this is to be contrasted with the following: In general, there exist open subsets $U$ in a profinite set $S$ that are not disjoint unions of profinite subsets. One way in which this can fail is, for example, you could take a product of profinite sets. I mean, disjoint unions take coproducts to products, and on profinite sets, these are totally disconnected. So you can't have non-trivial open coverings in this sense. But in general, the structure of these open subsets

Let $U$ be an open subset of a profinite set $S$. Then $U$ is a union of clopen subsets, so we can write $U = \bigcup_{n} S_n$ where each $S_n$ is clopen in $S$. More precisely, choose a presentation $S = \varprojlim S_n$ where the transition maps $S_{n+1} \to S_n$ are surjective. Then $U$ is the union over all $n$ of the images of $S_n \setminus f_n^{-1}(S \setminus U)$.

So now you've at least written it as a sequential union of closed subsets. Maybe I should have said, one way to think about open subsets purely in this language of pro-categories and profinite sets is to think about the closed things instead. Closed subsets should themselves be profinite sets, and then the closed subsets are precisely the injective maps of profinite sets.

The closed complement should be a profinite set. Someone injects, you take the preimage of this. Then if you want, you can write $U$ as a union over all $n$. Now these are subsets and are themselves profinite sets. Okay, so that's one nice thing.

Here's another one. Then $S$ is an injective object in the category of profinite sets. I will spell out what I mean by that. This means that whenever you have an injection of profinite sets $Z \to X$ and a map $Z \to S$, you can always find an extension $X \to S$.

Assume that these don't characterize all injective objects. Or are these exactly injective objects? So there are more injective objects. For example, any injective object in general is closed under taking products, so any product of profinite sets would also be allowed. But the profinite sets in particular are profinite.

So let me prove this. Actually, the first thing one should check is the case where $S$ is just $\mathbb{F}_2$. In this case, it means that the continuous maps from $X$ to $\mathbb{F}_2$ are in bijection with the clopen subsets of $X$. Or equivalently, any clopen subset of $Z$ can be extended to a clopen subset of $X$. Consequently, I don't know what an exercise to do.

Okay, let's assume we know this case. Then in general, just write $S$ as a limit of finite sets $S_n$. In general, the transition maps are not required to be surjective, but I will assume that all the maps are surjective. You can always assume that.

In this case, you will argue by induction on $n$. If you want to extend the map to $S$, you need to extend it compatibly to all the $S_n$'s. But if you've already extended the map to $S_n$, then extending further to $S_{n+1}$ means the whole situation decomposes into a disjoint union over all the fibers over $S_n$.

So you can assume that $S_n$ is just a point, but then $S_{n+1}$ is just some finite set. Then it's a very easy exercise to extend, maybe just have to extend a bunch of clopen subsets. Okay.

Exercise: Figure out why this argument doesn't apply to a general profinite set. You might still think that you can similarly extend as a limit along surjective maps. You can always do that and then try to inductively extend.

Just a small remark. If $Z$ and $S$ are both empty, do you need something? No, $Z$ is empty and $X$ is not empty. Okay, if $S$ is not empty, it's okay.

Okay, I think that's it for my general preparations about profinite sets. So let's now finally come to the definition of profinite condensed sets on a site with the following covers. We always take a profinite set $S$ and a cover of $S$ by other profinite subsets. This is the cover, and also surjectivity.

So let me, as last time, spell out what this really means. A profinite condensed set $X$ is a functor from profinite sets to sets satisfying the following conditions:

1
Here is my attempt at correcting the transcript with punctuation, capitalization, and paragraph breaks:

Just need to $M$ both of them individually, and then there is this funny condition. This comes from allowing all the surjective maps. This means that whenever you have any surjective map of profinite sets, then to give a map from $S$ to $X$, it's sufficient to give a map from $T$ to $X$. 

At least as a first approximation, you want that any map from $S$ to $X$ is determined by what it does on $T$. But actually, you also want to characterize which maps from $T$ to $X$ actually come from $S$, and these should be the ones that agree on fibers of this map, so to speak. The good way to say this is that the two ways you can make a map out of the fiber product, either first projecting to the first coordinate or the second coordinate, this should agree. This is what the sheaf condition for this surjection says.

Before expanding machinery, let me just tell you this key example to have in mind. Let's say $A$ is a compact Hausdorff topological space. Then we can define $\underline{A}$, and this is the thing that takes any $S$ to the continuous maps from $S$ to $A$. Here the presheaf on profinite sets precisely remembers how profinite sets map continuously into your topological space. This has all the fun properties.

Whenever you have a map between profinite sets, that remembers how a continuous map from one gives one to the other. It turns out that this condition is always satisfied. There's actually not a concrete topology, so if you would omit continuity then this would be clear. If you want a map from $S$ into $A$, it's sufficient to give one from $T$ into $A$, and then it factors over $S$ if and only if on the fibers the map is constant, which is kind of expressed by this.

But you're saying more than that here, because you ask that the maps have the continuous property. In other words, if you have just any map from $S$ into $X$ and you know that if you restrict to $T$ it becomes continuous, then it was actually continuous to start with. Equivalently, $S$ is actually a quotient and has a quotient topology from $T$. This is actually a general property for profinite compact Hausdorff spaces - they are actually quotients. In particular, $|A|$ is just $A$ as a set.

In general, for any condensed set $X$, we think of $|X|$ as the underlying set. But you can also evaluate on some of our other favorite profinite sets, and maybe the most important one is this one-point compactification of the integers. What does this correspond to?

It's a continuous map from $\N \cup \{\infty\}$ to $A$. In other words, it's a sequence in $A$ together with a limit point. So in other words, it's a convergent sequence, generally with the choice of a limit point, though most often there's at most one limit point. 

Similarly, when evaluating $X$ on this, where giving such a convergent sequence, we also have to give a witness for what the limit is.

Okay, and then you can do more wild things. You can evaluate this at the Cantor set, and well, this is what it is. One thing to note however is that it's just a set, but it comes equipped with all the continuous endomorphisms of the Cantor set.

Let me just give one remark here and then forget about this forever. If you have a condensed set, it's completely determined by what I will describe as a functor from profinite sets to sets. By $|X|$, the Cantor set together with the actions, where this is really just an abstract set. This guy here is just, so you could think of a condensed set purely algebraically as a set equipped with an action of the profinite monoid. But I think this is a strictly worse way to think about this. Don't do that.

But I mean, it makes the point that in some sense such a set is a very algebraic kind of thing. But why, maybe I should at least say why this is the case. This is precisely the case because if you want to know what the value on any other $S$ is, you can cover it by Cantor sets. And then for the fiber product

You're saying it into the category of what, exactly? Well, I think just this functor from sets to... No, sorry, this presheaf on this abstract monoid. Right, just consider sets equipped with an action of this monoid, this abstract monoid. I think this is a fully faithful embedding.

Maybe put a topology on this that... Actually, no, because you need to ensure that this funny covering condition... Here, I mean that for any surjection in the category, in particular, you need to ask the sheaf condition. And this amounts to some conditions that are possible.

Right, so this functor that takes a topological space to... At least for today, I want to stress the other direction, because I'm also discussing something related.

So this has a left adjoint that takes any condensed set $X$ and maps it to the underlying set (what you want to think of as the underlying set) and canonically equips it with a certain quotient topology.

So whenever you have anything that you want to think of as a continuous map from $S$ to $X$, you in particular get a map from $S$ (or rather, just the underlying set of $S$) to this $X^*$. This gives us a map from $S^*$ to $X^*$. Take this union here and endow all of these with their natural topology as a condensed space, and this was an isomorphism.

Okay, so the image actually lands in... This will always be like "mildly compactly generated". What is this? This is for fixed $S$, and all $X$. Or if you want, you could just take the full subcategory, because everything is a presheaf on this.

So for topological spaces, there's a notion of a compactly generated topological space, which is one where, when you want to test whether a map is continuous, it's enough to test it on compact spaces mapping to it. And this is by definition the case for this $X^*$, because if you want to test continuity from here to somewhere, by definition of the quotient topology, you only have to test it from here.

So you only have to test continuity on these condensed sets, but these are actually metrizable. So it's actually in this sense "mildly compactly generated". I hope you can imagine what this should mean.

And conversely, if you start with a mildly compactly generated space, first treat it as a condensed set and then go back, you're precisely recovering the correct topological space, because basically exactly this condition here, and because the condensed objects anyway will come back.

In other words, there is any... Basically, any underlying... So in other words, the kind of unit of the adjunction map in this case.

So they have the... I mean generally, any $X$ maps to this $X^{**}$, the unit of the adjunction. And on these guys, it's an isomorphism.

And so in particular, this means that these guys will form a full subcategory of all condensed sets. And it's... This is a very weak condition to be in this subcategory. I mean, virtually all the topological spaces that ever arise in nature... I mean, for example, we're mostly using these condensed sets in our, like, functional analysis, so to say, for topological modules. And any kind of Banach spaces, Fr\'echet spaces, whatever, they all have this property. They are all usable. So it's not so bad.

I wanted to make one more remark here about the relation to other similar tools that have been considered in the literature.

One is something that I think was quite influential. There's a paper of Johnstone called "On a topological topos", where he has a similar idea that, because topological spaces are not such a very well-behaved category, you should rather try to find a topos that is very well-behaved, which is very close to topological spaces.

So this is something that's achieved by this condensed sets. They form a topos, and one that is extremely closely related to topological spaces. And such things have been done before. One is Johnstone's topological topos. This is based on just the sequence spaces, so no larger profinite sets appear, only the sequence spaces.

And he uses a canonical topology. So on any category, there's a so-called "canonical topology", which is the finest one for which all the representable presheaves are sheaves. In general
Infinity that are really just infinite covers, so it's an infinite collection of them that covers, but no finite subcollection will cover. But this actually leads to a Banach-Alaoglu property.

You could also just use a finitary one and then get a version of astosch where I think most of what does also works. And then actually there's one that extremely close to what we're doing. I think a paper by es, if I remember right. Basically what they're doing is they take like profinite sets, but only finite. This is a finitary topos, so that's nice, but they don't allow all continuous maps.

I'm not sure if I will come to it today, but it's really important for the good algebraic properties for the functors we want to do to allow all here. I might come to such a point today.

I think they make explicit what their stuff is in terms of this picture, but I think when you only allow the discrete, it is slightly better to make this exclusive.

So you said that being finitely generated is a really weak property. It gives some wellability, for example, in schwartz, or even weak sequential implies sequential. Sequential just means that this sequence space is enough to check, and even that is basically always satisfied. That's why I mean Johnson I think came up with the idea to just use this one, because usually it's actually too small to around with a version that uses even smaller spaces like this.

But in the end, it's actually quite important for us to keep the countable sets in, because if you want like a countable set that surjects onto any metric compact space, for example any like closed manifold or something, you can always cover it by countable. And this is actually important for us, that you can always find from like one profinite the whole thing. Otherwise we couldn't control the other at all, it would become infinite.

Right, so I have here something about this, but look not now.

Right, so maybe this is all I want to say right now about condensed sets, and as I said, for us their main importance is as a home for doing homological algebra. So let's talk about like abelian groups.

Recall like on any topos, abelian groups always form an abelian category. In particular, colimits exist, and there's a set of generators. In particular, this applies to condensed sets.

And so in particular, it's definitely abelian, and now something must have happened that like in topological abelian groups, we run into this issue that they are not at all abelian categories. But now we have abelian categories, so let me briefly discuss how that's possible.

Dustin mentioned the inclusions can be problematic. So for example, you might take $\mathbb{Q}$ inside $\R$, which is a natural topology, or even more drastically, you could have $\R$ a discrete topology inside $\R$ with its natural topology. So these are maps of topological abelian groups, perfectly nice ones, but where the cokernel is kind of problematic.

So let me briefly just compute these cokernels and abelian groups, or like what happens if you take underline. Well, first of all, a point definitely just quotients.

But more interestingly, what happens to give a condenser, we also have to give the values at any set $S$. And so like, you have to be slightly careful when you take quotients, because now the sheaf condition actually becomes important. And like the naive answer you might guess is that this should be continuous maps to the reals mod continuous maps from $\mathbb{Q}$, where $\mathbb{Q}$ is discrete. These are just locally constant.

And you can actually prove that you don't have to sheafify in this case, and this is already a sheaf, and this is the true answer. Okay, so this means that this quotient still kind of remembers something about how there was a topology on this guy, even if you can't really phrase it in terms of the topology itself in this guy. It still remembers that on this part you should take all the continuous maps, but on this part you should only allow the locally constant ones.

And so now let's even do the more drastic thing where you modify all of $\R$ by a discrete guy. Well, then the underline set is just zero, right? It's $\R/\R$, which is zero.

But there should be a value at 
Very much nonzero. So here's some controlling this funny thing by observing that it has nontrivial maps on general. So, say again... Andity, it's not... Sorry, I again meant to... The, yeah, think... I mean, the rational numbers are not at all embedded in the real numbers. Okay, so let me just state the theorem and then maybe prove it next.

So, part of this I already said. Condensed light condensed... In particular, the exact... But even better, countable limits. Countable products. And for the people that know this funny XM, that's Soal 86 XM inen Le's hope paper, which is about some funny way that products can affect the co-limits. And this is satisfied for products.

This is worse than in all condensed being GS, where all products are exact. Here it's just the countable ones, but I don't know, I mean, most of them really only take countable limits, so it's not so bad. But it's one reason that we were at first a bit hesitant to make this switch. It's also now not anymore... It doesn't have enough compact projective objects.

But one thing that's extremely nice: there is a free guy guys stop. So you can take the light profile set convergent sequence, and then you can always build a free condensed group on there. This turns out to be internally projective. And this is really a property that's extremely specific to the light setting.

Within all condensed being groups, we have plenty of projective objects, but none of them are internally projective, except trivial cases like $\Z$. But also, all of them are really, really big. I mean, they all come from extremely disconnected sets, so they are kind of impractical.

This one wouldn't be projective in condensed groups, but within light condensed groups, it just so happens to be projective and even internally projective. And so this is the only setting of any variant of condense theory in groups where I'm aware of any non-trivial object that's internally projective. And like, the free on a convergent sequence is actually kind of important as a very basic object in the theory. So it's really nice that within this setting, it has its good categorical properties. And it's one of the main reasons we made the switch to the light setting.

The forgetful functor, say, from light condensed groups to the underlying light cond... This has a left adjoint, the free functors on objects. Any light profile sets with free being group on that. I will discuss it more next time. In particular, inside here you have light profile sets while... And in particular, you can take on the light profile set and infinity fre, okay?

So I have this and then discuss some other things on... One questions? Can I ask a question about the... So you said you stated two facts about light case that are not true in the general case. This is a technical question to understand the contra in fact.

I found... So the one can give cont example using the interval from zero to the first uncountable ordinal, which is a forfinite set. And so for the second statement, I take this cross itself and then the closed subset, which is the first uncountable ordinal cross the set union, the set cross this last element, and then the whole thing, the overall product. This is not the retract of the whole product. And for the first thing, I take the complement of the last element, and this is a big open which is not disjoint union of Clen. This is not difficult to see.

On the other hand, you state that there are cases where the shift topology is nonzero. So just for a general interest, I want to know if this is true in this case, and what is the reference for this fact? The shift topology can be... No, if you take the first uncountable order, treat it as a profile set, and remove the limit point, this is a case where the open is not a disjoint union of profile sets. Doesn't high... I think it has.

But you said that people know that sometimes it has... I think you can definitely put examples where there is... You can also take the Stone compactification of integers and remove a part of the boundary. Also pretty, maybe less computable in general.

Is it still true that light top is... You take category stop this... Yeah, you know, the underlying set, right? So that's enough for face.

There no further questions, and let's stop here and we resume on Wednesday.
\end{unfinished}
% !TeX root = ../AnalyticStacks.tex

\section{\ufs Light condensed sets II (Scholze)}

\url{https://www.youtube.com/watch?v=me1KNo3WJHE&list=PLx5f8IelFRgGmu6gmL-Kf_Rl_6Mm7juZO}
\renewcommand{\yt}[2]{\href{https://www.youtube.com/watch?v=me1KNo3WJHE&list=PLx5f8IelFRgGmu6gmL-Kf_Rl_6Mm7juZO&t=#1}{#2}}
\vspace{1em}

\begin{unfinished}{0:00}

Okay, so let me recall a little bit where we were last time. Last time, we had this category of profinite sets. There are several ways to think about this: either as sequential limits of finite sets, or as metrizable totally disconnected compact Hausdorff spaces, or as countable Boolean algebras. We equip this with a Grothendieck topology where covers are generated by the following two families: finite disjoint unions and all surjective maps.

As I kind of stated last time, this has a very important consequence that I want to stress again: sequential limits of covers are still covers. This property will actually be extremely crucial in order to have a good resulting theory of something like topological abelian groups that we want to use to have good homological algebra properties. This is something I maybe want to stress today a little bit, where this appears.

Then we had this category of condensed sets, which were these sheaves for the Grothendieck topology on this category of profinite sets. In general, whenever you have sheaves on some site and the generating site embeds (well, at least if it's a subcanonical site), you have a Yoneda embedding. You have the profinite sets sitting inside there. This sends any profinite set $S$ to the functor that takes any $T$ to the maps from $T$ to $S$, maps in profinite sets. If you prefer to think more concretely in terms of compact Hausdorff spaces, this is continuous maps.

One important property, which is also completely general, is that the image generates under colimits. You can think of general condensed sets as being built out of profinite sets by some kind of gluing procedure.

Then we also compared this to the category of topological spaces. In particular, we can embed this here as certain compact Hausdorff spaces. In fact, more generally, for any topological space $A$, you can build such a thing. I think I called a topological space $A$ last time $\underline{A}$, which is given by the same procedure. You send any $S$ to the continuous maps from $S$ to $A$. So this diagram commutes.

As I also said last time, and I kind of want to improve on something I said last time, this $A\mapsto \underline{A}$ is left adjoint. It takes any condensed set $X$ to, well, if you have a condensed set, in particular you have the value on a point, which we think of as the underlying set of this condensed set. Then we were equipping this with a certain topology.

Here we're taking the disjoint union over all possible maps from the countable set into $X$ of the resulting map from the countable set onto $X_{top}$. The reason for the countable set being that it's any profinite set that admits a section from the countable set. So it's enough to allow the countable set here.

But something which I kind of forgot about and that Yok Morita reminded me after my lecture is that actually, you can describe this quotient topology differently. Because the countable set, I mean, convergence of the countable set can be detected by sequences. Or more precisely, if you map all possible convergent sequences to the countable set, it's still a quotient map.

So actually, you can also consider all, let's call this $\beta$, all convergent sequences in $X$ of just the convergent sequence. This is also a quotient map in topological spaces.

So if you want, the countable set can actually be thought of as the colimit over all, say for example, countable closed subsets. So for example, finite unions of convergent sequences. This is in $\mathbf{Top}$. It will not be true in condensed sets, and I will discuss this in a second. But it is true in $\mathbf{Top}$.

This means that equivalently, you can describe this corresponding topological space as just the one where you just remember what the convergent sequences are. That's enough to determine this topology. In particular, this means that this funny notion that I was talking about of being compactly generated is actually just the same thing as being sequential. So that continuity can be checked by checking whether convergent sequences go to convergent sequences.

And so the thing I said last time, someone, more succinctly, is saying that sequential topological spaces embed into light condensed sets. Okay.

Okay, so from this perspective it seems that allowing the countable set didn't really help at all. But now I want to say why we allow the countable set. So is it the same or more general than first countable topological spaces? I forgot all my general topology. Does somebody know, is sequential the same as first countable? It could be, discrete on huge, but if the greatness of first countable... Because in the previous version of the theory it was first countable, I forgot, I think first countable with all points closed. But now we don't need all points closed because we don't have the set theory, right?

I mean, why allow the countable set? We're frozen, I'm frozen. I hope I'm un-frozen.

So the reason is that in any topos, for example, whenever you have sheaves on any site, there is an extrinsic notion of being compact and of being Hausdorff. So these are the following notions that I want to recall. Frozen again, frozen again, why? Oh my God.

So this general topos notion is what's called quasicompact. If any cover admits a finite subcover, here this general notion would amount to saying that there exists a surjection from the countable set onto $X$, or $X$ is empty. Yes, thanks.

And then there is this intrinsic notion of being quasiseparated, of being Hausdorff. And this notion was introduced by Grothendieck in SGA4. And of course, Grothendieck was using algebraic geometry lingo, so instead of Hausdorff he said separated. And because it's just a general topos notion, and not the specific separated notion in algebraic geometry, he called this quasiseparated.

So in general this would say that for all quasicompact, well maybe I'm assuming here something about enough quasicompact objects, but there are enough quasicompact objects. And for all quasicompact $Y$ and $Z$ mapping to $X$, so we hope that the technical problems got fixed, let's see, the fiber product between the two is quasicompact.

And again, let me simplify this here again, because one can always surject onto any quasicompact guy by something by the countable set. For all two maps from the countable set onto $X$, the fiber product, I should have chosen notation for the countable set, is still quasicompact.

So this I'm basically trying to say that, sorry, need not be surjective, just a map, that if you map a countable set into $X$, then it may not be injective. So this might factor over some quotient of the countable set. But you're somehow declaring here that it's the quotient by a closed equivalence relation on the countable set, which is one way to express some kind of Hausdorffness.

Okay, so in any topos you can talk about these quasicompact and the quasiseparated objects. And at least if your topology is finitary, the quasicompact guys are exactly those where like a finite union of the generating objects maps onto it.

So if we only allowed, for example, the convergent sequence as our basic guy in the test category, and the quasicompact objects would all be quotients of this guy, or at least be countable. So in particular, the countable set would not at all be quasicompact. And if you want to access the countable set, it would be precisely by such a colimit, that would be this huge colimit of smaller objects.

And this would mean that the intuition for what a compact object is, like the countable set is very much a compact object and you want that, this would be destroyed if you passed to such a topos.

And in fact, I mean, with our notion of what the generating objects are, you in fact get this really nice proposition that, by the way, quasicompact I very often abbreviate to "QC". I mean, in principle I hate abbreviations, but okay, this is QC and this is QS. And if both of them are assumed, then these are called "QQS".

So you can wonder, what are the compact Hausdorff, so to say, light condensed sets?

So this means that the topos, abstract topos theoretic notion of quasicompactness really very closely mirrors the idea of what a compact space is. And similarly, the general topos theoretic notion of what Hausdorff means mirrors precisely what you think Hausdorff should mean. And so this forces our Grothendieck topology to be finitary, because otherwise the basic objects wouldn't be quasicompact. And it forces us to include the countable set in the formalism.

Then you can also wonder more generally, if it drops the compactness hypothesis but still requires a Hausdorff condition, what are those things? So what are the quasiseparated condensed sets? Those also can be described in more standard terms.

So we know that the metrizable compact Hausdorff spaces are allowed. And in general, you should think of these guys as being rising unions of quasicompact subsets. In fact, they are exactly the ind-category, where all the transition maps are closed immersions of metrizable compact spaces. Let me write injections---I really mean all maps are injective, in which case they're automatically closed immersions of metrizable compact spaces. In the ind-category. So last time I was talking about the pro-category, which were formal inverse systems, and these are formal directed systems. So there is a category, where the objects are functors from a filtered poset towards metrizable compact Hausdorff spaces with all the maps injective. And functors, I mean maps, are the usual thing on ind-objects.

Yeah, so you could also drop the Hausdorff here and then drop this here. And let me just point out that within this category you have what topologists call the compactly generated spaces. Probably meaning sequential. But note that they are not the same, because in this category, as I said previously, you can write the countable set as a huge colimit of countable closed subsets, but not in the category of quasiseparated condensed sets.

I mean, these are both condensed sets. This is the prototypical example of an ind-system of metrizable compact Hausdorff spaces. This is also an ind-system where you just have one term, it's actually some quasicompact quasiseparated guy. But as condensed sets, they are very different. Because this is really just a formal colimit, and this is just one object. In particular, this guy is quasicompact and this one is not.

But actually, if you ask for ind-systems that are countable, then on such things, these all come from here. So some of the difference between these two categories only comes when the colimit category is pretty large, as it is in this case.

So I don't understand the subtlety. Is there a condition on---so if you take the condensed set which is the direct limit of the injections of the countable unions of images of the sequence space, does this give something by the equivalence of categories which is in this category MCG? It's not an equivalence of categories, right? This is just a full inclusion.

What is included in that? Ah okay, this one is included. Okay, I did not realize that this is what you were saying. 

All right. So let me just say something that's implicit. We said that something very nice, like the interval, is definitely a nice metrizable compact Hausdorff space. We said it should be quasicompact as a condensed set. And I said that quasicompact means that there must be a surjection from the countable set. In fact, that is true. You can find a surjection from the countable set, and in fact some kind of canonical one.

I mean, if you have a sequence $a_0, a_1, a_2, \ldots$ of either zeros or ones, you can send this to the number in binary $0.a_0 a_1 a_2 \ldots$, the binary expansion, which is an element in the interval. And any point in the interval admits such a representation. So you get the surjective map.

But we definitely need the countable set, or something as large as a countable set, to surject onto here. And so you can then recover this whole guy as a certain closed equivalence relation here, where the equivalence relation is precisely this nasty thing that $0.111\ldots = 1.000\ldots$. All right
Okay, so we're mainly interested in this formalism of condensed sets as a framework for doing some kind of algebra where all the objects have something like a topology.

To get this started, let's talk about condensed abelian groups. There are actually two ways to think about this, which I didn't say last time. Either these are abelian group objects in condensed sets, or these are sheaves on this category of condensed sets with values in abelian groups. Let me just write in symbols: sheaves on $\mathsf{Cond}$ with values in $\mathsf{Ab}$. Similar remarks apply to any kind of algebraic structure. If you have rings, for example, you can view them as ring objects in condensed sets or as sheaves of rings.

From the general theory of sheaves and Grothendieck topologies, we know that this is an abelian category, in fact a Grothendieck abelian category. So it has filtered colimits, etc. It also has a tensor product. Let me describe a little bit what the properties are.

The unit object is just the condensed set $\underline\Z$ associated to the integers as a discrete set. If you have two condensed abelian groups $M$ and $N$, then $M\otimes N$ is the sheafification of the presheaf that sends a condensed set $S$ to the tensor product $M(S)\otimes N(S)$. This is just a functor from condensed sets to abelian groups, a "presheaf", and then you can always sheafify.

There's a further property that I want to stress, which is also completely general. If you have a condensed abelian group, then you can forget its abelian group structure and just have an underlying condensed set. But this has a left adjoint, a kind of free construction. A "free condensed abelian group"---let me drop the "condensed" when I say that now---takes any condensed set $X$ to the free abelian group on $X$. Here, as in general for these colimit-type constructions like tensor products and left adjoints, you always have to sheafify. So it's the sheafification of the presheaf that takes any $S$ to the free abelian group on $X(S)$.

There is already some structure here that you don't often think about in topological abelian groups. You don't really often take the free group on a topological space. Here it exists, and it's actually a completely fundamental structure.

The idea is that this free abelian group on $X$ is some topological abelian group. What is its underlying group? It's just the free abelian group on the underlying set $X(\ast)$. So if you have any kind of topological space, you can just take its free abelian group. Of course, this machinery will then put some kind of topology on the free abelian group.

Let me actually discuss this in an example. Sheafification won't change the value on the point, because any cover of the point is split, so the sheaf condition is kind of vacuous on the point.

Here's an example, which is kind of related to the introduction I gave last time. We can take the real numbers $\underline\R$. Very soon I will forget to write the underline all the time, because implicitly everything has become a condensed set. But okay, so we have the real numbers as a condensed set, and then we can take the free condensed abelian group on that.

What kind of object is this? It's sums, if you want, of real numbers $x$ weighted by integers $n_x$, where the $n_x$ are almost all zero. You might think of these points $x$ as measures, and then these are finite sums of measures.

But now it also has some topology, where you kind of remember that these $x$'s are allowed to move continuously. But then, if you have $x + y$ in general, that's a non-zero element. However, when $x$ and $y$ become the same, then suddenly this collapses.

It's maybe not so clear how you would actually describe this topologically. If you wanted to describe this as a topological abelian group, you would have to declare what the open subsets are, and I think that's a little bit tricky to visualize.

But you can actually say what it is as a condensed set. First

Again, the $C_0(\Z)$ itself can be written as a rising union over all integers $n$ of subsets where you are only allowing sums $\sum_{i\in I} n_{x_i}[x_i]$ where the sum of the absolute values of the $n_{x_i}$'s is at most $n$. Everything is contained in something like this, as condensed sets.

So whenever you have a profinite set mapping into here, it will actually factor over one of these subsets. And these guys, they are compact Hausdorff and metrizable. It's a kind of fun exercise to figure out how to describe such a compact Hausdorff space.

I claim that whenever you have any compact Hausdorff space and you look at finite integral sums of points of them where the sum of the coefficients is at most $n$, there is a canonical compact Hausdorff topology on that. Classically, this takes a little bit of thinking. But in this formalism, this free construction just produces it for you automatically.

The subtle part is that there are some kind of non-trivial identifications you have to make. In general, $x+y$ is a non-zero element, but when they become equal, you have to collapse this to zero. To make this true, you actually have to use that in our Grothendieck topology, we didn't just allow finite disjoint unions but also effective epimorphisms. Otherwise, this wouldn't come out right.

Also, by general nonsense, if you start with something that already had a group structure, then if you pass to the free ring, this now has a ring structure where the multiplication comes from the addition. In particular, this is actually a condensed ring completely naturally. This is kind of related to the question I had in my first lecture, like how do you draw an element in all its real powers? It's just done by this construction.

These are general features. Now I want to mention a few features that are quite specific to this light condensed setting. The first thing is that countable products are exact. This might seem like an operation you're maybe not so often doing, but something you are certainly very often doing when you do some kind of functional analysis is to take sequential limits of surjective maps, and they are still surjective.

For example, in functional analysis, maybe you have some kind of Fréchet space and it's a sequential limit of Banach spaces. Maybe in that case it's not even surjective, but it will also come out right. You definitely want sequential limits to behave nicely. For example, if all the transition maps are surjective, you definitely want the limit to still be surjective. As I will argue in a second, this more or less forces you to go where your basic objects that define your site must be some kind of totally disconnected things.

You have these, and there is this other property that I will explain in a second. You have the sequence space $\Z^\N$, it's a profinite set or light condensed set, and you take the free ring on that. This turns out to be internally projective.

Recall that one way to say what projective means is that in any abelian category, you can define Ext groups. It means that $\Ext^i(P,-)=0$ for $i>0$. Internally projective makes sense when your category also has a tensor product, because if you have a tensor product, then you can define an internal Ext, and internally projective means that the internal $\Ext^i(P,-)=0$ for $i>0$.

The first two properties are actually things that are better in all condensed abelian groups, because all products are exact. Well okay, the third property is also solely still true. However, the third property is something that's only true in the light setting.

Let me just define the internal Ext. Basically, if you have a tensor product, then you can also ask for an internal Hom, which is some kind of adjoint of the tensor product. Similarly, the internal Ext will be some version of the same thing on Ext groups.

Okay, so let me prove this actually. Let me first note that one reduces to showing that taking products of surjections is surjective. The only thing that's not true for general products is that a product of injective maps is always injective and so on. The only thing that's not clear is that the countable product of surjections

For all $M$, you can take the product over all $n$ at most $M$ of $M_n$, and then the product over $N$ bigger than $M$ of $N_n$, and this surjects onto the product of the $N_n$'s. Now because a finite product is the same thing as a finite direct sum, they always preserve surjective maps. They're always exact, and so you can always, like, for finitely many coordinates, kind of do the lift. But then this guy here, I mean, this map is the limit now over $M$ of these things.

And so if we're asking whether a countable limit of surjections is still surjective, let's assume you have such a diagram. We have $M_0$, $M_1$, $M_2$, and $M_\infty$ is the limit of these, which certainly maps to $M_0$. We ask ourselves whether this is surjective. So what does it mean to be surjective in the sense of sheaves?

This means that whenever you have one of your objects generating your site, so any light profinite set, and a map from here, then, well, if it was a surjection as presheaves, you should immediately be able to lift that to here. But in fact, it's enough to find a surjective map from some---let's call it $S_\infty$---and a lift to here. So does there exist some other light profinite set surjecting onto $S$ and a lift to $M_\infty$? That's the question. That's what surjectivity on coverings amounts to.

But let's just see what happens. We definitely, well, right, so this is this map to $M_0$. But we know, because this map is surjective, we know that there is a light profinite set and a lift to here. And then again, because this map is surjective, there exists some further light profinite set and a lift to here.

And so now you can just take $S$ to be the limit of this diagram. So inductively, you construct light profinite sets with a map to here, and then take the countable limit of these surjective maps. As I said last time, and I think I recalled today, countable limits of surjections are still surjections in light profinite sets. Hence, this guy is still allowed as a cover in our Grothendieck topology.

So in the definition of your $X$ group in the sheafification, you drop the underline and the left $p$. So $\Z_p$ $s$, you mean $\Z_p$ $s$ underline? Yeah, let me drop the underlines. Okay, so $S$ is for me a light profinite set, and they sit inside of light condensed sets. So yeah, if you feel better, make an underline, I mean, yeah. And like, the Yoneda embedding has no decoration for me, it's just the same thing. But yeah, also in three, like, this threefold guy, it's like, maybe I should also underline $10$ here.

Okay, so before I go on to the proof of three, let me reflect a little bit on what happened here. This finishes one and two, and let me come back to that in a second.

So here, the critical thing---we definitely, for doing good homological algebra, like a functional analysis kind of homological algebra, you definitely want property two. But critical for two was exactly this property that countable limits of surjective maps are still surjective, that limits of covers in your Grothendieck topology are covers.

And I claim that this basically forces you to use totally disconnected spaces as building blocks. Why? You might also have the idea, and I think people are doing that, that if you want some kind of nice category of something like topological real vector spaces, you might work in this kind of smooth setting where you take your defining site to be like smooth manifolds, and the Grothendieck topology just the usual one of open covers of smooth manifolds.

But in that case, this property that countable limits of covers are covers is just not true. Because if you have, like, maybe just the real line, then you can cover it by two intervals. Each one of them you can again cover by two intervals, and then keep doing that. But then the intervals shr

Disconnected compact Hausdorff spaces. And initially, we took all of them. We realized it's slightly better to restrict to symmetrizable ones and to have all surjective maps as covers. Because you can actually show that any surjective map of topological spaces can be written as a sequential limit of actually open covers. So if you want open covers and their sequential limits, you need all of them.

Actually, I should maybe say that this is not some kind of ahistorical comment. This pro-étale topology was, in fact, first---I mean, so this stuff about condensed sets, this comes from something called the pro-étale topology, like originally in Bhatt-Scholze for schemes. And there, the wish was precisely that limits of surjective maps should still be surjective. Because this was---we wanted to have certain sheaves which were naturally certain inverse limits, and we want them to be well-behaved. And for this reason, we allowed these countable limits, or all limits of covers, to still be covers. So this is really the origin of this whole theory.

All right, so this was a small interlude. Excuse me. So if you have a surjective map of topological spaces, for example, you take the $\N \cup \{\infty\}$, two copies of $\N \cup \{\infty\}$. So all of the two sequences will converge to infinity. And then this is covered by two copies of $\N \cup \{\infty\}$, and they intersect at infinity.

You made a statement to the effect that---yeah, well, okay, not all transition maps are surjective. But actually, you know, I mean, actually something slightly stronger is true. I don't need all the transition maps to be surjective. I only need that the maps down to $S$ are surjective, and the limit is still okay. And in that sense, you can realize $S$ as okay in this other sense.

Okay, something slightly weaker would be---yeah, I'm not sure. I would have to think how much difference it makes.

Okay, so I want to prove this thing that this is internally projective. But before I do that, let me make a warning that this is a phenomenon that's only true once you pass to groups. So $\N \cup \{\infty\}$ is not at all projective in condensed sets.

And I actually kind of expected Gab wanted to go there with this remark. Because this is precisely the example that I need to do. So you can have a surjective map of topological spaces and the convergent sequence downstairs, so that there does not exist a lift. And one example for this would be to just take this to be the convergent sequence itself, and this to be like, I don't know, $2\N \cup \{\infty\}$ disjoint union $2\N+1 \cup \{\infty\}$. Some breakup of $\{\infty\}$ as like the limit of the even guys and the odd guys.

Then like on the integers, you only have one possible lift. But the even guards converge to something else than the odd guards. But miraculously, once you pass to groups---it so happens that if these were condensed abelian groups, in fact you don't---then any convergent sequence can be lifted.

All right, let me actually use a slightly different guy. So let $M$ be the free group on the null sequence. So you can take $\N \cup \{\infty\}$ and mod out by $\{\infty\}$, which is actually a direct factor, right? Because you have $\{\infty\}$ mapping here and then projecting to a point. So this splits, actually has a direct summand.

And so, and of course, the integers themselves, they are projective. That's okay. So the question is really whether this other part---so this classifies null sequences mapping out of $M$, is the same thing as giving a null sequence in the other guy. Or in other words, it's a free condensed abelian group on a null sequence. And free group on a convergent sequence, but then the limit point should be zero.

So if we want that this guy is internally projective---and let me actually focus on the projectivity, and then just
The property that $\infty$ goes to $0$... Because this is a surjective map of condensed groups, this means that there is some surjective map from a profinite set that lifts to $\N$. That's just what the surjectivity means.

Okay, now how do covers of $\N \cup \{\infty\}$ look like? $\N \cup \{\infty\}$ is like you have a discrete set and then it accumulates towards infinity. Now, in the pullback, you have some complicated $S$ upstairs here. But you can always make that smaller because, for each of the discrete points here, you can just pick any lift. I don't care which one, and then all these lifts together with keeping everything at $\infty$, this is a closed subspace. So there exists $S'$, a closed subspace, so that over the integers, it's just always a point.

You can assume that this $S$ here is somehow a different compactification of $\N$. Without loss of generality, we can always make $S$ smaller here, as long as it's surjective. So we can assume that $S$, the part over any finite thing, is just a point. Okay, but there are many compactifications of the natural numbers. For example, this one, so we can't expect that we can directly split that.

But now we actually use a property of profinite sets that I mentioned last time. Okay, so let $S_\infty$ be the fiber over $\infty$. This might be profinite, so there's a certain subspace in this, but everything here is profinite. In particular, $S_\infty$ is profinite. I stress it here because here it's really the critical point where profiniteness is used. This means that it's injective in profinite sets. In all profinite sets, it's also non-empty. And so there exists a retraction $r$ from $S$ to $S_\infty$.

Okay, so let me call this inclusion here $i$. Maybe this was my original map $f$ and this was a map $g$. Now we can just write down the lift. Consider the following map from $S$ to $\N^\sim$, which is the map $g$ that we had. But then from it, you subtract what you would get if you first use the retraction and then re-embed into $S$ and then apply $g$.

So if you look at $S_\infty$, then on $S_\infty$, these two maps are just the same because this was a retraction. Am I frozen again? So on $S_\infty$, it's actually just the constant map $0$. Was that a question? Am I frozen again? Do you see me now? Yeah, okay.

But $S$ surjects onto $\N \cup \{\infty\}$ and this map to $\infty$. Because this is actually a surjective map here, you can show that this is a pushout in profinite condensed sets. That's basically a version of this gluing condition that I gave. If I want to give a continuous map from here to somewhere, it's enough to give one here and here which agree on the overlap. But it's definitely enough to give one here, and then the question is just when does it descend down to $\N \cup \{\infty\}$? This means that on fibers it must be constant, but there's only one fiber in some sense where there's something to check, which is the fiber at infinity. And there we precisely assume that it all just comes from a point.

Okay, but now you're precisely in this situation. You have a map from $S$, you have the zero map, and they agree on $S_\infty$. So this means that actually this whole map factors over a map from $\N \cup \{\infty\}$ to here. Let me call that map $f$.

Okay, and now we basically have all the structure we need, and we just need to check that we actually produce the lift. So what do we have now? Now we have a map from $\N \cup \{\infty\}$ to $\N^\sim$ which vanishes on $\infty$. So in other words
Because this was a retraction, and so this means that, because of this pushout property, you really get an exact sequence. You have a map from the convergent sequence modulo infinity to $N$.

Now the question is whether we've actually lifted $f$, and the claim is that we did. To check that we did, note that this guy here is still surjected on by the free guy on $S$. So to see that this map here is $f$, it's enough to check for the composite.

The point is that if I take this thing that I used here as a correction term and I project that down to $N$, then this map from $S$ to $N$ vanishes on all of $S_\infty$. This term projects to zero in $N$, because its composite map from $S$ and $N$ vanishes on $S_\infty$. So the correction term indeed vanishes, and $g$ was a lift, so we're fine.

For internal projectivity, let me just not do it. It's essentially the same argument, you just have an auxiliary profinite set floating around that you also need to cover a bit. Again, it's critical that sequential limits of surjections are surjective.

Here's just one remark about the comparison to all condensed abelian groups. This is something that I already said a bit earlier, but just to say it more explicitly:

In all condensed abelian groups, products are exact and you have projective generators $\Z[S]$ where $S$ is what's called an extremely disconnected set. For example, it's a Stone-Čech compactification of some discrete set. These are huge things, they have cardinality $2^{2^{\aleph_0}}$.

One thing that's better there is that you really have projective generators. You don't have projective generators in light condensed groups. You have this one guy which is the free guy on a null sequence, but the free guy on a countable set cannot be covered by a projective object. This is slightly unfortunate, but actually not that bad in practice.

The good thing is that in light condensed groups, you have this really explicit projective object which is the free guy on the $\N$ sequence. Whereas here, these projective guys exist by the axiom of choice, but they are completely inexplicit and their precise structure depends on which model of set theory you're working with.

But one thing I want to point out is that none of those projective generators there are internally projective. Basically, if you take a product of two Stone-Čech compactifications, and $I$ and $J$ are infinite, this product is never projective. That's actually not so easy to see, but we can prove it. This is a rather severe technical issue in the category of all condensed abelian groups. It's extremely nice that you have these projective guys, but for many arguments you would really like to know that they are internally projective, and they are not. That's bad.

In light condensed groups, at least we have this one really nice internally projective object. That's good.

The other thing I wanted to point out explicitly is that the free guy on the $\N$ sequence is not projective in all condensed abelian groups. The light ones embed into the category of all condensed abelian groups, and you could ask if it's still projective there. But there, precisely the Stone-Čech nonsense gives the obstruction. When some universal compactification of the integers, the biggest one, is the Stone-Čech compactification, this rejects. You can show that there does not exist a splitting here.

Maybe another thing I should say is that this is also related to certain questions about Banach spaces that people have studied in literature. There, the only known injective Banach spaces—there's some kind of duality that what used to be projective in this condensed stuff will become injective in the category of Banach spaces—are continuous functions from some $S$ into $\R$, where $S$ is one of these extremely disconnected guys which I could have also allowed. Basically, such a Stone-Čech compactification or a retract of it.

It's known that all injective Banach spaces are retracts of continuous functions on a Stone-Čech compactification.

What about injectives? In your case, it is for light condensed abelian groups. Is it enough injective? For all condensed abelian groups, there are no nonzero injectives. In all condensed abelian groups, there are no non-zero injective objects.

Okay, yeah, they exist for set-theoretic reasons. I mean, for like general nonsense reasons in light condensed abelian groups. But I don't think you can write any of them down.

And what corresponds to this free abelian group sequence is the Banach space of null sequences. This is not injective as a Banach space. But it is what's known as separably injective, where you only test this injectivity against separable Banach spaces. This is very much related to the thing that this guy is not projective in condensed abelian groups, but it is in light condensed abelian groups.

Actually, I think I made this realization that it is projective in light condensed abelian groups after looking up this proof that this guy is separably injective. It was an open question in the Banach space literature whether the continuous functions on this guy is a separably injected Banach space. In the notes on complex geometry, Kan kind of proves that it's not separably injective. In particular, this guy also doesn't behave like a projective object, even when you test against light condensed abelian groups.

Anyways, maybe one thing to take away here is that you might think that all this totally disconnected nonsense shouldn't really appear when you do function analysis over the reals. But actually, people that studied Banach spaces intensively, they are very much studying continuous functions on totally disconnected things.

Okay, right. Yeah, maybe let me finish by talking about homological algebra. You also said something, I forgot in which context, about the axioms AB5 and AB6 of abelian categories.

In Grothendieck's Tohoku paper, you can find a lot of axioms that an abelian category might or might not satisfy. AB5 is the question of whether all products are exact. This is true in all condensed abelian groups. It's not true in light ones, but at least the countable ones are okay. When AB5 is satisfied, you can ask about AB6, which is a certain question about the commutation of infinite products with filtered colimits. This always sounds confusing, but you can actually make a statement that's true.

This property, AB6, is always true if you have enough projective generators. In particular, it's true in condensed abelian groups. If you restrict this question to countable products, then it's also true in light condensed abelian groups. So AB6 holds for countable products. One way to see this is that you have this fully faithful embedding of light condensed abelian groups into all condensed abelian groups. This commutes with all colimits and countable limits. In particular, these countable products are some of the correct products there, the ones that you would also compute in condensed abelian groups. But you can also just check it by hand.

In the last bit, I want to talk about cohomology. When you study topological spaces, you probably also care about the cohomology, in particular for manifolds where you would want that to be the singular cohomology. When you think about a topological space, we already have an idea of what the cohomology should be, just as we already had an idea of what a complex thing should be like, and so forth. But on the other hand, whenever you work in a topos, that topos somehow comes with its preconceived notions of not only what compact objects are, but also what cohomology is.

If $X$ is any condensed set and $M$ is any abelian group (it could even be a condensed abelian group, but let's restrict to discrete ones for the moment), you can define the cohomology of $X$ with coefficients in $M$. This is just something that is there whenever you work in the topos.

One way to define the cohomology of $X$ with integral coefficients, in terms of the formalism that I now introduced: You take R Hom in the category of $X$-groups in light condensed abelian groups (actually they would also be the same as in all condensed abelian groups, oh well, I'm doing light stuff so let me stick

So, um, if say $X$ is a CW complex, this thing is $X_\bullet$ or it's underlined. It's exactly the singular cohomology. This might seem a little bit weird because we didn't really put any geometry into the definition of these condensed sets, right? We were just using totally disconnected things, and suddenly using totally disconnected things, we are still able to probe whether that circle is not contractible. Um, but it comes out right.

Um, and, um, right, uh, so let me actually recall that. I mean, yeah, let me make one remark about this and then stop. So as I said, this is here the $\Z$ of these $X_\bullet$ groups, and I told you how to think about this, right? So this was, uh, the thing where you take finite sums of $X$-points of $X$ valued by integers. Um, and so this is actually---there's a Dold-Thom theorem which basically tells you that this guy is something like a model for the homology of $X$ once you, once you pass, once you treat this up to homotopy equivalence. This kind of guy is a model for the homology of $X$. Uh, and, well, for us the interval in condensed sets is like an actual interval, it's not a point. So we didn't yet pass to homotopy, uh, but one way in which you're doing that is by taking $X^s$ out of it, because an interval cannot map to something, uh, this being discrete. And so, uh, yeah, so this dualizing, I mean this is like homology, so the dual should be like cohomology, which fits the picture. Uh, and yeah, let me just, uh, stop here.

Other questions? Uh, all right, so we can also consider the internal $\Ext$-functor. Um, in this case, this would just unravel to an adjunction to, uh, also replacing $M$ by the continuous functions from $S$ to $M$, which is still an abelian group, so it doesn't really do that much more. Um, uh, maybe one comment to make is that, and maybe I won't talk much about it, but, um, for all sorts of things like continuous group cohomology and so on, sometimes it's not quite clear, like, what is a continuous representation of a group, what is the right notion of continuous group cohomology? Because often this is just represented by some explicit cochain complex, uh, but if you just treat your topological groups as condensed abelian groups, uh, then it's clear what an action on, like, a condensed abelian group should be, and it's clear what cohomology should be. There's some kind of general 2-categorical answer to what it should be, and this always gives you the expected answer. It's not always the same thing as the thing computed by continuous cochains, but when it's not, it's a better answer.

So, so here, when you, when you have a CW complex, if you have more generally a local system, I think it will give a sheaf on condensed $X_\bullet$ over $X$-underline, and then you can say that the usual cohomology, it's like singular sheaf cohomology. It satisfies cohomological descent, for, for at least for compact Hausdorff spaces, it satisfies cohomological descent of surjections of compact Hausdorff spaces. So to compare the two sides, it looks like using the theory of cohomological descent to, to, well, essentially to do the, should give, to do it with local systems. By, yeah, you can, you can use any coefficient system, really, on $X$-underline. This is, yeah, it's a really robust result.

Um, but if you, for example, work in, um, like, this topos of sequential spaces, where you only allow $\N \cup \{\infty\}$ as a generating object, then if you would try to compute this cohomology of, like, the interval, then you would first have to express your interval in terms of your generating object. So we would write this as this huge colimit of all these, uh, small countable sub-closed subsets in the interval, and then this $C$

\end{unfinished}
% !TeX root = AnalyticStacks.tex

\section{\ufs Ext computations in (light) condensed abelian groups (Scholze)}

\url{https://www.youtube.com/watch?v=EW39K0J7Hqo&list=PLx5f8IelFRgGmu6gmL-Kf_Rl_6Mm7juZO}
\renewcommand{\yt}[2]{\href{https://www.youtube.com/watch?v=EW39K0J7Hqo&list=PLx5f8IelFRgGmu6gmL-Kf_Rl_6Mm7juZO&t=#1}{#2}}
\vspace{1em}

\begin{unfinished}{0:00}
  All right, I think that's a signal to start. Good morning, welcome back. So today I want to talk about some EXT computations in condensed mathematics, and then some...

So I guess the basic claim is that this category of condensed abelian groups is a very convenient framework for doing homological algebra. It's a very nice abelian category. You can, without any problem, form a derived category. And because topological spaces essentially embed faithfully into condensed sets, topological abelian groups also basically embed faithfully (under some mild assumptions).

So now you have a nice home to play with them, but then because you have an abelian category, you can ask about EXT groups, and then you can wonder what they actually are. And for this theory to be somewhat useful, you want all the answers of these EXT computations to give you reasonable answers that are interpretable.

So the first theorem that I stated last time is the following theorem. One thing you can do is start with, for example, some nice geometric object like a CW complex X. And let's say M is an abelian group. Then one can look at the EXT groups in condensed abelian groups between the free condensed abelian group on X (or rather, on the corresponding condensed set), and M treated as a discrete abelian group.

Basically, if you treat X as a condensed set, then in any topos there is this internal notion of cohomology, which one way to express is as EXT groups in the topos. So this is the internal notion of cohomology of X with coefficients in M in the condensed world. And fortunately, it turns out that this precisely recovers the singular cohomology of X with coefficients in M. Okay.

So the first thing I probably want to do today is sketch a proof of this theorem. Maybe I should say that to a large part, in this lecture and maybe also the next one or two, I will still follow the first course that I gave on condensed mathematics some years ago. You can find lecture notes online, so for much of what I'm talking about today and last time, a reference is this "cond.pdf" that you can find on my webpage.

Okay, so proof sketch is that, well, as X is a CW complex, it's an increasing union of some $X_i$, where the $X_i$ are compact. It's filtrable, built from finite dimensional cells. And then on both sides, you take colimits (direct limits). Yeah, you probably need a slightly better statement there to really compare the complexes, not just individual groups. But basically, you can reduce to the case that you just have a compact space X, a compact CW complex.

Then there's actually a more general statement that works for any compact Hausdorff space X that may not be a CW complex built from cells. That's the Čech theorem. Again, you still have these EXT groups between the free abelian group on X and M. It turns out that in this case, you can always compute this as what's known as Čech cohomology on X with coefficients in M.

So here, \v Cech cohomology is defined in terms of the category of sheaves of abelian groups on $X$. You consider a topological space X (or the corresponding site), and abelian sheaves on X. Then you have a global sections functor from abelian sheaves to abelian groups (I'll write this as $H^0(X, -)$ to distinguish it from other notation). You apply this to the constant sheaf $M$ on $X$, and take derived functors.


It is known that Čech cohomology, when restricted to CW complexes, satisfies the Eilenberg-Steenrod axioms and agrees with singular cohomology. So for $X$ a CW complex, the two notions coincide. But not in general - in fact, singular cohomology is really only valid for CW complexes. In general you should use something like Čech cohomology (or really the same thing).

So here's a key example to keep in mind, of relevance to us: If $X$ is totally disconnected, then you can actually show that the global sections functor $H^0(X, -)$ is already exact. So in this case, Čech cohomology vanishes in positive degrees. Okay.
So here is a proposed correction of the transcript:

Compact, I mean... Yeah, I'm still... Sorry, yeah, the prof fin set. These are the global sections and the global sections are just locally constant maps from $X$ to $\N$, equal to zero, and there's no higher cohomology.

But if you would consider the singular cohomology, this is defined in terms of the singular chain complex, which is built from just mapping points and simplices into $X$. But from the perspective of all simplices, there's no way to tell apart $X$ from just a discrete set of points, right? Because simply connected... So any connected space will just factor over one point.

But when you... So you can also compute the singular cohomology of $\N$, and again it's $\R$ in degree zero, but in degree zero you get all functions.

Yes, continue. So this means that thinking about cohomology doesn't really see anything about the topology of $X$ anymore. The sheaf cohomology does.

I think another possibility is to consider locally contractible spaces. I believe that when you cross $X$ with an interval, this sheaf cohomology group should not change. I'm not completely sure what it looks like, and then one can do for locally contractible spaces a comparison with sheaf cohomology. And if in addition it is paracompact, then you get comparison to singular cohomology. But the usual results on... 

Yes, yes, right. Yeah, I think the key statement you really need about $X$ is that it's locally contractible in this funny sense. "Locally contractible" doesn't mean it's covered by opens, or that any point has a basis of open neighborhoods that are contractible. But rather that for any point and any neighborhood, you can find a smaller, possibly smaller neighborhood that can be contracted in the larger one. And under this funny assumption, that's the official definition of locally contractible.

You can actually show that yeah, I think everything that I said about CW complexes extends to this. But to compare sheaf and singular cohomology, the usual treatment requires paracompactness. I'm not sure if it is... So yeah, maybe paracompactness... 

Yeah, I think these paracompactness assumptions, they would kind of only matter in intermediate... If you go all the way back to here, I think it probably disappears again.

Yes, the question... So the question was, often one considers topological spaces up to weak equivalence. And then obviously any space is weakly equivalent to a CW complex, and for example $X$ is weakly equivalent to just a discrete set of points. But here I don't want to consider topological spaces up to weak equivalence, because otherwise I wouldn't be able to treat totally disconnected spaces at all. I'm not using here topological spaces as a way towards pure homotopy theory, I'm really interested in the actual topology.

Um, right. So okay, so here is what I want to prove. If I have any abelian group, then I can compare these sheaf cohomology groups. And this can actually be further upgraded, and I mean, I'm not doing this just for fun, actually...

So here's a first upgrade you can do. Sheaves were defined... So there are two sides. On the one hand, you can fix $X$, let's call it the topos of $X$. So this is what's used on the right-hand side. And on the other hand, you can consider the category of slices of the topos, defining light contravariant set-valued functors. You can consider light profile sets together with a map to $X$.

And once you pass to sheaves, or once you pass to the corresponding topos, it turns out that that's actually a geometric morphism here. So... These here are light sets with a map to $X$, and these are $X$, and there is a map here.

So what I'm telling you is that whenever you have a sheaf over $X$, you can pull it back to get a light functor on contravariant sets over $X$. And to define what the pullback is, you really only have to define it on those generating objects. So $\mathcal{L}$ of $U$, when $U$ is an open subset of $X$, is just $U$.

Okay, so here's the usual sheaves on $X$, and you can pull them back to get light sets over $X$. And correspondingly, you also have the topos of sheaves on 

This means that $X$ is a sheaf group, which are some kind of forms. In particular, for all $\mathcal{F}$ which are abelian groups on $X$, the cohomology on the site of condensed sets over this coefficients in $\mathcal{F}$ is the same thing as the cohomology. If you apply this to a constant sheaf, you recover the previous result. This is for $X$ locally compact.

The kind of $X$ I was having in mind is the case where $X$ is a compact Hausdorff space, probably profinite would be good. So if we apply this to the constant sheaf on the abelian group, we get the previous result.

Maybe this all sounds a bit scary, but the point is just the sheaf is defined in terms of cohomology on one side, and the other is basically also just cohomology on the other side. The original claim was that for a certain very specific sheaf, namely a constant sheaf, they have the same cohomology. But actually something much more robust is true: that any sheaf $\mathcal{F}$ which can be represented in terms of such a functor is fully faithful.

Okay, so let me sketch. We need that on any object, the pushforward of $\mathcal{B}$-cohomology sheaves on $X$, the adjunction map from $\mathcal{A}$ to the pushforward of the pullback of $\mathcal{A}$ is an isomorphism. This is a certain map in $\mathcal{D}^+(X)$, and then just a general fact that you can check whether such a thing is an isomorphism by checking it on stalks. This is really the key point where I'm kind of using the topos-theoretic formalism in order to make a reduction to checking something on stalks.

We want that $a_x$ maps isomorphically to---oh, right, we wonder what this maps isomorphically to. Now the key point is that you can actually have suitable base change properties that allow you to pull taking the stalk into this pushforward operation.

The key base change property, taking stalks at $x$, uses the following: this is where you actually use something about the nature of the stalks, namely you use a general cohomology result for taking the stalk of some kind of $\mathcal{F}$ and taking cohomology of some kind of formula. There are certain statements when you can interchange them. Such base change results hold in the generality of what's called coherent topoi.

Coherent topoi is just basically the same thing as being quasi-compact and quasi-separated. So on the end, what you see is that the key thing you really need at this point is that this condensed set corresponding to $X$, in this internal language of topoi, means to be quasi-compact and quasi-separated. This was precisely the thing I mentioned last time: that if you want to do this, for example, for the interval, you need to know that there is a surjection from one of the generating objects, the contractible sets, towards the interval.

So implicitly here you're using that you can cover something like an interval by contractible sets. Sorry, yes, thank you. Why did you use the boundedness assumption? Because otherwise this statement doesn't hold true in general. If you don't have things that are bounded to the left, there's always some issue of how cohomology changes with cosic limits, and there are all sorts of questions.

Basically, there are at least three versions of which way you can do this. There's Lurie's version, which may be the best one, and then there is just the rough category of sheaves, and then there is the left completion. If you want the functor to be fully faithful here, you should always take the left completion.

All of this matters if $X$ is not locally compact, or if it's five-dimensional, then everything's the same. But this kind of general statement, that it always holds for any Grothendieck topos, only works when you're bounded to the left. Otherwise there are some issues. I believe you want to regard the point as a limit of its closed neighborhoods rather than open neighborhoods, yes, exactly.

So if you actually want to execute what I gave here as a hint, then you actually have to use that. I mean, this is cofinal with all open neighborhoods of $x$, but when you're in a compact
Sure, I'll help clarify this transcript. Here's the revised version with added punctuation, capitalization, and paragraph breaks:

You can take the closed neighborhoods here, which has the advantage that if you pull back a closed neighborhood, you actually get some profinite set. You want to stay within the realm of proper, compact, separated objects, for which you need to take a closed neighborhood. But you can just do that.

So, we can pull in---taking the stack $X$, this basically means that we've reduced to the same statement, but now the space $X$ has to just become a point. Let's do this. Okay, but if it's a point, then all these are just abelian groups. Well, this is still something---abelian groups embed fully faithfully in there.

This basically means the integers, as a profinite group, are still projective. So the $X$---yeah, maybe you have to explain, but it's obvious that this $Hom(S, lim M_i)$, with this topology---you have to use that these are in between the open and closed neighborhoods. So if you pull back to open neighborhoods, but then the transition maps, if you have two open neighborhoods and their intersection, then the intersection on the closed neighborhoods sits in between. So it doesn't matter which one you use.

Let me just pause a second and let me try to understand what happened here more completely. So we're interested in computing $X(S, G)$ for some $X$, and we're interested in computing, well, let's just say, in arbitrary degree. Then we would like---now I took a very fancy approach, but we could try to do it more down-to-earth. How would you actually try to compute $X(S, G)$ in abelian groups? You would try to find a projective resolution, and if you cannot find one that's projective, at least you would try to find one that's a free resolution.

This actually breaks the proof in two steps. Step one is to show that if $X$ is actually totally disconnected, then you don't actually have to do anything. So in this case, $X(S, G)$---in this case it's one of these generating objects in our site, it is in this case some profinite set $S$. The continuous maps from $S$ to $G$ are equal to $G(S)$ in degree $0$ and nothing in positive degree.

Okay, so for $S$ being, for example, the profinite completion of the integers, this really just follows from the statement from last time that this is a projective object. But in general this is not true, and for the general case you still have to resolve this $S$. But for---so this comes down to the following:

If there was some, for example, some $H^1(S, G)$, you could always split this $H^1$ after you pass to a cover of $S$ by what it means that it's injective. Similarly, if there's any higher $H^i(S, G)$, you can always split it after passing to some kind of infinite resolution of this $S$. So concretely, what this amounts to is that for hypercovers---so this is a simplicial object in profinite sets---there's $S_0, S_1$, and so on.

Concretely, this means that $S_0$ is a cover of $S$, and then you can recover $S$ as a quotient by this fiber product $S_0 \times_S S_0$. But then the next one $S_1$ is required to surject onto the fiber product, and so on. Here, it takes a little bit of effort in writing.

So whenever you have such a hypercover, there's actually some completely general thing that in any site, if you have an object with a hypercover, then you can build the corresponding chain complex. This is actually always exact, and you need to see that when you dualize and pretend that this should be the answer you would want, if I now pass to the corresponding complex of continuous functions, that this is still exact. This is not automatic, this is what you have to prove.

So we need to show for all hypercovers---the exactness is automatic, what we need to show is that the corresponding complex of continuous maps is also exact. There are several ways to prove this. One is to use the same argument I used, which is to argue that in order to prove this exactness, you can treat everything here as a sheaf on $S$, because instead of taking global

Stalls, and then again when you pass to stalks, you realize the same thing. Where now you're covering a point they have to cover a point, but then again, because it's a point, their cover---there's nothing.

So that's one way. The other way is to prove something, some abstract lemma, that whenever you have a hypercover of profinite sets by profinite sets, you can always write this as some cofiltered limit of hypercovers of finite sets by finite sets. And then this writes this thing as some filtered colimit of corresponding things where everything is a finite set. But then for finite sets also, hypercovers are split and the exactness is automatic.

I think that the latter argument is the one I actually used in the lecture. The first argument is like cohomological descent in SGA 4, and I think that they also check---I forgot in which reference, but in one of the proper base change theorems for separated, proper maps. And then the arguments in cohomological descent go through, and it will give this. The cohomological descent spectral sequence will give you this result. Right?

Yeah, I guess the previous thing on the right was, I guess, it's called Mor. Either that, or you have a colimit, and it actually takes a little bit of unwrapping that you can always do this, but it's okay, to reduce to the case of finite sets where it's obvious.

Okay, so that's the first step if you try to do this more concretely, but maybe also the less interesting step, because we're still just treating these totally disconnected spaces. And now, step two is to treat general spaces.

And so we would again like to find such an acyclic resolution, and now we actually found a lot of acyclics, because now we know that for these connected guys at least this is all right. So now it's enough to find a projective resolution, but one by such joins.

So we want to resolve $\Z$ on $X$, but on $SS^{-1}S$ is lol. And so how does one actually do that? But one uses precisely that---one uses that you can always find the surjection from the constant set, so from some profinite set on to $X$. And then you get some equivalence relation $S_0 \times_{SS} \Z$. And this actually, I mean, if $S_0$ is already totally disconnected, then $S_0 \times S_0$ is, and this is a closed subspace if you take the fiber product. So this is actually always totally disconnected. You don't even have to do any further resolution.

So you can take Čech's nerve, where this is $S_0$, this is $S_{-1}$, this is $S_{-2}$. In general, these $S_i$ are just some $i+1$-fold fiber product. We have some hypercover, in fact Čech cover, of $X$.

And so again, by the general principle that hypercovers give you resolutions, you now get such a resolution of $\Z$ on $X$ by these three guys. This is our resolution by things that are Čech for functions.

And so this tells us that, if you're interested in $R\Gamma(X, \Z)$, this can be computed by taking these guys. And now we know that this guy here is computed by this complex, where you take here the continuous functions from $S_0$ to $\Z$. Which is some really awkward formula for something like singular cohomology.

So you take your nice base $X$, maybe the interval, cover it by some constant set, and then take all the fiber products. And then everywhere you take just the locally constant functions to the integers, and build this cochain complex. And the claim is that this computes $R\Gamma(X, \Z)$. 

Now, it's not so clear a priori that it really computes the right thing. What the previous proof amounts to is to check that, to again treat all of these things here as sheaves on $X$, so as global sections of some sheaves on $X$. And then to check that this computes the right thing, you can again somehow compute on stalks, and then you're done. And then we check that it resolves the constant sheaf.

Sorry, sir
So what are some examples of objects in here? Well, also the $\Z$ are in there, and then I don't know, the real numbers $\R$ are there, or something like the real numbers modulo the integers $\R/\Z$, or the $p$-adic numbers are in there. Also something nice like the adeles, which are the restricted product of the completions of the integers.

One nice thing about the adeles is that they sit in a short exact sequence where you have $\mathbb{Q}\hookrightarrow\mathbb{A}_{\mathbb{Q}}\twoheadrightarrow\mathbb{A}_{\mathbb{Q}}/\mathbb{Q}$, and $\mathbb{A}_{\mathbb{Q}}/\mathbb{Q}$ is compact. So there is some kind of structure theorem for these guys.

Each object can be broken up into three pieces: one piece is the discrete piece, one piece is a finite-dimensional real vector space, and one piece is a compact piece. As you see, there are some kind of interesting exact sequences in there. So you definitely expect that there is some kind of Ext$^1$ group, like $\Ext^1(\R/\Z,\Z)$, which classifies the extensions of real numbers by integers, something like this.

Two things that one knows about this category: it seems to be like an exact category, and computing Yoneda Ext from the category, one knows that all the Ext$^{\ge 2}$ groups are equal to zero for these objects. I don't know about Ext$^1$ groups.

So you can wonder whether something similar holds true now if you compute Ext groups inside the category of condensed abelian groups. Okay, so let's take any two locally compact abelian groups which are also metrizable. Then, again, this metrizable assumption can be ignored; it only comes from the restriction to metrizable condensed abelian groups. Then you can compute the Ext groups as condensed groups from the corresponding guys as topological groups.

I mean, just from the fully faithfulness, you already know that the Hom groups, they can't be changed; they must just be the usual topological homomorphisms. But a priori, there could be some weird Ext$^1$ groups that you didn't know about. Here you're computing Ext groups within the whole category of condensed abelian groups, and there could be some really weird extensions between them. Actually, later, maybe hopefully today, I'll give an example where this actually happens.

But here it turns out that's not the case. All the Ext groups, all the extensions of a locally compact group by a locally compact group in condensed abelian groups, they all are themselves extensions of locally compact abelian groups. You can really identify the abstract Ext group with the usual thing, where we're looking at short exact sequences $B\to X\to A\to 0$, so exact sequences in locally compact abelian groups up to the appropriate notion of isomorphism.

Let me give some key examples. Actually, something slightly better: you could even consider this. So first of all, you can compute Ext groups of anything against the circle group $S^1=\R/\Z$. This is actually $\mathrm{Hom}(-,S^1)$, so this is $\mathrm{Hom}$ as condensed groups, and it's precisely the Pontryagin dual group as a compact group.

Or another example, one could also try to compute some Ext groups like $\Ext^{\bullet}(\R,\Z)$, where some of the intuition is that the real numbers are something connected, so they can never map to the integers which are totally disconnected. So you would expect all these Ext groups to be zero, and indeed, if I remember correctly, that's true.

I think actually these are more or less the key examples. To understand this computation, what you have to do is, you can do a devissage where $A$ and $B$ can be reduced to these basic cases. So for example, if you map from something discrete, then there's not really anything to show, because then the Ext groups are easy to compute. So you can assume basically you have a compact abelian group

Algebraic geometry to do some computations. But the statement we needed was never actually put in by Grauert. There is an unpublished letter of Thom to Grauert, where he proves the result, but it's some unpublished---but it's a very nice theorem. Let me use $\mathbb{B}$ for this.

There is a resolution of the form, something functorial in abelian groups $S$. So what is a resolution? You're trying to resolve any abelian group $M$, and you're trying to find some kind of universal projective resolution of this. They're trying to resolve by free abelian groups, and there's of course a very easy way to at least find a projection onto them. You send some of the generators, given by some element of $M$, $M$ as an element in here, some the finite free abelian group generated by the elements of $M$, right? But of course, this is way, way bigger than this.

So the standard way actually to do this is to do some kind of monadic resolution, where you use a free abelian group monad. And then there is some kind of general thing that the next term---I mean, you might put here, and this would be (it's not what I want to do)---you could put as a free abelian group on the free abelian group on $M$, and then two maps here, and then the difference you use here. And then you can continue, but then these things get uncontrollably large. So this is not what you want to do.

And you realize that actually, you don't need something as big as this to generate the kernel. Because really, the only thing you really have to enforce is that when you add two elements of $M$, then they become the same as the sum, right? So basically, whenever you have a pair of elements of $M$, so here generators are certain pairs $(a,b)$, you can send that to $a+b$, $a+b$. And it's easy to check that this actually generates the kernel of this map, because once you prescribe those relations, then you can uniquely sum any such thing. And you realize you used to just ignore time.

And then you can continue. So each term will just be $\Z$ joined with $\Z^M$ to some power. And there are some transition maps that are given by some universal formulas like this, except nobody's able to write them down.

Do you have to take a finite direct sum or in each term of such powers? Or is it enough to have one power? Like when we originally wrote this up, we used to find some of these things. But you can actually, by some stupid argument, you can basically cover any such finite direct sum again by such a free guy. Okay, I see, I see. You will actually realize that there's a small issue with a zero, but you will figure it out.

You can just choose one term in each degree. And so all the differentials are given as some universal formulas, which actually have functoriality. All the differentials are universal. So it's a little bit of a mathematical result.

And surprisingly, the proof of the theorem uses a little bit of stable homotopy theory. So in some incarnation, it uses something like the finite stable homotopy groups. And that they appear in the proof is also the reason that you don't really know how to do this explicitly. Because at some point, you need to basically kill something like stable homotopy groups, $\pi_n^s(S^k)$, using surjections from finite free groups onto them. And you can do that, but you won't get anything.

But one very nice thing is that this is really functorial. And so this means that this immediately works in any topos. In any topos, you can write down the same complex, the same universal formulas. But whenever you have a sheaf of abelian groups, you can write down the same complex as sheaves of abelian groups on that side, and it will automatically be exact.

So functoriality, or really the universal formulas, also works for sheaves. And so we now get a resolution of our $A$ that we're interested in, where now all the terms are some $\Z[A]$ or $A^{\square n}$ or some such. And so in order to compute $X$ from here, there's some reduction to computing $\text{Ext}$
Only did the case where the target was the $\mathbb{S}^1$ group for the $\mathbb{S}^1$. I also need a case where the target are the real numbers.

Yeah, so if you look back at what I did in these notes on $p$-adics, then there are also proofs that for any compact Hausdorff space $X$, you can compute some of the $X$ groups of $\Z$ (join $X$ into $\R$) as also needed for all $X$ to compute $X$, so from all the $X$ for $X$, but now with real coefficients. And then the claim is that this actually doesn't have any higher cohomology, whatever $X$ is. And $\Z$ of course, it just gets a continuous $\R$.

This also works with $\R$ replaced by any Banach space, but it doesn't work if it's just $\R$. But it really uses local convexity because there are some partition of unity arguments in the proof. You know, for the partition of unity to behave nicely, you need that the target Banach space is locally convex.

Sorry, that's important. So as a preview of something that will happen later: when we consider real vector spaces in $\R$, we will actually have to consider non-locally convex cases. And so we will actually really be interested in situations of such computations where comp, and then we do not have something on all comp spaces. And then this means that we will actually have to resolve. So when we want to resolve by a simplex, we really have to go to totally disconnected.

All right, so let me give an example of how such computation will come out. So when you're trying to compute the $X$ groups of the reals against integers, then $X$ is--no I think that's right, it's a shift from the free on $X$.

All right, if you want to $\R$, you just have to map into $\R$.

Yeah, thanks. So when you computed, when you compute that, then you're resolving this by the $\R$ on $\R$ on $\R$'s and finite vector spaces, and then we know--I mean, finite vector spaces, that's definitely a CW complex. So we know that the $X$ groups, you can see, um, sorry there's some $2i$'s here, that this is, well, it's the same thing as a singular cohomology $\R$. And so this is of course $\Z$ in degree $0$ and $0$ otherwise.

And so this means that when you compute the $X$ groups out of this, then each term will just give you one copy of $\Z$. You would like this to be $0$ for all $i$ greater than $0$. You might be worried that there's some more lots of $\Z$'s remaining, and maybe you don't know what the differentials are. But one way to control this is to observe that you can do a stupid thing of also using this resolution for $\Z/2\Z$, which is also $0$ by several properties of the integers.

And then you realize that when you compute $X$ out of this sequence, it's the same thing as computing $X$ out of here, because each term individually has the same $X$ groups. And so then out of here is the same as out of here here, but this just results $0$.

Okay, so this is how one can leverage this knowledge about the $X$ groups from these free guys on reasonable things into such $X$ groups from locally compact groups.

Let me actually mention the variant of this argument. So one might be worried that it's kind of weird trying to do very explicit computations by using an inexplicit resolution. But some of the inexplicit nature of resolution somehow never becomes an issue. Just the existence of such a resolution, its properties is enough. But actually there is a resolution that is explicit and that can also be used. This is something known as the Eilenberg-MacLane or $Q$-construction, which was rediscovered in the process of this formalization as well.

So this is an explicit complex. It starts just like we expect.

Do this on the other face too. So you take $-a$, $c$, $-b$, $a+b$. I hope you can check that if you compose the two differentials, you get zero. If not, then there is some easy variant of this that should work.

Now you can imagine how you do this one step up. Imagine $\N^8$ is where the eight elements sit on the vertices of a cube. Then you take this side minus this side, and this side plus this side, minus the sum of the sides.

Then there is a theorem that this is kind of linear. More precisely, $Q(M)$ is always quasi-isomorphic to $Q(\Z) \otimes_\Z M$. In particular, if you look at all the homology groups, everything is free. This just means that the homology of this cubic construction is just a linear functor. In general, there's some extra $2$-torsion.

Yeah, this linearity in $M$ turns out to be kind of sufficient. So $X_i(A,B)=0$ for all $i$ if and only if all the $EX$ groups from this explicit complex vanish. Basically, all we are trying to prove is that they can be put into the form that some $EX$ group should vanish for all $i \geq 0$. Then we can use this explicit resolution of $A$. I mean, it's not quite a resolution of $A$, but for this purpose, it's good enough.

Another thing you can actually compute is what this thing is, using some stable homotopy theory. You can show that it is actually $\bigoplus_{i \geq 0} (\Z[S^i])^{\oplus 2^i}$ shifted into degree $-i$. So this is where a little bit of stable homotopy theory comes in. Basically, whenever you write down something explicit, the explicit answer should have something to do with stable homotopy groups of spheres.

Okay, this is just to say that instead of an explicit resolution, you can also do all of these arguments with this explicit complex instead. It doesn't really change any of the arguments, except you have a little bit more comfort that you actually know which objects you are dealing with.

Just to clarify, it's not direct. I think it actually is a direct summand. You think no? Yes, I think the statement is that there are $Q$ and $Q'$, so there is actually a modification that kills something extra, and then you just get this summand. But then you are removing the extra stuff.

All right, so finally I can say one corollary that's really important for the theory of condensed abelian groups. One thing you can now compute is an $EX$ group of something that's not at all locally compact anymore, but actually quite big. You can take the whole product of a countable number of copies of the integers, so the profinite integers. Or if you want, you can take the product of $p$-adic integers.

It turns out that these $EX$ groups are actually just a direct sum of countably many copies of $\Z$. This is very different from the classical answer, where if $EX$ was the naive dual of a countable product of copies of $\mathbb{F}_p$, which is a profinite group, now we are looking at some continuous things, and we just get the direct sum. The really critical thing is that there's nothing in higher degrees.

So this guy will be the compact projective generator of solid $\Z$-modules, which is a full subcategory of condensed abelian groups. All the solid $\Z$-modules should be solid, and then if you want it to be projective, you definitely want all the higher $EX$ groups to vanish.

Okay, so let's prove that. This actually uses a weird trick. Naively, you would now again try to resolve this by free guys on profinite sets, but the issue is that this is a really large thing. Here's a warning: if you treat this here as a condensed set, then this is a union over all functions $f: \N \to \N$ of the product over $n \in \N$ of the interval from $-f$

And so this approach would actually be extremely hard to execute, but there's a trick you can do. You can resolve now in the other direction, which seems weird. You can embed this into a product of copies of the real numbers. This is a similar thing.

At this point, we crucially use that products are exact in condensed mathematics. It might be easier to justify this specific case. Okay, so this is exact.

And now we're trying to compute $X$ against---so this maps the $X$ from here into the $X$ from here, and the $X$ from here. Let's do this one first. This is actually a compact Hausdorff space that is totally disconnected. It's still profinite, right? So it's a countable product of profinite groups.

And so we know what the $X$ groups are. The $X^0$ is a product of $\Z$ copies. It turns out that this is, well, this is discrete and this is something connected, so there shouldn't be any maps. And then it turns out that in degree one, the computation will show that this is just a direct sum of copies of $\mathbb{Q}/\Z$.

Nothing else. This is what the result of the previous lemma refers to. Then if you write the long exact sequence, you realize that what you need is that the $X^2$ groups from this product of copies of the real numbers is equal to zero for $i \geq 2$.

And now, still in the same kind of situation, we do some kind of huge diagram chase. And now there are two ways to finish the argument. One way is you can observe that if you do a similar operation now with this guy, then each of the terms here becomes a product of intervals. So each of the guys becomes an $H^n$-cube, and actually you can show that also for an $H^n$-cube it behaves like the highest degree---this comparison with singular cohomology is also true for the $H^n$-cube. There's no higher cohomology, so the argument we gave for the real numbers works also for this $H^n$-cube variant.

Oh, there's a slightly different way of arguing using a little bit of adjunctions. The source here is some condensed module, it has a module structure over the real numbers. And so this means that maps from the product into any module $M$ will always be the same thing as the internal $\mathrm{Hom}$ in condensed modules over this condensed ring $\R$, from this guy into $M$, since internal $\mathrm{Hom}$ and external $\mathrm{Hom}$ agree here by this general adjunction.

But now this internal $\mathrm{Hom}$ is already zero by the result for $\R$. And so the whole thing vanishes. So we don't really need to know exactly what this is, it's enough to know that it's some module over the real numbers. And then because the real numbers don't map into any module was real analytic functions.

Okay, okay, so that's one of the key computations that we needed, which kind of gives me---so I have maybe five minutes left. Let me talk about a fun theorem with some set theoretic stuff.

So here's a theorem. All right, let me consider the following conditions, consider the following assertion $(\ast)$:

For all sequential limits $M_0 \to M_1 \to \cdots$ of countable condensed abelian groups, and all possibly non-abelian condensed sets $N$, the $X^i$ groups from the sequential limit of $\underline{\mathrm{Hom}}(M_n, N)$ towards $n$, this is the colimit of $X^i(\underline{\mathrm{Hom}}(M_n, N))$. Sorry, and of course this vanishes at least for $i \geq 2$, because these are just $X^i$ groups and condensed abelian groups, and $X^i$ groups in condensed abelian groups agree for $i \leq 1$.

So they vanish at least for $i \geq 2$. And then the colimit also, it's actually equivalent to the following statement: that if I take the $X^i$ group of the product 

So first of all, it's not just that this is excluded by $\star$. In fact, $\star$ implies that the continuum must be really large; $2^{\aleph_0}$ must be bigger than $\aleph_\omega$.

Basically what happens, I think, is that if $2^{\aleph_0}$ is some $\aleph_n$ for some finite $n$, then you will get some $\Ext^n$ problems. But if you make it larger than all $n$, then you have a chance.

In fact, it cannot actually be equal to $\aleph_\omega$ by some GCH. What I proved is that actually the smallest bigger thing is consistent: $\star$ holds and $2^{\aleph_0} = \aleph_{\omega+1}$. You can also---the first possibility after this can be realized.

In fact, it holds whenever you have any ground model. Then you can extend this ground model by doing a Cohen forcing; it holds in the forcing extension.

By joining $\aleph_\omega$-many reals, Cohen invented this notion of forcing that takes one model of set theory and builds another one, a bigger one, in order to show that the continuum hypothesis may be false. This is like the most basic forcing. I mean, now they have billions of different types of forcing, but this is still the most basic one, where you're just adjoining new real numbers, so to say, to your model. And you're adjoining quite a lot of them; $\aleph_\omega$-many would be the minimal thing that you can do in order to have a chance.

$\aleph_\omega$-many, but once you join that many Cohen reals to your model, the simplest kind of forcing is done, which will ensure that this is true. It turns out, always.

Okay, and why do I mention this? Well, we don't actually ever in this course---I will never use this principle $\star$. But it's kind of neat to know that you can use it. Often when you try to compute certain things, it's easy to figure out what the answer would be if this was true, and then the things you really need, you can usually prove them without invoking this general principle.

But there are also some situations where you might want it. For example, in order to control $\Ext$ groups out of Banach spaces, where it really is the case that you get the expected answer under this principle $\star$.

But in general, these $\Ext$ groups are just some... 

Question from audience: Why is it $\aleph_\omega$ and not $\aleph_1$? I mean, as I said in the first lecture, it comes from having the smallest possible .

Another audience question: I have a basic question. When you said that, um, when you computed the $\Ext$ groups for $\mathbb{T}$ locally compact, and then you said that this is zero for $i > 1$, right? You said just because you ...

Lecturer: I don't know, I mean, I'm sorry, I didn't want to refer to--there's no simple reason. You actually have to do a computation. It comes out as zero, but in the end it's nice, it matches what ...

Audience: Thank you. I didn't catch the previous answer, but I want to ask again on this computation of $\Ext$. So for example, if you have a compact abelian group and you take the $\Ext^i$ to the reals, via the condensed formalism, you have a complex that computes it where the terms will be continuous functions on various powers of this compact group to the reals, right? You have to compute that this is acyclic in higher degree. So I don't see exactly how you do it. For the group cohomology complex, you have this averaging by integrals, but I don't know exactly what is ...

Lecturer: I didn't actually look it up in preparation for this. How does it work? So instead of mapping to $\R$, it might also be that you want to ... How does it work? No, so it's different. You use that, um ...
Where's--just go--I mean, I can't think right now, but you can find it in the notes.

I don't see any further questions, so let's--so in the set-theoretical setting, what was the colimit on the blackboard you're now looking at? What is the $X_\theta$? This is the colimit of what?

So he writes this thing as this huge colimit of all functions, and then you can write, indexed by the same index category, where all the terms are now something you can write down. And then it's precisely these $\delta$-limits that they are studying.

Okay, I understand now, because we know that there is a cohomological dimension result for $\mathrm{Alf}^n$. I see how it's related, because if I think there were all $\dim_\mathbb{Q}$ functions--suppose that of eventual dominance--then under, for example, continuum hypothesis, this would be the same as $\omega_1$.

And then such $X$-groups indexed by $\omega_1$ should be bad. But the specific order type of this poset of functions ordered by eventual dominance depends extremely much on the specific model.

\end{unfinished}
% !TeX root = ../AnalyticStacks.tex

\section{\ufs Solid abelian groups (Scholze)}

\url{https://www.youtube.com/watch?v=bdQ-_CZ5tl8&list=PLx5f8IelFRgGmu6gmL-Kf_Rl_6Mm7juZO}
\renewcommand{\yt}[2]{\href{https://www.youtube.com/watch?v=bdQ-_CZ5tl8&list=PLx5f8IelFRgGmu6gmL-Kf_Rl_6Mm7juZO&t=#1}{#2}}
\vspace{1em}

\begin{unfinished}{0:00}
  Right, so today I want to talk about solid abelian groups. The goal is to isolate a class of intuitively speaking "complete" objects. The point being that in the previous lecture, we've seen that when we work in the category $\Cond(\Ab)$ and start with reasonable examples, then the outcome is also reasonable. But when you instead do some free construction, for example you form some tensor product---I don't know, you take one power series algebra and tensor it with another power series algebra in abelian groups---then ideally speaking, there should be some kind of completeness, where this comes out as complete in both variables. But if you just naively form the standard condensed abelian groups, then the underlying object is still just the algebraic tensor product of these things, which is some nasty indescribable thing. Product-like condensed groups are not so nice.

So the idea is that you would want to define a subclass of complete objects, and then you would like to complete things, and hope to get a reasonable answer instead. Idea being, there should be some notion of completeness for condensed abelian groups. But we definitely also still want that complete objects form as nice a category as condensed abelian groups themselves, they should still be an abelian category.

You definitely want something like the integers to maybe be complete, and maybe the $p$-adic numbers should be complete. But then if you want this to be an abelian category, you also want some kind of extension of $p$-adic numbers by the integers to be complete. But this is a very non-Hausdorff thing. So you definitely can't phrase completeness as meaning convergent sequences have a unique limit. First of all, it's not really possible to say what a convergent sequence is, other than one that already has a specified limit point, in this condensed setting. Abstractly, "convergent sequence" isn't really a notion. But also, even if there was, you couldn't ask that they have unique limits, because you want these non-Hausdorff examples.

Okay, and so back in the day when we were first thinking about this stuff, this was really one of the key questions for us: how to define such a notion of completeness. Originally we wanted one fixed notion, where in particular the real numbers should be complete, the $p$-adic numbers should be complete, maybe all locally compact abelian groups should be complete. And still something where you can form cokernels and it's stable.

In the end, we couldn't directly make that work. But we realized that if we for the moment forget about the real numbers and just want something non-archimedean, then there is something that works quite nicely. So it turns out it's difficult to find a notion where the real numbers are complete, but there is a theory that works well for all non-archimedean fields.

Later we will recover the real numbers, but that's a different story. Today I want to talk about a theory that works well in the non-archimedean context. Originally we did things very differently, so today I actually want to give a presentation of the theory of solid abelian groups that's very different from the one you will find in the first lecture notes. It's a presentation that really only works in this nice way in the condensed setting.

Okay, but in the condensed setting, one can base everything on the following idea: being complete in some sense means any null sequence is summable. One basic nice fact in non-archimedean analysis is that sequences are summable if and only if they go to zero. You can just try to turn that into a definition. It turns out this is actually a definition that makes very good sense. So let me try to indicate how you formalize this idea.

Here's a formalization of the idea. Let me consider this projective object that we had: $\underline\Z^\N$, the constant sequence. We want to say something about convergent sequences, so this should play a role somewhere. Recall that this is an internally projective object. Let me recall what this means.

One way to say what this means is if you look at the internal $Hom$ functor into any condensed abelian group, ...

And what does it do? It takes some life Beilinson group, sends to life Beilinson group. That is, the one that takes any life problem to the internal hom from P. So the underlying, so if we evaluate this at a point, and I just get hom from P2M, but then I've enriched this back into the solid Abelian groups. And then the thing is that this functor is actually exact, but also preserves all limits and colimits. Can you find all that? But actually these are compact objects, right? Okay.

Okay, so this will become important in a second. And now we want to phrase the condition that any null sequence is summable. And one way to do is the following. So let me write something down. I consider the endomorphism F, which is the identity minus the shift map, which is an endomorphism of P. So complete P has a basis generated by some $\text{exp}(n)$, and we make $\text{exp}(n)$. So this is the basic space of null sequences $m_0, m_1$, and so on. And what does F do? If you just translate the formula there, it just sends such a null sequence to a new null sequence, which is $m_0 - m_1, m_1 - m_2$, and so on.

And what would the inverse be? The inverse should send a null sequence here--well, you should be able to recover $m_0$ as $m_0 - m_1$ plus $m_1 - m_2$ plus $m_2 - m_3$ and so on, by telescoping, right? And so if there is an inverse, the inverse has to be S, which takes a null sequence and produces a new sequence where the first term is just the sum of all of them.

So this is one way to phrase the condition that any null sequence in M is summable. It's a very structured way of saying this, because maybe to say that, you would only have to say it on the actual hom, on the internal hom. But it turns out to be much better if you ask the condition on the internal hom.

Okay, and so the goal of today's lecture is to understand this $C(\text{Solid } \text{Ab})$. And in the original approach, proving that this is an Abelian category was actually the last thing one could really prove in this live setting. It's actually the first thing one can prove.

If you consider the subcategory of $\text{Solid } \text{Ab}$, it's actually a reflective subcategory, stable under internal colimits like tensor products, all limits and colimits, even internal homs, and internal exact sequences if you want. This also contains some objects--it contains the integers. And once it contains the integers, everything you can build by limits and colimits and kernels and whatnot will contain all this. It's not everything.

It contains the real algebraic numbers. It does not contain the reals, because of course the reals are just wrong. Okay, I understand. Because there is, yes. So it's definitely current.

And so Jaron, when you introduce later this proposition, it basically tells you that this is an analytic ring structure on the integers here. Okay.

And so previously, I mean, the way we set up the theory previously, it was a bit of an art to construct analytic ring structures, because there were a lot of things you have to check, and none of them were easy to guarantee. And so there were basically only two examples one could construct, which is the solid ring structure on the integers, and then there is this liquid ring structure on the reals and some related rings. Those proofs were pretty hard, and they were really very handcrafted things. But it turns out that because of this internal reproductivity of P, it's now trivial to construct analytic ring structures, because for any endomorphism of P, you can ask such a condition, and all this will come for free.

I mean, yeah, so let's just try to prove that for example, it's stable under kernels. Say you have a map $M \to M'$ where this happens, and then you want to know that for the kernel it also happens. But the internal hom from P is an exact operation, so it just preserves the kernel. So of course the same condition is true for the kernel, the same for the cokernel. Or if you
You have stability under retracts and finite limits automatically in an abelian category. This limit is also clear, and this colimit also, because just the internal Hom has these properties. So it's all for free.

What about internal Hom from $X$ to internal $X$? This looks more difficult. It's also trivial, no? Because internal Hom you can also pull over, right? I mean, just by adjunction. Ah, but then you need only internal Hom on the second... I mean, you need only... Okay, let's try Hom, because maybe I didn't... I just wrote that down on the board and I didn't write in my notes.

Something guaranteed. Okay, so here is stability under... Let's first do the internal Hom, then see whether it works for the internal tensor product. Let's say $M$ is solid and let's say actually any $N$. What's the internal Hom from $N$ to $M$?

Okay, I see it, because you can permute... Write it down. See, so what is this? For this, we have to prove something about this guy theater. But by the Hom-tensor adjunction, that's also the internal Hom from $- \otimes N$ to $M$. But then, if $M$ is solid, you can commute the two Homs and it becomes internal Hom from $N$ into the internal Hom from $-$ to $M$. But then, if you come as $M$ here... So of course, another fun still.

And if you look at internal tensor product, then because internal Hom from $-$ is exact, you also have that the internal Hom from $-$ into the internal will also just come out as the internal tensor product from the completed tensor product. And then again, the internal tensor product from $N$ internal Hom...

Okay, and so... Yeah, so I mean, this is something extremely... that when you just enforce such an "is solid" condition on the internal Hom from $-$, you get these very... Okay, so everything is for free, except possibly that there is any object that satisfies it. But okay, so you can just compute that the internal Hom from $\Z^{(\N)}$ into the integers is just a direct sum of copies of the integers. Because if you have a null sequence in the integers, then it must be eventually zero because they are discrete.

Okay, it takes a little bit to see that it's really internally Hom, but that's true. Again by adjunction, if you want. And yeah, so if you just compute what identity on $\N$ here... So you want to show that any null sequence is summable. But any null sequence is eventually zero, so it's easy to sum it.

Okay, before actually focusing more on this specific example of like completeness condition, let me draw some... It's good to know. So just by some abstract adjoint fun, this means that there... this inclusion here... will typically have a left adjoint, some kind of completion. So fun that takes any abelian group to a solid one. This is what we call solidification. And so, this is characterized by the property that for all solid $N$, what do we have... that for all solid ones, the Hom from $N$ to $M$ is the same thing as the Hom... The completion does make it the right adjoint, as opposed to the left? It's the left adjoint to the inclusion, because I'm using the inclusion. So I should write some kind of fun here, which is inclusion fun. Because this... No no no, my question is that... So but you see that both are stable under all colimits. It also has a right adjoint, which I've never considered. Yes, but there is some... also some small solid sub, I guess.

Okay, but over solid objects... requires need product... product... making the solidification. So concretely, this is just if you have two solid abelian groups and you want to form the tensor product, you first form the tensor product inside the abelian groups and then pass to solidification again. There's a little bit of unraveling to show that this is really some half-tor operation. But the key thing that you need for this is that the class of solid modules is stable under pullback. And...

Okay. Is this required to be symmetric? No.

Symmetric. Let me just... So the existence of solidification is just an instance of trying. Basically, if you want to have a left adjoint, it should commute with all limits. Then, under very mild assumptions, it also exists and they are satisfied here.

For the symmetric monoidal structure, well, we define $I$ to just be the constant sheaf. Then we want the symmetric monoidal structure. So what we want is that for all $M$ and $N$, which are just condensed abelian groups, first we individually solidify them, then tensor them back in the category of condensed abelian groups, and then resolidify.

To check that, you check that the solidification is a left adjoint. So to check that this is nice, you check that there are maps into all... For all $Z$, which is solid, we want $\mathrm{Hom}(M\otimes^{\mathbb{S}}N, Z)$ to agree with $\mathrm{Hom}(M\otimes N, Z)$. On the right, $Z$ is still solid, so it's enough. But then the two $\mathrm{Hom}$'s are the same, so it's also the $\mathrm{Hom}$ from $M\otimes N$ to $Z$. This is not from the solidification, it's from the original one, but it's okay.

So we have a very nice category. It acquires its own enrichment, but now we would like to understand what it actually looks like. Actually, one thing we also need to understand is how to interact with the real numbers. We already said that the real numbers are certainly not solid, but something stronger is true. Namely, the real numbers do not admit any nonzero maps to anything solid. Or equivalently, the solidification of $\R$ is equal to zero.

Okay, so for this, note that $\R_{\mathrm{solid}}$ is a ring, because the real numbers are a ring, and if you solidify it, it stays a ring. For a ring structure, to show that a ring is zero, it's enough to show that $1$ equals $0$. Now you want to make use of the fact that in an abelian group, you can uniquely form sums of $\N$-indexed sequences. There are lots of $\N$-indexed sequences in the real numbers that are not actually summable, so you would expect that using such divergent sequences, one can produce a contradiction.

It actually took Dustin and me considerable effort figuring that out, but just yesterday, Ian and then also Kobe found an argument. Here's their argument. We consider the $\N$-indexed sequence $1, \frac{1}{2}, -\frac{1}{2}, \frac{1}{4}, -\frac{1}{4}, -\frac{1}{4}, -\frac{1}{4},$ and so on. I hope you can guess how it continues. You take $\frac{1}{2^n}$ and take each one $2^n$ times.

This is a sequence $\chi: \N \to \R$. Then in the solidification, we have a map $\chi: \N_{\mathrm{proet}} \to \R_{\mathrm{solid}}$. When you compose further to the solidification, we have this map $\N \to \N_{\mathrm{proet}}$, and then there exists a unique map $\mathfrak{m}: \N_{\mathrm{solid}} \to \R_{\mathrm{solid}}$ filling this diagram, because we are using that solidification forms a localization. So we can uniquely find such a "summable sequence" in the solidification, which intuitively speaking should be the thing starting the sequence with $1$. In particular, if you restrict to the inclusion of $0$ in $\N_{\mathrm{solid}}$, you get a map from $\ast$ to $\R_{\mathrm{solid}}$, an element of $\R$

Solidification is right exact if you have enough projectives. If $M$ is an $R$-module, then you can resolve $M$ by a solid projective resolution. Then, for example, if you solidify this resolution, you see that $R$ solid is zero. By a spectral sequence argument, the vanishing of Ext groups here reduces to the vanishing of Ext groups from the reals. But the Ext groups from the reals, they are $R$-modules. A different way to state this would be to observe that all the internal Ext's of $M$ against something solid will be $R$-modules, but the solidification of $R$ is actually zero, so any $R$-module works.

When you compute the internal Hom Ext's, since you don't have enough projectives, the classical approach is to use injectives. But then they are no longer solid. I don't know if that's solid, but of course if you have enough solid projectives... The internal Ext's I was referring to here, they are the ones in condensed abelian groups.

If you compute them by injective resolution of the right hand side, the solid argument... If you compute them by projective resolution of $M$, then by what we proved before, all the internal Hom's are solid. But if you are obliged to compute them by injective resolution... I think it's okay. This is definitely an $R$-module, right? But it's also solid. Why is it solid? Because internal Hom against something solid is always solid. That's something I previously said. But it doesn't follow immediately...

Okay, so everything on real numbers is completed by a limit argument. But on the $p$-adic numbers and so on, you get them by a limit of discrete things. So everything that's pro-discrete is definitely solid.

So, next goal is to compute the solidification of $\Z$. The idea is that when you solidify $\Z$, you need to adjoin lots of new elements, because $\Z$ has no non-zero divisible subgroups. But then the divisible subgroups are summable. So the sum of the sequence $1, 1/2, 1/3, 1/4, 1/5, \dots$ must now also be in $\Z$ solid, and so on. We expect there to be lots of new elements.

One way to make this precise: Consider the subspace $X \subset \prod_{n \in \Z} [-n,n]$ given by the union over all integers of the products $\prod_{k=-n}^n [-k,k]$. So it's a subspace of the whole product. Then there is a null sequence here, which is given by the sequences that are non-zero in only one term. Explicitly, let $e_n \in X$ be the sequence that is everywhere zero except the $n$-th spot is 1.

Then actually on solidifications, that $e_n$ sequence becomes zero. Actually, I didn't talk about the verification, so I would like to even say that not just on solidifications, but in derived solidifications, this $e_n$ sequence becomes zero. A different way of stating this is to talk about Ext groups against solid objects.

That the solidifications agree means that the $\text{Hom}$'s against any solid object agree. But in fact, all the higher Ext groups agree as well. I cannot read what you wrote, just after the $\P$. Is it the map from which $s_n$ to... And then here you have a sum of those basis vectors $e_n$ that are everywhere zero except in one spot where they are 1.
Let's organize this into sentences and paragraphs:

These $e_n$ are just the basis vectors, which is a null sequence in here because the sum has the topology of pointwise convergence. Can you send $n$ to the sequence where only the $n$-th coordinate is 1 and the rest are 0?

Yes, I think so. We know that this makes $\tilde{K}$ quite big, because the underlying abelian group of $\tilde{K}$ is actually an uncountable abelian group. You sum only along finite sums of these countably many basis vectors, but $\tilde{K}$ is a very large, uncountable thing.

So we've got quite a bit to prove. Let me draw some nice diagrams. Okay, maybe I won't do all the compatibility checks, but let me just draw the diagrams and then I'll leave a little bit of diagram chasing.

So we're trying to make good use of the fact that when you have solid abelian groups, the internal $\operatorname{Hom}$'s behave well. In other words, they commute with solidification.

Okay, so we have this diagram here. And then we have another map which is... Okay, so what are the maps here? Giving a map from $\prod_{n=0}^\infty \Z$ or something to here corresponds to a map from $\Z$ to the internal $\operatorname{Hom}$, meaning it corresponds to a null sequence in the internal $\operatorname{Hom}$. So to give this map, I have to give a null sequence of maps from the direct product $\prod_{n=0}^\infty \Z$ to itself, which are the projections to the coordinates greater than or equal to $n$. In other words, in the first $n$ coordinates it's the zero map, and on the others it's the identity.

And so then what does this composite do? This composite here also corresponds to... Let me first describe this composite here. This corresponds to the sequence of maps which are just projection to one fixed coordinate. Because if I think about the difference between two consecutive maps, only one fixed coordinate remains.

But the thing is, the projection to the $n$-th coordinate actually factors over the integers. So in each case, when I fix one $n$ here, the corresponding map is just projection to the integers and then adding back in. But this means that actually all of these maps that are parameterized here, all of them factor over the subspace $\tilde{T}$. 

So this means that this composite factors over this subspace here. I actually need to use that I take a bounded product, because if the coordinates stayed unbounded, I wouldn't actually get this characterization.

Okay, so what's the idea here? You have the identity endomorphism here, and you write this as a sum of a null sequence of endomorphisms. And the null sequence of endomorphisms is the projections to the individual coordinates. So the identity endomorphism here is the sum of the projections to any coordinate.

But all of these endomorphisms that you sum, they all factor over the subspace $\tilde{T}$. And one other thing I should say is that because the first map that I project is really just the identity, if I project to coordinates greater than or equal to 0, I'm not doing anything. This actually means that if I come back, at the 0 end, this $s$ here is the identity.

And so then I can solidify this diagram. I guess that's how to rewrite it.

Okay, so we have this diagram. Well, now this here is an isomorphism, because $F$ solidifies to an isomorphism just by definition of what solid means. And solidification is symmetric, so both of these maps become isomorphisms.

But now this means that this map actually is a split surjection, because you can find an inverse by first going here, then going here, and then going here. So this is actually a split surjection, which is almost what we wanted to prove. It's actually an isomorphism.

To show that it's an isomorphism, you have to show that when you circle around this diagram, you get the identity. And this is another diagram chase that maybe I won't do, it's not difficult. Yeah, so all the maps are isomorphisms.

And so this means that this actually becomes an isomorphism too. I claim something slightly stronger, namely the agreement

So there's a derived solidification functor. And then the same argument is true on solid modules. This $T$ that you solidified is $p$-completely solidified. A ring object now, it will be, but it's not obvious. Oh no, sorry, I think $T$ itself is already a ring object. It doesn't have a unit, does it? Well, just zero, if you want. I mean, I'm not sure. Wouldn't the identity be the constant sequence 1? Which is not, I'm not sure how you get a ring structure component-wise.

I guess, in fact, once we're here, we can actually compute the solidification with another example. So that if we take $S^1$ product, then the whole product, but I mean, this is already solid. We don't need to solidify. And it's also a condensed ring, okay.

And actually, this is a case where I do need a little bit of this $\varprojlim^1$ business. Because it's actually, so we have this sequence here, $\varprojlim S^1$ is pretty large, and then we have this funny $\varprojlim^1 S^1$, which is some weird non-separated guy, but whatever. And so what we have to see is that $\varprojlim^1$ for all $i$ equal to zero, and then this guy doesn't have any $\varprojlim^1$.

Now, of course, I've carefully planned my lecture. So now we know already one class of examples where the solidification exists, which is anything that admits a module structure with a real number. So the claim is that this guy is a condensed ring, which seems surprising at first.

But so why? Actually, there's a different way to write this. It's also the same thing as a bounded product of copies of the real numbers modulo unbounded products, where I define the unbounded product the same way, as an increasing union. And why is that? Because the difference between these two notions, mapping $A$ to $B$ to $C$ to $D$, is the same as first taking ratios here and ratios here.

So what is the ratio here, or $\varprojlim^1$, whatever. Here the ratio between these two is, of course, just the product of $z_i$. But the $\varprojlim^1$ here is also the same thing. Because if you want to surject onto a product of circles, then of course you can keep the projection, can keep them $z_i$ and $1$. So this definitely surjects onto here, but then the kernel is obviously just $B$ sequences of $z_i$. So if you want, you can draw some kind of three-diagram of short sequences justifying this. Like a short sequence here, short sequence here, and then they give you a snake lemma.

Right, and so this one is visibly a condensed ring. Apart from this discussion, the upshot is that the completion of $\Z$ is just the whole product, and the $\varprojlim^1$ is zero. So then, now where here on the right I'm taking the ext groups, all the ext groups I'm taking are currently still taking in condensed abelian groups.

Noting that, I mean, of course this is zero for $i$ greater than zero. Because, so in particular, we recover that $\Ext^i$ in condensed abelian groups from this product of copies $\Z$ to say, $\R/\Z$, which is a discrete abelian group, is just a direct sum of copies of $\Z$ and $\Z$ for $i$ positive and zero. So this is something that I also proved at the very end of the last lecture. And actually, I imagined it would be an input into today's lecture, but now actually one can present the argument so that it's a corollary here.

Did you use the solid analytic feature to show that $\Ext^1$ vanishes?" Yes, I did. Well, I mean, they're also in condensed abelian groups, but yes, I definitely used a different argument. Also, when you actually want to justify that this map here is surjective, I mean, it's actually, if you think in terms of

Why? I mean, this is just given by... Think that, but $p_T$ of $P$, you can take this to $P$. And basically, there's a free guy on, like, an $n \times n$ grid of elements that all jointly converge to zero. But then, well, you can just enumerate... Same, it's just, here is some kind of...

So, in particular, like, phrased in terms of... One way to think about power series algebras, in particular, two power series algebras together, and you get the power series algebra. And maybe up there I should have noted, in some cases, the product of $n$ is some compact projective object. No, I'm sorry.

Alright, so we understand quite a bit. We don't yet necessarily understand all the objects and groups, because there are other generators in life groups: $\Z$ join assets fin. And so we could wonder what their solidification is, but this can also be understood.

And then the... exist from the free... toward using solidification and on $X$ solidification. So you see that also these solidifications of these guys will be a product of... And finally, any assumption on $S$? Like, $S$ should be non-empty or infinite, or you should... Yes, yes. Yes, infinite.

Canonically, one way to write a morphism is that if you write $S$ as a limit of compact sets $S_n$ with surjective transition maps, this will actually just be the limit of these... These are all find of free being in groups, the... the transition maps. So it's easy to see that any such limit would be isomorphic to a product of copies of $\Z$.

So, in our previous way of setting up solid series, we were actually taking this formula as the starting point for defining what a solid guy is. So we were just, on all the generators of our category, all these $\Z$ join $S$'s for profinite $S$, we were by hand declaring what the solidification should be. We were defining it to this limit, and then we were checking by hand that this gives a valid pretheory. But this argument is actually much harder than the way I've set it up now. In some sense, the definition of what a solid group is, is much clearer, and then this really becomes a computation.

Okay, so let me quickly give a sketch. So we have our $S$ and it's written as this limit of finite sets. I assume that all the transition maps are surjective and I inductively choose... Well, first the section $i_\Z$ from $S$ second $\Z$ back into $S$. Then, on all elements of $S_1$ that are not in the image of $i_\Z$, you pick the first playing on on $S_0$. And then if you project back down to $S_1$, in particular, you've already some elements of $S_1$ which you've already lifted to $S$, but then there's new elements of $S_1$ that I didn't yet have. And so on, then I pick $p_i$ new $S$ sometimes on $S_1$ minus the image, and so then jointly these two things together now define for me a section from $S_1$, which on some elements is given by projecting to $S_0$ and then taking the lift you already have there. And the other elements, you make a new choice. And then you continue.

Okay, so this way, what in particular you will get is a countable sequence of elements in $S$, right? Because the union of all these maps is just a countable subset of this, but it will certainly be dense.

And so, let's also enumerate the elements. So, enumerate $n$ as you start enumerating at zero, then start enumerating the elements of $S_1$ that weren't in $S_0$, one and so on. And we get a map $d$ from $\N$, which is some of the free guy on this copy of $\N$, towards $S$. Right, on $S_\Z$ this is given by $i_\Z$, but on some higher $S_{n+1}$ it's given about the difference, the next in

Canonically, it should be this thing. In order to prod such an isomorphism, I have to produce such an isomorphism. If you carefully think about how you would actually go about doing that, you have to choose a new base of elements. Then it is precisely what you want to end up to.

The argument is actually very similar to the argument we already did. We have our carefully here, and then sectors, where again, giving such a map means I give a pro-sequence of maps from S to S. The pro-sequence I consider here is a sequence of $S \to S_{\leq n}$. So you start with the identity, which we have to do because we want this splitting here, and then take $I$ project to the $\leq n+1$ and take the splitting $\pi_n$.

Okay, so when I write these maps, I need the maps induced on quotients. It turns out that because some of these maps approximate the identity here, the same as $S_{\leq n}$, and so then this guy here, this corresponds to the sequence of the differences $\pi_{n+1} - \pi_n$. If you think about what these differences are actually doing, you realize that they are only changing something on a small part. I would mess it up if I try to say it orally, but you can check that you've exactly crafted things so that this difference will factor over the image of this $S_{\leq n}$.

The idea here is again that you take the identity here and try to write it as an infinite sum of maps, all of which factor over the submodule. For this, you use the sequence of functions $\Z_S$ which are factors of something of finite range $\Z_S$, and then you can do it. The argument is just the same as before.

What becomes critical here is that you really have a pro-finitely presented $I$. If I were to set up the theory of solid abelian groups back in alter groups, then the condition I gave that just talk about pro-sequences wouldn't be enough, because you will never see this $I$ as a quotient of the integers. You're really using that you can still understand this via its finitely presented quotients.

Okay, so now we have that $\mathrm{Solid}(\mathrm{Ab})_{\mathrm{ST}} = \mathrm{Coh}(\mathcal{O}_S^\mathrm{a.cg})$. In particular, this includes that it has a compact projective generator, a single compact object which is a full subcategory of products $\Z_S$, and it's actually internally projective. In fact, if you tensor it with itself, it becomes isomorphic to itself.

Also, in the solid case, being solid is equivalent to only having the $\mathrm{Hom}$-finite condition hold, which is what we took as our original definition. It had only $\mathrm{Hom}(\Z_S,-)$ preserve filtered colimits, and all maps $\phi : \Z_S \to M$ have a unique extension to $\Z_S^{\mathrm{a.cg}}$ from the free object, which is the colimit $\varinjlim_n \Z_S$.

And this, and then also because $\phi$ is actually zero. And maybe I didn't say, but you do it, you can do it by taking $M$ to be anything, and then you go from the universal property for $\Z[S]$ to the same one for any $M$, because you represent the solidification of $M$ as $\Z^S$. So the fact that any map from $M$, from other $M$ (I mean $M'$) to $\Z[S]$, factors through the solidification, and then by then is clear that this is the same as $\Z^S$. I think this also represents $M$. Part of it is clear, a direct fact.

So, time, so let me just give one kind of philosophical way of thinking about this solid condition. The very end condition was that all null sequences are summable. And something that we get out of this is that one can always integrate against certain kinds of measures. So one can also think of $\Z[S]$, it's actually the same thing as the continuous functions from $S$ to $\Z$, because the continuous functions, they have a colimit of the functions on $S$ to finite sets. This gives a description.

And so from this perspective, these are some kind of $\Z$-valued measures on $S$. And so this means that whenever you have a map from $S$ to $M$, where $M$ is solid, and whenever you have a measure on $S$, then this induces by what I said here, and this is the measure $M$ to something here. So we can and this to the integral of $f$ against $\mu$. In other words, whenever you have a map from some profinite set into $M$ and some $\Z$-valued measure on $S$, then you can form the integral.

One thing that I should stress here is that this characterization of solid at the end, I only talk about homs, not about internal homs. The first characterization I gave where I talked about differences, I had to talk about the internal homs. Here it's a problem enough.

All right, I'm out of time. Let me check why the composition is zero. Because these are $0$, solidification is also $0$. I'm slightly confused about my diagram, why it's still a presentation of $M$. That's maybe... First, you definitely get a right exact $M$ with a solidification. Solidification is right exact. But then, and you definitely always have this map here, but then the observation that there exists this map back, which is zero here, means that this actually retracts. So $M$ is a retract of its solidification by this argument. But retracts of solid guys are solid. So there is a single compact projective generator.

Right, so I mean we'll discuss this more next time. So one can give some purely synthetic algebraic descriptions of what the category of solid abelian groups is without talking about anything. And so you can say what they are, right? Products of sums of $\Z$, they are just infinite matrices which in every row are eventually zero.

Are you going to give a similar new way to define, a new definition for liquid modules? Or is this new formalism available only for the solid setting? The question is whether one can also characterize the liquid vector spaces in a similar way by mapping. I think it should be possible, and we're currently figuring out the details of what works. Let's see, we still have a few weeks.

\end{unfinished}
% !TeX root = AnalyticStacks.tex

\section{\ufs Complements on solid modules (Scholze)}

\url{https://www.youtube.com/watch?v=KKzt6C9ggWA&list=PLx5f8IelFRgGmu6gmL-Kf_Rl_6Mm7juZO}
\renewcommand{\yt}[2]{\href{https://www.youtube.com/watch?v=KKzt6C9ggWA&list=PLx5f8IelFRgGmu6gmL-Kf_Rl_6Mm7juZO&t=#1}{#2}}
\vspace{1em}

\begin{unfinished}{0:00}
e  so  today  I  want  to  uh  before  Dustin
takes  over  on  Wednesday  again  uh  today  I
want  to  give  some  some  compliments  on
solid
modules  sorry
being  so  just  briefly  recall  the
definition  so  with  P  has  always  denoted
the  subjective  object  which  is  as  a  fre
guy  on  the  nor
sequence  and  uh  then  we  consider  this
anamorphism  of  P  which  is  the  identity
minus  the  shift
n  which  takes  the  generator  n  to  n+  One
n  and
then
the  new  definition  of  a  solid  module  was
thatal  to  the  old  definition  but  on  prly
so  solid
is  Ed  by  F  on  interal
home
morphism  where  intuitively  speaking  this
is  a  map  I  mean  M  from  P  to  n  some  of
the  space  of  n  sequences  in
N  so  you  take  a  n  sequence  in  n  and  to
the  sequence  of
differential  and  intuitively  speaking
that  this  is  an  isomorphism  means  that
conversely  if  you  have  any  more  sequence
than  M  then  you  can  sum  it  because  if
you  think  about  it  then  this  m  Z  must  be
the  sum  of
all  so  so  this  looks  like  some
definition  of  derived
complete  modules  like  periodically
derived  complete  so  I  suppose  there  is
some  mathematical  relation  that  derived
complete  let  let  me  talk  a  little  bit
about  uh  little  bit  of
yeah  uh
I  mean  I  think  on  the  face  of  it  it's
not  related  but  okay  so  there  will  be
some  dve  complet  this  today  um  and  so
then  the  theorem  we  proved  last  time  was
that  uh  this  PO  guys  and  try  all
good
be  stable  under  all
fourel
extensions  let  me  just  say  limits  or
coits
um  tend
product
product  making  the  left
joint  so  this  inclusion  L
join
um  oh
solidification  um  making  this  left  joint
symmetri  mon  meaning  Comm
product  and  uh  it  is  is
generated  by  a  single  compact  projective
object
um  this  is  a
product
and  on  this
generator
uh  the  Sol  tender
product  is
justes
um  and  maybe  the  other  thing  I  should
say  is  that  one  can  compute  on  the
solidification  of  all  these  three
modules  on  a  Prof  finite  set  s  they  are
just  the
limit  of  the  three  modules  on  the  finite
set
and  okay  so  I  think  this  basically
summarizes  starting  from  last
time  and
uh
maybe  okay  what's  the  first  thing  maybe
the  first
thing  okay  then  then  let  me  first  uh
discuss  the  derived  enhancement  of  the
situation
um
um  so  let's  let's  assume  that  instead  of
uh
just  just  one  Con  in  group  you  really
have  not
Tak  then  actually  the  definition  still
makes
sense  and  we  take  that  as  a  definition
so  we  say  that  a  complex  of  like  which
is  solid
if
again  internal
alism  but  again  because  the  internal  H
from  p  is  actually  exact  this  actually
equivalent  to  asking
uh  that  for  all  I  in
the  the  internal
home  from
theology  so  there's  equivalent  to  asking
that  uh  all  the  homologies  of  a
also
and  again  this  is  also  true  uh  in  our
previous  discussion  of  s  being  groups
and  is  in  fact  a  general  property  for  we
call  an  analytic  R  structure  that  you
can  check  completeness  of  the  level  ofy
groups  classically  this  is  a  little
subtle  to  prove  but  here  again  it's  an
immediate  consequence  of  the  internal
projectivity  of
key
okay  so
uh  then  some  of  cor  area  of  this  C  said
uh
um  the  class  of  sub  solid  a  is  itself  a
triangulated
subcategory  or  if  you  would  work  in  the
infinity  setting  it  would  be  a
stable  inity
uh
and  stable  under  all
direct  and
so  all  liit  and
liit
uh
and  and  also  one  can  again  show  it's
also
stable  under  our
home  uh  and  again  from  same  proof  as
last  time  and  then  if  you  just  repeat
this  discussion  of  tender  products  now
at  the  level  of  deriv  categories  uh  you
actually  see
that
um  oh  actually  it's  also  true  that  uh
so  I  did  give  a  notation  to  the
subcategory  uh  and  I  didn't  want  to  give
itation  because  in  the  end  it  just  turns
out  that  if  you  take  the  drive  category
of  solid  guys  um  then  drive  C  are
functorial  so  you  get  a
map
this  is  actually  F
faceful  and  the  essential
image  ex  a  solid
base  this  is  actually  something  that's
not  completely  automatic  this  uses  that
um  these  project  generators  here  that
they
stay  they  some  have  exp  banishing  also
in  light  condens  being  groups  Orly
that's  the  solidification  actually  here
is
okay
um
uh  so  is  it  the  case  that  if  you  have
any  complex  possibly  unbounded  whose
terms  are  direct  sums  of  those
generators  these  generators  and  the  the
taking  solidification  turn  by  ter  is  is
a  good  arve  that  that  gives  you  the
yes  yes  that's
right
okay  I  yeah  solidification  preserves
like  on  the  level
of  for  more  maybe  forget  about
uh  the  thing  so  then  you  have  this  whole
sub  category  in  the  dra  of  see  solid
objects  and  again  by  thej  function  you
will  actually  incl  will  have  left  joint
some  kind  of  dve  solidification
fun  uh  which  again  will  commute  with  all
Co  elements  back  into  because  this  is
still  under  alls  and  then  this  and  on
the  generator  we  kind  of  determined  what
it  does  last  time  that  it  comes  this
thing  even  in  the  d  s  uh  so
yeah  um
okay  so  uh  also  also  on  the  of  D
categories  we  have  a  full  inclusion
really  um  so  it's  as  nice  as  it  could  be
and  uh  and  again  by  some  my  joint  fun
serum  uh  this  has  a  left
joint  hey  nothing  to  let  me  write  this  a
deriv  solidification
L
box  so  in  the  joint  Factor  theorem
usually  there  is  some  set
theoretical  condition  because  you  want
thatly  I'm  thinking  of  everything  as
Infinity  categories  and  then  everything
here  is  this  presentable  Infinity
category  and  then  L  has  this  general
joint  fun  serum  there  I  think  there  are
for  tri  categories  there  are  Ser  of  nean
and  others  which  would  also  to  justify
this  I  believe  uh  but  I  mean  they  often
assume  actual  compact  generation  which
is  not  quite  true  in  this  Cas  so  I'm  not
sure  but  L  version  is  definitely
sufficient  okay  so  so  for  B  for  would
need  the  abstract  Z  because  we  can  just
prove  the  existence  by  hand  in  this  case
because  on  on  a  class  of  generators  we
can  we  have  identified  that  the  exists
these  guys  already  know  that  they  exist
but  the  abstract  theorem  is  both  for
Left  Eye  joint  and  right  eye  joints
assuming  that  there  are  enough  limits
and  Co  limits  right  I  mean  so  there  one
of  presentable  Infinity  category  which
is  one  which  has  all  Co  limits  and  which
is  generated  by  the  set  of  objects  um
and  in  this  situation  every  functor  that
preserves  all  col  is  R
joint  and  every  functor  that  preserves
all  limits  and  all  sufficiently  filtered
full
limits  also
all  sufficiently  filtered  coits  so  it's
accessible  one
also
um  uh  and  uh  a
unique
uh  symmetric
product  prodct
um
making  uh  a  goes  to
R  and  actually  something  even  better  to
that  this  is  literally  the  derived  left
derived  fun  of  the  thing  you  have  being
categories  uh  and  similarly  you  can  also
I  mean  the  drive  tender  produ  is  also
obtained  by
driving  the  thing  you  already
havean
level  but  here  how  do  you  know  that  the
solid  category  is  presentable  in
your  before  you  you  prove  the
existence
um  yeah  I  think  there  are  General  theor
that  if  you  just  ask  locality  for  such  a
map  there  are  some  automatic  presented
work
okay
so  uh  so  once  we  pass  to  the  dri  t  i  can
St  things  from  last  time  that  really  the
deriv  Sol  application  of  z  s  is  still
limit  as  Z
end  so  it  is  uh  the  other  solation  there
are  no  higher
homies  and  uh  so  also  the  Der
solidification  of  p  is  this
perect
and  the  Der  C  product  of  s  to  copy  is
the
under  okay  so  you  might  think
that  nothing  arried  it  really  ever
appears  in  practice  but  not  so  um  here's
the  very  F
proposition  uh  let's  say  again  that  uh
for  simp  the  xcw
complex
so  recall  that  in  this  situation  we
previously  had  this  thing  that  the  X
groups  against
the  they  are  the  singular
Gres
right
um  but  now  we  can  also  interpret  the
homology  of  X  and  it  turns  out
that  a  singular  homology  of
x  is  exactly  the
iology  of  the  the
solidification  more
precisely  um  I  mean  there's  some  of  to
complex  Computing  singular
modin  and  this  is  really  isomorphic
to  the
rication  in  particular  the  RO  ification
is
actually  comes  from  the  complex  of  the
street  being
groups
so  again  this  is  some  kind  of  version  of
the  if  you
take  this  three  group  on  X  and  this
models  so  as  example  like  if  you  take
the  three  intervals  is  some
large  thing  right  and  the  underlying
thing  is  finite  sums
of  points  but  if  you  solidify  them  the
interval  just  goes  away  it  just  becomes
Z  and  but  if  you  take  the  circle  and
solidify  that  that  becomes  a  copy  of  Z
degre  zero  and  a  copy  of  zg
one  and  in  general  if  you  take  uh
this  is  Union  of  circles  of  all
dimensions  and  you  get  something  which
is  concentrated  degree  zero  but
solidification  goes  arbitrarily  in  part
to  the
left
um  right  and
so  uh  again  there's  the  formal  reduction
to  the  case  of
complex  so  justs
assuming  I  keep
theor
which  um  and  so  for  financy  w
complex  uh  how  do  we  actually  comput
this  so  so  assume
comp  and  also  it's  to  fin  complex  we
know  this  homology  is
finite  generate  eache  okay
so  I  I  may  miss  the  point  but  you  said
that  the  direct  solidification  with  GX
under  bar  is  a  singular  uh  complex  right
so  it  is  a
Qing  it  is  one  so  in  above  this
statement  you  said  that  say  yeah  sorry
it's  just  a  qu  isomorphism  ah  qu
isomorphism  sorry  I  see  sorry  not  the
yeah  sorry  it's  only
find  yeah  yeah  yeah  yeah  that  do  makes
sense  sorry  sorry  sorry  I  see  I  see
sorry
um  okay  so  how  do  we  compute  the  the
appication  so  uh  so  we  do  the  use  those
things  so  we  take  some  subjection  from
some  life  profile  set  onto  X
so  like  the  cont
said  and
then  uh  we  get  all  fber  products  and
this  usess  resolution  of  this  condenser
being  here  by  the  guy  on  S  and
then  uh  the  guy  on  the  F
product  which  is  again  some  light
profess  and  then  and  all  the  fiber
products
okay  so  that's  the  resolution  of  of  this
guy  and  so  to  form  the  d  uh
solidification  we  can  somly  the  D
solidification  to  all  the  terms  here  but
for  these  terms  we  know  that  it's
concentrated  degree  zero  uh  so  we  know
that  the  derived
solidification  is
computed
by  uh  this
let  me  not
underline
okay  now  this  see  might  seem  a  bit  hard
to  compute  uh  but  we  know  that  each  term
here  is  actually  just  internal
home  from  the  continuous  functions  from
s  into  z  z
right  because  uh  that's  something  I
discussed  at  the  end  of  the  last  lecture
I  mean  this  formula  that  this  limit  of
the  SN  can  be  uh  translate  into  this
formula  so  these  are  some  kind  of
measures  on
X  and  similarly  for  the
other
so  this  actually  means  that  all  the
terms  all  the  terms  in  this  complex  are
isomorphic  to  their  double  dual  right  so
the  continuous  functions  from  s  to  Z  is
besides  the  Duel  of  sorry  the  Dual  of  of
this  one  and  then  I  dualize
again
uhor  that's  solidified  is  actually
just
home
and  I  I  could  also  put  put  the  r  here  it
doesn't  change  high  in  those  cases  um
and  so  this  actually  means  that  uh  the
same  formula  will  be  true  for  this  guy
if  you  think  about  it  so
because  resoltion  so  this  means  that
also  this  gra  solidification  here  one
way  to  compute  it  is  that  it  un
isomorphically  to  the  r
form  R
form
X
but  now  we're  in  business  because  we
know  that  this  guy  here  is
just  some  isomorphic  to  quic  to  singular
Co  chain
complex  and  uh  because  we  know  that  the
singular  chology  for  fin  CW  complex  is
fin  in  each  degree  uh  it's  some  taking
the  Dual  of  singular  corology  against
singular
homology
okay  and  so  you  get  the
qu  the  Der  solidification  of  X  and
singular
and  so  this  means  that  to  some  extent
this  passage  from  light  condens  being
groups  or  solid  being  groups  uh  is  like
passing  to  homotopy  types  A  little  bit
like  at  least  for  CW  complexes  it's  some
contracts  the  interval  and  then  some
identifies  two  homotopy  equivalen  W
complexes  but  on  like  totally
disconnected  things  it's  much  finer
information  or
it's  yeah  you  could  still  invert
homotopy  equivalence  but  not  invert  V
homotopy
equivalence  so  is  it  the  case  for
locally  contractable  spaces  in  the  sense
that  you  explained  yes  yes  so  whatever  I
said  about  CW  complex  also  works  for  a
locally  contractable  session  actually
uhan  pointed  out  to  me  that  uh  the  par
prce  assumptions  actually  not  required
in  comparing  singular  chology  and  chief
chology
no  no  they  are  not  required  there  are
some  recent  papers  Su  that  effect
um  but  I  don't  care  some  sense  right  now
so  right  uh
so  I  know  it's  fun  it's  probably  not
really  relevant  right  now  but  F  to  know
um  okay
so  now  let  me  go  back  and  try  to
describe  the  the  category  of  solid  a
little  better
um
and  um  so
yeah
so
uh  we  want  understand  the
structure  Al  Sol  being
better  so  so  there  is  a  comp  projective
generator  so  there's  definitely  a  notion
of  find  generated
objects
and  these  are  precisely  the
Cs  of  a
disas  and  then  there's  a  notion  of  fin
presented
object  uh  these  are
uh  the  coreel  of
map  and  it  turns  out  that  so  the  modules
behave  like  modules  in  the  following
sense
uh  that's  a  presented  objects  form
itself  in  a
category
and  like  the  only  critical  thing  is
stability  under  kernels  but  once  you  has
that  it's  automatically  stay  under
kernels  Co
kernel  and  really  really  only  the
kernel  are  something  to  mention
um  and  the  whole  category  is  just  the
end  category  of
those
also  any  finally  presented  object
actually  AIC  to  cor  of  injective
Meth
this  actually  something  slightly  better
than  what  DUS  announced  in  the  first
lecture
uh  he  claimed  that  the  science  presented
objects  has  a  resolution  of  length  two
and  uh  but  actually  length  one  is  good
enough
any
and  there's  a
resolution  just
stop
okay  so  there
some  the  finite  objects  in  your  C  which
are  exactly  the  of  injective  maps
and  then  everything  is  a
f
um  I  the  key  spe  for
this  is  the  following  that  uh  so  trying
to  understand  all  the  five  presentive
ones  then  okay  you  have  a  kernel  you  can
always  find  the  subjection  onto  M  and
then  there's  some  kernel  and  the  kernel
you  still  know  is  at  least  a  finally
generated  subm  module  of  this  and  so
then  it  would  be  good  to  know  that  all
the  fin  generated  sub  modules  of  this
are  actually  such  a  product  and  this  is
what  I  claim
now
zero  no  sorry
sorry  let's  say  product  of  copies  of
Z  generically  it  be  product  course  could
be  zero
fin
um  so  this  definitely  implies  this
statement  here  but  it  also  implies  like
okay  so  object  are  always  St  on  the  C
and  extensions  um  things  that  they're
stable  under
kernels
uh  but  then  I  mean  this  also  always  for
reduces  to  identifying  the  finally
generated  sub  modules  of  like  you're
generating  objects  and  if  those  are  all
finally  presented  then  then  you're  in
good  shape  and  this
some
so  the  Ser  is  really  easy  to  deduce
from  okay  so  let's  proof
Sy
so  we  know  that  uh  it's  finally
generated  so  we  know  that  there  is  some
subjective  map  from  product  of  copies  of
Z  onto  M  which  injects  back  into
of  by  the  way  uh  this  is  again  the
result
it's  previously  a  lot  of  what  I  was
talking  about  about  the  general
properties  of  the  car  Sol  being  groups
they  basically  extend  to  the  full
condensed  setting  not  just  the  line  one
this  Serum  is  again  one  which  only  holds
in  the  L
setting
um  and  be  because  we're  again  using  some
countability  in  just  a  second  um  okay  so
so  we  have  some  here  let's  go  g
um  and  we  know  that  the
uh  the  MS  from  a  product  back  to  Z  uh
they  are  given  by  direct  sum  so  we
actually  know  that  this  is  dual  to  a
m
uh  in  the  other  direction  from  a  direct
sum  of
these  and  basically  our  task  is  to  show
that  whenever  we  have  such  a  such  a
anamorphism  of  direct  sum  of  these  and
when  we  dualize  then  this  image  here  is
itself  a  prod
right
so  for  the  proof  I  will  use  the
following  fact  which  I'm  not  sure  if
it's
known  uh  that  well  known
uh  uh
let  N  accountable  grou
um  that  into
AAL  I  mean  it's  just  an  abstract  yeah
discret  no  condens  that  EDS
into
uh  Z
ISE  okay
maybe
um
uh  okay
um  let  me  leave  this  as  an  exercise
um  I  no  is  definitely  fult  if  you  drop
the  accountability  assumption  because  if
you  take  an  actra  prodct  of  copies  of  an
AB  being  group  it's  definitely  not  a
free  being
group  um  but  one  thing  this  implies  is
that  if  I  look  at  the  image  of
AG  uh  then  this  is  definitely  countable
and  it  embeds  in  the  direct  sum  do  also
into
D  implies  that  the  image  of
H
three  for  this  you  don't  need  this  it's
just
image  okay  I  will  use  a  spe  again  in  a
second  so  I  want  mention  it  way  to
justify  this  to  use  f  but  okay  may  it's
easier  uh  so  the  image  is  free  um  but
then  you  have  a  subjection  here  from
this  direct  sum  onto  another  free  group
so  this  splits  in  particular  like
there's  there's  a  kernel
here  but  this  must  actually  split  back
so  the  of
H  uh  split  as  a
summon  I  mean  maybe  actually  this
completely  irrelevant  maybe  I  should
focus  more  on  the  potion  but  let  me  just
try  to  understand  a  little  bit  about  the
structure  of  such  such  snaps  and  what  it
implies  on  the  Dual  side
um  uh  so  we  kind  of  split  but  this  means
that  uh  you  can  basically  replace  this
by  the  other  direct  factor  which  is  also
freeb
and  replace  H  by  the  corresponding  thing
because  then  this  corresponding  this
prodct  will  split  into  two  two  copies
and  this  n  will  just  embed  into  one  of
them  uh
so  we  can  definitely  assume  that  H  is
indexes  um  Okay  so  so  now  we  have  an
injection  I'm  going  direct  some  of
these  and  now  we  have  some  qution
here  and  now  the  next  thing  is  to
understand  structure  of  the
potion
you
um  so
uh  there  is  a  certain  quent  of  Q  which
is  the  image  when  you  embed  it  into
double  um
sorry  let  me  just  write  as  a  produ  over
all  over  these  over  all  possible  maps
from  two  into
V  so  you  can  look  at  all  possible  maps
that  Q  maps  to  a  fre  group  or  just  to  to
Z  and  takes  the  corresponding  maps  to  Z
and  then  there  will  be  some  quo  here  uh
and  then  by  the
fact  uh  this  guy  is  actually
free  okay  and  so  this  means  fre
or  particular  projector  splits  at  a
direct
summon  and
so  uh  and  also  if  you  look  at  the  teral
of  this  m  um  then  it's  easy  to  see  from
this  that  this  Q  Prime  will  have  no  more
maps  to  Z  right  because  if  this  had  a
map  to  Z  then  because  this  is  a  direct
sum  there's  another  map  from  q  q  to  Z
which  would  make
Q  okay  so  this  means  that
uh  this  has  this  potion  Q  Bar  which  is
free  and  so  this  splits  back  here  but
then  also  back  here  if  you  want  and  so
then  you  can  also  get  rid  of  that
Su
uh  okay  so  without  L  of
generality  so  you  can  replace  some  Q  by
Q  Prime  and  then  there  are  no  more  hph
to  Z  those
generality
um  Q  has  no  more  maps  to
Z  a  different  way  to  think  about  what
this  I'm  doing  here  is  to  observe  that
um  the  category  of  countably  Genera
three  being  groups  has  core  kernels  and
this  Q  Bar  would  be  the  core
kernel
okay
okay  so  where  are  we  in  the  proof  so
we're  trying
to  show  that  whenever  we  have  a  MTH  of
direct  sums  of  Z  then  once  we
dualize  the  image  of  this  map  is  itself
a  product  of  copies  of
Z  and  now  we  split  it  off  the  kernel  we
split  it  off  part  of  the  cor  kernel  and
now  we're  down  to  a  situation  where  we
have  this  map  and  Q  has  more  maps  to  Z
and  we're  trying  to  understand  what
happens  if  we
dualize  okay
so
um  so  we  have  here  product  of  copies  of
Z  cop  of
z  uh  this  map  here  is  our  old
map  d
and
uh  all  right  uh  sorry  theal  this  is
precisely
uh  the  internal
home  from  Q  to
Z  and  then  the  would  be  actually
X1
um
and  no
but  so  we  ensured  that  the  home  from  Q
to  Z  is  Zer  but  actually  it  turns  out
that  the  interal  home  from  Q  to  Z  is
zero
as  the
points
are  home  from
q  t  z  join  s  into  Z  which  by  using  T
Junction  this  different  order  is  the  hom
from  Q  into  the  continuous
MV  which  is  a  discrete  guy  and  where
this  guy  is  Sol
three  uh  and
so  uh  but  Q  has  no  maps  to  Z  and  just  no
map  any  being  group
and  okay  and
so  so  what  does  this  mean  so  this  means
that  the  image  of  G  is  actually  just
this  product  of  G's  so  some  the
reductions  we  made  in  the  beginning
precisely  ensured  that  actually  our  m  g
became  injected
just
and  okay  I  guess  strictly  speaking  I  did
some  of  these  reductions  it  could  have
happened  that  an  infinite  direct  sum
became  a  finite  direct  sum
but
just  take  the  direct  s  with  an  in  thing
whatever
um  uh
okay  so  that's
good
um
um  right  so  actually  call  a
proof  is  that
the
uh  any  M  which  is
a  p  presented  solid  gu
um  here's  a  direct  fun  or  a
product  copies  of
Z  and  the  group  of  the  form
X1  of  few  against  C  for  some  countable
being  group
q  without  naaps  to
Z
and  so  if  Q  is  actually  a  torsion  group
uh  that  these  are  precisely  is  a  profile
General  there  are  some  kind  of  weird  non
separated
guys  okay  uh
uh  ah  and  here's  another  fun
um  uh  so  this  this  call  is  again
something  that's  only  true  in
the  uh  in  the  live  setting  namely  that
the  product  of  copies  of  V  is  actually
flat  so  next  Che
it's  not  always  true  that  projective
objects  are  um  we  could  check  that  once
we  pass  the  P  numbers  um  something  like
this  holds  true  in  like  without  the  LI
assumption  but  we  were  kind  of  confused
about  the  case  of  solid  being  groups  and
then  but  then  Sim  observed  that  actually
it's  false  uh  in  all  solid  being  groups
the  flatness  but  once  you  restrict  to
the  light  connect  becomes  true
again  which  is  also  kind  of  the  reason
that  we  never  saw  any  obstruction  to
this  in  actual  mathematical  practice
um  and
why  uh  so  we
need  for  all  and
solid  The  Rock  Solid  and  the
productes
the  Sol  T  Ro
T  Co  limits  so
generality  to
prevented  uh  so  we  actually  have  a
resolution  of  length
one
uh  but  now  I  T  this  with  a  product  of
copies  of  Z  then  this  map  stays
injected  this
St  I  mean  it's
just  I  mean  actually
this  actually  what  you  really  see  is
that  if  you  take  the  d  uh  it's  called  T
this  the  prod  of  this  you  actually  just
get  a  product  of  c
of
for
because  so  you  can  just  write  this  as  a
fil  the  fin  presented  guys
and  all  right
uh
so  uh  and  one  thing  that's  really  nice
about  this  whole  tender  product  is  that
uh  really  gives  a  lot  of  I  mean  many
many  computations  come  out  right  um  in
sometimes  non  obvious  ways  just  like
maybe  previously  a  lot  of  EX
computations  came  out  right  now  obious
ways  and  so  maybe  uh  let  me  do  some
tensor
computations
um  so  often
uh  you  maybe  care  about  things
like  you  start  with
some  in  being
gr  and  a  speed  being
group  and  then  you  form  the  per
completion  of  them
and  and  then  in  this  for  in  this  such
situations  uh  one  often  users  with
complet  the  T  of
product  uh  which  is  like  completion  of
the
usual
if  you  do  some  kind  of  formal  geometry
or
something
um  and  here's  a  the  that  is  actually  uh
what  the  s  t  product  does  uh  so  actually
like  in  full  generality  for
any  being  group  is  actually  better  to
replace  this  by  soal  deriv  De  completion
which
is  the  deriv  PO  operation
um  very  where  this  deriv  thing  just
means  a  complex  represented  by
multiplication  by  where  this  sits  in
degree  zero  is  this  sits  in  homological
degree  minus  one  or  homological  degre
one
uh  almost  always  this  agrees  with  usual
comption  it's  degre  zero  but  in  complete
generality  that's  a  better  operation  um
and  so  generally  uh  and  then  it  makes
sense  to  talk  about  derive  I  mean  these
things  are  the  derive  complete  s
um  uh
and  uh  so  here's  here's  a
proposition
uh  and  because  we're  doing  the  is
slightly  better  to  from  the  start  uh
work  in  the  dra  category  and  let  me
assume  I  have  things  which  are  conc  non
negative  degrees  so  go  to  the
left  so  see  I  have  two  of
them  and  they  are
complete
so  there  isomorphic
to  D  of  the  D  production  from  P  to  the
n  n  obviously  it's  actually  also  a
condition  you  can  check  on
grps
uh
then  also  there  Sol
products
um  let  me  prove  this  in  a  second  let  me
just  Note  One  cor  which  is  for  example
that  if  you  take  a  I  don't  know  infin
direct  some  of  copies  of  Z  and  take  a
period  completion  of  that  and  then  take
product  of  that  with  such  a  period
completion
of  you  just  get  the
similar
so  this  means  that  in  such  kind  of
situations  of  more  or  less  formal
geometry
uh  d  i  complete  things  the  S  product
does  what  you  would  expect  it  to
do  and  there's  really  nothing  special
about  the  number  P  here  I  me  could  work
with  any  base  ring  and  solid  modules
structure  with  that  ring  and  then  any
element  of  that  ring
would  so  this  PRD  completion  of  the  free
free  ailan  group  you  consider  it  as  a  as
a  solid  as  a  light  condensability  group
or  as  a  is  it  is  it  solid  I  found  the
completion  in  condenser  being  groups  so
it's  it's  a  all  the  z  p  to  the  end  all
the
but  takes  a  limit  to  requires  the  non
condens
strategy  and  it  is  solid  the  it  is  solid
because  solid  I  mean  the  class  of  Sol  is
stable  under  all  limits  liit  and  so  on
so  start  with  C  which  is  solid  take  a
direct  Sol  take  to  end  it's  solid  take  a
limited  um  uh  remk
uh  there's  nothing  special  about
P  for  any
R
um  any  elements
or
and  and  and  then  which  are
solid  a  for  now  I  mean  this  at  this
moment  it  really  just  means  a  modules
inside  solid  B
groups  uh  which  are  the  T
complet
then  the  dve  10  the  product  of
M
is  I  should  mention  is  actually  a  COR
with  the  proceeding  so  we  know  there's
definitely  a  map  from  here  to  here
because  the  standard  product  represents
B  maps  and  so  there  is  obious  B  map  to
here
um  uh  but  both  sides  are  deriv  promete
the  right  hand  side  by  definition  the
left  hand  side  by  the  proposition  um  so
to  check  whether
we  can  actually  check  after  reduction  P
because  for  dve  complet  things  is  can  be
checked  modular  P  but  modular  P  both
sides  just  become  theet  and  you're  just
taking  the  usual  TCT  of  product  of  FPS
and  product  a  direct  sum  of  FPS  and  a
direct  sum  of  FPS  and
just
um  okay  so
uh  let  me  give  a  pro
sketch  uh  so  first  of  all  maybe  in  this
situation  you  can  actually  work  with  a
zp  instead  because  uh  the  D  centerct  of
zp  was  CP  uh  is  just
zp  um  that's
because  you  can  take  C  power  series
T  C
power  and  then  you  fren  by  T  minus  p  and
uus  p  um  then  tus  P  you  get  here  uus  P
get  here  but  also  when  you  take  this
power  Series  in  two  variables  and  but
tal  to  P  I  mean  then  you  get  zp  power
but  thenal  to  zp  um  so  this  actually
means
that  uh  the  drive  ceg  of  solid  zp
modules  is  automatically  full  sub  of
solid  Z
module  so  being  a  ZT  module  is  not  a
structure  and  a  solid  being  it's  just  a
condition  and  if  everything's  complete
then
uh  uh  everything  becomes  a  zp  module  so
so  m\&n  is  actually  here
and  everything  is  taking  place  in  the
subcategory  okay  and  let's  actually  let
me  do  the  case
where  which  is  maybe  the  most
destructive  case  uh  where  m  is  really
just  uh  take  a  direct  sum  of  copies  of
zp  uh  but  you  complete  a  direct  sum  and
then
small
then  in  order  to  compute  what  the  Sol  T
product  does  we  have  to  write  this  in
terms  of  our  generating  objects  and  this
is  actually  a  slightly  non  trial
exercise  because  there  is  actually
rather  large  to  liit  of  compx  detective
objects  um  so  what  What's  Happening  Here
is  that  this  actually  is  a  CO
limit  overall
functions  from  n  to  n  uh  which  go  to
Infinity
so  they  only  find  the  many  values  where
below  some
constant
um  all  the  FR  of  copies  Over  N  of  to  F
of  so  what's  happening  here  so  first  of
all  why  is  there  a
map  so  whenever  you  have  a  map  which
goes  to  Infinity  you  can  take  the
product  of  these  copies  of  P  to  the  F
and  DP  and  this  actually  match  to  the
completed  direct
sum  because  modu  each  power  of  P  almost
all  of  the  terms  in  this  product  will  go
to
zero  that's  because  this  is  going  to
Infinity
um
but
uh  uh  and  quite  obviously  also  an
injector  um  but  then  you  still  have  to
show  that  it's  a  surge  action  so  you
have  to  show  that  whenever  a  live  Prof
set  map  is  completed  direct  sum  it
actually  effected  over  one  of  these
terms  but  if  you  have  a  map  into  this
uh
completion  then  yeah
Fe
Vector  think
any  light
from  and  the  map  from  s
this
um  I'm  just  thinking  whether  in  the  most
General  case  I'm  actually
um  uh  so  pick  any  sest  and  the  map  from
to  copies  of
D  which  is  by  the  permission
liit  and
so
so  we  have  I  don't  know  G  here  and  so  we
have  a  collection  of  M  GN  here  and  then
G  each
GN  vors
over  some  direct  sum
over
uh
integers  at  most  some
uh  a
of  um
right  because  uh  this  is  a  like  I  mean  a
compact  object  and  Maps  lot  so  it
must  what  is
this
and
uh
so  okay  so  let  me  go  right  so  so
there  are  just  a  few  bunch  of  elements
where  you  you  get  something  mod  P  one
and  then  there  is  maybe  a  larger  Bunch
where  you  get  something  non  zero  uh  mod
P  P  squ  and  then  there's  something  even
larger  where  you  get  something  non  zero
mod  cubed  but  then  you  can  just  find
some
slowly  very  slowly  increasing  function  f
uh  uh  sub  effect  over  this  product
right  so  what's  what's  my  diagram  here
uh  these  are  my  M  these  are  my  M  and
okay  this  is  AF  one  this  is
AF
okay
so  so  this  means  that  this  tend  of  M  and
M  um  each  of  those  so  it's
a  of  f  of  of  to  the  F  of
n
p  and  then  I  take  Sol  product
with  a  similar  expression  on  the  other
side  and  let  me  call  now  this  function
for  the  other  Factor  T  and  let  me  call  a
variable  on  the  other  side
n
zp  and  the  S  TCT  canes  Coit  so  I  can
pull  the  Coit  out  so  it's  a  CO
liit  over  two
functions  from  n  to  n  which  goes  both  to
infinity  and  uh  and  then  the  individual
hand  up  these  I  mean  both  of  these  are
just  products  of  zps  uh
and  yeah  I
mean  I  probably  didn't  say  it  so  far  but
product  of  CPS  they  just  become  the  B
INF  product  again  just  for  like  for
product  of  c
um  and  so  now  it's  the  product  of  N  and
m  in  M  of  P  to  the  F  of  n  plus  P  of
m
c
and  now  this  has  an  obvious  map  so  f  on
it  over  all  functions  let  me  call  them  H
which  are  now  functions
on  both  coordinates  which  just  go  to
Infinity
on  yeah  is  a  function  of  two  variables
uh  and  then  you  put  the  function  of  H
and  N  here  uh  where  this  this  is  uh
completed
and  now  at  first  you  might  think  that
this  will  surely  not  work  out  because
here  you're  allowed  to  quantify  over  all
arbitrary  slowly  increasing  functions  of
two  variables  whereas  here  you're  just
getting  those  that  are  sums  of  functions
of  one  variable
individually  but  then  there's  just  maybe
at  first  slightly  surprising  but  not  Al
how  to  prove  combinatorial  L  that
whenever  you  have  such  a  slowly
increasing  function  H  you  can  always
find  one  that's  even  slower  increasing
uh  which  is  a  sum  of  two  functions  of
individual
coordinates  leave  it  to  as  an  exercise
to  figure  that
out  but  that's  what  I  mean  that  the  Sol
T  Pro  like  for  non  obvious  reasons  giv
you  the  correct
awesome
um  and  so  yeah  General  argument  some
saying  so  you  can  redu  the  case  at  n  and
M  are  just
some
uh  yeah  completed  direct  sums  of  copies
of  the  generators  and
then  all  as  the  same  argument  you
actually  have  to  be  slightly  careful
because  upor  you  can  also  have  the  case
that  M  and  M  are  completions
of  uh  a  direct  sum  which  is  over  an
uncountable  index  set  and  then  you  still
want  that  uh  but  then  there's  actually
an  argument
that  the  uncountable  case  just  by  more
formal  aru  to  cont
Cas  because  this  der  completions  they
always  commute  with  I  mean  you  can
always  ruce  uncountable  things  to
contable  things
because
just  anys
um  right  so  this  is  a  nice  computation
um  there  are  some
even  there  are  some  further  computation
that  come  out  right  uh  which  some  really
appear  when  you  do  someone  solid
functional  solid  functional
now  so  let  me  work  over
QP  for
Simplicity  uh  but  most  one  of  what  I'm
saying  uh  works  over
any
me
so  then  again  uh  solid  PP  displaces  or
even  the  D
category  it's  just  the  full  subcategory
well  P  modules  and  those  where  p  is
invertible  and  it's  the  full  subeg  of  be
group  and  if  you  form  10  prodcts  and
just  some  St  was  in  the
sub
um
and  uh  this  category  itself  it  now  has  a
compact  projective
generator  well  here  will  be  a  product  of
zps  and  here  you  have  to  invert  p  take
of
zp  this  is  a  slightly  curious  kind  of
topological  Vector  space  or  like  condens
vector  space  but  actually  comes  from  a
topological  Vector  space  so  it's  it
looks  like  one  where  some  the  unit  which
would  have  a  norm  where  the  unit  ball  is
this  thing  where  the  unit  ball  is
compact  but  actually  there's  is  General
thing  you  cannot  have  bner  spaces  or
Norm  Vector  spaces  of  infinite  Dimension
ver  unit  ball  is  compact  so  it  actually
turns  out  that  if  you  were  trying  to
endulge  this  with  a  norm  in  the  kind  of
obvious  way  where  this  would  becomes  the
unit  ball  then  the  nor  map  would  not  be
a  continuous  map
and  so
this  yeah  it's  not  really  a  norm  Vector
space  is  what  it  is
uh  so  uh  these  things  have  appeared  a
little  bit  also  in  the  classical
function  literature  uh  May
firstan
called  um
uh  I  mean  the  the  more  usual  objects  the
consider  thetic  function
analysis  which  space  you  wrote
Smith  theic  Smith
space  and  she  was  uh  studying  the  analog
of  those  things  over  the  real
numbers  they  can  also  consider  some  type
of  topological  Vector  spaces  where  which
as  well  ining  Union  uh  gotten  by  scaling
out  some  compact  convex
set  the  more  usual  objects  we  consider  B
spaces  and  at  least  separable  examples
of  those  exactly  the  on  that  we
considered  previously  you  take  it
accountable  copies  of  P  complete  and
then
and  there  certainly  also  in  this
category  and  we  just  saw  how  we  would
present  in  terms  of  the
generation
just
in
and  these  things  are  actually  in
reality
uh  Mi
places
SP  so  either  you  put  separable  and  light
on  those  sides  or
you  so  separable  on  the  bio  side  means
you  can  take  accountable  D  somewh  and
light  some  on  the  other  take  accountable
product  and  this  is  just  the  obvious
where  Take  A  Smith  space
St  but  you  can  also  take  B  of  space
like  so  yeah  so  the  generator  that  we
use  here  the  sp  there  and  sometimes  du
of  spaces  okay  we're  basing  everything
on
them
uh  uh  here's  again  a  curious  remark
um  and  maybe  people  ask  me  this  so  it
seems  relevant  uh  so  what  happens  if  you
try  to  do  the  sage  well  this  is  the  smth
this  projective  and  even  turn  projective
so  there  actually  deriv  so  um
uh
um  so  if  it  takes  the  r  home  here  it  is
just  the
internal
Zer  that's  true  and  kind  of  obvious  um
but  if  he  go  the  other  way  you  can  all
wonder  if  you  take  a  b  space  and  takes
the  drive  to  of  that  one
is  it  just  the  usual
dual  and  the  answer  is  uh  depends  on
your  model  of  the  set
the  in  the  same  way  as  this  came  up
before  so  again  it  fails  under  the
contines  but  under  this  principle  star
that  I  mentioned  that  is
consistent  uh  it's
because  a  BOS  space  is  a  special  type  of
Pro
Group  limit  of  direct  sums  over  end  of
QP  mode  p  toz
mzp  and  so  this  AR  de  like  between  Pro
things  and  this  precisely  the  thing
that's
controlled  this  is
princi
Yeah  the  more  I  think  about  the  more  I'm
actually  temped  to  just
uses  work  model  is  true  uh  because  you
have  the  choice  of  either  these  higher  X
being  some  junk  that  really  has  no
meaning  at  all  all  them  being
zero
um  okay
um  but  other  like  what  I  DUS  and  I  have
been  trying  to  do  this  this  question  was
never  really  relevant  so  that's
sometimes  you  run  into  this  thing  I  mean
it's  definitely  like  if  you  do  some  kind
of  computations  now  with  in  this  kind  of
solid  function  analysis  you're  doing
lots  of  home  lots  lots  of  tenses  lots  of
homes  and  so  on  at  some  point  surely  you
will  at  some  point  take  some  our  home  of
b  space  against  D  somewhere  and  then  you
can  decide  whether  you  work  in  a  model
where  it's  just  some  joint  or  you  work
in  this  model  where  it's  what  it's
supposed  to  be
so  um  but  what  I  really  wanted  to  say  is
uh  so  you  know  you  also  have  to  fre
spaces  um  which  are  uh  countable  limits
of  FAL  spaces  along
them
these  are  really  uh  this  really  pop  up  a
lot
um  and  there  is  a  notion  of  uh  there's  a
standard  notion  of  computer  trans  for
them  uh  in  particular  it's  the  one
that's  compatible  with
limits
so  on  B  bases  it's  a  usual  BF  tender
product  and  when  you  have  f  space  inv
such  a  accountable  limit  and  it's  the
other  one  also  in  some  corresponding
limit  of  complete  T  of  products  of  B
spes  and  so  like  a  general  thing  that
usually  when  you  define  such  T  prods  and
function  analysis  is  that  you're  trying
to  make  some  compatible  with  limits  here
right  for  example  in  this  case
uh  which  is  not  readed  because  our  Sol  T
produ  was  defined  to  be  compatible  with
Co
limits  and  usually  there  are  very  few
funs  that  are  compatible  with  limits  and
to  limits  and  certainly  thect  is  not
compatible  with  all  limits  uh  but
fortunately  it  so  turns  out  that  in  this
situation  it  does  give  the  correct
answer  so  here's  a
proposition  if
CMW  this
let  say  considered  as  topologic  spes  for
the
moment
uh  then  you  can  pass  the  condensed  World
from  line  and  then  take  the  Sol  tender
product  or  even  the  dve  Sol  tender
product  and  it  turns  out  that  this  is
really  just
uh  the  completed  T  PR  spaces  and  then
pass  in  particular  it's  deg
Z  for  example  if  you  take  a  product  of
copies  of
p  and  take  another  prod  of
P  it  becomes  the  by
infinite  which  might  seem  like  it's  a
standard  thing  on  compact  projective
generators  but  it's  absolutely  not
because  these  things  are  very  very  far
from  being  compact  projective  only  these
kind  of  unit  balls  there  there  as
compact  projector  cars  but  this  is  a
huge
space
okay  and  so  let  me  again  just  give  an
proof  on  an  example
in
zamp  I  me  the  general  proof  is  in  some
sense  combining  uh  the  proof  I  did  there
with  those  poit  with  the  one  I  will  do
here  uh  and  we  kind  of  spray  by  was
doing  a  minimalistic  approach  to
combining  this  but  Sasha  e  actually  has
a  very  fancy  way  to  combine  them  that  he
needed  in  some  of  categorical  stuff  um
okay  so  so  so  okay
so  how  do  we  write  this  product  of  tpce
so  it's  again  such  a  huge  po
limit  and  this
time  you're  taking  such  a  unit  balls  but
uh  where  you  makes  the  denominators
larger  and  larger  so  you  take  of  P  the
minus  FN  *
DC
um  where  again  f  is  some  function  from  n
to  n
but  now  we're  taking  forward  over  ever
faster  increasing  functions  so  it's
certainly  going  to  Infinity  but  but  very
fast
say
um  right  because  again  if  you  have  any
complex  project  I  mean  any  profile  that
mapping  into  here  in  each  factor  it  will
Factor  over  some  uh  over  some  pie  of  F
mCP  and  then  you  just  well  take  this
function  and  I  mean  generically  it's
probably  a  function  that  I  don't  know
just  for  any  n  you  give  some  value  but
for  each  new  number  you  can  make  it  much
much
larger  okay  so  uh  and  so  if  you  take
this  tend
product  it's  a
Col  by  same  Principle  as  before  now  we
have  two
functions  which  go  to  Infinity  very  fast
um  of  a
product  n  n  of
tus  FN
minus
zp  and  again  this  next  to  a  similar
col  or  functions  which  are  functions  of
both
uh  variables  going  to  Infinity  very  fast
um
and  again  you  might  naively  think  that
surely  if  I  have  two  variables  I  can
build  functions  that  are  so  fast
increasing  that  I  can  never  dominate
them  by  something  that's  uh  some  of
functions  in  both  variables  but
turns  out  again  by  basically  the  same
argument  as  there
is  you  can  always  dominate  any  such
function  by
function  all  right  that's  fun
um  all  right
uh
I  should  maybe  say  I  mean
uh  this  is  some  f  poet  here  but  the
precise  structure  of  this  Index  postet
this  is  actually  where  all  the  set
appears  so  here  we  kind  of  don't
actually  have  to  understand  what  this
looks  like  because  this  combinator  can
just
prove  but  for  these  X  computations  you
really  have  to  he  some  starting  with
such  a  description  in  the  case  of  b
space  and  then  do  that  and  so  then  this
this  this  huge  call  liit  becomes  a  huge
derived  limit  and  then  you  really  have
to  Grapple  with  a  structure  of  this  kind
of  postet
of  functions  that  grow  arbitr  fast  and
uh  this  is  something  that  very  much
depends  on  your  model  of  set  here
particular  on  the  cardinality  of  SP  set
just
a
all  right
um
yeah  I  think  that's  basically  all  I
really  wanted  to  say  but  if  there  have
questions  we  have  a  few
minutes  so  for  functional  analysis  in
some  very  large  B  of  spaces  does  it  make
a  difference  if  you  work  in  light
condenser  or  all  condensed  sets  it
doesn't  make  a  difference  because  all  B
spaces  are  lied  all  for  spaces  are  lied
no  matter  whether  they're  separable  or
not  it's  just  the  pred  to  Smith  bases
only
they  uh  they
see  okay  it's  clear
Factor  yeah  because  it's  actually  have
some  kind  of  uncom  dimensional  V  Space
it's  still  the  filtered  just  naive
filtered  col  of
okay
okay  any  other
questions  is  the  sorry  it's  a  very  V
question  but  is  the  relation  between  the
homology  of  the  C  complex  and
ification  kind  of  the  accident  or  should
should  I  take  it  some
deep  I  mean  can  I  can
I  can  I  expect  that  we  will  have  another
example  in  I  don't  know  in  a  condensed
Spectra  or  something  like  that  for  other
well  instead  of  uh  like  taking  the  dra  C
being  groups  you  could  everywhere  work  a
category  of  Spectra  and  then  you  could
also  yeah  I  mean  maybe  a  good  point  to
point  out  the  following
um  there's  Spectrum  call  little
K  like  to  think  of  as  some  you  take  the
algebraic  Cas  of  the  complex  numbers  but
you  take  into  account  the  topology  that
the  glnc  has
right  um  uh  and  some
makes  this  homotopy  say  uh  and
classically  this  a  little  bit  hard  to
phrase  what  this  is  supposed  to  mean  I
believe  orbe  there  way  to  do  anyway  so
one  way  you  can  however  do  it  in  our
formalism  is  you  can  take  the  cas  of  the
complex  numbers  as  a  condensed  spectrum
because  I  mean  the  complex
numbers  this  is  a  condensed
ring  and  so  this  actually  means  that  F
Cas  series  is  kind  of  has  some  automatic
structure  as  condensed
Spectrum  right  because  to  give  a
condensed  Spectrum  I  have  to  give  a
function  for  each  maybe  live  profile  set
I  have  to  produce  a  spectrum  but  uh  I
can  just  take  the  case  theory  of  the
continuous  function  from  s  to
C  um  so  this  this  is  sorry  so  far  this
is  not  at  all  true  um  because  here  the
pi1  for  example  will  be  C  Star  as  a  as  a
condensed  being
group  but  uh  now  you  can  take
KFC  and
solidify  like  just  like  we  had  solid
draft  of  solid  being  groups  and  draft
car  of  condensed  being  groups  you  also
have  solid  Spectra  inside  of  all
condensed
Spectra  and  this  sonification  does
exactly  what  you  wanted  to  do  it  somehow
contracts  all  the  deal  NC  that  this  was
build  out  of  Toopy  Ty  essentially  by  the
proposition  I  mentioned  in  lecture  today
and  then  you  see  that  this  is  actually
exactly
the  and  I  mean  you  could  do  the  same
thing  really  not  just  for  the  complex
numbers  but  for  any  algebra  over  the
complex  numbers  or  even  a  category  over
the  complex  numbers  and  then  take  the
case  as  a  condensed  spectum  and  solidify
and  this  actually  gives  a  construction
of  what's  known  as  a  semitopological
case
mhm  so  it  is  solidification  of  what  semi
topological  uh
so  if  a  is  any  kind  of  C
algebra  was  it  SC  topology  or
uh  with  a  topology  I  don't  care
um  then  you  can  similarly  take  the  cas
of  a
and
solidify  and  this  recover  something
that's  known  as  a  sar  topological  case
the  I  some  some  some  hope  in  the  area
that  when  you  have  a  proper  smooth  DG
category  of  the  complex  numbers  there's
some  way  to  define  its  topological  case
Theory  and  I  think  this  was  one  of  the
reasons  people  started  uh  defining  the
same  toy  Cas  Theory  so  the  actual  one
should  be  a  module  over  periodic  KU  but
you  can  just  take  this  one  and  just
inverts  the  B  element  then  you  have  a
candidate  for  something  that  might  be
the  correct  thing  but  nobody  knows  how
to  prove  much  about  it  I
think  can  you  get  the  similar  result  for
KO  or  KR
sure  uh  just  write
R  yeah  I
mean  uh  I  mean  all  the  obvious  VAR  yeah
so  in  particular  KO  is  take  Cas  of  the
real
numbers  and  there's  something  I  know  PSP
or  whatever  it  is  and  then  maybe  related
to
the  so  here  the  K  is  the  periodic  or  the
the  full  spectr  I  mean  in  those  notation
I  the  algebra  Cas  Spectrum  uh  it  turns
out  to  be  connective  for  the  inputs  that
I'm  using  uh  maybe  not  quite  obviously
so  but  it's  true  um  and  so  there  is  a
connective
KU  uh  it  is  much  more  subtle  to  recover
the  periodic  you  I  mean  you  can  of
course  just  invert  what  El  at  the  end
but
uh  and  sometimes  some  this  antic  case
the  that  do  an  I  develop  which  is  a
different  way  to  approach  this  kind  of
stuff  is  aimed  at  defining  some  kind  of
version  of  the  cas  the  of  the  complex
numbers  but  now  analytic  ring  where  the
answer  well  certainly  has  a  St  negative
degrees
Ian  some  FS  is  it  somehow  related  to  the
description  of  the  small  KU  or  small  K
or  as  some  configuration  space  like
description  given  by  gam
seal  um  I  don't  know  about  the  ah
sorry  there's  some  uh  do  like
description  for  uh  small  K  and  for  using
a  like
vector  given  by
SE
uh  I  I  think  this  statement  is  much  more
naive  I  mean  this  is  really  just  this
statement  is  really  just  one  way  of
encoding  the  intuition  that  or  maybe  see
but  I  you  you  you  some  in  some  sens  this
is  false  some  something  that's  true
before  group  completion  namely  that  if
you  take
uh  the  infinity  monoid  classifying  space
of  c  and  then  pass  to  homotopic  types
here
um
then  then  I  mean  KU  is  a  good  completion
of  that  and  basically
solidification  uh  factors  over  passing
to  homotopy  types  and  so  this  is  seconds
um  but  they  have  I  don't  think  there  are
any  figuration
spaces  any  other
questions  to  what  extent  does  the  Sol
tens  product  combine  two  topologies  so
say  we  have  m  and  n  and  m  x  complete  and
directly  and  there  was  a  thing  that  I
mentioned  that  if  you  take  power  series
T  and  then  U  then  it  becomes  the  UT  edic
one  so  there  it  does  do  this  thing
um  and  this  more  b  space  setting  it
doesn't  do  it
I  so  when  you  TOR  over  a  and  you  have
this  exotically  complete  statement
now  when  you  can  take  the  completion  of
the  Ring  a  but  this  one  is  no
longer  the  derived  completion  does  it
make  a  difference  for  the
proof
uh  yeah  so  sorry  I  actually  realized
that
probably  most  General  version  for  St
might  not  be  expected  true
um  I  was  getting  confused  about  this
um  well  I  mean  if  you  want  to  compute
some
direct  product  of
a  solid  or  whatever
um
one  it's  kind  of  resolved  by  the  the
absolute  one  and
then  you're  also  taking  a  proct
a  and  so  to  show  that  this  here  is  the
completed  it's  enough  to  know  that  all
the  other  ones  are  and  this  way  you  can
kind  of  reduce  the  case  that  the  ring  a
is  actually  just
z  z  x
here  so  then  the  DED  X  completeness  of
this  one  reduces  to  the  one  here  here
here  but  a  itself  is  derived  X  complete
I  mean  if  it's  basically  X  complete  it's
also  derived  X  complete  um  so
uh  you  can  kind  of  reduce  this  statement
to  the  case  that  the  ring  is  really  just
the  join
x  what  did  you  mean  by  passing  to  otop
type  in  BGN  C  so
glnc  is  viewed  as  a  topological  space
and  therefore  as  a  as  a  condensed  set  or
or  notari  I  mean  yes  so  when  I  when  I
Define  KFC  one  way  to  do  that  is  to  take
this  classifying  space  of  DG  onc  and
then  I  have
some  uh  infinity  group  in  condensed
groupoids  um  and  then  I  group
completed  but
then
uh  yeah  so  some  physically  I'm  here
using  the  category  of  condens
Alima  which  contains  condensed  sex  or
condensed  groups  and  so  on  as  one  for
sub
category  but  contains  Ana  or  typ
whatever  there  another  for  sub  which  are
some  of  the  discret
objects  um
and  then  on  a  certain  class  of  objects
here  there's  a  left  joint  which  is
passing  to  the  homotopic  type  uh  and
this  is  not  always  defined  but  on  some
object  it's  defined  and  for  example  on
CW  complexes  it  is  defined  during
testifying  space  of  Jo  and  so  and  so
what  I  meant  there  was  that  I  kind  of
appli  this  left
joint
all  right  let's  stop
here
\end{unfinished}
% !TeX root = AnalyticStacks.tex

\section{\ufs The solid affine line (Clausen)}

\url{https://www.youtube.com/watch?v=fUjn2rGw9SA&list=PLx5f8IelFRgGmu6gmL-Kf_Rl_6Mm7juZO}
\renewcommand{\yt}[2]{\href{https://www.youtube.com/watch?v=fUjn2rGw9SA&list=PLx5f8IelFRgGmu6gmL-Kf_Rl_6Mm7juZO&t=#1}{#2}}
\vspace{1em}

\begin{unfinished}{0:00}
e
today  we're  going  to  be  talking  about  a
little  bit  of  geometry  maybe  the  solid
Aline  line  um  so  let  me  start  with  a
recap
um  of  what  we've  seen  so  far  um  so  we
had  this  category  of  solid  abilon
groups  um  which  was  a  full  subcategory
of  uh  light  condensed  a  bilion
group  um  and  this  was  some  kind  of
analog  of  uh  complete
non-  archimedian
topological  aan
groups  but  it's  kind  of  um  from  a  formal
perspective  easier  to  work  with  it's  an
ailan
category  um  in  fact  it's  yeah  so  it's
ailan  category  closed
under  uh  limits
colimits
extensions
uh  and  many  other  things  besides  and  um
also  we  had  this  left  ad  joint  called
solidification  um  and  there's  a
symmetric  monoidal  structure  here  which
you  think  of  as  a  completed  tensor
product  and  and  such  that  this
completion  functor  is  symmetric
monoidal
um  and  uh  the  definition  was  in  terms  of
this  object  p  uh  which  was  kind  of  the
free  module  on  a  null
sequence  so  the  definition  was  that  m  is
solid
uh
uh  so  the  definition  was  that  it's  solid
if
um  uh  this  map  from  null  sequences  to
null  sequences  given  by  identity  minus
shift  is  an
isomorphism  um
okay  so  that's  what  we  had  last  time  um
now  I  want  to  start  doing  a  little  bit
of  geometry  so  we're  going  to  be  modest
and  look  at  the  Aline  line  which  is
actually  the  most  important  case  um  so
about  this  P  so  it  turns  out  that  this  p
uh  even  before  solidification  so  p  is  a
ring  so  there  are  two  shets  one  can  look
at  because  you  can  shift  one  to  the  left
and  one  to  the  right  yeah  or  bothos  is
equivalent
uh  I  want  the  one  no  I  think  only  one  is
good  I  mean  I  want  the  one  that's
injective  uh  so  p  is  a  ring  and  maybe
I'll  it'll  be  clarified  in  a  sec  maybe
well  p  is  a  ring  and  in  fact  there's  a  a
ring  map
uh  so  from  the  polinomial  ring  in  one
variable  to  p  and  the  shift  is
multiplication  by
T
um  uh  so  that's  that's  what  you  have
before  solidification  but  after
solidification  something  uh  extremely
nice  happens  so  when  we
solidify  well  we  actually  already
explained  what  the  free  module  on  P  is
so  the  the  the  uh  there's  a  compact
projective  generator
here  uh  which  was  a  countable  product  of
copies  of  Z  that's  kind  of  the  basic
object  from  which  everything  is  built
and  you  can  get  that  object  object  just
by  solidifying  this  basic  um  sequence
space  p  and  if  you  take  into  account  the
ring  structure  what  happens  when  you
solidify  is  you  just  get  the  power
Series  ring  in  one
variable  so  in  the  solid  context  the
universal  carrier  of  a  null  sequence  is
just  this  power  Series  ring
um  but  moreover  there's  a  a  very  nice
property  of  this  situation  so  so  LMA  how
do  you  multiply  things  in  P  that  looks
like  a
cent  by  something  which  is  not  uh  the
null  sequences  is  like  ring  out  unit
usually  yeah  so  let  me  explain  how  you
get  the  ring  structure  just  very  briefly
so  it's  uh  it's  better  to  think  of  this
construction  as  being  attached  to  n
rather  than  n  Union  Infinity  so  more
generally  if  you  have  a  set  say  or  a
countable  any  countable  set  you  can
Define  measures  on  it  as  the  free  module
on  any  compactification  module  the  free
module  on  on  the  on  the  boundary  and
it's  independent  of  the
compactification  and  it's  functorial  for
proper
Maps  so  now  and  it's  symmetric  monoidal
you  can  see  that  using  Independence  of
the  compactification  if  you  take  the
product  of  two  countable  sets  then
measures  on  one  tens  or  measures  on  the
other  is  measures  on  the  product  so
using  that  and  the  fact  that  addition  on
the  natural  numbers  is  a  proper  map  has
finite  fibers  you  can  see  that  addition
on  natural  numbers  induces  this  ring
structure  on
P
okay  all  right  um  it's  the  same  and  you
know  addition  on  natural  numbers  also  in
some  sense  induces  this  so  that's  why  uh
yeah  all  right  uh  so  the  Lemma  is  that
uh
so
uh
so  if  you  do  the  tensoring  in  the  solid
world  uh  of  this  power  Series  ring  with
itself  over  the  polinomial  ring  uh  you
just  get  the  power  Series  ring  again
maybe  multiplication  map  is  an
isomorphism  you  could  see  and  even  the
even  the  derived  tensor
product  and
well  this  is
um  this  is  quite  Elementary  uh  because
we  know  how  to  well  we  know  how  to  do
tensor  products  in  solid  Z  Peter  gave
lots  of  examples  of  calculations  in  this
category  and  the  tensor  product  of  two
of  these  basic  guys  uh  is  just  another
one  of  them  um  and  so  if  you  do  uh  so
proof  If  you  do  Z  power  series  T  let's
say  T1  uh  tensor  over  solid  z  uh  Z  power
series
T2  that's  just  uh  Z  power  series  T1
T2  by  the  basic  calculation  of  the  solid
tensor  product  and  then  if  you  want  to
get  to  this  uh  you  just  do  modding  out
by  you  just  identify  the  two  different
variables  T1  minus  T2  and  uh  Power
Series  ring  modulo  T1  T2  is  is  the  power
Series  ring  in  one
variable  um  there's  maybe  another
perspective  Peter  also  mentioned  this
fact  that  if  you  take  the  solid  tensor
product  of  two  derived  complete  things
then  the  result  is  still  derived
complete  so  well  maybe  you  have  to  worry
a  little  bit  about  this  but  morally  if
you  want  to  check  this  is  an
isomorphism  both  sides  are  Drive  T
complete  you  can  check  at  modulo  T  and
modul  T  it's  really  a  triviality  um  so
kind
of  okay  um  the  the  derived
solid  uh  Tor  product  is  it  defined  as
the  derived  T  product  then  you  solidify
or  is  it  how  how  is  Def  exactly  yeah
it's  the  you  can  either  say  it's  the
derived  the  left  derived  functor  of  the
solid  tensor  product  or  you  can  say  that
you  take  the  derived  tensor  product  and
then  derived  solidify  it  it's  it's  all
the
same  and  do  need  to  know  there  are
somehow  enough  flat  things  to  construct
the  the  you  don't  need  to  know  that  but
you  but  we  but  it  is  actually  true  that
all  of  these  guys  are  flat  I  don't  know
if  Peter  did  Peter  mention  that  I  forget
uh  yeah
um
okay  uh  yeah  so  a  a  prior  you  need  to
resolve  both  variables  but  since  you
have  flat  objects  you  don't  you  would
only  need  to  resolve  one  um  okay  so  what
is  the  interpretation  of  this  so  this  uh
this  corresponds  to  sort  of  the  apine
line  you  could  say  and  then  since  this
uh  satisfies  this  item  potent  property
uh  sort  of  like  a  a  when  you  invert  an
element  in  a  ring  you  kind  of  get  the
same  thing  you  can  think  of  this  as
corresponding  to  some
Subspace  of  the  apine  line  so  a  Subspace
as  opposed  to  just  a  random  space  with
them  with  a  map  to  the  Aline  line  and
that's  the  interpretation  of  this  item
potency  so  then  you  could  ask  okay  how
how  should  one  think  of  this  Subspace
what  Subspace  is  this  um  so  the  naive
thing  to  say  would  be  that  it's  the
formal  completion  of  the  Aline  line  at
the
origin  um  well  because  it's  a  power
Series  ring  but  that  naive
interpretation  is  not  the  correct  one
and  so  so
so  does  not  correspond  to  a
formal  uh
neighborhood  of  zero  in  A1  and  the
reason  is  that  that  that  interpretation
is  not  stable  under  base  change  so  so
look  at  base
change  by  which  I  mean  this  is  all
implicitly  over  Z  but  we  could  tensor  it
to  any  ring  or  any  solid  ring  and  see
what  pops  up  um  and  let  me  do  a  little
calculation
so  so  let's  uh  let's  base  change  to  a
non-  archimedian  field  say  I'll  just
take  the  simplest  example  QP  and  we'll
see  what  pops
up  uh  so  we
take  the  Aline  line  over
QP
um  uh  oh  sorry  well  maybe  I'll  just  say
yeah  we  we  take  we  take  QP  and  then  we
do  a  we  we  do  a  solid  tensor  product
with  uh  this  thing  here  and  we  want  to
compare  this  to
qpt  um  so  this  so  QP  is  zp  with  P
inverted  and  the  solid  tensor  product
commutes  with  co-  limits  in  both
variables  so  this  is  the  same  thing  as
zp  solid  tensor  product
uh  Z  power  series  T  and  then  you  invert
P  but  this
zp  uh  has  a  resolution  by  two  copies  of
so  so  Zu  U  maybe  mod  U  minus  P  it  has  a
resolution  by  two  copies  of  this  power
Series  ring  uh  so  we  know  how  to  do
these  tensor  products  and  it  just  does
the  naive  thing  of  you  can  pull  in  the
limits  uh  so  the  result  of  that  is  that
you  get  zp  power  series
t  uh  and  then  invert  p  on  the
outside  which  is  not  the  same  thing  as
the  formal  completion  at  the  origin
namely  QP  power  series  T  so  it's
contained  in  there  and  it's  the  subset
where  the
coefficients  uh  are  bounded  in  piic
Norm
so  so  what  is  the  interpretation  then  of
of  this  ring  uh  this  ring  is  the  ring  of
functions  on  the  open  unit
dis  uh  in  in  A1  over
QP
so  if  you  you  could  ask  uh  if  you  take
say  maybe  after  an  arbitrary  non-
archimedian  field  extension  of  QP  you
could  ask  when  you  plug  in  a  value  in
that  field  uh  when  will  this  when  will
such  a  series  converge  and  the  answer  is
it  will  converge  if  and  only  if  the
number  you  plug  in  has  absolute  value
strictly  less  than  one  the  Ring  of
bounded  functions  actually  what's  that
it's  the  Ring  of  bounded  holomorphic
function  of  open  Unity  ah  yes  uh  what
not  the  full
because  sometimes  yeah  there's  another
one  Ah  that's  true  yeah  that's  the  bound
thank  you  yeah  yeah  that's  right
sometimes  people  use  the  the  not
necessarily  like  the  robing  and  stuff
which  is  not  about  I  don't  know  that  all
kinds  of  things  yeah  I  don't  know
what  well  okay  certainly  all  these
functions  Converge  on  the  open  unit  dis
um  but  they  don't  converge  past  that
um  so
uh  well  so  it  yeah  so  then  the
interpretation  to  that  this  suggests  is
that  uh  that  uh  this  uh  ZT  uh
Z  this  should  be  thought  of  as
corresponding  to  some  sort  of  open  unit
dis  and  it's  just  that  over  a  discrete
ring  like  Z  you  can't  tell  the
difference  between  the  open  unit  dis  and
the  formal  completion  at  the  origin  but
the  difference  pops  out  when  you  make  a
a  base  ch  change  to  something  that  has
more  more  freedom  to  play
with
okay
um  but  in  uh  rigid  geometry
uh  uh  it's  not  really  the  open  unit  dis
that's  fundamental  uh  usually  it's  the
closed  unit  dis  um  so  what  about  the
closed  unit  disc
well  one  way  to  understand  the  closed
unit  disc  if  you  understand  what  an  open
unit  dis  is  you  can  imagine  putting  the
open  unit  disc  at  Infinity  instead  so
then  you're  looking  at  then  you  get  the
complement  of  the  closed  unit  disc  and
then  that's  one  way  of  talking  about  the
sorry  the  complement  of  the  you  yeah  you
get  the  complement  of  the  closed  unit
disc  and  the  complement  of  a  thing  is  is
just  as  good  as  describing  the  thing  so
let's  take  this  guy  and  put  it  at
Infinity  instead  so  the
complement  well  this  is  all  just  at  the
level  of  a  uh  loose  thinking  so  we  can
take  this  power  Series  ring  in  one
variable  uh  and  we  can  tensor  it  over
well  but  let  me  call  the  variable  T
inverse  instead  because  I  want  to  be
thinking  of  putting  it  at  infinity  and
then  I  tensor  over  ZT  inverse  with  uh  ZT
inverse  so  then  I'm  puncturing  at
Infinity  so  that  I  actually  live  in  the
apine  line  so  I  have  a  homomorphism  from
ZT
um  uh  and  what  is  this  ring  well  I'm
just  taking  this  power  Series  ring  and
I'm  inverting  the  the  variable  there  so
another  way  of  describing  this  is  as  Z
power  Series  in  t
inverse  so  that's  the  open  unit  dis
centered  at
Infinity
um  uh  so  if  we  want  to  understand  the
closed  unit  dis  we  should  in  some  sense
be  uh  kind  of  localizing  away  from  this
complement  or  another  way  of  saying  that
is  that  we  should  be  killing  this  object
so  so  to
get  we  need  to  kill  uh  kill  this  thing
here
so
um  so  well  I'm  just  going  to  make  just  a
preliminary  so  this  Z  power  series  T
inverse  is  a  module  over  ZT  as  I  just
exhibited  and  it  also  has  a  very  simple
resolution  so
uh  note
um  so  uh  well  ZT  inverse  we  can  think  of
that  as  let's  call  let's  disambiguate
and  call  the  variable  U  um  we  can  think
of  it  as  Z  power  series  U  and  then
invert  U  um  but  one  way  to  invert  U  is
to  adjoin  the  thing  that  you  want  to  be
uh  the  generator  uh  I  mean  want  to  be
the  inverse  and  then  enforce  the
equation  that  that  says  that  they're
inverse  to  each
other  um  and  when  you  look  at  it  like
this  uh
so  so  I  I'll  rewrite  that  so  we  get  ZT
inverse  uh  has  a  two-term  resolution
where  you  have  Z  power  series  U  bracket
T  and  Z  power  series  U  bracket
t  uh  and  here  you  have  uh  U  minus
one  so  two  term
resolution  and  what  are  these  objects
well  these  are  just  the  base  changes  of
our  fundamental  uh  p  uh  not  to  solid  Z
but  to  solid  z  uh  and  then  adjoin  a
variable  T  so  so  these  are  resolutions
by  the  compact
projective  uh
generator  uh  of  the  category  of  of  z
braet  t  modules  in  solid  ailion
groups
so  killing  this  thing  should  be  the  same
thing  as  requesting  this  map  to  become
an
isomorphism  and  now  this  suggests  the
following  based  on  an  analogy  with  the
definition  of  solid  up  there
so  so
definition  so  let's  say  let's  let's  say
we  have  a  one  of  these
guys  let's  say  that
is  ZT
solid
um  if  and  only  if  when  you  take
uh  internal  H  from  uh  P  to  M  and  then
internal  H  from  P  to
M  uh  and  then  you  take  the  map  which  is
given  by  so  now  U  corresponds  to  the
shift  and  T  is  the  extra  thing  we  have
acting  on  M  because  m  is  a  z  bracket  T
module  so  we  have  shift  time  T  minus  one
uh  we  want  this  map  to  be  an
isomorphism  and  then  this  note  up  here
is  saying  that  that's  the  same  thing  as
requesting  that  if  now  I  should  maybe
pass  to
Arham
uh
it's  the  same  thing  as  requesting  that
there  are  no  R  homs  from  this  object
we're  trying  to  kill  into
M  okay
so  now  uh  theorem  is  basically  that
um  this  definition  the  name  is  well
chosen  so  that  this  ZT  solid  Theory  over
ZT  is  very  similar  to  the  solid  Z  Theory
over  z  uh
so  so  let's
say
and  we  can  even  I  mean  everything  is
also  inside
condensed  little
light  this  uh  is  an  aelon
category  closed
under  limits  and
colimits
extensions  um  so  if  uh  if  m  is  an
arbitrary  say  condensed  ZT
module  uh
then  internal  x  uh  and  N  is  in  solid
Z  ZT  then  internal  X  over  ZT  from  uh  M
to  n  uh  is  also  in  solid
ZT  uh  there  exists  d  uh  there  exists  a
left  ad
joint  to  the  inclusion
uh  call  it  uh  let's  say  well  I'm  going
to  at  least  temporarily  denote  it  by
upper  T  solid  we're  forcing  the  variable
T  to  be  solid  you  could  say  so  that's
uh  z  z  and  Z  is  the  same  thing  one  you
wrote  in  the  first  yes  I  wrote  I  use
different  notation  yeah  sorry  yeah  this
is  this  is  supposed  to  be  the  same  as  as
this
yeah
yeah  um  there  exists  symmetric  monoidal
structure  on  this  solid  this  drive
solidification  over
ZT  um  and  then  uh  the  drived
analog  uh  holds  as
well  so  the  derived  analogs  of  all  these
claims  hold  and  moreover  the  drive
theory  is  naive  in  the  sense  that  an
object  in  the  derived  category  is
derived  solid  if  and  only  if  each
homology  group  is  solid  in  the  sense  of
the  ailion  category  and  it  is  the
derived  category  of  the  aelan
category  um  okay  so  that's  all  the  kind
of  formal  stuff  but  but  that  in  the
solid  Z  Theory  it  was  also  important  to
to  get  a  to  understand  what  was  going  on
it  was  important  to  understand  the  the
basic  compact  projective  generator  which
is  always  gotten  by  just  well
solidifying  the  sequence  space  um  so  let
me  make  a  claim  about  that  so  the  if  you
take  the  Z  power  series  well  maybe  I'll
just  I'll  say  let's  not  think  of  it  as  a
ring  we're  thinking  of  it  as  a  module
now  so  if  you  take  a  product  of  copies
of  z  uh  you  base  change  it  to  Z  bracket
T  and  then  you  solidify  you  get  well  a
product  of  copies  of  Z  bracket
t
um
so  maybe  a  little  remark  about  an
interpretation  of  it  so  I  said  solid  Z
was  kind  of  analogous  to  complete  non
archimedian  topological  ailan  groups  non
archimedian  means  Say  by  definition  that
there's  a  basis  of  neighborhoods  of  the
identity  consisting  of  a  billion
subgroups
um  so  what  would  be  the  analogous
interpretation  of  solid  ZT  it  would  be
that  you  have  a  as  you  have  a  complete
non-  archimedian  thing  where  moreover
there's  a  basis  of  neighborhoods
consisting  of  ZT  submodules
and  you  you  can  think  of  I  mean  the  and
for  a  basic  example  of  a  a  ZT  a  ZT
module  which  is  non  archimedian  but  does
not  satisfy  that  property  you  can  think
of  this  ring
um  you  can  not  find  a  basis  of
neighborhoods  of  zero  which  are  stable
under  multiplication  by  T  because
they're  stable  under  multiplication  by  T
inverse  instead  and  so  but  in  some  sense
the  the  theorem  is  that  if  you  kill  just
that  one  guy  then  you've  explained  the
difference  between  the  two
Notions  um
okay
so  so  in  the  previous  Theory  without
light  it  was  a  little  bit  different  it
was  not  defined
using  this  P
but  well  okay  it  basically  was  done  done
this  way  uh  I  mean  maybe  we  didn't  make
this
explicit  and  we  maybe  more  talked  about
uh  just  this  but  we  definitely  talked
about  this  um  yeah  but  it  does  make  it
more  clear  to  to  to  think  of  it  that
that  way  with  the
p  okay  so  the
proof  yes  confused  about  the  ZT  itself
is  ZT  solid  ah  so  it  is  that's  part  of
the  that's  part  of  the  theorem  because
I'm  claiming  in  particular  that  this  is
ZT  solid  and  and  ZT  is  a  retract  of  this
so  it's  something  that  we  need  to  prove
and  we  will  prove  it  but  it's  not  solid
what  ZT  is  UN  solid  ZT  oh  no  it  is  it  is
so  as  an  ailan  group  you  mean  or  yes  no
no  every  discrete  ailan  group  is  solid
because  it's  generated  under  co-  limits
by  Z  no  it's  an  important  point  Thank
you  for  for  bringing  it  up
yeah  other
questions  okay  and  Z  ZT  power  series  is
also  solid  ZT  power  series  is  also  solid
yes  yes  because  well  because  it's  a
limit  of  things  I  mean  you  can  build  it
from  limits  and  Co  limits  from  ZT  so
it's
yeah  all
right  uh  so  for  the  proof  uh  well  the
all  of  these
properties  so  all  except  the
last
those  are  that's  exactly  the
same  those  the  arguments  for  all  of
those  things  were  completely  formal  just
based  on  the  fact  that  you  have  this
internally  projective  object  p  and
you're  asking  that  some  endomorphism  of
internal  HS  from  P  to  M  become  an
isomorphism  that  was  all  that  Peter  used
when  he  was  proving  the  analog  of  these
claims  for  uh  solid  Z  so  the  fact  that
we  have  a  good  formal  the  theory  is
already  contained  in  there  and  then  in
Peter's  lecture  the  hard  part  was
identifying  the  free  modules  that  when
you  solidify  this  sequence  space  p  that
you  actually  just  get  a  product  of
copies  of  Z  fills  up  the  whole
thing  thankfully  uh  so  that  that  part  is
actually  going  to  be  easier  here  because
we  already  have  solid  Z  and  um  and
basically  thanks  to  this  interpretation
of  killing  some
object  so
so  the  claim  uh  the  key  claim  is
that
um  the  that  you  can  give  an  explicit
formula  for  this  derived
solidification  uh
so  I  apologize  for  the  janky  notation
now  this  looks  really  bad  uh  D  derived
derived  solidification  with  respect  to  T
so  uh  that  this  is  just  an  internal  ROM
over  z  braet
t  uh
from  the  quotient  of  this  ring  by  this
ring  to
M  and  note  that
this  this  is  this  does  have  the  required
map  from  M  uh  because  m  is
ROM  over  oh
sorry  oh  sorry  minus  one  uh  shift  by
minus  one
uh  soor  I'm  just  trying  to  read  what
that  last  thing  says  that  that's  and  m  t
t²  yeah  l  so  derived  solidification  yeah
the  deriv  functor  of  this  thing  here
yeah  which  I  I  kind  of  prefer  actually
to  think  of  as  just  the  derived  analog
of  it
um
okay
and  the  AR  underline  is  either  in  all
condense  or  only  in
solid  uh  it's  the  same  so  basically
because  of
this  I  mean  they  yeah  the  derived  the
the  so  when  you  compute  of  course  you
have  Gren  the  categories  so  you  have  a
well  defined  even  Rec  bounded  complexes
but  how  do  you  check  exactly  I  think
Peter  must  have  briefly  said  something  I
don't  remember  but  how  do  you  check  that
the  the  AR  or  underline  are  the  same
for  some  Junction  arum  I  don't  remember
exactly  I  mean  you  basically  I  don't
know
I  like  I  said  I  I  sort  of  tend  to  think
on  the  derived  level  from  the  beginning
but  the  basic  point  is  that  everything
has  a  resolution  by
these  uh  internally  projective  guys  and
then  on  those  they're  the  same  and  then
it's  some  derived  limit  of  that  and  it's
just  I  di  internally  projective  in  h  in
uh  in  all  condensed  in  light  condensed  I
mean  so  but  it's  really  not  that's
that's  really  not  necessary  either  I
mean  this  also  works  in  the  in  all
condensed  so  I  mean  I  I  claim  it's
formal  and  I  also  claim  I  don't  want  to
get  into  the  detail  right  now  so  uh  yeah
so  let's  maybe  discuss  after  if  you
still  have
questions  um  okay  uh  right  so  the  proof
of
claim  so  this  follows
formally
from  uh  the  fact  that  this  ring  Z  power
series  T  inverse  uh  is  item  potent
over
uh  over
ZT  which  follows  from  the  very  first
description  I  gave  of  it  as  the  base
change  of  the  we  check  that  the  power
Series  ring  is  item  potent  and  we  put
that  at  Infinity  um  and  the  item  potency
is
preserved  so  for  example  using  so  that
the  item  potency  of  that  so  what  is  this
thing  here  it's  just  the  the  homotopy
fiber  of  the  inclusion  of  this  into  this
and  this  is  the  the  base  ring  and  you
can  easily  check  that  item  this  item
potency  is  equivalently  to  equivalent  to
the  derived  item  potency  of  this  object
and  that  means  that  if  you  take  this
expression  and  you  apply  it  again  you
get  the  same  thing  back  so  that's  kind
of  an  item  potent  operation  and  again
the  same  item  potent  will  prove  to  you
that  if  you  take  this  operation  and  you
Arham  from  this  guy  uh  you  get  zero  so
this  thing  is  ZT  solid  and  that's  not
quite  everything  you  need  to  check  but
it's  basically  everything  you  need  to
check  it's  an  item  potent  operation  and
it  uh  m  is  solid  if  and  only  if  this  map
is  an
isomorphism  um
so  it's  a  it's  just  completely  formal
and  I  won't  write  out  all  of  the  all  of
the
details  um  so  maybe  I'll  just  say  note
so
this
yeah
okay  so  that's  a  a  formula  for  the
derived  solidification  of  a  general  Z
bracket  T  module  but  uh  something  nice
happens  if  your  Z  bracket  T  so  we  still
haven't  proved  the  claim  um  but
something  nice  happens  if  your  Z  bracket
T  module  is  base  changed  from  a  a  solid
Z  module
so  and  next
claim  uh  is  that  the  functor
from
the  sort  of  pullback  functor  from  solid
Z  to  solid  ZT  so  sending  a  a  module  M  to
uh  you  take  M  tensor  Z  ZT  and  then  you
solidify  um  this  functor  uh  is  T
exact  uh
preserves  limits  and
colimits  and  it  sends
uh  z  uh  to
Z  to  uh
ZT  so  if  we  if  we  prove  this  then  we're
in  particular  we  in  particular  get  this
claim  because  this  is  of  that  this  is
that  functor  applied  to  product  of
copies  of  Z  I  claim  the  funter  commutes
with  products  so  then  we  could  just  try
to  it's  enough  to  understand  what
happens  with  Z  but  I  already  also
claimed  that  Z  goes  to  ZT  so  uh  we'd  be
done  so  the
proof  um  so  we  just  take
this  formula  for  the
solidification  um  and  we  plug  in  the
case  where  m  is  uh  also  induced  so  we
get  that
so  M  tensor
ZT  uh  LT  solid  this  is  the  same  thing
as  I'll  just  write  it  m  brackets  T  by
that  I  mean  m  t  for  ZZ  brackets  T  um  and
now  we're  Computing  an  rhom  over  ZT  so
the  way  to  do  that  is  to  disambiguate
the  two  occurrences  of  T  and  then
equalize  them  at  the  end  so  that's  uh
calculated
so  um  let's  say  Z  power  series  U  well
this  is  so  u  z  power  series  U
um
uh
uh  so  two  term  complex  I'll  write  it
vertically  so  to  speak  um  so  then  you
equalize  U  and  T  or  no  sorry  U  and  T
inverse  no  uh
yeah
um
uh  or  so  let  me  let  me  say  we  no  I
should  I  should  say  rather  that  we
equalize  T  and  the  and  the  and  the  shift
operator  on  here  which  is  induced  by
multiplication  uh  by  T  here  so  let's
say  let's  denote  that  by  shift  here  and
then  I'm  switching  variables  uh  from  T
to
U
um
okay
uh
but  uh  note  that  now  we're  taking  R  homs
in  solid  Z  but  this  is  one  of  our
compact  projective  guys  so  huming  our
humming  out  of  that  commutes  with  Co
limits  and  this  Mt  is  a  CO  limit  uh  it's
just  uh  so  we  can  put  that  out  here  like
this  and  then  we're  taking  uh  this  thing
bracket  T  and  we're  modding  out  by  T
minus  something  so  that  means  that  or
we're  not  modding  out  by  it  that  modding
out  would  be  the  co-  fiber  we're  doing
the  fiber  instead  but  we  also  had  a
shift  and  those  two  actually  cancel  and
in  the  end  what  you  get  is  that  this  is
ROM  over
z  uh  from  u  z  power  series  U  uh  to
M  and  that's  the  key
formula
oops  and  uh  we've  kind  of  along  the  way
we  sort  of  of  lost  the  ZT  module
structure  on  this  thing  uh  but  you  can
recover  it  it's  induced  by  uh  so  the
multiplication  by  T  is  here  given  by
sending  U  to  zero  uh  u2  to  u  u  Cub  to
u2  Etc  so  if  you  do  that  there  then
that  induces  an  endomorphism  here  and
that's  the  multiplication  by  T  on  this
resulting
thing  sorry  I  I  miss  what  you  say  so
what  what  is  the  isomorphic  to  this
green  uh  this  this  this  bit  a  somewhat
silly  notation  uh  is  homotopy  fiber  of
this  map  or  some  kind  of  shift  of  a
mapping  cone  and  that's  the  same  as  this
it's  also  the  same  as
this  okay  and  now  we're  done  basic  well
now  this  functor  again  this  is
internally  projective  in  solid  Z  so  this
fun  is  T  exact  and  it  preserves  limits
and  Co  limits  in  solid  Z  but  limits  and
Co  limits  in  solid  ZT  or  are  calculated
on  the  underlying  level  because  it's
just  part  of  a  module
category
um  and  then  the  last  claim  is  that  Z
goes  to  uh  just  the  usual  polinomial
ring  in  one  variable  but  what  happens
when  you  plug  in  Z  here  you're  taking  R
homs  from  this  product  of  copies  of  Z  to
Z  it  turns  into  a  direct  sum  of  copies
of
z  um  and  you  can  check  that  uh  what  you
get  is  just  the  usual  Z  bracket  T  even
with  the  ZT  module  structure  or  really
there's  a  natural  comparison  map  which
you  see  to  be  an
isomorphism  okay  so  there  was  this  some
someone  prove  it's  not  related  directly
to  the  theory  but  someone  proved  just
obr  and  gr  at  home  from  the  product
infinite  product  cop  of  Z  to  Z  is  a  is  a
direct  but  this  is  not  here  you  are
doing  it  in  this  is  easier  this  is
easier  yeah  yeah
yeah  okay
so  uh  we  can  also  do  other  examples  of
such  a  thing  so  let's  do  another  example
so  that  finishes  the  sorry  that  finishes
the  proof  of  the  theorem  so  we  now  have
a  grip  on  this  solid  ZT  Theory  and  I
want  to  advance  the  interpretation  that
the  solid  ZT  theory  is  is  like  uh  like
working  over  the  the  ZT  without  the
solidification  is  like  the  Aline  line
and  ZT  with  the  solidification  is  like
the  closed  unit  dis  that  was  kind  of  the
interpretation  that  I  started  with  um
but  let's  do  an  example  again  in  non-
archimedian  let's  base  change  to  a  non-
archimedian  field  and  see  what  happens
uh  so  let's  take  another  example  in
above  so  let's  look  at  QP  T  so  that's
functions  on  the  Aline  line  and  now  we
want  to  restrict  to  the  closed  unit  dis
in  the  sense  that  we've  just  described
so  we  take  the  T  solidification  of
this
um  so  we  this  is  an  instance  of  this
this  funter  here  um  and  I  just  said  this
functor  has  all  the  properties  in  the
world  so  it  commutes  with  co-  limits  so
I  can  again  take  the  T  to  the
outside  oh  sorry  the  one  over  P  to  the
outside  and  it  also  um  commutes  with
limits  so  I  can  take  the  the
ptic  so  I  can  take  limit  Over  N  of  Zod  P
to  the
NZ  bracket
t  uh  T  solidified  and  then  one  over  P  at
the
end  um  and  now  we're  applying  this
funter  there  to  Zod  P  to  the  N  but
that's  just  two  copies  of  Z  so  um  you
get  the  same  answer  as  for  Z  it's  just
discreet  so  this  is  just  inverse  limit
Over  N  of  Z  mod  P  to  the
NZ  bracket
t  uh  one  over
P  or  in  other  words  it's  ZT  P
completed  P  completed  one  over  p  and
these  are  the
functions  uh  on  the  closed  unit
dis  kind  of  as  as
desired  okay  so  maybe  it's  I'll  take  a
five  minute
break  uh  maybe  I  forgot  to  give  the
interpretation  of  the  coefficients  here
so  again  this  is  some  power  series  with
coefficients  in  QP  but  here  the
coefficients  uh  tend  to  zero
periodically
um  so  it's  a  smaller  ring  of  functions
so  it  converges  on  a  bigger  region  and
you're  allowed  to  be  now  on  the  bound
well  boundary  so  to  speak  I
mean  yeah  it's  yeah
yeah
um  okay  now  I  want  to  let's  play  a  fun
game
uh  so  I've  been  talking  about  the  open
unit  disc  and  the  closed  unit  disc  or
maybe  I'll  say  yeah  or  the  closed  unit
disc  and  the  complement  of  the  closed
unit  disc  but  in  it's  kind  of  there's
always  this  fun  question  of  what  exactly
is  open  and  what  exactly  is  closed
because  in  in  rigid  geometry  the  closed
unit  disc  is  considered  as
open  um  and  the  open  unit  dis  well  I
suppose  it's  also  open  in  a  sense  but
it's  not  quasi  Compact
and
um  so  let's  play  a  game  also  you  can  use
a  belov  Viewpoint  which  changes  somehow
is  another  way  another  then  it  is  closed
not  way  yeah  so  I  claim  that  our
formalism  is  going  to  give  a  definite
answer  to  this  question  what's  open  and
what's  closed  yeah  okay  well  all  right
um  where  was  I  oh  yeah  what's  closed  and
what's
open  so  let's  let's  take  a  look  at  the
the
the  the  categories  you  could  say  we've
attached  to  the  various  geometric
objects  so  for  the  apine  line  we  said
we're  just  thinking  of  it  as  the
polinomial  ring  in  one  variable  but  then
the  linear  algebra  category  is  just  ZT
modules  and  solid  ailan  groups  and  let
me  actually  go  to  the  derived
perspective
so
uh  um  but  then  inside  there  we  had  this
derived  solid  so  we  had  D  of  a
solid  uh
ZT  um  but  actually  it's  better  to  think
of  that  as  a  as  a  quotient  really  um
because  the  the  symmetric  monoidal
functor  is  the  derived  solidification
functor  uh  which  goes  like  this
so  uh  but  then  on  the  other  hand  we  had
the  complement  uh  and  the  complement
was  uh  the  open  unit  disc  or  rather  the
open  unit  just  move  to  Infinity  so
that's  this  thing  and  that  corresponds
to
d  uh
mod  uh  luron  Series  in  t
inverse  um  solid  Z  I  think  someone  asked
before  but  I  for  about  that  you  had  the
solidification  which  is  one  ad  there
should  be  another  on  the  other  side  wait
wait  wait  wait  uh  so  then  and  then  this
is  just  the  natural  includ  well  it's
it's  a  forgetful  functor  a  priori  you're
forgetting  the  extra  module  structure
but  it's  in  fact  a  fully  faithful
inclusion  because  of  the  item  potency  uh
the  item  potency  of  this  thing  means
that  there's  at  most  one  ZT  module
structure  on  any  ZT  Z  lauron  series  T
inverse  module  structure  on  an  e  z
module  um  so  this  is  actually  like  a
localization  sequence  or  what  have  you
so  this  is  the  symmetric  monal  quotient
of  this  by  this  thing  which  is  kind  of
an  ideal  in  there  um  and  what  kind  of
adjoins  do  we  have  and  which  of  them  are
well  behaved  and  so  on  so  well  we  know
that  this  one  has  a  right  ad  joint  uh
which  is  given  by  the
inclusion
um  but  it  also  has  a  left  ad  joint
so
so
um  and  the  left  ad  joint  is  given  by
sending  M  uh  to
uh  I  guess  the  the  fiber  so  you  take  M
and  then  you  base  change  it  to
Infinity
uh  and  you  take  that  uh  cone  diagram
here  so  it's  the  fiber  of  uh  M  mapping
to  m  t  over  ZT  with  the  Z  lant  series  T
inverse  so  uh  and  this  left  ad  joint  is
in  a  sense  better  behaved  than  the  right
ad  joint  the  the  naive  inclusion  and  by
the  the  measurement  that  it's  better
behaved  well  they  both  commute  with  co-
limits  but  but  this  one  satisfies  a
projection  formula  with  respect  to  this
fundamental  functor  which  the  symmetric
monoidal  functor  so  this  left  ad  joint
is  kind  of  linear  over  these  two
symmetric  monoidal  categories  so  this  is
better  so  uh  and  I  I'm  going  to  call
this  one  J  upper  star  and  this  one  J
lower  shriek  and  this  one  J  lower  star
this  inclusion
here
um  and  on  the  other  hand  for  here  uh  we
have  this  funter  here  which  I'll  call  I
lower  star  the  inclusion  there  uh  so  it
has  a  left  ad  joint  also  I  upper  star
which  is  just  the  base  change
functor  uh  which  is  symmetric  monoidal
so  that's  kind  of  the  the  more
fundamental  one  from  this  perspective
but  it  also  has  a  a  joint  I  upper  shriek
which  is  given  by  some
arom  um  but  it's  less  well  beh  paved
again  uh  because  here  this  one  satisfies
a  projection  formula  with  respect  to
this  one  so  this  is  a  linear  this
functor  is  linear  over  this  symmetric
monoidal  category  just  like  this  funter
is  linear  over  this  symmetric  monoidal
category  so  star  is  an  adint  the
inclusion  has  anint  on  it  does  but  that
one  we  don't  talk  about  yeah  now  this
was  the  question  I
yeah  yeah  um  because  already  this  one's
not  as  nice  as  I  mean  the  good  ad  joint
is  is  actually  up  here  so  um  so  the
interpretation  that  this  suggests  so
this  is  exactly  like
uh  uh  if  you  have  a  if  you  have  a
topological  Space
X  and  then  you  have  z  a  closed
subset  and  then  U  uh  the  complement  open
then  you  get  if  you  have  on  DED
categories  of
sheaves  uh  you  have  exactly  the  same
thing
where  uh  you  have  the  open  over  here  you
have  this  thing  satisfies  a  projection
formula  you  happen  to  have  this  other
guy  but  it's  not  as  well  behaved  um  and
then  you  have  I  lower  star  uh  from  d
shees
z  uh
z  um  I  upper  star  and  then  I  upper
shriek  so  it  satisfi  formally  speaking
it  behaves  exactly  the  same  way  and
actually  they're  both  special  cases  of
the  same  thing  which  is  you  have  a
symmetric  monoidal  category  um  and  you
have  an  item  potent  algebra  in  it  and  uh
it  it  generates  the  whole  situation  in
this  case  you  have  this  category  and
this  item  potent  algebra  here  you  have
this  category  and  then  the  push  forward
of  the  structure  sheath  from  the  closed
sub  or  push  forward  of  the  constant
sheath  from  the  closed  subset  and
it  completely  analogous
formalisms  um
so  that  tells  us  that  from  our
perspective  uh  the  closed  unit  disc
should  definitely  be  thought  of  as  open
and  the  open  unit  dis  should  definitely
be  thought  of  as  closed  so  as  far  as  I'm
concerned  the  question  is
settled  um  but  also  uh  this  this  another
funny  thing  is  that  if  you  think  about
this  perspective  it's  also  telling  you
that  zarisky  opens  and  usual  algebraic
geometry  should  be  thought  of  as  closed
and  I  think  that's  also  true
um  um  I  can  explain  why  people  want  to
know  uh  all
right  but  not  for
usual  you  don't  like  no  you  start  you
start  from  the  usual  theory  of  like
patching  of  sh  like  is  used  in  many
context
like  in  I  mean  it  was  used  but  then  the
for  usual  is  used  for  the  usual  topology
like  oh  the  risk  of  a  but  then  you
somehow  develop  it  further  and  decide
that  now  that  the  risk  opens
are  well  close
this  I  think  I  I'd  like  to  try  to
convince  you  but  but  not  right  now
uh  all
right
open  sorry  which  import
open  I  don't  understand  the
question  no  I  heard  the  question  I  don't
understand  it  what  because  you  said  the
closed  things  correspond  to  an  ENT
object  in  symmetric  monal  category  so
you  said  that  there  risky  open  is  like
closed  yeah  yeah  so  for  example  a
distinguished  open  is  just  the  ring  with
the  function  inverted  that's  item  potent
yeah
yeah  so  I  mean  it's  so  so  maybe  I  should
say  quasi  compact  opens  or  closed  it's
probably
yeah  the  deriv  direct  image  of  the
structure  ship  is  important  yeah  yeah  um
okay  all  right  so  now
uh  um  yeah  so  this  by  the  way  this
uh  yeah  so  this  this  we're  kind  of  going
to  be  guided  by  this  sort  of  thing  in  in
setting  up  the  definition  of  analytic
stack  and  so  on  like  the  idea  of  so  one
of  the  things  we  discovered  is  that  when
you  move  to  this  condensed  solid  what
whatever  context  that  you  you  actually
get  six  funter  formalisms  in  large
generality  on  derived  categories  of  so
to  speak  quasi  coherent  sheaves  and  um
and  they  have  I  mean  they  really  have
nice  interpretations  so  for  example  if
you  think  of  this  thing  as  being  what
sits  at  Infinity  then  it  makes  sense
that  this  is  extension  by  zero  because
you're  taking  your  sections  and  then
you're  killing  the  ones  that  live  near
the  boundary  and  it  really  all  just
plays  quite  nicely  and  we  showed  how  to
give  proofs  of  things  like  sidity  and  so
on  using  these  formalisms  and  um  so  we
take  this  perspective  seriously  that  the
derived  categories  and  the  funs  between
them  are  going  to  dictate  to  us  what
what  the  geometry  looks
like  um  there  was  oh  there's  another
thing  I  remembered  in  the  break  that  I
forgot  to  mention  is  that  so  I  said  that
this  is  T  exact  and  preserves  limits  in
Co  limits  but  I  want  to  caution  you  that
this  this  uh  T  solidification  from  from
solid  ZT  is  not  t
exact  uh  but  it's  only  uh  so  it's  only
off  by  one  so  so  um  I  said  it  was  given
by  R  H  humming  from  this  object  which
has  a  two-term  compact  resolution  so
it's  only  off  by  one  from  being  T  exact
so  it  sends  everything  connective  goes
to  something  connective  everything
anti-c  connective  goes  to  something
that's  at  most  a  one  shift  of  something
anti-c  connective  um  so  it's  not  deriv
t-  solidification  is  not  exactly  exact
but  it's  it's  very  controlled  but  the
composed  map  is  yes  exactly  so  each  the
first  map  is  the  exact  and  the  composed
map  and  the
second  map  is  not  except  that's  right
yes  it's  a  bit
funny  all
right  um  okay  so  now  I'm  going  to  say
something  like  I  want  to  motivate  what
I'm  going  to  do  in  the  rest
of  so  I'll  say
Vista  so  where  what  is  one  place  we're
probably  going  to  go  is  uh  so  we  want  to
look  at  solid
rings  so
IE  commutative  algebra  objects  in  this
tensor  category  solid  z
um
uh  and  then  we  want  to  see  uh  so  sort  of
generalizing  our  discussion  when  R  is
the  a  r  is  z  bracket  T  we  want  to  see
some  subsets  like  closed  unit  discs  open
unit  discs  things  like  this  but  of
course  when  you  have  a  big  algebra
you'll  get  many  more  such  subsets  and  we
want  to  organize  the  organize  uh  what
you  see  in  a  in  a  nice  way  and  it's  not
necessarily  the  most  General  thing  you
can  do  but  what  we're  going  to  do
basically  is  just  take  the  things  you
things  you  see  over  Z  bracket  T  and  kind
of  Base  change  them  along  all  possible
maps  from  ZT  to  r  that  is  uh  all
possible  functions  in  R  um  and  what
we're  going  to  end  up  with  is  some  is
the  statement  this  this  derived  category
of  uh  mod  R  solid
Z  even  this  tensor  category
uh
localizes  uh
along  the  valuative  Spectrum  the
spectrum  of  all
valuations  uh  of  well  the  underlying
usual  commutative  ring  of  this  solid
commutative
ring  localizes  so  you  take  the  valuative
spectrum  of  r  ah  just  of  R  viewed  as  a
as  a  discrete  ring  yeah  as  a  discrete
ring  and  the  valuative  spectrum
is  okay  it's  like  a  residue  field  and
valuation
without  the  valuation  doesn't  have  to  be
integral  on  R  just  any  evaluation  yes
that's  right  it's  a  big  which
is  and  it  has  the  topology  which  is  okay
I  know  the  topology  but  it's  kind  of  the
it
was  yeah  is  topology  which  induces  the
topology  and  things  like  spa  and  so  that
yeah  but  it's  kind  of  because  sometimes
one  gets  Ms  which  are  not  SP  in  this
okay  but  okay  it's  possible  to  consider
so  it's  yeah  yeah  it's  possible  to
consider  it  indeed  and  it's  you  know  a
priori  more  General  okay  and  so  so  I  I
won't  remind  you  what  evaluation  is
right  now  but  I  will  but  I  will  remind
you  that  this  space  is  similar  to  usual
spectrum  of  a  commutative  ring  in  that
it  has  a  basin  basis  of  uh  quasi  compact
opens  I  mean  it's  a  spectral  space  and
it  even  has  a  particularly  nice  basis
which  is  in  some  sense  analogous  to  the
distinguished  apine  opens  in  algebraic
geometry  and  while  a  distinguished  Aline
open  in  algebraic  geometry  is
parameterized  by  a  single  element  of  the
ring  uh  here  you  kind  of  have  to  take  a
little  bit  more  data  so  so  this  uh  so
the  basic
opens  are  the  so-called  rational
opens  uh  so  X  you  have  to  take  finitely
many  functions  uh  and  then  you  form  this
thing  here  where  the  interpretation  is
that  uh  you  invert  G  so  we're  inside  the
zisy  distinguished  open  for  G  but  then
we  shrink  further  by  requiring  that  the
the  sort  of  F1  be  the  the  absolute  value
of  F1  be  less  than  or  equal  to  the
absolute  value  of  G  the  absolute  value
of  f  to  all  of  the  absolute  values  of
these  guys  should  be  less  than  or  equal
to  the  absolute  value  of
G  um
G  yes  that's  what  I  said  yeah  g  yeah  so
so
uh
right  and  so  I'm  saying  that  this
localizes  on  this  meaning  you  have
actually  a  sheath  of  categories  a  sheath
of  symmetric  monal  categories  um  and
what  you're  going  to  attach  to  this
thing  uh  is  a  version  of  the  solid
theory  of  the  solid  Theory  so  you  look
at  this  those  m  in  D  mod  are  solid
Z  such  that  well  first  you  want  to  say
that  multiplication  by  G  on  M  is  an
isomorphism  and  second  you  want  to  say
that  if  you  take  this  uh  internal  homs
from  P  to  M  uh  and  internal  HS  from  P  to
M  uh  and  you  take  fi  *  shift  minus  one
this  should  be  an  isomorphism  for  all  I
so  you  can  or  in  other  words  uh  oh  fi
over  G
sorry  so  in  other  words  you  want  so  you
think  of  these  all  these  guys  as  Maps
from  say  spec  spec  R  so  to  speak  to  the
apine  line  then  you  want  that  g  lands  in
the  uh  standard  you  know  zisy  nonzero
Locus  and  you  want  that  the  FI  over  G  uh
land  inside  the  closed  unit
disc
so
so
so
okay  so  I'm  not  going  to  go  into  details
about  that  um
but  uh  where  where  should  I  go  now  I
don't  know  maybe
here  this  as  PB  just  refers  to  the  local
corespond
to
uh  I  mean  yeah  the  only  I  mean  I  don't
uh  you  can  think  of  it  as  a  local  but
it's  also  a  topological  space
um
yeah  and  the  points  have  a  nice
description  and  so
on  uh  okay  so  what  do  I  want  to  do  today
well  or  part  part  maybe  partly  do  today
so  in  particular  so  I'm  claiming  this
category  local  izes  along  the  space  and
in  particular  you  get  a  structure
sheath
uh  uh  on  the
space  and  um  what  I  want  to  do  in  the
next  bit  so  goal  for  rest  of
lecture
is  make
this  this  structure  of  Chief
explicit  and  uh  compare  well  maybe
probably  start  to  compare  to  hub's
Theory  so  we're  we're  eventually  going
to  produce  this  data  by
very  easy  formal  means  but  it  requires
some  language  and  setup  so  we  can't  do
it  yet  but  I  want  to  make  at  least  this
part  of  it  explicit  uh  already  at  the
beginning  okay
so  let's
see  so  um  let  me  make  a  so  ah  let's
let's  let's  start  with  this  generality
so  we  have  a  solid
ring
um  um  and  let's  take  an  element  F  in  r
or  really  I  should  say  in  the  underlying
discret  ring  of  R  in  case  there's
ambiguity  um  and  that  in  particular  well
that  that  gives  you  a  map  from  ZT  to  R
which  sends  T  to  f
um  and  then  we  can  sort  of  see  how  these
Loi  that  we've  identified  uh  can  be  or
correspond  to  properties  of  R  so  let  me
make  a
definition  um
so  uh  f  is  topologically  nil
potent  uh
if  uh  this  map  Factor
through  this  homomorphism  I  should  say
factors  through  the  power  Series
ring  and  uh  f  is  power
bounded  uh  if  uh  well  I  want  to  say  that
if  if  this  map  well  the  map
geometrically  factors  through  the  closed
unit  dis  um  but  the  way  to  say  that  is
uh  if  so  let's  say  if  R  is  actually  so  R
is  a  algebra  over  ZT  in  particular  it's
a  module  over  ZT  and  we  can  ask  that  it
be
solid  in  the  first  definition  the
factorization  is  unique  yes  it's  Unique
if  it  exists  because  of  this  item
potency  so  that's  an  interesting  fact
actually  this  is  also  the  free  module  on
a  null  sequence  and  so
um  so  even  though  there  are  sort  of
house  solid  aing  groups  that  have  like
non-h  house  dworf  Behavior  Uh  still  uh
kind  of  this  limit  is  unique  if  it
exists
um
okay  so  basically  in  the  definition  of
solid  you're  imposing  that  certain
limits  exist  uniquely  even  even  though
you  have  non-house  dorf
Behavior  Uh  okay  so  that's  the  same
thing  as  saying  that  uh  if  you  take  H
from  P  to  R
h  p  to  r  f  *  shift  minus  identity  that
this  is  isomorphism  that's
just  oh  someone's  talking  hello
yes  yes  the  first  condition  lit  saying
that  St  nor  sequence  which  is  the  PO  of
and  zero  so  it's  literally  yeah  yeah
it's  I  mean  yeah  I  was  going  to  I  mean  I
was  going  to  explain  the  relation  with
classical  definitions  but  it's  it's  yeah
it's  quite  immediate  for  this  one  that
it's  the  same  as  the  classical
definition  so  maybe  I'll  just  repeat
what  Peter  said  uh  so  this  is  the  same
thing  as  uh  this  P  thing
solidified  so  yeah  this  map  sends  T  to  F
but  it's  a  ring  homomorphism  so  it  sends
t  to  the  N  to  F  to  the  N  um  and  then
saying  that  that  factors  through  p  is
the  same  thing  as  saying  that  this
sequence  uh  so  1  f  f  s  so  on  extends  to
a  null  sequence  which  is  B  basally
exactly  the  same  thing  as  being
topologically  nil
potent
um  yeah  the  new  phenomenon  in  the  solid
case  is  that  it's  Unique  if  it  exists
such  a  limit  that's  kind  of
fun  uh
okay  it  factors  through  in  the  sense  of
Rings  or  in  the  sense  of  the
uniqueness  factorization  okay  you  can
ask  factorization  as  a  condens  group  or
in  the  sense  of  rings  in  the  sense  of
rings  yeah  so  you  could  have  another
factorization  just  in  the  sense  of
condens  yes  that's  true  that's  true  yeah
yeah  yeah  yeah  I  was  maybe  being  too
imprecise  earlier  thank  you  yeah  or  in
the  sense  of  ZT  modules  is  enough  or  I'm
not  sure  I  don't  think  so  so  yeah  thank
you  for  the  thank  you  for  the  comment
yeah  just  as  Rings  just  as
Rings  okay  um  so
all  right  so  here's  a  a  Lemma  giving
basic  properties  ah  so  no  sorry  let  me
say
so  so  we  write  uh
r0  so  this  is  again  kind  of  standard
notation  in  Huber's  Theory  so  set  of
power  bounded
elements  and  our
z0  uh  set  of  topologically  nilp  putting
elements
which  is  a  Subs  well  yeah  I'm  going  to
prove  that  uh  yeah  so  Lemma  is  that  R
zero  uh  inside  this  ring  here  uh  is  an
integrally
closed  sub
ring  and  uh  R  z0  first  of  all  is
contained  in  r0  and  it  is  a  radical
ideal
so  I'm  still  confused  about  this  n
sequence  you  something  is  n  sequence  in
the  just  the  sense  that  the  map  extends
to  a  another  sequence  then  you  don't
know  that  it  factors  as  a  ring  that  this
is  a  ring  map  that's  that  seems  correct
to  me  yes  okay  so  it  is  doesn't  mean
topologically  important  in  your
definition  in  your  sense  but  on  the
other  hand  maybe  uh  yeah  no  I  mean  yeah
you're  right  uh  yeah
well  it's  the
same  it's  okay  that's  a  good  point  but
if  you  have  something  that  comes  from  a
uh  if  you  have  something  that  comes  from
a  house  doorf  topological  ring  uh  for
example  what  if  if  the  target  is  quasi
separated  then  independently  of  asking
about  the  algebra  structures  then  the
limit  is  unique  if  it  exists
so  yeah  oh  quasi  separated  was  this  uh
analog  of  how  DF  in  the  in  the  condensed
setting  so  so  still  if  you  start  with
yeah
I  then  it  is  automatically  a  ring  map  if
you  yes  because  of  density
basically  okay  so  we  are  not  sure
whether  topologically  OS  in  the  null
sequence  sense  is  always  the  same  as
this  well  okay  I  mean
house  door  I  mean  yeah  all  right
um  so  hello  um  okay  so
proof  so  why  is  this  uh  a  subring  um
well  I  guess  maybe  the  first  thing  to
check  is  that  it  has  a  unit  that's  part
of  what  I  mean  by
subing  um  but  if  you  look  at  it  that's
the  definition  of  solid  that's  one  way
of  saying
it
so  put  f  equals  1  here
okay  um  okay  now  I  want  to  show  that  if
f  and  g  are  in  there  uh  I'll  prove  that
it's  a  subing  by  showing  that  any
polinomial  so  if  f  is
a
uh  and  if  you  apply  any  polinomial  to  f
and  g  then  you're  still
uh  in  there  uh  so  how  can  we  do  this  so
kind  of  maybe  there's  different  ways  of
doing  it  but  I  think  the  cleverest  one
is  maybe  so  we  so  we  can  look  at  the  map
from  the  polinomial  ring  in  two
generators  to  R  which  sends  X  to  F  and  Y
to
G  and  our  hypothesis  is  that  R  is  a
solid  ZX  module  and  it's  a  solid  zy
module  and  what  we  want  to  conclude  is
that  for  any  map  here  uh  from
the  polinomial  ring  in  one  generator  T
that  R  is  a  solid  ZT  module
okay
so  so  we  can  so  first
resolve  uh  R  by  it's  it's  a  solid  it's
definitely  solid  as  an  ailan  group  so  we
can  resolve  resolve  resolve  R  by  these
guys  we  can  make  a  res  resolution  of  r
by  direct  sums  of  our  compact  projective
generators  um  but  we  know  that  R  is  a  z
bracket  X
solid  uh
so  but  uh  then  so  we  so  if  we
solidify  uh  if  we  if  we  solidify  with
respect  to  X  then  what  does  this  turn
into  it  turns  into  direct  sum  um  ah
sorry  R  is  also  a  ah  I'm  sorry  I'm  sorry
we  should  I  should  put  I  should  give
myself  my  variables  uh  X  and  Y  cuz  R  was
also  a  module  over  Z  bracket  XY  um  and
now  I  solidify  with  respect  to  X  um  and
I  get  direct  sum  of  product  of  copies  of
z  braet
x  bracket  y's  by  the  properties  of
derived  solidification  that  I  proved
earlier  but  then  and  then  it  doesn't
change  R  so  that's  still  a  resolution  of
R  and  then  we  solidify  uh  with  respect
to  Y  and  we  get  a  direct  sum  of  produ  of
copies  of  Z  bracket  X  Y  Again  by  the
same  uh
reasoning  so  uh  in  total  we  see  that  R
can  be  resolved  by  these
guys  but  each  of  these  is  clearly  uh  but
each  of  these  is  solid  over  ZT  because
it's  a  limit  it's  a  co-limit  of  limits
of  things  which  are  solid  over
ZT
this  is  solid  over  ZT  because  it's
discreet  um  and  then  this  is  a  product
and  that's  a  direct  sum  so  in  total  it's
it's  solid  over
ZT
okay
uh  so  I  didn't  directly  use  the
definition  of  solid  instead  I  used  the
description  of  these
generators
so
um  right  now  how  about  showing  that
our  uh  the  topologically  nil  potent
elements  are  all  power  bounded  um  you
can  use  a  very  similar  argument  so  what
we  have  is  a  map  from  uh  power  series  X
to  R  which  sends  X  to
f  um  which  means  that  R  becomes  a  Z
power  series  X
module  uh  so  now  if  we  resolve  R  by
direct  sums  of  products  of  copies  of  z
uh  uh  we're  allowed  to  tensor  this  with
uh  Z  power  series  T  but  we  know  how  to
compute  this  tensor  product  they're  both
just  products  of  copies  of  Z  so  this  is
direct  sum  of
uh  product  copies  of  Z  power  series  T
and  this  is  also
solid  over  ZT
all  um  so  we  conclude  that  R  is  solid
over  ZX  oh  t  i  I  switched  to  T  somehow
switch  to  X  somehow  so  so  we  conclude
that  this  composition  uh  R  is  actually
solid  as  as  a  ZT
module
oh  I  didn't  prove  oh  I  forgot  to  prove
it's  integrally  closed  oh  I'm  sorry
uh  uh  oh  let's  let's  do  that  uh
so  so  it's  again  a  very  similar  argument
so  let's  say  that  we  have  a  an  equation
of  the  form
uh
um  so  where  all  CI  are  power
bounded  um  so  again  we  just  make  the
universal  things  we  have  Z  bracket  x0  up
to  xn  minus  one  and  over  that  we  have
the  ring  where  you  adjoin  another
variable  uh  let's  call  it  t  and  then  you
set  the  equation
uh
uh
yeah  and  we  have  our  solid  ring  R  and  by
hypothesis  we  have  a  map  here  such  that
when  we  compose  to  here  uh  R  becomes  a
solid  module  over  each  of  the
variables  and  what  we  want  to  show  is
that  when  you  compose  here  uh  it  becomes
a  solid  module  over  this
variable
so  uh  I'll  just  say  it  quickly  in  words
you  use  the  same  trick  so  you  resolve  R
first  as  just  a  zx0  up  to  xn  minus  one
module  in  solid  Aon  groups  and  you
solidify  with  respect  to  each  of  the
variables  you  find  yourself  built  out  of
product  of  copies  of  this  ring  but
you're  also  a  module  over  this  so  you
can  tensor  up  to  this  but  that's  a  a
finite  free  module  over  that  ring  so
then  that  will  just  go  inside  the
products  and  you  find  that  you're
resolved  by  a  direct  sum  of  products  of
copies  of  these  guys  and  then  because
this  individually  is  solid  and  solid  is
closed  under  limits  and  colimits  you
deduce  that  that's  solid  as  well  so  the
key  here  is  just  that  this  is  finite  as
a  module  over  this  so  that  tensoring
with  it  you  can  bring  inside  the  product
finite  free  yeah
although  well  that's  not  really
necessary  the  Rings  no  Theory  and  you
can
resolve  yeah  finite  would  be  enough  in
fact
yeah  you  mean  finite  with  all  with
representation  by  finite  well  it's  an
ethereum  ring  I  mean  yeah
yeah  okay
um
so  uh  oh  and  the  rest  so
yeah  I  mean  I'll  leave  the  rest  to  you
it's  completely  analogous  arguments  like
why  uh  why  it's  an  ideal  and  why  it's
a  why  it's  a  a  radical  ideal  even  um  so
fun  exercises  in  that  style  of
argument
so  um  so  now  we  can  describe  this
structure
sheath  uh  so  now  suppose  we  have  a  again
a  solid  ring
and  then  we  have  G  and  F1  G  comma  fub1
comma  FN  in  R
solid
um  then  the  claim  is
that  well  so  now  yeah  so  there  exists  a
universal  or  an
initial
solid
ring
uh
so  and  I'll  write  I  want  to  just  I  don't
want  to  use  exactly  the  same  notation  as
who  so  I'll  decorate  it  with  a  solid  in
the  super  stri  script  um  with  a  map  from
R  uh  such  that  uh  so  first  of  all  G
becomes
invertible  uh  in  there
and  the  second  thing  is  that  uh  fi  over
G  is  power
bounded  uh  in
R  uh  for  all
R  so  that  kind  of  uh  encodes  the  idea  I
was  talking  about  with  you  want  fi  over
G  to  go  to  the  closed  unit  dis  you  want
GI  to  be
invertible  um  so  the  proof
is  uh  you  can  just  construct  the  guy  so
um  you  can  start  with
so  well  you  could  first  invert  G  but  I'd
rather  maybe  I'll  do  invert  g  at  the  end
so  what  you  can  do  is  you  can  take  R  and
adjoin  just  polinomial  variables  X1  to
xn  and  then  solidify  with  respect  to  all
of
them  which  recall  um  does  something  when
R  is  not  discret  remember  we  had  the
example  of  one  variable  and  R  was  QP
then  this  gave  us  the  Tate  algebra
um  uh  and  then  we  can  say  that  these
variables  are  supposed  to  be  fi  over  G
so  maybe  uh  GX  IUS  F  so  G  gx1  minus  fub1
uh  gx2  -
F2  gxn  minus  FN  and  then  it  doesn't
matter  at  which  point  you  invert  G  but
let's  do  it  at  the
end  and  um  this  kind  of  obviously
satisfies  the  correct  Universal  Property
so  we  first  we  freely  adjoin  solid
variables  uh  then  we  impose  the
relations  which  guarantee  that  kind  of
thing  there's  one  one  small  thing  to
check  which  is  that  after  you  take  this
in  which  these  are  definitely  solid  and
then  you  do  these  operations  you  need  to
see  it's  still  solid  but  that's  because
it's  all  Co
limits
um
okay  uh
so  um  so  this  is  the  kind  of  thing
you're  looking
at  um  and  now  I  want  to  make  a  a  caution
here
uh  yeah  know  exactly  so  just  a  sec  so
this  is  this  is  not
exactly  the  value  of  the  structure
chief  on  uh  X  F1  FN  over
G  it
is  pi  0  of  the  value  of  the  structure
so  you  have
AED  yes  so  I  said  that  this  I  when  I
remember  when  I  was  discussing  this  six
funter  formalism  and  this  localizing  the
category  I  was  always  careful  to  say  it
was  the  derived  category  and  in  fact
these  things  really  only  work  uh  these
six  funter  formalisms  and  so  on  really
only  work  at  the  derived  level  and  what
that  means  is  in  what  you're
producing  a  prior  is  a  derived  chath  not
an  ordinary
chath  um  so  so  that's  uh  okay  but  but
now  I'll  say  but  in  in  most  practical
cases  almost  all  practical
cases  uh  all  Pi  I  equals  z  for  I  bigger
than
zero  so  in  practice  it  doesn't  seem  to
cause  trouble  but  it's  an  important
thing  to  keep  in  mind  and  the  second
warning
the  second  warning  is
a  so  even
if  R  is  an  is  is  very  nice  some  kind  of
Huber
ring  uh  then  uh  this  quotient  uh  may  not
be
uh  so  this  uh
this  may  not  be  uh  quasi
separated  so  um  okay  so  it's  a  kind  of  a
nonous  of  quo  of  of  of  restricted  form
of  Series  in  gender  yes  exactly  so  this
this  will  always  be  some  this  will
always  be  the  usual  Tate  algebra  and
many  generators  that  follows  basically
from  the  arguments  we  gave  for  any  sort
of  Huber  ring  R  but  then  when  you  take
this  quotient  the  ideal  generated  by
these  elements  might  not  be  closed  so  in
principle  you  could  be  having  a
non-house  dwarf  quotient  here  um  so  you
start  when  you  say  Huber  ring  you  mean
complete  hu  yes  I  mean  complete  Huber
ring  thank  you  yes  yes  um  but  again  but
this  theory  is  defined  for  complete
ubering  as  well  or  no  this  theory  is
only  defined  for  complete  Huber  Rings
the  theory  I'm  discussing  because  you
cannot  Define  the  condensed  of  any
topological  thing  you  can  Define  it  but
it  won't  be  solid  unless  unless  the
thing  is  complete  in  general  ah  okay  you
wanton  Sol  it  so  you  won't  complete  and
then  the  other  guys  when  you  do  this  it
is  is  still  solid  but  doesn't  come  from
it's  kind  of  nonous  of  yes
yes  yeah  but  again  uh  in  almost  all
practical  cases  or  maybe  all  practical
cases  uh  it  is  quasi
separated  I.E  house
dwarf
um  so  and  then  so  so  what  I  was  trying
to  aim  for  is  the  relation  to  hu's
Theory  so  so  I  but  I  I  so  I  Pro  I  guess
I  probably  won't  get  there
uh  so  if  so  I  think  we'll  probably  we'll
definitely  discuss  this  more  detail
later  but  just  as  a  preview  so  if  if  R
is  a  a  Huber  ring  and  don't  worry  if  you
don't  know  what  that  is  it's  just
certain  nice  kind  of  topological  ring
that  people  use  in  non  archimedian
analysis  ah  and  I  let  say  complete  h  no
just  hubber  I'm  just  doing  the  Rings  now
I'm  not  not  doing  the
pairs
um  and  if  uh  so  and  if  the  ideal
generated  by  F1  up  to  FN  uh  inside  R  is
open  which  is  the  condition  that
describes  rational  opens  in  the  space  of
continuous  valuations  as  opposed  to  the
space  of  AR  arbitrary  valuations  then  in
this  case  hubber
defines
uh  uh
the  Ring  of  functions
here  and  you  could  ask  what  the
relationship  is  with  this  thing  that
comes  from  the  solid  Theory  um  and  this
is  the  uh  quasi  separ
ification  uh  of  this  more  generally
defined  thing  that  we  have
here  so  the  solid  thing  or  maybe  for
emphasis  I  should  maybe  put  the  pi  Z  as
well  so  you  can  get  the  Huber  thing
functorially  from  this  more  uh  General
thing  in  particular  from  our  structure
Chief  but  they're  not  necessarily  equal
in  general  but  in  all  practical  cases
sort  of  they  are  equal  that's  the  the
general  outline  of  the
story  so
General  okay  is  not  and  so  so  in  general
if  you  always  use  the  structure  shft  do
always  satisfies  the  yes  so  it's  always
shy  yes  yeah  so  and  yeah  so  maybe  that's
an  important  point  to  mention  there  was
this  little  flying  the  ointment  in
Huber's  theory  that  for  in  the  general
setting  uh  he  had  he  defined  a  structure
sheath  except  it  wasn't  a  sheath  it  was
only  a  prief  and  in  all  all  practical
cases  it  was  a  sheath  but  still  it  was  a
little  bit  the  general  theory  was  kind
of  missing  something  something  for  that
reason  um  that's  fixed  by  this  so  if  you
don't  if  you  no  longer  care  about  things
being  quasy  separated  and  non-derived
then  you  get  a  a  good  a  good  plain  old
structure  sheath  and  you  get  even  more
you  get  derived  category  of  quasy  gent
she  which  localizes  and  also  you  you  get
the  possibility  of  um  uh  of  of  defining
all  of  these  things  even  without  this
condition  being
present
um  okay  I  think  probably  I'll  stop  there
thank  you  for  your
attention
yes  do  we  have  any  use  of  don't
continuous
varations  any  use  yes
uh  uh  I  know  Coles  likes  them  so  he  he
came  he  has  this  has  this  notion  of  AD  o
spaces  and  and  those  correspond  to
certain  yeah  but  you  don't  really  use
them  um  I  think  he  used  them  but  you  oh
me  well  look  I  I  just  build  the  theory  I
don't  he  calls  him  ad  hoc  spaces  I  don't
think  he  developed  the  theory  he  just
kind  of  did  it  in  an  example  um  so  he
wanted  to  he  wanted  to  exactly  include
things  like  this  open  unit
disc  kind  of  uh  so  this  yeah  things  like
the  zp  I  think  in  his  paper  where  he
does  this  GL2  mod  P
langland  uh  is  that  right  yeah  I  don't  I
don't  know  so
so  he  wants  to  this  this  thing  is  a
prototypical  example  of  a  topological
ring  that's  not  a  Huber  ring
so  um  so  in  the  usual  theories  of  viid
analytic  geometry  the  open  unit  disc  you
can't  think  of  it  as  an  apine  space
because  the  the  ring  that  it  corresponds
to  is  not  one  of  your  allowed  rings  so
they  always  so  you  always  have  to  think
of  the  open  unit  disc  as  a  union  of  the
closed  unit  discs  contained  in  it  it's
this  non-  quasy  compact  thing  to  fit  it
inside  Huber's  Theory  but  Coles
Advocates  that  sometimes  it's  better  to
to  just  work  with  this  thing  as  if  it
were  apine  and  that's  something  that  our
Theory  easily
accommodates  yeah  there  are  people  who
who  did  I  forgot  the  name  of  this
and  construct  some  notion  of
rigid  I  forgot  the  name  some  draw  some
paper  on  this  on
like  variant  of
three  I  about  the  name  of  the  there  was
some  paper  some  time  ago  but  not
sure
okay  any  other  questions  yes
yeses  every  Sol  z
albra  uh  yes  so  you're  yeah  so  this  this
notion  of  analytic  ring  we  haven't
defined  it  yet  it  kind  of  organizes  a
lot  of  discussion  but  indeed  yeah  if  you
have  a  if  you  have  a  solid  if  if  you
have  algebra  in  solid  Z  modules  you  get
what  we  call  an  induced  analytic  ring
structure  so  it's  just  all  you  take  all
I  mean  I  said  the  module  category  you
have  and  that  that  sits  inside  condensed
our  modules  and  that  is  an  analytic  ring
yeah  but  in  case  of  Z  we  had  something
more
natural  more  natural  it's  arguable  it's
more  complete  uh  there're  two  things
they  exist  they  do  I  mean  they  you  want
them  both  I
think
and  can  be  extended  can  be  extended  to
any  discreet  ring  right  yes  that's
right  not  in  general  not  not  General  Z
solid  Z  algebra  well  for  a  general  solid
Z  Algebra  I  don't  know  of  any
necessarily  know  of  any  like  completely
canonical  well  like  maybe  I  mean  one
thing  you  could  do  is  you  could  take  all
of  the  Power  bounded  elements  and  force
all  of  them  to  be  solid  that  would  be
kind  of  the  maximally  complete  thing
that  you  can  get  via  the  stuff  that
we've  developed  Ved  but  there  could  be
there  could  be  further  completions  of
that  as  far  as  I  know  I  mean  I
uh  you're  welcome  yes  with  the
definition  of  solid  anal  Rings  we  used
in  the  very  first  lecture  if  we  use  this
instead  of  a  regular  ring  here  do  we  get
something  similar  to  like  Huber's  theory
for  Huber  pairs  instead  of  just
regular  could  you  could  you  repeat  the
question  I'm  not  not  sure  I  understood
in  the  very  first  lecture  we  find  like
solid  analytic  ring  solid  analytic  Rings
yes  yes  yes  with  the  ring  and  the  direct
category  uh  if  we  try  applying  something
similar  here  to  analytic  Rings  what  do
you  mean  what  do  you
mean  ah  you  mean  like  this  discussion  of
this  this  thing  here  oh  yeah  yeah  yeah
you  can  do  that  yeah  but  do  we  recover
like  oh  yeah  you  you  recover  the  theory
of  with  with  the  r+  in  there  as  well
yeah  so  elements  R  plus  you  require
solid  solidity  exactly  exactly  not  for
all  of  the  guys  exactly  exactly  so  you
can  you  can  you  can  yeah  and  we  will
discuss  this  you  get  you  can  add  that
extra  flexibility  into  the  picture  yeah
and  the  fact  that  R  z0  is  this  is
automatic  that  those  are  the
to  they're  automatically  solid  exactly
yes  okay
yeah  so  it  actually  it  actually  fits
remarkably  well  with  Uber's  Theory
a  Prett  the  like  you  take  tens  product
of  analytic  R  you  have  to  first  like
solid  one  another  have  to  take  yeah  yeah
usually  yeah  so  here  it  still  the  same
here  you  don't  have
well  these  all  these  solidifications  all
all  commute  with  each  other  so  there  you
don't  need
to  yeah  they're  all  given  by  R  homing
out  of  out  of  something  and  any  two  R
homing  out  of  commute  with  each  each
other
yeah  other
questions  okay  well  see  you  on  Friday  or
next  Wednesday  or
whenever
\end{unfinished}
% !TeX root = ../AnalyticStacks.tex

\section{\ufs Huber pairs and analytic rings (Scholze)}

\url{https://www.youtube.com/watch?v=dIwBTJNN7a0&list=PLx5f8IelFRgGmu6gmL-Kf_Rl_6Mm7juZO}
\renewcommand{\yt}[2]{\href{https://www.youtube.com/watch?v=dIwBTJNN7a0&list=PLx5f8IelFRgGmu6gmL-Kf_Rl_6Mm7juZO&t=#1}{#2}}
\vspace{1em}

\begin{unfinished}{0:00}
  Right, so last time Dustin started talking about the relative solid theory and the relation to adic spaces. I want to kind of continue with that.

Okay, so I guess I want to talk about the relation between these basic objects that appear in Huber's work, that are called Huber pairs, and the kind of basic objects that appear in analytic geometry.

Motivation: we've seen several examples of a pair of a "present" notion of "complete" and the "same". The first one we discussed was the integers ($\Z$) and solid $\Z[1/p]$-modules inside of all condensed $\Z$-modules. But then last time, Dustin discussed the example where you take the polynomial algebra over $\Z$ in $T$, and then within all condensed $\Z$-modules, you can somehow just take the ones that are complete in the sense of this $T$. So you're taking modules over this algebra inside of solid abelian groups.

But then Dustin argued that it's actually quite natural to also isolate a stronger condition, giving the solid $\Z[T]$-modules. Geometrically speaking, this corresponds maybe to some kind of line, and this corresponds to the unit disk. In these cases, the condensed thing was actually just the classical thing.

But this notion of completeness is still interesting. You could also take, I don't know, $\F_p$ or $\Q_p$ or whatever, and then take solid $\F_p$-modules or solid $\Q_p$-modules. In those cases, these are actually just the ones where the underlying condensed group is solid. There's no meaningful way to strengthen this.

Okay, so the notion of an analytic ring captures this situation.

On the other hand, if you learn the adic stuff, then you run into this definition. I mean, the basic objects there are these Huber pairs. So let me recall what these are. These are Huber's definition, although of course Huber used different names.

A Huber ring is a topological ring $A$ with an open subring $A^+$ and an ideal $I \subset A^+$ such that there exists some finitely generated ideal $J \subset I$... Because now it's not clear what "ideal" means, because there are two rings in place, $J$ is an ideal of $A^+$, a subset $A_0 \subset A$ that has the same $I$-adic topology.

Let me give examples in just a second. Let me just finish the definitions.

And then Huber has this notion of a ring of integral elements. This is an open and integrally closed subring $A^+ \subset A$ containing $A^{\circ}$, the power-bounded elements. Second...

And third, a Huber pair is the pair $(A^+, A)$ satisfying these conditions.

Okay, so what are the examples to keep in mind? Let me first give some stupid examples:

Any discrete ring $A$ is Huber, so in this case you can take $A^+ = A$ and $I = 0$. Any ring that itself is an adic ring for a finitely generated ideal $I$ is Huber, so in this case you take $A^+ = A$ and $I = I$.

Maybe actually interesting is when you have something like a Banach algebra. So for example, $\Q_p$ is Huber, or any non-archimedean field is Huber. In this case, you take $A^+ = \Z_p$ and $I$ generated by $p$, which really is only an ideal in $\Z_p$ and not in $\Q_p$.

So basically, the idea is that Huber rings are basically certain kinds of localizations or something like this of such adic rings with some adic topology.

And so also whenever you have any

Kind of Banach algebra over $\Q_p$ or some other non-archimedean local field. There's also always Huber rank. As a zero, you can take the unit ball in your Banach algebra, and as the ideal, you can take the one that comes from a uniformizer.

Remark: The completion of any Huber ring is again a Huber ring. We will generally only consider complete examples now. In the classical sense, allow all convergent sequences mod $\mathfrak{m}^n$.

If you start with a so-called ring of definition, one which has the $I$-topology for some finitely generated ideal $I$, then it will be the case that the completion of $A^+$ will be an open subring of the completion of $A$. This is just the definition of a Huber pair.

At least when one uses Huber rings and Huber pairs to do adic spaces, then the adic spaces associated with a Huber ring and its completion are definitionally the same. In this sense, non-complete Huber rings have at most a technical role.

We will only consider complete ones. So from the following, whenever I say Huber ring, Huber pair, and so on, I always assume that the underlying Huber ring is complete. Okay?

And so, last time, Dustin discussed some notions of topologically nilpotent elements, power-bounded elements, and so on. This can also be defined here.

If you have a Huber ring, then you can define topologically nilpotent elements and power-bounded elements in $A$. This is the set of all $f \in A$ such that $f^n \to 0$ as $n \to \infty$, which once $A$ is assumed to be complete, is really just a condition.

And these are all the elements such that the set of its powers is bounded. It's actually equivalent to saying that it's contained in some ideal of definition, some ring of definition.

Actually, a different way to think about this set: First of all, you have these canonically defined objects, $A^+$ and $A^{\circ\circ}$. But on the other hand, we used these things in the definition that there is some ring of definition and some ideal of definition of the same. These will always consist of power-bounded elements, and the ideal consists of topologically nilpotent elements.

For example, in the case of $\Z_p$, $\Z_p$ is actually the $A^+$, and $(p)$ is actually the $A^{\circ\circ}$. In this case, these inclusions are equalities. Of course, this can't in general be true, because you could also take as ideal of definition the ideal generated by $p^2$, but then you don't have all topologically nilpotent elements.

But in some sense, the $A^+$ gets, in fact, the collection of all such $A^+$'s, and the collection of all such $I$'s. They form filtered collections, and $A^+$ is actually the colimit of all possible such rings of definition, and similarly, $A^{\circ\circ}$ is the union of all ideals of definition.

Alright, this was part one of the definition, together with a little bit of discussion. Part two was---oh no, sorry, there is no part two. I wanted to define, I wanted to say that if I also have a ring of power-bounded elements, I can define an $A^+$, but it's kind of weak because---sorry, okay.

So when you learn Huber's theory, at first, I think it's extremely hard to appreciate the significance of this ring of integral elements $A^+$. It is somewhat necessary to set up the theory, but it's kind of hard to feel why it's necessary.

But it actually turns out that the theory that we develop using the condensed mathematics gives you a very good understanding of what it actually does. Namely, precisely---here's an example.

I actually have several possible examples. For instance, you could take $A = \Q_p\<T\>$, and in this case, $A^+$ is just $\Z_p\langle T \rangle$. For example, you could take just 

So maybe I should give this definition of analytic rings. By the way, sorry, maybe I can make one notation remark. Huber uses the single letter $A$ to denote the whole pair consisting of some topological ring and a ring of integral elements. We will follow him.

Also, when I discuss analytic rings, I want to use single letters to denote my analytic rings. But then they will have an underlying condensed ring, that's $\underline{A}$, and we needed some symbol to denote that underlying condensed ring. We didn't come up with anything good, so we chose to just follow Huber's lead. Okay.

So let's say $A^\triangle$ now some lightning bolt, and then I want to say what is an analytic ring structure on this thing. So what is an allowed class $\mathcal{C}$ of complete $A$-modules? Okay, here's the definition. It's equivalent to the one that Dustin gave in the first lecture, but presented from a slightly more elementary perspective.

Okay, so it's a full subcategory $\mathcal{C}$ of $\mathcal{D}(A^\triangle)$, the category of condensed $A^\triangle$-modules, together with an $A$-module structure. That's the data I just said. Now I will make a lot of conditions on this, but those are conditions that we had already seen before, twice. We stated that solid $A^\triangle$-modules have a lot of nice properties, and it was a long, long list. Sometimes, because we don't want to state this list all the time, we make this definition.

So first of all, $\mathcal{C}$ should be stable under kernels and cokernels. But it's also stable, in fact, under all limits and colimits. All extensions, so if you have an extension of two things in $\mathcal{C}$, it should also be in $\mathcal{C}$.

Then there is a Tor-amplitude condition you want $\mathcal{C}$ to check. It's also stable under all $X \mapsto X^{\oplus I}$ for some set $I$. And $\mathcal{C}$ contains $A^\triangle$ itself.

[Dustin:] So can I ask a question? Does it imply, maybe one can prove from this in some way, for example, that $\mathcal{C}$ is a Grothendieck category? The condition that I allude to is the existence of a set of generators. Is it automatic under these conditions?

[Peter:] Yes, yes. Did you hear his answer?

[Dustin:] Yes, he said yes. And does it imply that the Ext groups in the subcategory are the same as the Ext groups in the full category?

[Peter:] No. I don't know what he said, but the answer is yes. I know the answer is yes, but you didn't hear... I mean, in this presentation, the derived category might not be in degree zero. So if you really phrase it at the derived level, you have to be slightly careful when you say that, right? Because it might not be the case that the thing I will define, the derived category of $\mathcal{C}$-modules, is the derived category of $\underline{A}$-modules. It's not. Dustin, do you hear what I say?

[Dustin:] Yes, I hear what you say. You said that in the category, the thing I will define as a certain triangulated category or stable $\infty$-category as a full subcategory of condensed $\underline{A}$-modules with some properties, which is the correct one. But in general it will not be the same thing as the derived category of $\underline{A}$-modules.

[Peter:] Yes, this is the heart of a t-structure. This is those whose homological groups are in this. So it doesn't imply that the Ext's are the same.

[Dustin:] It does not. And sometimes it does, sometimes it is true.

[Peter:] But like in most practical cases it will end up being true. In full generality, no. Okay. So I can proceed?

[Several people:] Proceed.

[Peter:] Now Dustin put me...made me confused. So I want to claim that there is automatically a left adjoint to the inclusion.
So the claim is that the left adjoint---which I will write as sending a module $M$ to its base change from $A_\infty$ to $A$---mod $A$ is just purely notational for now, but I will think of it as the modules over this analytic ring $A$.

This base change functor has kernel, the $\otimes$-ideal in $\mathrm{Mod}(A_\infty)$, and $\mathrm{Mod}(A)$ acquires a unique symmetric monoidal structure making the base change a symmetric monoidal functor.

Let's sketch the proof. We already discussed the existence of the left adjoint, which is formal nonsense. If it's not, just make it part of the definition. The question is about this kernel being a $\otimes$-ideal.

So what does it mean to be a $\otimes$-ideal? The left adjoint $F$ definitely preserves colimits. To show it's a $\otimes$-ideal, we have to show that if we have something $M$ in the kernel and $N$ is anything, then $M \otimes N$ is still in the kernel.

Assume a module $M \in \mathrm{Mod}(A_\infty)$ such that $F(M) = 0$, meaning it has no maps to any $A$-module. We want to show that for all $N \in \mathrm{Mod}(A_\infty)$, we have $F(M \otimes N) = 0$. 

This means showing that for all $L \in \mathrm{Mod}(A)$, $\mathrm{Hom}(F(M \otimes N), L) = 0$. By definition of $F$ being a left adjoint, this is equivalent to showing $\mathrm{Hom}(M \otimes N, L) = 0$.

Using the Hom-tensor adjunction, this is the same as $\mathrm{Hom}(M, [N,L])$, where $[N,L]$ is the internal Hom. But we assumed $\mathrm{Mod}(A)$ is stable under all limits, in particular internal Homs. So $[N,L] \in \mathrm{Mod}(A)$. Then again $\mathrm{Hom}(M, [N,L]) = 0$ because $F(M) = 0$ by assumption.

The symmetric monoidal structure on $\mathrm{Mod}(A)$ has to be given by taking the tensor product in $\mathrm{Mod}(A_\infty)$, seeing this as a colimit in modules, and then completing again. The question is whether this makes the base change functor symmetric monoidal.

To check this, for all $M,N \in \mathrm{Mod}(A_\infty)$, we can either first tensor $M$ and $N$ and then apply $F$, or we can first apply $F$ to both of them and then tensor in $\mathrm{Mod}(A)$. This has to give the same result.

I think a better statement is that if a map in $\mathrm{Mod}(A_\infty)$ becomes an isomorphism under localization, then tensoring it with anything else also makes it an isomorphism. This follows from the same type of argument, by mapping into the subcategory and using the internal Hom. I did this in the solid case, and it's the same argument here.

Then the point is that, for example, $f$ is from $M$ to its completion, which becomes an item of localization because an important operation. And so if I-- this within some-- the same-- stage it.

So this is some structure you automatically have on a triangulated category $C$ of modules, some kind of localization of condens-- the underlying ring. And requires in terms of product. And now we pass-- D-- let not just say that a structure on an underlying lightens string a triangle.

Then undine the of a modules, the full sub-subset for all $Z$ of all-- let me still just call them $N$, so compx of modules. So group-- let me think homologically, all the homo groups line. Define for Jesus.

And okay, so here-- here's already the warning: there is a natural comparison function from the of mod $A$, but it's not always. And in essentially all, I mean basically, yeah, all cases I'm aware, it will come out to be an equivalence, but it's just not a general effect.

But yeah, so the good thing we are to focus on is the thing that we simply call $D$. And so the previous proposition has an analog on dou.

So E of A triangle triang-- so I mean, probably in one or two lectures we will probably switch to the infinity categorical language, where we would say "stable infinity category" instead. For now, it's not really requir--, so let me just TR more classical terms.

Stable under all-- so again, in stable infinity case we could say "stable under all limits and colimits", but general limits and colimits are not well behaved work Str categories. But you can say something equol-- and stable under all s-- and for product, which are well behaved to what they're supposed to do.

I'm trying to say, right, the inclusion again has left on, that I will call the dve $S$, the product. And again, this has a property that if you have something, it becomes an isomorphism here. Then if you T-- it with anything else, it's still the same. And because this is now a triang c--, you can actually phrase this equ-- in terms of the co--. So if you have something in the kernel of this, then you tend up with something sa-- the.

And then again, this the T-- here. So if one wants to to do the previous type of argument for this fact, then one lends into the question of whether-- of course there is internal or in the full derived category by unbounded and so on. But is the question is whether if you have internal oh-- from anything to something in $D(A)$, then it lies in $D(A)$, right? And this is not-- because of unboundedness, I don't-- of course you have a bounded complex, you have a spectral sequence. I mean, you still have to to to work with that. Here you have unbounding in both directions, I can see in one direction you have IND light conding is still repete. So the derived category is left complete and so I think you can control control the question. Okay, let let me do this in a second when I come to the pro s-- the product maying face chang--.

All right. So see, you have-- I don't know, $M'$ to $M$, $M'$, $M'_1$, $M'$, and there a modules. And let's assume two of them, and by shifting it doesn't matter which to, are $D(A)$. Then we want to show that $M$ is also to show. But for this we just look at long sequence. So we have $H(M')$ have $H$, and these are $A$. So if I have some quent here and have some cur-- kernel right? And this is a kernel of the thing which a mod is a qu-- of a modules. So this of both a modules and then this one is an extension, right?

Here we use stability on the kernels and cions, and then we use stability on extends. No I think standing here actually realize okay. So and the directed to the $H$, and not standing realize that possibility the product, countable ones, they definitely reduce to.

And then the uncountable ones? That was okay when we were working in the full condensed setting and preparing the lecture. I overlooked that there might be an argument to do here.

Dustin, should I just assume that there is a claim on the level of the categories? I'm sorry, I was busy with the chat. What's going on? Why is it stable under all products?

Why is what stable under all products? The subcategory of complete ones? Oh, all products! Instead of just countable products, right? Oh, all products exist, but is it exact in your category? Yeah, this is a problem. $\infty$-limits and all products are not exact. Yeah, this is a problem.

Okay, we'll have to think about this. It's not an actual issue in some sense, but I screwed up the definition. So we should ask for the existence of left adjoints. I mean, Dustin did in the first lecture and I just threw it out when I prepared the lecture. For existence of left adjoints. The definition, I definitely want the admitted left adjoints.

Um, sorry. All right, so now I made this next thing actually part of the definition that exists.

So, let's say $M$ is complete. And then, is any condensed... No, sorry. What I want to show is $N$ is in $D^-(A)$. And then there anything... Then you have to show that for all $L$ in $D(A)$, $\underline{\mathrm{Hom}}(L,N)$ are complete.

And first of all, because $\infty$-sites are what's known as replete, this means it's closed under countable limits of surjections. This notion was introduced in my paper with SAG, but on the pro-side. And one thing we saw there is that this implies that any such... Sorry, for all $K$, I didn't use the letter $K$, right? So for any $K$ which is, for example, a condensed abelian group or module, $K$ is isomorphic to the derived limit of its truncations in degrees at most $n$. Some kind of limit, usually of abelian groups.

I mean, it's somewhat true, right? When you truncate up to some degree and then just take a limit of these things, you're somewhat stabilizing to the correct answer. In general, that's an issue because you're taking a countable limit here. In general, countable limits need not exist. But under this assumption, you can control them.

So this means that I can certainly assume that $L$ is bounded here, right? What I need... So first of all, and to show this, I can again use the adjunction. And I assume that $M$ has sheaf completion. So it suffices to show that the internal hom in condensed abelian groups from $\underline{L}$ to $\underline{M}$ is... Sorry, it's not complete, because then you can rewrite this as a hom from $L$ into this guy. But I assume that $M$ has trivial completion, so it doesn't have to do anything.

Okay, so this I want to reduce to the level where I kind of had the statement that if $L$ is in $D^-(A)$ and $M$ is in $D^+(A)$, then all the internal homs are in $D^+(A)$. The issue though, as already pointed out, is that here we need to ask this condition for all possibly unbounded complexes. That's why I mentioned this fact. So this at least allows us to assume that $L$ is in $D^-(A)$.

So I can assume $L$ is in $D^{\leq 0}(A)$. I usually put this going to the right. Because also all truncations, they are still in $D(A)$. But because the condition was just on the other hand, $N$ can be written as a colimit of the truncations to the left. I mean, this is always true, that there's a colimit of truncations $\tau_{\geq -n} N$. And $\tau_{\geq -n} N$ is in $D^{\geq -n}(A)$.

This is much easier because colimits forgetful to direct sums are always good. And similarly, you can pull the colimit into a limit. And because we know
Okay, I think that's fine. Once you have that, the existence of the tensor product is just the same formal diagram chase that I didn't execute previously, but did earlier.

Another thing I should have mentioned as part of the general theory, but didn't, is that $D^{\\leq 0}(A)$ has a natural t-structure, making it a stable $\\infty$-category. The left adjoint is not generally t-exact, as we've already seen that solidification could turn something unbounded on the left. Still, this left adjoint preserves connective objects.

A t-structure is where you have a notion of truncation of complexes, a notion of complexes which live in certain non-negative degrees or certain non-positive degrees, and they satisfy all the usual properties. We definitionally made this a triangulated subcategory which is stable under all the different functors.

So this inclusion is t-exact, and it's a completely general fact that if you have a left adjoint to a t-exact functor, at least it preserves the connective part. Let me check whether this maps to anything which is concentrated on the right, but this is a left adjoint, so you can compute the $\\mathrm{Mor}$ in the larger category. But then this is still in this category.

In particular, you can talk about the heart. The heart is also definitionally just $A$. If you take this and pass to the heart, this is $A$. If you take the tensor product and pass to the heart, in this sense the derived and abelian level are compatible.

Then there's the other question: if you start here and just animate all these constructions to $\\infty$-categories, do you recover those constructions? This is just not true in general. In general, you don't even recover $D(A)$. Even if you do, there are separate questions about whether you recover the correct functors, and again, not in general. I think if you do recover the correct categories, you also recover the correct functors by functoriality. But the tensor product is a bit subtle. Again, in practice it is true that $C(A)$ is just the animation, and all these functors are correct.

With that out of the way, I'm almost done with my lecture, unfortunately.

Okay, back to the comparison. When we had Huber rings, we had these topologically nilpotent elements and so on. Dustin already gave a variant of this. First of all, Huber rings themselves are Huber pairs $(A, A^+)$. These condense to rings of course. All this is actually fully faithful.

Actually, I should denote these as $A^\solid$ and $A^{\solid+}$ where the $\solid$ means solid. Last time, Dustin already gave a definition that for a solid ring $A^\solid$, we can define subsets $A^\blacksquare \circ$ and $A^{\blacksquare \circ\circ}$ of the underlying condensed ring.

Let me recall, $A^\blacksquare \circ$ was the set of elements in the underlying ring such that the corresponding map $\Z[T] \to A^\blacksquare$ factors through $\Z\langle T \rangle$. There was some discussion about how much structure you need to check here. The condition was that it factors as condensed rings. It's actually enough to check that it factors as condensed modules over $\Z$.

Then $A^{\solid\circ\circ}$ was defined as the set of elements $f$ such that there is a sequence $(f_n)_{n \geq 0}$ with $f_0 = f$, $f_{n+1}^p \in f_n A^\blacksquare \circ$ for all $n$, and $f_n \to 0$. This comes together with a smallness condition that $f$ times the shifted sequence makes it $I$-adically Cauchy.

If you apply this to the case where this solid ring $A^\solid$ arose from a Huber ring, then this is precisely the set of topologically nilpotent elements, and those define the $\circ\\circ$-elements. When you regard 

Is precisely the same thing as $H \subseteq A$ being power-bounded. Dustin showed last time that this is always an integral statement.

This here is always a what's the definition? Yeah, sorry. Given $f$, I can again---let me write again why this means I can "$a$" and then the condition is, I'm already speaking of modules over $C$, modules over the closed unit disc, as we motivated last time. This means that $\lvert f(a) \rvert$ should be at most $1$. So it should be also power-bounded. Okay.

But now, I can also make the point: Assume $A$ is an analytic ring structure on a solid $R$-algebra. Then I can also define an $A^+$. I realized I didn't define this, so let me do this in just a second. Such that the map from $D(T) \to A$ is the same as always.

This map induces a map of analytic rings, from $\mathbf{V}$ (the corresponding solid module). Yeah, that's precise.

So something that I should have said previously but forgot. Let $\phi\colon R \to S$ be a map of condensed rings, $M$ an $S$-module, and $N$ an $R$-module. Then a map of modules $f\colon M \to \phi_* N$ (where $\phi_*$ denotes restriction of scalars) is equivalent to a map of condensed $S$-modules $\phi^* M \to N$ (where $\phi^*$ denotes base change). In this case, you can pass to the left adjoints on the level of derived categories, because you can check it on the level of modules.

Once you pass left adjoints, the left adjoint to restriction of scalars (i.e., what I term the base change functor) becomes the left adjoint here. So you get it also left adjoint here, which is base change. If you want, you can compute it by first base changing as condensed modules and then completing it. You also get a derived functor.

Okay, so the claim is that first of all, once you have such an analytic ring, you can get data as in Huber. I will immediately check that this actually automatically satisfies his list of conditions that he puts on his ring of integral elements.

Conversely (I'm not sure if I have time, I hope I can say it), whenever you have a ring of integral elements in Huber's theory, you can actually produce an analytic ring which is in some sense the initial one.

Okay, so right. First of all, I can also rewrite this power-boundedness condition. It's all those $s$ such that $1 - sT^*$ (which is an endomorphism) maps $D(T)$ into $A^+$.

Changing notation: $P$ was this "$P$" called---it's always the spectrum based on the normed $R$-algebra. And we characterized being solid over $\Z[[T]]$ by, well, being solid (but this we already asked), and that $1-sT^*$ is a morphism on this projective generator.

So what's actually to ask is that if I look at this thing here (an object in $\mathcal{D}(A)$), then this is a "why". If I admit such a map, then this already happens here---like, already here, $1-sT^*$ becomes an automorphism of this object (definitionally). But on the other hand, because this precisely characterizes the solid modules here, you can also show the converse.

Like, if I want to show that I have such $A^+$, I need to show that all complete $A^+$-modules here restrict to complete $A$-modules here. But being complete precisely meant that if I tensor $1-sT^*$ into there, it becomes an isomorphism. And so just translates.

So basically, whenever you have any element $s$ of your underlying ring, you can ask this condition that $1-sT^*$ becomes a morphism of $P$. And this will define for you an analytic ring structure by general theory. Yeah, basically whenever you have an endomorphism of your complete projective object $P$, declaring that this should be an "as morphism" for all $s$ in the ring structure---so you can take
Modules... But then probably it's equivalent to what you have written, if you think about it, but in the derived way. I just want to confirm that the two versions are equivalent. Yes, right? Because you can actually detect analytic $R$-structures on the level of modules. It's enough to check it on the underlying level, the ind-level. That is also true, yeah. Okay, so it's equivalent.

I'm... Uh, yeah, so with derived $I$ would be more confident, but I think the argument just sketched goes through. So here... Right, so the point is that this subset $A_L$ automatically satisfies the conditions. It always contains the topologically nilpotent elements, which are always an open subset. $\{u\in R\mid \sum_{i\geq 0} a_i u^i$ converges$\}$ in particular, this is open. When $R$ comes from $\Q_p\langle T \rangle$, it's always containing the power-bounded elements, and it's always integrally closed.

So why is that? Well, if $f\in R\langle T \rangle$ is an element, then we actually get a power series... That's almost... Yeah, if I have a module that's actually $\Z_p[[T]]$-module in $\Solid_A$, that's automatically $p$-torsion-free, actually. I mean, this proof is exactly the same as last time, just because all I used was arguments about modules being solid over one ring implying they're solid over another, and so on. Yes, let me just state it again. Okay.

So these are actually all there. But this means that whenever I have a module here over $A$, it in particular becomes a module here, which is solid. Everything is solid. So it must be an underlying solid ring.

If $A^+$ is integrally closed, then I get a map from here, $\{a\in A\mid \forall M\in A\text{-}\mathrm{Mod}, M \text{ solid} \implies aM \subseteq M\}$, to $A$. But in particular this means that $A^+$, the underlying condensed ring which we always assume is a complete $\Z_p$-module, by restriction... So if $f$ is an element in $A$, right, so this group scheme inclusions $A^+\to A$... And therefore, something that's the same argument. So in fact, yeah, the argument that Dustin gave there was already talking not just about a triangle, but about any module. And so if you just run his argument, to see this is what he actually...

Okay, so... Right. Therefore, if $A$ is any triangle, I have the solid analytic $R$-structure $\mathcal{A}_1$ which lives over $\Z_p\langle T \rangle$ and consider this morphism of rings of integral elements. I just gave you a recipe here that was taking some such analytic ring structure here and produced a ring of integral elements $A^+$ in here.

And it's actually functorial. So you can actually show that $f:A\to B$, one arrow is contained in the other. Yeah, I mean, if you have... And you get an inclusion of the analytic rings, and this actually has a left adjoint, I assume.

So whenever I have a ring $A$, so I can produce an $\mathcal{A}_1$-ring structure on the solid $A$-modules. Yes, so $\mathrm{Solid}_A$ admits an $\mathcal{A}_1$-ring structure. It's unique because the ring of endomorphisms is unique, and then it's just a condition. It's just a condition that all the $A$-modules are actually solid.

A different way to phrase this is to ask that $1-\varphi$ is an automorphism of $A^+\otimes\Q_p$. It doesn't matter.

All right, so I wanted to say that it has a left adjoint. So if I have a morphism of pairs $(A,A^+)\to (B,B^+)$, then I can send this to the ring $B$ associated where mod $B^+$ by definition, all those

Usually there's just one or two elements, something where you really have to check. All the same thing as those modules, and only for those two elements.

So yeah, to connect to the beginning, for example, like $Z[Z]$-pairs, these are just $Z[Z]$-modules. If I take $Z$, that's the pair. So if I only put $Z$ here, then I'm only asking that it's $Z$-solid over $Z$. So I take all $Z[T]$-modules and $Z$-solid $Z$-modules. But then when I take this, it becomes $R_0$ for $A^+$. So let me just state one last proposition.

So when you start with $H$, and then go to $A^+$, then if I go back, this actually matches back to $A^+$. So if I write, I have an analytic ring and then I can take its plus ring. So this is actually a plus equivalence, the left adjoint functor, from $R_0$ to $A^+$.

So all brings, but they're all---and yeah, I mean, I'm really still quite struck by how closely the theory of solid analytic rings really matches Huber's classical theories. If you restrict to analytic rings where you only allow yourself to put conditions that one minus some element times the shift operator on $T$ becomes an isomorphism, then you're precisely getting those analytic ring structures that are induced by rings of integral elements in Huber's sense. So which is kind of very strong aposteriori motivation for this definition.

All right, I should stop.

\end{unfinished}
% !TeX root = ../AnalyticStacks.tex

\section{\ufs Localization of solid analytic rings (Clausen)}

\url{https://www.youtube.com/watch?v=lJTLj8gYAtg&list=PLx5f8IelFRgGmu6gmL-Kf_Rl_6Mm7juZO}
\renewcommand{\yt}[2]{\href{https://www.youtube.com/watch?v=lJTLj8gYAtg&list=PLx5f8IelFRgGmu6gmL-Kf_Rl_6Mm7juZO&t=#1}{#2}}
\vspace{1em}

\begin{unfinished}{0:00}

Welcome back, everyone. Today, we're going to be doing a little bit of localization in the setting of solid analytic rings. But first, I want to start with a recap of last time.

Last time, we defined the notion of an analytic ring, which is a pair $(R, \triangle R)$ where $R$ is the whole object and $\triangle R$ is the category of $R$-modules. This $\triangle R$ is a full subcategory of the category of $R$-modules, satisfying some closure axioms. Specifically, $\triangle R$ is closed under all limits, colimits, and extensions. Additionally, for any $M$ in $\triangle R$ and $N$ in $\operatorname{Mod} R$, the internal $\operatorname{Hom}$ from $M$ to $N$ also lies in $\triangle R$. Finally, the unit $R$ itself lies in $\triangle R$.

There were a couple of technical points that came up last time that I'll take the time to address now. One was regarding the existence of left adjoints. There was a claim that this was automatic, justified by a theorem of Adamek and Rosický called the "reflection principle." This states that if $C$ is a presentable category and $D$ is a full subcategory of $C$ closed under all limits, and if there exists a regular cardinal $\kappa$ such that $D$ is closed under all $\kappa$-filtered colimits, then $D$ is presentable and the inclusion $D \to C$ has a left adjoint.

I also want to make another remark concerning a technical point that came up in the last talk. Recall that we were discussing the derived analog of this definition, where we defined the derived category of $\triangle R$ as the full subcategory of the derived category of $\operatorname{Mod} R$ consisting of those objects whose homology groups all lie in $\triangle R$. We wanted to show that this derived category satisfies analogous closure properties under limits and colimits. The key point for proving closure under limits was to show that the product of a family of objects in $\triangle R$, viewed as objects in the derived category concentrated in degree 0, still satisfies the condition of lying in $\triangle R$. The subtlety here is that while countable infinite products are exact in the setting of condensed abelian groups, arbitrary infinite products are not. So the product functor has right derived functors, and we need to ensure that the product lies in $\triangle R$ for all degrees. This can be proved by considering the direct sum instead of the product.

Alpha, then this guy is a retract of $\R^I$ product $\alpha$ in $M$, because termwise it's obviously this system is a retract of this system here. But this is just the same thing as $X \in \text{Hom}_M(\oplus_I \R, M)$. Okay, and what about product of infinite complexes, unbounded below complexes? Yeah, the argument given in the last lecture, I mean this was treated in the last lecture, ah, because there are kind of it's possible, so that you can reduce to countable. Exactly, okay. 

So that's that, and I think also that I don't know, maybe you know the reference. So if you have in general a category $\mathcal{C}$ and subcategories $\mathcal{C}_i$ that are closed under coproducts and filtered direct limits, then in the derived category when you consider objects whose cohomology lies in the subcategories, this is also present. This, I think, can be shown. I don't know the reference for this, I don't know. Okay, but this is just an elementary thing and you can do it for complexes, and I think this can be used to give a less...I mean, once you know this, which actually one can formulate in terms of every complex being the limit of smaller ones with some bound, then this can slide well.

Yeah, I don't think, yeah, to this without using the, I don't think it's necessary to use this. I mean, actually you can kind of explicitly construct the left adjoint on the derived level just by taking left derived functors of the left adjoint you have on the abelian level. Well, maybe it doesn't quite work like that, but you can. Well, yeah, I'm willing to believe that the $\infty$-category version can be avoided anyway. Let's... you can first... the actually presentable, because you're just... that all the... right, right, right, right, yeah, category. Yeah, that's a good argument, yeah, because this is not necessarily the derived category of $\text{Mod}_R$, but what Peter said was that there's a general principle about presentable categories being closed under limits and the category of categories as long as you have functors that commute with colimits, and the homology functors commute with colimits, so then you can see that this thing has to be presentable, and then the $\infty$-categorical adjoint function theorem, the more naive version proved by LRI, would give you the left adjoint. Okay, anyway, enough of that. 

Right, uh, let's move on to some real math. Okay, so and when you did before without light, you did everything, so is there a close relation between the notions in the light and the general set, that is, so a condensed ring in the light gives one in the general, and subcategory, subcategory, and so on, maybe, or is it more twisted? Well, the first statement is completely accurate: light condensed rings embed fully faithfully into all condensed rings. But when it comes to the analytic ring structure, it's a little bit more subtle.

In general, $R^+$ is an integrally closed subring containing all the topologically nilpotent elements. We saw that if you have a solid analytic ring structure on a solid ring, such that every module over it is solid as an underlying Abelian group, then the collection of $f$ for which this is an isomorphism for all $M$ is actually an integrally closed subring containing all the topologically nilpotent elements. So you could always just throw in these guys, take the subring generated by them, and take the integral closure, and that will not change the theory.

Okay, and then there was an example. If this is a Huber pair, meaning this is a Huber ring, then the integrally closed subrings $R^+$ like this are the same thing as the open integrally closed subrings, and these are exactly the $R^+$'s in Huber's theory. So the general setup we have for an arbitrary solid ring of the possible choices of $R^+$ when you specialize to the case of a Huber ring, it recovers exactly the choices of $R^+$ you have in Huber's theory.

From now on, whenever I say Huber ring, I'll mean complete Huber ring unless otherwise specified. There is also sometimes people use derived complete things, and if you have a derived complete thing, it also seems to give a condensed thing, and then it's also possible to say that this is also solid. Yes, actually I might talk about that fun story later in the lecture.

Moreover, in the Huber case, the $R^+$ is actually recovered from the analytic ring structure---it's basically equivalent to the $R^\triangle$ in Huber's theory. In the general case, I didn't quite claim that if you start with an integrally closed subring satisfying these conditions and you form that theory, there might for some other reasons also be other things that $f$ that satisfy this property for all your $M$, but in the Huber case, you can show that there aren't---we didn't do it last lecture, but it was done in a previous lecture.

Okay, questions. Now we're going to discuss localization. Let me make a remark which might be a bit shocking at first glance, but it's actually trivial. If $R$ is a solid ring, and we have this $R^+$ satisfying these conditions, and again you can feel free to assume it's an integrally closed subring containing the topologically nilpotent elements, note that this condition defining the analytic ring structure is just a condition that you're imposing for all $f$ in $R^+$, and $R^+$ was by definition a subset of the underlying discrete ring $R$. So all the data that you're using to define the analytic ring structure actually already appears at the discrete level. 

Then you get another pair, just with the same $R^+$ and the power-bounded elements in the discrete case are just all the elements, so certainly it's still going to be power-bounded inside there. And this observation shows that if you're interested in solid modules over your original $R$, $R^+$, you can take the ones over where you have a discrete ring, and then that already has all of the information about the analytic ring structure. And all that remains is to observe that $R$ will be a commutative algebra object in here, and you just kind of abstractly take $R$-modules in this Abelian category. It's important to note that since we're doing condensed modules, even when you have a discrete ring, you have a huge amount of

Is actually base changed from the discrete case in this completely naive way. Okay, so we're going to discuss localization. How these categories glue, but it's actually going to be sufficient to treat the discrete case, because if you understand how this category glues, then you could just put the $R$-module structure on top of that and you'll understand how this category glues. And I want to stress from the beginning that I'm talking just about one kind of example of gluing---I'm not claiming this is the most general, but it is nonetheless quite general. It's just a certain framework for gluing, you can call it.

So now, let me make an analogy. So we're going to be in the world of discrete rings now. If $R$ is a commutative ring, then we have its usual derived category of $R$-modules. This localizes over the Zariski spectrum of $R$, and I'll say more precisely what that means in a second.

What is this $\text{Spec}\,R$? It's the set of prime ideals, and there's a basis of quasi-compact opens closed under finite intersection---these are the so-called distinguished opens. Let's say $U_f$ are the set of prime ideals $P$ such that $f$ is not in $P$, so $f$ maps to something nonzero in the residue field. There's a structure sheaf, and its value on this distinguished open $U_f$ is $R_1/f$, the localization of $R$ at $f$. It's important to note that neither $U_f$ nor $R_1/f$ determine $f$, but they do determine each other. In fact, $U_f$ gets identified with $\text{Spec}\,(R_1/f)$, and this matches up the distinguished opens.

If $V$ is contained in $U$, then you get a base change functor. The theorem, which is completely classical, is that this pre-sheaf is actually a sheaf of $\infty$-categories. In this setting, you also have a sheaf of abelian categories, and even a hypersheaf, but I'll just focus on the derived categories here. The key difference is that in the setting I'm about to discuss, the localization maps won't be flat in general, unlike in the discrete case.

So, this is in $L$ also. First of all, what I said should be correct, I think. I'm sorry, my brain is a little not working very well right now. I actually zoned out while you were talking. My apologies. Do you have any object, and you have any object, in the right category of the topos restricted to you? Val, in the sense, yes, then this is a hypersheaf, no, chief, yeah, hypersheaf, well, why would it automatically be a hypersheaf?

No, I think I was able to do it in some classical, more classical formulation, but it's a... Is it? So, what do you know about this statement? Oh, yeah, the usual derived category. I maybe it is, maybe it is a hypersheaf, yeah. I don't know, I don't know. I mean, I don't know the statement. It's, I guess, now that I think about it, it sounds plausible, but I mean the hypersheaf, but there's certainly Lurie proves it's a sheaf, yeah, and hypersheaf, maybe, I don't think proves it in one of his books, yeah, I'm sure.

Yeah, there is also something that I saw that you mentioned, I mean, in some text that they found on the internet, instead of looking at the derived category, you can look at Fun$_\infty$ or Shv and this is the same as the derived category of the hypersheaves, if you do hypersheaves with values in the derived category of abelian groups, that's the same thing as the derived category of the category of sheaves of abelian groups. Yeah, and this is also proved, I assume, somewhere in Lurie.

Now, I'd like to move on. So, maybe we have this discussion at another time. Okay, so that's one part of the analogy. And the second part is, so now we have this $\R$, $\R^+$, a discrete Huber pair. So, that just means $\R$ is an ordinary commutative ring and $\R^+$ is an integrally closed subring.

Well, then we've assigned to this this $\mathcal{D}_{\R,\R^+}$ solid, and the claim is that this localizes on something else, on the valuative spectrum of this pair $\R, \R^+$. Okay, so what is this? 

So, that was the set of prime ideals, and the kind of purpose of a prime ideal in this setting is to let you know where functions vanish or don't vanish, so kind of you could think of it that way, so kind of a binary condition of whether you're zero or nonzero. And in the valuative spectrum, you are allowed some more refined information, not just information about whether a given function vanishes or doesn't, but given two functions, you can ask whether one is bigger than the other. And the way you can measure that is by means of evaluation, so this is a function from $\R$ to $\Gamma \cup \{0\}$, where $\Gamma$ is an abelian group written multiplicatively. And then there are axioms, like multiplicativity, $v(fg) = v(f)v(g)$, and the non-archimedean condition, $v(f+g) \le \max\{v(f), v(g)\}$. And we also involve the sub

Okay, so the point here is that we now have a much bigger category, and there's more flexibility for how to localize. It connects with this classical discussion of valuations. If you've never seen this before, then you can look, for example, at the rational numbers. Maybe you know the classification of valuations there. There's the trivial valuation, which I guess corresponds to equality, where for every prime ideal, you have the trivial valuation, where it's zero if your element is zero and one otherwise. But then also for every prime $p$, you have a $p$-adic valuation.

So you have the generic point of $\Spec \Z$, you have the special points of $\Spec \Z$, but then you have these things in between, which are nearby $p$ but not equal to $p$, these $p$-adic valuations. Oh, sorry, I was talking about $\Z$ not $\mathbb{Q}$. But then the fact that you can classify those is kind of misleading, because once you add an extra variable, then all of a sudden things explode, and there's many different kinds of valuations, basically because in a surface, you can have lots of different kinds of curves passing through a given point, and you have valuations of so-called higher rank, which introduce additional complications into the theory. 

I'm not going to go too much into this, but yeah, so I'll stick to mostly formal aspects for now. Okay, so let's continue the table of analogies.

So we had $\Spec R$, and we had this particularly nice basis for the topology, quasi-compact, closed under finite intersections, and each of them was also of the same form as the global guy, just for a different input datum, $R_1$ over $F$. And we have the same thing here. We have a basis of quasi-compact opens, closed under finite intersection, and these are called the rational opens in this case. They depend on the choice of some elements in your ring. You choose $F_1, \dots, F_n$ and $G$ inside your ring, then you can form this thing, and what is it? It's the set of those valuations $V$ satisfying all of these conditions, such that moreover $V(G)$ is non-zero, and $V(f_i) \leq V(G)$ for all $i$.

So in some sense, it lives inside the distinguished open, the Zariski open, given by just deciding $G$ should be non-zero, and then we use this extra flexibility of we can also impose inequalities, so we're shrinking this Zariski open a little bit using some inequalities, and we still get an open subset. Okay, continuing.

So there's a structure sheaf, but actually, there are structure sheaves. On this $F_1, \dots, F_n$ over $G$, you have one thing which just takes the Zariski localization, but then you also get a choice of integral elements, and that you get by it's going to be, it has its going to be a subring of here, and you get it by taking the integral elements you had before, or rather their image in there, and then adjoint, and then looking also at these elements $F_1/G, \dots, F_n/G$, and then that might not be integrally closed, so you take the integral closure. Basically, you just look at all of the elements which the valuations in your open subset think should be less than or equal to one, so you've kind of already have it for this by fiat, and you forced it for these, and then the collection of those things is an integrally closed subring. 

And then again, you have this nice recursive property that $U(F_1, \dots, F_n/G)$ is just the same thing as the valuative spectrum of $\mathcal{O}_U^+$, and this matches up rational opens. And here is another place you can see the kind of necessity of including the data of this $\mathcal{R}^+$ in the general theory, because if you could have said, "Okay, well, I want a bigger

Let me put this: If $U$ is contained in $V$, then we get the pullback map. In fact, there's a map of analytic rings from $\mathcal{O}_U$ to $\mathcal{O}_V$ in the sense of the previous lecture. So we have a map of condensed rings, which is just in this case a map of discrete rings, such that if you have a complete module here, then when you restrict scalars, it's also complete here. That's the kind of forgetful functor, and then that always has a left adjoint, which is this base change functor. Explicitly, you get it by taking your module here, abstractly tensoring up from this ring to this ring, and then re-completing in this theory here.

The theorem---I've kind of run out of space, but maybe I'll put it. This precedes that one over there is a sheaf of $\infty$-categories, and I'll put the warning that this is not true on the level of abelian categories. In contrast to the classical case, these pullback functors are not t-exact in general, because the pullback involves a solidification, which, as I said, is not a flat operation and does bound topological dimension. Yes, it does, it'll be bounded by $n$, the solidification is bounded by---I mean, the homology is zero up to $n$, yeah.

Okay, so I think we'll take a 5-minute break before I get to the proofs. The proof---is it complete or not? Probably not in general, but there's an abstract result that if the space has finite cohomological dimension, then hypercompleteness is automatic. This is sometimes useful. If I start with a solid ring, I can take its underlying set, which is discrete, and we mentioned it's the same as the module versus the condensed thing. Is there a case where this is actually a point of this algebra? Hmm, I don't think so. So these tend---for example, like, I don't know, $\Z$. We showed that the $\Z$ power series $T$ is idempotent over the $\Z$ polynomial $T$, but this discrete ring is going to be way too big. I think this is not going to be idempotent; there's no extra reason why this should be, I mean, I didn't think about it carefully, but I would assume the answer is no.

Okay, so I've stated the theorem, and now I want to explain the proof. But to motivate it, I'll give a certain proof of this classical theorem here. There are many different possible arguments in the classical case, especially because these localizations are flat, there's lots of flexibility in how you set things up. But I want to describe a particular argument for this claim here, which will kind of translate over without too much difficulty to this case here. So let me erase some boards. There was maybe one remark that one can make in both settings that I forgot to make. I said I defined a sheaf of $\infty$-categories on these rational opens. Okay, not every open subset is rational; they're just a basis for the topology. But there's this general result that when you have a basis for the topology closed under finite intersections---that condition being actually necessary in the $\infty$-context---then a sheaf on that basis uniquely extends to a sheaf on the whole space in the naive manner of taking limits of an arbitrary open. Yeah, so we're only describing this sheaf of categories on the rational opens, but after the fact, you get also a category attached to an arbitrary open, whether or

Okay, this is a very simple example of two elements which generate the unit ideal inside this ring. The claim is that if you want to check something as a sheaf, you only need to check the sheaf condition in this one specific situation. This was originally in Quince's proof of the---anyway, it's easy, but there was something of Qu and he proved the cell conjecture.

Okay, it reduces to the fact that if you have some vector bundle on a fine space over a ring which is derived over a local ring, then it is extended from the ring, and he did it by reducing to this, and it was a bit trickier. Quince is a clever guy, so let's give the proof.

Well, I said you know we can describe algebraically the covers. If you, in general, the covers would be described like this: you take $F_1$ up to $F_N$ in $O$ generating the unit ideal, such that there exist $X_1$ to $X_N$ in $O$ with $X_1 F_1 + \dots + X_N F_N = 1$. And the general cover is the $U$ of $F_i$---oh, darn it, I can't believe I didn't think of that, will not be okay, because you cannot generate the empty set from non-empty things. Damn it, I should know better by now.

But, um, plus empty cover of empty set. Okay, checking the sheaf condition there just means you check that the value of your sheaf on the empty set is the terminal object in the category that is the target of your sheaf. Okay, so that usually can be done without much difficulty.

Okay, any question or comment from Bon? No? Okay, but note that this cover here is refined by another cover where you take $F_i * X_i$. This is a smaller distinguished open, and those still generate the unit ideal because of the same expression. So then we can assume just that $F_1 + \dots + F_N = 1$, and then you can do an induction on $N$.

Here, we have an object in the derived category and an object in the DED category, and then you give yourself extra data of an isomorphism between them. But it's not an isomorphism in the usual derived category, it's an isomorphism in some infinity version. So you can imagine, for example, if this is represented by a complex of projective objects, then you'd actually want to give a chain homotopy equivalence between their images.

Let's say they're bounded above just for simplicity, and then you make an infinity category out of that. So you define some notion of chain homotopy there, and so on.

Right, so then what does essential surjectivity mean? It means you can glue in the derived category. If you have a chain complex here, a chain complex here, and an explicit identification between them, maybe you choose some quasi-isomorphic models and make a chain homotopy equivalence between them. Then that collection of data uniquely comes from an element here, up to quasi-isomorphism. So the point being that you actually have to specify the data of the chain homotopy equivalence here in order to get the well-defined object there. That's the essential surjectivity.

The fully faithfulness says something else. It says that if you have two objects here, and you want to know the homs between them, so you can think of calculating Ext groups, for example, the Rhoms between two objects here, you can get it by base changing here and taking Rhoms, base changing here and taking Rhoms, and then doing a homotopy pullback of those complexes for Rhoms.

Okay, so that's kind of how to think about this result. It lets you glue objects that are defined locally in a derived sense, but it also lets you do global Ext calculations by localizing.

Okay, but how do you formally prove such a statement? Note that each base change functor has a right adjoint, which is just the forgetful functor, from the derived category of $R_1$ over $F$ to the derived category of $R$. And then it actually follows formally that this functor also has a right adjoint.

You can explicitly describe what this right adjoint is. If you have a pair $M, N, \alpha$ where $\alpha$ is an isomorphism, you just apply the right adjoints to each of these objects and then take a limit. So you take $M$ crossed over $N$ with $M_1$ over $F$, which is the same thing as $N_1$ over $F$.

Okay, so the trick to get used to conversing by two op is just to have an easy diagram, yes, in principle you could also do this argument without doing the reduction, but it's certainly easier to talk about it this way, because it's finite many intersections.

I think in the end, once you get to the statement we're trying to prove with the valuative spectrum, then you really probably don't want to - well, I don't know, maybe you could organize it cleverly, but I think doing the reductions makes it much easier.

Okay, right. This is good news, I mean, this is the great thing about proving something as a sheaf of categories---you have an automatic candidate for the inverse, it's some right adjoint. So you have a functor you want to prove is an equivalence, you have a right adjoint, that means you have a unit and a counit you need to check are isomorphisms. So then you need to check, one of them will be a map in this category, and one of them will be a map in this category.

For example, for the unit, you need that if $M$

Okay, so then the proof of the solid analog. You use the fact that when you invert $1 - f$, this is the intersection of the two, that is, those which are---yes, yes, yes, yeah, you want to know that when you pass to right adjoints, this thing is just the intersection of those two things as well. Yeah, maybe I should have added that to the list. Um, yeah, thanks.

Well, we have to understand it in a sensible way, I guess you're right. And of course, we have to know the language of $\mathcal{L}$ to make it precise. Yes, so it's not---maybe to well, I mean, it's the right adjoints are fully faithful, so it really is just kind of an object-wise condition you could say, just---yeah.

Okay, so what's the analog? So again, we have the site of rational opens $U$ in this Valuation spectrum, and the Grothendieck topology, open covers. And then we have a lemma that this topology is generated by---the empty set cover, and for all rational opens $U$ and all $f$ in $\mathcal{O}_R$, we have to take care of two different kinds of covers. So, we have $U$ covered by $1/f$ and $U \setminus V(f)$. This is a refinement of the Zariski cover we had previously, where you just inverted $f$ and $1/f$, and that was a cover. This is a smaller, a refinement of that, which still covers.

The last thing I think---one, it's enough to do it when $f$ is in $\R_+$. Okay, I don't think that will be helpful, but uh, it's---that's nice to know. Yeah, it's like in rigid geometry, in Tate's original work, where he proved a simplicity theorem. He reduces to---he didn't have rational domains, but he has this two types of covers, for which you can prove acyclicity, and it turns out that it generalizes to the analytic case.

I will not give the argument for this, it's---it's just it's a bit more complicated, because well, the Valuation spectrum is more complicated than Spec, but the idea is basically---well, the idea is somewhat similar, you could say, but it's actually a somewhat complicated argument. So, but it's---Uber, there is maybe a statement.

That it's enough to have a rational $f_1, \dots, f_n$ generating the unit ideal, and then you can check just for those. Yes, yes, yes, you can do some little bit of work to reduce to exactly, exactly, exactly, yes.

So, Huber shows that every $G$-cover is refined by one of the following form: take $f_1, \dots, f_n$ generating the unit ideal, and then look at $U(f_1, \dots, f_i^\wedge, \dots, f_n)$. The collection of these $U$'s is a refinement of the usual Zariski cover you get when $f_1, \dots, f_n$ generate the unit ideal, but you can check just on valuations that it still covers the valuative spectrum. Then you do something similar to what we did previously---you can assume one of the elements is equal to 1. You keep playing and playing and eventually you get the desired thing.

Okay, so now what are we reduced to, analogous to there? If $(R, R^+)$ is a discrete Huber pair and we take $f \in R$, then we need...

There were two different kinds of covers: $U(f)$ and $U(1/f)$. So I'll do the first one. $D(R, R^+) \to D(R[1/f], R^+ + fR^+)$ is just the solidification, $T$-solidification, from $\Z[T] \to R$ with $T \mapsto f$. By definition, these are both analytic ring structures on the same ring, and the only difference is that in the second case, we've enforced the extra condition that $x \in R^+$ is solid in $R[1/f]$. This gives us the category we want.

For the second cover $U(1/f)$, this is $D(R, R^+) \to D(R[1/f], (R^+ + R^+f)^\mathrm{int})$. This is first inverting $f$, then $T$-solidifying for $\Z[T] \to R$ with $T \mapsto 1/f$. The modules here are a full subcategory of $R[1/f]$-modules where $fx$ is invertible, with the extra condition of being "solid".

I claim that this whole process of inverting $f$ and then solidifying with respect to $1/f$ is also described by just an $\R\mathrm{hom}$. Namely, take $\Z[T] \to R$ with $T \mapsto f$, not $1/f$, and take $\R\mathrm{hom}_{\Z[T]}(-, R)$. This is the localization which kills the idempotent algebra $\Z[[T]]$ in solid $\Z$-modules.

So, this is kind of the new notation fitting it in the general framework. Before, for Uber pairs, you wrote the $\mathcal{D}R^+$ integral clause. Okay, you wrote $\mathcal{D}$, so let me explain what the point is here. This localization was supposed to be given by first inverting $F$ and then doing the solidification. But again, this solidification---the first claim is that this functor already inverts $F$. 

If you have something $F$-torsion, it's going to be killed by this, and the reason is you're killing this whole guy. And therefore, in particular, you're killing any module over this guy. But everything $T$-torsion is a module over $\Z[[T]]$. So this automatically inverts $F$, because anything $T$-torsion is a $\Z[[T]]$-module. 

So if you have a solid Abelian group, which is a filtered colimit of things killed by powers of $T$, then it is a $\Z[[T]]$-module. That's just a condition, so you can reduce to checking for something which is uniformly killed by some power of $T$, but then it's obviously a $\Z[[T]]$-module because it's a module over the truncated power series ring.

So then it would be the same thing to write this formula where you invert $T$, but then if you do that, it's exactly the same thing as $T$-solidification as described by the previous formula. Inverting $F$ and then solidifying $1/F$ is just the same thing as doing this here.

Okay, so basically all you need to check now, if you look at those conditions, most of them we already know. It's a localization, kind of by construction, the localizations commute, because they're both given by $\mathcal{R}$-homing out of some object, and any two functors $\mathcal{R}$-homing out of an object commute with each other, just because the tensor product by a functor and the tensor product being commutative.

So what does this translate to in terms of these idempotent algebras which determine these localization functors? It translates to a simple condition on these idempotent algebras. If you take $\Z[[T]]$ and tensor it in solid $\Z$-modules over $\Z[[T]]$ with $\Z((T^{-1}))$, you get zero. If you have something that dies on $\mathcal{R}$-hom out of this and dies on $\mathcal{R}$-hom out of that, then by messing around using this condition, you conclude that it just has to be zero.

What's the interpretation here? You can think of this as localizing away from the open unit disc, and this was localizing to the closed unit disc or away from the open unit disc centered at infinity. The reason those two cover intuitively is because if you take the closed unit disc centered at infinity and the closed unit disc centered at zero, then their union is the whole space. But in terms of the complements, that's saying if you take the open unit disc and the open unit disc at infinity, then they don't intersect, and that's exactly the algebraic translation of that fact.

Similarly, for the second kind of cover, you need that $\Z[[T]]$ tensor

We assigned to this discrete Huber pair. But if you then want to get a sheaf, you send a rational open. You have to send that to modules, our modules in this D-O-of-U, discrete... I don't, maybe I want to say, so O for the discrete ring, okay?

So this recovers the topology in the G-model, nisr, okay, yeah. So the and in particular, so what is the unit object in this category? So you take R, and then you invert G, and then you derive-solidify with respect to all the FI over G's.

So necessarily when you do this object for a completely general solid ring, you're going to end up with some derived phenomena here. I was hoping to get to it today, but we'll probably discuss exactly how that happens later. I want to also make another remark, which is that this sheaf, Dr-r+, actually lives over a much smaller subset, a closed subset. Let's say SP-R-star-R-plus-r. If you remember the set of topologically nilpotent elements, then here you add the condition that if F here is topologically nilpotent, the valuation has to be strictly less than one. 

So for an individual F, that's a closed subset, and then it's a big intersection of such things that's a closed subset of this topological space. My claim is just that if you take this sheaf of categories and you restrict it to the open complement, you just get zero, so it's really living over this closed subset here.

On the other hand, Huber considers Spa-R-r+, which is the continuous valuations less than or equal to one on r+, and that's the same thing as a potentially smaller subset, generally smaller subset, satisfying a stronger condition, saying that if you're topologically nilpotent, then for all gamma in gamma, there exists an N in N such that the valuation of f to the N is less than gamma. So there's some subtlety here.

The space that Huber localizes over is actually smaller than the space that we localize over. But Huber shows, and I should say that this Dr-r-plus does not live over, but there does exist a retraction, which is actually a quotient. By definition, it was a subspace, but you can actually realize it as a quotient, and then you do get a sheaf of categories on Spa, and that's the correct way to get a sheaf of categories on Huber's topological space. It's the retraction that's kind of the good map in the sense that this is the quasi-compact map. So in general, you get more flexibility for localization using this picture than with Huber's picture, and the kinds of extra things you get are something that we already discussed, like the so-called functions on the closed unit disc, which arise from the structure sheaf in this general setting but don't arise from the structure sheaf in Huber's setting. You can analyze these things, but I think I've now said enough. Thank you for your attention.

Rational opens here you can actually parameterize it by similar data, but with an extra condition that these things generate an open ideal. And then if you pull those rational opens back here, you get exactly the corresponding rational opens as expected. But if you take a general rational open here not satisfying that condition, and then restrict it---no, no, if it does satisfy that condition and you restrict it, you get the correct thing. But if it doesn't satisfy that condition and you restrict it, you get something new which is not necessarily even quasi-compact, not a rational open. So you have to write it as a union of rational opens, and this is kind of like taking the open unit disc and writing it as a union of closed unit discs, which is a typical example of that phenomenon.

You said something about getting a structural shift last time. Can you comment on this, or will it come later? I was hoping to get to it again today, but I didn't. So the structural shift would just be you take $R$, which is living globally, and you apply the localization functor to get something living here instead. And that's this object here, and one can analyze it and so on and so forth. And it's also some center of a certain category. Is there a way to think of it as a center of a center? I don't know, but these are your symmetric monoidal categories, so it's just the unit, I mean, the unit of the symmetric monoidal derived category in this sense.

You claim that when you take $\mathcal{M}od_R$ of this, this is actually a good thing, which so it is associated to the---in good cases it is associated to the rational---except that sometimes you have to do. So this is, I mean, this will also correspond to an analytic ring, but you know, in the derived sense. So you have to---the notion of analytic ring that we've discussed so far, you had an ordinary condensed ring in a full subcategory. Here you need to not just remember that ordinary derived, you need to remember some derived enhancement of it as well. But then it is enough to just remember the ordinary abelian category of modules over the ordinary thing.

And besides the derived stuff, there's also a quasi-separated issue, where the value of the structure sheaf might be different from Huber's, even if it lives in degree zero, it might---the quotient might not be by a closed ideal, and so it might still differ from Huber's. But again, in practical cases, that doesn't show up. And I guess even inverting $G$ can introduce non-quasi-separated behavior in general. We'll discuss this in coming lectures, all of these.

In the so, considering those two spaces in $\mathcal{U}$-theory where you have a retraction, which I think he maybe used a slightly different notation, but anyway, this $\mathcal{S}^{p,v}$ and $\mathcal{A}^{i}$. So you have this subspace living in a slightly bigger thing, and there is a retraction which is spectral. So you have shifts, you can consider shifts on both things. What I said, I think, is correct, that the restriction to the subspace is like the direct image. Okay, then you have a shift, like in this shift of categories and some sense on the full $\mathcal{S}$, and then you take the direct image yes to the subspace, which is like restriction.

Something I probably, but the question can also be asked about, so you have particular sheaves on the full $\mathcal{S}$, which are direct images by the inclusion of sheaves on the subspace. So the question is, like, in this context, so you have your, let us say, you have a rational open in the figure $\mathcal{C}$, and you consider its intersection with the smaller $\mathcal{S}$. You said that you can write it as the union of $\mathcal{C}$, so you can evaluate your shift by in this way, by inverse limit of those. Is it equivalent to the shift to the value on the original thing in the big $\mathcal{S}^{p,v}$? So like, whether the shift of categories is the direct image of its restriction to the subspace, which---so you have on $\mathcal{S}^{p,v}$ $\mathcal{R}$ plus

\end{unfinished}
% !TeX root = ../AnalyticStacks.tex

\section{\ufs Analytic adic spaces (Scholze)}

\url{https://www.youtube.com/watch?v=YLQt_tV4tHo&list=PLx5f8IelFRgGmu6gmL-Kf_Rl_6Mm7juZO}
\renewcommand{\yt}[2]{\href{https://www.youtube.com/watch?v=YLQt_tV4tHo&list=PLx5f8IelFRgGmu6gmL-Kf_Rl_6Mm7juZO&t=#1}{#2}}
\vspace{1em}

\begin{unfinished}{0:00}
Good morning. Today I want to talk about what is called analytic geometry.

I immediately want to point out that there is some conflict of terminology here. Basically, using what we've said so far, all rigid analytic spaces give rise to analytic spaces in our sense. But within the world of rigid analytic spaces, there was already a qualifier called "analytic." 

Within the context of any space, what is this conflict of terminology?  I didn't come up with a good name for them, but there's one name that's already to some extent used in this world, so I will call them "T" in this lecture. Peter, could you write a bit bigger, as it's a bit unclear. 

I don't want to overuse the word "analytic" in this.

Here's the definition where this name is already used: it's called the "affinoid" case. This is terminology that's already in use. If it has a topologically nilpotent unit, it's also called "adic."

The example to keep in mind is, for example, $\T_p$, where $p$ is a topological unit. In fact, any non-archimedean locally complete field will do. We always assume completeness. Anything above you will also be called an "affinoid space."

This is usually called "analytic" if it is covered by such affinoid subsets. We didn't really discuss the full definition of rigid analytic spaces, but we discussed the algebra that their rings have, and there's some way of doing them which will not be all that important for us right now. Basically, we told you what the open subsets are, and then you kind of do the obvious thing.

It's actually equivalent to a property that the speaker discusses in the green text, which I won't go into.

The speaker also clarifies that we did not define "analytic spaces" in our sense, and that there is a subtlety about sheafiness in the classical theory of rigid analytic spaces.

The speaker then mentions the question of whether in the theory of localization and Dustin talked about last time, when you do this, you get some condensed thing, possibly over a rational domain, and it is unclear if this always gives the right thing.

The speaker then explains that if you have any space, you can associate to each point a completed residue field, which can either be the residue field or a complete valued field, and it is asked that there are no points where this is a discrete valuation ring.

Basically, the setup is that the regular fields are always of these forms. And when you have such a complete non-archimedean field and there is some unit in there, you can lift this to a small neighborhood, and that's why these are called "adic."

The intuition is that there is a world of schemes, which sits on some world of formal schemes, maybe with some adjectives. These all sit with the context of rigid analytic spaces. But then these "adic" spaces are some of the ones where you start with a formal scheme, but then remove all the scheme-like points, so they are at the other end.

The speaker then mentions something like $\T_p^0$ that can also be considered a kind of $\mathbb{SP}_1$, but says the generic fiber in this sense is unclear.

Finally, the speaker wants to say a few words about the structure of such things.

Definitions. So, let's say $\pi$ is a topological unit. You can then take the ring of definition, because you can always assume that they contain any given power-bounded element. I mean, some particular power-bounded $\pi$. You could also do it such that you take any $\pi$, and then some power of $\pi$ will anyway be a unit. And then actually, it's automatically the case that $\mathcal{O}_K/\pi$ is a zero ring.

That's an actual exercise in looking at the axioms of the topology. So you might think that it could have some rather more subtle topology, but actually, it's always supplementary. And then you can actually write $\mathcal{O}_K$ with Banach algebras. For example, you could say that the norm of a function $f$ is $2^{-\max\{m|f\in\pi^m\mathcal{O}_K\}}$. This means that the norm of $\pi$ is $1/2$ the norm of $\pi^{-1}$. Of course, there's some kind of arbitrary choice that I made here, both in uniformizing $\pi$ and in the number $1/2$ here.

The trivial counterexample is the zero ring, where the norm of $\pi$ would be zero. I don't think it has a topological unit, or maybe that's a counterexample to show that anything that is an algebra over $\Z$ is probably okay. So, whatever you figure it out.

There is an equivalence between certain $p$-adic and algebraic geometry of rings. The objects are $p$-rings admitting a $p$-valuation such that the norm of $\pi$ is basically between $0$ and $1$, and the norm of the inverse is also $1$. But as a moment, you actually only care about the topology used by the valuation, and everything here is completed $p$-ring and continuous. Of course, the zero ring is still a counterexample to what happens to zero.

Okay, so these are the basic algebras that we consider here. You could also talk about this in the language of $p$-algebras, but you shouldn't really fix the norm, but you should just ask defined by some norm with this property, because we have...

Right, so now I want to state a theorem that I proved some 10 years ago or so, and the theorem that I found extremely striking and surprising at the time. As was discussed, there are all these issues that for general Huber rings or Huber pairs, what Huber defines is not always actually a sheaf. But let's just say that $A^+$ is quasi-coherent if what Huber defines is a sheaf of Huber rings. And I did not at all expect that in those cases where you can prove that something is a sheaf, you can probably also prove other nice properties, because then that's probably the reason that things are well-behaved. What I did not at all expect is that you could prove any non-trivial theorem whose only hypothesis is that something is a sheaf, and then other good things happen, because just asking for the sheaf condition is the first thing that could break, but I didn't expect that if you're in a situation where it is a sheaf, why should other...

But here, I proved the following statement: If $A$ is a quasi-coherent sheaf of finite projective modules over a Huber pair $(A,A^+)$, then any finite projective module over $A$ can be extended to a sheaf of finite projective modules over $A^+$. And I mean, $A$ is just a direct module, so certainly this is still a sheaf. It's a sheaf of finite projective modules, and actually, it's one that's locally free.

A locally free $\mathcal{O}$-module is called a vector bundle. Whenever you have a ringed space, you can always talk about these modules which are locally free or of finite rank. These are called vector bundles with respect to that ringed space. It's easy to see that this is a functorial construction, but the highly nontrivial, surprising fact is that there is actually an equivalence. When you can glue vector bundles, you can define the category of vector bundles. Somewhat later, they also proved that there is some version of this that works for some kind of coherent modules, but I don't want to state the precise result because it's extremely...

Let me just say there is a result that if the ring is nice enough, then you recover the expected result that you can do this even for rings, but you have to be careful because localization might not preserve all the properties you want.

All right, so this is a very nice result. Basically, the aim at the time was just to prove that on perfectoid spaces you can define vector bundles, but the argument actually worked whenever the space is quasi-compact and quasi-separated. I could kind of follow their argument line by line, but it's a rather tricky argument where you really have to do some changes of bases to make things better. It's actually an analysis argument.

Today I want to explain different proofs that we can give using our general theory of solid modules. I should maybe say that some 10 years ago, when this result came out, I was also talking to K and he was already telling me that if you work in the derived category and look at some kind of $\mathbb{I}^{\mathbb{N}}$-up modules, then you can just glue. I was like, okay, but what does it even mean? We have all these problems, I think it's not a structure sheaf, and it seems highly non-obvious how vector bundles and so on would work. So what does this extremely general result actually tell you about such constructions? I don't know whether he ever figured that out, but the goal of this lecture is to show that this is not just some fancy, abstract result, and to get down to something more concrete.

Now, the idea of continuous valuations means that when I form such a rational subset, I should assume that the ideal generated by 1/a is an open ideal. But actually, once you have a topology where 1 is a unit, then a must be equal to 1.

First of all, $H$ is given by taking the completed power series algebra in variables $G_1$ to $G_N$, and then modding out by the ideal generated by $dG_1 - F_1, \dots, dG_N - F_N$, and taking the closure.

This is basically why that works on this open subset. But first of all, you didn't invert $G$, because now the ideal generated by $G$ contains $F_1$ up to $F_N$, but they generate the unit ideal. So $G$ has become invertible even without taking the closure.

And then on this, you have $T_1$ which is $F_1/G$ and all its completed power series. This is what could happen on this subset where this is less than or equal to that.

But I mean, Huber was always working with complete topological rings, so when you take a quotient by something, you also take the quotient by the closure. And then, for example, this is a $B$-algebra, and the quotient by any closed ideal is still a $B$-algebra, so it's still okay.

This is maybe not so hard to show, and you all recall the endomorphism part. The image of $a$ is just the thing that was defined last time. You define a category ring structure by asking for this completion for all these elements, and then you somehow look at what the completion of the unit actually is. This is what the $S$ uses, and it has a very similar formula.

But then, take the derived quotient, where here or and elements $I$ want to $R$. This derived quotient signifies the derived base change from only algebra, but they just modify all the variables $X_i$, so basically you're setting $X_i$ also to zero, but not just in the stupid way, but in the derived way. And completely, this is computed by a possible complex, and in the middle there are some $Ext$ there, a cotangent differentials.

And this is just by taking the standard resolution of $\Z$ over here. It will be an animated condensed thing, and that's why I will consider this. I don't really want to talk about animated things today, but yeah.

So, it's certainly some kind of algebra. Let me just give a sketch of it, and for the second part, so you can write $A^1/d$ as $A^1$ mod $d$, and now it's not really necessary whether I take the derived or not, but it's also turn the derived $S$ because what I will write down is a regular sequence.

And then you solidify, solidify, solidify these, so some exact operations category. This thing is just computed by a possible complex, which is a complex of finite complexes of some $C$ thing. Just understand what happens when you solidify these things, but this is exactly what Dustin already computed that something.

Okay, and so certain things that saying it's the same thing as stating all, which is a context, but zero is just a usual structure. And then you take the $C$ separated a small inter, the inclusion from light or not, it doesn't really matter, probably separated condition set inside of all condition set, it hasn't left the join. I'll take any $x$ to the quotient separated portion, and this is making the operation of taking the maximum $h$

Has a good effect that if $x$ has some kind of algebraic structure like being a group or ring, and so on, and the functor that is a preservation of final product is available both for the adic setting and all condensed sets, yes. I mean, also, if you start with the adic condensed and then from the maximal condensed also adic and so on. So, thank you.

Right, so completely if you write $X$ as a condensed set, it's telling you to compute the joint, and you can always write any condensed set as a quotient of a discrete set. Just take different profiles depending on it. Then, there exists the minimal compact injection $\overline{R}$ of $R$ that is also an equivalence relation and containing $R$. So, it's in some sense taking the closure of $R$ intuitively speaking. But sometimes this adds new things where the relation is not spreading, you add that as well, and then you do some kind of transfinite induction or you just intersect all possible containments.

Then $X$ is this quotient, this is $X \text{mod} R$, because this inclusion relation was called the compact injection. So, for example, if $X$ happens to be a group, then you can always also find a surjection from the separated condensed to the group, and then the relation is just the quotient by the subgroup. In that case, you're really just taking the quotient by the closure of the subgroup, the closure in the sense of small injections.

Okay, so here's one other thing I should recall, which is what do compact injections have to do with spaces? Let's say $X$ is any sequential space. To recall, it's a topological space for which the condensed set is fully faithful in the closed subsets of $X$. Relatively to the compact injections, taking any closed subset here, why is it actually a further compact injection? While to check that, the definition is that whenever you pull back to some profinite set, then the preimage is a compact injection. But this precisely means that the other way is also compact, and so the preimage in this case is just the closed subset of this profinite set by the closed subset here.

Okay, and so back to the sequence, what is this? Well, this is just the thing, and then you take the extension. Okay, and then if you take a separated quotient, well, this guy is already separated, and then by the above, and because this here is actually sequential, comes from a sequential topological space, this precisely means that I should just replace this one here by the closure, which is what.

Right, and so this gives us some relation, but we're interested in a somewhat more precise relation, which is the following theorem, potentially due to Lurie, except that he didn't use this language to phrase it, but I mean the proof is really his. That a $p$-adic completion is $p$-adically complete if and only if all $p$-power series all have bounded denominators. 

Let me give the proof. This is clear because by the general descent results from the last lecture, we have the sheaf of categories, but in particular, you get a sheaf of rings, and we just take the units. And the non-trivial and I think somewhat surprising result is that if it so happens to be a sheaf, then actually it is the right thing. Let me get the other direction. And before I start, note that to check this property that all $p$-power series are $p$-adically separated, it suffices to show that this is true on a further refinement of the cover, because you can always recover it by all the smaller values

To get the good properties of all these views, we need further refinement. I don't want to go through all the commutative refinement, but just note that some following concepts are also always refined.

We can always cover any space with a Zariski-type cover, with at most one and at least one nonempty intersection. So the other type of cover is not necessary when you look at new units. This reduces to the following key steps.

Assume we have a sequence of $x_n$ that is Cauchy. Actually, what does "Cauchy" mean here? I mean, this is a sequence of elements in $\R$, so let me denote it as $a_n$. This satisfies the Cauchy condition, which says that for any $\epsilon > 0$, there exists $N$ such that $|a_m - a_n| < \epsilon$ for all $m, n \geq N$. 

This is always true, just because any element in the limit can be written as a sum of $f^n$ and $1/f^n$, where the $f^n$ come from one side and the $1/f^n$ come from the other side.

Assume this Cauchy condition, which is always true for a Pro-$H$ space, because it's just one specific instance of the Cauchy condition.

Then, the derived object by $f$ and the other thing is also the identity. In other words, it agrees with what we defined earlier. What does "the $D$" mean? It means that modification by $-f$ is actually injective here and has closed image.

Let's prove this. What you have to see is that if you look at the map, and then multiply by $f$ or $1/f$, this is injective with closed image.

But actually, because of the star condition, if you want to show injectivity, let's assume you have something in the kernel that also still lies in the kernel when you replace $a$ by one of those two rings. But then over those two rings, it's easy to see that this map is injective.

And similarly, this map actually has closed image, because the image is precisely the kernel of the next map. Using this, you can also check that if it has closed image over those rings, it must also have closed image here.

So it's enough to check it when we replace $a$ by those two rings. In other words, we can assume that either the absolute value of $f$ is less than or equal to 1, or the absolute value of $f$ is greater than or equal to 1.

If $f$ is less than or equal to 1, then $T - f$ can be shown to be a closed ideal, because you can just successively "peel off" the highest coefficient. Something similar holds in the other case.

Basically, if you have an element here, you can really just by looking at coefficients see what happens. And also, one of them becomes a torsion module anyway, because after not doing any further localization, it's only the other one you have to check.

So this is a funny argument that $K$ found, that if you just assume the Cauchy condition, then you can reduce this problem to the simple open covers where you just take a portion by one element. Because then the question of what the $R$-portion is is really just the question of whether this one $M$ here is injective and has closed image.

Okay, so this finishes my discussion about the Cauchy condition and the relation between the $H$-structure and so on.

To the second part---so, here are some definitions. What does the "me" mean? Let $R$ be a ring with a certain condition structure for now. Then, something like this definition $S_6$, I believe. If it can be represented by complexes up to some degree $n$, but $\Z$-graded, then all these terms must be finitely generated. You could also say, try to project this with negative degrees as well, because I want to say something else in a second.

And then, importantly, if your ring was an $\mathfrak{a}$-adic ring, then this will just be the condition that each group is a coherent module---the case just finitely generated module. But if you're not in an $\mathfrak{a}$-adic range, then there's some coherent module structure form in some category, usually. And because you didn't ask, the relations between the relations are finitely generated, and so some of the infinite complexes they capture the idea that you have modules which are finitely generated with as many relations, as many further relations.

So the point is that over in the $\mathfrak{a}$-adic case, it's a good class of finitely generated modules behaving nicely; you don't have that over a general ring, but the ring still has a nice class of complexes. And let $\mathcal{P}$ be perfect if there is such a representation which just wanted to find it---just a finite complex, projective modules.

Then the theorem is that, let's say $\mathcal{A}^+$ is any $H$-module. Then sending any $A$ to the following things: so, on the one hand, you can take perfect complexes, all the ones or you can take two coherent ones that are, say, in degrees greater than or equal to zero or greater than or equal to $n$ for any $n$. We could also take all perfect complexes, but you could also take those perfect complexes which have such a representation in some interval. So, perad means those that can be represented using a complex sitting in certain degrees.

These are all functors that take a real subset to the infinity category, and they all agree, and I claim that they all share some nice properties on the right. I'm always just thinking of $A$ as a cochain complex being a reason; there's no dependency on you on the right-hand side. I did not, thank you.

And I guess finally, I'm now extending all these notions to animated rings. And so, okay, so either you know what all these things mean when this guy is just an animated ring, then it's true as stated, or you secretly that you maybe in the $\C$ case where it's just $\C_0$, and then I just told you what they are. Either way, these are nice properties, in particular, the $\C$ case, if I expect the vector bundles, this cover.

Yes, Peter, is there a reason you didn't state the like pseudo-coherent $\mathcal{A}$-module? No, okay. You have to be careful that the good way to make $\mathcal{P}$ restrictions on $\mathcal{C}$ and a perfect complex are different. Yeah, for perfect complex, you assume that there is a representation as an actual complex that's in some range of degrees. For coherent sheaves, you can just make a more naive thing like bounded below, bounded above, although no, there is actually a reason. If I no, there is a reason because look, I don't think localizations are flat in any

Okay, this will have a meaning at some point when this is animated. We didn't discuss this yet, so if you feel more comfortable, just assume this sits at zero, and this is a sheet of any categorical and so this certainly contains fully faceful, just the modules over the underlying $\mathcal{S}$ ring, which is always true for any anticyclic ring whatsoever.

And this certainly contains all these other subcategories. But yeah, basically some kind of a finite solution here. And so by virtue of this being fully faceful, it means that the only thing we actually have to prove in all these settings is that if you have an object here, which locally happens to lie in all of the sub-categories, and actually so globally--maybe at the expense of pretending that I can replace all of you by $\mathcal{A}$ again--that this is such that over the base change from $\mathcal{A}$ plus, one of the sub-categories, then so does the conditions that we put.

Okay, and so maybe the first idea you might have is that well, let's first check that this condition of being a discrete module in the sense, and then the success conditions. But this actually does not work. Warning, this is a warning that Dustin already made, but I want to reiterate it because it's important. The condition does not just say globally, it's do locally, which will not be globally. Let me actually quickly sketch one example. I'm not sure it's the easiest example, but somewhat instructive.

Let's consider the curve, so it takes the $\mathbf{T}_m$ spheres over, let's say, $\P^1$, and then you can take the quotient by, let's say, other units. This what's called a table, different with the analog of taking $\C^\times$, complex space $\C^\times$, and $\P^1$ by some topological element. And then this becomes an adic curve, complex numbers do similar, or whatever rigid geometry started. And let's assume that our $\mathcal{A}^+$ is some large open subset here, so basically remove a small disc around the origin. But large enough so that there is non-vanishing global monodromy. But should really be sub-only and not this. And then there is $\mathcal{C}$, the projection that--yeah, ring. And then you can find what I will call a lower streak of sub here. What is this? Well, locally, this map is just split, and so locally on the base, like the $\P^1$ are just discs, a union of copies of the base taken by the integers, and then the sheaf is just a direct sum of copies of this same. And so this is how it's defined locally. And well, it's obviously a sheaf, and so it glues to some sheaf on the base. So this defines for you an object $\mathcal{A}^+$-solid, because locally it does. And it is locally discrete because locally, as I said, this cover of splits, and this is just a bunch of copies of your base. But you can check that globally it is not split.

Okay, so we can't hope to use this category instead. We will use two other notions. We said you can first also define this kind of pseudo-coherent, where now I mean, yeah, it's again the complexes which are represented by something that's bounded to the right, so goes to the left, and the cohomology are in this case, I just say they are

And you actually just use---I first of all being bounded to the right is a property that globalizes because you're locally bounded to the right, and then globally you're just some kind of finite limit of that, so you're still bounded to the right. And so then you need to check this $x$ condition, and the only thing you need is that these localizations have the finiteness amplitude. Otherwise, there would be some issue, but during that localization, it's fine.

So this finiteness condition on the big category---this can be checked, and then actually once you have the finiteness condition, the thisness is also something that can be globalized. But this actually needs to be proved, and some equivalent conditions need to be checked.

So the issue there is that it's in some sense easier to work with the second subcategory of $n$-nuclear objects, and these are somewhat more general than just the $B$-nuclear objects, but not the $C$-nuclear ones. Let me just give a quick definition here. So all the $B$-nuclear ones will be modular, but things are somewhat more general than that. In particular, they contain all the $B$-nuclear objects, but not the $C$-nuclear ones.

The name is inspired by this class of nuclear vector spaces defined in functional analysis. It's not a completely precise translation, but in spirit it is. The condition is that the internal Hom to $P$ factors through a trace-class map. This in some way encodes the idea that all maps from the dual of $B$ tensor with something are trace-class.

In general, the nuclear modules are generated by shifts of just $B$-module functions. You can generate them by full limits, and in general, referring to more general analytic rings, by where these are all trace-class, meaning they come from elements in the dual. Whenever you have an element in the standard pro, you can produce an $M$ from that, and these are all trace-class. And then if you take a sequential form of trace-class maps, you can check using that internal Hom commutes with limits that they always have, and conversely, when you try to present nuclear objects in terms of how they're generated by compact projectors, the map will actually factor through a trace-class map.

Okay, and so yeah, in particular, over $\Q_p$, this is stuff generated by $B$-nuclear objects, so this is still a rather large class of guys. It will actually, I mean, there are actually $C$-nuclear and so on. One very nice property of this nuclear module category is that it forms a universal $L^1$ approximation in some sense to the right-hand side, and using this, you can also define the $L^1$ spaces.

So back to here, you have the subcategory of $n$-nuclear objects, and the finest class that is globally closed. And so you want to check that if you have some $C$ that satisfies this condition locally, then you want to check it globally. And for this, the ideal situation is that just the two sides of the morphism can localize---the right-hand side definitely localizes, as it's just some tensoring. The question is, does the left-hand side also localize? And it actually does, and the key thing is that you can use the same argument that was used last time to show the different localizations commute with each other, because they actually show that all the localizations are given by internal Hom from some object.

Strictly speaking, for an algebraic localization by inverting an element, it's not literally an internal hom from some object spaces and rational subsets, like inverting an element is not a rational subset. But in general, there's a suitable localization. In any case, you can always refine further so that it is.

I just wanted to mention that, in case there was some confusion. Actually, with other types of localizations, it's also commutative with elements, because it limits. 

So, now we have two classes of things that are in the sense of a special category, and the question of commutative localization seems to suggest that the sketchy argument given doesn't use that you can put anything, not necessarily compact, instead of $P$. This commutative localization is contrary to what we have for schemes, where you have to take the underlying hom.

However, in this case, the localizations commute with all products, because they have a left adjoint, which is extremely important for the theory of compact SP-coherent modules. In these situations, the pullback functor has a kind of coherent lower shriek functor, which is the left adjoint. So the pullback preserves all limits, but it also satisfies the projection formula. This is precisely equivalent to the internal hom, so it's true for anything instead of $P$.

For the specific localization we're interested in here, to Zariski open subsets, this is actually always true. We made two steps in our proof: we isolated two different classes of modules that can be glued, and the last step is to isolate the intersection and show that the two central modules over $S$ are just an inclusion. They are certainly in the SP category, and this is actually called...

So then we finally show that we can glue these two ones, and there's still a little bit to show that all the other super- and subscripts I put also glue, but this is actually easy.

So far, I didn't do anything that seems remotely like analysis. But for doing vector bundles, there is actually some new curious power series that you have to undertake somewhere. So somewhere, there has to be some actual work.

Let me show how it's done. We can assume we have some complex, and by shifting, you can assume it starts at $\Z$. I also said you can assume that these are all just...

Okay, you have such a complex. How do we know that it is nuclear? Now I certainly have a map from $\P$ to here, which is just the identity here, and my complex is just zero elsewhere. This is a map from $\P$ into $C$. Being nuclear, it means that it's trace class. If you actually have a section in here, then again you can factor this over some approximation to $C$, and when you think a little bit about what this means, you realize that there exists a diagram where $G$ is $PR$, but then this $F$ can actually be split back. You get a morphism here, and what you see with this is that the map from $\P$ to the complexes over the morphism from $G$ to $\P$ for some $G$ of $\P$ is just some playing around with the fact that $\P$ is projective and so on, and the definition of nuclearity reduces this.

So in the end, you see that there is actually a way to put a unit here, and then it's actually enough to show that this guy here is actually perfect. Certainly, this representation doesn't show that, but they claim, as far as the
Particular compact operator, and then any perturbation of a Fredholm operator by a compact operator is still a Fredholm operator. Some have properties that are related to the kernel and cokernel dimensions.

This is an argument that you just have to execute slightly more carefully in this situation. So, there's some Fredholm operator on this like base, something like this. And then, you can show that, yeah, the cone on the Fredholm operator is a perfect complex, which is a better way of thinking about it as the K-theory generator.

This is the only place where you actually have to play with some, I mean, you have to have a matrix representing the map, and then you play with that. It's easy, but there's this step where you actually feel like you're doing a little bit of analysis.

Okay, before you erased the blackboard, you had a factorization where $P$. You said that you factorized it through $P$, and one map is trace class, right? But it seems to me that you have to use the $\text{Hom}$ space. I mean, the data will not give you that the map from $P$ to $P$ is, you need some homotopy to do it. I mean, I think the data is probably that where it's executed, look at the complex geometry, not where we do precisely the same argument like this linear vector space.

So, it's an unfortunate thing that I'm probably screwing up if I try to do it right here. But in any case, the data probably means that the map from $P$ to the complex is homotopic to a map that factorizes through a finitely generated free module, up to homotopy. And then you can use this finitely generated free module to modify the complex, so that now the degree zero part becomes zero, and then you can just keep going.

\end{unfinished}
% !TeX root = ../AnalyticStacks.tex

\section{\ufs Solid miscellany (Scholze)}

\url{https://www.youtube.com/watch?v=87-wuqGA8GE&list=PLx5f8IelFRgGmu6gmL-Kf_Rl_6Mm7juZO}
\renewcommand{\yt}[2]{\href{https://www.youtube.com/watch?v=87-wuqGA8GE&list=PLx5f8IelFRgGmu6gmL-Kf_Rl_6Mm7juZO&t=#1}{#2}}
\vspace{1em}

\begin{unfinished}{0:00}
e
okay  so  I'd  like  to  start  um  welcome
back  everyone  uh  so  I  think  um  we're
about  ready  to  move  on  to  a  new  topic
but  I  just  want  to  um  finish  off  our
discussion  of  the  solid  Theory  with  some
miscellaneous
miscellaneous  uh  facts
so
um  so
recall  that  we  had  this  uh  procedure
that  if  uh  we  had  a  pair  say  a  A+  for
this  is  some  solid
ring  and  this  is  A+  uh  a  sub
ring  of  power  bounded
elements  uh  oh  sorry  just  one
um  then  we  Associated  to  this  well  a
certain  solid  analytic  ring  uh  which
was  specified  by  this  mod  a
A+
um  and  it  was  um  obtained  by  uh  forcing
all  of  the  elements  in  A+  to  be  Sol  to
become  solidified
variables  um  but  recall  that  we  also  had
the  thing  on  the  level  of  the  derived
category  and  this  is  so  what  would  if
you  if  if
A+  if  A+  is  not  a  subing  of  a  z  ah  then
then  when  you  enforce  this  condition  you
might  lose  the  uh  a  a  might  not  might  no
longer  live  in  that
category  so  you'd  have  to  apply  the
localization  and  get  a  new  a  a  sort  of
sort  of  the  completion  in  some  sense
okay
yeah  um  but  we  also  had  this
uh
so  and  recall  that  this  was  defined  by
uh  this  was  defined  by  as  it  was  defined
as  a  full  subcategory  of  here  consisting
of  those  objects  whose  homology  lies  in
here  for  all  I  and  we  showed  that  uh  you
have  a  nice  left  joint  with  the  exact
same  properties  as  the  left  ad  joint
here  symmetric  monoidal  and  so  on  and  so
forth  but  there  was  this  subtlety  that
this  is  not  necessarily  um  well  in  the
full  generality  of  an  analytic  ring  this
is  not  necessarily  equal  to  the  derived
category  of  this  and  the  basic  reason  is
that  if  you  take  some  basic  object  here
and  apply  derived  solidification  it
might  not  live  in  degree  zero  uh  that's
kind  of  the  obstruction  um  so  in  fact  uh
the  generator  here
so  so  there's  a  compact  projective
generator
which  is  equal  to  you  take  the  compact
projective  generator  we  know  In  Love
from  the  solid  Z  Theory  and  then  you
simply  tensor  it  in  the  sense  of  the
solid  Z  Theory  with  this  solid  ring
a  um  and  that  we  know  that  it  doesn't
matter  whether  you  take  the  derived
tensor  product  or  the  ordinary  tensor
product  there  because  as  Peter  showed  in
one  of  his  lectures  this  is  flat  in
solid
z  um
so  that's  what  means  that  if  you  did
made  that  the  if  you  did  the  derived
analog  of  this  definition  it  would
literally  just  be  the  derived  category
of  this  ailion
category  um  but  then  what's  the
generator
here  compact  generator  generator  of  this
derived  category  uh  sorry  uh  of  this
thing
so  so  what  do  you  do  you  take  this
object  here  and  you  localize  it  with
respect  to  this  derived  localization
functor
uh  so  you  derived  a  plus
solidify  and  uh  maybe  I'll  say  uh  so
theorem  so  first  the  I  want  to  explain
is  that  this  this  lives  in  degree
zero
and  that  means  it's  a  a  compact
projective
generator  for  the  aelan
category  uh  mod  a  A  plus
solid  and  uh  and  the  derived  category  of
this  ailan
category  uh  is  the  is  the  drive
category  we  defined
there  so
as  long  as  your  underlying  solid  ring  is
not  derived  in  any  way  then  in  fact  the
whole  theory  is  you  know  or  that  doesn't
have  any  the  whole  theory  is  kind  of
flat  so  to  speak  uh  determined  on  the
aelan  level
um  does  it  so  is  it  a  change  from  what
was  said  before  or  or  because  I  remember
that  it  was  not
clear
in  the  previous  talk  I
mean  it  was  not  clear  that  this
was  well  that's  that's  why  I'm
explaining  it  now  yeah  yeah
um  okay  so  so  how  what  is  the  how  do  you
access  this  derived  as  solidification
so  to
so  being  being  uh  being  A+  solid  means
that  you're  solid  for  any  variable
mapping  in  so  um  what  I  want  to  claim  is
that  uh  this  uh  so
claim  is  that  the  this  derived  A+
solidification
um  uh  you  can  actually  write  this  as  a
filtered  Co  limit  so  it's  all  happening
element  wise  in  A+  and  you  can  write  A+
as  a  union  of  of  rings  which  are
finitely  generated  over  the  integers  so
let's  say  R  subset  A+  R  finite  type  over
z  um  and
then  you  just  forget  to  write  the  square
in  a  A  plus  d  oh  yes  thank  you  yes  I  did
yeah
uhhuh  yeah
because  iritating  asking  the  difference
between  when  there  is  a  square  yes  yes
uh  sorry  that's  yeah  um  so  you  so  you
can  take  M  which  is  a  priori  an  a  module
but  you  can  A+  maps  to  a  and  therefore  R
maps  to  a  you  can  treat  it  just  as  an  R
module  um  and  then  you  can  do  the
derived  uh  uh  r
solidification
um  so  this  is  a  filtered  Co  limit  so  the
point  is  well  there's  two  things  to
check  first  of  all  you  need  to  check
that  well  there's  a  map  you  need  to
check  that  this  is  derived  A+
solid  um  and  you  also  need  to  check  that
um  in  this  functor  if  you  have  something
which  is  uh
well  yeah  and  then  you  need  to  check
that  if  you  map  out  to  anything  derived
uh  A+  solid  then  it  doesn't  see  the
difference  between  this  expression  and
this
expression  um  so  why  is  it  derived  A+
solid  um  because  derived  a  plus  solid  uh
is  the  same  thing  as  uh
so  is  the  same  thing  as
derived  uh  our
solid  uh  for  all  our  subset  a  plus
finite  type  just  because  it's  simply  an
element  wise  condition  on  the
algebra  um  and  uh  so  so  certainly  the  r
term  in  this  thing  is  derived  R  solid
but  also  any  further  term  in  the  uh
filtered  Co  limit  after  the  r  term  is
also  going  to  be  derived  R  solid  because
it's  even  has  an  even  stronger  property
of  being  derived  like  RP  Prime  solid
where  RP  Prime  is  bigger  than  R  um  so
there's  a  co-  final  system  of  things  in
this  filtered  cimit  which  are  derived  R
solid  for  any  fixed  R  and  we  know  that  a
filtered  Co  limit  of  uh  of  R  solid
things  is  R  solid  so  it's  going  to  be  R
solid  for  all  R  and  therefore  it's  going
to  be  A+
solid
um  and
um  so  and  then  mapping  if  you  map  out  to
something  a  plus  solid  then  in
particular  it's  R  solid  for  every  R  and
by  a  similar  argument  you  see  that
there's  no  difference  on  maps  from  M  to
this  or  this  filtered  cimit  to  this
okay
so  so  it  suffices  to
analyze  to  show  uh  that  if  you  take  this
product
z  uh  tensor
Z  solid  a  uh  and  then  you  derived  R
solid  uh  R
solidify  uh  this  lives  in  degree
zero  um  but
uh
this  uh  we  can  do  it  in  two  steps  so  we
can  take  we  can  see  this  as  you  take  you
first  base  change  to  the  solid  R  Theory
uh  so  you  take  this  thing  and  then  you
derived  R
solidify  uh
uh  and  then  you  tensor  in  the  derive
tensor  in  the  solid  R  Theory  um  with
a  so  it  suffices  to
see  uh  that
um  uh  it  suffices  to  see  that  well  first
of  all  that  product  z  uh  tensor  zolid
are  derived
solidify  uh  well  this  lives  in  degree
zero  and  is
flat  uh  with  respect  to  our  solid  tensor
product
okay
so  by  the  way  let  me  not  to  before  I
continue  with  the  so  basically  we  have
to  analyze  R  solid  when  R  is  finite  type
over  the  integers  and  prove  analoges  of
what  Peter  already  explained  for  zolid
so  uh  we  have  to  calculate  this  basic
generating  object  make  sure  it  lives  in
degree  zero  and  we  also  have  to  check
that  it's  flat  um  but  before  I  go  on  and
do  that  let  me  note  corollary  which  is
kind  of  a  corollary  of  this  filtered
cimit  type  of  argument  so  uh  if  um  so  we
have  a  compact  projective  generator  of
this  aelon  category  and  I  basically  just
analyzed  for  you  that  it's  gotten  by
some  filtered  cimit  of  uh  the  thing  you
see  over  a  finite  type  ring  um  so  let  me
say  that  uh  so  so  if  R  is
arbitrary  commutative  ring
uh  then  well  one  thing  you  can  say  is
that  uh  solid
R  um  by  which  I  mean  solid  R  comma
R  uh  so  you  solidify  all  of  the  all  of
the  elements  in  R  uh  this  uh  is  equal  to
in  of  the  finitely  presented
uh  elements  in  in  solid  are  um  and  the
finally  presented  things  are  the  things
which  are  just  co-  kernels  of  uh  of  maps
between  finite  direct  sums  well  you
don't  need  that  but  of  these  comp  of
this  single  compact  projective  generator
that  you  have  here  um  and  so
also  uh  if  you  want  to  know  about  this
finitely  presented  category  it  formally
reduces  to  the  finite  type
case  uh
um  so  in  some  sense  all  of  the
completion  is  happening  at  the  level  of
finite  type  sub
subrings
um  and  then  the  rest  is  just  some
algebraic  filtered  cimit
um  all
right  uh  so  right  so  now  we  need  to  do
this  analysis
here
um  so  I'll  make  a  a
claim  uh  which  is  basically  that  it  it
works  just  like  in  the  case  of  solid  Z
so  if  you  take  product  Z  tensor  solid  z
r  and  then  you  derive
solidify  uh  then  this  is  just  the  same
thing  as  a  product  of  copies  of  R  and  if
I  forget  to  say  an  index  set  it's  always
going  to  be  countable
um
okay  sorry  finite  yes  thank  you  yeah  our
finite  type  over  Z  thank  you  very  much
so  by  this  discussion  if  R  is  not  a
finite  type  what  you  would  instead  be
seeing  is  so  if  R  not  finite
type  then  uh  it's  equal  to  the  union
over  all  finite  type
subrings  of  the  product  of  copies  of  RP
Prime  which  is  smaller  than  the  product
of  copies  of
r
okay  uh  so  proof
so  so  so  first
consider  uh  the  polinomial  ring  on
finitely  many  uh
variables
um  then  we've  uh  essentially  already
seen
this  um  so  we  saw  that
the  derived  uh  T  solidification  of
of  a  free  module  um  commutes  with  all
limits  and  colimits  and  uh  and  sends  Z
to  just  the  to  to  Z  bracket  T  so  um
well  so  if  you  do  X1  solidification  so
uh  uh  then  that  just  gives  you
product
um
uh
uh
uh  X2  xn  and  then  you  just  keep  going
you  do  X2  solidification  X3
solidification  and  so  on  and  it  always
just  throws  the  variables  inside  the
product  instead  um  and  in  the  end  you
you  do  end  up  with  product  of  zx1  up  to
xn  so  here  we're  so  here  we
recall  uh  that  the  if  you  do  if  you  want
to  if  you  do  ZT
solidification  of  a  a  module  tensor  over
Z  with  Z  bracket  T  um  this  commutes  with
limits  uh  in  m  and  it
sends  Z  to  Z  bracket  T  and  we're  also
using  that  uh  the  different
solidifications
commute  so  that  you  can  actually  do  the
solidification  with  respect  to  all  of
them  by  just  doing  them  one  by  one  and
when  you  do  the  X1  solidification  and
you  X2  solidified  it'll  still  remain  X1
solid  you  also  use  that  if  you  solid  if
it's  X  and  Y  solid  then  it  is  Sol  for
everything  yes  in  the  subing  generated
yeah  yeah  yeah  exactly  so  that  you  know
so  that  to  solidify  for  R  it's  enough  to
solidify  for  all  the  variables  in  R  yeah
don't  well  right  but  we  we're  using  it
oh  well  this  is  yeah  this  I  guess  it
also  follows  yeah  that's  true  the
argument  for  that  was  basically  this
anyway  but  okay  um  it's  probably  good  to
mention  it  in  any
case
um  okay  um  now  in  the  general  case
so  let's  say  we  have  a  a  quotient  so  X1
xn  rejecting  on  to
R  uh
then  uh  sorry  uh  product
z  uh  tensor  Z  solid
R  derived  R
solidify  um  you  can  get  it  in  two  steps
again  you
can  uh  zolid  uh
uh  uh  and
then  uh  you've  already  solidified  with
respect  to  a  generating  set  for  R  so
you're  actually  already  are  solid  and
the  only  thing  you  need  to  do  is  do  an
algebraic  base  change  from  this  ring  to
the  ring  R  so  derived  everything  should
be
derived  um  but  we  already  figured  out
what  this  is  that  this  was  the  product
of  zt1  up  to
TN  and  then
we  uh  so  we're  left  with  just  analyzing
this  um  and  proving  that  it's  equal  to  a
product  of  copies  of  R
but  uh  so
zt1  up  to  TN  is
nean  that  means  R  can  be
resolved  by  potentially  infinite
resolution  but  a  resolution  by  finite
free
modules  and  for  each  of  those  finite
free  modules  it's  clear  you  just  bring
it  into  the  product  um  and  then  it
follows  that  the  same  holds  for  r  on  the
level  of  this  derived  tensor
product  um  so  then
so  you  can  look  at  the  class  of  zx1  up
to  xn  modules  for  which  this  claim  is
true  with  r  being  replaced  by  that
module  and  that  class  contains  the  Ring
Of  course  and  therefore  it  contains
anything  that  can  be  resolved  by  uh
direct  sums  of  copies  of  the  ring  as  a
also  the  product  the  infinite  product  of
copies  of  zt1  TN  is  I  think  flat  over
zt1  TN  again  it's  a  property  of  Nan  or
coherent  yeah  at  least  in  the  as  a  the
usual  product  but  here  it's  condensed
Mass  prod  but  still  I  think  for  every
condensed  set  it's  the  value  of  this  on
it  probably  is  yeah  yeah  that's  another
that's  another  argument  that's  another
possible  argument  but  I  mean  for  me  this
is  a  little
more  a  little  more  more
straightforward
yeah  yeah  because  the  flat  I  I  believe  I
don't  remember  the  flat  thing  just
uses
coherence  ahh  okay  so  the  flat  the
product  of  flat  is  flat  is  coherent
dring  but  now  you  use  ideal  is  is  a
finite  Ty  to  actually  see  that  when  you
quotient  out  you  get  a  product  of  copies
of  R  otherwise  it  would  you  would  get  a
limit  of  product
of  of  Co  liit  of  product  of
finite  fin  type  so
and  and  again  it  is  condens  set  by
condens  set  it's  the  same  it's  it's
verified  by  Elementary  way  yeah  I  mean
for  me  this  is  pretty  Elementary  but
okay  I  I  don't  know  no  because  in  theory
to  compute  this
tensel  L  you  have  to  resolve  on  the  left
I  mean  t  is  different  meaning  I  mean
when  you  B  change  from  of  ring  to
another  ring  you  are  supposed  to  resolve
the
original  not  okay  you  can  also  compute
it  by
resolving  the
other  I  don't  need  to  I  mean  I  I  really
this  argument  just  works  in  the  D
category  I  don't  need  to  I  don't  need  to
worry  about  resolving  this  it  just  I  the
the  argument  is  I  I  claim  there  are  no
gaps  in  this
argument  no  because  when  you  reserve  a
ZN  you  are  living  the  categories  of  R
and  you  have  a  d  category  of  some  of  no
no  no  no  no  then  but  the  map  is  Def  find
already  so  the  yeah  the  to  cheog  you  can
pass  to  the  D  categor  if  no  I  have  a
resolution  in  this  algebraic  category
and  then  I  just  I  have  this  tensor
funter  taking  me  to  this  DED  category
and  that  tensor  funter  preserves  co-
limits  and  I  get  a  resolution  there  it's
it's  really  um  it's  really  fine  um  okay
so  yeah  this  is  this  is  quite  technical
I'm  afraid  um  but  uh  so  we've  proved  now
this  uh  this  claim  that  this  drive
solidification  lives  in  degrees  zero  and
that  was  the  first  thing  we  needed  to
show  but  the  second  thing  we  needed  to
show  is  that  it's  flat  uh  with  respect
to  the  tensor
product  um  so  for  this  we  want  uh  we're
going  to  analyze  this  category  of  solid
R  modules  in  a  little  more  detail  so
this
structure
of  solid
R
um  so  again  so  let  me  say  maybe  theorem
so  again  R  finite  type  over
z  uh  then  well  something  we  already  know
um  from  this  being  a  compact  projective
generator  is  again  that  solid  R  uh  is
just  the  in  category  of  its  compact
objects  which  are  the  finitely  presented
ones
um  so  these  are  the  things  that  are  co-
kernels  of  maps  between  the  compact
projective
generators
um  but  this  CLA  second  claim  is  uh  so
you  could  say
coherence  uh  this  category  this
subcategory  uh  is  an  ailan  category  is
is  closed
under  uh  under  kernels  co-  kernels  and
extensions
this  is  same  as  this  Corine  uh  which
Coraline  yeah  the  first  part  is  the
first  part  is  obvious  or  sorry  well  the
first  part  follows  from  the  fact  that  we
have  a  compact  projective  generator  of
this  category  then  uh  the  compact
objects  will  all  be  built  from  finite
co-  limits  from  that  and  they  will  be
generators  and  it  follows  formally  that
it's  the  end  category  of  of
that
um
so  what  we  really  need  to  check  is  that
we  have  this  this  is  an  a  bilon
subcategory  now  let  me  remind  you  a  bit
about  some  classical  commutative  algebra
so  a  commutative  classical  commutative
ring  is  called  coherent  if  you  have  the
analogous  property  in  the  setting  of
discrete  modules  so  that  the  finitely
presented  R  modules  form  in  a  billion
subcategory  but  you  can  check  by  some  by
playing  around  with  short  exact
sequences  that  you  only  need  to  there's
only  really  one  thing  to  check  uh  which
is  that  every  uh  finitely  generated
ideal  uh  is  actually  finitely  presented
and  those  same  arguments  in  this  setting
uh  show  that  it  suffices  to
check  so
every  finitely  generated
subobject  of  product  of  copies  of  R  is
finally
presented  so
so  CF  coherent
Rings  now  I'm  going  to  use  a  bit  this
notion  quasi  separated  so  note  that  uh
if  you  have  any  subobject  of  product  of
copies  of  R  well  this  is  quasy  separated
I.E  house  dwarf  and  any  subobject  of
anything  quasy  separated  is  also  quasy
separated  so  this  is  also  quasy
separated  um  so  it  suffices  to
show  for
this  it  suffices  to  show  that  every
finitely  generated  quasi  separated  uh
solid  R  module  is  actually  finitely
presented  so  let  me  make  a  Lemma
so  so  if  m  in  solid
R  is  quasi
separated  uh  then  the  following  are
equivalent  um  one  is  that  m  is  finitely
presented  two  is  that  m  is  finitely
generated  and  three  is  that  well  you  can
say  exactly  what  M  looks  like  uh  so  M  uh
is  an  inverse  limit  of  mn's  or  each  MN
is  a  finitely  generated  R  module
discrete  R  module  um  and  the  transition
Maps  forus
projective
and  the  and  the  category  of
these  uh  is  just  the  countable  Pro
category  of  the  finite  of  the  category
of  finitely  generated  R
modules  so  the  maps  between  such  inverse
limits  are  just  the  the  maps  of  pro
objects  Pro  objects  with  subject  of
transition  yes  thank  you  thank  you  yes
yes
yes
yeah
yeah  okay
so
so  the  non-trivial  implications  are  two
implies  3  and  3  implies
1  um  so  for  two  implies
3  uh
so  well  we  finally  generated  by
definition  means  we  have  Sur  some
surjection
from  product  of  copies  of
R  and  then  there's  some  kernel
um  but  since  m  is  quasi
separated  uh  it  follows  that  this
inclusion  is  a  quasi  compact
map  of  of  of  condensed
sets  which  means  so  uh  this  object  is
quasi  separated  so  it's  just  some
filtered  Union  of  compact  house  doorf
spaces  and  this  means  that  when  you
restrict  this  inclusion  to  any  of  those
compact  house  door  spaces  you  get  a
closed
subset  but  um  this  is  one  of  those
sequential  spaces  metrizable  spaces  in
fact  um  and  so  being  a  closed  subset  on
any  compact  subset  is  the  same  thing  as
just  being  a  closed  subset  so  this  is
the  same  thing  as  just  a  closed  subset
in  the  topological
sense
uh  with  the  product
topology  CL  Subs  which  is  also  in  our
module  this  is  the  same  associate
condensed  module  is  the  one  that  we  are
talking  about  right  uh  so  let  me  so  then
um  but  now  we  can  look  at  for  any  n  we
can  look  at  the  projection  so  we  have  K
uh  subset  product  R  we  can  look  at  the
projection  onto  like  the  first  n
coordinates
uh  and  then  we  can  let  knen  be  the  image
in  here
uh  then  it  follows  that  uh  K  is  just  the
uh  the  inverse  limit  of  K
NS  uh  and  these  have  subjective
transition  Maps  I.E  so  K  is  as  in
three  no  you  know  that  m  is  as  in  three
but  this  also
uh  right  I  wanted  to  know  that  m  is  as
in  three  but  this  also  follows  yes
um  uh  and  that  yeah  then  yeah  and
M  analogously  be  uh  inverse  limit  Over  N
of  MN  uh  where  MN  is  the
product
uh
okay  and  note  that  this  argument  shows
that
um  if  you  so  in  particular  if  you  start
with  an  object  as  in  three  uh  yeah  okay
so  now  let  yeah  so  now  um  we  also
see  uh  to  finish  I.E  to  show  that  3
implies  1  it  actually  suffices  to
show  uh  3  implies
2  because  if  we  know  that  this  guy  is
fin  generated  then  it  fits  into
something  like  this  and  we  just  prove
that  the  kernel  is  of  the  same  form  and
then  it  will  follow  that  it's  fin
presented
um  um  so  but  then  this  is  actually
Elementary  so  you  have  like  M1
subjecting  onto  M2  subjecting  onto
M3  you  can  make  some  uh  finite  free
module  rejecting  onto
M1  um  and
then  you  have  some  here  uh  kernel  of
this  map  here  uh  and  then  you  can  make
some  R  direct  some  D2  rejecting  onto
that  uh  and  then  uh  you  can  make  a
system  R  direct  some  D1  R  direct  some  D1
plus
D2
uh
uh  um  and  in  the  inverse  limit  you  get  a
product  of  copies  of  R  rejecting  on  to
m
so  in  fact  uh  if  you're  wondering  about
surjection  on  the  level  of  condensed
objects  in  fact  you  can  even  get  a
topological  splitting  for  the  map  of
topological  spaces  so  by  kind  of  section
here
make  a  compatible  section  there  and  so
on  if  you're  not  worried  about  our
linearity  which  you're  not  then  uh  you
can  get  this  quite
easily
um  okay  so  that's  the  proof  of  the
Lemma  so  the  quasy  separated  objects  are
just  these  kind  of
classical  these  classical  R
modules  um  and  that
also  uh  proves
the  um  it  proves  the  uh  the  coherence
here  in  extensions  is  easy  extensions  is
formal  yeah  yeah  it  doesn't  require
anything
uh  because  it's  projective  projective
generator  so
it's  yeah  you  can
lift
exactly
I  mean  actually  uh  because  you  have  this
compact  projective  generator  you're
equivalent  to  the  category  of  modules
over  a  ring  and  so  it's  really  it's  you
can  actually  just  quote  the  usual
theorem  from  commutative
algebra  um  okay  yes  all  right  like  so
here  um  so  if  M  was  finally  generated  in
particular  the  kernel  also  had  this
really  nice  property  of  being  of  having
property  three  as  well  uh  if  M  was  quasi
separated  and  finitely  generated  yeah
and  so  that  in  particular  implies  that
if  m  is  finally  you  can  find  a
resolution  of  by  really  nice  like  yeah
that's  right  that's
right  you  said  you  said  something  about
Comm  in  fact  the  endomorphisms  of  the
projective  object  is  an  associative  yes
oh  yeah  sorry  yeah  not  not  commutative
algebra  sorry  thank  you  thank  you  it's
not  communative  algebra  it's  it's  module
Theory  over  non-commutative  ring  yeah
thank  you
yes
wait  we're  still  proving  the  theorem  uh
we're  done  with  that  theorem  where  did
we  use  the  fact  that  was  finite  type
over  Z
sorry  ah  well  it  it  was  used  in  order  to
say  that  this  was  the  compact  projective
generator  so  when  we  were  so  a  prior  you
take  product  Z  tensor  up  to  R  and  then
solidify  with  respect  to  R  and  only  in
the  finite  type  over  Z  case  when  you  had
this  rejection  from  The  polom  Ring  yeah
but  for  any  narian  ring  you  can  take  the
product  of  copies  of  yeah  you  if  if  you
just  say  that  you  build  a  category  that
has  this  as  compact  projective  generator
with  the  endomorphisms  what  they  have  to
be  then  the  rest  of  the  arguments  here
work  in  the  for  an  arbitrary  etherion
ring
yeah
okay  um  right  and  we're  not  quite  done
with  the  uh  the  first  theorem  because  we
still  need  to  show  that  uh  product  of
copies  of  R  is  flat  so  now  but  so  now  we
make  a  claim  that  if  uh
m  is  finitely
presented
uh  in  solid
R  uh  then  if  you  take  M  derive  tensor
product  R  solid  with  product  of  copies
of  R  you  just  get  a  product  of  copies  of
M  and  in  particular  this  lives  in  degree
zero
no  um  and  that  implies  uh  that  this
theorem  or  this  claim  implies  that  this
guy  is
flat  uh  because  the  derived  tensor
product  with  any  finally  presented
object  lives  in  degree  Z  but  every
object  is  a  filtered  cimit  of  finitely
presented  objects  so  the  drive  tensor
product  with  any  object  will  live  in
degree  Z  which  is  the  same  thing  as
saying  you  have
flatness  so  incidentally  to  define  the
the  right  pens  of  maybe  we  didn't  spend
time  on  this  but  in  the  usual  approach
you  need  to  know  there  are  enough
objects  to  relative  to  T  to  define
classically
t  so  here  what  is  the  approach  what's
the  meaning
assigned  d  t  do  you  prove  that  there
enough  L  it's  not  really  necessary  so
first  on  the  level  of  the  derived
category  of  light  condensed  ailan
groups  um  I  mean  you  can  take  this
perspective  of  sheaves  with  values  in
the  OR  hypers  sheaves  with  values  in  the
derived  Infinity  category  and  then  you
can  just  sheify  the  sectionwise  derived
tensor  product  of  ailan  group  derived
sheify  the  sectionwise  tensor  product  of
derived  ailan  groups  and  that's  a  fine
definition  of  the  derve  tensor  product
on  that  basic  thing  and  then  it  just
gets  ported  to  all  of  the  other  contexts
uh  you  know  you  have  a  ring  object  in
there  you
have  uh  then  you  just  do  a  relative
tensor  product  and  then  uh  you  have  some
localization  but  we  already  argued  how
to  get  the  symmetrical  structure  on  the
localization  of  given  given  by  an
analytic  ring  structure  and  so  on  and  so
forth  so  it's  not  actually  to  make  the
definition  it's  not  actually  necessary
to  know  anything  about  flat
objects  so  for  for  condensed  a
billion  ah  okay  so  let  us  say  without
Sol  you  just  have  the  T  of  okay  there  we
know  we  know  initially  there  enough  FL
in  indeed  in  fact  in  in  light
condensability  groups  we  do  know  there
are  enough  flat  objects  so  that's  I
didn't  need  that  Infinity  stuff  really
to  you're  right  yeah  yeah  but  I  mean
yeah  I  have  to  say  I  was  kind  of  off  the
top  of  my  head  just  saying  how  I  think
about  it  yeah  these  free  object  free
object  on  a  light  condensed  set  will
itself  be  flat
and  um  yeah  okay
uh  okay  so  where  was  I  ah  so  we  have  we
have  to  Pro  to  finish  the  uh  discussion
of  so  far  we  just  need  to  prove  this
claim
um  so
proof  um  so  we  are  finally  presented  so
we  can
make  uh  a  surjection  here  where  the
kernel  is  finitely
generated  but  the  kernel  is  also  uh
quasy
separated  um  so  you  know  it'll  also  be
finitely  presented  well  we  can  then
it'll  we  can  basically  just  continue  to
make  an  infinite  resolution
so  uh  and  that  reduces
us  to  the  case  uh  where  M  itself  is
equal  to  this  product  of  copies  of
R
um  and  then  uh  and  then  we  can  remember
that  this  was  product  of  copies  of  Z
tensor  R  and  then  R  solidified  and  then
that  uh  will  reduce  us  to  the  analogous
fact  for  products  of  copies  of  Z  that  it
distributes  over  uh  infinite
products
so  for  prod  infinite  product  of  Z  for
Z  ah  then  you  reduce  to  to  Li  condens
abil  groups  is  infinite  produ  of  Z
and  yeah  there  are  even  the  thing  so
then  you  have  this  nice  object  that
solidifies  to  a  product  of  copies  of  Z
this  uh  this  thing  we  were  calling  P
basically  n  Union  infinity  and  that
light  condensed  set  has  the  property
that  if  you  take  its  product  with  itself
it's  it's  isomorphic  to  itself  in  the  in
the  expected  way  um  so  or  like  that  that
object  for  n  times  itself  is  the  same  as
that  object  for  n  cross  n  and  yeah  you
just  well  anyway  it  was
explained
um
okay  um  any  other
questions
because  this  is
projective  uh  this  is  the  prod  copies  of
the
projec
generate  okay  but  if  you  did  not  define
t  l  is  a  classical  Left  Right  F  you
cannot  use  the  fact  that  for  projective
object  I  mean  this  requires  a  classical
treatment  yes  but  right  but
so  the  actually  theun  of  the  T  right
that  that  I  could  have  added  to  the  um  I
could  have  added  that  to  uh  yeah  that's
right  that's  another  important  claim  in
addition  to  the  fact  that  the  derived
category  of  the  ailan  category  is  the
derived  category  that  we  defined  you
also  have  that  the  derived  solidifi  or
derived  solidification  is  the  left
derived  functor  of  solidification  you
have  that  the  deriv  tensor  product  is
the  left  dve  functor  of  the  tensor
product  and  you  can  resolve  an  either
variable  this  all  follows  once  you  know
all  of  this  this  flatness  and  living  in
degree  zero
stuff  so  again  to  so  the  foundational
way  you  define  tensor  L  for  your  deriv
category  of  solid  is  what  I  mean  without
using  flat  objects  yeah  so  the  way  you
do  it  is  that  you  of  course  it  lives  in
some  dve  category  without  without  solid
so  you  can  just  take  tensor  there  and
solidify  yes  exactly  of  course  you  have
to  know  that  that  somehow  then  part  of
this  story  is  that  solidification  commut
I  mean  if  two  things  have  the  same  after
I  mean  you  have  to  know  but  then  to
define  derived  without  solidification
then  then  it  is  just  derived  over  this
is  just  modules  over  a  ring  supp  you
already  know  over  Z  of  the  find  yeah
then  over  a  ring  you  use  flat  object  or
what  do  you  use  you  you  never  have  to
use  flat  objects  to  make  the  definitions
okay  so  how  do  you  define  the  the
derived  tenso  product  for  without  the
solid  yeah  yeah  you  have  a  you  you  have
a  commutative  ring  in  solid  Z  and  you
want  to  do  dve  tensor  product  I  mean  it
explicitly  it's  you  just  you  know  it's
some  geometric  realization  of  a  bar
construction  where  all  the  tensor
products  over  solid  Z  are  derived  I  mean
so  Lori  sets  it  up  if  you  have  a
commutative  algebra  object  in  a
symmetric  monoidal  Infinity  category
then  you  get  a  tensor  structure  on  our
modules  in  there
um  okay  so  you  can  have  in  this  General
setup  you  have  some  way
to  yeah  I  mean  it's  the  it's  basically
like  you  take  M  tensor  Ln  M  tensor  LR
tensor  Ln  You  Know  M  tensor  LR  tensor
squar
okay  and  then  you  have  some  simplicial
object  and  you  take  the  co  limit  all  in
this  Infinity
sense  okay  so  you  have  this  General  way
of  viewing  the  tens  of  products  and  then
it  always  in  all  cases  there  enough  flat
objects  and  you  can  use  flat  or  okay  in
in  the  in  in  what  we've  discussed  so  far
the  solid  theory  yes  that's  what  we've
just  proved  although  as  you  see  it's  not
a  formal  argument  by  any  means  you
really  have  to  analyze  uh  what's  going
on
other
questions
okay  um
so  oops  so  I  want  to  spend  just  a  little
more  time  on  the  homological  algebra  of
solid  R
um  prove  because  Peter  also  explained
something  else  in  the  case  of  solid  Z
basically  that  every  finitely  presented
module  it  doesn't  just  have  an  infinite
resolution  by  products  of  copies  of  Z
but  it  actually  has  a  two-term
resolution  by  products  of  copies  of  Z  so
that  kind  of  at  least  from  the
perspective  of  finitely  presented
objects  you  have  sort  of  homological
Dimension  One  for  solid  Z  just  like  you
have  for  usual  Z  so  um  I  want  to  present
a  generalization  of
that  um
so
so  uh  theorem  so  if  R  is  finite
type  over  Z  and  is
regular  a  regular  ring  of  Dimension
d  uh  then  if  you  have  a  m  in  solid  are
finitely  presented  and  you  have  n  and
solid  are  arbitrary
say  uh  then  X  I  even  internal  X  why  not
uh
uh  the  XS  vanish  uh  in  degrees  bigger
than  the
dimension  and  the  to
also  right  so  a  corollary  of  this  will
be  that  if  you  have  a  finitely
presented  uh  module  then  you  actually
get  a  projective  resolution  of  length  uh
in  most  well  depending  on  how  you  define
length  D  or  D+  one  yeah  maybe  yeah
um  and  then  we  already  saw  that  the
projective  objects  are  flat  so  this  also
implies  a  bound  on  the  tours  so  so  a
corollary
is  so  if  if  m  is  and  that  and  that
passes  from  fin  presented  to
arbitrary
uh
let  me
say
hi
yeah
okay
um  let  me  caution
uh  uh  this  this  this  this  statement  does
not
extend  to  non-finitely
presented  M  so  unlike  unlike  in  the  case
of  discrete  modules  over  a  nean  ring
regular  nean  ring  of  finite  Dimension
but  there  you  actually  have  the  X
Vanishing  uh  for  all  modules  um  there's
some  extra  argument  for  going  from  the
finitely  presented  case  to  the  general
case  that  argument  does  not  work  in  this
context  and  in  fact  um  I  think  you  can
have  arbitrarily  High  AXS  even  for  solid
Z  so  let  me  give  a  here's  a  fun  a  fun
exercise  you  can  try  to  do
so  you  take  two  distinct  prime  numbers
uh  this  is  just  just  cute  so  here's
something  that's  not  finitely
presented  uh  because  you're  quotienting
by  something  which  is  z  Z1  over
L  um  then  if  you  do  there's  an  X2  which
is  nonzero  and  it's  given  by  switching
the  roles  of  P  and
L  uh  so  that's  just  if  you  want  to  have
a  little  fun  with  some
calculations  and  the  n  x  to  and  you
claim  also  there  arbitrary  highx  I
believe  that  so  certainly  there  so
certainly  uh  there  exists  a  model  of  zfc
in  which  there  are  arbitrarily  highx  so
there's  logicians  analyzed  a  particular
uh  X  Group  which  I  think  Peter  also
mentioned  in  one  of  his  lectures  but  I
think  even  in  zfc  it  should  be  possible
to  produce  a  arbitrarily  High
non-vanishing  X  I  mean  even  under  in  any
model  of
zfc  I  mean  you  run  to  a  priori  if  you
want  to  go  from  the  finitely  presented
case  to  the  general  case  you  have  to  an
analyze  derived  inverse  limits  along
some  arbitrary  filtered  system  but  do
you  have  a  boundary
cardinality  because  if  you  are  in  a
model  with  the  Contin  hypothesis  for  FS
yeah  and  if  the  you  look  at  or  maybe
generalize  I  don't  know  if  you  look  at
all  possible  finite  types  of  modules
yeah  so  they  M  it's  related  to  maths
from  product  of  Z  to  product  of  Z
yeah  then  cality  of  this  space  is
like  one  of  those  some  finite
aln  then  we  have  chological  Dimension
results  for  this  yes  yes  yes  yes  yes  so
it  suggests  that  that  there  the  would  be
bound
for  sub  for  quotients  of  s  by  by  of  a
project  something  but  then  there  is  a
classical  argument  in  ring  which
probably  extends  to  this  case
that  that  if  you  know  the  X  Vanishing
for  this  kind  ring  mod  IDE  and  you  have
it  for  arbitrary  yeah  but  so  but  you
need  to  but  that  the  analog  of  that
argument  shows  you  need  to  analyze
something  like  an  arbitrary  subm  module
of  product  of  copies  of  Z  not
necessarily  finally  generated  and  yeah
and  yeah  and  then  yeah  but  I  guess  yeah
there's  a  bound  on  the  cardinality
there  yeah  so
yeah  you're  right  that  the  cardinality
of  what's  that
Peter  it  could  be  that
something
have  maybe  yeah  it's  possible  yeah
um
but  if  the  cardinality  of  the  Continuum
is  larger  than  alfn  for  any  n  then  then
you  should  probably  get  um  okay  uh
arbitrary  higher  X  anyway  let's  let's  uh
let's  not  go  there
um  right  uh  where  are  we  ah  so  let's
let's  try  to  prove  this
so
um  so  the  first  claim  so  first  we're
going
to  so
first
case  uh  we're  going  to  take  n  so  ah  so
we  well
so  so  by  the  Sorry  by  the  same  argument
um  sorry  just  a  sec  let  maybe  I'll  do
that  at  the  end  so  the  first
case  let's  say  that  n  is  actually
a  know  a  classically  finitely  generated
module
over  so  the
uh  nean  ring
R
um  and  then  let's  take  M  to  be  quasy
separated  a  finally  presented  solid
R  so  then  that's  an  inverse  limit  of
mn's  uh  over  all
n  uh  with  surjective  transition
maps  and  in  this  case  the  claim  is
um  that  X
I  uh  from  M  to  n  uh  or  I  guess  it
doesn't  even  need  to  be  finely  generate
let's  just  say  discreet  uh  is  the  same
thing  as  the  filtered  Co  limit  of
xti  over  R  so  this  is  R  solid  um  xti
over  R  from  MN  to
n
didn't  you  Pro  that  par  present  impli  qu
separation  uh
no  no  certainly  not  uh  I  mean  in  solid
the  solid  Z  case  something  like  ptic
integers  mod  integers  will  be  fin
presented  but  not  quasi
separated  so  what  what  did  be  Pro  proved
that  if  you're  quasi  separated  and
finally  generated  then  you're  finally
presented
uh  and  if  you're  fin  presented  are  you
ah  you  are  not  necessarily  qu  SE
exactly  exactly  yeah  so  for  example  zp
modulo
z
uh  is  finally
presented  so  if
a  map  from  product  of  copies  the  product
of  copies  of  R  is  the  coal  is  not  clly
separate  not  necessarily  that's  right
okay  I  misunderstood  this  uhuh  yeah
that's  important  um
yeah
um  so  we  need  to  show  that  so  we  need
that  the
arom  uh  you  can  pull  out  the  inverse
limit  uh  basically  uh  by  the  way  the
claim  for  internal  follows  from  the
claim  with  underlying  it's  just
replacing  n  by  like  continuous  functions
with  values  in  N  for
some  uh  for  some  profinite  set  s  um  but
so  uh  we  have  the  mitag  Leer  so  we  have
the
resolution
uh  some  kind  of  usual  identity  minus
shift  sort  of  thing  or  maybe
it's  um  or  shift  times  F  or  some  some
kind  of  shift  map  um  so  this
distinguished  triangle  here  so  that
tells  you  that  the  question  of  pulling
rhom  the  inverse  limit  out  of  the  ROM  is
the  same  as  the  question  of  uh  pulling
out  a  product  and  turning  it  into  a
direct  sum  um  and  then  we  can
resolve  product  MN  by  product
R  to
reduce  to
ROM  product  r  n  equals  direct  sum
R
RN  uh  which  follows  from  this
being  uh  projective  and  yeah  and  the
basic  calculation  on
home
okay  so  from  this  it  follows  that  we  get
the  desired  bound  uh  on  X  if  n
is  uh  a  classical  discret  R  module  and  M
is  finally  presented  and  quasy
separated
now  you  can  always  resol  so  now  M
arbitrary  finally  presented  not
necessarily
quasi  separated  then  you  can  always
resolve  oops
M  by  product  of  copies  of  R  and  then  the
kernel  will  be  both  both  uh  finitely
presented  and  quasy
separated  and  uh  then  you  analyzing  the
long  exact  sequence  in  X  mapping  out  to
n  you  would  get  the  desired  result  but
with  uh  you'd  lose  one  degree  of
chological  Dimension  you'd  get  that  x  i
vanishes  for  I  greater  than  D+
one  um  so  the  xti  vanishing  here  for  I
greater  than  D  when  you  Chase  through
the  long  exact  sequence  would  only  give
the  vanishing  in  degrees  greater  than  D+
one  there
so  uh  so  this  naive
argument  only
gives  uh  x  i  m  n  =  0  for  I  greater  than
D  +
1  so  we  have  to  do  a  little  bit  of  extra
work  to  improve  this  for  that  conclusion
there  we  didn't  use
regularity  uh  well  we  did  because  I
should  maybe  recall  the  classical  fact
so
weall  so  if  R  is  regular
mean  Dimension  D  then  the  classical  X
groups  you  know  between  discrete
modules  so  this  so  that  the  the  question
now  for  quasy  separated  things  in  the
solid  context
reduces  to  the  classical  question  in
discret  commutative  algebra  and  that's
where  we  use  regularity
yeah  um  so  we're  going  to  do  a  less
naive  argument  so  we're  going  to
continue  the  resolution  one  more  step
so  so
instead  so  we  have  a  product  of  copies
of  R  we  can  Sur  again  from  a  product  of
copies  of  R  and  now  at  this  point  we
take  a
kernel
um  and  we  get  a  resolution  like
this  um  this  is  still  quasy  separated
and  then  it  suffices  to
show  uh  to  so  to
prove  uh  x  i  m  ah  different  n  I'm  so
sorry
um  ah  so  let  me  let  me  say  x
uh  for  I  I  greater  than  D  it  suffices  to
show
uh  for  I  greater  than  D  minus  2  so  we
bought  ourselves  one  extra
uh
uh  drop  um  because  we  continued  the
resolution  One  Step
more  um  sayal  Z  and  one  have  to  be  head
that's  true  yeah  so  this  will  I'll  be
assuming  so  assuming  D  greater  than  or
equal  to  two  the  case  D  equals  1  well
you  can  use  similar  arguments  in  low
Dimension  yeah  thanks
Peter  one  the  same  as  for  integers  I'm
not  entirely  sure  but  a  modification  of
this  argument  for  D  greater  than  or
equal  to  two  will  will  also  work  for  D
equals
1  um  right  so  it  suffices  to  prove  this
x  Vanishing  so  now  we  want  to
um  uh  now  we  want  to  find  a  nice
expression  of  n  as  an  inverse  limit  of
finitely  generated  R  modules  so  there's
actually  two  different  two
ways  of
writing
uh  n  as  an  inverse
limit  of  a  finitely  generated  R
modules  so  the  first  way  is  just  the
what  we  have  the  argument  we  already  had
where  this  is  um  this  is  the  kernel  of  a
map  between  quasi  separated  things  so
it's  a  closed  subm  module  here  um  so  we
can  always  just  look  at
the  um  well  maybe  I'll  make  that  the
second  method  meod  sorry  the  first
method  would  be  uh  we  analyze  this  map
here  um  this  is  a  map  from  product  of
copies  of  R  to  product  of  copies  of  R
again  that's  the  same  thing  as  a  map  on
the  Pro  system  so  if  you  uh  if  you
restrict  to  any  initial  chunk  of  this
then  there's  some  corresponding  initial
chunk  here  where  the  the  map  projecting
onto  that  factors  through  projecting
onto  this  chunk  and  then  just  a  map  of
discrete  modules  there  so  if  you  reindex
and  allow
like  uh  finite  free  modules  here  instead
so  uh  then  you  can  you  can  assume  that
uh  this  map  here  comes  from  an  inverse
limit  of  compatible  maps  from  um  you
know  the  initial  chunk  the  nth  initial
chunk  here  to  the  nth  initial  chunk  here
just  by  just  by
reindexing  um  and  then  if  you  take  the
kernels  of  those  Maps  so  you  take  uh
so  well  let  me  say  I  don't  really  need
it  to  be  a  product  anymore
so  so  we  take  an  inverse  limit  of  this
type  and  we  say  that  this  is  given
termwise
um  uh  so  then  we  can  take  so  we  can  let
uh  so
let  uh
so  then  we  have  n  equals  inverse  limit
Over  N
NN  but  the  other  thing  we  can  do  is  we
can  let  uh  n  Prime  n  which  will  be
contained  in  NN  uh  contained  in  FN  be
the
image  of  n  mapping  to  inverse  limit  of
FN  mapping  to  FN  or  the  particular  FN
inverse  limit  FM  mapping  to  the
particular  FN
um
so  uh  number  one  has  uh  so  number  so
each  of  them  each  of  these  presentations
of  n  has  something  good  and  something
bad  about  it  so  uh  good  news  for
one  is  that  you  have  this  uh  XD  minus  2
Vanishing  x  i  x  i  NN  x  equals  zero  for
all  I  bigger  than  exactly  what  you  want
D  minus
2  um  but  the  bad
news  no
guarantee  that  this  system  NN  is  MOG
leer  uh  which  is  needed  for
uh  X  to
I  uh  NX  equals  filtered  Co  limit  Over  N
of
xti  and
NX  so  that's  uh  that's  unfortunate  so  in
the  AR  the  argument  I  presented  here  in
the  case  where  the  transition  I  said  the
transition  Maps  were  subjective  but  the
argument  really  only  used  that  it  was  a
MOG  Lefler  system  um  to  get  this
resolution  here  um  and  then  in  two  prime
the  situ  in  two  the  situation  is
opposite  uh  MOG  leer  in  fact  the
transition  maps  are  rejective  by
construction  but  no
guarantee  that
X  um  D  minus  one  uh  n  Prime  n  x  equals
z  so  n  Prime  n  is  this  condensed  module
of
no  no  it's  just  discreet  so  any  any
subobject  of  a  discret  object  is  itself
discreet  all
right  so  we  know  that  this  is
a  but  we  know  that  n  is  the  limit  of
both  in  the  condens  it's  limiting  the
condenses  of  both  of  these  systems  yes
okay  but  and  certainly  of  you  can
produce  an  example  of  map  from  limn  to
limn  Prime  for  which  the  the  the  kernels
would  not  be  Meed  left  yes  exactly  yes
you  can  yeah  URS  in  nature  yeah
yes
um  right  so  what  we  have  to  do  is  find
something  in  between  that  has  the  good
properties  of  both  so  I  apologize  this
argument  is  a  bit  technical
by  the  way
um  the  reason  we  have  this  nice  property
here  uh  is  that  it's  a  kernel  of  a  map
between  uh  projective  objects  uh  so  it's
it's  like  the  it's  the  you  know  so  if
you  took  the  co-  kernel  then  it  would  be
two  steps  away  from  that  Co  kernel  so
the  vanishing  in  degrees  greater  than  D
for  the  co-  kernel  will  imply  The
Vanishing  in  degrees  greater  than  D
minus  2  for  that  long  kernel  over  there
that's  why  you  have  this  and
um  so  and  but  for  this  NN  Prime  you  you
don't  know  that  it's  a  kernel  of  a  map
between  projective  things  you  only  know
that  it's  sitting  inside  a  projective
thing  so  You'  get  Vanishing  of  XD  one
better  than  for  a  general  module  but  you
wouldn't  get  Vanishing  a  priori  in
degree  D  minus  one  so  that's  what  we
need  to  fix
um  so  what  is  the  difference  so  find  so
we're  going  to  find  uh  n  Prime  n  sitting
in  between
them
uh  and  such  that  we  get  X
Vanishing
uh
uh  uh  and  uh  NP  Prime  n  is
mler
so  if  we  get  this  then  we're
done  uh  because  the  inverse  limit  uh  we
would  get  the  desired  X  bounds  for  the
inverse  limit  of  the  NN  primes  but
that's  sandwiched  in  between  these  two
things  which  have  the  same  inverse  limit
which  is  the  thing  we're  interested  in
so  the  vanishing  of  X  for  this  inverse
limit  would  imply  it  for  uh  our  our
desired  Vanishing  for
n
okay
so  now  we  use  a  bit  of  commutative
algebra  so  so  what  is  the  obstruction  we
have  a  module  where  we  know  the
vanishing  in  degree  D  but  we  don't  know
the  vanishing  in  degree  D  minus  one  uh
the
obstruction  uh
having
uh
uh  this
is  uh
that  so  that  the  obstruction  is  in  some
sense  concentrated  at  the  closed  point
so  it's  a  question  of  depth  so  the
Outlander  bbam  formula  tells  you  that  uh
at  a  for  a  regular  nean  local  ring  the
projective  dimension  of  a  module  plus
the  depth  of  the  mod  module  is  equal  to
the  dimension  of  the  Ring  um  and  because
we  have  the  uh  we
have  uh  the  one  better  estimate  on  the
projective  Dimension  um  the  you  know  the
only  rings  that  are  going  to  give  us
obstruction  are  the  local  rings  at  at
Prime  ideals  which  are  actually  maximal
ideals  so  that  the  local  ring  has
maximal  Dimension  and  the  only
obstruction  is  going  to  be  moving  from
situation  of  depth  one  at  that  maximal
point  to  depth  two  so  so  is  the  fact
that
at  a  maximal
ideal  uh  so  n  and  Prime  n  only  known  to
have  depth
one
yeah  yeah
yeah  yeah  yeah  many  closed  points  and
you  know  that  it  lies  inside  the  other
one  because  the  other  one  is  as  this  yes
is  reflexive
yes  and  so  uh  and  then  you  have  to
establish  but  only  close  point  the
classical  L  argument  work  that's  that's
correct
yes
okay  all
right  maybe  I'll  just  repeat  what  oer
said
so  so  what  you  can  do  if  you  let's  let's
say  for  Simplicity  you  only  have  a
problem  at  one  maximal  ideal  in  general
you'd  fix  a  problem  at  finitely  many  and
then  if  you  increase  the  number  of  them
eventually  the  situation  would  stabilize
by  an  etherum  this  so  um  so  let's  say
there's  a  problem  only  at  one  maximal
ideal  then  you  can  look  at  the  inclusion
of  so  Spec  R  minus  that  closed  point  x
uh  closed
Point  into  Spec  R  and  then  you
replace
n  Prime  n  by  uh  J  lower  star  J  upper
star  of  n  Prime
n  and
uh  that  receives  a  map  from  n  Prime  n
but  it's  in  fact  an  inclusion  due  to  the
depth  one  assumption  uh  on  our
module  um  but  then  on  the  other  hand  by
this  procedure  of  this  extension  and
restriction  so  depth  is  you  can
characterize  depth  in  terms  of  local
chology  at  the  maximal  ideal  and  uh
that's  exactly  what  comes  up  when
analyzing  the  difference  between  this
and  say  the  drive  version  of  this  and
using  that  you  can  see  that  this  uh
improves  the  depth  so  this  is  depth
two  uh  at
X  um  on  the  other  hand  if  you  were  to
perform  the  same  construction  for  NN
since  we  already  know  this  has  depth  to
you  wouldn't  have  been  changing  it  so
you  really  are  producing  something  uh
sitting  in  between  which  fixes  the
problem  at  a  given  closed  point  then
there's  the  neoness  which  tells  you  well
it's  only  finitely  many  Clos  points  that
are  going  to  be  involved  so  you  can  by
the  same  procedure  you  can  fix  the
problem  all  the  problems  for  n  Prime  n
um  and  in  a  compatible  manner  in  the
tower  okay  so  that's  then  you've  got  it
to  be  depth  to  at  all  the  closed  points
which  gives  you  the  desired  X  Vanishing
for  these  uh  N  double  Prime
ends  uh  and  then  what  about  the  mid
leer
so
so  uh  mtag  ller
so  this  Tower  N  double  Prime  n  sits  in  a
short  exact  sequence  with  n  Prime
n  uh  and  the  quotient  n  Prime  n  mod  n
Prime
n  so  to  show  that  this  is  Midler  you
need  to  know  that  this  is  mogler  or  it
suffices  to  show  that  this  is  MOG  Lefler
and  that  is  MOG  Lefler  this  one  we  know
because  the  transition  maps  are
subjective  but  this  one  this  one  here  is
supported  at  but  it's  supported  at
finitely  many  closed
points  and  it's  a  finitely  generated
module  over  the  ring  and  that  implies
that  it's  finite  as  an  ailan
group  because  remember  our  ring  was
finite  type  over  Z  the  maximal  ideal  is
all  of  course  have  you  know  the  resid
field  is  a  finite  field  and  a
compactness  argument  shows  that  any  uh
inverse  system  of  finite  aan  groups
satisfies  the  mogler  property  so
compactness  gives  M
left
okay  it's  a  bit  of  a
yeah
yeah
um  okay  so  that  gives  the  X  Vanishing  so
so  the  to  sum
up  summary  of  what  we've  done  so  far
we've  seen  that  if  m  is  fin
presented  solid  R  and  if  x  I've  switched
to  x  uh  is  a  discrete  module  you  know
over  the  ring  then  we  get  the  desired  X
Vanishing
um  now  uh  so  now  suppose
x  if  x  is  quasi  separated  and  finally
presented  in  solid
R  you  write  X  as  an  inverse  limit  of
xn  uh  with  surjective  transition
Maps
um  and  then  the  R  homs  into  that  will
just  be  the  uh  inverse  limit  of  the
derived  inverse  limit  of  the  R  homs  into
each  of  those  terms
now  in  principle  you  might  get  one  worse
again  because  there  could  be  a  limb  one
in  the  last  inverse  system  but  because
that  last  uh  uh  in  the  that's  that's
talking  about  the  XDS  into  this  but
since  these  transition  maps  are
subjective  and  D  is  the  largest  degree
at  which  you  have
nonzero  X  Vanishing  you  actually  see
that  on  XDS  the  you  get  surjective  uh
Maps  so  that  there's  no  limb  one
potentially  giving  you  something  in
degree  D+  one  um  so  the  you  get  the
claim  for
D  um  and  then  uh  X  arbitrary  finitely
presented  then  you  resolve  it  by
uh  and  the  X  with  values  in  X  can  only
be  better  uh  then  the  X  with  values  in
these  two  so  you  get  that  you  get  that
situation  as  well  and  then  finally  for
an
arbitrary  so  for  X
arbitrary  uh  so  if  you  write  X  as  a
filtered  Co  limit  of  x  highs  or  these
are  fin  presented  then  uh  X  to  I  uh  from
any  finit  presented  M  to  X  is  the
filtered  Co  limit  of  x  i  oh  no  I
shouldn't  use  I
XJ  M  to  XI  by  pseudo
coherence  of  M  so  you  can  resolve  M  by
product  of  copies  of  R  and  then  this
follows  from  product  of  copies  of  R
being  compact  projective  what  about  X
line  it's  the  so  yeah  I  should  from  this
point  on  so  uh  this  claim  for  an
arbitrary  discret  module  implies  the
same  claim  for  underline  because  it's
the  same  as  uh  s  valued  points  of  this
thing  is  the  same  thing  as  X  I  from  M  to
continuous  functions  from  s  to  this
discrete  thing  which  is  just  another
example  of  a  discrete  thing  so  then  you
get  this  there  and  then  from  that  point
on  all  of  the  arguments  actually  work  at
the  uh  at  the  internal  a  level
okay  including  the  Lim  one  argument
including  the  limb  one  argument  yes
because  you  reduce  it  to  to  an
X  you  reduce  it  to  showing  that  a  limb
one  vanishes  in  condensed  to  billion
groups  and  the  terms  in  the  system  are
discrete  uh  and  the  system  is  MOG  Lefler
and  that  those  properties  are  preserved
by  and  if  you  take  continuous  functions
with  values  in  that  it's  still  a  system
of  discreet  things  and  the  transition
maps  are  still  mler  or  subjective  even
yeah  and  and  in  the
condensed  a  bilan  group  again  you  have
up  to  only  up  to  Li  one  yes  you  don't
have  more  because  you  can  compute  it
termwise
yes  because  because  products  are  because
countable  products  are
exact  that's  the
reason  ah  okay  the  the
countable  ah  okay  okay  and  you  can
compute  the  Lim  one  term  by  no  maybe
this  is
not  you  can  compute  Lim  one  on  each
object  not  necessarily  project  okay  you
don't  have
projective  Compu
one  it's  not  like  in  any  site  where  it's
it's  delicated  you  cannot
compute  but  here  you  can  compute  the  Lim
one  on  any  test  condensed  Set
uh  no  no  for  some  things  yeah  so
certainly  for  this  P  object  you  can
because  that's  projective
um  and  then  that's  enough  for  solid
because  we  know  that  the  solidification
of  P
generates  um  so  that's  that's  one
argument  you  could  give  I'm  sure  there
are  other  arguments  as  well  but  yeah  I
mean  but  you  know  the  limb  one  is  always
just  going  to  be  the  sheif  of  the  uh  you
know  the  condensed  limb  one  will  always
be  the  sheif  of  the  naive  sectionwise
limb  one  and  so  if  prove  the  vanishing
of  the  of  the  limb  one  section  wise
that's  enough  to
prove  yes  I  missed  where  coherence  comes
from  ah  right  so  that  comes  from  the  the
the  claim  that  the  finitely  presented
objects  form  an  ailan  category  which  has
as  a  corollary  that  for  any  finitely
presented  objects  you  can  build  an
infinite  resolution  where  all  of  the
terms  are  products  of  copies  of
R  you're
welcome  and  to  have  non  qu  separated
presented  object  the  ring  has  to  be  of
Dimension  at  least  two  or  non  quasy
separated  finitely  present  has  to  have
Dimension  at  least  one  at  least  one
yeah  so  if  you  have  a  if  you're  over  a
finite  field  then  the  every  every
finitely  presented  object  is  quasi
separated  but  once  you  move  to  say  the
integers  then  then  as  I  said  zp  mod  Z  is
an  example  of  something  that's  not  quasi
separated  but  is  finitely  presented
also  summary  the  step  of  reducing  to  uh
reduce
from  qu  separated  fin  presented  to  fin
to  discrete  one  uh  is  that  EAS  to  it's
not  L  one  here  like  which  Step  sorry
this  step  here  yeah  you  see  that  there's
no  limb  one  because  the  the  only  the
only  limb  one  that  can  contribute  to
degree  D  plus  one  is  the  limb  one  of  X
D's  and  because  because  on  the  level  of
these  guys  we  know  that  there's  nothing
there's  no  X  for  any  discrete  module  in
degree  D+  one  you  see  that  if  you  have  a
surjective  map  uh  with  then  uh  XDS  into
it  will  also  be  surjective  because  the
obstruction  is  an  XD  plus  one  of  the
kernel  which
vanishes  so  then  the  XDS  is  also  a  mular
system  yeah
okay  that  was  uh  that  took  longer  than  I
anticipated
um  I  had  a  couple  of  other  topics  I
wanted  to  discuss  but  at  this  point  I
probably  have  to
choose
for  so  maybe  I  will
um  U  maybe  I  just  want  to  make  a  point
um  and  illustrate
it
um  so  so
recall  so  if  R  is  now  a  discret
commutative
ring  so  we  we
showed  that
that  solid  uh  or  D  let's  say
RZ  uh  solid
localizes
over
uh
this  valuative  spectrum  of  of  the
Ring
um  this  localization  had  the  property
and  this  is  maybe  why  it's  um  convenient
to  anal  one  reason  why  it's  convenient
to  analyze  that  if  you  on  the  on  this
discret  level  if  you  take  the  unit
object  here  which  is  r  and  then  you
restrict  to  any  any  of  these  basic  quasi
compact  opens  then  you  still  just  get
another  discrete  ring  in  fact  if  it  was
the  rational  open  like  given  by  F1  up  to
FN  over  G  then  the  ring  was  just  R1  over
G  so  you  with  this  localization  you're
kind  of  staying  in  the  world  of  of  your
ambient  ring  being  discreet  but  this  is
by  no  means  uh  there  are  other
are
also  other
ways  to
localize  the  same  category  this  is  by  no
means  the  most  General  possible
localization  it's  the  one  we  discussed
because  it's  the  one  that's  most  closely
related  to  Huber's  Theory  and  our  goal
in  this  class  is  to  explain  well  the
basic  definitions  in  our  Theory  and
their  relations  to  more  classical
theories  of  analytic  geometry  but  I  just
want  to  point  out  very  briefly  that  um
there's  also  so  a  whole  different  avenue
you  can  go  uh
for
um  localizing  this  which  even  more
radically  I  guess  departs  from  Huber's
formalism  so  we  already  saw  a  small
departure  in  that  we  were  allowed  to  do
slightly  more  localizations  of  a  Huber
ring  than  before  because  of  this
difference  between  valuations  that  are
less  than  one  on  topologically  Neil
potent  elements  and  ones  which  are
valuations  which  are  continuous  in  Huber
sense  um  but  kind  of  even  more  drastic
things  are  possible
so
um  so  in
fact  so  if  uh  let's  say  C  subset  Spec
R  is  any  constructible
subset  so  that  that  means  an
intersection  of  a  quasi  compact  open
with  the  with  a  complement  of  a  quasi
compact  open  so  it's  some  locally  closed
subset  of  specr  which  is  sort  of
finitely  presented  and  of  those  uh
finite  Union  of  those  yes  yes  yes  yes  so
the  ba  the  basic  let  me  say  the  basic
objects  which  um  sort  of  form  a  basis
for  the  constructible  topology  would  be
something  like  you  take  D  of  G
uh  uh
intersect
um  the  complement  of  yeah  the  common
zero  set  of  finitely  many
functions
uh
uh  uh  then  you  can  define  an  item  potent
algebra  in  uh  in  d  r  z
solid  uh
namely  uh  to  this  on  the  level  of  this
basic
object
um  D  of  the  complement  of  the  zero  locus
of  G  and  then  intersect  the  zero  locus
of  F1  to  FN  uh  to  this  you  can  assign
the  thing  you  get  when  you  take  R  and
then  you  invert  G  and  then  you  uh
derived  complete  along  F1  up  to
FN  which  uh  this  final  object  only
depends  on  the  the  constructible
subset
um  and  you  know  by  the  way  if  R  is  an
etheron  then  this  derived  completion  is
just  the  usual  completion  but  I  want  to
emphasize  that  I'm  taking  this  derived
completion  this  is  a  discret  object  but
I'm  taking  this  dve  completion  uh  in  uh
solid  ailan  groups  or  deriv  category  of
solid  ailan  groups  so  I'm  so  to  speak
putting  the  inverse  limit  topology  on
this  uh  derived  inverse  limit  here
um  no  such  things  then  you  you  say  that
you  ah  you  view  your  constructive  subset
as  a  union  of  locally  closed  you  also
look  at  the  intersection  so  on  so  for
each  of  them  each  fin  intersection  you
do  this  and  then  you  take  the  what  okay
you  take  the  limit  of  this  type  thing
and  it  won't  be  concentrated  in  hom
they'll  have  some  chological  yeah  so  is
it  the  case  the  intersection  of  two
basic  thing  what  you  get  is  a  TENS  of
product  yes  okay  so  it's  it  is
okay
uh
intersection  oh  yes  yes  that's  right
that's  right  sorry
uh  right  so  maybe  I  shouldn't  actually
say  that  you  can  assign  item  poent
object  to  any  constructible  subset  maybe
I  should  just  say  that  you  can  uh  or  is
it  okay  or  no  well  let  me  just  say  that
you  can  assign  a  item  potent  object  to
any  basic  constructible  subset  so  you
want  to  take  the  kind  of  the  derived
inverse  limit  of  that  important  object
Associated  to  all  basic  thing  inside  the
given  thing  but  then  you  have  a  problem
to  prove  properties  of  this  I  don't  know
if  it  follows  that  it  is  an  algebra
orru  looking  those  subset  May
what  yeah  constructible  locally  closed
is  if  it's  construc  locally  closed  Som
take  R  GMA  of  this  so  it  will  could  have
some  chological  degree  depending  on  the
on  number  of  Alpha  and  C  to  cover  yes
yes  yes  yes  so  that's  but  thank  you
Peter  that's  that's  uh  that's  better
yeah  so  then  the  only  derived  Behavior
comes  from  uh  kind  of  uh  Union  of
principal  opens  thanks  yeah  that's  what
I  should  be  saying  um  uh  okay  so  now  why
are  these  item  potent
so  so  recall  that
um  uh  recall  from  one  of  Peter  Peter's
lecture  that  um  the  solid  tensor
product  theyif
me  uh  item  potent  means  that  yeah  thanks
so  let's  say  that  this  let's  say  that
this  is  a  item  potent  means  that  a
tensor  over  R  uh  solid  derived  a  is  the
same  thing  as
a  um  of  derived
complete  RZ  solid
yeah
um  is  derived
complete  so  to  check  the  item  potency
you're  asking  that  that  that  implies
that  uh  so  you're  asking  whether  a  map
between  derived  complete  objects  by  this
fact  is  an  equivalence  and  that  you  can
check  by  reducing
modulo
um  uh  modulo  the  uh  regular  or  the
sequence  F1  through  FN  that  you're
um  that  you're  looking  at  and  then  it
just  becomes  kind  of  an  obvious  fact
about  a  a  ring  tensor  itself  being
itself  uh  over  ring  tensor  over  itself
being  itself  rather  any  localization  of
a  ring  tensor  over  that  ring  with  the
localization  is  itself
so  you  did  something  like  this  before
yes  in  one  of  Peter's  lectures
he  proved  the  well  at  least  a  special
case  and  and  claimed  it  worked  in
general  that  the  if  you  take  a  solid
tensor  product  of  two  derived  complete
things  um  exactly  in  a  setting  like  this
so  you  have  a  ring  and  you  have  finitely
many
elements  um  or  any  number  of  elements  I
suppose  and  the  drived  solid  tensor
product  of  connective  objects  so  in
homological  degrees  uh  uh  will  still  be
derived  complete  why  because  we  proved
it  or  yeah  we  Pro  well  Peter  gave  a
argument  for  the  special  case  of  like  P
um  and  then  but  the  the  general  case  is
quite  similar  so  he  he  gave  the  the
heart  of  the  argument  of  the  general
case  yeah  you  have  to  write  yeah  you
have  to  you  have  to  do  some  work  you
have  to  do  do  some  analysis  but  it's  not
um  yeah  it's  something  we've  we've
discussed  in  previous
lectures
um  okay  so  now  uh  and
then  so  now  the  claim  that  I  would  like
to  make  is  that  if
C1  up  to  CN  are  locally
closed  uh
constructible  and  if  their
Union  say  set
theoretically  is  all  of  Spec  R
than  uh  these  guys
cover  let's  say
these  the  corresponding  a  A1  these  item
potent
algebras  uh  cover
uh  Dr  RZ
solid  uh  in  what  sense  so  in  the  sense
that
uh  so  if  you  have  m  in  deriv  category  of
RZ  solid  uh  such  that  M  tensor  a  i
equals  Zer  for  all  I  then  m  equal
zero
but  C  yes  thank  you  thank  you  very  much
yeah
yeah  so  uh  and  this  implies  that  you  get
uh  Dr  RZ  solid  is  some  limit  over  so  in
the  first  term  you  have  product  of
product  over  I  of  AI  modules  in  DRZ
solid  and  then  you  have  product  over  I  J
uh  less  than  J  it  doesn't  matter  modules
over  the  tensor  product  AI  tensor  i  j  in
the  same  thing  and  then  so  on  you'll
have  some  finite
check  uh
thing  what  sorry  cover  if  they  love
yeah
I  like  that  they're  they're  they're
giving  it  a  big  hug  that
cover
um  uh  yeah  so  if  they  love  DRZ  solid
uh  okay
anyway
um  and  the  basic  idea  uh  idea  of  the
proof  is  that  it  suffices  to
show  that  R  is  generated  by
uh  AI
modules  for  varying
I
um  but
um  so  so  for  if  I  just  take  the  example
of  uh  Spec
R  is  DF  Union  Z  of  f  uh  then  you  use
then  r  1  over  f  is  okay
because  it  comes  from  here  and  then  the
difference  between  R  and
r1f  uh  is  um  so  generated  in  a  finite
manner  finitary
manner  um  but  this  is  the  union  of  like
R  mod  f  to  the  N  with  some
shifts  um  this  is  actually  uh  the  fiber
SAR
here  uh  and  this  is  actually  an  RF
complete
module  so  it  uh  lives  in  in
ZF  um  so  if  you  take
all  modules  over  RF  inverse  and  all
modules  over  RF  complete  uh  and  then
generate  things  in  just  by  triangulated
category  nonsense  eventually  you're
going  to  hit  R  and  then  tensoring  with
anything  you'll  see  that  you  hit
anything  and  then  using  that  you  can
check  that  you  get  this  kind  of  covering
situation
so
uh  now  let  me
just
um  say  what  this  looks  like  in
examples
so  uh  so  all  of  these  yeah  so  we  have
the  sort  of  the  category  attached  to
each  element  in  our  cover  the  category
attached  to  the  pair  wise  intersections
is  just  modules  over  this  tensor  product
which  will  also  be  item  potent  so  let's
take  a  so  example  uh  let's  say  r  equal
zxy
um  so  let's  take  C1  equal  Z  X  Plus  or  -1
y  uh  so  we've  got  the  whole  Locus  where
uh  X  is  non  zero  and  then  we  can  look  at
the  locus  where  X  is  zero  but  Y  is  also
non
zero  um  and  then  we  can  look  at  the
locus
where  both  X  and  Y  are
zero
that's  the  power
series  uh  oh  sorry  I  should  add  X  and
then  complete  at  long  X  so  that's  z
braet  y  plus  or  minus  one  power  series
X
um  so  then  what  happens  when  you  start
taking  intersections  of  these  guys  the
most  interesting  thing  is  what  pops  out
when  you  take  the  intersection  so  the
algebra  attached  to  C1  intersect  C2
intersect  C3  um  you  can  just  calculate
these  things  using  the  fact  using  this
basic  fact  that
um  uh  uh  solid  tensor  product  of  derived
complete  things  is  derived  complete  um
so  what  do  you  get  so  when  you  do  C1
intersect  C2  you're  tensoring  this  with
this  that's  just  inverting  X  so  then  you
get  uh  z  y  plus  or  minus  one  laurant
series  X  and  then  you  should  tensor  that
over
zxy  uh  in  Z  solid  with  this  third
guy
XY  um  now  both  X  and  Y  are
inverted
um  uh  but  if  you  imagine  not  inverting  X
then  you'd  have  a  power  series  here  and
you  could  use  this  derived  X  complete
trick  to  see  that  you  can  move  this
power  series  onto  the  inside  here  and
you  get  that  this  is  Z  laurant  series  y
power  series  X  with  X  inverted  which  is
just  laurant  series  X
so  you  get  this  kind  of  object  here
where  you're  in  you  you're  iterating  op
so  you  iterate  in  general  when  you
intersect  a  lot  of  these  things  together
you  iterate  completing  and
inverting  and  it  famously  it  matters  in
which  order  you  do
this  um  but  it  all  comes  from  this  sort
of  commutative  situation  nonetheless  in
this  way  of  setting  it  up  here  and  I
want  to  make  the  point  that  the  so  these
things  are  like  kind  of  like  higher
local  fields  that  people  have  studied
and  they  tried  to  study  them  from  their
perspective  of  topological  Rings
topological  fields  and  there  were  just
terrible  problems  Matthew's  not  here
right  now  but  he  even  wrote  a  paper
explaining  that  everything  is  horrible
um  but  uh  if  you  put  them  in  the
condensed  world  then  they're  just
perfectly  well-  behaved  objects  that  you
can  work  with  um  it  arises  from  this
natural  procedure  and  it's  item  potent
even  in  the  DED  sense  over  this  ring  and
um
it's  just  uh  it  just  functions  uh
functions  quite
well  um  so  I  invite  you  to  try  to  draw
pictures  of  these  covers  um  to  get  an
understanding  for  what's  going  on  with
these  kinds  of  constructions  but  um  okay
I  think  I'll  stop
here
yes  for  finite  Type  zra  R
expain  R  solidification  of  the  countable
product  of  C  and  R  is  just  a  count
countable  product  of  yes  and  I  thought
it  also  true  for  something
like  the  ring  of  power  Z  that's  true
which  is  not  finite  time  that's  right
but  it's  also  not  discret  so  so  yeah
it's  detered  by  yeah  so  you  could  also
take  a  finally  generated
ring  uh  and  an  arbitrary  ideal  and  look
at  the  completion  and  then  you  get  the
same  statement  so  the  okay  uh  the  basic
the  compact  projective  generator  is  just
the  product  of  copies  of  the  itic
completion  of
r  i  see  and  that's  a  same  time  if  ring
is  not  fin  at  type  you  also  said  it's
Union  of  five
is  it  easy  to  see  for
this  no  but  Z  double  bracket  so  sorry
that  was  a  statement  about  discret
rings  so  in  general  if  you  maybe  if  you
have  some  uh  ring  that's  complete  with
respect  to  a  finitely  generated  ideal
you'll  get  the  good  answer  if  your  ring
modulo  that  ideal  is  finally  generated
maybe  this  is  a  good  way  to
if  is  the  ring  module  the  ideal  is
finite  type  over  Z  then  then  you'll  get
the  the  naive
answer  I  think  yeah  sounds  reasonable  at
least
yeah  solidification  solid  is  determined
by
underr
yes  so  the  same  formula  doesn't  work  for
Z  bra  what  so  Z  double  bracket  T  is  a
discrete  ring  yeah  as  a
ring  you're  looking  at  Z  double  bracket
t  as  a  discrete  ring  yeah  well  then  it's
just  no  then  no  then  the  doesn't  no  it
doesn't  hold
no  but  if  you  if  you  look  at
Z  what  to  say  if  you  look  at  Z  double
bracket  t  uh  non-discrete
modules  in  Z  double  bracket  T  discreet
uh  solid  Theory  that's  the  same  thing  as
same  as  in  in  uh  in  ZT
solid  Theory  so
um  once  you've  decided  to
yeah  yeah  once  you're  complete  then  it's
enough  to  solidify  uh  stuff  modulo  that
ideally  you're  complete  along  okay  yeah
so  does  that  address  your  your  concern
yeah
okay  so  what  again
you  discrete  ring  and  takech  the
associal  condensed  ring  and  you  consider
solid  relative
to  to  ah  relative  to  the  elements  of  the
Ring  yeah  okay
just  each  element  of  the
ring  and  then
you  so  if  you  consider  this  condensed
ring  with  the  product  topology  you  could
either  solidify  every  element  in  the
underlying  discret  ring  or  you  could
just  solidify  T  and  it's  the
same  in  fact  you  don't  even  need  to
solidify  T  I
mean  you  don't  need  actually  it's  just
it's  just  the  same  as  in  zolid  theory  so
if  you  look  at  a  Z  power  series  T  module
in  zolid  then  it'll  automatically  be
solid  with  respect  to  everything  so  in
fact  this  follows  because  I  mean  we
showed  that  the  topologically  nil  potent
element  ments  will  always  be  solid  um
and
uh
yeah  um  and  then  modular  topical  noo
elements  all  you  have  is  z
which  is  is  there  paper
on  picture  I  mean  paper  of  P  about  this
about
what
this  here  he  gave  he  talked  about  it  in
this  class
yeah  are  you  wondering  if  it's  written
down  somewhere  yes  it  is  uh  Guido
Bosco  wrote  down  at  least  the  case  of  P
like  complete  and  I  think  also  I  think
also  isn't  it  right  Peter
uh
uh  I  don't  remember  the  title  of  guo's
paper  but  I  think  also  Lucas  man  wrote
yeah  Lucas  man  wrote  The  General  case  in
his
thesis  and  the  in
the  I  think  so  I  forget  but  I  forget  the
title  of  the  paper  in  which  it's  contain
but  he  has  some  appendix  on  ptic
functional  analysis  from  the  solid
perspective  the  title  of  the
paper  you  say  I  said  the  in  this  in  this
here  it's  has  some  appendix  to  one  of
his  papers  whose  title  I  can't  remember
and  the  appendix  is  called  something
like  solid  uh  functional  analysis  or
solid  ptic  functional  analysis  something
like  this  which  kind  of  writes  down  the
the  argument  that  Peter  went  through  it
was  before  but
okay
yeah  just  yeah
yeah  very  quick  um  maybe  this  was  a
computation  done  earlier  in  the  course
I'm  just  forgetting  but  like  regarding
the  control  of  the  X  groups  like  The
Vanishing  of  of  them  like  so  we  had  the
proposition  for  finite  type  so  if  I  just
like  for  example  like  zp  zp  the  ptic
integers  yeah  was  there
computation  oh  so  you're  wondering  about
these  homological  Dimension  results  for
zp  yeah  so  there  you  get  uh  yeah  now  I
have  to  remember  I
think  um  yeah  there  you  should  get  also
it  should  just  only  be  X  ones  uh  yeah
that  that's  actually  even  that's  easier
to  show  than  in  the  case  of  of  solid  Z
because  zp  is  compact  so  you  can  there
you  can  directly  see  that  all  of  the
finitely  presented  objects  they're  also
they're  they're  it's  quasy  separated  and
it's
the  they're  also  all  those  objects  are
quasy  separated  and  they're  just  like
yeah  inverse  limits  of  finite  zp  modules
uh  countable  inverse  limits  of  finite  zp
modules  and  that  makes  the  whole
analysis  much
easier  so  to  summarize  so  for  let  us  say
R  is  a  field  yes
then  the  fin  presented  are  qu  separate
in  fact  they  are  all  given  quite  simply
by  you  just  have  the  either  infinite  or
final  product  of  the  cas  so  that's  okay
for  our
domain  except  the  fact  that  you  can  have
a  the  parar  group  is  maybe  not  trivial
but  still  it's  a  product  of  uh  you  you
have  a  resolution  you  have  product
copies  of  R  and  then  the  Kel  is
something  like  product  of  maybe  ah
actually  you  can  use  the  the  AR  Swindle
to  get  rid  of  this  anyway  so  you  can
it's  always  this
this  okay  so  it  is  a  resolution  of
length  to  like  like  before  because  we
don't  have  to  worry  okay  and  so  we  get
and  then  in  dimension  at  least  two  you
do  your  the  pro  the  uh  that  we  you
discussed  yes  okay  that's  the  yeah  and
then  there's  if  you  wanted  to  move  to
non-discrete  rings  then  you'd  have  to  do
more  analysis  and  so  on  and
yeah
yep  okay  thanks
everyone
\end{unfinished}
% !TeX root = ../AnalyticStacks.tex

\section{\ufs What happened so far? (Scholze)}
\label{sec:12-what-happened}

\url{https://www.youtube.com/watch?v=VMgZSP9sRdo&list=PLx5f8IelFRgGmu6gmL-Kf_Rl_6Mm7juZO}
\renewcommand{\yt}[2]{\href{https://www.youtube.com/watch?v=VMgZSP9sRdo&list=PLx5f8IelFRgGmu6gmL-Kf_Rl_6Mm7juZO&t=#1}{#2}}
\vspace{1em}

\begin{unfinished}{0:00}

Good morning. For the people here, you can fill out the survey there now or anytime this week or next week.

Alright, so where are we in the course? Let's try to take a little bit of a broader view. What happened so far? We started with this idea of condensed sets, the being cured, and so on---condensed mathematics. We tried to make the point that this is a good framework for combining homological algebra and functional analysis, with the goal of using this framework to develop a general, good notion of geometry.

At some point, we wanted a notion of completeness. We went in a certain direction, namely concentrating on solid modules---first over the integers, but then over more general rings of finite type, as discussed last time. This is closely related to a specific framework we discussed, which is extremely closely related to Huber's theory of adic spaces.

There were certainly some conceptual ideas that emerged from this, namely that the notion of completeness shouldn't be something that you define once and for all, but rather is an extra datum that is part of the data of what a commutative ring is in the setup. Completeness is a more relative notion.

So what are we trying to get through in the course? We have this category of analytic rings, and for reasons we already saw a little bit, when you localize, some of the structure might become not well-defined anymore. We might change the precise definition here a little bit to work with derived rings from the start, but I don't want to go into that today.

We want to start from this category of analytic rings and, as in usual algebraic geometry where you start with commutative rings and then build schemes by gluing the spectra of commutative rings, we want to have some procedure where we start with this category of analytic rings and then do some kind of gluing to produce some notion of analytic spaces, or maybe we'll directly go to some category of spectra.

Within this world, we want to find all sorts of examples. We definitely want our theory to accommodate geometry over the real or complex numbers, not just non-Archimedean geometry. So first, we would like to see more examples.

Some key questions are: How do the real numbers fit into this? Can we put a natural structure on the real numbers that is suitable for doing complex or real geometry? And can we also do non-Archimedean geometry using solid modules? What is a meaningful way to combine the two settings?

I want to develop one way of thinking about this, with the following example in mind. Maybe the most prime example of what we should be able to do in some world of geometry is the famous Tate elliptic curve.

It's actually, if you look at Fargues and Fontaine's paper, I think it's the one called "The Geometrization of the Local Langlands Correspondence," there really is also one of the first applications of a Serre's ideas, again discussing like the "tilting" of Deligne-Mumford stacks. This is a generalization to higher dimensional Beilinson-Bernstein varieties.

So, you have the étale spectrum of like $\mathbb{W}_{\mathbf{R}}$ series. Over there, you can look at some of the Fargues-Fontaine curve, which is an analytic space over $\mathbf{R}$. Informally speaking, that's all---there's some coordinate $T$ here, and $T$ should have the property that the absolute value of $T$ is bounded between some $\mathbf{Q}_p$-valued elements. 

And then you look at the part of the multiplicative group where the absolute value of $T$ is bounded by a power of one of those $\mathbf{Q}_p$-valued elements, and then acting on this, you have multiplication by a character. This is actually a totally discontinuous operation because it multiplies the absolute value of $T$ by the absolute value of the $\mathbf{Q}_p$-valued element, which is something between zero and one. So, it's actually a free and totally discontinuous action. And so, in the world of étale spaces, you can really pass to the quotient, and the taking this quotient is the nicest kind of quotient, because otherwise you just get where the quotient is like locally split, and locally the space just looks like this.

And so, you can define the Fargues-Fontaine curve to be this analytic Deligne-Mumford stack, and then you quotient by the $\mathbf{Z}$-action. And I mean, how does it look like? You started with, I don't know, $\mathbb{A}^1_{\mathbf{R}}$, and then by modifying by the $\mathbf{Q}_p$-valued element, it's moving everything towards the origin, but then when you take a quotient, this somehow is the same thing as taking some annulus of radius one of this $\mathbf{Q}_p$-valued element and then identifying the boundary.

And so, using this, you can actually see that this is some---yeah, it's actually proper, so it's proper and separated, and it's also smooth because it's locally just a Deligne-Mumford stack. It's a proper, smooth, connected analytic curve over $\mathbf{R}$. And actually, also, I mean, there's a group structure on the Fargues-Fontaine curve, given by some subgroups. It actually has a group structure.

And basically, in any world of analytic geometry, there should be a theorem that something that's proper, smooth, and one-dimensional is always algebraic, because you can always find an ample divisor by just taking any closed point and then taking the corresponding inverse ideal sheaf. This gives an ample line bundle, and then if you have any kind of version of GAGA or your favorite theorem which should always show that they imply the properness and algebraicity of the thing, and so it's definitely true here.

So, there---actually, so here, there implicitly there is some GAGA for over a nice noetherian base, Fargues and Fontaine's pairs, but also the statement about relative dimension one,

The coefficients of $Q$ to the $n$ in here are just polynomial in $n$. When you do this geometric series, the new coefficients still stay polynomial in $n$. This is a funny observation, because it implies that after just defining something formally as a power series, when you actually write down the equation, you realize that this is something you can specialize to any Archimedean value between 0 and 1.

You can certainly take $\C^\times$ as a complex analytic space, as a complex manifold, and for any complex number between 0 and 1, this picture makes literal sense---you can literally take that portion and you get an actual torus. So you can do that, and the analytic description makes perfect sense.

We would really like to have a way to do geometry so that we can perform this kind of construction of $\Z[Q]$ not just over the full power series algebra, but really over some subalgebra where we put some growth condition on the coefficients, so that we can also later on specialize to more general parts. The precise growth condition that you get here is quite a bit stricter than just the observation that it converges when the absolute value of $Q$ is less than 1.

This convergence is just a subexponential growth of the coefficients. It's pretty clear that geometrically, there should be a direct geometric way of seeing that this power series should converge when $Q$ is less than 1. There's some kind of geometric reason for the coefficients having at most subexponential growth, but it's not so clear how you would see that they actually have at most polynomial growth.

Let me actually talk a little bit about the geometry of the space of continuous valuations on $\Z[[Q]]$. You have this topologically nilpotent unit, and this is an extremely convenient structure to have, because it allows you to compare absolute values of all other functions against the absolute value of $Q$. There is a unique map from this space to the Berkovich spectrum, where the absolute value of $Q$ is sent to some pre-specified element between 0 and 1, say $1/2$. For most of the valuations we care about here, this map will be injective, but there are some rank 2 valuations where there's a little bit of extra information in the geometry that's not remembered by this map.

Next, let's consider a set of all "absolute values", but now they really take values in $\R/\{0\}$ instead of just $\R$. Here, you ask the following things: First, there are some basic properties, like the norm of $1$ is $1$, and the norm is multiplicative. Second, the valuations satisfy a strong triangle inequality, so it considers the usual triangle inequality, but also requires that they are all bounded by the specified norm on $\R$. 

This naturally leads to the product of copies of $\R^{>0}$ enumerated by all elements $x$ in $\R$, endowed with the Subspace topology. Actually, because of this construction, you can always replace $\R^{\ge0}$ here by the interval from $|x|$ to $\infty$, and then an arbitrary compact product of compact $\mathcal{H}$-spaces is still compact $\mathcal{H}$. So this is a nice compact $\mathcal{H}$-space.

The only difference between this space and the $p$-adic spectrum is the possibility of having higher rank things in the $p$-adic spectrum, but these just give rise to some minimal changes in the space. Implicitly, I'm endowing the integers here with a norm where the norm of $0$ is $0$ and the norm of all non-zero elements is $1$, since they all contain a unit for the ring I'm considering.

So I still want to understand a little bit about this geometry. It's more or less the same thing as the Berkovich space of the integers, which has a much richer geometry than the usual spectrum of the integers. Here's a proposition: The norm on $\Z$ I'm considering is the one where the norm of $0$ is $0$ and the norm of any non-zero integer $n$ is $1$, since they all contain a unit. There is also a more natural norm you can put on $\Z$, which is just the absolute value. For each prime number $p$, you have the $p$-adic absolute value, which also satisfies all the required properties, but with a choice of what the absolute value of $p$ is. This gives rise to a "ray" for each prime, and the inverse limit of joining these rays corresponds to maps to complete $p$-adic fields, including the usual rational numbers $\Q$ with the standard absolute value.

All right, let me also already discuss the side variation of this. You can also take the $\mathfrak{B}$-space of the integers with the usual absolute value, so $\lvert n \rvert = |n|$, the positive version of it. 

This has the same pictures. The difference between these two things is just the boundedness condition. Previously, the absolute value of 2 could never be 2 like it would be for the real numbers. But now, the absolute value of 2 can be 2.

So this is actually some of the same picture. There's again this thing, for each $\F_p$ number part, but now there's something extra---there is some kind of Tate interval. This corresponds to the map from $\Z$ to $\R$, and then you take the usual absolute value of $r$ raised to any power $\alpha$, where $\alpha$ goes from 0 up to 1. 

It's better to think of this if you take the usual absolute value, Peter, yes, can I vote that you use $\alpha$ instead of $p$ when talking about raising an absolute value to the $p$ power? Thank you. Just in time, to the $\alpha$. If I parameterize this in terms of some fixed absolute $p$, where $|p| = 1/p$, then this line here corresponds to the $\alpha$ where now $\alpha$ can be anything from 0 to infinity. But for the real numbers, you fix the usual absolute value, then you can raise that to any real power, and it will definitely satisfy the conditions, but actually, if you want the triangle inequality to be satisfied, you realize that this happens only if $\alpha$ is at most 1. So in this sense, this line for the real numbers is actually some stops in the middle compared to the others.

Okay, right. So where are we? We are trying to understand a little bit about the geometry of what the spectrum of $\Z[\frac{1}{q}]$ actually looks like, and it's basically the same as the $\mathfrak{B}$-spectrum, and this is fiber over the $\mathfrak{B}$-spectrum of the integers. Now let's actually try to understand what are all the fibers of this map.

For example, if you take the fiber at a prime $p$ of just $\F_p$, well then it's just one point. Then, as usual, some of the formation of the closed space commutes with some kind of fiber products, and so the rings $\F_p[\frac{1}{q}]$ are just a point, because they're already non-fields, and you fix the value of $q$ to be the half.

In characteristic $p$, the thing has just one fiber, which is this kind of the wrong series $\Q_p$. Maybe I should have said, I mean, this proposition is basically just 0, right, because of the absolute values. But then you can also have this spectrum of the integers, and then for each $p$, you have this half-line of $\F_p$s. If you base change this whole half-line of $\Q_p$s, someone looking at, you're in some sense taking, now this is some kind of punctured open unit disc, and now you're this base change to $\Q_p$ will actually be some kind of punctured open unit $\Q_p$, so this will actually be a punctured disc, and this map here to this line 0 to infinity, depending on how you parameterize it 0 to 1 or whatever, this should be an incarnation of the radii matter, but actually in a slightly funny way, it's say 1 over the $\log$ of the radius or something like this, I won't get it straight.

So there is a whole punctured unit disc here, which has an origin and a boundary. Whenever you fix a specific point on here, then this fiber will be some specific annulus in here, where the absolute value is fixed. And now you can wonder what happens as you move towards the invisible, anyway, but I'm always tempted to try to see the image is this circle here, and then when you move towards

Whereas when you move upwards on this Ray, go towards $\Q$, then you will get other points, and they will get closer to the origin. Okay, so what does this thing actually look like?

There is one special point which is $\Q$ and then there are special points which are $\Z$. And on the way there, you have the punctured open disc. The region in the middle, the punctured open disc, and the slightly mind-bending thing is how the different parts are glued to each other. I mean, the whole thing is a compact Hausdorff space, so it makes sense like to ask when you go in this direction, where you end up.

And so, if you move towards the puncture of this open unit disc, you move towards this point, and you move towards the boundary, we end up towards this point. So this whole space is a space that has $p$-adic regions for each $p$, each $p$-adic region is a function open $\Q_p$, and then they glue to each other in this funny way, where for each one, if you go towards the center, you will end up at the common point, which is is a kind of generic point, and for each one, when you go towards the boundary, you end up at the characteristic."
Now it has some kind of function open over the rational numbers, both $p$-adic places and the archimedean place. And everywhere when you go towards the center, you always end up with the center point of the picture.

The slightly awkward feature of this picture is that I had to now specify the absolute value of $\Q$ in advance, and then as the archimedean part of this picture, this punctured open just stopped at the boundary, although from the perspective of the $p$-adic curve there was no reason for stopping at all.

Right, so then one---but like this is precisely the kind of picture in which we would like to combine non-archimedean geometry. I mean, this space has parts which are literally complex analytic or real $p$-adic analytic, but they sit together in one job. And so one kind of PR version of the question about existence of analytic $\R$-structures is now like $\mathbb{K}^1$ and this guy or more any algebra for the vector---was a natural, and this has been a question that was very much in our minds back when we first found out about the solid theory and then tried to really go further.

It actually turns out that to define this, it's slightly better to work with not just precise the Hahn convergence condition, but functions which converge on some rate, some slightly larger dis. That's more techn. Generally speaking, for any kind of $\mathbb{B}$ offering which has a topological unit in the usual way, you can produce this kind of liquid analytic, and the resulting theory will be extremely close to $\mathbb{B}$ which theory, and this is something that we want to discuss at some point today.

However, I want to also talk about something else, something that we only found out a few weeks ago. Once you have this liquid and structure over $\Z$, maybe greater than a half, you can literally repeat the construction of the $p$-adic curve that I did in the beginning over this ring. And then someone show that the $p$-adic curve is definitely defined over this ring, and then okay, in the end you could also make a half larger and larger and would get that it's defined over the whole open unit dis, but you would not get this way the strange Tooumi draws bound on the coefficients.

There is actually a description of the "liquid" structure of the three modules. Yes, let $S = \bigoplus_{i=1}^n \Z$ as usual. Then in this ring, we take $\mathcal{H}(Q) = \R$, and this $S$ defines the three "liquid" modules on $S$ over this ring. The hard part is to prove that this actually gives an "analytic" ring structure defined as follows.

Just like $\R$ is the union of all intervals $[a,\infty)$ for $a > 0$, also the "free" modules are this union over intervals $[a,\infty)$. But then also, as in the free discrete case, we have this union over all size values, and this time I maybe want to index them by some real numbers $c > 0$. But then once you fix the radius and size, you're just taking an inverse limit of such free modules of the part where the norm is at most $c$. 

A similar situation where you can show that if you bound the norm here and give this a certain kind of topology, this actually becomes a compact $\mathcal{H}$ thing, the limit is still compact. This naturally suggests itself when you try to define some notion of "complete" modules over this ring, because the $p$-adic norm on this ring is defined just this way for one element, and then of course if you have a free module, you're just summing the absolute values.

Let me actually just give a concrete example. In the beginning of my lecture today, I was asking two questions: first, is there a natural structure on the reals, and second, is there one which allows you to combine things. Now I'm starting to answer them in the opposite order. 

First I said that there's something that combines them. Let me now also give the answer to the first question: what are the "analytic" ring structures on the reals? You can specialize from $\Z[t^{1/2}, t^{-1/2}]$ to $\R$ by sending $t$ to some number $T$ here less than $2^{1/2}$. This defines a point of the $\mathcal{S}p$ space that maps to the $\mathcal{B}$ space of $\Z$, and so must actually correspond to some power of the usual absolute value on the reals. 

So inside the $\mathcal{B}$ space of $\Z$, you have this half-interval for the reals, where $\alpha$ from 0 to 1 corresponds to the absolute value on $\R$ to the $\alpha$ power. And now this map is actually realizing some isomorphism between these two things, where the value of $T$ is mapped to some real number $R$, and this can be made explicit: $T = R^{2}$.

Okay, so the value of $T$ determines some $\alpha$, and I could have just told you the formula $\alpha = \log_2(T)$. But this would seem slightly curious---what does it mean? The meaning is that you have a point in the $\mathcal{S}p$ space that maps to the $\mathcal{B}$ space of the integers, and you get some point there. This is the $\alpha$. 

So we have a structure here, and we get one here. But the "complete" modules

The norm $\|x\|_\alpha$ is a sum of $|x_i|^\alpha$. Note here that $0 < \alpha < 1$, so this is not one of the usual kinds of norms that you would usually consider in real functional analysis. This is not locally convex. Usually when you put $L^p$ norms, the $p$ lies between 1 and $\infty$, but here we're going to the left and using this non-locally convex norm.

This norm does satisfy the triangle inequality, but it doesn't satisfy the usual scaling by real numbers. You raise it to the $\alpha$ power, but usually you'd ask that it satisfies the scaling with respect to the usual real numbers, which this does not do.

However, if you just look at the unit balls in a 2-dimensional vector space, they have a very peculiar shape, something like that. So you might be tempted to think that this is a stupid way to put a ring structure on the reals, and you should do something else. But this does work.

I guess that the completed modules on $\R$ should be the bounded measures on $\R$. One way to define them is as the dual to continuous functions, but that does not work. The way to describe them would be to do a similar construction but with one more parameter $\alpha$. This would be closely related to the usual theory of solid modules, in about as clean a way as the usual theory of linearly complete modules.

We were really hoping that this $\alpha$-norm structure on the reals would be impressive, but it isn't. However, it's related to some interesting things in classical functional analysis. One thing is that the category of complete modules is not stable under extensions, which was realized around 1918. But once you allow some locally convex vector spaces into the picture, the theory works again, though you're forced to use strictly convex norms.

I should also discuss one other thing quickly. If we specialize this construction to the $p$-adic numbers, you might expect that we're just trying to extend over the "missing part" at the real numbers, so the $p$-adic part wouldn't change much. But that's not true---you actually get new liquid structures on $\Q_p$, with an $\alpha$ parameter that can be anything between 0 and $\infty$. These $\alpha$-liquid modules on $\Q_p$ can be described in a similar way to the real case.

You need them because the Slo Theory already works with you. So I want to ask a small question just to clarify possible confusion. You have the $L_\beta$ for different $\beta$, which are again defined in the same way, just by some powers of the $\beta$, without taking one over $\beta$. This is a slight mismatch. What I know is that if I literally specialize the definition of $A$ to $\infty$, I would be summing the supremum norms, which is not what I'm doing when I do this $\infty$ here. The supremum here is the actual...there's a little bit of a mismatch in notation between here and here, but I think it's okay.

But what is the when you take the union over $\beta < \alpha$? I think usually it was a filtered union. That is, is it the case, because here you have to be slightly careful that the way you index the constants, some should change when you increase $\alpha$. So why the limit, the union over $\beta$, what are the inequalities between the $L_\beta$ here, the Tate unit over all $\mathcal{C}$? And then, is it literally the case that these spaces get larger and larger as you increase $\beta$?

Okay, think of it like this: in some sense, you can think of a whole series of series that go from 0 all the way to $\infty$, where you put some norm here. At this end, you have solid modules, and then at each point $\alpha$ here, you have the $\alpha$-liquid ones. In terms of the class of modules, being solid is a much stricter condition than being $\alpha$-liquid. So the class of modules that you allow here becomes larger and larger as you make $\alpha$ closer to 0. And as you go to $\alpha = 0$, you have all condensed objects, because if you naively put some kind of $L_0$ norm there, meaning that there's only a bounded number of coefficients that are non-zero, and then you put some bound, this actually recovers a free condensed module without any competition by variant of what I said at the beginning about the integers.

Okay, I guess it's time for the new thing. Right, so back then we were trying to look, find natural candidates for how what the free compact modules could be, and then it was some hard matter of proving that they actually define the existence of being an antintic brain. But in this course, we have a different mindset. It is that to the $\mathcal{F}$ and the structure, we find something natural, the morphisms of this projective object $P$ that you want to be morphisms. The free outno sequence should come after completion. And because in life being groups, this $P$ has these very strong properties, like being compact and internally projective, this will always define a drink structure, so it has become extremely easy to produce analytical instrues.

Here's an example: it turns out that to define the adic curve, you only really need two things. The first is that $P$ should be topological, it's new, and should be unit clear, and both of these have clear meaning in terms of the underlying cont ring. I mean, so being new means that you have a map from basically this guy and up with the natural ring structure, sending 1 to $\mathcal{C}$. But then you need some completeness for your modules, you need to be able to some certain sequences. And really the only thing you need is that $1 - Q \cdot \text{shift}$ acting on this projective $P$ is an isomorphism. And it's clear that there is a universal example of such an antic ring. I mean, you just take the free generated by a unit, that's just a condensed ring, and then this condition puts some ring structure on this ring also, which is good to much. The key is that there is an initial such example. Existence is easy, the hard part is a description. Hey, so it's a pair of a triangle, and you can actually describe a triangle, and well, let me describe the underlying ring. This is precisely those sums such that the PO and thus, if you like, the claim is that we can do analytic geometry as usual over such rings. And so you can just repeat the construction to the curve, and then you will see for geometric reasons,

Already, the free complete guy---a triangle, for example, would be a free non-complete guy, but a triangle is a free complete guy. It is the union over all $a > 0$ of something. So, as usual, this is a limit of a sum of something. 

I mean, of the part of the free module, let's say, where the coefficient of $Q^n$ has $\ell^1$-norm at most $n + n^k$. In the future lecture, I'll give a more precise description. But basically, you're asking for a similar description as before, but now you're asking for some polynomial growth conditions on the coefficients.

The condition here is that the coefficient of $Q^n$ can grow at most like a polynomial in $n$ of degree $k$, but I also have to allow the presence of negative coefficients. Anyway, then the limit as $n$ goes to infinity has $\ell^1$-norm at most $n + n^k$.

If you want the ring to be the thing where you have coefficients that grow at most polynomially, and you think about a way to encode that on the modules, then this is what you would be doing. Peter, do you really mean $\ell^0$-norm or $\ell^1$-norm? Because it's about a free module on $\Z$, so let's say $\ell^1$-norm.

Within doing analytic geometry over this space, you will get a geometric way to construct the Tate curve really over the kind of smallest ring, which is actually the sum of something. This also means that if you specialize back to $p$-adic numbers, then the Ganga series is very close, but not quite at zero. It's a fun exercise to take this description of the free modules and base change it to $\Q_p$.

Let me start now.

\end{unfinished}

% !TeX root = ../AnalyticStacks.tex

\section{\ufs Generalities on analytic rings (Clausen)}
\label{sec:13-analytic-rings}

\url{https://www.youtube.com/watch?v=38PzTzCiMow&list=PLx5f8IelFRgGmu6gmL-Kf_Rl_6Mm7juZO}
\renewcommand{\yt}[2]{\href{https://www.youtube.com/watch?v=38PzTzCiMow&list=PLx5f8IelFRgGmu6gmL-Kf_Rl_6Mm7juZO&t=#1}{#2}}
\vspace{1em}

\begin{unfinished}{0:00}
In the last lecture, Peter was---well, we spent a lot of time talking about solid analytic rings. In the last lecture, Peter started to introduce different kinds of classes of analytic rings which also work in the archimedean context, so these "liquid" and "gaseous" analytic rings.

What I want to do today is I want to talk about some generalities on analytic rings, partially in service of the story Peter is telling, and partially because it sort of should be done at some point. It's going to turn out that when discussing these new examples of analytic rings, it's kind of nice to have a good handle on the category of analytic rings and how to manipulate it. So I'll just be presenting some various facts about analytic rings and their category.

Let me start by recalling the definition. So an analytic ring is a pair $(R, \mathcal{M}_R)$, where $R$ is a condensed, commutative ring, and $\mathcal{M}_R$ is a full subcategory of the category of condensed $R$-modules, closed under limits, colimits, and extensions. Additionally, any $\mathcal{X}$-group in $\mathcal{M}_R$ mapping to any condensed $R$-module still lies in $\mathcal{M}_R$. Finally, the unit object, the underlying ring itself, should lie in this full subcategory. Let's call this last condition "$\star$".

Now I'll make another definition. A "non-complete analytic ring", or "pre-analytic ring", is a pair as above, except we don't require the $\star$ condition. This concept arises when trying to produce an analytic ring structure on an interesting condensed ring, as the $\star$ condition can be quite complicated to verify. Often, there is a simpler ring that maps to the condensed ring, and it's easier to produce a pre-analytic ring structure on that simpler ring. Then, there is a completion procedure that allows one to go from a non-complete analytic ring to a complete, honest analytic ring.

We've seen examples of this when discussing rigid spaces. Recall the solid $R^+$ theory---when specifying the category attached to a rational open, we took the category of $R$-modules and imposed some conditions, which actually made it into a pre-analytic ring. But the $\star$ condition was not necessarily satisfied, so one would need to change the $\mathcal{M}_R$ to basically the structure sheaf to get an honest analytic ring.

In general, what you want to do with these non-complete analytic rings is to complete them to get something that is an analytic ring in the honest sense. That's the procedure I want to discuss right now.

A map of pre-analytic rings is just like a map of analytic rings---a map of underlying condensed rings such that the restriction of scalars of any module in the domain category lands in the module category of the codomain.

The completion function, the $R$-completion going to mod $R$, and then you have the $S$-completion going to mod $S$. So, this is a completion functor exhibiting this as a localization. And similarly here, this condition here is the same thing as saying that if you go here and then you complete, that actually factors through here necessarily uniquely because this is a localization. So, it's saying that if you have a map here which is kill, which goes to an isomorphism here, then it goes to an isomorphism here. So, then you get a symmetric monoidal base change functor from mod $R$ to mod $S$.

So, that defines the category of pre-analytic rings. In other words, there's a fully faithful inclusion of the category of pre-analytic rings into the category of analytic rings, and this inclusion has a left adjoint. Moreover, the left adjoint sends the triangle mod $R$ to---and some explanation will be required, but we take our triangle and we complete it with respect to this pre-analytic ring structure here. So, we apply the left adjoint to the inclusion from mod $R$ into mod $R$ triangle, and then we take, so to speak, the same mod $R$.

So, first, we need to make sure that this is of the appropriate type. I mean, I need to explain how mod $R$ is actually a full subcategory of mod $R$ triangle hat, in order for this to parse as a claim. And then I need to check that this setup satisfies the same axioms as this setup, so it actually defines a pre-analytic ring.

So, first of all, this mod $R$ triangle hat is a commutative condensed ring. This is a sort of a completely general phenomenon. We have this mod $R$ triangle, and this localization functor to mod $R$, this completion functor, this $R$-completion, and this is symmetric monoidal and cocontinuous, and it has a right adjoint, which is the inclusion in this case. But when you have a symmetric monoidal functor with a right adjoint, then the right adjoint is automatically lax symmetric monoidal. And in particular, if you take the unit object here and then you apply the right adjoint, it gets a commutative algebra structure. So, mod $R$ triangle hat is a commutative algebra in this category, in this tensor category, and therefore a fortiori in condensed abelian groups, so it's a condensed commutative ring.

Moreover, in this general situation with the right adjoint to a symmetric monoidal functor, if you take any object here and apply that right adjoint, it gets the canonical structure of a module over this commutative algebra, again just purely formally. So, there's this natural factoring. We have mod $R$ including into mod $R$ triangle, and this actually factors through mod $R$ triangle hat modules, and then the forgetful functor. And the next thing I should check is that this is actually fully faithful, so that we can indeed view mod $R$ not just as a full subcategory of mod $R$ triangle, but of mod $R$ triangle hat.

Completion is idempotent. You find that this is indeed just a triangle M tan, okay?

So, we have to verify all of the axioms, and this is actually a little bit more subtle because we need to make sure that the calculations are going to work out correctly. Let me remark that this argument for full faithfulness will also work for $X$, provided that the derived completion on our triangle is the same as the completion on our triangle.

Recall that we had this notion of the derived category of an analytic ring, which was a full subcategory of the derived category of our triangle, and we had the derived completion functor, which was not necessarily even on something sitting in degree $0$. Its $\pi_0$ or $H^0$ is the same as this, but it could have higher homology. For example, in this situation with these weird non-sheafy adic spaces, we don't have that. But in general, we don't have that. So what we can see is that everything works if we replace our triangle hat by our triangle hat, the derived thing, and use the notion of a derived analytic ring. Meaning that if you ask not to give an analytic ring structure on this ordinary ring, but an analytic ring structure on this derived enhancement, then using the same kind of formal argument, you'll be able to check all of these properties.

So, let's introduce a notion of a derived analytic ring. An analytic $E_\infty$-ring is a pair $R_\triangle, \mathcal{M}_R$, where $R_\triangle$ is a connective condensed $E_\infty$-ring, and $\mathcal{M}_R$ is a full subcategory of $R_\triangle$ that is closed under limits, colimits, and such that the internal $\mathcal{H}om$ from anything in $\mathcal{M}_R$ to anything in $R_\triangle$ is still in $\mathcal{M}_R$.

The proposition is that for any connective condensed $\infty$-ring $R_\triangle$, there exists a bijection between the set of pre-analytic ring structures on $R_\triangle$ and the pre-analytic ring structures on just the condensed commutative ring which is gotten by taking $\pi_0$. This projection on the level of pre-analytic ring structures restricts to a bijection on the level of analytic ring structures. And what is this gotten by? Here, you have this $\mathcal{M}_R$, and you send that to just those things in $\mathcal{M}_R$ which happen to lie in the heart.

As $D = 0 \oplus \pi_0 R \triangle$, so you have a module concentrated in degree zero over some derived ring, that's just the same thing as giving a usual module over the $\pi_0$. And in this direction, it's given by sending $\operatorname{Mod}_R$ to the set of those $M$ in $D \geq 0 R$ such that $H_* M$ is in $\operatorname{Mod}_R$ for all $i \geq 0$. 

So, is this definition that $D > 0$ might not be closed on the limits? Didn't I ask it to be a whole category? I can't repeat the question, but it goes to the greatest than minus one. I mean, yeah, I still have, but we gave the argument fixing that. So when you have this assumption, you can see that that holds. When you take an arbitrary direct sum of copies of the unit, then you'll see that products of a fixed, even higher derived products of a fixed, given fixed element will be still in the category. And then if you have an arbitrary product, then you can write it as a retract of a product with a fixed element by taking the direct sum of all the elements.

So, let me make a remark. In a special case where $R \triangle = \pi_0 R \triangle$, we see that the two definitions are consistent, meaning in the case where the notion of rings overlaps, namely where your derived ring is just concentrated in degree $\mathbb{Z}$, so it is a classical ring, the derived definition of what an analytic ring structure or a pre-analytic ring structure is matches up with the naive definition we gave on the level of $\infty$-categories.

The second remark is this finishes the proof of the proposition on completion, because what we can do is we can just... So there's no obstruction to proving the proposition on completion in the derived setting, and then we can move it down to the $\mathcal{A}n$ setting by going from the analytic ring structure on this derived thing that we produced to the analytic ring structure on its $\pi_0$, just by applying this procedure.

Monoidal category, then you can look at commutative algebra objects in it, and there's an $\infty$-categorical version of that, which implicitly involves things like $\infty$-operads or something. But really, an $E_\infty$-algebra over $\mathbb{Z}$ will just be a commutative algebra object in the derived category of $\mathbb{Z}$ in that $\infty$-category, with the usual derived tensor product.

So if you believe in this $\infty$-category mumbo jumbo, you don't need to think about it in being built in terms of topological spaces and operads explicitly. You can kind of just plug into some simple categorical formalism.

Okay, so that's the notion of completion. Maybe the good thing to say is that you have an analytic ring over here, which gives you an analytic ring over here, such that the analytic ring structures here are the same thing as the analytic ring structures here, and such that each $\pi_i$, not just $\pi_\mathbb{Z}$, is complete in this category here.

I'm not going to prove the proposition, as this was proved in one of the older lectures. You can look there for the argument. Now I want to move on to the next topic, which is colimits in the category of analytic rings, although maybe I should make another remark.

It's important to note that the derived categories can change when you take an analytic ring structure on $\pi_k$. It gives you an abelian category, but we also argue it gives you a derived category. That derived category is not necessarily going to be the same as the derived category you get on $R^\triangle$ when you move along this equivalence, because the higher homology in $R^\triangle$ can make things a little different. 

So when we're doing this completion on the level of naive analytic rings sitting in degree zero, it's important to note that the derived category is not the correct one. If $R^\triangle /R$ is the completion of $R^\triangle /R$, then it's not necessarily true that the derived category of this thing is the same as the derived category of that thing, unless you use derived completion on the left-hand side. If this ring that you calculate has higher homology, then you really should be considering a derived analytic ring instead of an ordinary analytic ring.

Okay, so now we have this notion of completion of analytic rings, and this lets us discuss colimits of analytic rings. The procedure is to take the colimit in pre-analytic rings and then complete. What is this colimit in pre-analytic rings? It's actually quite naive. If you have a diagram $R_i^\triangle /R_i$, then the colimit is just the tensor product of the $R_i^\triangle$ modulo the tensor product of the $R_i$.

Is the co-limit of condensed rings given by taking the co-limit of condensed commutative rings? No, sorry, the co-limit of $R_i$ mod $r_i$ is the pair of the co-limit over $I$ of the $R_i$ and that's the co-limit in condensed commutative rings. Then you take a certain full subcategory, so-called mod cimit $I$ ini $I$ $R_i$, where this is the those $m$ in mod co-limit $R_i$ such that when you restrict scalars to $R_i$, $m$ lies in mod $r_i$. It's just an intersection of a bunch of categories where those categories are just required when you restrict scalars, you lie in the corresponding complete thing.

This is quite straightforward to see from the definition of the category of analytic rings. By definition, a map is a map of rings satisfying a certain property, so you certainly want the co-limit of the rings here, and this definition is just tailored so that you have the correct property. Checking that this satisfies the axioms of a pre-analytic ring is also not difficult. Closure under limits and colimits is elementary, and the closure under the axioms is also not so difficult.

Let me now take some examples. The claim is that for filtered co-limits in analytic rings, the completion is unnecessary. The filtered co-limit of $R_i$ mod filtered co-limit of $r_i$ is already complete. That means that for every $I$, you lie in mod $r_i$, but from $I$ on in this filtered co-limit, you lie in mod $R_i$ by definition, and so from $I$ on, that's a co-final collection, so you can calculate this as a filtered co-limit of things which satisfy the condition, so it also satisfies the condition.

An example of this is that if $R$ is a discrete commutative ring, then solid $R$ is the filtered co-limit of solid $r_i$, $i$ in $I$, if $R$ is the filtered co-limit of the $r_i$.

If you understand filtered co-limits, then the next thing you should try to understand is pushouts. If you have an $R \to A \to B$ diagram of analytic rings, then you need to complete the pre-analytic ring which is a tensor over $R$ of $B$, and then the full subcategory of those $m$ in mod $A$ tensor $R$ $B$ such that as an $A$-module, that lies in mod $A$ and as a $B$-module, it lies in mod $B$. It is important to complete here, because this thing has no reason to lie in mod $A$ or in mod $B$. The completion functor here a priori involves iterating an $A$-completion and a $B$-completion, alternating it one category or the other. The exact same remarks apply in the infinity category if you know the notion of derived analytic ring.

In general, you'd have to take this thing $A$-completed as an $A$-module, then $B$-complete that as a $B$-module, then $A$-complete that and $B$-complete that, pass to a sequential co-limit, and that would be the description of the completion functor. In practice, it's usually not so bad, but that's a priori what you need to do.

You enforce all of the relations described by your rational open. So if it was $\mathcal{F}_{1}$ over $G$, then you require that $gx$ is invertible on your modules, and that $\mathcal{F}$ over $G$ is a solid variable with respect to all of your modules. And then the pushouts---so a new analytic ring, I hesitate to call it $\mathcal{O}(U)^{+}$, but okay, I'll do it anyway. Maybe. And then if you take $\mathcal{O}(U)^{+}$ tensor over $R$ with some other $\mathcal{O}(V)^{+}$, then what you're going to get is the same thing for the intersection of these rational opens: the analytic ring corresponding to the intersection of these rational opens. I guess in the discrete case, our discrete it literally is just this. 

So the notion of pushout in analytic rings is corresponding to intersection of rational opens here, and that just follows from the definition of this. So yeah, these pushouts are in general kind of the most important construction, because they're geometrically what's supposed to correspond to pullbacks. And this business of having to complete them makes for an additional subtlety compared to the usual algebraic geometry.

Okay, so questions about that. Is the procedure easier when you just want a finite product? No, the case of relative tensor products is no easier. The case of absolute tensor products is no easier than the case of relative tensor products. So in fact, I could have---I didn't really need in this first part to say filtered, sifted is enough if you know what sifted means. And then general colimits, just as general colimits can be decomposed into filtered colimits and pushouts, general colimits can also be composed with sifted colimits and coproducts. So concretely, the completion functor for this relative tensor product is just the same as the completion functor for the absolute tensor product.

Okay, now I want to have a little bit of fun, well, depending on your definition of fun. I want to prove the following theorem. Suppose $R$ is an analytic ring. Then the Frobenius map, which goes from $R$ to $R/P$, induced by $x \mapsto x^p$, is certainly a map of condensed rings. But this is actually a map of analytic rings, from $R$ to $R/P$. And the $R/P$-modules are just the $R$-modules that are complete when viewed as $R$-modules.

You should probably assume $R$ has characteristic $p$, in which case it's really just a map from $R$ to $R$. What do we need to do to prove such a claim? Well, according to the definition, what we need to do is to see that if $M$ is in $\mathrm{Mod}_R$, then the Frobenius pushforward of $M$, which lies in $\mathrm{Mod}_{R/P}$, actually lies in $\mathrm{Mod}_R$. This is not so obvious how to check. We have these maxims of analytic rings, they're all about closure, categorical closure properties in linear algebra, like limits and colimits and so on, they say absolutely nothing about the Frobenius. They give no kind of hint as to why this should be true. However, let me make a remark: there is another perspective on maps of analytic rings.

To be a map, a map of analytic rings, you need to check a priori for all $S$-modules. But, by co-limits, it's enough to check it for a generating class. In fact, you know, so for these guys, in fact you can even take $T$ to be the countable set if you like, so that this is the thing that you actually get, is that this is an $R$-module. So the map from the free $R$-module on $T$ to it factors through $RT$, and this is functorial in $S$, functorial in $T$.

Moreover, well, there's a small extra condition that in particular is satisfied by them. So, if you do have a map from $T$ to $R$-triangle, a map of condensed sets from $T$ to $R$-triangle, then there's two things you can do. One, you can make a map from $R[T]$ to $R$-triangle, an $R$-triangle linear map, because by definition of analytic ring, this thing is complete. You go to $R$-triangle $[T]$ and then you go to $R[T]$. But then you also have this map here that we've assumed exists from $R[T]$ to $S[T]$, but on the other hand, you can compose this to $S$-triangle, and you get this thing here. So this is just a map of rings, this is the thing we posited to exist, and this exists for the same reason this exists, and that square will commute when you have a map of analytic rings.

So I want to claim---conversely, if $R \to R$-triangle to $S$-triangle is a map of condensed rings, such that there exist these maps from $RT$ to $ST$ satisfying the conditions above, then $R \to S$ is a map $R \to S$.

This is nice because you don't have to explicitly think about how you'd build $S[T]$ from $R[T]$-modules if you just have the maps that would kind of indicate it. Then you can actually do it. Let me maybe give a hint of the argument---it's in the notes from a previous iteration of this. A hint of the argument would be that $\text{Mod}_R$ is monadic over condensed sets, so you have a forgetful functor from $\text{Mod}_R$ to $\text{mod}_R$-triangle, which in turn forgets to condensed sets, and that satisfies the hypothesis of Barr-Beck. It's basically a localization, or the right adjoint of a localization, followed by a forgetful functor. And so this category can be understood as the category of whatever they call it, algebra over some monad here, which is kind of the free $R$-module monad. And then if you want to show that every $S$-module is an $R$-module, it would be enough to produce a map of monads from the free $R$-module monad to the free $S$-module monad. Well, the very first step in producing such a map of monads would be giving the map from the free object here to the map of the free object here, and then there's a condition you need to check, compatibility with the monad structure. And if you play around enough, you can reduce what you need to check to just this commutative diagram here.

Okay, so let's continue discussing Fenu's. So now what are we reduced to? We need, for every compact Hausdorff space $T$, a

Recognize that each of those cross terms is a norm from $M \otimes \mathbf{P}$, so it's a nice exercise if you've never done it.

Sorry, continue over here. I should say you have a group homomorphism---that's the special thing that happens here.

And what is the relation of this construction with the Frobenius on a ring? If $M = R$ is a commutative ring, then you have this thing which goes from $R$ to $R \otimes_\mathbf{P} \mathbf{T}^\mathbf{P}$ and $\pi_0$ of that. But then you can use the multiplication map from $R \otimes \mathbf{P}$ to $R$, to go to this where now the $\mathbf{T}^\mathbf{P}$ action is trivial. Here you're using that the ring is commutative, so that the multiplication is a $\mathbf{T}^\mathbf{P}$-equivariant map from $R \otimes \mathbf{P}$ to $R$. But now since the $\mathbf{T}^\mathbf{P}$ action is trivial, this is just the same thing as the norm map, which is just summing over the action of the group, but the group is trivial, so you're just summing over $\mathbf{P}$ copies of 1. So this really is just $\mathbf{R}_\mathbf{P}$. And if you trace through this, this is the Frobenius.

Okay, and now if $M$ is an $R$-module, then this map, the so-called Frobenius map from $\mathbf{T}^\mathbf{P}$, is a map of abelian groups. In fact, it is a map of $R$-modules if you Frobenius twist on the right-hand side. In other words, this abelian group level Frobenius is kind of linear over the ring level Frobenius, in the appropriate sense.

What this suggests is that we should try to apply this abelian group version of Frobenius and see what happens. So now we take $T$ a profinite set, then we get $R[[T]]$, and we can always do this construction. Now we're viewing $R[[T]]$ as a sheaf of abelian groups on profinite sets, and we can apply this construction here. We tensor it $\mathbf{P}$ times, and we get $\mathbf{T}^\mathbf{P}$, and then we take $\pi_0$. So that's all just happening at the level of sheaves of abelian groups.

We can always map this to any further completion we like, and in particular we can map it to $\pi_0 R[[T]]$, which is the free $R$-module on the profinite set $T$. We have this sort of Künneth formula, so the tensor products of the free modules are just the free modules on the product, and then $\mathbf{T}^\mathbf{P}$ is acting here as well.

We wanted to make a map from $R[[T]]$ to itself which is Frobenius-linear, and we've gotten to $R[[T]]$ to the $\mathbf{P}$. How do we compare them? Well, we can put $R[[T]]$ into this via the diagonal embedding, and that's also equivariant for the $\mathbf{T}^\mathbf{P}$ action if you make $\mathbf{T}^\mathbf{P}$ act trivially here.

The claim is that this map is an isomorphism. If you buy that, then you're basically done. $\mathbf{T}^\mathbf{P}$ is acting trivially here, so the Tate construction is just modding out by $\mathbf{P}$, so this is indeed $\mathbf{R}_\mathbf{P}[[T]]$. And for similar reasons, the composite is actually going to be Frobenius-linear and give you the desired construction. You also have to check that condition, but the full argument is in the notes

Well, we're going to more directly use them now. So the proof is: the first claim is that R[SCP] is actually mapping injectively into R[S], and the reason is that any inclusion of light profinite sets has a retraction. This was a fact that Peter proved in the second lecture of this course. So if you apply it to the inclusion of CP and S, you find there's a retraction. So then if you hit it with any functor, it'll still be injective.

So what does this imply? This implies by the long exact sequence in take-chology that it suffices to show that if you take R[S] mod R[SCP], this thing has vanishing take-chology.

The second claim is that generally, if you have T inside S, then $R[S] mod R[T]$ only depends on the locally compact space, the locally light profinite space which is the complement $S\setminus T$. It's not like a group, where you can choose.

Okay, so that's not a precise claim. What's the more precise version of the claim? If you have S' mapping to S, and then you have T included here, and you form the pullback to get T', and if this is an isomorphism over S minus T, so you're sort of blowing up T, so to speak, or you're choosing a different compactification of S minus T, then R[S'] mod R[T'] maps isomorphically to R[S] mod R[T], and this holds because this square here is a pushout in condensed sets, which is a little exercise you can see using the definition of the Grothendieck topology.

And then the third thing is that if X is a sigma-compact, totally disconnected, locally compact Hausdorff space, and CP acts with no fixed points, then X is actually isomorphic to some coproduct of CP many copies of Y, where CP is acting by permuting those copies. Sigma-compact means countable union of compacts.

Okay, so if you combine two and three, what do you deduce? You deduce that in that setting, R[S] mod R[SCP] is actually a direct sum of CP many copies of some guy X, where CP is acting on this set.

Copies of some guy and then you compact a in a different way by just compactifying $Y$ and then taking the $\mathbb{C}P^n$ many disjoint union copies of that to get a compactification of $X$ and then use that to calculate the quotient here. Then you'll find that the module is induced, and an induced module has vanishing Tate cohomology.

Okay, so this fact that $\mathcal{F}$ is a map of analytic rings, it's not just a cute fact. I'll make a remark, but I don't think I'll go into the details. This theorem on $\mathcal{F}$ implies that if our triangle mod $R$ or let's do it in this setting of a derived analytic ring or a pre-analytic ring, and we give our triangle the structure of an animated commutative ring or condensed ring, which gives new functors like derived symmetric powers on $D^{\geq 0}_R$, which are the things used to build free animated rings or the monad describing animated rings over our triangle, then these symmetric powers descend. If a map $M \to N$ goes to an isomorphism in $D^{\geq R}_R$, then its symmetric powers also go to isomorphisms there. This allows you to do normalization or completion of pre-analytic rings in the animated context, giving a good category of animated analytic rings.

Okay, so one last topic. I'll call it "killing algebra objects". Recall for motivation that we presented the solid $\mathbb{Z}$-theory as the analytic ring structure on the integers where $M$ is in solid $\mathbb{Z}$ if and only if the internal Hom from $P$ to $M$, shifted by -1, is an isomorphism. This is equivalent to saying that $\mathrm{Hom}(A, M) = \mathbb{Z}$, where $A = P / (T-1)$. This has an algebra structure in the ambient category $D_{\mathbb{Z}}^{\mathrm{cond}}$. 

In general, if $C$ is a symmetric monoidal category, we can consider "killing algebra objects" $A$ in $C$, and the category of modules over $A$ in $C$. The condition that a map $M \to N$ goes to an isomorphism in the localization $D^{\geq R}_R$ allows you to extend symmetric power functors from this thing to that localization.

Let's say we have a monoidal, presentable, stable infinity category, and let's include that the tensor product commutes with co-limits in each variable. Let's say A and C is an algebra object, and it really, I only need it in some kind of weak sense. So let's say that we have a multiplication map, literally just A tensor A goes to A. We have a unit, so there's the unit object in the symmetric monoidal category, and then we require that the multiplication is either left or right unital.

So let's say one is the unit for A, and then the multiplication is an isomorphism. You said you don't require any associativity or nothing, yeah, it's really, really weak. Okay, then we can define D subset C to be the full subcategory of those M in C such that the internal Hom, which implicitly is an RHom here, from A to M equals zero. So we're killing A, we're declaring that A should be equal to zero, but also in an internal Hom sense. And then the goal will be, in favorable situations, to give a formula for the left adjoint to the inclusion, to the inclusion. Sorry, is K the name of the mathematician or just "killing"? Killing, killing, killing means not the mathematician responsible for the killing. I'm that this kind of seem, probably he didn't do it, yeah.

Okay, maybe I should give some examples. Well, there was the solid example that I just described. So there was also solid ZT, which was also obtained by killing some endomorphism of, or requiring some endomorphism of P to go to an isomorphism, but that endomorphism of P was also just given by multiplication by some element with respect to the ring structure on P, so it's the same thing as killing the cofiber, just like this. There's also kind of pure algebra examples. So for example, if you take the usual derived category D of R, and then you look at R mod f, so that's called D of R, then what is D? Well, D is D of R1 over F, so it's kind of inverting f in some algebraic sense, is an example of this. Another thing you could do is call D of R, then you could take this algebra instead, R bracket f inverse in D of R, then what is D? D is the sort of like f-complete derived subcategory, and we're looking for a formula for the derived F completion.

Okay, so let's define a functor F from C to C by the following. F of X is the internal Hom from C to X, and there's a natural transformation from X to F of X because this is the same thing as internal Hom from 1 to X, and by construction, C maps to 1, so you get a map in the other way on the internal Hom. I want to claim that this is a first step towards constructing the required localization. Claim: if M is in D, then applying P to M to this map is an isomorphism. So Homing to M living in D doesn't see the difference between X and F of X.

Okay, so for the proof, it suffices to see that the fiber of X mapping to F of X is an A-module. The reason for that is to show that Homing out of this map to M being an isomorphism is the same thing as saying that the Homs from the fiber to M should be zero. But by definition, internal Homs from anything in A to M is zero. Therefore, if it's an A-module, then it's actually a unital---maybe I should say it's a retract of A tensor

And one should actually potentially be a bit careful about which map one writes down here. What I want to do is I want to take $F$ applied to the previous map, as opposed to taking the instance of the previous map with $X$ replaced by $F(x)$.

Okay, and then you can continue on like this: $F(f(f(x)))$. And then I claim that, so let's define $F_\infty(x)$ to be the colimit of this sequence. Then I claim that, if either (1) this colimit stabilizes for all $X$, so if for example every map from here onward is an isomorphism, or (2) the internal $\text{Hom}$ functor from $C$ to $C$ commutes with colimits, or really maybe only needs sequential colimits, then $X \mapsto F_\infty(X)$ is the left adjoint to the inclusion of $D$ inside $C$.

For the proof, you have to check two things. One, you have to check that if you map from $X$ to anything in $D$, that's the same thing as mapping from $F_\infty(X)$ to anything in $D$. But we just proved that claim for $X$ going to $f(X)$, and this thing is given by hitting an instance of that map with the functor $F$. But the functor $F$ is going to preserve the property used in the proof here that the fiber is an $A$-module, because internal $\text{Hom}$ to an $A$-module will still be an $A$-module. So the exact same argument shows that each of these maps also satisfies the same property that internal $\text{Hom}$ out of them doesn't see the difference between one guy and the next. And then you have a colimit, and internal $\text{Hom}$-ing out of that is just an inverse limit of internal $\text{Hom}$-ing out of all the other ones. So that formally just passes through to the colimit.

Okay, so in principle, for example, you can just you could at least attempt to use this formula to compute solidification. It's all actually quite explicit. In that case, this $\text{Hom}$-ing out of $C$ is just...

It's also fun to try to unwind these cases and see that you recover the classical formulas. For example, the well-known formula for $M_1/F$ can be rewritten as a sequential colimit, which is exactly the same as this sequential colimit formula. 

Here, it looks different because it's an inverse limit, but what's actually happening is that this inverse limit is just the first iteration of $f(x)$ applied to $M$, and all the maps after that are isomorphisms. So it's a way to understand the left adjoints using this formula.

I should also mention that this situation occurs when $a$ is idempotent. But this is not the case here, so taking the completion would not give you the same result. You'd be doing something else, not just algebraically inverting $F$. 

I went a little over time, so I apologize for that. Thank you, and see you next week.
\end{unfinished}
% !TeX root = AnalyticStacks.tex

\section{\ufs Gaseous modules (Scholze)}

\url{https://www.youtube.com/watch?v=krq6jCy-dhE&list=PLx5f8IelFRgGmu6gmL-Kf_Rl_6Mm7juZO}
\renewcommand{\yt}[2]{\href{https://www.youtube.com/watch?v=krq6jCy-dhE&list=PLx5f8IelFRgGmu6gmL-Kf_Rl_6Mm7juZO&t=#1}{#2}}
\vspace{1em}

\begin{unfinished}{0:00}
e
so  last  Friday  I  was  uh  discussing  the
the  paper  look  the  curve  and  uh  I  was  um
I  was  in  some  sense  already  introducing
a  new  kind  of  analytic  rink  structure
that  we  called  the  guess  analytic  R
structure  and  so  today  I  wanted  to  pick
up  there  um  and  uh  but  some  to  analiz
this  guess  and  on  the  GRE  structure  uh  I
will  need  to  use  some  of  the  results  at
Dustin  um  explained  on
Wednesday
okay
mod  and  okay  so  so  the  plan  roughly  for
today  is  uh  to  First  say  something  about
um
guess  I  call  the  guest  dis
spacement  which  is  a  certain  certain
version  of  arithmetic  long
series  some  in  some  sense  some  kind  of
suffering  of  the  wrong  Series  in  some
variable  that  I  was  s  called  Q  last  time
um
then  actually  want  to  take  a  little  bit
of  time  to  talk  about  the  corresponding
Ser  of  the  real  numbers  guess
real  and  finally  I  want  to  like  the
motivation  for  this  was  uh  coming  from
ptic  curve  and  last  time  I  someone
claimed  that  this
G  structure  would  be  good  for  that  and  I
want  to  start  a  little  bit  explanation
for  that
so  I  want  to  say  just  a  few  words  uh
already  now  about  the
construction  of  the  table  of  the
C
um  but  at  that  point  we'll  have  to  do
some  more  serious  analytic  geometry  and
I  think  that  will  be  a  point  where  we
then  we  start  to  develop  more  a  general
formalism  to  really  carry  that
out  all  right  um  okay  so  let's  start
with  gu  space
l
over  here
um  so  we  call  it  the
motivation  uh  was  to  find
the
minimal
a  triangle
a  um  over  which  can  find  bit
AER  necessarily  bit  imprecise  up  the  r
take  Li  the  curve  there's  an  example  of
Li  the  curve  and  the  universal  one  would
be  the  the  Mod  Space  will  so  this  is  not
what  we  want  to  do  we  really  want
something  that  is  some  of  the  form  DM  Q
to  the
Z  in  some
antic  depending  on
some  um  and  so
the  or
following
that  there  should  be  this  element  Q  This
should  topologic
unit  in
the
one  and  the  second  the  r  that  I
mentioned  and  I  think  is  quite  natural
but  where  I  didn't  so  far  didn't  really
explain  and  maybe  I  will  get  to  the  to
the  what  the  end  of  today  why  it's
really  the  good  condition  is
that  uh  if  p  is  a
free  a
module  on  a  n
sequence  let  me
it
let  um  so  p  is  a  free  uh  light  condenser
being  D  on  a  n  sequence  um
then  1us  Q  *  the
operator  acting  on  P
Tender
Is  So  This  is  similar  to  how  we  Define
the  solid  analytic  R  structure  where  we
so  head  Q  to  one  instead  and  of  course
then  we  didn't  ask  it
one
um  uh  I  mean  this  is  some  weaker
condition
because  saying
that  for  no
sequence  a  n  and
a  we  stand
for  the  sum  of  a
n  right  so  if  you  think  about  what  the
inverse  of  this  would  be  then  be  one
plus  two  time  shift  plus  sare  time  shift
and  so  on  particular  Zer  coordinate
evaluate  such  we  can  form  some  kind  of
geometric  series  uh  in  Cube  uh  with  the
coefficients  from  a  no
sequence  and  we  certainly  I  mean  we
certainly  want  some  condition  here  right
because  if  you  didn't  ask  any  kind  of
completen  this
condition  then  we  would  just  work  with
all  a  modes  it  wouldn't  be  completed  in
any  sense  but  we  do  want  to  to  able  to
form  some  kind  of  infinite
sering  all  right  and  so  something  that  I
said  last  time  was  just  clear  but  maybe
I  should  really  say  a  few  more  words
about  this  um  is  that  there  is  certainly
an  initial  analytic  ring  uh
prop
and
um  you  may  or  may  not  actually  work  with
animators  or  infinity  or  whatever  Rings
uh  in  the  setup
um  the  proof  also  shows  that  there's
initial  animated  such  uh  later  on  we
will  I  will  actually  compute  the  initial
animated  one  and  show  that  it's  actually
con  Zer  and  so  then  agre  with  initial  I
can  reallying  the  previous  sense
was
let  me  introduce  an  notation  so  let  me
denote  the  free  free  condens  strings  on
a  topologically  ne  topologic  ne  poent
element  by  Z  ad  joint  Q  head  so  this  is
defined  to  be  Z  joint  Q
infinity  infinity  equal  to  zero
um  so  actually  this  is  just  uh  P  right
because  I  mean  it's  just  a  condensing
being  group  I'm  just  thinking  Z  join  any
Infinity  mod  Infinity  which  is  precisely
my  definition  of  p  but  then  I'm  ending
it  with  a  spring  structure  where  I'm
interpreting  this  n  here  is  small  the
powers  of  some  some
variable  but  I  want  to  keep  T  sort  of  a
module  separate  from  from  this  kind  of
ring
so  so  then  we
certainly  one  corresponds  to  a
map  from  this  Z  joint  you  head  and
then  and  let  me  call  this  thing
here  a  triangle
zero
so  we  want  one  then  we
want  a  triangle  to  to  come  with  such  an
M  right  because  it's  a  free  free
condensed  ring  uh  equipped  with  a
topolog  NE  poent  unit  this  a  fre  guy
with  topolog  NE  poent  element  and
this  okay  so  then  we  can
form  uh  stre  all  sequence  and  sender  it
up
to  this  guy
and  we  have  already  here  we  have  oneus
2  and  then  we  can  Define  certain  for
subcategory  um  of  all  the  modules
over  the
string  uh  like  we  want  this  to  become  a
nice  LM  so  in  other  words  whenever  we
map  this  to  something  over  a  module  over
analytic  ring  this  should  become  anism
so  we  can  simply  declare  this  to  be  all
those
modules  over  a  z
triangles
such  um  1-  2
*  acting  on  the  internal
H  um
from
p
uh  this  is  extremely  analogous  to  how  we
defined  solid  modules
right  the  ring  that  was  just  the  iners
and  then  we  ask  a  similar  condition  for
qal  to
one
and
uh  because  thisal  home  from  P  such
excellent  property
um  it's  immed  to  check  and  this  is  what
I  did  for  solid  modules  um
that  uh  the  subcategory  has  all  the
stability  properties  that  we  ask  for  for
antic  gr  so
then  um  the  fair  triangle  mod
a  is  some  what  we  call  nothing  called
last  time  with  preanalytic  bra  so  kind
long  analytic  bra  so  it  doesn't  satisfy
I  me  satisfy  all  the  properties  of
analytic  ring  except  for  the  property
that  the  ring  itself  is
complete  the  base  ring  is  really  just
this  uncompleted  uh  thing  so  far  um  and
we  can  see  something
about
sh  but  then  properties  one  and  two  they
p  pris  mean  that
analytic  over
analytic
and  so  now  we  just  use  uh  what  do  pro  on
Wednesday  that  uh  the  inclusion  from  of
analytic  ring  to  panal  rings  has  left
joint  so  you  can  always  complete  such  a
panal
and
so  yeah  s  also  explained  last  time  it's
actually  better  to  uh  when  you  have  a
general  suching  to  complete  it  in  the
sense  of  animat  antic  ranks  um  because  I
prior  the
completion  uh  of  of  the  unit  in  this  me
the  drive  completion  might  not  SRE  zero
and  then  the  better  object  to  consider
is  the  drive  completion  in  this  case  it
will  turn  out  to  be  the  case  that  this
drive  completion  actually  sits  in  degre
Z  so  there  will  not  be  a
difference  uh  so  that's  one  reason  that
uh  we  someh  switch  in  this  discussion
about  completion  of  analytic  structures
um  the  reason  for  the  final  stretch  of
the  talk  last  time  was  that  we  now  want
to
compute  uh  this  analytic  ring  structure
and  for  this  DUS  gave  some  general
recipe  for  computing  uh  this  comption
factor  in  some  some  cases
so  want  to
compute
um
um
okay  so  let  me  first  give  give  um  give
some  formula  uh  for  what  this  comp  looks
like  I  already  stated  it
uh  very  briefly  last  time  but  let  me
deliberate  okay
so  uh  the
following
grab  the  initial  analytic
animat
uming  is
um  but  it  turns  out
to  look  can  get  form
us
yeah  so  first  of  all  a  triangle  but  also
all  the  pre
modules  all  the  three  a  modules  on  live
Prof  sets  uh  they  sit  in  degre  zero
so  they  have  no  hom
Z  and  there  are  also  CA
separation  so  we're  not  describing
some  uh  secondly  this  a  TR
uh  it  is  actually  a  sub  ring  of  the  long
Ser
R
um  let  me
first  gives  a  formula  for  the  underlying
gr  even  a  point  so  these  are  those
sums  2  to  the
n
such  the  absolute  value  of  n  i  mean
certainly  there  are  zero  for  for  very
negative
n  very  long  series  um  but  then  when  then
go  large  they  should  have  most
po
and  so  this  is  tying  things  back  to
discuss  T  of  the
curve  has  the  property  that  when  you
write  down  the  corresponding
y  property  um  and  then  he's  completing
it  to  get  this  anal  but  that
a  a
question  oh  I'm  sorry  I  didn't  know  we
were  miked  up  yeah  I  was  I  answered  it  I
think  so  um  happy  to  answer  it  for
everyone  it's  yeah  well  the  question  was
was  whether  the  um  whether  you're
discussing  a  specific  analytic  ring  here
or  some  general  situation  I'm  I'm  just
I'm  just  here  I'm  describing  of  course
the
initial  one  coming  from  the  position  the
inial
animator
okay  so  the  uh  right  so  I'm  taking  the
underlying  ring  here  in  two  levels  right
if  you  have  an  analytic  ring  that  first
of  all  has  an  underlying  condensed  ring
it  this  has  an  underlying  ring  so  this
is  this  one  um  and  now  if  I  also  want  to
describe  the  condensed  structure  then
here's  a  way  to  do  that  uh  let  me  try  to
write  big  so  that  there  are  no  questions
about  indices  um
uh  take  a  union  over  K  and
NS
um  so  basically  this  is  K  is  giving  me
the  the  polinomial  grow  and
like  Pol  of  degree  K  and  is  giving  me
the  order  of  the  pole  of  of  my  the  wrong
suit  broadly  speaking  and  so  then  I'm
taking  a  product  over  all  n  which  are  at
least  minus
n  uh  of  those  in  which  lie  in  the
interval
from  n  plus  n  to  the  k  n  plus  n  to  the
k  um  right  integer  in  these
bounds
times  so  so  each  of  I  mean  this  is
always  a  finite  set  so  this  product  here
is  is  a  like  Prof  finite  set  and  then
taking  the  Union
this  describes  this  as  a  light  or  as  a
light  set  first  of  all  but
then  so  subset  of  here  and  stable  under
the  ring  scure
here
the  way
what  I'm  not  sure  why  start  call  me
switch  to  my  usual  name  s
um  so  this  describes  this  thing  and  then
he  also  describes  the  free
modules  uh  so
for  and  live  profile  said  written  as  an
limit  of  fin  s
i  um
Jo  s  has  a  sar
form  uh  to  to  write  it  um  right  so  first
of  all  again  you  can  think  of  this  as
uh  like  one  way  to  say  it  if  it's  Liv  in
like  also  on  Z  wrong  series  Q  you  can
take  to  guide  on  S  and  you  can  solidify
that  so  something  we  understand  let  me
recall  this  is
just  limit
over
power  or  in  other  words  these  are  some
kind  of  Z  long  series  two  value  measures
on
this
proficient
um  so  it  sits  inside  there  um  so  in
particular  I  only  just  describe  it  as
a  cond
set  because  the  group  structure  will  be
the  one  on  the  target
um  and  so
uh  One
Way  think  about  this  is
that  well  he  have  an  element  here  some
kind  of  measure  which  is  a  sum  overall  M
and  V  of  some  measures  m  m  *  Q  to
M  um  and  then  again  I  asked  I  put  a
gross  condition
on  the
mm  um  so  first  of  all  actually  say  the
mm  they  are  actually  just  some  of  D
measures  so  for  any  fixed  power  of  Q  you
cannot  really  take  infinite  sum  here
this  really  must  just  be  sum  of  D
measures
this  uh  but  even  more  than  that  you
asked  uh  that  some  these  are  not  too
large  nor
sub  poal
grow  and  I  guess  I  mean  this  is  probably
rather  describing  on  line  like  not  the
condense  structure  so  let  mebe  the  cond
structure
now  uh  it's  very  similar  to  the  above  so
it's  again  Union  over
case  and
ends  uh  play  the  same  row  as
above  uh  but  now  we  also  need  to  squeeze
in  this  inverse  limit
overall  I  to  get  something  for  for
S  and  then  now  we  just  do  the  obvious
things  we  take  M  greater  equal  to  minus
n  of  uh  the  measures  on  Z  join  s  i  which
are  of  Norm  at  most  n  plus  m  to  the
K
time  right  so  here  you're  taking  allow
to  take  some  some  differences  of  at  most
M  plus  m  to  the  K  basis
elements  so  if  he  did  this  when  s  was  I
was  just  in  one  element  would  be  down  to
do  that  thing  there  um  so  this  is  just
some  finite  set  uh  this  was  some  like
Prof  finite  set  if  it  takd  still  set
take
you  uh  Peter  I  wonder  if  might  be  a  bit
clearer  to  just  like  drop  the  n  greater
than  zero  only  like  only  take  the  term  n
equals  z  and  then  just  invert  Q  on  the
outside  yeah  but  then  I  would  have  to  do
this  funny  thing  where  I  we  have  to
replace  n  plus  M  here  by  something  like
n  plus  two  in  order  for  the  zero
coefficients  be  allow  oh  yeah  to  be
allowed  to  goow  yeah  okay  yeah  all  right
and  then  it  looks  artificial  again  so
yeah  or  you  could  you  could  have  a
constant  C  in  front  like  polinomial
growth  yeah  well
anyway  yeah  I
was  okay  this  is  correct  and  you  can
certainly  find  many  other  ways  to  rise
this
in  and  I  thought  if  I'm  already  taking  a
union  then  inverting  Q  at  the
end  seems  like  an
operation  many  I  don't
know
a  matter  of  taste  I
guess  all
right  is  it  okay
um  so  I  don't  want  to  give  the  the  full
argument  here  but  I  do  want  to  provide  a
sketch  I  want  to  at  least  explain  where
the  Pol  normal  growth  condition  suddenly
appears  because  this  maybe  AOS
surprising  thing  because  when  we  Define
this  G  Gus  analytic  R
structure  we  didn't  think  of  any
specific  grow  conditions  at  all  right  we
were  just  asking  that  one  two  time  soon
is  convertible  and  then  we  wonder  what
what  we're  getting
um  and  this  is  certainly  also  how  we
what  happened  to  us  when  we  thought
about  this  business  so  we  just  well
let's  write  down  this  condition  it  seems
natural  and  see  see  what  we  get  uh  this
was  was  what  came
out  uh  so  so  the  way  to  compute  is  to
use  the
uh  formula  from  the  end  of  St
leure  so  there  there  was  a  formula  how
you  compute  some  kind  of  completion  uh
in  cases  where  the  competance  condition
is  given  by  uh  asking  that  the  home
inter  oral  derive  come  from  some  algebra
is  Zer  so
note  that
um
um  this  condition  that  this
interal  uh  is  equal  to  the  point  than
um  so
that's  uh  let  me  call  it  c  um  so  this  is
T  the  a
triangle  mod
oneus  but  now  I  actually  want  to  I  do
want  to  think  about  p  as  a  ring  suddenly
icept  I  didn't  want  to  do  that  and  I
want  to  keep  my  notation
separate  but  at  this  point  I  do  want  to
remember  that  P  has  a  as  a  admits  a  ring
structure
um  and  so  then  this  is
actually
um  it's  a  free  guy  on  the  top  logic  new
element  that  X  head  before  and  then  aad
was  also  a  three  new  variable
and  then  okay  so  I  had  tover
q  and  then  one  minus  I  mean  shift
operator  is  multiplication  by
X  by
one  so  this  is  actually  a  commut
r  uh  so
keep  uh  so  it's  equiv  to  the  F  condition
uh  using  this  P  so  the  condition  is  that
the
r  over  a
triangle  uh  from
B  right  because  this  R  home  is  literally
just  the  cone  of  the  homs  and  and  and
the  homes  are  in  degree  zero  because  of
these  excellent  properties  of
P
and  also  uh  this  B  has  a  property  that
internal  home  from  B  commes  with  all
limits  uh  I  mean  same  property  for
p
so  uh  this  was  one  of  the  conditions
under  which  uh  Dustin  explain  the
formula  for
comption  I  mean  the  r
completion  the  three  analytic
RS
a  z  triang
a  is  given
by  um  taking  a  module  M  which  is  in  the
drive  category  of  a
triangle  gra  your  condens  a  z  triangle
modules  um  and  taking
this  theal
colum  of  the  following
diagram
so  m  m  to  the  internal
AR
from  I
willot  the
fiber
Co
um  can  take  the  r  home  from  I
into  and  because  oops  this  fiber  this
naturally  maps  to  one  so  Dually  on  our
Homestead  met  from  M  this  and  then  there
was  some  last  time  about  how  to  per  the
next  NS  but  I  think  doesn't  really
matter  all  that  one  um
go  to  find  the  Alum  twice  which  is  of
course  just  the  Alum  from  the  tender
product  all  right  so  that's  some  kind  of
formula  and  uh  as  D  mentioned  last  time
I  mean  there  are  some  situations  where
it's  quite  a  useful  formula  for  example
if  you're  in  a  situation  where  you're
just  algebraically  inverting  some
element  F  so  then  this  would  be  some
bring  a  mode  F  then  the  is  usual  way
inverting  F  where  this  Lally  just  always
the  same  module  and  multiplication  by  F
and  so  in  the  column  get
M1
um  but  you  could  also  try  to  use  this
formula  to  compute  solid  confusion  uh
but  this  turns  out  to  be  extremely
confusing
and  yeah  maybe  one  thing  that  will  what
happens  that  really  only  the  co  liit
that's  well  behave  both  in  the  Sol  case
and  this  case  here
um  and  in  the  col  I  claimed  that  you  get
something  nice  and  degre  zero  qu
separator  and  so  on  but  each  of  the
individual  terms  will  actually  be
spread  far  out  and  have  some  non  some
higher  homology  which  probably  not
separate  and  so  so  the  individual  terms
are  a  bit  hard  to
control  um  right
so  uh
uh  anyways
so  so  how  do  we  actually  go  about
Computing  this  so  This
I  uh  I  mean  it's  built  from  one  and  B
one  is  not  so  hard  of  course  and  B  was
just  built  from  P  basically  by  this
conone  so  to  to  understand  what's
Happening  Here  the  key  is  to  understand
what
uh
um  uh  one  key  is  certainly  to
compute  the  internal
home  uh  from
pnal  home  from  B  CH  from  a
triangle  a  triangle  but
also  they
trying
right  so  basically  want  to  evaluate  this
formula  on  the  M  which  are  some  three
guys  on  this  and  then  see  what  we
get
and  so  here's  the
LI
um  maybe  make  let  me  make  a  remark  that
uh  you  don't  actually  have  to  invert  Q
in  any  of  this
um  we  can  proceed
without  inverting
F  and  then  you  you  will  have  similar  the
completion  will  basically  be  the  same  as
have  to  drop  all  the  terms  which  have
negative
positive  um
uh  and  because  yeah  slightly  easier  to
to  write  down  the  formul  if  I  don't  have
to  invert  it  would  just  extra  operation
have  to  do  it  every
um  so
then  if  I  don't  bre  you  then  this  is
really  just  the  free  di  to  new  element
which  is  just  P  so  we  one  thing  we  have
to  understand  is  some  what  is  internal
home  from  T  to  P  where  we  think  of  this
here
say  Z  joint  X  and  just
joint  and  uh  there's  a  formula  for
this  so
uh  so  this  these  three  condensed  to  be
groups  on  some  Pro  Set  they  always  given
as  this  Union  over  all  ends  of  like
things  which  have  bounded  Norm  end  and
so  this  will  some  pull  through  this
internal  home  and  so  this  will  also  be  a
union  Over  N  bigger  than
zero  where  now  I'm  somewh  taking  the
part  of  this  algebra  here  where  some  of
Norm  at  most  n  so  at  most
n  uh  terms  are  allowed  to
appear  but  then  when  I  do  that  uh  the
new  so  now  I'm  allowed  to  do  a
overall  n  which  is  like  the  power  of  Q
dividing
by  and  so  then  that's  so  you  take  V
joint
Q  not  Q  to  the
N  but  then  inside  this  you  take  sums  of
at  most  n  basis
elements  you  take  sums  *  Q  to  the  AI  *  Q
to  the  I  where  the  sum  of  the  absolute
values  of  the  AI  said  most
then  and  then  you  take  polinomial  in
X
um
which  not  sure  I  should  call  x  x  dual
but  never  mind
call
um  uh  but  it's  also  just  a  form  a
variable  so  this  is  the  part  of  poal
algebra  where  each  coefficient  satisfies
expound
okay  and  so  there's  a  similar
formula  uh
for  internal  home  from  P  which  again
think  of  joint
except
towards
uh  the  ad  joint  you
have
adjin
some  light  profile
sets  and  and  uh  it's  just  the  same  sort
of  we  take  Union  of  all  bigger  than
zero  and  take  an  invers  limit  over  all
NS  and  then  you  take  V  joint
Q  to  the
N  Jo  as
I  but  inside  there  again  this  part  that
some  of  at  most  and  some  are  difference
of  at  most  end  basis
elements
maybe  I  should  put  some
operates
um  and  then  I  take  a  pols  in  in  the  Dual
variable
um  uh  right  for  each  term  settis  is
done  all  right  so  Peter  just  a  point  of
information  about  uh  the  way  things  look
on  the  screen  here  so  if  you're  close  to
the  the  boundary  of  the  board  uh  like
far  right  of  the  right  hand  board  or  the
far  left  of  the  left  hand  board  it's
quite  a  shadow  and  it's  we  basically
don't  see
anything  it  said  still  okay  also  to  far
in  um  that's  better  that's  readable  with
some  effort
yeah  anything  limit  over  the  uh  uh  yeah
sure  yes  of
course  what  is  the  Stop  and  the
exponents  of  the
X  sorry  X  maybe  I  should  do  the  EXP  y
it's  it's  just  a  different
okay  I  didn't  want  call  X  because  it's
some  of  the  Dual  of  X
or
now  but  course  maybe  I  use  the  head  to
have  some  specific  meaning  here  the  star
was  didn't  have  any  meaning  that
s
uh  so  in
particular  uh  one  way  to  uh  a
consequence  of  this  set  this  internal
home  from  P
top  is  actually  just  the  Subspace
of
uh  Z  joint
Y  and  it's  a  Subspace  with  a  certain
kind  of  uh
funny  uh  funny  condition
that
um  s  this  um  so  some  kind  of  let  me
notes  as  far  as
why  power  series  and  cubes  bounded
because  in  some  sense  you're  asking
that  yeah  there's  some  B  boundedness  uh
in  terms  of  the  coefficients
of  wise  and  similarly
the
from  from  te  to  this
um  there  also  a  subring  of  U  you  take  Z
joint  uh  Y  and  then  you  take  s  and  then
you
take  and  again  it's  a  substate  of  this
where  each  coefficient  of  Y  uh  satisfies
a  certain  boundness  property  so  again  I
will  denote  this  here  by  the
join  y  join  s  power  series  and
Q  um  Peter  there's  been  a  request  to
explain  again  what  the  less  than  or
equal  to  N  means  in  the  description  of  h
p
top
uh  for
finite
that's
I  I  didn't  use  I  foring  here  right
um  the  join  I  less  equal  to  n  a  certain
sub  space  of
fre  three  free  on  I  and  set  of  those  Su
over  I  of  some  a  itimes
I  so  that  is  sum  of  the  absolute  vales
of
the  is  it
most
all  right  so  if  you  take  this  formula
for  the  internal  home  from  p  and  just  do
these  formal  manipulations  that  you
express  I  in  terms  of  the  other  thing
sorry  sped  way  too
high  um  if  you  just  do  this
manipulations
uh  then  you  for  example  see  that  the
Rome  uh  from  sorry  uh  I  said  I  didn't
invert  didn't  invert  Q  so  let  me
explicitly  say  just  for  Z  Jo  Q  head  and
okay  so  I  can  also  be  defined
corresponding  version  inverting  you  have
um  uh  so  if  you  compute
that  then  this  is  a  two-term
complex  um  but  those  terms  are  see
computer  up  there  it  should  take  this  uh
this  power
series  over  the  one  which  has  some  bound
coefficients
and  previously  there  was  on  transition
that  was  1  -  x  *  Q  uh  but  now  Y  is  some
kind  of  dual  object  X  so  now  actually
transition  turns  out  to  be  just  Q  minus
or  y  q  minus
one  and  so  this  sits  in  degree  zero  and
this  sits  in  homological  degree
one
and
uh  if  you  do  the  similar  computation
with  more
variables  hey  uh  s  effector
here
then  you  get  a  similar  picture  but  just
the  K  of  the  variables  y  so  this  is
computed  um  by  complex  in  uh  y  variable
so  inste  what  take  is  you  take  Z  and
then  you  join  y1  up  to
YK  and  then  again  you  take  such  b  power
Series  in
Q  and  then  I  mean  some  of  this  was  the
derived  potion  by  just  one  variable  tus
Y  and  now  you  take  a  derived  potion  by  Q
minus  y1  up  to
Q  corresponds  to  Tak
fil
and  I  mean  you  can  do  the  same  with  with
a  with  a  sets  in  right  so  if  you  have  a
profile  sets  somewhere  then  you  just
plug  the  profile  sets  here  in  the  middle
and  then  get  the  same
for
all  right  s  so  so  now  we  have  a  somewhat
concrete  formula  for  these  things  right
so  if  you  want  to  compute  the  underlying
ring  we  literally  just  have  to
understand  the  co  liit  of  all  K  of  these
things
so
thus  the  ring  we're  interested  in  is
just  the  co  liit  over
all  of  this
same  take  y  want  to  YK  if  you  found
it  then  you  take  a  d  quent  and  here
turns  out  to  be  a  case  where  it's
really  very  important  that  each
individual  level  to  Tak  the
D  full  CL  on  because  there  will  be  high
homology
then  here  plug
s  what  are  the  transition  Maps  a
transition  sorry  yeah  I  should  have  said
so  this  is  similar  question  to  what  the
transition  mat  in  call  that  D  wrote  down
last  time  um  you  just  I  mean  each  of
these  is  a  subing  of  the  next  one  where
you  just  don't  have  the  new  variable
it's  not  a  sorry  all  these  things  are
actually  not  range  because  if  you
multiply  two  things
uh
what  confusion  but  um  let  me  not  try  to
say  whether  they  r  or  not  but  uh
certainly  there  sub
modules
all  right  um  so  so  far  there  was  no
apparent  polom  grow  condition  at  all
there  was  just  some  kind  of  boundness
that  appeared  for  more  transparent
reasons  but
now  uh  let's
analyze  uh  like  the  of  this  or  rather
separate
Co
so  I  mean  this  for  Simplicity  you  can  do
the  full  analysis  but
uh  it  takes  a  lot  of  concentration  so  I
think  there's  already  enough  interesting
stuff  going  on  without  these  addictional
technicalities
um
so  right  so  so  what  is  the  H  zero  so  you
really  just  take  this  ring  and  then  what
out  by  Q  minus  y1  of  the  Q  minus  y  k  or
if  it  takes  a  separate  equion  really  by
the  the  closure  of  all  these  things  yeah
um  so  in  other  words  here  I'm  are
formally  allowed  to  uh  replace  any
recurrence  of  any  of  these  variables  by
I  that  it's  really  just  set  into  a  q
right
um  so  this
uh
there  a  question
com  what  is  the
image
of
uh  this
thing  uh  mapping  towards  power
Sol  where  all  the
Yi  so  let  me  again  recall  what  this  was
so  this  was  the  union  over  all  end
of  of  the
limit  over
M
of  the  part  of  this  thing  which  was  of
Norm  at  most
n  so  this  wasn't  literally  written  as  a
finite  free  module  but  I  mean  you  take
the  basis  by  giving  my  1  Q  of  to  Q  to
one  uh  and  then  you  pend  the  variables
by
one  but  in  particular  it  will  actually
be  sufficient  to  analyze  the  part  where
all  of  these  are  say  just  equal  to  one
uh  Z  or  one  this  one  um  so  here  we  just
allowed  to  take  some  kind  of  bounded  sum
but  then  we  have  all  these  y  ones
through  y  k  here  at  the  posal  but  for
each  of  them  individually  I  can
take  I  can  take
a  I  mean  just  each  coefficient
individually  is  bounded  so  for  example
if  I  like  for  each  end  we  have  to
analyze  what  the  image
is  but  then  which  which  multiples  of  Q
which  like  integer  times  Q  can  I
achieve  well  this  integer  time  Q  like
for  each  occurrence  of  Q  I  can  decide
whether  I  want  to  keep  it  a  q  or  whether
I  make  it  a  y12
YK  but  for  each  y1  y  k  i  can  Som  use  it
most  end
times  but  then  in  total  like  The  K  uh  I
can  I  can  get  this  variable  q  k  n  k  *  n
times  because  yeah
there
um
uh  and  then  actually  maybe  the  nend  is
rather  red  pairing  so  let's  ignore  the  N
so  if  n  was  equal  to  one  uh  then  we  can
some  get  a  coefficient  up  to  K  here  for
this  Q  because  I  can  split  each
occurrence  of  Q  between  the  different
variables  and  then  if  I  look  at  q
squ  then  I  have  all  the  monomers  of
degree  2  hereos  in  the  y1  YK  so  there
like  k/2  mon  have  at  disposal  each  one  I
can  use  like  at  most  n
times  so  I  get  n  *  K  two  and  then  in
degree  some  big  m  i  sorry  it's  K  plus
one  I  guess  I  have  K  plus  M  over  choices
for  for  these  monomials  here
uh
so
the
coefficient  of  Q  to  the
m  can  lie
in  likeus  n
*  uh  n  plus  k
k  actually
same  and  maybe  I  I  didn't  uh
okay  I  would  have  to  be  slightly  more
careful  here  the  precise  noise  but  I
mean  basically  something  like  that  um
and  so  the  question  is  how  does  this
grow  how  does  this  grow  in  m  and  in  M
this  is  precisely  polom  of  degree  K
right  and  then  this  just  the  polom
function  of
the
yeah  so  whenever  I  have  something  that
grows  in  where  the  coefficients  grow  in
m  just  like  poom  then  because  yeah  I
have  all  these  monomials  here  at
disposal  I  can  somewh  split  up  all  the
individual  summons
uh  to  make  them  bounded  in
here  and  and  there's  c  a  lot  of  choices
about  how  you  go  about  doing  this  and
when  you  want  to  prove  that  actually
this  uh  this  complex  end  up  ends  up
being  in  degree  zero  you  Som  have  to
show  that  a  different  way  of  arranging
the  terms  here  uh  is
actually  uh  like  differs  from  the  other
one  uh  by  uh  something  bounded  some
bounded  some  in  the  next  C  of  the
complex  and  this  will  actually  not  be
true  for  giv  K  but  when  you  allow  your
to  increase  your  k  a  little  bit  then  you
can  do
it
Peter  there's  a  question  in  the  chat
there's  a  question  in  the  chat  yeah
which  is  uh  or  you  want  to  read  it  or  uh
oh  yeah  so  there  was  a  question  about
this  Z  join  y  join  s  whether  I  can
interchange  the  order  and  indeed  you  can
interchange  the
order
so  the  kind  of  argument  you  have  to  do
here  to  see  that  this  is  well  behave
this  a  little  reminiscent  of  the
arguments  you  have  to  prove  that  liquid
stres  well  behav  but  it's  several  orders
of  magnitude  easier
right
so  it's  again  a  situation  where  you  have
uh  some  kind  of  system  of  complexes
where  each  term  has  some  kind  of  norms
and  then  you  have  to  like  see  that  if
you  have  something  with  differential
zero  you  can  bounded  pre  image  but  here
it's  really  you  can  really  just  do
it
I  takes  some  concentration  but  it's  not
all
all  right  all  right  so  um  this  is  all  I
wanted  to  to  say  about  uh  the  proof  of
this  box
ination
um  uh  now  they  I  want  to  talk  just
briefly  about  Gaz  G  real  Vector
spaces
so  let
me  condens  or
light  we  started  emitting  the  light  at  a
couple  of  times  so  throughout  this
course  we're  always  working  light
filling  and  it  might  not  always  be
mentioned
yes  if  I  take
the  internal
home  from  like  P  was  always  ining  the
free  with
integers
here
and  then  I  choose  one  top  poent  unit  in
the  real
say  so  now  you  might  actually  Wonder  uh
doesn't  matter  that  it  shows  a  half  here
and  it  doesn't  so  I  mean  so  in  this  Ser
of  liquid  analytic  of  this  liquid  drink
structures  on  the  reals  there  was  this
extra  parameter  that  you  have  to  choose
your  how  non  convex  you  allow  your
vector  spaces  to  be  um  here  it  seems  you
also  have  to  make  a  choice  but  actually
it  doesn't  matter  the  condition  is
independent
of  number  like  a
half  you  could  Al  take  a
negative  any
top
um  actually
why  so  it's  easy  to  see  and  uh
intuitively  believable  that  if  he  makes
this  number  smaller  then  the
conditioning  becomes
weaker  uh  because  here  some  of  the
series  you're  trying  to  to  to  sum  there
like  even  faster  Decay  but  on  the  other
hand  if  you're  trying  to  sum  a  geometric
Series  where  the  coefficients  are  like
one  over  two  to  the  end  then  you  can
always  um  the  finite  sum  where  you  Su
the  first  few  terms  and
then  have  telescoped  uh  not  really
telescope  but
um
there's  another  question  in  the  chat  I  I
actually  missed  it  uh  so  I  can't  read  it
um  let  me  see  if  I  can  read  by  the  way
general  question  nonl  condens  sets  will
not  be  used  anymore  in  the  Ser
originally  right  so  there  was  a  question
uh  like  originally  extreme  disconnected
set  play  quite  of  a  crucial  role  how  set
up  the  Ser  because  gave  us  enough
projective  objects  and  it  is  really
quite  convenient  to  have
um  so  for  much  of  what  we're  doing  we're
kind  of  forgetting  about  the  full
category  of  condensed  sets  but  on  the
other  hand  for  some  arguments  actually
is  still  quite  convenient  to  have  the
ambient  framework  of  all  condensed  sets
because  I  mean  the  light  ones  they  still
embed  into  all  condensed  sets  and  for
some  so  if  you  want  to  check  that
something  is  an  isomorphism  you  can
still  do  that  by  checking  the  values  and
extremely  disconnected  and  this  is  still
convenient  for  a  few  arguments  um
so
yeah  yeah  for  and  the  other  setting  is
also  nice  but  like  for  for  the  specific
application  analytic  geometry  we  found
it  more  convenient  to  make  this
restricture  because  there  are  a  couple
of  things  in  particular  this  internal
projectivity  of  P  that  we're  using  all
over  the  place  that's  only  from  this
uh  maybe  related  questions  like  can  you
Al  like  the  Ser  of  liquid  real  Vector
space  that  also  worked  not  just  in  liing
but  the  full  setting  you  can  ask  the
same  question  about  the  gasius  anding
structure  whether  you  can  also  Define
that  on  all  condens  modules  I  think  you
can  but  the  argument  we  gave  here  does
not  work  because  we  use  internal
projectivity  of  P  so  you  would  have  to
argue  slightly  more  carefully  to  see
that  there  is  a  gr  structure  of  G  real
Vector  spaces  in  the  full  condens
setting
um
uh  generally  is
onlyus  yeah  and  this  is  basically  a
telescopic
argument
all  right  I  mean  if  if  she  wanted  some
the
you  can  always  rewrite  like  at  a  finite
sum  and  then  new  sum  where  you  still
have  a  no  sequence  of  coefficients  and
now  powers  of  Q  to  the  end
here  uh  all  right  so  you  you
have  let's  say  um
and  so  anything  that's  P  liquid  for  any
P  or  Alpha  liquid  for  any  Alpha  uh  is
always  cases  so  this  is  extremely
General  class  of  condensed  Vector  spaces
uh  but  it's  still  a  workable  class  for
many
purposes  um  and
so  uh  you  can  now  wonder  what  are  the
three  uh  guesses  real  Vector
spaces  and
here
so  first  of  all  so  here  here's  a
proposition  what  does  it  have  to  do  uh
with  uh  what  we  had  previously
so  our  guess  which  notes  the  analytic  R
uh  corresponding  to  gu  is  Vector
spaces  so  again  it's  clear  that  this
defines  structure  because  of  this  good
properties  of  P  um  this  just  gotten  by
taking  this  one  we  had
is
makes  a  completed
one  and  the  g  mod  over
this  and  then  in  some  moding  out  by
oneus  2
*  setting  Q  equal  to  a
half  you  take  this  one  you  complete  that
antic  as  we  did  and  then  you  set  Q
so  it's  easy  to  see  that  when  you  do
this  to  this  uh
ring
uh  Point  number  gross  condition  that
you're  getting  you  actually  get  the  real
numb  H  you  don't
all  you  don't
all  yeah  no  I  mean  like  if  if  you  take
this  completed  ring  and  one  by  oneus  PQ
you  really  get  the
real
I  so  these  are  like  functions  on  like
like  the  real  part  of  this  thing  is  like
a  function  open  unit  the  real  numbers
and  then  if  you  set  P  equal  to  a  half
there  like  this  one  dimensional  thing
you  just  get  real
numbers  in  particular  um  this  tells  you
what  the  the  free  gas  miners  are  you  can
compute  them  the  free  G  your  vector
Spaces  by  taking  the  fre  guys  here  then
taking  the
poent  uh  but  now  the  description  becomes
a  bit  confusing  because  when  we  wrote
down  the  gross  conditions  that  we  allow
here  it  was  some  kind  of  Pol  normal  grow
condition  in  q  but  now  Q  has  become  a
constant  so  it's  not  immediately  C  clear
how  you  phrase  these  Rose  conditions  uh
directly  on  the  real  numbers  uh  but  you
can  do  it  and  here's  the
description  so  the  free  guess  is  uh  real
vectors  space  on  some  light
profs  uh  has  the  following  crazy
description  we  take  a  union  over  again
some  case  which  corresponding  to  this
poal  draws  that  we  had  previously  and
now  there  is  a
constant  then  I  take  a  limit  over  I
and  then  I  take  a  certain  bounded
subset  um  in  the  three  real  Vector  space
on  the  fin  set  s
i  uh  which  satisfy  some  condition  on  how
large  I  all  to  be  and  the  condition  is
that  if  you
sum  um  Su  the  size  and  measure  it  or
their  size  measure  it  in  terms  of
certain  functions  FK  that  I  will
introduce  in  a  second  there  at  most
C
so  formally  this  is  very  similar  to  what
the  three  liquid  Vector  space  look  like
and  there  the  the  Norms  you  were  taking
there  were  like  you  were  raising  to  some
power  also  or  maybe
Al
um  generally  the  functions  that  you
would  like  to  put  there  um  they  should
be  increasing  of  course  because  larger
larger  exes  should  be  calized  more  um  on
the  other  hand  you  want  the  transition
apps  to  be  well  defined  and  for  this  you
need  the  functions  to  be  uh  concave
turns  out  um
so  well  this  could  be
reasonable  um  these  to  FK  they  should
yeah  they  should  measure  how  lot  to  real
number  is  in  terms  of
some  let  it  some  real  number  greater
equal  to  zero
um  uh  I  mean  they  should  be
increasing  and
conct
so  basically  you  need  that  FK  of  X  Plus
y  should  be  at  most  FK  of
X  Plus  FK  of
Y  she  wants  a  transition  Ms  there  to  be
well  defined  because  the  transition  maps
are  somehow  summing  some  of  these  X's  um
and  this  is  Ed  by
theity  uh
but  on  the  other  hand  you  don't  really
care  what  your  function  is  for  large
values  of  of  of  X  because  uh  really  what
you're  trying  to  think  out  is  just  which
infinite  sums  are  allowed  and  of  course
in  these  infinite  sums  almost  all  the
terms  should  be  very  very  small  so  when
I  choose  such  a  function  FK  really  only
the  only  thing  that  matters  is  the
behavior  near
zero
okay  this  is  a  Prelude  to  saying  that  uh
so
FK  is
any  uh  increase  in  comp
functions
such  that  for  small
X  um  it's  given  by  the  following
it's  uh  the  absolute  value  of  log  X  to
Theus
K  so  I'm  saying  it's  this  way  because  I
mean  in  principle  I  just  like  to  take
this  function  and  in  the  neighborhood  of
zero  that  is  a  nice  increasing  concave
function  but  that  at  some  point  uh  it's
becomes  convex  goes  to  infinity  and  then
does  some  nonsense  because  like  if  you
said  x  equal  to  one  log  of  one  is  zero
one  to  the  minus  K  is  some  nonsense  so
um  so  I  can  write  down  this  function  for
all  X  but  I  can  do  it  for  small  X  and
then  at  some  point  I  just  aritar  extend
and  it  doesn't  do
me
what
all  right  so  this  is  maybe  a  little  bit
hard  to  just  so  let  me  let  me  give  a
sort  of
diagram  uh  so  in  the  liquid  in  the
liquid
story  uh  there  this
function  have  like  some  the  of  x  to  the
beta  where  beta  is  some  number  between  Z
and
one  so  there  some  function  so  this  is
always  increasing  in  concave  so  I  can
just  always  use  it  so  some  function  like
that
um  and
then  the  locus  where  the  sum  of  the  X  to
the
beta  um  in  the  two  variable  cases  will
be  some  some  kind  of  lus  like
that  some  sling  on
convexing  uh  in  the  G
case
um  it's  some  function  that's  extremely
steeply  ascending
near  near  uh  zero  so  even  if  it's  just  a
tiny  number  it's  often  treated  as  rather
large  by  this  by  the  function  FK  then
then  the  actual
function  that  I  wrote  there  would  make  a
turn  and  then  go  up  and  okay  instead  I
just
cut  I  can  make  it  linear  if  I  want  to
right  need  not  be
strictly
um  and  then  if  I  look  at  the  set  where
the  see  this  will  similarly  be  something
that  extremely  closely
tied  uh  to  to  the  coordinate  X
so  you're  allowed  to  take  certain
infinite  sums  but  basically  you  have  to
ensure  that  the  coefficients  have
exponentially
P  but  not  quite  that  actually
so
from  the  liquid
the  we  can
form
some  of  some  and
and  there  exist  some  detail  some
prespecified
Al  such  that  the
sum  of  the  X  the
BET  finite  so
there  in  principle  we  like  to  just  ask
usual  stability  but  as  I  said  this
doesn't  really  work  but  when
you  slightly  restrict  the  class  of  sums
for  which  they  ask  us  that  they  are
always  like  part  of  the  structure  ofid
re  Vector  space  to  be  the  speed  want  and
this
works  uh  in  the  G  the  you  have  much  more
spring
what's  uh  so  yeah  the  alha  liquid  Series
so  the  liquid  Ser  depends  on  exra
parameter  uh  Cas
Theory  can  only  form  those
sums  or  theic  say  but  the
some  ni  um  and  this  actually  two  one  to
say
xn  up  to
reordering  if  you  sum  in  the  order
starting  with  the  largest  ones  and
descending  uh  then  the
xn  F
uh  one/
xn  sorry  the  should  have  prob  the
exponential  growth  by  which  I  mean  the
following
so  you  can  form  such  a  su  only  if  exist
someon  greater  than  equals  to  zero  and
maybe  some  constant  C  greater  than  zero
such  that  the  abute  value  of  xn  is  at
most  c
times  a  half  to  the  end  to  the
absolute  right  so  if  I  didn't  put  the
EPS  on  here  then  this  would  be  some  kind
of  exponential  decay
uh  uh  it's  enough  to  have  some
fractional  power  of  n  here  in  the
exponent  um  and  if  they  satisfy  some
Decay  condition  like  this  then  then
you're  allow  to  S
guesses  so  why  does  this  expression
appear  here  well  if  you  take  a  half  to
the  end  to  the  Epsilon  and  then  apply
this  kind  of  procedure  here  then  the  log
turns  this  into  to  the  Epsilon  and  then
to  the  minus  K  you  get  one  over  n  to  the
Epsilon  *  K  uh  and  this  becomes  summable
when  I  don't  know  Epsilon  *  K  becomes
very  all  right  so  some  rather  crazy  type
of  grow  conditions  from  get  was  real
numbers  but  it's  still  a  workable  thing
because
um  like  do  agation  CA  on  complex
geometry
and  uh  at  least  all  the  all  the  stuff  in
complex  geometry  would  work  as  well  as
Gus  Vector  spaces  because  the  only  thing
that  we  really  needed  that  this  algebras
of  conver  phoric
functions
um  where  some  kind  of  thing  in  the
theory  and  some  nuclear  uh  and  this  is
still  true  uh  in  this
story  because  the  transition  Maps  when
you
form  like  when  you  restrict  all
functions  on  one  this  all  function  on
small  this  then  you're  precisely  if  you
think  in  terms  of  their  power  series
expansions  you're  rescaling  all  the
power  series  coefficients  by  but  fixed
call  of  the  number  less  than  one  so
actually  the  transition  map  they  have
exponential
decay
so  yeah  so  I  think  the
guest  uh  story  would  be  just  as  good  for
this  discussion  of  complex
geometry  all  right
um  yeah
so  yeah  so  the  claim  is  that  you  can  do
this  computation  of  the  three  gas
elector  spaces  uh  forly  from  the  one  I
did  over  this
uhing  in  the  variable
Q  takes  a  little  bit  of  unraveling  but
it's  not  it's  not  all  that
hard  all  right  so  uh  finally  let  me  try
to  do  some
geometry
yes  so  the  goal  is  to
define
the  GU  is
uh  okay  let  me  again
recorded
and  so  the  idea  is  something  that  we
already  used  a  couple  of
times  we  have  this
fixed  to
and  like  an  add  B  some  use  that  when  a
fixed  to  unit  you  can  use  this  to
normalize  absolute  values  uh  so  that
some  of  when  you  want  to  understand  how
large  some  other  function  is  uh  you  can
see  some  how  large  this  compared  to  to  Q
um  uh  so  we  can  use
Q
measure  the  size  of  any  other
function  so  basically  you're  you're
wondering
like  how  do  powers  of  the  absolute  value
of  uh  F  from  comp  to  PO
of
um  and  uh  one  way  to  organize  this
information  um  is  in  the  following
place
and  now  I  will  start  to  use  this  formal
of  analytics  STX  uh  that  this  whole
course  like  the  main  aim  of  introducing
um  and  this  language  is  really
convenient  to  phrase  before  play  because
I  could  give  a  more  downt  version  of
this  but  it  would  be  much  more  confusing
um
and  I  think  it's  a  really  neque  way  of
packaging  the  information  I  will  maybe
explain  a  little  bit  about  those  so  the
exist  the
morm
T
um
from  let's  say  first  from  the  f  l  over  a
um  and  on  the  fine  line  you  have  the
coordinate
function  in  particular  like  when  you
want  to  do  something  for  any  function
you  can  something  do  for  the  universal
function  which  is  the  function  t  on  an
A1  right  so  when  you  have  any  function
on  the  a  then  this  gives  you  M
from  the  spectrum  of  A2  with  A1  let  your
function  T  to  your  given  one  so  to
define  something  for  any  function  is
enough  to  do  it  for  an  A1  Anders  Cas  U
and  so  then  there  is  some  kind  of
absolute  value  of  T
function
measuring  measuring  against  the  ABS
value  of  Q  uh  that  goes  to  well  in  this
case  it  can  actually  be
infinite
uh
because  some  could  be  some  kind  of
unrounded  function  maybe  a  better  way  to
think  about  that  you  have  one  on  the
projected
line  and  if  you  have  the
point  this  homogeneous  coordinates  he
want  to2  then  this  goes
abute
implicit  here  is  that  something  is  down
to  earth  as  this  closed  interval  uh  will
have  a  certain  very  crazy  Incarnation
and  an  antic  St  in  all  setup  and  uh  this
is  required  to  make  S  of  this  mths
um  so  simpit  here  again  is  again  so  if
you  really  want  to  go  to  zero  Infinity
here
uh
um  I  again  need  to  normalize  the
absolute  value  of  Q  and  so  I  what  I
always  do  is  to
let  but  I'm  me  in  my  last  lecture  it
kind  of  suddenly  mattered  at  one  point
uh  what  I  show  there  but  here  it  doesn't
matter  again  and  always  just
resale  but  let's  assume  we're  in  a
formalism  where  something  like  this
makes  sense  where  we  have  some  kind  of
way  measuring  the  absolute  value  on  this
same  something  there  so  then  we
can
Define  what  I  call  the  analytic
Chay  has  a
pre-image  of  the  open  part  so  this  is
intuitively
speaking
the  union  over  all  n  of  the  locus  where
the  absolute  value  of  of  T  is  bounded
between  some
Powers  some  powers  of  the  absolute  value
of
G
uh
right  and  this  was  something
that  already  entered  at  the  beginning  of
the  last  my  last  lecture  um  that  this  is
what  you  just  start  with  if  you  want  to
define  the  T  with  the
curveent  and  then  find
that
um  and  then  you  still  have  acting  on
this  multiplication  by
two
and  uh  this  map  here  is  actually  some
kind  of  multiplicative  map  so  if  you
mute  value  of  product  the  prodct  the
absolute
values  where  this
if  it  so  happens  that  one  is  zero  and
the  other  is  infinity  then  there's  no
claim  being
um  but  in  particular  on  this  Locus  I  me
you
have  this  analytic
here  to  Z
Infinity  you  have  multiplication  by  Q
here  this  corresponds  to  by
multiplication  by  a  half
here  and  this  map  here  is
actually  this  is  actually  some  kind  of
proper
uh
that
because  P1  is  proper  zero  Infinity  is
proper  so  the  map  is
proper  um  proper  map  to  here  um  and  then
if  you  pass  the
equion  um  then  you
have  GM
Z  I  mean  this  must  be  totally
discontinuous  I  mean  free  and  totally
discontinuous  is  actually  because  it's
it  is  here  right  so  this  then  is  just
F  over  Z
INF  half
Z
which  copy  of
circle
um  so  so  the  m
is  it's  locally  as  is  proper  but  now  S1
is  again
proper
um  and  so  this  means  that  this  one  is
proper  but  of  course  it's  also  locally
asmic  to  this  one  which  is  an  open
subset  of  P1  and  P1  ought  to  be  smooth
uh  so  it's  and  a  curve  so  this  must  be  a
proper  smooth
curve  and  this  is  the  D
right  and  so  uh  this  is  how  we  want  to
argue  how  we  think
construct  so  take  a  look  to
curve  all
right
questions  this  Z  Infinity  is  also  def
over  this
a  you  can  Bas  change  it  to  a  you  don't
have  to  write
okay  and  this  always
that's  you
can
from
what  is  the
usual  there  will  be  a  fun  from  like
condens  s  towards  analytic
steps  and  so  zero  Infinity  I  consider  as
a  life  set  and  this  will  have  an
Incarnation  as  analytic
St  this  will  actually  mean  that  the
theory  of  analytic  stes  will  be  related
to  the  condens  story  in  two
ways  one  because  like  the  analytic  Rings
themselves  are  found  some  on  on  condens
things  uh  which  is  the  role  that  condens
have  played  so  far  but  suddenly  there
will  be  another  role  that  the  condens
Stu  plays  because  yeah  condens  will  also
have  an  ination  in  the  worlds  actually
two  kind  somewhat  AAL  condens
Direction  because  the  way  this  this
thing  will  be  realize  is  antic
St  it  actually  just  use  the  sced  rins  it
will  not  use
any  interesting  condensed  structure  on
the  ring
level
any  thanks  Peter
yeah
\end{unfinished}
% !TeX root = ../AnalyticStacks.tex

\section{Stacks (Clausen)}

\url{https://www.youtube.com/watch?v=EEH_0QhrgEg&list=PLx5f8IelFRgGmu6gmL-Kf_Rl_6Mm7juZO}
\renewcommand{\yt}[2]{\href{https://www.youtube.com/watch?v=EEH_0QhrgEg&list=PLx5f8IelFRgGmu6gmL-Kf_Rl_6Mm7juZO&t=#1}{#2}}
\vspace{1em}

The topic of today's lecture will be stacks.

So far, we've discussed the theory and examples of ``analytic rings''. Next, we will explain how to use these to build geometric objects, the ``analytic stacks''. We already saw a clue of what phenomena we want to include in this world of analytic stacks in the previous lecture, namely the Tate curve over the gaseous base ring.

Today we won't give the precise definition of analytic stack, but will provide motivation from algebraic geometry. The paradigm here is you have commutative rings, and you want to think of these as describing some sort of basic geometric objects which are called affine schemes.
\begin{align*}
  \text{commutative rings} &\leadsto \text{affine schemes}\\
  R &\mapsto \Spec R
\end{align*}
These two categories are anti-equivalent. Then you want to allow yourself to glue these affine schemes together to make more general geometric objects. 

There are two aspects to this gluing:
\begin{enumerate}
  \item What gluings are allowed?
  \item How to identify the results of different gluings? More generally, what are the maps/what is the category we get from these formal gluings of affine schemes?
\end{enumerate}

A classic example is the projective line
\[ \P^1 = \A^1_+ \cup_{\G_m} \A^1_- \]
where we glue together a ``plus'' version of $\A^1$ and a ``minus'' version of $\A^1$ along the multiplicative group $\G_m$. But we want this object to have symmetries like $\PGL_2$, which don't respect this decomposition. So there has to be something said about what are the maps between different gluings.

It's better to think of (2) first: specify an ambient category containing the category of affine schemes, and then single out a full subcategory by specifying allowable gluings of affine schemes in the larger category. This is one way to describe a class of geometric objects in algebraic geometry.
\[\xymatrix@R=2em@C=1em{
\text{category} \ar@{}[rr]|-{\supseteq} \ar@{}[dr]|-{\rotatebox{-45}{$\supseteq$}} && \text{affine schemes = $\text{CRing}^\op$}\\
& \text{single out $\{*\}$} \ar@{}[ur]|-{\rotatebox{45}{$\supseteq$}}
}\]

In the classical approach to schemes, you take the ambient category to be locally ringed spaces, so a commutative ring gives you $\Spec R$, and then a scheme is a locally ringed space which is locally isomorphic to some $\Spec R$. So the allowable gluings between affine opens are gluings along open subsets, in terms of viewing $\Spec R$ as a locally ringed space.

There's a more modern approach which says we shouldn't try to be clever about choosing an ambient category---we don't have to find the concept of a locally ringed space. We can just formally build an ambient category based on the category of commutative rings, and then work there. The category where you're allowed to glue arbitrary affine schemes is called the \textbf{category of presheaves} on affine schemes, $\Psh(\Aff)$. This is the universal category in which you can glue; more formally, it's the initial category-with-all-colimits with a functor from $\Aff$, and giving a colimit-preserving functor out of $\Psh(\Aff)$ is the same as giving a functor out of $\Aff$. There's a set-theoretic technicality here, so caution is needed; we'll discuss that later. The category $\Aff$ is not a small category, so one has to be careful when taking functor categories out of it.

Now you've formally allowed yourself to glue, but you haven't explained how you should identify the results of two different gluings. If you pass to sheaves for some Grothendieck topology, that explains how to identify gluings; more generally, how to map between two of these formal gluings. The more covers you put in your Grothendieck topology, the more maps you're going to be making, which might not be so evident from the perspective of the cover. For example, the automorphisms $\PGL_2$ of $\P^1$ may not be apparent. But you don't want to add too many elements to your cover, or else you might destroy information by identifying too many things; so there's a delicate choice to be made in the Grothendieck topology.

For example, you could take the Zariski topology, and then schemes is a full subcategory of Zariski sheaves on affine schemes such that they're locally representable, in the sense of open covers.
\[
  \text{schemes} \subseteq \Sh^\Zar(\Aff)
\]
You can define the notion of an open inclusion in $\Sh^\Zar(\Aff)$ just by reduction to the case of $\Aff$: a map $X\to Y$ in $\Sh^\Zar(\Aff)$ is an open inclusion if for every $\Spec A\to Y$, the pullback
\[\xymatrix{
  \Spec B \ar[r] \ar[d] \pullback & X\ar[d]\\
  \Spec A \ar[r] & Y
}\]
is an open inclusion $\Spec B\to \Spec A$ in $\Aff$ (this includes the condition that the pullback is in $\Aff$).

This is the perspective we're going to take on defining analytic stacks from analytic rings. However, we don't want to take schemes as a model, because more general geometric objects come up and are relevant, namely \emph{stacks}.

\subsection{Why stacks?}
% \subsection{\yt{12m13s}{Why stacks?}}
Often moduli spaces in algebraic geometry have the form $X/G$, where $X$ is a variety and $G$ is a group variety or scheme acting on $X$. So we have a quotient of some variety by some group of automorphisms.

\begin{example}[\yt{12m48s}{Moduli of elliptic curves}]
  Consider the moduli stack of elliptic curves $\M_{\text{ell},\Z[\frac16]}$, where we invert $6$ to simplify the discussion. Then this is $X/G$ with
  \begin{align*}
    X &= \Spec\Z\left[\tfrac16\right][A,B][\Delta^{-1}]\\
    G &= \G_m
  \end{align*}
  where $\Delta$ is the discriminant.

  This parameterizes elliptic curves with affine equation in Weierstrass form
  \[ y^2 = x^3 + Ax + B. \]
  It then turns out that the only isomorphisms between two elliptic curves given by the above equation are given by scalar multiplication with certain weights on the $x$ and $y$ variables, and that gives the $\G_m$ action.

  There's also many other ways of presenting the same stack. For example, you can add level structure and then mod out by a finite group,
  \[
    \M_{\text{ell},N} / \GL_2(\Z/N)
  \]
  If $N$ is sufficiently large, $\M_{\text{ell},N}$ will be represented by a variety, and then you quotient out by the finite group $\GL_2(\Z/N)$. The Grothendieck topology that we choose should be such that this is identified with $\M_{\text{ell}}$ our category, so we should at least allow \'etale covers into the story for this, and that is indeed the classic choice.

  These quotients exist in the category of schemes, and they give $\A^1_{\Z[\frac16]}$, the so-called ``$j$-line'', implemented by the $j$ function. However, this is not a good quotient. One way of measuring that is: on the moduli stack of elliptic curves, there's a natural line bundle $\omega$,
  \[ \omega = \Lie(E)^*, \]
  which is the dual of the one-dimensional vector space of tangent vectors at the origin. In other words, it's the cotangent space of the universal elliptic curve.

  This means you can write down a line bundle on $\Spec\Z\left[\tfrac16\right][A,B][\Delta^{-1}]$ which is equivariant for the $\G_m$-action, or a line bundle on $\M_{\text{ell},N}$ which is equivariant for the $\GL_2(\Z/N)$-action. However, this line bundle doesn't descend to the quotient $A^1$, so it's a bad quotient in the sense that you can have equivariant objects on the top, but they don't come from something on the bottom. Even more basic, the universal elliptic curve over $\M_{\text{ell}}$ can't be defined over $\A^1$.
\end{example}

The problem in the above example is that the action is not free. There are some elliptic curves with extra automorphisms, and because the action isn't free, the naive quotient in schemes is collapsing too much. The solution is to take the quotient in a more refined (2-)category: (sheaves of) groupoids.

Working in groupoids, there's a notion of groupoid quotient
\[ X \gq G \]
where:
\begin{itemize}
  \item the objects of $X\gq G$ are the elements of $X$
  \item $\Hom_{X\gq G}(x,y) = \{g\in G \mid gx = y \}$
\end{itemize}

There's always a map $X\to X\gq G$, and the fibers are all isomorphic to $G$, where ``fiber'' means a pullback
\[\xymatrix{
  X_x \ar[r] \ar[d] \pullback & X \ar[d]\\
  \{x\} \ar[r] & X \gq G
}\]
So this sort of allows us to pretend that every action is free; the map $X\to X\gq G$ is always like the total space of a $G$-bundle.

The only trick is you have to interpret fiber product in the sense of 2-categories: in a pullback diagram of groupoids,
\[\xymatrix{
  A \times_C B \ar[r] \ar[d] \pullback & B \ar[d]^-g\\
  A \ar[r]_-f & C
}\]
an object of $A\times_C B$ consists of a triple $(a,b,\gamma)$ where $a\in A$, an object $b\in B$, and $\gamma$ is \emph{an isomorphism} between $f(a)$ and $g(b)$ in $C$.

So it's a way of taking a quotient such that you don't really care about the difference between a free action and a non-free action. And it's such that you formally have that line bundles on $X\gq G$ are the same as equivariant line bundles on $X$ and so on and so forth.

This leads to notions such as Deligne-Mumford stack, or more generally, Artin stack. These are all full subcategories of \'etale sheaves on affine schemes.
\[
  \text{Deligne-Mumford stacks, Artin stacks}
  \subseteq
  \Sh^\et(\Aff)
\]
In the discussion of schemes, we had Zariski sheaves that we asked to be locally representable, where ``locally'' is in the sense of open covers. We also could have used \'etale sheaves, it doesn't change the resulting category of schemes. For Artin stacks, you more or less require that you have a smooth cover by affine schemes, and maybe some technical things you want to put in there as well.

The basic example is, let $X$ be an affine scheme, for simplicity let's say over some base $S$. Let $G$ be a smooth group scheme over $S$ which acts on $X$. Then you pass to the quotient in the stacky sense, $X \to X\gq G$.

So these are more general geometric objects, and this is really good for moduli theory. The theoretical justification for that is the \textbf{Artin representability theorem}, which gives concrete criteria for when a functor is represented by an Artin stack. But it's still constrained by the need for a smooth cover; in particular, this means finitely presented. So if you had some non-finite type group scheme acting on something, you wouldn't necessarily be able to take the quotient in Artin stacks.

Now you might say, why do you care? It turns out there are many $X\in\Sh^\et(\Aff)$ which are geometrically relevant, but are \textbf{not} Artin stacks.

\begin{example}[\yt{23m40s}{Formal schemes}]
  \label{ex:15-formal-schemes}
  Let $R$ be a noetherian ring, and $I\subset R$ an ideal of $R$. We want to consider the formal spectrum $\Spf(R_I)$. There are different options as to how to encode this thing, and what it should mean. One we've already discussed is Huber's theory, which includes formal schemes as an example, and is based on viewing $R$ as a topological ring. Grothendieck's theory of formal schemes is also based on viewing $R$ as a topological ring, but localizing along a smaller subset than in Huber's theory.
  
  Another way of looking at it is it's just the union of all the $n$th order infinitesimal neighborhoods of $\Spec(R/I)$ inside $\Spec(R)$,
  \[ \Spf(R_I) = \bigcup_n \Spec(R/I^n). \]
  So giving some (finite-type) data on a formal scheme should be the same as giving compatible collections of data at all of these finite levels. For example, vector bundles on $\Spf(R_I)$ should just give a compatible collection of vector bundles in the usual sense (finitely-generated projective modules),
  \[ \Vect(\Spf(R_I)) = \lim \Vect(R/I^n). \]
\end{example}

This is a very different gluing from what you think about with schemes, and even when you think about Artin stacks. There are no smooth covers in sight here; instead, you're taking some union of infinitesimal thickenings and getting a new object.

There are other examples: for some moduli problems, you really are quotienting by an infinite-dimensional group. One example that's dear to the hearts of both homotopy theorists and number theorists is the moduli of one-dimensional formal groups, the \textbf{Lubin-Tate space}. In this case the group you have to mod out by is coordinate changes on a one-dimensional formal scheme, and then there's infinitely many coefficients you have to specify. So you have an infinite-dimensional group you have to mod out by, and that doesn't fit into the framework of Artin stacks. 

This suggests using a different Grothendieck topology, say fpqc instead of \'etale, if you want to accommodate infinite type phenomena in your covers. But in addition to these, there's also a very remarkable class of examples started by Carlos Simpson.

\begin{center}
  Simpson: ``Every linear algebra category is $\QCoh(\text{some stack})$''
\end{center}

Simpson didn't literally say this, and it's much too strong to be true; this is just an slogan interpretation of his work. His work gives lots of examples of this phenomenon, where you have a natural linear-algebraic category, and then it turns out you can write down some stack whose quasicoherent sheaves are that category. There are fun examples of this already in the world of Artin stacks.

\begin{example}[\yt{28m22s}{Representations}]
  Let $G$ be an algebraic group over a field $k$. Then a special case of a non-free quotient is a point $*=\Spec k$ with an action of $G$, so
  \[ BG = * \gq G. \]
  A quasicoherent sheaf on $BG$ should be a $G$-equivariant quasicoherent sheaf on the point; but a quasicoherent sheaf on a point is just a $k$-vector space, and the $G$-equivariance exactly means you have a representation of $G$. So we get
  \[ \QCoh(BG) = \operatorname{Rep}_G(\text{$k$-vector spaces}). \]
\end{example}

\begin{example}[\yt{29m20s}{Filtered objects}]
  We can also consider $\A^1/\G_m$ for the natural action of $\G_m$ on $\A^1$ by scalar multiplication. This is a funny stack, because there's an open locus in this stack, corresponding to a $\G_m$-invariant open locus in $\A^1$, namely $\G_m$ itself. On that open locus, you're taking $\G_m/\G_m$, which is a point. So this has a point as an open subset, and the closed complement is $0/\G_m=B\G_m$. In this case, (flat) quasicoherent sheaves are given by filtered $k$-vector spaces.
  \[
    \QCoh^{\text{flat}}(\A^1/\G_m) = \{\text{$k$-vector spaces equipped with a $\Z$-indexed filtration}\}
  \]
  The restriction to flat modules is just to stay in the abelian world rather than going derived.

  Note that at the origin, we have
  \[
    \QCoh^{\text{flat}}(B\G_m) = \{\text{$k$-vector spaces equipped with a $\Z$-indexed grading}\}
  \]
\end{example}

These are Artin stacks, so there's nothing exotic there. Here's a more interesting example.

\begin{example}[\yt{31m11s}{de Rham stack}]
  \label{ex:15-dR}
  Let $k$ be a field of characteristic $0$, and let $X/k$ be a smooth variety. Then you can form, and Simpson did, what's called the \textbf{de Rham stack} of $X$. There is a presentation
  \[ X \epi X^\dR \]
  but this is not quotienting out by a group action, it's just some by equivalence relation. The equivalence relation in question is that which identifies two points if they're ``infinitesimally close'' to each other. Equivalence relations are supposed to live in the product $X\times_k X$, and what you do is you take this product and you formally complete along the diagonal,
  \[ (X\times_k X)_{\widehat X}. \]
  Then $X$ mod this equivalence relation gives $X^\dR$. The formal completion should be taken in this sense of Example \ref{ex:15-formal-schemes}, i.e.\ the union of the different scheme structures that are available on the diagonal as a closed subset.

  (The definition of $X^\dR$ makes sense in arbitrary characteristic, but some things we're about to say will not be true in positive characteristic.)

  What are quasi-coherent sheaves, let's say vector bundles, on $X^\dR$? This is the same thing as vector bundles on $X$ equipped with some kind of descent datum, but that descent datum exactly amounts to a flat connection.
  \[ \Vect(X^\dR) = \{\text{vector bundles on $X$ + flat connection}\} \]
  That's Grothendieck's interpretation of what is a flat connection. It's exactly giving descent, identifying infinitesimally close points.
  
  There's also cohomology of the structure sheaf. This gives de Rham cohomology, which is the natural notion of cohomology in the world of vector bundles with flat connection.
  \[ R\Gamma(X^\dR, \O_{X^\dR}) = R\Gamma_\dR(X/k) \]

  This is not even close to being an Artin stack either, and for different reasons from the moduli of formal groups. Here we're modding out by some formal scheme giving an equivalence relation.
\end{example}

\begin{example}[\yt{35m2s}{Prismatization}]
  More recently, Bhatt-Lurie \cite{APC,BLPrismatization} \citeme\textcolor{red}{$F$-gauges} and Drinfeld \citeme define stacks whose $\QCoh$ capture coefficient systems for various $p$-adic cohomology theories in characteristic $p$ or mixed characteristic. For example, there's a stack capturing de Rham characteristic $p$, but it is not the one of Example \ref{ex:15-dR}. You have to use the divided power envelope of the diagonal instead of the formal neighborhood of the diagonal.
  
  So there are stacks capturing prismatic cohomology, de Rham cohomology, and crystalline cohomology, as well as filtered versions of these. Moreover, the comparison theorems in prismatic cohomology between all of these various cohomology theories can be explained, so to speak, ``geometrically'' in terms of the stacks. (It's arguable how geometric these kinds of stacks are :)
  
  Maybe the better way to say it is that, a priori these comparison theorems are about comparing linear algebra categories, e.g.\ vector spaces. But it turns out there's a more fundamental explanation, which is that you have an isomorphism of \emph{stacks}. You then deduce comparison theorems of cohomology theories by passing to quasicoherent sheaves. So you promote a comparison of cohomology theories to an isomorphism of stacks.
\end{example}

I want to give another example of this phenomenon. We've seen de Rham cohomology in characteristic zero, some $p$-adic cohomology theories as well, but what about Betti cohomology? Here's a fun example which actually has quite a bit of relevance for the course, so that's why I'm going to mention it.

\begin{example}[\yt{38m6s}{Betti stacks}]
  ``Betti cohomology'' is the algebraic geometers' term for, if you have a complex variety, then you take singular cohomology or sheaf cohomology with constant coefficients on the underlying topological space with the analytic topology. For example, if you have a compact Hausdorff space $S$, then I claim that you can make a stack.
  
  How do you do it? You use the old familiar idea: you find a surjection from a profinite set $T$, and then you have some fiber product $T\times_S T$.
  \[\xymatrix{
    T\times_S T \ar@<-.5ex>[r] \ar@<.5ex>[r] & T \ar@{->>}[r] & S
  }\]
  Since $T\times_S T$ is a closed subset of a product of two profinite sets, it'll also be a profinite set. Write $T_0\defeq T$ and $T_1\defeq T\times_S T$.
  
  So your compact Hausdorff space $S$ is a quotient of an equivalence relation in the category of profinite sets. We can then apply $C(-,\Z)$ (continuous $\Z$-valued functions) followed by $\Spec$ to get a groupoid in the category of schemes, in fact in the category of affine schemes. We define the Betti stack of $S$ as the quotient of this equivalence relation in the category of sheaves for the fpqc topology on affine schemes. 
  \[\xymatrix{
    \Spec C(T_1,\Z) \ar@<-.5ex>[r] \ar@<.5ex>[r] & \Spec C(T_0,\Z) \ar@{->>}[r] & S^\Betti
  }\]
  You could also replace $\Z$ with a commutative ring $k$.

  What are quasi-coherent sheaves on Betti stacks? These are just usual sheaves of abelian groups on the topological space $S$.
  \[ \QCoh(S^\Betti) = \Sh(S,\Ab) \]
  That's a fun exercise. So, coherent cohomology on the Betti stack $S^\Betti$ is just usual topological cohomology of the topological space $S$. More generally, we can embed condensed sets into stacks via the above procedure, using the presentation of a condensed set via profinite sets.

  \note{(something I didn't hear)}

  What you have to check to see that is that, if you have a surjective map of profinite sets, then it goes to a faithfully flat map on the level of continuous functions. That's not that hard to do: $C(T_i,\Z)$ are filtered colimits of continuous functions of finite sets, which as rings are copies of products of $\Z$. In fact, for \emph{any} map of profinite sets $T_1\to T_0$, the induced map $C(T_0,\Z)\to C(T_1,\Z)$ is flat, and then if it's surjective it's faithfully flat. You also need to check that fiber products in profinite sets correspond to relative tensor products; this follows again by a reduction to finite sets.
\end{example}

So now we're faced with this somewhat baffling array of different stacks, some of which don't resemble Artin stacks in the least. But we want them because they're convenient ways of encoding different linear-algebraic and geometric phenomena. 

\textbf{Question:} how to define a reasonable subcategory of $\Sh^\fpqc(\Aff)$ containing all these examples?

\textbf{Answer:} $\Sh^\fpqc(\Aff)$ (modulo set theory)

In other words, it's not clear that there's any other answer to this question than the entire category $\Sh^\fpqc(\Aff,\An)$. So there is content in the answer, but it didn't necessitate introducing anything wasn't present in the question. 

\subsection{Desired examples of analytic stacks}
% \subsection{\yt{47m44s}{Desired examples of analytic stacks}}
This is also the approach we will take in defining analytic stacks. We will define a Grothendieck topology on $\AnRing^\op$ and then take sheaves with respect to it (again modulo set theory). Before getting into the details of exactly which Grothendieck topology, and these set theoretic technicalities as well, let's see what kind of phenomena we want to capture, so what the examples should be. In some sense, we've already seen some.

\begin{example}[\yt{48m6s}{Adic spaces}]
  We certainly want that any adic space in the sense of Huber should give rise to an analytic stack. We already explained how the basic ingredient in adic spaces, namely Huber pairs $(R,R^+)$, give rise to analytic rings, and we explained something about how the formalism of analytic rings lets you glue. But we didn't quite discuss how you can use that to then glue these analytic rings together to get some kind of analytic stack.

  We saw that at least you can localize the category of modules over that analytic ring along Huber's spectrum, but we didn't quite discuss how you can use that to then glue these analytic rings together to get some kind of analytic space. But we certainly want the kind of gluing that shows up in Huber's theory, gluing along rational open subsets in the topology defined by a basis of rational opens, we want that kind of gluing to be allowed and to give you an analytic space.
\end{example}
  
\begin{example}[\yt{49m8s}{Complex analytic spaces}]
  We also want any complex analytic space to give an analytic stack, say over $\C^\gas$ or $\C^{\liq_p}$. So the kind of gluing allowed should also incorporate gluing along open subsets in complex analytic geometry.
\end{example}

\begin{example}[\yt{49m53s}{Algebraic stacks}]
  Another even more basic thing is we want the world of analytic geometry to be a generalization of the world of schemes, and even of algebraic stacks (in some sense---maybe not precisely the fpqc topology discussed above, but a slight modification of that). These should live over the universal base $\Z$, with trivial analytic structure ($\Mod(\Z)=\Cond\Ab$). So universally, over any analytic ring, if you have some algebraic object, you can get an analytic object.
\end{example}
  
\begin{example}[\yt{51m4s}{Banach rings}]
  Any Banach ring $R$ should also give rise to an analytic stack. This in some sense matches Berkovich's theory, in the same way that the affinoid analytic stacks coming from pairs $(R,R^+)$ match Huber's picture. There's a small interesting tidbit here, which is that the stack that we'll assign to a general Banach ring will actually not be affinoid, it will really be a stack. So, if you take $\Z$ with the usual archimedean norm, it will go to an actual stack that's not affinoid; instead, it's a stack which in some sense corresponds to $\M(\Z)$, the Berkovich spectrum of $\Z$, so that's a fun little twist.
\end{example}

\begin{example}[\yt{52m22s}{Coefficient systems}]
  As above, there should be analytic stacks whose $\QCoh$ capture various coefficient systems for cohomology (sheaves of abelian groups, vector bundles with connection, prismatic $F$-gauges etc.)
\end{example}

\begin{example}[\yt{52m58s}{Tate curve}]
  We want to be able to define the Tate curve as well as its uniformization over $\gasopen$. We also want to have machinery to prove it's algebraic. So we have this curve that we define via uniformization. We take the analytic $\G_m$ and we quotient by the multiplication by $q$, and we get something which Peter argued was a smooth proper curve, and it has an identity section. So then, if you have some Riemann-Roch theorem, then you can see that you have a projective embedding. And if you have some GAGA theorem, then you'll be able to see that it has to be algebraic. So we want Riemann-Roch, and we want GAGA.
\end{example}

\begin{example}[\yt{52m20s}{Comparison theorems}]
  On the theme of GAGA, we want that various linear algebraic comparison results should promote to isomorphisms of stacks. GAGA is one such.
  
  GAGA is a general principle which applies in different contexts. But for example, in the world of complex analytic spaces, it says that if you have a proper algebraic variety $X$ over the $\C$, so it has an analytification $X^\an$ which is compact, then it says that coherent sheaves and their cohomology in the algebraic and in the analytic sense agree.
  
  So that's a question about making a comparison between two linear algebraic categories. It's saying algebraic coherent sheaves are the same as analytic coherent sheaves. And one thing we would like our formalism to do is to promote that to an isomorphism of stacks.

  Another example, again in the complex analytic context, would be the comparison between Betti cohomology and de Rham cohomology. This should also promote to an isomorphism of stacks.
\end{example}

\subsection{Analytification}
% \subsection{\yt{56m48s}{Analytification}}
Recall that adic geometry in the solid context really got started once we noticed that there's a nice subset of $\A^1_{\Z^\solid}$, namely the closed unit disc, which we were thinking of as open. Or you could think of it in terms of its complement. or the translation of that back to the origin. In the end, algebraically speaking, this came from an idempotent algebra
\[ \Z\psr T \in \Solid_{(\Z[T],\,\Z)} \]
Once we had this idempotent algebra, then we could move it to infinity via the change of variables $T\mapsto T^{-1}$, and then that gave us the closed unit disc, and then that let us tie into the $(R,R^+)$ theory, where we could ask that the elements in $R^+$ actually land in the closed unit disc as opposed to just being maps to $\A^1$.

Similarly, over $\gasopen$, you can define a ``subset'' of $\A^1_{\gasopen}$, corresponding to an idempotent algebra of ``functions which are convergent in some (unspecified) disk around the origin''. So it's germs of functions defined at the origin, so to speak. Formally, it's going to be the filtered colimit
\[ P[q^{-1}] \defeq \colim\left(P \xra q P \xra q P \xra q \dotsb \right) \]

In the solid case, $P$ was itself idempotent and turned into $\Z\psr T$. In the gaseous setting, $P$ is not idempotent in $\Mod_{\Z[T]}(\Mod_{\gasopen})$, but it becomes idempotent after we take this colimit to shrink the open unit disc down to the origin.

% \begin{remark}
% Sorry, we write sheaves and what are they valued in? One group? Points, yeah. So, let's say group points, but of course, I mean, you might as well, yeah, I mean, but all the examples are just groupoids. And in fact, many of the examples, even though they're called stacks, are actually just sheaves of sets. So, like Simpson's Duram stack, for example, is just a sheaf of sets. It's a Duram space somehow, but anyway, somehow they call these things stacks anyway.

% But yeah, so it's going to be similar to the situation with analytic rings and where we had sometimes theoretically we had derived things that were animated or whatever and so on. Theoretically we had them, but in practical examples, they didn't really show up. And similarly here, theoretically we're going to allow sheaves with values in anma, but in practical cases, they'll be at most groupoids. I mean, that's, yeah.
% \end{remark}

The idempotent algebra $P[q^{-1}]$ satisfies many of the same properties as $\Z\psr T$, and it again lets us import Huber's theory of pairs to this context. What you do is you take this idempotent algebra functions of germs at the origin in the affine line, you move it to infinity and you get germs at infinity.

Let $R$ be of finite type over $\gasopen(*)$, and assume that $\Z$ is bounded in $R$. Then you get an analytic ring structure on $R$, namely $(R,\Mod_R(\Mod_{\gasopen}))$. Every element $f\in R$ gives a map
\[ \AnSpec(R)\xra f\A^1, \]
and we pass to the ``subset'' where all such maps land in the locus ``away from $\infty$''. This gives a new analytic space $\Spec(R)^\an$, the ``analytification'' of $\Spec(R)$.

\begin{example}
  When $R=\Z[T^\pm]$, we get $\G_m^\an$. After base change to a ring where $2$ is bounded, this is the object of the previous lecture, the thing we quotiented out by to get the Tate curve.
\end{example}

So now we have two different contexts (solid and gaseous) in which you can import Huber's theory of pairs into the world of analytic spaces. It turns out that they can be glued together, and so can the idempotent algebras. First notice that the gaseous theory makes sense over $\Z[q]$: all we had to do to define the gaseous theory was we had to write down the endomorphism $1-qt$ of $P$, and then ask that it become an isomorphism in our theory. And to do that, you didn't need to require $q$ to be topologically nilpotent. So this gives an analytic ring $\Z[q]^\gas$.

\begin{itemize}
  \item if we set $q=0$, we get the uncompleted $\Z$ theory
  \item if you set $q=1$, you get the solid theory, $\Z^\solid$
  \item if you require $q$ to be topologically nilpotent and a unit, you get the gaseous theory, $\gasopen$
  \item if you work away from the locus where $q$ is topologically nilpotent, then you are working over a theory where you force both $q$ and $q^{-1}$ to be gaseous, and that implies that $1$ is gaseous, which means you're in the solid theory, so you get $\Z[q^\pm]^\solid$.
\end{itemize}

Then we can define a certain quotient of
\[ \Spec(\Z[q^\pm]^\qgas) \]
which more or less parameterizes choices of a notion of ``analytification''. So, anytime you have this variable $q$ which you've declared to be gaseous, you can form this colimit and you'll get an idempotent algebra at zero, and you can move it to infinity, and then you get a notion of analytification, as discussed.

However, different choices of $q$ can give rise to the same thing. If two choices of $q$ differ by a bounded unit, so something in $\G_m^\an$, then you'll get the same algebra. So the quotient we want is 
\[ \Spec(\Z[q^\pm]^\qgas)/\G_m^\an \]
where $\G_m^\an$ acts by multiplication on $q$.

This is a different role of stacks in the theory. Here we have a stack which is a quotient of ($\AnSpec$ of) an analytic ring by an equivalence relation, and which is in some sense parameterizing choices of analytic geometry over a given base ring. Let $R$ be an analytic ring with a map
\[ \Spec R^\tri(*) \to \Spec(\Z[q^\pm]^\qgas)/\G_m^\an, \]
which we call a ``gaseous structure'' on $\Spec R^\tri(*)$ (and again assume $2\in R$ is bounded). Then we get two functors from $R^\tri(*)$-schemes to analytic stacks over $R$, plus a natural transformation.

Say $X=\Spec A$, where $A$ is an $R^\tri(*)$-algebra. We can view this as an analytic stack over the uncompleted $\Z$, and then you can base change that to $R$ to get $X_R\defeq X\times_\Z R$. But then you also have a subset
\[ X^\an_R \xra\subseteq X_R, \]
the analytification over $R$, given by requiring that all the functions land in the part of $\A^1$ that's away from $\infty$. Now there's a general theorem.

\begin{theorem}[\yt{1h18m18s}{GAGA}]
  If $X\to\Spec(R)$ is proper (and finitely presented), and every element $f\in R(*)$ is bounded, then $X_R^\an \to X_R$ is an isomorphism.

  \note{some discussion between Peter and Dustin about whether finitely presented is really necessary. It might not matter for GAGA, but could for some other things.}
\end{theorem}

\begin{remark}
  This implies completely formally that $D(X^\an_R)=D(X_R)$, which is some form of GAGA. Non-formally, maybe with some more conditions on $R$ (but satisfied in practice), this implies that $\Vect(X^\an_R)=\Vect(X_R)$, which is classical GAGA. 

  Basically you can do algebraic geometry over the uncompleted $\Z$ theory, and then that means you can do algebraic geometry over any analytic ring just by base change.
\end{remark}

One of the things that maybe I should have been emphasizing before launching into this whole discussion is that over the completed $\Z$ theory, you can basically do algebraic geometry. And then that means you can do algebraic geometry over any analytic ring just by base change. For example, we were considering $\A^1$ over an arbitrary base ring. It's not analytified yet, but when you have extra structure on your analytic ring, then that picks out a choice of what it means to analytify an algebraic variety.

All the classical GAGA theorems are special cases:
\begin{itemize}
  \item complex-analytic GAGA after Serre (using $\C^\gas$ or $\C^{\liq_p}$)
  \item Grothendieck's formal GAGA: \note{look up statement of formal GAGA}if you have again complete noetherian ring $R$, and a proper scheme over it, then coherent sheaves on that is the same thing as coherent sheaves on the formal scheme you get by formally completing or, in other words, compatible collections of coherent sheaves on all the various $n$ potent thickenings there.

  In terms of Huber pairs, you would work over
  \[ \Spa(R^\land_I, R^\land_I) \to \Spec(\Z^\solid) \]
  so $\Spa(R^\land_I, R^\land_I)$ inherits the notion of analytification from $\Z^\solid$ based on the closed unit disc.
  
  \item non-archimedean GAGA over $\Q_p$ or any complete non-archimedean field. For this you would take your analytic ring to be $\Q_p^\solid$. But you don't put the gaseous structure on it which factors through the map to $\Spec\Z^\solid$, rather you put the gaseous structure on it which corresponds to
  \begin{align*}
    \Spec(\Q_p^\solid) &\to \Spec(\Z[\hat q^\pm]^\qgas)\\
    q &\mapsto p
  \end{align*}
  That's the one that picks out the notion of analytification that corresponds to usual analytification of algebraic varieties over your non-archimedean field.
\end{itemize} 

\begin{remark}
  You could try to use the other gaseous structure on $\Spec\Q_p^\solid$ where you factor through $\Spec\Z^\solid$. But GAGA won't apply in this setting, since then $1/p$ will not be bounded, and if you make it bounded in the solid setting you'll kill everything.

  Nonetheless, there is still a different gaseous structure on $\Q_p^\solid$ obtained by factoring it through $\Z_p^\solid$. In a sense the above gaseous ring structure corresponds to some kind of overconvergent version of rigid geometry, while the gaseous ring structure that factors through $\Z_p^\solid$ corresponds to usual rigid geometry.
\end{remark}

\subsection{\ufs Addressing set-theoretic technicalities}
\begin{unfinished}{1:29:22}
things that maybe I should have been
emphasizing before launching into this
whole discussion is that over the
completed Z Theory you you basically you
can do algebraic
geometry
um and then that then that means you can
do algebraic geometry over any analytic
ring just by just by base change so
that's um so that's kind of like
considering the apine line like I was
considering the apine line for example
over an ariary base ring it's not it's
not analytify yet but when you have
extra structure on your analytic ring
then that uh then that picks out a
choice of what it means to analytify
also in algebraic
variety um and
so
uh and I'm also conf exit are fine or
yeah when I was describing kind of
explicitly what it is I was looking at
the case where X is apine but in general
both things glue and so you then then
you can Define uh you can doesn't need
to be it doesn't need to be Aline no
yeah
um so so all the all classical Gaga
theorems
uh are special
cases so
there's um so for example groen's formal
Gaga well the complex analytic
Gaga um so Sarah's original Gaga um take
for example C gaseous or P liquid or
whatever you
like
um uh there's also formal
Gaga uh which is you know if you have a
a again complete nean ring and then you
have a a proper a proper uh scheme over
it then coherent sheaves on that is the
same thing as coherent sheaves on the
formal scheme you get by formally
completing or in other words compatible
collections of coherent sheaves on all
the various n potent thickenings there
um then you would take a spff well maybe
I should say
Spa so in terms of Huber pairs you would
work you would work over this um which
lives over a solid
Z and therefore inherits the notion of
analyics based on
the the closed unit
dis
um it also includes like non-
archimedian GMA
Gaga uh say over I don't know over QP or
any complete non- archimedian field so
there you would take your analytic ring
you would take to be a spec of say QP
solid
um but you wouldn't put the gaseous
structure on it which factors through
the map to spec Z rather you'd put the
gaseous structure on it which
corresponds to mapping topc of Z q hat
plus or minus one Q
gasas uh where Q goes
to uh Q goes to P say so St say standard
choice of topologically mil potent
unit um that's the one that picks out
the notion of analy ification that
corresponds to usual anal analy
ification of algebraic varieties over
your non- archimedian field um you could
have also chosen the other gashes
structure on this guy where you factor
through uh spec Z solid um and that
would give you a different statement of
Gaga in fact a different theorem so yeah
so we're
Mark and three could also
choose
uh the solid
z uh gaseous
structure on uh on solid
QP that gives a an a priori different
Gaga
theorem oh the the analy ification of A1
will then just be the open unit or the
closed unit
dis so um yeah so there's so let me
illustrate H that cannot
beis it has
what
sorry dation
sorry1
ofp
canale so they must
also1 so iation of A1 in that context is
oh wait ah you're right you're right
you're right I'm sorry you're right I'm
sorry yeah that's that that open your
dis is when you just force T to be
analytic but there's no reason
why
uh that should also Force something like
uh T over P to be analytic yeah no ah
same Gaga thank
you yeah I miss uh I I was not the same
guys it's
just no it is the same isn't it because
you you would
make you should always assume that your
whole
ah
you're right I forgot that Axiom shoot
yeah so I'm sorry you need
ah oh boy than uh thanks again so you
need
uh to
assume uh that every element in here is
is bounded
uh
uh yeah
so I was kind of making the same mistake
again over and over um yeah I'm sorry uh
so you don't get a Gaga statement at all
in this context because
um one over p uh is never going to be if
you try to make one over P bounded in
that sense then you're just going to
kill everything so in in the solid QP
setting but nonetheless um there's a
point I wanted to make which is that
there is still a different gaseous
structure on solid QP obtained by taking
um uh obtained by factoring it through
solid zp
and in a sense the difference between
this gaseous ring structure and this
gaseous ring structure is sort of the
difference between usual rigid geometry
and um and some kind of overon
convergent version of rigid
geometry
um
yeah uh
okay
so is there something weaker than
n something weaker than n could you
explain the
question a result that would put uh
identify x r with some
less uh
regular could you explain uh okay maybe
okay yeah this assumption that this uh
bound it yeah yeah
so what is the issue if it's if you
don't assume that it's empty like the if
you don't assume that everything in here
is bounded then when you try to do the
analyics that I described then in
particular you're always forcing all of
the scalers IE elements of of here uh
you're forcing them to be bounded anyway
so if you don't if you don't have this
assumption then your analyics changing
your your base so we have so a special
case is when you take x equals Spec R
and you try to analytify that what this
assumption is saying is basically that
then it doesn't change it like specr is
its own
analytify ring where it's where this
thing has been forced to be bounded but
in the case of solid QP if you try to
force with that with that with that
gaseous structure factoring through
solid Z if you try to force all of the
scalars to be bounded um you're just
going to get the zero ring because
yeah yeah
um
right
yeah um so yeah so that was the
uh yeah x equals specr was giving a
problem there my
apologies
um
okay
uh right
so that was um kind of explaining one
example of where a linear algebraic
comparison result uh can be promoted to
an isomorphism of
stacks um more generally we want
relations um so
relations between the various kinds of
stacks uh various
examples so for already in Peter's talk
he had this map from the T
curve uh with parameter Q to the
topological Circle
so and we want this to make sense as a
map of analytic stacks
so where I already explained how such a
thing is an algebraic stack and
algebraic Stacks will embed into
analytic Stacks even over the the
initial analytic ring um and then we
have the Tate curve and so this kind of
analytic space should map to that kind
of analytic space but also if x is a
complex analytic
space uh then you want
um X to map
to uh its underlying topological Space X
of c
um and if say R is a banak
ring then you want its well sort of its
liquid spec uh to M map to the burkovich
Spectrum
um
so
uh yeah so there's all these we're going
to have all these various different
classes of analytic spaces and we want
the category to be such that we can make
maps between them when we expect
comparisons um so for example it's well
known that the coherent chology of a
comp complex analytic space localizes on
the underlying topological space um
that's kind of more or less by
definition in a lot of ways um and that
kind of linear algebraic relation is
supposed to be explained by the
existence of such a map in this in this
large category of analytic
spaces um so all right
um I
think
uh maybe I'll say one word about the set
theoretic technicalities just to get it
out of the way
um
so so addressing
uh
um
so um well let's go back to our
uh algebraic analog so so this was a
Comm the opposite category of
commutative
Rings um so we have fpqc sheaves on
there but again this is not a legitimate
definition because this is not a small
category and this by definition is a
full subcategory of functors from uh
commutative Rings uh to say sets or
groupoids or whatever you like um and
there's well-known problems with such
things that the
morphisms in this funct category involve
a some more than a sets worth of data so
you have morphisms that aren't sets but
are some sort of bigger entity so that's
a real pain um so what is the
fix
um uh consider so instead
of instead of pre- shees uh on on F
consider uh what are called accessible
prees
um which are the same thing as
accessible
functors uh from commutative rings to
sets and what are accessible
functors
uh if and only if there exists a
cardinal Kappa regular
Cardinal uh such that uh fun F such that
f commutes
with with Kappa filtered
coletes
um so what does this mean uh the the
first choice of Kappa is Alf not
um um and in general when you think of a
regular Cardinal you shouldn't try to
think about what it is as a set what you
should really think of is the collection
of sets that are of smaller cardinality
that's what Kappa is really indexing
it's don't think of it as the
cardinality of some set think of it as
indexing the so this corresponds to
finite
sets um those are the sets of
cardinality smaller than Kappa and then
the notion of Kappa filtered Co limit in
that case is just the usual notion of
filtered Co limit so if your funter
commutes with filtered colimits you're
okay but if Kappa gets
bigger uh you have
fewer uh Kappa filtered Co
limits so to be a so for example if you
take Kaa equals lf1 then you're indexing
the countable
sets and you're only then a Kappa
filtered Co liit would be one where
every countable set has a cone as
opposed to just every finite set so
there are fewer kapper filtered colimits
that means there's more examples of
functors which commute with Kappa
filtered
colimits um and
uh uh
Kappa um so so the nice thing uh is that
every uh for
every X in
F uh the un
uh the on its image under the on
embedding is always
accessible and in
fact
accessible is equivalent to being a
small co-
liit of representable functors
hxs so if we think of
these apine schemes is the basic
building blocks and Co limits as our
gluing procedure then we're saying kind
of an obvious thing we're only allowed
to glue a sets many worth of things
together at a time and that's exactly
what's captured by this notion of
accessible
functor and
moreover uh fpqc shif
foration uh preserves accessible sheaves
or pre- shees
sorry that was actually that's a theorem
of
Waterhouse
um so that um
you can um yeah so you can also impose
the sheath condition at will without
running into set theoretic
technicalities um but this accessible PR
is not presentable right it's not a
presentable C never there
AIC yes okay yeah we have we have to
prove it yeah yeah yeah it has to be
proved so what what makes this work so
these are not kind of in some sense non
trivial claims the claims made here so
what makes this
work so the first thing is that this
category that we're working with uh is
presentable in fact compactly
generated
uh that's what makes the theory of
accessible prees work so it makes that
the condition they're commuting with
Capa filtered col is the same as being a
small cimit of
representes um and then there's
something about the gro de topology
which is that
every uh fpqc
cover
um uh in
F is a a filtered is a there exists a
Kappa such that um and in this case it's
lf1 is a is a Kappa filtered
limit of Kappa comp
compact uh cap fpqc
covers so every uh every flat cover of a
ring is a alf1 filtered Co limit of
countably presented flat covers of that
ring um and this kind of condition
having some a priori bound
on the basic fpqc covers I cardinality
bound is what you can use to um to check
this kind of fpqc shic result and um
so well so maybe I'll say then so
then at
least I we haven't explained what our gr
de topology is but we' least explained
what the category of analytic Rings is
so so there's a theorem um
so analytic
Rings is uh
presentable in fact it's Kappa compactly
generated for
Kappa uh true to the ALF not plus
so
so so that you Capa small means
cardinality less than Continuum um
so the uh that's what lets you kind of
formally avoid all of these uh set
theoretic difficulties involved with
taking pre- sheaves and sheaves on a um
a big
category uh maybe
yeah oh no I won't add anything that's
fine okay thank you for your attention
this category of analytic Rings is
naturally ened of a condensed animal
right I mean
mapping mapping home mapping space
is a condens condens animal yes so we
could also consider ened seeds
instead of usual SE it it seems also
natural to me but you take non-enriched
we take non-enriched yeah and
um it's confusing
the it's
confusing
um what can I
say
um
so in particular this analy the category
of analytic uh Stacks is
not um no it's not that's right
yeah um right different
way because also okay but how do you use
that to get an
enrichment uh
I
mean yeah well maybe there's some
adjoint funter is
there I mean you're saying I mean okay
it's yeah it's not even clear it's
enriched over itself
right
well I mean I don't know yeah but no I
guess I guess you're right you're you're
saying you just take your profinite set
or whatever and then you cross it with X
and map it to y yeah no yeah that's
right yeah so there is an sorry thanks
yeah so you you could take take so in in
analytics tax is enriched over light
condensed sets because you can define an
s-v valued point to be just a a map like
this um but uh it's different from the
enrichment there's essentially no
relation with the enrichment you had on
analytic Rings because we imported these
things via kind of very trivial analytic
Rings just continuous functions and it
um yeah it's all all in all it's a
little bit confusing and um yeah so for
example I think in some recent Works
in in with ptic coefficient systems and
like like V vishan was telling me that
she wants to work with like sheaves of
solid aelan groups on the proall side
site and then you know on the protol
site you already have profinite sets
there so you're somehow you have you're
doing condensed in two orthogonal
directions at the same time but
apparently that's the correct thing for
her and it's just it's it's all a little
bit confusing
but
yeah
yeah I want to explain how this space be
seen as an analytical space um well this
by this I just meant the topological
space um and maybe I make some
assumptions to make sure it's compactly
generated otherwi I mean sorry
metrizable otherwise I have to modify a
little bit what I mean maybe but um yeah
just so the same way that uh that this
thing was a a topological space and
therefore an analytic stack so but this
you could think of as providing an
analytic stack structure on this a
non-trivial analytic stack structure on
this this this topological space
here
yeah um what do you mean by two years
Bond what do I mean by by
one CU is
bounded two is bounded two is bounded ah
yeah so what I meant was that the yeah
so we have some Spec
R and I was looking at the apine line
and then so two is a a function on this
well if you have any element in the
ground ring it gives you a section of
the projection and you can ask that that
section be away from the the local ring
at Infinity so so we had this ring of of
uh ring living at the origin this
filtered cimit of P along scaling the
coordinate by Q then we can put it at
infinity and we can ask that this F not
meet that Locus at Infinity in the sense
that well in the sense that if
you that if you take some relative
tensor product like R mod f and then
that that ring that lives there then you
get zero um
so
um uh right and that if that happens we
say that f is bounded or that f is of
analytic so um it doesn't yeah it
doesn't get too close to
Infinity um so you have to ask that
condition for the scaler two um in order
for some of the claims I made to hold
um yeah and for the Gaga you have to ask
that condition for all elements in the
Basse ring
um
yeah I feel particularly silly about
forgetting that because I mean basically
we we already basically wrote up the
argument in the previous lectures on
complex analytic spaces and that aim was
explicitly included there so I just kind
of forgot about it
um okay other
questions
maybe at Point you said so you wrote
what what was the G structure on Spec R
on Spec R yeah uhhuh said this amount
to Spec
R G GM did it say did I say spec RQ
plusus one I meant spec zq plusus one ah
okay spec oh I'm sorry I'm sorry I meant
Z I meant Z I'm sorry but is it possible
to
elaborate or maybe or
maybe
well I mean the point is that if you
have a map to that stack then you get to
Define such a Ring The Ring of germs at
zero
um uh which lets you run this machine of
of building
analytic so yeah but so why you divide
by G because you don't really want to
specify exactly you don't want to
specify Q so for example you know if you
have like a a tate ring or whatever an
analy T Tate Huber ring or whatever then
there exists a topologically no potent
unit but you know it doesn't play any
role I mean so or at least up to some
bounded difference it doesn't play any
role so that's exactly what this
quotient is doing it's saying you have
some say topologically nil potent unit
um but it shouldn't it should only
matter up to some bounded difference and
so you define this GM analytic and then
you say that if you have q and you
multiply it by something bounded both
both away from zero and away from
Infinity then that should play the same
role in particular it should give rise
to the same item poent algebras and the
same theory of
analytic and so
on by the way I don't see you multiply
by something about it I mean rather that
you can mully by something one or you
raise to raise to some
power but this gets
confus s model is not really antic gen
at
all uh it
is I mean you can have something that's
like Q nor itself but if you multip by
something bound that's too large then
you would actually not have something
anymore yeah but
we that's
true
um that's true wait
uh there but it's not
really I thought I wrote the thing
down
um but maybe we can just
sort I thought I wrote this thing
down um just a
second
ah okay sorry yeah maybe it's not the GM
maybe it's the uh so you can also Define
the locus
where uh the coordinate is bigger than
or equal to one or sorry some in some
over convergent sense and the locus for
T is less than or equal to zero maybe
it's the thing that's uh that lies in
between
there um that's not a yeah that's a
group yeah
um we will discuss later in the
course
yeah but now I'm concerned okay anyway
okay we'll discuss it later in the
course thanks everyone
\end{unfinished}
% !TeX root = ../AnalyticStacks.tex

\section{\ufs Analytic stacks and 6 functors (Scholze)}

\url{https://www.youtube.com/watch?v=BV0-dlAuS3U&list=PLx5f8IelFRgGmu6gmL-Kf_Rl_6Mm7juZO}
\renewcommand{\yt}[2]{\href{https://www.youtube.com/watch?v=BV0-dlAuS3U&list=PLx5f8IelFRgGmu6gmL-Kf_Rl_6Mm7juZO&t=#1}{#2}}
\vspace{1em}

\begin{unfinished}{0:00}
So, last time we were starting the discussion of -- and we really didn't get into the actual definition that we used, but we were trying to give an overview of what the definition should look like and what kind of examples it should accommodate. And so today, I want to go more into -- like, I want to talk about analytic spaces. But something that I think is important is how we set up this definition. It's actually a series of six functions.

Okay, so last time, we saw some motivation for making a definition of -- I want to be in some category, just -- this definition, modulo some set-theoretic issues that I'll explain in a moment. And then, in that series of -- some other things related to sheaves, Hyper-sheaves, and so on. I will probably also say something at the end of this lecture.

So, we would like to define our geometric object by which gives us our basic building blocks, the spaces, and then specify some topology that tells us how we're allowed to glue them together. The key question is, which Grothendieck topology is appropriate to put on this? There was some motivation for this definition, and lots of examples were mentioned that we will be taking up again once we have the proper definition of the category.

Before I get there, let me remark that what we're doing here is kind of very, very close to something we did at the beginning of the course. We defined condensed sets, or "lightly condensed sets," on the category of profinite sets. And there, we were playing a similar game, where we had some basic objects that we started with, and we're building a larger category by allowing ourselves to glue them in a certain way. The gluing that we allow was specified by the Grothendieck topology that we chose. And there, we chose a rather general Grothendieck topology, allowing all, in particular, all surjective maps of profinite sets.

And now, we're repeating this game with a much broader class of basic geometric objects, corresponding to these analytic rings. These are able to model all sorts of different geometries -- they can model aspects of just algebraic geometry, they can model some $p$-adic geometry, some complex geometry, whatever. And all these different kinds of geometry are somehow built, you can put them together in this world of analytic stacks and build spaces out of all of these things together.

So, the key question is, which Grothendieck topology are we going to put on this? Here are some things that we want. The most important variant -- if you have a triangle of analytic animated rings, then the primary invariant that we're interested in is its derived category of complete modules. Recall that this was defined as the full subcategory of the derived category of the condensed animated modules, such that all the homology groups are...
Complete. Okay, so this is the primary variant, and we want that some a met the v a is a sheet. But this is, it's a drive. The only sensible way to phrase this is that it's an $\infty$-category. So we actually need to treat this as an $\infty$-category from now on.

Let's start with something we want. We want this, so this already means that for any analytic stack $x$, we will be able to define the $\infty$-category of $\mathrm{PD}$-crystals on $x$. This is just the limit over all $a$ that belongs to something. That's good, but actually we want something flatly more. This is some business with the six functors.

So we also want $x$ to have a structure of the six functors. Let me say a little bit about what this is. There are six functors: there is always a !-functor, which has some kind of right adjoint, which is an internal Hom; then there is a pullback functor, which has a right adjoint, which is the pushforward; and then there will be two more that we would like to have, you would have to like a $!$-functor and the right adjoint.

So this structure should exist: each $\mathrm{D}(x)$ should have a symmetric model structure, and then whenever you have a model, you can ask for an internal Hom object, which has the usual relation to the model. If such an internal Hom exists, it's called a closed symmetric structure.

Then whenever you have a morphism $f$ from $y$ to $x$, there should be a pullback functor, and this should actually be compatible with the tensor product, should be a tensor functor. And this should have a right adjoint, which is the pushforward. This is something we just get automatically because, like when you have a morphism of rings, you can base change modules, and also then the same functor on $\mathrm{D}(x)$ and $\mathrm{D}(y)$ will be compatible with base change.

But then there are these lower shriek functors, not shriek functors, and so this is a more delicate kind of structure which only exists for certain $f$ from $y$ to $x$. We want the functor that lifts this cohomology, this compactly supported cohomology, and this should satisfy two properties:

One is the base change property, so whenever you have a morphism $Y$ and you have any base change of this, first of all the base change should be again in the $!$-setting where this works, but secondly, there should be a natural isomorphism between taking first the $!$-functor and then pulling back, versus first pulling back and then taking the $!$-functor.

And another thing that it should satisfy is a projection formula isomorphism, that for $a$ in $\mathrm{D}(x)$ and $b$ in $\mathrm{D}(y)$, you're trying to say how these lower shriek functors interact with the existing structure, and so the first thing is that with a pullback they should just commute in this sense, and then with a tensor product, you have the following property that when you have $a$ and $b$, then taking the $!$-functor of the tensor product is the same thing as the tensor product of the $!$-functors.

This is the kind of structure that arises in a lot of different contexts in mathematics. The most classical, in some sense, is if you take some nice topological space, something locally compact, then this has a structure for morphisms where this is literally the compactly supported cohomology. And then you can derive that $f$ what with setting.

Actually, the first time this was developed, not for this case, but it was for $\mathit{D}$-modules on schemes, I think, but then it was realized that there's, I don't know, you can also do this for $\mathit{D}$-modules, and you can also, there are lots of different settings where you can do this. One setting where it was however not so much developed is the setting of varieties, now some kind of coherent settings or quasi-coherent settings, where usually you don't really have a notion of compactly supported cohomology. There is this appendix of Deligne's "Residues and duality" where he kind of

Don't say "um", but actually, our goal will naturally go through a series of $c_{c}$ that behaves rather well, and we absolutely want to have it.

Before going on, let me make a remark, because I think Gouvea will complain in a second that there's a completely imprecise definition here of what the six-functor formalism is. When I write this isomorphism here, there is no natural comparison between the two---these are just two random functors. But they should naturally be identified. So you have to supply this isomorphism. But once you start supplying this isomorphism, then you run into trouble, because now I don't know---you can like base change twice, and then there are comparison maps here, comparison maps here, and one for the composite. Of course, you would hope that they commute, but then you start to wonder how many different such things you can write down and which kind of compatibilities you have to enforce.

I think for a while, it was some kind of open question what a really good and minimal way to encode all the data that is present in a six-functor formalism. There has been work by Liu-Zheng where they do this in the world of $\infty$-categories, and there's been work of Gaitsgory and Rozenblyum where they actually've Centede, although their approaches have different names and so they don't really talk about complexes with compact support at all. In the classical setting, they've set up some kind of notion of what such a six-functor formalism is. That treatment is however $\infty$-categorical, and so that's difficult. Then, maybe, Lucas Mann really isolated a key structure that you need.

So, yeah, there is a goal. I gave a course about this last winter, so let me simply refer you there. But something I was also personally taking away from this course is that if you're just interested in the six-functor formalism, then this kind of dictates everything, including the growth topology.

Okay, right. So, before starting the discussion in our specific case, let's recall the usual definition of such an $\infty$-functor. You do this by specifying two collections of morphisms: there's a class of proper morphisms, and here an $\infty$-functor will actually just be a functor. But if you want to make this definition, you better check that it satisfies the properties that an $\infty$-functor should satisfy---you need proper base change and the projection formula to hold. This time, the situation is better, because when you take this to be a functor, there's actually always a natural base change transformation. So here, I'm not supplying data, I'm just asking for conditions. When you want to declare a morphism to be proper and an $\infty$-functor to be an $\infty$-functor, there's something you can simply check---whether the natural base change transformation is an isomorphism in the case of this morphism $f$, and similarly for the projection formula.

There's also a class of open immersions, where instead you ask that the right adjoint of $f$ for open immersions is the left adjoint of base change. Again, you need to check that this is a reasonable definition, so you need to check that it satisfies base change and projection formula. But this time, again, if it's a left adjoint, there is a natural comparison map between these two things going in the other direction, and you ask that these things are satisfied.

The general morphisms $f$ that are $\infty$-stackable are the ones for which this $\infty$-functor is defined. They're taken to be the composites of an open immersion and a proper morphism. We have open and proper that go far, and the maps that you can choose are the ones which you can somehow compactify, and then you want to declare that the $\infty$-stack is compactly generated.

The definition of properness comes up with a really big caveat: when you write down this as a definition, it's not really a definition, because it chooses the compactification. In general, there are many possible ways to compactify. So you have to show that this definition is independent of the choices, canonically, because you really want to get a final structure. You don't just need to show that it's unique up to isomorphism, but even the isomorphism is unique up to higher isomorphism, and so on.

Fortunately, there is a theorem that was essentially proved by X, and then slightly streamlined in this formulation by Y. Under really minimalistic assumptions on the classes of proper maps, open embeddings, and some general finite maps (so essentially just what I said, except, for example, you want that a composite of finite maps can still be compactified), the theorem is that Z can always produce such a compactification. The precise theorem is in these lecture notes or in Y's work.

In particular, these assumptions include no condition whatsoever on uniqueness of compactifications. You don't assume something like the two compactifications can be dominated by a third one. The question in the chat is whether we can use this formalism for characteristic classes, index theorems, and so on. Yes, that's the kind of things we hope to do with this.

Now, let's apply these general ideas to our setting. We start with the category or $\infty$-category C, which is the category of analytic rings, or "affinoid analytic spaces" if you want. We will generally call an object in here the analytic spectrum of some A, but the underline does not mean it's a topological space---it's just a symbol to say we're taking a geometric object and passing through the opposite here.

We need to figure out which morphisms here should be proper, which should be open embeddings, and at this point we should forget any preconceived notion of what a proper morphism in algebraic geometry is, and really just look at what the formalism tells us. It turns out that a morphism of analytic rings from B to A is proper. This notion of properness is different from the usual one you use in algebraic geometry.

Context: He was talking about a pair of animated condensing light condensing and the subcategory of the connective derived category satisfying some conditions. 

Right, so whenever you have any analytic geometry and just an animated algebra over the underlying animated condensed algebra, then you can always endow this one with an analytic ring structure where completeness is just completeness where you restrict the relation to the condensed one. Whenever we have any map of analytic rings, there's an induced one, and then there is some kind of localization where you're just taking the same ring but then doing a formal completion. This geometrically corresponds to the process of compactifications, although this may or may not be an open embedding.

Here's a proposition: if a proper map of analytic Huber pairs satisfies the projection formula, then it is an isomorphism. The only reason this seems at all sensible is because it matches the notion of properness that applies to formal schemes. In the case of formal schemes, there's always this completion sequence, and if you have a map of formal schemes where the completion upstairs is the smallest possible thing determined by the completion downstairs, this is already proper and satisfies all the good properties that a proper map should satisfy.

The projection formula says that for a proper map of analytic Huber pairs $f: A \to B$, and any $M$ over $A$ and $N$ over $B$, we have $f_*M \otimes_B N \cong f_*(M \otimes_A f^*N)$. This should be intuitively true, since base-changing $M$ from $A$ to $B$ and then tensoring with $N$ should be the same as tensoring $M$ with the base-change of $N$. The key is to be careful about the precise meaning of the symbols and check the details.


Actually, the thing must be the same. All right, and so, in general, you just have to do the same argument more carefully. Namely, I mean, in general, you can just take $n = B$; then the left-hand side becomes the base change from $M$ to $B$ in the sense of analytic rings, but the right-hand side is a base change from $M$ from $A$ to $B$ purely in the sense of complete $A$-module. And so, that these agree is precisely the assertion that it's in very---

So, you get this equation where the $T$ has these meanings, and that they agree that the base change in the sense of analytic is just the tensor of $A$ is precisely saying that this must be induc because for the IND ener structure is precisely how you comp to base change in general when you compute the base change for rings. And first, you do the base change for juices, so you do this, but then afterward, you would still have to make it complete as a $B$-module. But here, it's saying that you don't actually have to do it.

Okay, was a question? I'm not sure what the question is, like the map from the fine line to a point is not proper in usual algebraic geometry, but if you pass through this word, it's counted as a proper here because it satisfies the projection formula and also satisfies proper base change and so on. I mean, actually, in classical base change for complex geometry, you don't need to ask for proper; you only need to ask for quasi-compact and quasi-separated.

So, Dustin would be able to give a really good solution to this. This is good. So, it's just a class of proper maps, it's on the base stack, and proper base change and the projection formula for proper base change, same base change, or it tells you that after any base change, the projection formula still holds. But also, proper base change---he would be able to check if you haven't induced any the structure, then you can just unravel what all the symbols mean, and it comes out right.

All right, so let's get to the class of open immersions, and again, I ask you to forget any preconceived notion of what an open immersion is. I'll give examples in a second. Let me actually call it $\mathcal{J}$, just for psychological comfort. Merge, if the $P$ map $B$ of modules $\mathcal{A}$-dit, the left three, set the projection from---

So, let me give a first example and then a non-example. A non-example is any open immersion in algebraic geometry that's not also a closed immersion, open-closed immersions. Okay, there, I don't know, they're kind of stupid, but like, if you take $\mathbb{G}_m \times \mathbb{Z}$ inside of some $\mathbb{Z}[t]$ with like, trivial ring structure, this is not an open immersion because usually, in algebraic geometry, taking the tensor of modules virtually never commutes with any infinite products, right? If you want left joint, it should mean that the pullback should commute with products.

But here's a key example: if you take the Spec of $\mathbb{Z}[p^{-1}]$ solid, this will be an example, or also sometimes more primitive because this one can be realized as the base change of this one, not quite, but essentially. The key thing to know is that if you take these joint solid and advance this into the induced one solid, this is open. This is something that already came up in

In terms of this, from this funny object, this thing should receive a map from $M$. Indeed, this complex it gives us a homological degree in $\mathbb{Z}$. This complex, which on cohomology gives us this junction, I think if you look back at Dustin's lecture on the certification of a $\mathbb{Z}$-joint, this is the formula he gave. This means that actually, if you tensor with this object, then this would become the left joint. 

It is $\mathfrak{m}$-injective in this one $\mathfrak{S}$, and here this means the comp. And so this means that this left joint exists on the image of $\text{Jappa}$, but $\text{Jasta}$ is essentially surjective. This also says that this left joint is given by tensoring some module, which is exactly what you need to check that projection.

Why is this reasonable? Why does this have anything to do with the intuitive idea that these lower $\mathfrak{S}$-functors should be some kind of compact support? In particular, like of the structure sheaf of $\mathbb{X}$, the upper stalk of $\mathfrak{S}$ should be like the compactly supported coherent cohomology in some sense of the affine line. 

Well, what should it take to give a compactly supported thing? You should vanish near infinity. Giving a function, that's an element of $\mathbb{Z}[[T]]$, and now you want to say that vanish at infinity. But infinity is going to get functions the $\frac{1}{T}$ series, and so you would like to take functions that vanish near infinity. 

The key thing making this work here is that you get some localization with these properties. The key thing you need to check when you want to check that, for example, this is what the completion does, is that $\mathbb{Z}[T^{-1}]$ is an important algebra. It's an object if you tensor it with itself, it's still itself. This is in fact the general description of these open immersions.

Given some $X$, the open immersion $J$ from some $Y$ into $X$ are equivalent, or anti-equivalent, I'm not sure right now, to idempotent commutative algebra $C$ such that the internal Hom from $C$ preserves some contivity assumption. Here, $J$ maps to the following. You can check $J^\dagger$ of the unit, and this will always map back to the unit, and the cone of that map is always an idempotent algebra.

The projection formula implies that the étale module is really just the stalk of the unit and the rest. This means that this thing is completely determined by the right module. So this means that the right module is actually just given by the internal Hom from the unit. Because the left adjoint tensoring with some object is our Hom from the unit, we know that the completion is really just given like this. This means that, in particular, this new analytic geometry structure is completely determined by this object, and then you just have to specify, or equivalently by this co-map, because then we can recover it by taking the fiber from one to C.

Then you just need to supply the conditions on this so that this completion really determines the analytic $R$-structure. For this, you need to check two things: that this completion commutes with all co-limits, and that it should preserve connectivity. All the other properties of the analytic $R$-structure are just some formal procedure to check.

Basically, when you have an open immersion, there is always this item but commutative algebra describing a space at infinity. In the sense of this open immersion, you have this open immersion, and then there's some kind of complementary closed subset determined by some algebra which is functions at infinity. This is an important algebra. The general idea is that whenever you have an open immersion, there is a complementary thing at infinity which is described by some algebra.

As a corollary of this, you can check that the class of étale maps is stable under base change. So we've isolated what the proper open immersions are. This leads us to a stable map. If you factor open and proper, we already know which proper map to take because I already told you about this canonical characterization of any Mori structure and then some kind of localization. This $F$ it always specifies some canonical $\overline{F}$, in this case the compactification is canonical, which is nice.

This $J$ always exists, and it will also always have the property that $J$ does for the factful. The real only condition is that you have the left adjoint, the projection formula. An interesting example here is if you take the solid structure on $\mathbb{Z}$, which is a complexification, but precisely this one over $s_0$. More generally, if you have any $M$-finite type algebras and take the solid relative ring structure, then these maps will all be étale, or if you have a map of pairs, then being étale is exactly this condition that Huber calls being of $+$-weakly finite type, which just means that the subring of integral elements $A^+$ is generated by just finitely many new elements.

Huber was defining these kinds of finite notions also for $\mathbf{p}$-adic spaces, but now they are also the ones where you can define relative étale functoriality for proper open immersions as well. Yes, that's next. There is a little bit to check, like if you take a composite of two étale maps, that's true, basically just by composition on the base change. And then for the $\mathbf{s}$-functor, there's one thing you have to check, which is some interaction between the lower and the lower star functor for open immersions and proper maps. Again, that's a very straightforward check.

I didn't discuss in this lecture all the little X's you have to put to make this work. There are some little bits to check, but each of them is a really simple check. As a proposition, basically a corollary to this general construction of success, some the fall re, you can say that this data, like of proper maps, open maps, satisfies all the required conditions to get enough étale cohomology on our category $C$, which I recall.

Now, a general question I had: Given a functor $f: X \to \mathcal{C}$ from some category $\mathcal{C}$, one may want to pass to a larger category of geometric objects built by gluing objects in $\mathcal{C}$, just like schemes are built from affine schemes. This was maybe the original question that Lurie and Joyal were interested in when they wanted to extend from schemes to $\infty$-stacks or higher $\infty$-stacks.

I proved some general results about how one can go about extending a functor from one category to another. This was also used by Lukas Main, who started rephrasing it, and when I got to Scholze, I again slightly rephrased what they did. In my notes on six functors, I tried to analyze this question and pin down the best topology for such extensions.

I didn't completely settle on a very precise combination of this topology, as it's slightly ugly at one point. But the takeaway is that the covers should basically be those that satisfy "universal star descent" and "universal codescent." I'll explain what I mean by this in a moment.

Let me just state a theorem that in the case we're in, you should ask for universal star and universal codescent, which seems like a lot to check. But actually, you need to check much less---to some extent, it's always any six functor, but I think it seems better here.


Even stronger, it satisfies to the end. So, if you're not thinking about---usually we have a ring and you think about the category of modules, but now we can go one categoric level higher and think about presentable stable $\infty$-categories linear over $\mathcal{D}(A)$, and it turns out that asking for Schröder descent is equivalent to asking for descent at a 2-categorical level, which is actually how the proof is obtained, but this is probably not one I want to do now in the last five minutes.

Let me just end by giving the definition of the $\mathrm{gr}$ topology. That was the first step to fall on. Well, generally, on the one hand, just by finite just unions. So, whenever you have a $\mathrm{gr}$ topology, it's just a union of several, then it's covered by those. This includes the empty cover of the empty set.

And these satisfy some compatibilities, $\mathcal{F}$, so this is a rather general class of things. It's actually very, very close to related to this notion of descendability that we defined, which is some nice notion of descent that's satisfied for virtually all faithfully flat maps, at least all those that are countably presented, which will be another motivation actually for us to at one point switch to the $\mathrm{\ell}_{\mathrm{tx}}$ setting, because there, things are countably generated.

Particularly, if we restrict to proper maps, then this condition of Schröder descent is precisely the condition that the map of algebras descends. Yeah, have that, and then, okay, they, like in the beginning of my lecture, I said that there are two small wrinkles---one about set-theoretic issues, so again, we should look at accessible things, and for this, we should check that this $\mathrm{gr}$ topology has some approximation properties, which means that specification of something accessible stays acceptable.

The other, some hypercompleteness issues, is that we actually don't want the thing that's just sheaves on this, because also in the $\mathrm{\ell}_{\mathrm{tx}}$ setting, we're actually considering hypersheaves. So, we actually want to allow a certain class of hypersheaves, but it's the same ideas that go into the classical notion of hypersheaves.

Okay, basically, $\mathrm{\ell}_{\mathrm{tx}}$ will be hypersheaves on $\mathcal{F}$, $\mathrm{\ell}_{\mathrm{tx}}^{\ \mathrm{fin}}$. All right, let me take a question. What do you mean by finite unions? Well, I mean you can have like $\mathcal{F}$-$\mathrm{\ell}_{\mathrm{tx}}$ objects, they have finite unions. I mean, like rings and finite products, and so whenever I take such an $\mathcal{F}$-finite union of $\mathcal{F}$-$\mathrm{\ell}_{\mathrm{tx}}$ objects, it should stay in $\mathrm{\ell}_{\mathrm{tx}}^{\ \mathrm{fin}}$, which is like it's covered by the individual ones.

Right, no, just this is a cover whenever $I$ is a finite set and $x_i$ are in $\mathcal{F}$, then this is also an 

It's an extremely wonderful structure that Jacob Lurie defined, this category of presentable Infinity categories with colimit-preserving functions. In there, you have a subcategory of stable ones, or also, whenever you work over some ring, you have some categories which are linear over some base, e.g. the derived category of a ring. These are, loosely speaking, DG categories, where the functions are colimit-preserving. This has a well, maybe a tangent structure. 

Now you can also ask whether the association that takes such an algebra or the spectrum of it to this Infinity one or Infinity two category, it doesn't matter which one you choose for this purpose of asking for descent, whether this satisfies descent. By modular categories, I mean objects in this same Jacob Lurie's formalism. 

For example, the derived category of something over a ring A is in this category, as is the derived category of an A-algebra. If you glue such things locally, then you are given the descent data. And why is it Infinity? The hom between two such things is viewed as an Infinity one category, actually, because you can't forget about the non-invertible 2-morphisms, and then you get an Infinity one category. 

For the question of descent, it turns out that the 2-categorical structure of this thing doesn't matter much, and the non-invertible 2-morphisms will automatically satisfy the descent condition. For these questions, I would really recommend reading the paper of Artem Yurievich Belorussov.
\end{unfinished}
% !TeX root = ../AnalyticStacks.tex

\section{\ufs !-descent continued (Clausen)}

\url{https://www.youtube.com/watch?v=rN_iM7Z8vdE&list=PLx5f8IelFRgGmu6gmL-Kf_Rl_6Mm7juZO}
\renewcommand{\yt}[2]{\href{https://www.youtube.com/watch?v=rN_iM7Z8vdE&list=PLx5f8IelFRgGmu6gmL-Kf_Rl_6Mm7juZO&t=#1}{#2}}
\vspace{1em}

\begin{unfinished}{0:00}
  Okay, let's get started. I will continue the discussion that Peter started last time on !-descent.

Let me recall the setup. We have analytic rings, and we are going to build some geometric objects on them, somewhat in the model of scheme theory. The first thing we do is define what the affine things are - we formally define them to be the opposite category of analytic rings. An object in this category, let's call it $X$, corresponds to an analytic ring, which we can denote as $\mathcal{O}_X, \mathcal{D}_X$. An analytic ring consists of an animated condensed ring and a certain full subcategory of modules over that, in the full derived category of modules over that ring.

I will denote the first coordinate as $\mathcal{O}$ of $X$, and the second coordinate as $\mathcal{D}$ of $X$ in this notation. Here, $X$ is really just a formal symbol.

In the derived context, there is a way to define the full derived category. For some purposes, it's easier to just restrict to the non-negative part, as the two formalisms are equivalent - you can go back and forth between them. But for the discussion today, it's actually much better to consider the full derived category as the primary object. The reason has to do with a descent phenomenon.

It turns out that the full derived category will have this !-descent property, but the non-negative part will not. In other words, for an analytic ring, we have a nice $t$-structure on this category, but for a general analytic stack, there won't be a $t$-structure on the derived category of the global object, which is obtained by gluing these local derived categories.

For an analytic ring, the category $\mathcal{D}_X$ is by definition a full subcategory of the derived category of $\mathcal{O}_X$-modules. This has a natural $t$-structure, and the claim is that this $t$-structure is detected all the way down in the condensed Abelian groups, and it induces a $t$-structure here as well.

We singled out the notion of a \'shable\' map, which means it can be factored as an open immersion followed by a proper map. For a proper map $f: X' \to X$, this means that an element $m$ in $\mathcal{D}_{X'}$ lies in $\mathcal{D}_X$ if and only if its image in $\mathcal{D}_Y$ lies in $\mathcal{D}_Y$, where $Y$ is the target of the map. In other words, you just take the class of complete modules on the analytic ring $X'$ and inherit the notion of completeness up to $X$.

We claimed that the composition of such maps is also \'shable\', but I forgot the details now.

Peter claimed it last time, I believe. But he didn't give the argument. It's quite straightforward, though. 

Let me continue the discussion. The open immersion means that the functor $J_!$ always exists. This should be a localization, and the kernel should be just modules over some idempotent algebra. It will necessarily be a compact object because, by the definition of a map of analytic rings, the right adjoint of this commutes with colimits.

This is just in the pure category theory sense. There is a notion of localizing by a multiplicative set in usual categories. But these are presentable categories, and the good definition is what Peter said: the right adjoint should be a fully faithful functor.

We already know, as part of the discussion of analytic rings, that the right adjoint exists. So this implies that there is a left adjoint.

Let me continue the discussion. We had a claim or a theorem. The remark is that for a proper map $\pi$, the right adjoint $\pi_*$ is nice: it commutes with colimits, satisfies the projection formula (i.e., it's $\mathcal{D}(Y)$-linear), and commutes with base change.

For an open immersion $J$, there exists a left adjoint $J^!$ or $J^\natural$, which has similar nice properties.

The theorem discussed last time was that there exists a six-functor formalism on $\mathcal{D}$ such that the class of "shabable" maps $f: X \to Y$ is characterized by having the property that for $f$ proper, $f_! = f_*$, and for $J$ an open immersion, $J^!$ is the left adjoint.

This class of shabable maps has good closure properties: it is closed under composition and base change.


If $F_1: X_1 \to Y$, $F_2: X_2 \to Y$, ..., $F_n: X_n \to Y$ are maps with the same target, then $f_i$ is $\mathsf{shtuka}$-able for all $i$ if and only if the map $F$ from the disjoint union over $i$ of $X_i$ to $Y$ is $\mathsf{shtuka}$-able. 

And these disjoint unions in this category, those finite disjoint unions in this category, they correspond to finite products in this category of analytic rings, and they're kind of just naively defined coordinate-wise.

So for finite finite coproducts, maybe I'll do just a reminder of some example of this six functor formalism. Let's take $Y$ to be $\mathsf{Spec}$ of the solid $\mathbb{Z}$-theory, and let's take $X$ to be $\mathsf{Spec}$ of the solid $\mathbb{Z}[T]$-theory, with the natural map from $X$ to $Y$. Then we can take $X'$ to be $\mathsf{Spec}$ of the solid $\mathbb{Z}[T, T^{-1}]$-theory. 

So this is a proper map, and this is an open immersion. And the complementary algebra is equal to this $\mathbb{Z}[[T]]^{-1}$. And then, for example, $j_! (\mathcal{O}_X)$ is this two-term complex. And then that's also the formula for $\pi_! \pi_*$, which is just the forgetful functor, forgetting that you have a module structure over $\mathbb{Z}[T]$. Intuitively speaking, this is functions on the affine line, and this is functions localized near infinity, localized near the missing point. And so this is like functions on the affine line which vanish near infinity, so to speak, because you're taking a fiber. So that's kind of compactly supported cohomology of the structure sheaf on the affine line, so to speak.

And this is what six functor formalisms are supposed to be doing in general: they're supposed to be specifying some notion of compactly supported cohomology, relative compactly supported cohomology, which behaves well in families. The key property of this $j_!$ is that it commutes with base change in complete generality.

So the algebraic objects are proper, even if you're dealing with something like the affine line, which is not proper in traditional algebraic geometry, but it's kind of compensated by the existence of this solid theory, where you have a new version of the affine line, the solid affine line, which is not proper anymore.

Okay, so a $\mathsf{shtuka}$-able map satisfies Čech descent. If you take $\mathcal{D}_Y$, $\mathcal{D}_X$, and $\mathcal{D}_{X \times_Y X}$, and then continue like this, using the $f_!$ functors to define this diagram, this induces the $\check{\mathcal{C}}$ limit of the $\mathcal{D}_X$ to the $\mathcal{D}_{X/Y}$. The naive thing would be to consider the star descent, you
This condition, what about the composition of Shri? Does it involve interchanging proper and open, or is it quite straightforward? The essential reason it's straightforward is that there's a canonical candidate for the $X'$ in the factorization. Your $X$ contains this algebra $\mathcal{O}_X$, the structure sheaf, and then you can just take $X'$ to be $\mathcal{O}_X$ and then $\mathcal{D}(X')$ should just be the full subcategory of $\mathcal{D}(\mathcal{O}_X)$ consisting of things whose image in $\mathcal{D}(\mathcal{O}_Y)$ lands in $\mathcal{D}(Y)$. 

So you have an open immersion and then a proper map, and then you claim that this, in étale spaces, the... Okay, so you use the other... Yes, okay, and you check that this is canonical.

The second goal for today is to be able to give examples of Čech covers, so that you can then apply these descent results. I don't remember if it was defined in the previous talk I saw, but what the six functor formalism is exactly. I think he referred to Lurie, so I'm going to actually discuss it today, because the precise way it's encoded will help give some arguments. So it's in fact the next topic that I'm going to turn to.

The existence of this six functor formalism on $\mathbf{S}$ follows rather immediately from work of Lurie, as reinterpreted by Gaitsgory and Rozenblyum. Following Gaitsgory, Rozenblyum, and Lurie, Gaitsgory encodes the six functor formalism in terms of span categories. Let me now set this up.

Suppose we are given an $\infty$-category $C$ with all pullbacks, and a class of maps $S$ in $C$ stable under composition and pullback. Then we get a span category $\operatorname{Span}_{C,S}$. The idea is that this is going to encode the six functor formalism, where you have shriek maps defined for maps that lie in $S$, and you have star maps like upper star defined for any map whatsoever.

This has objects the same as in $C$, but maps $X \to Y$ given by diagrams $X \leftarrow M \to Y$, where the right-hand map lies in $S$. The composition is given by pullback. A two-functor formalism on $C$ with respect to $S$ is just a functor from $\operatorname{Span}_{C,S}$ to some category of categories.


Suppose you are given a functor $F_!$ from the derived category $\mathcal{D}(X)$ to the derived category $\mathcal{D}(Y)$. Here, this functor $F_!$ is giving you a functor $F^*$ from $\mathcal{D}(X)$ to $\mathcal{D}(Y)$. These functors are stable under composition and include having the identity.

Yes, it does. This is what happens when you compose an empty set's worth of composable maps. If you have a pair of composable maps, the functoriality amounts to the base change formula for the $F_!$ functors. The fact that $F_!$ commutes with $F^*$ when you have a Cartesian square in your category.

So, a two-functor formalism versus a six-functor formalism, but two is the essence of six in this case. The other functors are: a base change formula, compatibility of $F^*$ under composition, and it also encodes compatibilities between the base change formulas and the compositions that you have on these things, so there's higher-order data implicit in this.

Provided this admits a right adjoint, then you get the right adjoint as well. And the remaining functors are some tensor product operation you have defined on each of these, and some adjoint to it, some internal Hom. If you have this and it satisfies a certain property, then it has some adjoint.

To encode this tensor product and the expected interaction of it with the functors here, namely projection formulas, we use a symmetric monoidal structure on this span category, induced by the Cartesian product in $C$. We need to assume that $C$ has a terminal object and pullbacks, so that we have products.

This symmetric monoidal structure on the span category is not the Cartesian product in the category anymore, it's just some symmetric monoidal structure. We then request that your functor $\mathcal{D}$ from this span category be lax symmetric monoidal with respect to that tensor product we just defined and the symmetric structure here given by the Cartesian product of categories.

The basic data is that you have $\mathcal{D}(X) \otimes \mathcal{D}(Y)$ should map to $\mathcal{D}(X \times Y)$, plus some compatibilities which are conveniently encoded in this being a lax symmetric monoidal functor. In particular, $\mathcal{D}(X) \to \mathcal{D}(X)$ using the diagonal, and then you can probably reconstruct this by pulling back over $Y$. But then one can wonder why you need the product in the category---it's not enough to specify $\mathcal{D}(X) \to \mathcal{D}(X)$ for every $X$ in some coherent way, you need to encode the compatibility with the $F_!$ functors somehow.


Object, oh boy, yeah. I'm sure it comes for free from, so it lacks a unit. I mean, there should be some unit.

Yeah, yeah. Okay, so I have to advance this story a bit. I said that if you just have these three functors satisfying certain properties, then you automatically get all six functors. And there's a very convenient way to organize the passage from 3 to 6, and that's a bit of higher categorical magic defined by Lurie. This is Lurie's magic category called PRL.

I have to say a little bit about this and how it works. Here, the objects are the presentable $\infty$-categories. "Presentable" means you have all small colimits and you're in some sense controlled by a small subcategory, in the sense that there's a small subcategory such that the whole thing is gotten by formally adjoining some sufficiently filtered colimits. So you should be $\kappa$-compactly generated for some $\kappa$. The morphisms are the colimit-preserving functors, and that's equivalent to admitting a right adjoint.

So let me tell you a bit more about this magic category. PRL has all small limits, and the forgetful functor PRL to the $\infty$-category of categories preserves them. That's nice--limits in this category exist and are completely naive. But also, PRL has all small colimits, and this is the really remarkable thing: you can also access these colimits in a completely naive way. There is a constant version of the forgetful functor where you take a category C and send it to C, but then you take a functor from C to D and send it to its right adjoint. You can make this into an honest functor, and this functor preserves colimits, which translates to limits in the $\infty$-category of categories.

So also the colimits in Lurie's magic category are just calculated as naive limits of the underlying categories, but with respect to passing to the right adjoints of the functors in your diagram. This is quite magical, because in the $\infty$-category of categories, colimits can be very difficult to calculate. But this makes it easy.

In particular, the condition of shriek descent for maps of fiberable maps of abelian groups is actually the same thing as cocartesian descent for the lower shriek functors in PRL, which is a much more convenient way of thinking about it.

There's more magic in PRL--there's also a symmetric monoidal structure, which is maybe the main theorem in Lurie's book. It's a tensor product characterized by a universal property, just like you would expect of a tensor product. Maps in PRL from C tensor D to E are the same thing as functors from the product that commute with colimits in each variable separately. And this tensor product on PRL also commutes with colimits in each variable separately, so you get a very nicely behaved tensor product on this category.

In fact, the internal Hom from C to D is just the $\infty$-category of colimit-preserving functors from C to D.

Of course, we only remember isomorphisms. So you can somehow recover the full mapping category by using this tensor structure. That's one way of recovering it at least.

In principle, PRL should be considered as an Infinity 2-category because there's a whole category of maps between two objects. But you don't really need to remember the Infinity 2-categorical structure because it's just the internal equivalence or isomorphism in this theory.

There was a reference to a paper on Infinity 2 in some places. So do they define Infinity categories and all of this in general? I haven't kept up with that literature; so far, I've been fine with just Infinity 1.

Okay, I want to continue a little bit more. If you have a PRL, it is now a tensor category, a category with a symmetric monoidal structure. You can ask what is a commutative algebra in this tensor category, meaning an object equipped with some symmetric multiplication and higher coherences about it being sufficiently commutative. What this means is that C is a symmetric monoidal presentable Infinity category, and the tensor product on C commutes with colimits in each variable. Some people call this a presentably symmetric monoidal Infinity category.

A basic example for us would be D of X for an algebraic variety X. You can then consider modules over C in PRL, which are certain presentable Infinity categories tensored over C. This allows you to say they are enriched over C in some sense. You have a C-internal hom from M to N, which is characterized by a universal property relating maps from C tensor M to N.

The category of modules over an algebra R1 tensor D R2 should be the same as the D of the tensor product, unless you're in characteristic 0. In that case, you get something bigger. But let me continue the story a bit more.

Mod C PRL also has all limits and colimits, and the forgetful functor Mod C PRL to PRL preserves them. It also has a tensor product which is a tensor product over C, a standard relative tensor product. 

If you have C in PRL and then commutative algebra objects R and S in C, you can consider left modules over R or S, which will be in Mod C PRL. Then Mod R C tensor over C Mod S C is just Mod R tensor S C.

Course, you say, "C" tends to him, but of course this comes with, yeah, there's more coherencies and there is a good way to formulate this system of coherencies. In this, we have to read some higher algebra. You have to read higher algebra, but you know, yeah, you, yeah, it's not easy to read higher algebra, I know, but it's possible. People have done it, and someone even wrote it, which is even more amazing, yeah.

Okay, so Tser products. Okay, so now I want to connect this to analytic rings. So, note, there's a functor or, well, maybe, so, yeah, so now I can say the good way to encode, or a good way at least to encode a six functor formalism, is a lax symmetric monoidal functor from Span(C's) to PRL with this tensor product here. So, we're no longer using the Cartesian product on Cat(Infinity), but this tensor product on PRL. But it really just amounts to a condition on the kind of formalism we had in the other sense. It's just the condition that when you look at this D of X cross D of Y going to D of X cross Y, that that should commute with colimits in both D and X and D and Y separately.

So, it's just a condition on this formalism here, but it's best to think of it as being a formalism with values in PRL, okay. Now, I want to connect with our specific example. So, note, analytic rings map to PRL by sending our triangle, D of R, to D of R. Well, in fact, as I already said, it maps to C alge(PRL), but, in fact, it maps to C alge(PRL) over a certain other commutative algebra object, namely, the derived category of condensed Abelian groups. So, everything by its nature lives over this derived category of condensed Abelian groups. So, we have a presentably symmetric monoidal category with a functor from this presentably symmetric monoidal category, and then I claim this functor commutes with colimits and detects isomorphisms under, yes, under, thank you. We're in the world of algebra, so I should say "under".

Yeah, we have this category here, and then we have an object in this category, and this notation means that you consider an object in this category together with a map from this object to that object. So, it's this slice category or "under" or "over" category or something. Condensed, every condensed abil, co, gives you, by pullback, some guy in of, okay. I mean, it's just, yeah, I mean, the initial object here, I want to say, it always SS to see over over IM of the addition, ah, okay. That's an analytic ring to what's the initial object, Z. Okay, so for Z, you get D of, condensed, so you don't need them, ah, okay, CLA this F, just, yeah, just a sec, just a sec.

In particular, so if you want to, the derived category of a pushout, so, let's say, a, so this is pushout in analytic rings, which was this kind of slightly subtle operation in this perspective on analytic rings because you had to complete some pre-analytic ring structure and so on, but actually on the level of the categories, it's quite naive. So, it is just this Lurie tensor product, this relative tensor product in PRL. So, the proof is not so difficult. Let me indicate what's going on in the proof, just in this special case, which is really all we need. In the case where A and B are both proper over R, I'm not assuming any Schreier ability, but still, let me use this language of proper maps. In the case when A and B are proper over R, it's just an instance of this general fact here. Did you say that commutative and now got confused, all this? So, somehow you get, you said the modules as limits and colimits, and then did you say that symmetric? Oh, I didn't say that C alge has colimits. I should have maybe discussed it, but it does. And, as usual, pushouts are calculated by relative tensor products. Pushouts in C alge(PRL) are calculated by relative tensor products in PRL. Okay, and it has limits also. Those


You can still factor any map of analytic rings as a proper map followed by a localization. For localizations, it's quite easy to check the universal properties to compare the two sides of this. Granting the case of proper maps, you then just have two different quotients of your category that you're trying to identify, and you can identify them by looking at the descriptions.

For a proper map, the algebra over the bigger ring is the algebra of the source. However, for general analytic maps, the algebra over the localization is not necessarily the algebra of the source. But you can still use this reduction to the proper case. Every map of analytic rings still factors as a proper map followed by a localization, though this localization won't have a left adjoint.

This functor is not fully faithful for a technical reason. To be fully faithful, you'd have to be able to recover the triangle just from the data of the D of R with its tensoring over condensed abelian groups. You can certainly recover the underlying condensed abelian group of the triangle, but there's not quite enough structure here in the animated ring context to recover the full animated ring.

As a corollary, if you take Y in f and consider f-shriek over Y, the category of X mapping to Y via a schable map, then the six functor formalism applies. We can look at the span category of this, but now we don't need to restrict to schable maps anymore, because every map in this category is schable. This span category is a priori lax symmetric monoidal, and in fact, it is symmetric monoidal.

For X to Y schable, the object D of X in modules over D of Y in PRL is dualizable with respect to the relative tensor product over D of Y, and in fact, it is canonically self-dual. With respect to this self-duality, the dual of the pullback map from X to X' over Y should be a map from the dual of X to the dual of the dual of D of X to the dual of D of X'.


So, you just get a map in the other direction from $D(X)$ to $D(X')$. This map is none other than the lower shriek functor. The proof is that you can check this universally in any span category. We have a symmetric monoidal functor, so if you want to, the image of a dualizable object will be dualizable, and if you exhibit a duality pairing here, you get one there and so on.

So, why is every object dualizable in this Span category self-dual? You have a unit, which is a counit, that is the same thing going in the reverse direction. Then you just check the triangle identities; it's completely straightforward. And then you can see that with respect to this self-duality of every object, passing to the dual is just the same thing as transposing the span.

So, we have a proposition: if $f: X \to Y$ is a schematic map, then the following are equivalent. Peter said this last time but didn't give the proof, and now we're going to give the proof. One is that $f$ satisfies shriek descent. Two is that for all $M$ in $D(Y)$, you have descent for $M$. In other words, what Peter equivalent---oh, yes, yes, yes. By definition, shriek descent means that you have descent with respect to this weird pullback, these real weird shriek pullback functors on the $D$'s. But the conclusion here is that you get it also automatically for the star pullback instead.

And then the third condition is kind of a categorified version of star descent, studied by Grothendieck and Rosenberg, which is a remarkable thing. If you take this whole category, then that assignment, assigning to $Y$ this category, also satisfies descent in the sense of $\infty$-1, but if you have---I'll explain more or less why it's the same, requiring this for $\infty$-1 and for $\infty$-2, but I'm not going to try to say I know what an $\infty$-2 means.

So, do you have also a shriek version of two? Yes, you have a shriek version of two, but that's completely formal from one. Let's get into the proof, and then it'll become more clear. Or maybe let me state a corollary first. A corollary is that a pullback $f$ satisfies shriek descent, implies any pullback also does. You can see that from two and using this symmetric monoidal that we discussed, but also, I think in the course of the proof, a simpler explanation will arise for why shriek descent is closed under pullbacks, which is important for using this to define a Grothendieck topology in the first place.

Okay, why the $P$? Because the $D$ of the---yeah, because the pullback in analytic rings is already calculated by a relative tensor product. So, if you took, for example, $M$ to be $D(Y')$, then this would give you star descent for the pullback, but you could take $M$ more generally to be any $D(Y')$-module. And then you see that condition two is stable under pullback, under base change.

So, we have star shriek descent, but I explained that the best way to think of that is that you have this colimit in $\mathrm{PRL}$, but I also explained that colimits in module categories are the same as on the underlying---so that's the same thing as modules over

Now, let's take the internal Hom out to $M$ for any $M$ in $\mathrm{Mod}_D(Y)$. Then, on the right-hand side, we're taking the internal Hom from the unit to $M$, so we just get $M$, and it's being identified isomorphically with some limit. So, the colimit pulls out of the internal Hom limit of the internal Hom in $D(Y)$ with $M$.

But I then explained over here that this thing is self-dual, it's dualizable, and in fact, self-dual. So then, you can actually move it over here with a tensor product. So, this is the internal Hom in $D(Y)$ with $M$, and I said that the manner in which it's self-dual is such that it converts shriek functors into lower shriek functors into upper star functors. So, then this is exactly a proof that one is equivalent to two.

Of course, you have to verify that the construction you get is really the expected one in some higher sense. You have to produce the data making this coherent identification of the dual of this self-duality and the dual of this, identifying with this, universally in the span category, and then map it into modules over $D(Y)$. I haven't actually done that, but there's also another independent argument for this which doesn't require such things. I just thought I would present this because it feels the clearest to me, even if it's slightly difficult to make technically work.

Okay, and then two implies three. We have to check this statement here. The map has a right adjoint, which is basically you take a system of modules here and you forget them all to $D(Y)$, and then you take the limit in this category here. So, it's like taking a system of modules $M_n$ and then you take the limit of the $M_n$. And you want to check that the unit and the counit of this adjunction are isomorphisms.

For the counit being an isomorphism, it's true after you base change up to $X$. This is because $D(X)$ is dualizable over $D(Y)$, which implies that you can pull the limit out. And then, the covert is split on pullback to $D(Y)$ or $D(X)$, and there's an obvious base change property.


We are completely okay then. Let's see, because it's dualizable, any limit commutes with it. But this is actually not correct from the start. Oh, I should change the whole system from D of Y to D of X, yeah, that's good. So then, thanks.

After base change to D of X, in other words, you take this whole collection here and you base change to D of X, the limit can commute. So you can do it in the system after tensoring, and after tensoring you get the system of the Adic categories for the fiber product space, a fine space, so to speak. Because we said that the D of the fiber product is D of the tensor product of the D for fine things, yes or no?

Okay, and then the covering is split, and so by the usual argument, this means that you can play with some Amitsur cohomology. Okay, then what is point three, the base change property? Oh, that's so that when you tensor the thing up to D of X, you can identify that system with the corresponding system for the pullback of the cover to D of X. So it's purely formal.

Well, actually, I think this argument is in Kirill Mathews' paper, and maybe it's also in Gates-Goresky-Rosen-Bloom, and so on. I mean, which you referred to as the original reference for descend ability, it's like the reference I wrote down last time.

Okay, so then this collection, and then the image of this, and then this, those two things are the same, isomorphic, after you tensor to D of X. But then two also implies that tensoring D of X detects isomorphisms. Done. Okay, it's a bit of magic, but you also get now a strong form of three with the two infinity.

Let me quickly explain the argument for three implies two, which is kind of more or less explaining why you get a strong form of three. If you take mapping spaces, if you have a what's that, Peter said. Right, you're right, that this argument shows that two is exactly the claim that the unit is an isomorphism. So clearly, three implies two, but I was going to give a different argument which maybe also explains why you get an infinity 2 categorical three implies two, because three means that this functor is an equivalence, but this functor has a right adjoint. So being an equivalence is the same as the unit and the counit being isomorphisms, but when you unwind what it means for the unit to be an isomorphism, you get exactly two. So clearly, three is stronger than two.

The idea is to go from $\Delta^n$ to this, and then if you let $N$ vary, you recover the whole well-completed simplicial space associated to this category, just in terms of mapping spaces.

I'm saying that what you get a priori is the space of objects in this $\infty$-category that you're interested in, but if you then vary the source by tensoring with presheaves on $\Delta^n$, and that tensoring kind of commutes with everything, then you get not just the set of objects, but the set of morphisms, the set of composable pairs of morphisms, the set of, etc., etc.

Yes, I was just giving it a little more explicitly. $\Delta^n$ is a finite category, and I have to put it in this big world of $\operatorname{Prl}$, so I just take presheaves of whatever spaces. If I want this to be a tensor over $\mathcal{D}$ of $Y$, then I should do presheaves with values in $\mathcal{D}$ of $Y$, and $\Delta^n$ is the simpli[cial] finite category or opposite.

Okay, so this was roughly half of what I wanted to do today, so let's go for another two hours, shall we? No, I'm kidding. We're taking a little break. I guess the next lecture is scheduled for the 10th of January, which is a Wednesday, so I'll be picking up and actually finishing what I intended to do today.

This was some consequence of \v Cech descent. Maybe just mention one more corollary of this, which is that the topology of \v Cech descent is subcanonical, so the \v Cech $\infty$-category fully faithfully embeds into the sheaf category. So those are some consequences of \v Cech descent---you get some very strong descent results. And next time, I'll talk about how you produce examples of \v Cech covers.


\end{unfinished}
% !TeX root = ../AnalyticStacks.tex

\section{\ufs !-topology (Clausen)}

\url{https://www.youtube.com/watch?v=vRUmXU8ijIk&list=PLx5f8IelFRgGmu6gmL-Kf_Rl_6Mm7juZO}
\renewcommand{\yt}[2]{\href{https://www.youtube.com/watch?v=vRUmXU8ijIk&list=PLx5f8IelFRgGmu6gmL-Kf_Rl_6Mm7juZO&t=#1}{#2}}
\vspace{1em}

\begin{unfinished}{0:00}

Let's begin. So I'll be talking more about this Zariski topology, but it's been a little while since we've last met, and so maybe I'll start with a recap.

Recall---well, let me start at the very beginning. We have this category of analytic rings. The objects in here are pairs---there are different choices about exactly what sort of data you want to put in the second piece. I'll take the sort of full derived category instead of some connective derived category or some abelian category, where this is a condensed---and in general, we want to say animated ring, so we're allowed to have some derived phenomena. This is a certain full subcategory of what you could call $D$ of $R$-triangle, which are triangle modules in derived condensed abelian groups, satisfying certain nice closure axioms for this, which I won't recall right now.

And then we define the category of affinoid spaces to just be the opposite category. And we singled out two---well, three, I guess---classes of morphisms of affinoid spaces. 

So $f$ from $X$ to $Y$ in $\mathcal{F}$ is proper if the pullback map from $\mathcal{O}_Y$ to $\mathcal{O}_X$ has a good right adjoint. Good means that the right adjoint commutes with pullbacks and also with colimits, and satisfies the projection formula. 

The maps called open immersions are those where the pullback map $\mathcal{O}_Y \to \mathcal{O}_X$ is a localization and has a good left adjoint, again in the same sense as before. For such an open immersion, there will always be some idempotent algebra in $D$ of $Y$ which is somehow the "functions on the closed complement" and determines this situation.

Called shable if it factors as $X \xrightarrow{f} Y$, where $f$ is proper and $Y \xrightarrow{g}$ is an open immersion. There are good closure properties for all three classes of maps: they are closed under composition, closed under base change, and contain all isomorphisms. Furthermore, if you have a map like this and a map like this, and they're both shable, then any map between them making the triangle commute is also shable.

We then get a six-functor formalism on $\mathcal{F}$, where the class of maps for which the important functor, the lower shriek functor, is defined is exactly the class of shable maps. The lower shriek functor $f_!$ is $f_*$ for $f$ proper, and $f_!$ is the left adjoint to $f^*$ if $f$ is an open immersion. This can be obtained by passing to the diagonal, as the closure properties imply that the classes are closed under passage to diagonals.

We then had a definition: a shable map $f: X \to Y$ in $\mathcal{F}$ is a shriek cover if the map from $\mathcal{D}(Y)$ to the limit of the Čech nerve is an equivalence, where we use the upper shriek functor for the transition functors. We then had a result that for shable $f$, the following are equivalent: (1) $f$ is a shriek cover, and (2) we have a Čech-type equivalence, where we use the lower shriek functors and take the colimit in the category $\mathrm{PRL}$ (presentable $\infty$-categories with colimit-preserving functors) or in modules over $\mathcal{D}(Y)_\mathrm{PRL}$. Here, $\mathcal{D}(Y)$ is a presentably symmetric monoidal category, so the tensor product commutes with colimits.

Colimits in each variable make it a commutative algebra object in PRL with respect to L's tensor product. Then you consider modules, so it's a presentable $\infty$-category which is tensored over $\mathcal{D}(Y)$ in a kind of colimit-preserving way.

The third condition was the "shriek cover" condition. You have this "shriek descent," where you use this funny twisted pullback. But it turns out that it implies that you get descent with the star pullback, and in fact it implies you get the same result with coefficients for all $M$ in $\mathrm{Mod}_{\mathcal{D}(Y)}(PRL)$. 

If you take $M$ to be $\mathcal{D}(Y)$ itself, then you see that this condition is just the usual descent, that is, descent with respect to pullbacks. But in fact, you can even tensor that with any module, and you still have descent.

The fourth condition was this kind of "two-descent," this categorified version, which says that the whole category of possible $M$'s satisfies descent. Here, the only thing that makes sense is star descent, but I'll emphasize it's just some naive pullback, where the base change functors are just relative tensor products in PRL.

One can ask, instead of just the fpqc topology in algebraic geometry, using quasi-compact, whether you have descent for an arbitrary collection of maps, not necessarily finite. We think that is true, that you could try to make an analogous infinitary version of the Grothendieck topology, and it will end up being finitary anyway. Peter checked this carefully, and the answer is yes - every cover will have a finite subcover.

In the case of just finite sets, if you have the conerter set and take all morphisms from $\P$ to the conerter set, which is big, this is universally submersive. This means that when you cross with any topological space, to check if something is open, you can check the pullback.

Of course, you can ask whether you have the right descent properties, which is not so clear. But at least for opens, it seems to work. Peter discussed this with some logicians, and they had some conclusions. It can happen that you have some profinite set and some infinitary cover by other profinite sets which satisfies the descent, at least in some examples, using the perspective of the derived "shriek" descent condition. Equivalently, along the "lower shriek," you can show that it is finitary, but it's really something that only works when you ask for "star descent," not for "shriek descent."

So those are the equivalent conditions for being a "shriek cover." Now, I want to talk about how you can check these conditions.
That image doesn't have to have any closure properties whatsoever. It doesn't have to be a stable subcategory or anything like that. But then you can take the thick subcategory generated by it. This means a closure under just finitary operations like cones, retracts, and shifts.

So, you get a priori a bigger category. You can ask whether it's the whole thing, or whether it just contains the unit. Let me write $1_Y$ for what you would normally think of as the structure sheaf of $Y$, the unit object in this symmetric monoidal category with respect to its tensor product.

The condition again is that this unit object can be written in a finitary manner in terms of things which are lower shriek from $\mathcal{D}(\mathcal{X})$. Can it be written in an infinitary manner? Yes, because it's a sheaf, it follows.

Oh, you're saying that if I say this condition but with infinitary, does it automatically imply the same condition but with finitary? I don't know what the relation between those is. I think the key point is that in the finitary case, this is a compact object in $\mathcal{D}(Y)$. So I think indeed they should be equivalent.

Maybe it's even better to say that if it's a proper, universally locally acyclic map, then you satisfy this condition. And the converse holds, provided either $f$ is proper, or $f$ is universally locally acyclic. Let me explain what that means: it means that $f_!$ is "good", which means it commutes with pullbacks, colimits, and satisfies the projection formula.

Those properties of $f$ are not true in general, and they're not equivalent - there are cases where some hold and others don't. There may be some non-trivial implications, but the projection formula clearly implies that $f_!$ commutes with colimits, so I didn't need to write that separately.

The main claim is that the "universally locally acyclic" condition is kind of like an equisingularity property for $f: X \to Y$, where the dualizing object is compatible with pullback. But the converse fails in general. Let me give a counterexample...


Category. So let's call this mapping by J, and this mapping by I. This induces a map from the disjoint union. Then J lower shriek of the unit is this compactly supported thing, which is the fiber of the homotopy fiber of ZT going to the series T-inverse. But I lower star, well I lower shriek of the unit there, I is proper, it's a closed immersion, so this thing just gives a Zariski series T-inverse. It's clear that you can build Z of T from this fiber and this guy by just one cofiber sequence.

Then the lower shriek from X the disjoint union will just be given by restricting to Z. Take the lower shriek there and restrict to Z, take the lower star there, and then take the direct sum of those two objects. So you'll get this guy direct sum this guy. Therefore, by closure under retracts, if you allow closure on retracts, you get each of them, and then you get ZT individually.

So it is not a sheet cover, because for the cover you just get Y and Z separately, somehow you get the derived of Y cross the derived of Z, not the derived of Y. Okay, here you can have some X or something that is not the same in Y, and then you lose some Z. 

So this condition is something like just set-theoretic surjectivity on underlying sets, or maybe it's like a cover in some constructible topology or something. And then if you want to go from that to some honest descent, you need to assume some properties. This is analogous to, in topological spaces, there's descent for open covers, but there's also descent for finite closed covers or proper descent, but you can't mix open and closed and still expect to get descent.

So this is some sort of set-theoretic cover condition, and then if you assume either that you're some generalization of open, which is this étale, or some generalization of closed, which is this proper, then you get honest descent. But you can't mix them.

So the proof - let's make explicit what this shriek descent according to the definition means. We're using these upper shriek functors, and there's some standard category theory which tells you that this comparison functor itself will then also have a left adjoint, which is given by taking a colimit in this category D of the G lower shrieks of them. So in particular for fully faithfulness, this amounts to the claim that if you take this colimit, it should be an isomorphism in D of Y for all M in D of Y.

If you take M to be the unit in Y, this is a geometric realization, it's a colimit over this Delta op. You can filter that, you can always write this as a colimit over the natural numbers and then a partial totalization. And this is a finite colimit, that's a nice fact about the simplex category. But then, since the unit object is compact, if you write it as a filtered colimit of something...

Let's look at sets of cardinality less than or equal to D. Since the unit interval Y is compact, we can deduce that it is a retract of some partial totalization, where each of these lies in the image of the shriek functor from D of X to D of Y. All of these structure maps from these iterated fiber products factor through D of Y. This is equivalent to a finite colimit, a colimit over a finite simplicial set. 

In the stable setting, the difference between this for D and this for D-1 is just given by one of these objects up to a shift. The successive cofibres are described in terms of these individual objects. So in the stable setting, some kind of rewriting simplifies all of this.

A finite colimit is defined in a higher topos. This is not more general than a colimit over a usual category, but you have to be careful, as a finite category in the usual sense might not be finite when considered as an infinity category or a simplicial set.

For part two, the hypothesis implies that every M are generated by things in the image of the shriek functor, using the projection formula. We want to deduce at least the fully faithfulness. We can assume that M is of the form F shriek N, and then with a base change result, we can reduce to a split situation.

And all of these maps $G$ are like compositions of pullbacks of $f$, so if $F$ has one of these two properties, then all of the maps $G$ will as well.

In the proper case, $G_!$ equals $g_*$, and the base change follows from the $f^* g_! \cong g_* f^*$ base change by passing to adjoints. And in the proper case, we have a sort of $G_*$ is $G_*$ of $1$ tending to $G^*$, and the base change follows more directly from again the $f^* g_* \cong g^* f^*$ base change. So there is some---you have to make sure the two base change comparison maps that you have are equal, but okay.

In order to conclude from one being an isomorphism that the other one also is, but okay. 

So that was that---proves fully faithfulness for two. So that gives full faithfulness in the $!$-descent, but the essential surjectivity, or so the other adjoint, or the unit or the counit, or whichever it is, the one going from here to there and then back up here again---that's proved in the same way, handled similarly using base change to reduce to the situation where the cover is split. So you're pulling back to $X$ where the cover is split, so we have a functor where we want to claim is an isomorphism. We've identified an adjoint to it, and what I've just explained is that if you do the functor and then the adjoint, that's the identity. And you'd also need to check that if you do the adjoint and then the functor, you get the identity, and I'm claiming that's handled similarly using base change along $X$ going to $Y$, where it happens for formal reasons because the cover is split.

Okay, so we're part of the way towards---so now we've, this is kind of a more concrete thing that you might hope to be able to check. So you have to be in one of these two situations. Let me explain some special cases.

Special cases. So one will be closed covers, finite closed covers. We defined a notion of a proper map and we defined the notion of open immersion, but we didn't define the notion of closed immersion. So if $f$ is proper and, well, one way of saying it is that the pullback from $\mathcal{D}_Y$ to $\mathcal{D}_X$ is a localization---the right adjoint exists, and the right adjoint is fully faithful. Nothing more, no, but a localization in category theory---in which context is defined now for categories---of which kind I forgot. I know that people like in Gabriel and some---I don't remember now what the. So let me say that---so this is a funtor which admits a right adjoint, so when the right adjoint is fully faithful, which is what I'm calling localization, it follows that you have a universal property for limit-preserving functors out of here, namely they're the same thing as limit-preserving functors out of here which kill every object, or sorry, which invert every map which is sent to an isomorphism by this functor. So this is an analog of localization for triangulated categories, for example.

Algebra: So tensors are the product of two copies of itself. $\mathcal{D}_{\mathscr{Y}}$ is itself again via the multiplication map.

Since $f$ is proper, let me make a warning. On the level of these $\infty$-topoi, it's not generally true that closed and open immersions are in bijection, with the same target. So an open immersion is not necessarily going to have a closed complement, and a closed immersion is not necessarily going to have an open complement.

It's close to being true that an open immersion has a closed complement. The only...Let me expand on this. Given an open immersion $U \to \mathscr{Y}$, we get an idempotent algebra $a$ in $\mathcal{D}(\mathscr{Y})$ such that $\mathcal{D}(U) = \mathcal{D}(\mathscr{Y}) / a$. But it's not true in general that $a$ lives in $\mathcal{D}^{\geq 0}(\mathscr{Y})$. This is the condition needed to get a closed immersion in $\mathsf{Aff}$. If you start with just an idempotent algebra, it doesn't necessarily correspond to a closed immersion in $\mathsf{Aff}$ because it doesn't necessarily have a correct underlying animated ring. It's only the connective ones that correspond to animated rings, not the non-connective ones.

In the case of usual schemes, can you recall what are the open... Well, it depends on which functor from schemes to analytic spaces you're using, so we'll go into it. But I want to say that this is analogous to some complementary phenomenon in scheme theory, where for an affine scheme, you have a closed immersion, but the open complement might not be affine. It might not correspond to an open immersion in affine schemes, but it's still a scheme. And it's kind of similar here, even in situations where this is not connective, usually you will get a closed complement which is an analytic space, it just won't be an affine one.

In the case of schemes, like taking the complement of $\mathrm{Spec}(A)$, you get $\mathrm{Spec}(A)$, which is important. But does this correspond to a closed immersion in this setup? When you take $\mathcal{D}(a)$, what do you get? Do you get $\mathcal{D}$ of some...which you call an open...a closed here? But I'd rather let me again get to the comparisons with the classical theories a bit later in the lecture, although this is going much slower than I anticipated.

I could give an example. If you look at $\mathscr{Y} = \mathbb{A}^2_{\mathrm{sol}}$ and $U$ to be $\mathbb{A}^1_{\mathrm{sol}}$, then I invite you to do the very good exercise of figuring out what this idempotent algebra is, and then the corresponding $a$ has a nonzero $\mathcal{H}^{-1}$.

The situation on the other side is somehow even worse. Given a closed immersion, well, actually Peter described the condition required for there to be a complementary open. A closed immersion corresponds to an idempotent algebra in the $\geq 0$ derived category of $\mathscr{Y}$. Then you need for there to be a complementary open in $\mathsf{Aff}$, you need that the internal $\mathcal{R}\mathrm{hom}$ from the fiber of the unit of $\mathscr{Y}$ going to $a$, which is a functor from $\mathcal{D}(\mathscr{Y})$ to $\mathcal{D}(\mathscr{Y})$, you need this commutes with filtered colimits and preserves $\mathcal{D}^{\geq 0}(\mathscr{Y})$. That's the formula for what would be the localization functor to the complementary open, and you need that that actually defines an analytic ring structure in our sense, which amounts to these conditions.

Right, so as I said, some of this will be fixed by allowing general analytic spaces, not necessarily

Okay, I was talking about finite closed covers right as an example of descent. So, suppose we have finitely many closed subsets $Z_1, \dots, Z_n$ with closed immersions into $X$, and all of these are in the image of some map $f$. When do we get a cover? Well, the disjoint union of the $Z_i$ mapping to $X$ is a cover if and only if the structure sheaf satisfies the most naive form of descent. 

This is actually going to terminate at a finite stage because it's a finite closed cover. The condition is that the structure sheaf is a sheaf, which is given by the tensor product of the item-potent algebras associated to the closed immersions. 

Why is this the criterion for being a sheaf cover? If this is satisfied, then we are in the image of the lower-star functor from the disjoint union of the $Z_i$, since each of these is closed. This implies $\mathfrak{f}$-descent, which in turn implies the fancy $\star$-descent, giving the structure sheaf condition.

Now, what about open covers? It turns out that every open cover has a finite refinement. So let's consider the case of a finite open cover $U_1, \dots, U_n$ mapping to $X$ by open immersions. The claim is that the disjoint union of the $U_i$ is a sheaf cover if and only if the tensor product of the corresponding item-potent algebras in $\mathcal{D}(X)$ is zero. The reason is that the unit object is compact, so this condition is equivalent to one of the algebras being zero, which happens if and only if the cover is a cover.

For simplicity, let's focus on the case $n=2$. Then the $\star$-descent condition gives that...


If you look at what $\star$ descent means and use the formula for the upper $\star$ functor, which is this kind of localization formula, then you find that the claim of $\star$ descent is exactly the claim that you have a pullback of this form.

Wait, I'm sorry, I'm getting myself awfully confused right now. No, I'm getting myself very confused. This is the $\star$ descent for the closed complement. I'm sorry, I'm sorry.

Of course, if something is a module over $A_1 \otimes A_2$ and it goes to zero on the $U_i$, so it goes to zero in each stage of the simplicial diagram, so it goes to zero apparently in this limit in the $\infty$-categorical sense. So if this condition is not satisfied, then there is a $\star$ descent.

I apologize, I kind of assumed I would be able to do this off the top of my head and I didn't think about it carefully. 

Let me say, the $\star$ descent for this cover, where you have two elements and both of them are mapping by monomorphisms into $X$, then the descent, which a priori involves some Čech nerve which is some infinite thing, it actually reduces to some Mayer-Vietoris. As is kind of standard, it's the same thing as $D(U_1 \cap U_2)$ being the pullback of $D(U_1)$ and $D(U_2)$, with the upper $\star$ maps. Then you can check that you have the map functor from this to the pullback, and again it has this left adjoint, so the claim for the unit gives that the unit of $X$ receives an isomorphic map from $J_1^{\star}$ of the unit on $U_2$ or $U_1$.

You have some kind of Mayer-Vietoris sequence like this, so this is a cofiber sequence. And then you have formulas for everything. I apologize for not explaining this very well, but you have formulas for all of the functors involved in terms of the corresponding idempotent algebras. If you work it out, it's just going to amount to the condition that $A_1 \otimes A_2$ is equal to zero, meaning that this condition will be directly equivalent when you write down what everything means to $A_1 \otimes A_2 = 0$. 

Open immersions are special, so $J^{\star}$ is one for open immersions. The $J^{\star}$ of this, how is it given in terms of the algebra? This term, for example, will be the fiber of the unit mapping to $A_1$. I made a mistake by not preparing this properly, because I thought it would just come to me, but yeah, I'm sorry for messing this up.

Let's give some examples now. The first example is Zariski covers. Note that there is a functor from the usual category of commutative rings to the category of analytic rings, which sends a commutative ring $R$ to the pair $(R, \mathrm{D}(R))$, where $\mathrm{D}(R)$ is the full derived category of $R$ modules in the category of condensed $R$-modules.

Viewed on the level of opposite categories, and maybe a remark is that this functor commutes with fiber products. In fact, it also sends the terminal object to the terminal object. In other words, relative tensor products in commutative rings are also relative tensor products in analytic rings, which follows from our discussion of relative tensor products in analytic rings. 

Moreover, the relative tensor product---I mean, the derived one---covers $f$ a map to $f$ a shriek covers. But now it's occurring to me that I forgot to remind what this means. So, note that we get a Grothendieck topology on $F$ by saying that a sieve over $X$ in $F$ is a shriek cover if it contains finitely many $Y_i$ mapping to $X$, such that the disjoint union $\coprod Y_i$ maps to $X$ as a shriek cover in the previous sense.

The key behind this, besides the obvious properties of finite disjoint unions, is that if you have a shriek cover, then any base change is also a shriek cover. This is a consequence of the discussion of colimits in PRL. Basically, the base change functor on the level of Mod-PRL just commutes with colimits, so if you have the condition there, then base changing, you get the condition. 

The proof is simple. Indeed, Zariski covers go to closed covers in the sense just discussed. If you have a principal open in Spec $R$, given by inverting some element, that inverting an element gives you an idempotent algebra, which defines a closed cover on the level of these guys. And the condition we had to check is just usual Zariski descent.


The whole formal neighborhood of that, and this then it acquires some fuzz. I claim that what you should really think is that the fuzz is making this thing really behave more like an open subset, and the Zariski open should really be thought of as closed, and it should have some kind of tubular neighborhood. So it should really be a closed subset, and then the formal neighborhood is the open complement. That's the picture I would like to suggest.

And when you go to the solid world, then you can again have an open version of puncturing. So you can name this boundary, maybe something like $\Z\langle\text{series } T\rangle$ base changed along $\Z[T]$ to $\R$ where $T$ goes to $F$, something like that. You can name the boundary and then you can remove it, so it's going to be a closed subset that you can remove, and you get an open subset. Then you're back to the usual way of thinking of having a Zariski open.

But then it's not---you take $\R[T]$, $\Z$, and then $\Z\langle\,\rangle$. So I'm saying once you move to the solid framework, then you can name this boundary here, which before was just heuristic, and its name is this. That will give it an input in algebra in $\mathcal{D}$ of $\R[z]_{\text{solid}}$, and the complementary open is like $\mathcal{D}$ of $\R[1/f(z)]_{\text{solid}}$, so $\Z[1/f(z)]$.

I mean, there are two different ways you can embed schemes into analytic spaces---one is based on sending $R$ to $R\langle z\rangle$, and the other is based on sending $R$ to $(R, R)$, and this is the one I'm discussing right now, where you base change to the solid $\Z$. And then the Zariski opens look closed, but if you use this one that corresponds to always removing the boundaries, then the Zariski opens actually go to open immersions.

So what do you mean by boundary? I don't really know what I mean by boundary, if I don't just mean the formal neighborhood minus the middle. Intuitively, I'm claiming you're removing kind of an open piece from this chunk, and then the closed complement should intuitively have some boundary. I mean, there will be boundaries at infinity too if your thing isn't proper, and what's happening at infinity is also important, but let's pretend we're in a proper thing or something.

In particular, we get a functor from Zariski sheaves on derived schemes to sheaves on $\F$, which is in fact the pullback of some topology. So it commutes with colimits and finite limits, which is a consequence of this. This is one way of embedding usual algebraic geometry into what I haven't quite defined what an analytic space is, but it's close enough for practical purposes. We're going to have some kind of hyperdescent condition we want to impose as well, which I thought I was going to get to today but I clearly didn't. But this is basically analytic spaces or analytic stacks, modulo a couple of technicalities.

Conditions and when you analyze this, you use the weaker one with which the Serre-Swan theorem covers, although if you did a stronger one, it would still give the same claim, because it would be just a further localization of this. On the other hand, yes, well, I'm not going to touch the other side - one could, but the risk is that it also depends on whether you have... Yes, yes, it does. But I'm not going to care too much about that side. It was said in some talk, I don't remember who, that maybe Peter said that there would be something intermediate between Čech descent and full hypercover.

I was going to discuss it today, but it's not going to happen. Okay, so the other subtlety is a set-theoretic subtlety, because the category of commutative rings is not small, so you have to be careful considering presheaves and sheaves on it. And the same if you take only those which are accessible - yes, yes, exactly. And then you have to prove that sheafification preserves this accessibility and so on and so forth. But okay, I don't think in the remaining 7 minutes I could do justice to the next topic, which was going to be adic spaces. I could rush through it right now, but I don't think that's a good idea, so I'll stop here.

I wanted to know what, for example, should cover... Oh, oh, oh, oh, sorry, I didn't understand your question. I'm sorry. Whether in a tall cover also gives you a sheaf cover - yes, it does, it does. Right, so this is something I should do in a few minutes.

So this was all motivated by Matthew Emerton's theory of descendability. So we were talking in AF, and then we were saying we have this derived category of anything in AF, and it's built on this condensed framework. But it's clear that the discussion is very categorical in general, and we could just try to make the exact same definitions in the world of ordinary algebraic geometry instead, using the usual derived category of a ring, and see what kind of definitions that gives.

If you take the same definitions, but with the pair R and DR, where R is a usual commutative ring or maybe derived, and DR is the usual derived category, then some simplifications happen. First, every map is proper, which is quite clear - well, maybe that's the main simplification that happens, that every map is proper. This implies that every map is also sheafifiable, and then the proposition I discussed earlier also holds in this setting, where there's no condensed thing involved.

The upper shriek and lower shriek here are not the same as the ones in Grothendieck duality theory - they're just some categorical adjoints. Nonetheless, it turns out to be useful in this descendability discussion, because every map is sheafifiable and every map is proper. And then we deduced that a shriek cover, X to Y, is the same thing as saying that the unit in Y lies in the image of F lower star D of X to D of Y, and this is exactly the definition of descendability in Matthew's work, or one of the equivalent characterizations he gives.

What about the classical Grothendieck flat and faithfully flat descent? Is it related? Let me give some examples. Actually, every kind of purely formally descendible map, R to S in an $\infty$-category sense, gives a sheaf cover in the usual sense via this functor. And Matthew gives many examples of descendable maps, such as a tall cover.

So, this is kind of funny. You need this, well, we apparently probably really need this countably presented hypothesis. That means that you have a map from $A$ to $B$ which is faithfully flat, but also $B$ is presented as an $A$-algebra, or maybe even just countably generated is enough, so $B$ has a presentation as an $A$-algebra with countably many generators.

The condition is that on $\pi_i \Z$, it is faithfully flat and the $\pi_i$ are countably generated. This is related to some limit, higher limit. I mean, if it is a limit of $A_n$'s, that's fine too, but in practice it's the same thing. If it is only countably generated, then you don't get it. You need it to be countably presented, because you need to be able to reduce to a countable base ring.

The point is, in a lot of situations where you have classical descent, you get this even stronger Čech descent, which also gives you descent and much else besides. But also, there's some issues if you have $R \to R/I$ where $I$ is a nilpotent ideal. The Čech descent is defined on the level of derived categories, so the descent you get for this does not imply that $D(R)$ is the same as $D(R/I)$ obviously.

The reason is that when you do the descent, you're doing everything on the derived level, and you end up with terms like this in the limit diagram, and those are not the same as $R/I$. Going to $\pi_i \ZR$, does it have the same property as $R \to \pi_i \ZR$? No, you can have a polynomial algebra with a degree 2 generator, and this has a module which by inverting that degree 2 generator goes to zero when you mod out by $X$, but is non-zero.

So, the analog for simplicial rings of a nilpotent ideal is that you should ask that the ring be truncated, so it has only finitely many homotopy groups. Then going to $\pi_i \ZR/I$ would work.

More generally, if you have a proper map of derived schemes, then you also get Čech-descendable. That's a generalization of this. And more generally, any faithfully presented cover, that's kind of a combination of this and that. There's a huge class of very, very much, but you do have to remember that in non-flat cases, the descent involves a higher topos and so on. All of these kinds of things will go to Čech covers in our setting.

Question: Was it equivalent to have a map of commutative rings be descendible in the ordinary commutative algebra Matthew sense, or for the image under this functor to be a shable map, a strict cover in our sense?

So, what's the argument? This is important to understand in this kind of relative sense. The descent and this pro-object formalism mean that this pro-object is just an old discrete pro-object. Okay, yeah.

Peter was pointing out that there's another characterization of descendibility, which is in terms of just the rings - like A, B, B tensor A B, and so on. You have to have descent, you have to get a limit diagram. But then you have to get even more: you have to get a very stable limit diagram. It has to be a pro-isomorphism between the tower you get from this co-augmented thing, the n-index tower, and that pro-object should be pro-isomorphic to the constant pro-object A.

In that condition, it's clear that it's independent of which framework you put it in, because the pro-category - the usual D(R) sits fully faithfully inside D(condensed R), and therefore Pro of usual D(R) sits fully faithfully inside Pro of D(condensed R). So for the pro-thing, it's enough to work in the homotopy category - it's enough to work in a weaker framework.

What you can do is pass to the fiber, and then you want to know about a tower being pro-zero. It's enough to look at it in the homotopy category. Okay, thank you again for your attention. 

When a tower is pro-zero, is it then the case that it is uniformly pro-zero? That is, because there is some finite domain, it's enough for any stage to add a fixed number? Exactly, yes, in Matthew's work. And it works with Simpson, who works in a very general setup, just like commutative algebra objects in presentably symmetric monoidal categories. He only looks at the proper case, so every map is proper in our sense.

So the setup is a symmetric monoidal category C, and then you consider an algebra A in C. He defines descendibility in this context. It specializes to the condensed world but only discusses the proper maps, not the arbitrary shable maps.

\end{unfinished}
% !TeX root = ../AnalyticStacks.tex

\section{\ufs Analytic stacks (Scholze)}

\url{https://www.youtube.com/watch?v=T9XhPCI8828&list=PLx5f8IelFRgGmu6gmL-Kf_Rl_6Mm7juZO}
\renewcommand{\yt}[2]{\href{https://www.youtube.com/watch?v=T9XhPCI8828&list=PLx5f8IelFRgGmu6gmL-Kf_Rl_6Mm7juZO&t=#1}{#2}}
\vspace{1em}

\begin{unfinished}{0:00}
So today, we will finally define what analytic stacks are. It's not so difficult. Recall that we have this category of analytic R-spaces. This was actually, I think, presentable. That's been proven - it has all co-limits, and it's generated by a set worth of the usual analytic rings and spaces.

We can make the following definition. I don't expect this to be accessible, but the word "accessible" is just there to deal with some set-theoretic issues. From analytic rings or Fr??chet analytic spaces, toward... Recall, the word "accessible" here, there will be a condition in just a second. This means that it commutes with filtered colimits for sufficiently large U. It's such that, sorry, let me rephrase the conditions as follows.

We have some open cover U of A, and then some hypercover U of A, for which the colimit of the nerve of this cover is isomorphic to A. Also, X of A is isomorphic to the spectrum of the colimit of the sheaves on this nerve. By the way, I write "unspec" where Dustin just wrote "spec" - I don't think we've settled on a final notation, just not to get it confused with the usual spec, let me write "unspec".

This is some form of descent that's somewhere strictly between... Can I ask just a minor technical question concerning the notion of hypercover? In usual algebraic geometry, do you hear me? Yes, I hear you.

So let's consider the case in usual algebraic geometry when you have, for example, an fpqc cover or something like this. There could be two meanings of fpqc hypercover. One meaning is that the map from X0 to X and the map from Xn to the coskeleton are fpqc. The other meaning is that there are coverings for the fpqc topology, which is H. Sometimes it causes some concern, what is the meaning. 

In our context, you use the strict meaning, I suppose. That everything, I mean, you... Yes, so I want all maps to be fpqc, but then there is no difference between, if it's refined by a stable cover, then it's already a cover because of the S condition. Ah, okay, so here it is satisfied that if some, if A is dominated by it's already fpqc cover, so okay, thank you.

Right, and I also need to say that commuting finite products, so commuting products including the empty product, geometrically just means that X evaluated on a disjoint union is the product.

So this condition is, I mean, up to issues, and it's something that's somewhere between 2 and 3. For Čech sheaves, you would only consider some Čech nerves of covers, and those would, by definition of what a Čech cover is, always satisfy this condition, so for Čech nerves of Čech covers, you always have this condition, so it's always a Čech sheaf. But for Čech hypersheaves, you would ask this condition for all Čech hypercovers, whether or not they satisfy this condition, but we definitely want our derived category to satisfy, to be, some abs property, so we need to restrict to classifying the Čech for which this holds.

Let me just state this as a remark and not prove it. Sheafification of an accessible pre-sheaf is successful. This is the analog of the theorem of Waterhouse. This means that you can actually form colimits in this category, because sheafification forms the colimit in the category of sheaves. But then you need again to enforce the strong sheaf condition, so you need to sheafify, and you can check that it preserves the set-theoretic assumptions.

Examples or remarks: For any... You can look at the functor which takes B to the maps from A to B. I have a small technical question concerning the sheaf-ification, which I believe you mean, sheaf-ifying it to satisfy this precise condition, that is not the Čech sheaf and not the Čech hypersheaf, but this intermediate condition. Yes, but then in order to construct this...



Unification---it seems that you need to know that the pullback of this kind of hypercover satisfying the condition on the $\infty$-category is also such a hypercover. But again, this follows from the same argument. So this is actually equivalent to $D(A)$ being the colimit along the Lourish maps of the $VA_\bullet$ in the $\infty$-category of presentable $\infty$-categories. This is a condition that base changes, because the $\infty$-categories base change well.

You use all of this---this was discussed in the previous talk. This is all magic about $\infty$-categories. Then you use it, it is the tensor product when you base change.

Okay, so first of all you have all the fine objects. For any affinoid ring $A$, you can consider the object $\mathrm{Spf}(A)$ which takes any $B$ to the homomorphisms from $A$ to $B$. This already satisfies all the required properties, like commuting with products.

Now you have a hypercover, some $B_\bullet$, and then you want to map to $B$. In particular, you want the limit of the $B_\bullet$ to be $B$. So you map to all the $B_\bullet$ and you also map to $B$ itself, because it's the limit.

This brings up the analytic rings. This fully faithfully embeds the analytic rings into the $\infty$-stacks. The accessibility condition is precisely that you're a small colimit of objects in the essential image. This is similar to how we think about condensed sets.

So what is the $\infty$-category of analytic $\infty$-stacks? For any analytic ring, there is an object which is the analytic spectrum. All the others are built by some gluing procedure, by some colimit of these fine objects. The hypercover condition tells you the ways in which you're allowed to glue.

As will become clear later, we want this very general topology because it means that analytic spaces that seem completely different can actually be the same object, just represented in different ways as a colimit of fine pieces. Often there is a geometric picture that they should secretly be the same, and to say they're really the same, we need to use rather strong topologies.




Rings, and then there's a way to extend them to the full class, but some left extension. And then, I think this way can probably show that...okay, okay. 

All right, so there is a fun from like derived schemes. The derived schemes, they also of course embed into like the so-called $\infty$-topos. We can go to taking particular, like any Spec $\operatorname{Spec} A$ for a ring $A$, and take a condensed thing, which is really just the and, um, and all condensed modules. And if you want, you can also, as Dustin discussed, you can basically put the fpqc topos in here, except you have to do this funny countability assumption, only one fpqc.

So basically, like any fpqc stack can also be mapped, and there's really not much to show here, right? I mean, you definitely just have to spin on rings, and then you just have to show that whenever you have a $\infty$-topos cover, it goes to such a strict cover, but that's basically by definition. And again, you could also use this funny thing between sheaves and hypersheaves here in the $\mathcal{C}$-topology here if you wanted to.

Right, uh, so ah, right, maybe I can mention that the pro-étale site is actually fully faithful into the condensed $\infty$-topos view. It's a full back where you actually uses the $\mathrm{Tan}^{*}$ functor. Do you do you need $QC$? I think you can get rid of this because you can certainly reduce to the quasi-compact case, and then there's this argument of offer that you can write as a limit of quasi-separated things. You can how do you reduce to the quasi-compact? Well, you can certainly, if you want to understand Homs from $X$ to $Y$, you can cover $X$ by, but you can $X$ compact, but then comp-wise they can definitely comp. Oh yeah, okay, there other argument that you can make the compact so can.

Alright, so you have some algebraic geometry sitting in there. Uh, next you have Stein spaces sitting in there, which well, these are good from the fromology one, so these are a plus. Like when you talk about Stein spaces, you always assume that this is Stein, and so what I'm saying now, let me also assume this because otherwise I would run to some longer discussion about exactly what I want to do.

And so I can just restrict this to the one plus of the analytic ring $A^{+}$ solid. So note that this way we get a different fun, get two different functions. So there are actually two ways to embed schemes into analytic spaces. You can take a Spec $A$, you can also match this to $\underline{\operatorname{Spec} \mathcal{O}_{A}}$. And then for there's a question in the chat, Peter, it says for derived schemes you're mapping the trivial analytic ring structure, exactly here I'm currently using the trivial analytic ring structure.

Right, no. And so you can either now take this further and look at $A$-modules and solid $\Z$-modules, or you can look at relatively solid $A$-modules. And so now there, and if you wanted to, you could put it right here suitably formulated and put it then also put it derived here. So then there are like three functors from schemes or derived schemes to analytic stacks,

Of course, this immediately suggests a generalization: for any analytic ring $A$, we can consider the "relative" scheme over $\Spec A$, which can be viewed as analytic spaces over $A$. To do this, we need to consider the "condensed" version of $A$, which has an underlying discrete ring. We can then look at schemes over this condensed ring.

For example, we might have the complex numbers with their usual ring structure, and then consider the usual schemes over the complex numbers as a special case of this framework, by just taking the condensed version of the complex numbers.

Now, you mentioned something slightly puzzling about "derived Huber pairs". This refers to a derived version of the notion of a Huber pair, where we have a derived ring with a $\pi_0$ part and all the higher $\pi_i$ parts. I don't want to go into the details of this here, but it is something that can be defined.

In terms of the fullness of the functor, we know that it is definitely fully faithful on the $\F_p$-linear case, because then it reduces to just maps of rings, and we know that Huber rings are fully faithful for analytic rings.

For the general case, we know it's fully faithful under some mild coherence assumptions. But we don't have a full fully faithfulness result, basically because we don't have a good version of an "analytic Grothendieck topology".

There's also the fact that, in contrast to the scheme case, the "adicity" functor doesn't commute with pullbacks. So you have to be a bit careful there. But if you restrict to the "Tate-adic" or "adic" case, then it behaves as expected.

There's also an interesting feature on the right-hand side: if you take a fiber product of $f$-analytic spaces, it's always just the usual tensor product of the corresponding rings. This is not true in the "adic" case, where you can have a fiber product of adic spaces that is not itself adic anymore. There's some subtlety there that can be explained in a nice way using some derived techniques, but I don't want to get into that here.

You can also do this "non-Archimedean geometry" over the real and complex numbers. For example, you can have complex analytic spaces mapping to analytic spaces over the Gaussian complex numbers. The main issue here is that complex analysts usually don't tell you what an "analytic space" is, but they could. Any complex analytic space can be written as a union of what we call "Stein" subsets, which behave very much like affine objects.

What is a Stein subset? For example, it could be the vanishing locus of some ideal inside a polydisc of complex numbers of absolute value at most 1. The algebra of functions we put on such a Stein set is the algebra of holomorphic functions defined in some neighborhood. It turns out that this Stein algebra is excellent, and if the Stein set is actually a manifold, it's regular and has all the nice properties you could hope for.

So when talking about coherent sheaves on a complex analytic space, you can really just talk about finitely generated modules over these Stein rings, for each Stein subset. There is a small caveat, which is that you could have an ideal such that the number of connected components of its zero locus is infinite, and this would prevent the Stein ring from being Noetherian. But for Stein sets that are actually the Riesz closure of a polydisc, the Stein algebra is Noetherian.

Politics is not arbitrary or compact. Okay, I remember that it was for another compact. 

Right, but if you look at complex geometry, there is a very close analog of the notion of an étale subset. Everything really has an algebra, and everything is very similar to how you do rigid geometry. Similarly to the Cech complex, which is actually a cosimplicial object over the algebra of continuous functions on the spectrum of that algebra, endowed with the Gauß-Hecke complex norm. This has a natural topology, like uniform convergence on compact subsets. So this also has a natural topology. You could also define direct and ind settings, whatever. It's a so-called dual nuclear Fréchet space.

Again, this is fully faithful on the analytic structure. You view it as an algebra in the Gauß-Hecke theory, and then you just take the induced analytic structure. So you just check on the underlying ring, which defines a condensed ring that is actually an algebra over the complex numbers. Because it's dual nuclear, it's actually Gauß-Hecke, and you can just induce up the Gauß-Hecke analytic structure from the complex numbers to here.

Not every dual nuclear space is Gauß-Hecke, but these are actually even nuclearly Gauß-Hecke. What is projective? It means some locally closed immersion into projective space, so maybe like an open subset of a closed subset.

Before I go to the next example, there's actually nothing special about using complex manifolds. You can do similar definitions for real analytic spaces, smooth manifolds, or even $C^0$ topological spaces. You take the ring of continuous functions and do the same construction. In each of those cases, you can decide whether you want real or complex-valued functions, and they give you two different functors where one is just the base change of the other.

Now comes a more involved example. This is showing that all the usual theories of algebraic geometry and geometry that are known can be incorporated into this framework. You could also directly define some notion of stacks. You can imagine many possible notions of a complex analytic stack by taking sites with complex analytic spaces and defining a stack over that topology, maybe just open covers.

But now comes the final example, relating this back to condensed sets. Since everything has become a stack, let's take condensed non-étale sheaves. These actually map into the framework we've been discussing. This works even for just schemes, without needing to go analytic. There is something condensed on the right, namely the analytic rings that were condensed rings, but we're not really using this condensed structure here - it comes from the speed objects.

Right, so if $S$ was a finite set, it would just be a finite disjoint union of copies of $\operatorname{Spec} \Z$. In general, as $S$ is the limit of finite sets, this is the limit of these finite schemes. And so then you can access further to the $\mathcal{I}$. 

And here's the reason that we chose this really funny version of between chiefs and hyper chiefs. So, by definition, they are hyper chiefs of a certain profin-topos, the one that we always use. And so to get this, you have to show that if you have any hyper cover here of a profin set by profin sets, then it goes to something for which you enforce the $S$ here. It definitely goes to a hyper cover, and so then there's a small thing you have to check that it actually is the same condition on the $\mathcal{D}$-category.

Basically, the argument that factorially flat maps satisfy the $S$-condition also proves that countably presented sheaves satisfy it. So I should say that here, the lightness condition is again important, because it's always true that if you have a stative map of profin sets, then if you look at the corresponding map of continuous functions, it's always flat for any profinite space. But we need a descendible map for our business, and so we only know that countably presented factorially that maps are descendible, which is another reason that we have this lightness condition.

All right, so here's a remark. There are different ways of embedding schemes into $\infty$-stacks, and now there are also different ways of embedding something like topological manifolds into $\infty$-stacks or the complex or real numbers. Either you can do the thing where you use the algebra of continuous real-valued functions to do this, or you can treat the topological manifold purely as a topological space or factorially into condensed sets and then go from condensed sets to $\infty$-stacks. This gives you a different thing, which is actually defined over the integers, right, because this $\infty$-stack just has integer-valued functions. These are completely different incarnations of a topological manifold, but there's actually again a map between them.

So let's consider the following example. You can take the two-sphere as a topological space and then treat it as a topological manifold, and this gives you an $\infty$-space which, in the compact case, it really is just the $\mathrm{UN}$ of the algebra of continuous functions. And let me work everywhere over the complex coefficients, which category of, let's say, $\mathcal{C}$-sheaves. What else could I do? I could take $S^2$ and treat it as a real analytic manifold and again build an analytic space over the guess, the complex numbers. Or, you could think about all other possibilities, like $C^\infty$, real analytic, and so now you would have here the algebra of--what's a good notation for real analytic functions? $\Omega$, yeah, $C^\Omega$ sometimes it's called. And I don't know, think about all possible other algebras like $C^\infty$ functions, $C^k$ functions, between they would all also give $\infty$-spaces.

But then, you can also like $S^2$ is like one version of $\P^1$ of the complex numbers, so you can also treat this as a complex space. And well, then you want to get something with the guess, the complex numbers. This is not $\mathcal{O}_\F$ anymore, right? I, there are not so many global homomorphic functions here. It is what it is, it's not $\mathcal{O}_\F$, but it's some glued from two copies of $\mathcal{O}$ over the disc, over the holomorphic functions on the disc, right? So you can cover this by two discs, glued along the over-convergent holomorphic functions on the circle. And similarly, this would be glued, but from two copies of $\mathcal{O}$ over the disc, glued along the over-convergent holomorphic functions on the circle.

Okay, so clearly, like making it more



Take again just $S^2$ and $S^3$ just as a condensed set, and go to the space of functions on this complex. So here, I'm using the functor from Lecture 6. Basically, what happens is that here you want to take the locally constant functions on $S^2$. Well, there are none on $S^2$, but if you secretly think of $S^2$ as being the quotient of a profinite set by a profinite relation, which you can always do by this condensed perspective, then on those profinite coverings you do have locally constant functions, and then you can do this gluing. 

Secretly, to evaluate what this is, you have to remember that secretly $S^2$ is a quotient of a profinite set by a profinite relation, and then you can find locally constant functions. Here you have algebraic functions, here you have holomorphic functions, here you have real-analytic functions, and here you have continuous functions---all of them give you different incarnations of what the two-sphere might be.

There's already one really funny thing, which is something that Dustin already mentioned a couple of lectures ago, that one incarnation of the GAGA principle is that it's actually an isomorphism of analytical spaces. So this is one kind of GAGA statement, which is now not just a statement about some coherent sheaves or some derived category of coherent sheaves or whatever, but really before you pass to linear algebra, on the level of spaces, there is an isomorphism.

I mean, this is an instance where you see how the way you build these things is completely different---one is built from just continuous functions, one from holomorphic functions, and the strong work topology that we impose is precisely there to ensure that you can have the possibility of such interesting results.


The C of modules, of course, is just the category of modules of the algebra C. A vector space with an action of the continuous functions. But as here, there are just sheaves of C-vector spaces on the topological space.

One can show that if X is maybe locally compact, then the sheaf has a question: Peter, why do we consider the gas analytic ring structure on the left?

The reason is that otherwise, this is unclear. For example, if you want functions from complex spaces towards analytics, you need the intersections to match up. If you have an intersection of compact line subsets, then the intersections should again be something compact Stein. And you want this to be mirrored on the side of analytics text in our picture. Which means that if you compute the corresponding completed tensor product of analytic rings of functions on D and the other disc, and then glue them together, you should get functions over convergent functions on S1. And this is a computation that comes out correctly in the gas analytic ring structure, but definitely doesn't come out for the usual analytic ring structure, because then you would just take the usual tensor algebraic tensor product, and this is just nonsense. So you need to complete the tensor product, and then it comes out right.

So for X-dimensional schemes, then you can look at the category of Tex, where A is any, and B changes to the spectrum of A. And this is just sheaves on X, sheaves on just X, some reasonable topological space. Those values, so yeah, so you can think of general sheaves as being some kind of functions on an associated space.

I realized before that I should have said that the functors that take any spectrum A towards the A-modules, or also towards presentable $\infty$-stacks, are linear over T, satisfy the send conditions, and hence induce functors on all things analytic. So for any stack, you can define the direct image and descent, and also you can define what is like a sheaf of categories over X. So here it is again the intermediate condition, but for this functor version, between the center type for any. This will actually both of these things will actually be part of some general six-functor formalism, and at some point we will talk more about that. And basically, all design criteria for the GR topology was actually precisely that, that for any analytic d, we definitely want to be able to talk about the C-sheaves. And at some point we also realize it's probably good to enforce that we can also talk about sheaves of categories, and then we are basically taking the strongest possible GR topology where this is true.

What is Perfil? That is the thing which is gotten by descent from the association Spec A goes to the category of A-modules in PERL, which is a 2-category, yes, a 2-category, but we don't have to care, we can just see it as an $\infty$-1 by neglecting the non.

Right, okay, so this is some general examples that maybe we want to cover in more detail in the remaining lectures, but I did want to come back to the point where we started discussing analytics, which was the construction of the T of the curve, and discuss this in a little more detail now, because now we have the language of talking about this.

So let me recall what we want to do. We have this universal ring, denoted Brr, endowed with the topological unit 2, such that when you take this usual Guy, which is a free Guy on a normal sequence, and you tensor it up to the string, then the operator that's 1 minus Q* shift, so shift is the endomorphism of P that just shifts all the integers on S, and this was this ring structure we introduced some weeks ago. And where an on-computation was that you could actually compute what's the underlying string of this was, and it was this algebra of Laurent series in Q, which has a certain funny to normal condition on their coefficients. So the underlying condensed ring is this algebra, but maybe the underlying ring of the underlying condensed ring, because now I'm just telling you a set, maybe.


Did I tell you the current condensed structure on this? Okay, so this is some funny ring of formal power series, integral series, with rather strong growth of the coefficients. And then we recall that the goal was to finalize the analytic curve over $\mathbb{A}^1$ or really some $\mathbb{G}_m^\triangle$.

This would be a scheme such that if I take the incarnation of $\mathbb{E}_Q$ as a scheme over $\mathbb{A}^1$, this can be written as a quotient of a $\mathbb{G}_m$ over $\mathbb{A}^1$ under multiplication by $Q$ as an operation. And I already a couple lectures back gave the outline of how I thought it should go.

The first step is to see that there is a certain kind of norm, and I will talk about this more in just a second, which goes from the $\P^1$ array, but really the $\P^1$ incarnated as a scheme over $\mathbb{A}^1$, toward the infinity. Intuitively, this tells you how large a point is. Having a point here is basically having a point here, so maybe nothing, think from a field or something like this, and then an element of that residue field. So there's something like saying that on any residue field, there would be a norm $\Z \to \Z$, like the absolute value, but now this map here is really meant to be a map of analytic stacks.

The left-hand side is now an analytic stack. We can take $\P^1$, make it into an analytic stack by this general functor that gives it a trivial ring structure, but then we can base change it to $\mathbb{A}^1$. And maybe here I should write that I mean, so here you're incarnating this by treating it as a condensed set, and we could or could not base change it to $\mathbb{A}^1$. Yeah, we can. You have to choose some value for the absolute value of $Q$, like say $q$. So it should stand for some $q$, $q$ is a section of this, and it's multiplicative in fact, is a unique such thing with the properties I will mention. 

So it should be multiplicative in a sense I will make precise in just a second. So I won't talk about these norms second. Let's say, to make it unique, I should precisely specify where $q$ goes. Meaning, like $q$ is actually a section from the spectrum of $\mathbb{A}^1$ back into this, and this should go to a constant value, something like half. And then you can define the analytic $\mathbb{G}_m$ array as a subset of the $\P^1$ array as the preimage of an open subset $\Z\setminus\{\infty\}$. And then you still have $q$ acting there, and then can take the quotient by $\Z$. And then, by an argument I already sketched last time, this is basically a projective curve. And then you have to prove the zero-dimensional algebraization theorem that would apply to all projective curves with a section, at least, that it's algebraic, so it is in the image of different schemes over the underlying ring.

Towards the end, I mean, you need to make the line, it's not clear even geometrically, but here it's okay. We discussed this, you need some GAGA or exact GAGA, and something which is again where is a soft, soft. Well, in our approach to GAGA, you don't really get SAGA, which is much more general than GAGA, but it doesn't really tell you anything like Riemann-Roch for something you don't a priori know is algebraic. I mean, so that's kind of a separate issue.

Right, so one key notion that comes into this is a notion of norm that we want to have here. So let me actually define this in general. It's a slightly awkward notion of a norm $\mathcal{A}$ that is not really a norm on the underlying set of $\mathcal{A}$ or anything like that, but it's rather


One other thing I want to mention is about the notion of "norms" for elements. We will also follow something, and here's how I want to phrase that. So inside the $\P^1$, I can first look at the locus $\mathbb{A}^1$ where I didn't really have just functions, but then I can look at the locus where the function is "near-potent", and that is some locus given by the zero locus of a certain polynomial. And then you join the two variable, so in other words, it's again just this $\P^1$ treated as an algebra. This maps to the algebra $\mathrm{UNP}(a)$, which is just another ring. Instead of taking the norm for some $s$, I want to say that this composite factors over the closed interval $[0, 1]$, not the half-open one. It turns out that the better notion is this one, for reasons I can't fully explain yet, but it's the one that behaves well.

I should also say that such a factorization here turns out to be really just a condition, because there's really a monomorphism, and similarly for these other conditions like zero goes to zero, I want to say that it factors over the subset $[0, \infty]$ again, that's just a condition.

Alright, so these are some first properties. But then when you have a "norm", you usually ask for some version of multiplicativity, and also some behavior with respect to addition. It turns out we just forget about the addition part, but we keep the multiplicativity condition. However, I'm not working on $\mathbb{A}^1$ but $\P^1$, and the reason I work with $\P^1$ is that even on $\mathbb{A}^1$ I would want to allow some functions that could have infinite norm, and if I'm allowing infinity on the target of this map, I might as well allow it on the source as well.

Now for multiplicativity, I have to be slightly careful about handling the zero times infinity cases. One way to do this is as follows: I consider the locus $X$ inside $\P^1 \times \P^1 \times \P^1$ where $XY = Z$, and similarly there is a smooth surface $\mathrm{xar}$ inside $\P^1 \times \P^1$. On the open part, you can look at the incarnation on the level of schemes of this closure, and one incarnation just on the real numbers, and then you have $X$ incarnated as a schematic morphism to $\P^1$, incarnated as a choice, and this matches to $\Z_{\infty}^3$, and inside here you have this $\mathrm{xar}$.

I wonder if it would be the same as asking first that the map giving the norm is equivariant for inversion on both sides, and then just asking for multiplicativity when you restrict to $\mathbb{A}^1$. I think so, yeah, that was kind of the definition I thought we were using, but maybe...

Yes, but even on $\mathbb{A}^1$ you have to be careful, because $\mathbb{A}^1$ can map to $[0, \infty]$, it could map to $\infty$. Oh, I should have said that I should have said the pre-image of $0$ instead of $\mathbb{A}$

Might the norm of two be unbounded? Usually, the absolute value of two is less than or equal to two, which implies a triangle inequality. However, there could be an example where the absolute value of two is infinite. 

On this Gaussian analytic base, I will try to construct such a norm in the remaining minutes. In particular, we can then look at what the norm of two is. It's some function from the analytic spectrum of a Gaussian towards the extended integers, and it's surjective. 

The method I already stated on the board is that there is a unique norm on the Gaussian space, taking some number between zero and one. It doesn't matter which one, because you can always rescale the target by some exponential and still be alright.

Let's say x is some kind of finite-dimensional locally compact space, and y is then I claim there's some kind of Tannakian duality describing what maps from y to x are. It turns out that to give such a map from y to x, it's just enough to give a functor from sheaves on x towards sheaves on y, with some property. But actually, you don't really need to specify the values of all.

Let me first see. This is the same thing as the functor F from the category of abelian groups on x towards the category of abelian groups on y, such that these functors are compatible with all pushouts and pullbacks, and they should be linear over the base field. The condition is that locally on y, the following happens: on x you have lots of important algebras, namely for any closed subset of x you can take the constant sheaf. This should be connected, and the image under the functor F should remain connected. In general, giving such a map, you can always do it locally, so the condition you have to enforce is just some strict local condition.

It turns out that this is the condition that locally on y, the pullback or the sheaf locally for the topology remains connective. And note that the full category of sheaves on x is actually generated by these guys on the closed subsets. So to define the standard functor, you really only have to declare these important algebras. Describing the map from y to some such guy here is completely determined by specifying, for each closed subset of x, an important algebra, and if they're all already connected, then you're good to go. The only thing you have to check somehow is that the Tor product behavior of these important algebras exactly matches the intersection behavior of the closed subsets of x, and that's what it means to be a tensor.

This is actually a set, even though a priori it's an analytic object. To realize this space over the appropriate category, we need to find some important object. Typically, you have to find something that should correspond to the pre-image of an interval from zero to some number $R$. If you know where these pre-images should go, then due to the compatibility on the involution, you also know where intervals going from somewhere to infinity should go. Everything else can then be written as some kind of colimits and intersections of those.

Really, to describe the SC norm, you just have to say what is the pre-image of the interval $[0, R]$, in other words, what's the locus where the norm is at most $R$. This should be the analytic spectrum of a certain ring, generated by sums of $n \cdot T^n$ that converge for $R' > R$. 

To produce such an object, you can start with a sequence in $A$ and multiply it by suitable powers of $Q$ to make it satisfy the desired condition. This can be described as a colimit of $T$ subject to a condition on $R$, which can be explicitly written down in terms of the Gauss algebra. The key idea is that the absolute value of $Q$ should be $1/2$.

If you form certain tensor products of such algebras centered at 0 and $\infty$, the tensor products will behave in the way you'd hope, matching the intersection behavior of intervals from 0 to $\infty$. This is a computation that relies on working over the Gauss algebra.

I should probably stop here, as I'm running over time. Let me know if you have any other questions!

Theories, yeah? So my question is, can you define an analytic space theory using your language? Because if I wanted to put an additional structure in an analytic space, perhaps I want a general definition for all analytic spaces - put the structure there. That's exactly what we're doing; that's what an analytic stack is. So you can say, "I have an analytic space theory, and this reduces to each known case."

Well, I - are such analytic space? Analytic space theory? I don't know what - yeah, I have to know what you mean by that. I mean, like something that would generalize all the... Yeah, that's what we're - that's what we're trying to do with this concept of analytic stack. Yes, you shouldn't be scared, it shouldn't be put off by the fact that we change the name from "space" to "stack" - that's basically a technicality, but that's the goal.

So you have a definition, like "an analytic space is an analytic stack that..." Well, you could try. Yeah, that's my question - like, do you have a definition, like "an analytic space is an analytic stack satisfying some axioms"? Do you have... Well, we have too many - maybe. Well, what I mean is, like, do you have the ultimate one, kind of like the umbrella one? From - we we tried, but we could never...

I mean, you could just say that the functor of points takes values in sets. No, F-ones are not okay, sorry. What if you ask that it takes values in sets? And even fine analytic spaces are not - not fine. Well, yeah, but the classical ones are. So anyway, the classical, the F-line, like, takes a ring to its underlying... Because the test category consists of derived things, you can... Yeah, yeah, that's true, that's true. No, but anyway, you don't have a definition of analytic... But I mean, you could ask that it's the counit of F along monomials. That's the - sorry, what's the condition? You could ask that it's the counit of F along monomials.

Okay, okay, thanks.

\end{unfinished}
% !TeX root = ../AnalyticStacks.tex

\section{\ufs Normed analytic rings (Clausen)}

\url{https://www.youtube.com/watch?v=wk_wInYTasQ&list=PLx5f8IelFRgGmu6gmL-Kf_Rl_6Mm7juZO}
\renewcommand{\yt}[2]{\href{https://www.youtube.com/watch?v=wk_wInYTasQ&list=PLx5f8IelFRgGmu6gmL-Kf_Rl_6Mm7juZO&t=#1}{#2}}
\vspace{1em}

\begin{unfinished}{0:00}
  e
okay  so  now  um  continuing  the  dis  I  want
to  continue  the  discussion  from  last
time  I  want  to  start  by  um  maybe  going
over  a  little  bit  of  what  Peter  said
perhaps  adding  um  some  details  uh  or
some  explanations  so  the  the  main  topic
is  uh  normed  analytic
Rings  um  so  we  had  this  now  we've
officially  defined  this  framework  of
analytics  stacks  that  we're  working  in
um  and  we  saw  uh  that  there  was  an
embedding
of  uh  well  light  condensed  anama  uh  not
well  at  least  a  funter  we  saw  that  there
was  a  funter  to  analytic
stacks  and  it's  induced
by  so  these  things  are  generated  by
light  profinite  sets  so  say
t
and  to  one  of  those  you  assign  I'm  just
going  to  write  it  I  know  this  is
probably  overloaded  notation  but  you
just  you  have  this  discrete  ring
of  this  is  a  discret
ring  of  continuous  functions  on  your
profinite  set  with  values  in  the
integers  um  it's  a  discrete  ring  um  and
you  can  view  it  with  the  uh  maximal
analytic  ring  structure  so  uncompleted
uh  so  you  look  at  all  condensed  modules
over  this  discrete  ring  or  derived
condensed  modules  over  that  discrete
ring  and  um  that's  the  analytic  ring  uh
that  you  associate  and  then  its  spec  is
an  analytic
stack  um  and  uh  the  thing  that  makes
this  funter  nice  is  that  this  uh  this
assignment  here
um  this  uh  sends  hyper
covers  uh  to  shriek  hyper
covers  uh
satisfying  uh  shriek  Cod  descent  or
descent  uh  so  so  apparently  I  don't  we
Pro  that  those  are  equivalent  sh  descent
C  descent  this  the  talk  oh  um  it  it's
the  exact  same  argument  to  you  mean  in
the  hyper  case
or  okay  I  I  remember  Peter  refer  to
something  like  this  in  some  explanation
so  the  when  you  have  a  you  can  formulate
one  that  the  categor  is  equivalent  to
the  one  to  the  inverse  limit  by  sh  or
the  direct  limit  by  and  those  are  it's  a
purely  formal  fact  about  PRL  so  proved
by  lur  that  those  are  equivalent  so  it
doesn't  need  anything  on  the  shape  of
the  diagram  or  or  whatever  yeah  um  so
it's  yeah  uh  so  these  were  the  things
that  uh  these  were  the  things  which
defined  the  notion  of  analytic
Stacks  um  but  it  also
preserves  finite  limits  so  for  example
pullbacks  well  let  me  say  finite
limits  so  if  you  take
a  pullback  of  light  profinite  sets
that's  a  that's  just  a  a  filtered
inverse  limit  of  pullbacks  of  finite
sets  and  then  it  goes  to  a  filtered
co-limit
of  uh  the  same  situation  for  finite  sets
so  this  reduces  to  finite  sets  and  for
finite  sets  it's  kind  of  completely
clear  um  and  that  so  and  those  two  facts
kind  of  imply  that  this  funter  here  so
that  that  the  induced  funter  here  that
you  get  from  this  uh  so  this  is  a  cimit
preserving  uh  Co  limit  and  finite  limit
preserving  um  which  is
nice
um  so  um  now  I  want  to  mention  uh  that
uh  that  uh  there's  a  paper  on  the
archive  by  Rock  uh  gregoric  uh  so  so  he
studed  is  the  analog  of
this  uh  with  a  analytic
stack  replaced
by  by  uh  fpqc  C
sheaves  um  in  in  usual  sheaves  fpqc
sheaves  in  in  usual  algebraic
geometry  over  a  field  um  and  he  wrote  a
nice  article  which  he  recently  posted  on
the  archive  um  which  goes  into  some  some
detail  about  properties  of  well  I  mean
he's  working  in  a  slightly  different
setting  but  much  of  it  is  much  of  it  is
the  same
um  yeah  so  I  would  recommend  reading
that
uh  okay  um  right  okay  and  then  uh  so  the
the  the  other  example  that  Peter
discussed  was  uh  example  of  a  light
condensed  set  let's  let  me  start  with
this  setting  so  let's  say  uh  K  is  a
compact
how  is
DF  I  think  a  question
the  how  does  Rec  topological  space  from
the  associated  antic  St  is  it  the  set
state  of  point  in  some  suitable
sense  so
the  I  don't  see  that  there  is  a
well-defined  notion  of  underlying
topological  space  of  an  analytic  stack
um  so  us
yes  that's  right  it's  a  you  have  you
have  an  underlying  uh  light  condensed
set
um  but  um  I  don't  see  an  underlying
topological  space  Oh  yeah  I  I  guess  then
you  could  map  this  to  topological  spaces
okay  so  you  have
underlying  so  when  you  have  a
stock  uh  no  the  stock  is  is  something
with  it's  a  generalization  of  ring
spaces  here  but  what  is
well  we're  not  really  using  like  a
diagram  You  cover  by  SP  by  kind  of
formal  spec  of  analytic  rings  then  you
have  the  the  the  world  superal  diagram
things  and  then  how  do  you  get  Space  how
do  you  get
the  you  cannot  take  spec  in  the  no
although  I  think  you  can  consider  all
open  substat  maybe  it  makes  sense  in
your  set  to  consider  open  you  could  you
could  use  monomorphisms  of  analytic
Stacks  to  try  to  there's  different  ways
to  okay  there's  actually  different  ways
to  try  to  extract  an  underlying
topological  space  from  an  analytic  stack
and
um
well  yeah  I'm  not  sure  it's  worth  saying
anything  completely  General  in  the  world
of  arbitrary  analytics  Stacks  but  um
well  once  when  we  introduce  this  notion
of  normed  analytic  ring  we'll  see  that
when  you  have  a  normed  analytic  ring
then  um  there's  a  nice  way  of  extracting
underlying  topological  spaces  for  things
over  that  normed  analytic  ring  let  me
let  me  continue  the  story  having
unsatisfactorily  not  addressed  the
question  um  so  let's  uh  like  so  if  you
have  a  compact  house  dworf
metrizable  and  let  me  add  finite
dimensional  and  um  this  actually  uh  is
the  same  thing  as  saying  that  K  is
embedded  as  some  closed  subset  of  uh
some  finite  product  of  copies  of  the
closed  unit  interval  um  so  these  things
are  very  easy  to  imagine
um  uh  then  well  then  we  can  view  this  as
a  as  a  light  condensed  set  in  particular
a  light  condensed  ana  um  and  then  we  get
an  Associated  uh  analytics
bace  um  and  Peter  made  a  claim  about  uh
the  fun  of  points  for  this  for  such  a
for  the  analytic  space  Associated  to
such  a  k
um
so
um  well  or  maybe  there's  there's  two
claims  so  and  two
claims  so  the  first  is  that  so  first  of
all  what's  the  most  important  invariant
you  have  if  if  you're  given  an  analytic
stack  it's  its  derived  category  so  we
saw  that  shriek  implies  star  descent
which  means  that  the  derived  category  of
a  stack  is  well  defined  by  pullback
upper  star  pullback  um  and  uh  this  um
this  D  of  K  I  mean  the  analytic  stack
Associated  to  K  by  abuse  of
notation  uh  this  is  just  sheaves  on
K  uh  with  values  in  derived  to  bilon
groups
um  what's
that  uh  thank  you  yes  thank
you  and  more  generally  if  you  base
change  this  stack  to  an  analytic  ring
then  you  put  uh  D  of  R
there
um  uh  right  uh  and  the  second  claim  is
that  if  you  want  to  map  say  Spec  R  for
an  analytic  ring
R  to  this  k
um  uh  this  is  equivalent  to  giving  a
symmetric  monoidal
functor  uh  from  sheaves  have  now  not
condensed  aelan
groups  uh  to  D  of
retric  monoidal  cimit
preserving  uh  funter  uh  such  that  uh
shriek
locally  on  r  on  Spec
R  what's  that  oh  thank  you  yes  yes  yes
DZ
linear  yeah
um  such  that  Shri  locally  on  R  we  have
that  uh  F  of  the  connective  part  so  she
K  with  values  and  say  d  of  Z  greater
than  or  equal  to  zero  um  lands  inside  D
of  R  greater  than  equal  to  Z  the
connective  part
there
um  so  how  much  the  fin  where  is  the
final  dimensional  use  is  it  some
technicalities
about  I  don't
know  you  can  Define  the  analytic  stock
for  any  case  without  Dimension  right
right  so  the  finite
dimensionality  um  I  we're  going  to  see
uh  where  the  finite  dimensionality  comes
in  I'm  going  to  sketch  the  the  argument
for  this  um
so  okay  so  well  the  first  remark  I  want
to  make  is  that  both  well  the  left  hand
side  is  a
set  well  maybe  I'll  say  both  are
sets
um  so  we're  in  this  world  of  analytic
Stacks  which  is  based  on  some  sheaves  of
Ana  so  a  priori  everything  is  an  Ana  I
mean  your  K  is  implicitly  some  co-limit
of  things  and  could  be  introducing
higher  homotopy  but  because  of  this
property  that  this  funter  preserves
pullbacks  um
so  well
so  um
so  this  is  a
monomorphism  uh  in  condensed  Ana
in  in  light  condensed  Ana  uh  so  by  the
property  of  pres  and  monomorphism  means
when  you  pull  it  back  along  itself  you
get  itself  again  nothing  fancy  then  you
get  the  same  in  analytic
stacks  and  that  tells  you  that  um  if  you
have  two  maps  can  agree  in  at  most  one
way  so  you  have  a  set  and  not  a
generala  um  and  over  here  the  reason
this  complicated  looking  space  of
functors  is  a  set  and  not  anama  is  that
um  you  know  for  well  it's  D  ofz  linear
so  it's  actually  the  same  thing  as  just
if  you  take  I  don't  know  sheaves  of  an
and  you  just  ask  for  a  CO  limit
preserving  functor  like  this  um  and  then
this  is  generated  by  these  uh  like  the
representable
ones  um  and  uh  but  well  if
we
um
uh  but  uh  so  in  particular  over  here
everything  is  determined  by  what  it  does
to  the  free  things  on  these
representable  guys  um  but  then  we  can
pass  to  the  complimentary  closed  and  we
and  it's  also  determined  by  the  um
so  by  the  um  so  this  this  thing  is
determined  by  uh  where  it
sends  uh  you  know  the  the  the  con  and
chief  on  a  closed
subset  um  and  then  this  should  just  uh
should  be  an  item  potent
algebra  in  the
Target  or  should  go  to  an  item  potent
algebra  under  f  it  should  go  to  an  item
potent  algebra  in  the
Target  um  and
um  and  then  uh  you  just  have  to  have
some  conditions  that  are  satisfied  so
you  just  have  to  specify  a  collection  of
item  potent  algebras  in  the  Target  and
then  some  conditions  have  to  be
satisfied  so  let  me  write  down  here  so
um  so  this  is  what  implies  that  uh  left
hand  side  is  a
set  but  uh  for  in  the  right  hand  side  um
is  just  the  set  of  uh  maps  from  you  have
closed  subsets  of  K  to  item  potent
algebras  in  D  of  R  uh
such  that  um  uh  inter  filtered
intersection  uh  goes  to  filtered  Co
liit  um  and
Union  uh  so
uh  Union  goes  to  well  the  analogous
construction  so  there's  a  if  you  have  a
union  of  closed  subsets  you  can  kind  of
use  a  Myer  viat  Taurus  to  express  the
the  uh  constant  sheath  on  the  Union  in
terms  of  the  constant  sheath  on  the  two
pieces  and  the  intersection  and  that
gives  kind  of  an  algebraic  formula  for
what  the  value  on  the  union  should  be
which  you  can  write  down  just  at  the
level  of  item  potent  algebras  and  you
ask  that  that  that  Union  goes  to  the
that  algebraic  construction  you  have
here
um  and  then  and  then  you  still  want  this
uh  this  condition  here  but  that's  just
another
condition  um
so  this  looks  like
sh  this  looks  like
Theory
K  the  direct  category  she  on  K  yeah  yes
this  uses  the  fin  Dimension  yes  that
does  yeah  okay  and  then
the  the  equivalence  with  sending  various
pieces  the  equivalence  of  giving
something  on  the  on  CL  on  H  on  closed
outpaces  is  it  does  it  use  this  doesn't
use  no  so  when  you  work  switch  one  with
shifts  or  so  far  I  haven't  used  finite
dimensionality  if  you  take  this  in  the
sense  of  lury  then  everything  I  say
Works  uh
generally  so  for  I  think  an  arbitrary
topological  space  k  but  um  certainly  for
a  compact  house  door  space
um
uh  okay  um  right  and  so  the  the  point  is
that  item  poent  algebra  is  form  a  this
is  actually  just  a  post
set  um  so  there's  a  a  prior  it's  an
Infinity  category  but  one  checks  that
that  it's  just  a  postet  so  a  map  if  it
exists  is  UN  you  can  uniquely  write  it
down  um  so  then  we  just  have  a  the  right
hand  side  is  just  you  have  you  just  have
to  give  a  map  of  poets  to  specify  all  of
this  um  a  map  of  postet  satisfying  some
simple  conditions  to  specify  such  a
symmetric  monoidal  blah  blah  blah  blah
blah  functor  and  so  on
um  the  next  thing  to  note  is  that  okay
we  should  now  now  I'm  claiming  these  two
sets  are  in  bje  and  I  should  first  write
a  map  and  the  map  uh  the  map  in  this
direction  is  obviously  going  to  send  F
uh  to  F  upper  star  where  F  upper  star  is
um
um  so  well  a  priori  D  of  K  well  D  of  K
is  this  thing  but  we  can  restrict  to  the
full  subcategory  where  um  you  require
the  values  to  be
discreet  um  or  I  could  say  dcon  Z  linear
functors  from  uh  shees  on  K  with  values
and  dcon
z  um  and  then  I  so  but  I  should  that
gives  one  of  these  things  but  I  should
maybe  explain  why
the  uh  that's  a  derived  category  of
condensed  a  bilan
groups  light  condensed  yeah  light  sorry
light  yeah  okay  so  you  have  a
uh
uh  so  F  goes
to
uh  spec  ah  by  definition  spec  out  okay
is  H  is  a  map
from
the  first  you  have  to  give  a  map  on  the
Ring
of  but  this  is  part  of  this  functor  the
map  on  rings  is  is  included  in  KN
the  no  so  what  is  it
your  F  from  spe  R  to  K  means  that  you
have  a  of  analytic  R  well  it  means  you
have  have  some  say  shriek  hyper  cover  of
this  uh  and  a  map  of  that  shriek  hyper
cover  to  the  hyper  cover  of  this  by
profinite  sets  oh  because  K  is  not  okay
yeah  and
uh
okay
um  and  if  is  totally  disconnected  then
it  is  the  same  as  a  map
of  if  K  is  totally
disconnected  yeah  so  yeah  so  maybe  yeah
if  if  K  is
profinite  then  the  left  hand  side  is  the
same
as  uh  map  of
rings  uh  from  continuous  functions  on  K
with  values  and  Z  to  uh  just  the
underlying  ring  of  this  uh  an  analytic
ring  that's  by  by
construction  um  that  those  are  the
same
um  okay  and  this  actually  explains  why
this  condition  is  satisfied  I  mean  for
arbitrary  K  because  for  arbitrary  K  you
can  Sur  from  a  profinite  set  T  and  then
this  is  a  a  surjection  so  by  definition
there  will  be  some  shet  cover  Spec  R  uh
where  you  factor  through  a  map  to
T  but  once  you  factor  through  a  map  to  T
then
you're  uh  your  F  uper  star  is  just  being
induced  by  this  functor  but  this  is  just
a  filtered
cimit  of
uh  of
um
uh  yeah  filtered  colon  of  C  of  like  K
nz's  where  this  is
finite
um  and
um
well  what  do  I  want  to  say  the
um  so  the  the  so  well  what  what  I  want
to  say  maybe  yeah  I  don't  know  if  that's
the  right  remark  to  make  so  in  the  case
when  K  is  profinite  then  so  you  want  to
prove  that  the  image  of  every  connective
object  is  connective  what  do  you  know
you  know  that  the  image  of  the  unit  is
connective  because  the  unit  has  to  go  to
the  unit  in  D  of  R  which  is  this
underlying  ring  which  is  connective  by
definition  um  but  then  uh  on  any  clopen
subset  we  then  have  to  go  to  something
connective  because  it  will  be  a  retract
of  that  unit  um  but  then  the  clopen
subsets  generate  the  connective  part
under  co-  limits  because  they  generate
all  open  subsets  under  co-  limits  and
then  everything  is  generated  under  that
by  Co  limits  so  everything  will  be
everything  uh  in  this  thing  here  in  the
case  where  this  is  profinite  everything
here  will  be  a  cimit  of  retracts  of  the
unit  objects  and  therefore  we'll  land  in
here  um  in  the  case  where  case
profinite  um
so  in  this
case  the  connectivity  is
automatic  um  so  that's  why  this  is  is  a
well-defined
map
um  and  uh  if  you  want  to  go  backwards  so
given  such  a  symmetric  monial  functor  or
such  an  association  with  itm  potent
algebras  um  it's  enough  to  go  backwards
uh  assuming  this  condition  because  you
can  work  shriek  locally  because  both
sides  are  shriek  sheaves  or  sh  satisfy
shriek  descent  so  to  go
backwards
um  uh
conversely  uh  if  you're  given  such  a
symmetric  monoidal
funter
um  let's
see  on  the  connective  level  um  then  you
want  to  produce  uh  this  map  here  um  what
you  can  do  is  you  can  take  uh  oh  sorry
this  was  okay  uh  you  can  take  this  hyper
cover  by  take  a  hyper  cover  by  profite
sets  or  just  well  I  guess  just  a  cover
is
fine  once  you  start  with  a  surjection
from  a  profinite  set  oh  sorry  what  am  I
doing  uh
uh  um  each  of  these  will  automatically
be  profinite  sets  because  they'll  be
closed  subsets  of  the  product  of  two
profinite
sets
um  uh  then
um
uh
then  uh  then  you  get  a
corresponding  uh  diagram  of
rings  uh  communative
algebras  uh  in  D  of  R  greater  than  or
equal  to  zero  so  you  just  take  so  if
this  is  pi  uh  so  Pi  Z  and  then  this  is
pi  1  and  then  Pi  2  and  so  on  so  you  have
um
Pi  0  lower  star  of  the  constant  Chief  Pi
1  lower  star  of  the  constant
chath  and  so  on  um  have  the  constant
Chief
um  and  these  things  are  all  going  to  be
connective  um  and  in  fact
um  you  have  AA
formula  uh  which  gives  that  this  is  a
this
thing  uh  is  the  I  a  check  the  co-  check
nerve  I  don't  know  of  just  the  first
map  um  so  because  these  maps  are  proper
uh  any  map  of  pro  you  know  compact  house
door  spaces  is  proper  for  using  proper
base  change  you  can  see  that  uh  like  the
whole  the  whole  diagram  is  just
determined  by  the  first  part
um  so
um  uh  and  now  you  can  apply  your
symmetric  monoidal  functor  F  so  you  can
take  uh  F  of  this  pi0  lowest  r  z  and  now
this  is  a  commutative  algebra  object
um  uh  well  first  of  all  I  should  be
saying  r  Pi  lower  St  Z  but  it's  actually
concentrated  in  degree
Z
yeah  yeah
um
um  and  then  in  fact  you  can
say  in  fact
animated  um  so  it's  not  it's  not
difficult  to  produce  the  animated
structure  but  let  me  not  get  into
it  um  and  then  you  take
uh  uh  to  R  Prime  to  be  the  analytic  ring
defined  by  um  f  of  Pi  0  lower  star  Z
modules
uh  and  then
um
um  just  the  induced  analytic  ring
structure  of  you  you  just  take  this  ring
object  in  here  and  look  at  that
underlying  ring  and  just  modules  over
that  ring  in  in  the  thing  you  already
have
um  and  then  you  can
check  uh  that  this  procedure  uh  induces
a  map  from  the  check  nerve  of  this  RP
Prime  mapping  to  R  so  spec  so  spec  RP
Prime  mapping  to
R  to  Spec
R  uh  and  then  by  the  construction  you
get  an
uh  so  is  this  z  underl  z  the  same  Z  I'm
sorry  where  are  we  on  the  top  board  yes
z  underl  z  yeah  and  then  have  Z  right
are  they  the  same  they  are  not  the  same
so  uh  this  one  maps  to  the  constant  Chi
Z  and  uh  that  one's  the  the  the  co  the
the  co  co-  fiber  when  Z  is  the
complimentary  closed  subset  to  the  open
subset  U  so  they're  not  the  same  but
they  determine  each  other  in  a  canonical
way  so  you  said  something
to  so  this  is  actually  a  single  sh  in
any  case  it's  a  some  f  because  it's  def
is  a  symmetric  mon  you  get  some  Comm
object  some  der  like  in  you  say  there  is
canonical  way  to  view  it  is  a  simpl
right  right  and  but  this  is
not  I  think  in  characteristic  zero  it
should  be  yes  in  characteristic  zero
it's  it's  clear
not  because  of  the  abstract  way  you
define  you  have  the  condition  it's  not
clear  that  you  that's  right  yeah  it
requires  a  little  extra  argument  so  you
can  see  for  example
that  yeah
it  requires  an  extra  argument
but  so  probably  the  work  that  you're
doing  with  Maxim  will  make  it  more  clear
because  so  I  so  what  what  uh  what  what
is  being  pro  probably  carried  out  at  the
moment  is  a  kind  of  a  categorical
perspective  on  animated  commutative
rings  so  they  you  can  s  single  out  what
properties  of  the  derived  category  of  an
e  infinity  ring  or  the  C  you  should  just
say  category  of  modules  over  an  e
infinity  ring  what  structure  do  you  need
to  put  on  that  to  promote  the  E  infinity
ring  to  an  animat
uh  thing  and  it's  something  like  derived
symmetric  power
functors  and  um  you  have  them
on
uh  you  have  them  on  I  mean  maybe  there's
an  even  simpler  argument  sorry  I  think
in  this  case  there  might  be  a  simpler
argument  so  you  want  to  produce  a  sorry
uh  let  let's  let's  leave  that  aside  for
the  moment  and  address  it  at  the  very
end  okay  so  yeah  but  yeah  sorry  I
um  okay  so  where  are  we  ah  yes  so  then
what  we
get  is  a  map
from  the  check
nerve  of  Spec  R
Prime  mapping  to  Spec
R  uh  to  the  check
nerve  of  uh  t0  mapping
Decay  and  um  then  we're  going  to  be  then
we'll  have  produced  a  map  by  by  by  kind
of  by  descent  we  have  produced  a  map
from  Spec  R  to  K  so  then  if  we  just  so
we  as  long  as  we  know  that  this  is  a
cover
okay  um  but  note  that  this  is  a  proper
map  by  construction  we  built  it  as
something  where  the  so  to  speak  the
completeness  condition  isn't  changing
when  you  go  from  R  to  RP  Prime  um  you're
just  extending  the  ring  so  by  the
Criterion  we  had  for
shable  uh  things  it's  enough  to
see  uh  that  the  unit  object  which  is  our
triangle  uh  lies  in  the  category
generated  by  the  image  of  a  so  in  the
image  of  the  the  forgetful  functor  uh
sorry
D  of  R  Prime  to  D  of
R  and  for  that  by  applying  F  it's  enough
to
see  using  yeah  the  lower  shriek  but
lower  shriek  is  lower  star  so  it's
really  just  a  forgetful  functor  from  and
the  image  this  means  the  image  and
closing  by  closing  by  say  finite  Yeah  by
cones  and  retracts  and  so  finitary
operations  yeah
okay  um  then  applying  F  it's  enough  to
see  that  the  constant  chief  on  K  is  in
the  subcategory  generated  by  the  a  lower
star  functor  so  H
sorry  sorry  sheaves  of  t  on  T  values  in
DZ  U  so  Pi  lower  star  sheaves  on  K
values  in  D  of
Z
okay  um  or  in  in  in  uh  or  in  ail
Matthew's  language  we  need  to  see  that
uh  Pi  lower  star  of  the  structure  chif
on  t0  is
descendible
right  um  now  remember  that  we  this  was
kind  of  we  could  choose  this  arbitrarily
so  it's  actually  enough  for  me  to
produce  a  cover  by  a  profinite  set  for
which  I  can  check  this  descend  ability
um  so  we  can  reduce  to  by  pullbacks  we
can  reduce  to  k  equals  like  01  to  the
N  um  and  then  I'll  I'll
discuss  just  the  case  n  equals  1  because
it's  simpler  to  write
down
uh
so
um  gu  you  can  also  reduce  to  this  case
what's  that  yes  that's  true  by  by  kith
yeah  you  can  also  reduce  to  this  case
yeah  that's  true
yeah  so  this  means  when  you  take  the
fiber  P
some  so  there's  some  yeah  we  didn't  get
into  much  details  about  how  to  make
these  sorts  of  arguments  but  aille  gave
some  tool  kits  which  are  nice  um  so  then
in  this  case  you  take  the  the  usual  kind
of  caner  set  so  you  have  the  so  I'm
going  to  produce  my  cover  of  the  closed
unit  interval  by  by  uh  by  the  caner  set
so  what  I  but  let  me  write  it  in  a
specific  way  so  I'm  not  going  to  not
going  to  write  so  profinite  said  a
priori  you  want  to  write  it  as  an
inverse  limit  of  finite  sets  but
sometimes  you  can  instead  write  it  as  an
inverse  limit  of  compact  house  door
spaces  and  in  the  limit  you  just  happen
to  get  a  Prof  finite  set  and  that's  the
best  thing  to  do  here  so  what  we  can  do
is  we  can  take  you  know
uh  the  two  halves  of  the  closed  interval
and  and  their  disjoint  Union  is  a  space
mapping  to  the  unit  interval  um  and  then
we  can  do  the  same  thing  again  this  the
the  base  to  the  base  to  exp  of  the  yeah
so  then  that's  a  so  then  we  have  a
sequence  of  spaces  mapping  to  the  unit
interval  where  in  the  inverse  limit  you
actually  just  get  the  caner  set  um  which
is  then  forming  a  cover  of  the  closed
unit  interval  so  what  does  this  mean  on
the  level  of  these  push  forwards  so
these  the  push  forward  from  the  caner
set  will  then  be  the  sequential  Co  limit
of  the  push  forwards  from  each  of
these
um  now  there's  a  general  fact  about  this
descend  ability  that  if  you  have  a
sequential  co-  limit  it's  enough  to
establish  descend  ability  with  uniform
exponent  of  nil  potent  for  each  of  the
finite
ones  um  and  for  each  of  the  finite  ones
you  have  descend  ability  basically
because  you  have  a  my  vus  cover  for  the
closed  subsets  and  there's  the  point  is
the  reason  you  get  a  uniform  bound  on
the  exponent  is  because  in  each  case
there's  only  double  intersections  that
you  need  to  be  concerned  about  and  no
triple  intersections  so  in  the  end  you
get  something  like  exponent  of  n  pot  two
for  each  of  these  individual  things  and
then  three  or  four  for  the  the  co-  limit
or  something  like  this  each  and  the
composition  of  the  individual  thing
is  no  when  you  have  the  individual  in
because  to  adjacent  one  you  have  only
adjacent  and  then  so  when  you  divide  K
you  claim  that  this  is
a  ah  okay  and  then  there  I  understand
that  when  you  go  to  Power  like  01  to
some  power  then  the  the  the  estimate
will  become  where  else  yes  and  therefore
like  for  the  cube  you  cannot  do  it  you
can't  expect  anything  that's  right  yes
so  this  is  where  the  the  finite
dimensionality  really  comes  in  so  it
plays  a  sort  of  a  technical  role  in
being  able  to  ignore  the  difference
between  this  and  the  derived  category  of
shav  of  a  being  groups  but  um  but  uh
this  is  the  this  is  really  where  you
need  finite
dimensionality
okay  uh  okay  now  let's  come  back  to  the
the  question  about  e  Infinity  versus
animated  commutative  so  um  now  we  have
this  we  have  this  object
um  oh  wait  is  this  going  to  work  just  a
sec  oh  maybe  this  isn't  going  to  work  uh
no  let's  not  come  back  to  the  question
about  e  Infinity  versus  animated
commutative  with  my
apologies
um  okay  let's
actually
um  right  oh  I  want  to  oh  so  at  the  end
of  this  I  want  to  make  a
remark
um  I  maybe  it's  good  to  say  that  at
least  in  this  example  of  this  H  he  can
ow  five  by  hand
structure  because  you  only  have  to  do
for  Al  because  sequential
of  yes  that  was  the  argument  I  was  yeah
that's  the  argument  I  was  going  to
suggest  and  then  I  was  thinking  in  my
head  um  about  why  for  item  potent
algebras  it's  automatic  and  it  wasn't
quite  clear  to  me  um  but  yeah  but  I
think  you  can  do  it  by  hand  I  mean  if
you  knew  the  categorical  structure  well
we  should  we  should  think  more  carefully
about  exactly  how  to  do  this
um  and  report  back  next  time  uh  so  okay
um  but  let's  move  on  I  want  to  make  a
small  remark  before  we  move  on
um  remark  so  by  similar  arguments  or
basically  the  same
arguments
um  if  you  have  a  a  map  like  this  um  then
or  sorry  if  our  functor  like  this  it's
equivalent  um  a  prior  you're  saying  that
you  have  this  connectivity  estimate
shriek  locally  on  Spec  R  um  but  um  it's
equivalent  to
ask  uh
that
um
so  uh  over  a  proper  cover  a  cover  by
proper
Maps  you  get  the
connectivity  so  you  don't  actually  have
to  when  you're  wor
there's  over  AR  that  mod  give  you  an
ring
structure  on  on  your  R  could  you  say
that  one  more  time
Peter
don't  connect  algebra  over  any  antic  R
yeah  then  you  can  use  this  to  Def  new
notion  of  complete  modules  which  are
modules  over  the  sky  oh  yes  uhuh  I  what
you  call  thetic  ring
always  I
over  H  that  that  that  seems  to  work
indeed  yeah
yeah
yeah  um  but  are  all  of  the  intermediate
guys  connective  as  well  yeah  I  guess
they  are  well  okay  yeah  so
yeah  um  okay  or  maybe  connectivity  isn't
even  important  there  anyway  all  right  uh
oh  yes  um  and  another  remark  is  that
this  will  actually  be  a  little  bit
useful  now  I've  not  lost  ah  I'm  sorry  oh
we're  over  here  in  the  setting  of  number
two  in  the  the  the  theorem  here  okay  so
so  let's  say  you  want  to  produce  a  map
like  this  by  this  procedure  so  you  want
to  produce  a  symmetric  mordial  funter
and  then  you  have  to  check  this  annoying
condition  that  on  some  shet  cover  of
specr  you  have  this  connectivity
condition  I'm  saying  or  or  let's  yeah  or
I  don't  know  you  have  no  I'm  no  I'm
making  a  different  claim  sorry  that  uh
let's  say  you  have  a  map  like  this  then
you  know  that  shriek  locally  you  get  uh
this  but  in  fact  what  we  see  in  the
proof  is  that  um  after  a  proper  map  even
after  just  a  proper  map  from  a  proper
cover  of  specr  you  you  can  ensure  this
condition  this  comes  from  after  proving
the  equivalent  yeah  yeah  yeah  yeah  yeah
not  a  prior  not  a  priori  exactly
yeah  um  another  remark  um  is  that  if  you
have  Z
closed  uh  closed  inside  K  or  this  is
actually  more  General  but  um  and  then  U
is  the  complimentary
open  uh  then  they  are  also  each  other  as
complements  uh  in  analytic
Stacks  so  these  two  do  determine  each
other  the  associated  analytic  Stacks  do
determine  each  other  in  the  naive  way
I.E  if  you're  given  Z  and  you  want  to
know  U  as  an  analytic  stack  its  functor
of  points  is  you  just  map  to  K  such  that
when  you  pull  back  to  um  to  Z  you  get
the  empty  analytic  stack  that's  the  same
thing  as  mapping  to  U  and  vice
versa  so  that  that  you  can  check  on
profinite
sets  where  it's  um  where  it's  quite
Elementary
um
okay
uh  so  now  let's  get
to  sorry  why  did  do  right  see
toag  um  how  else  did  you  want  me  to
write  them  only  what  what  it
means  doesn't  mean  anything  um  it's  just
because  the  well  I  wanted  to  write  for
example  I  wanted  to  write  this  one  this
one  right  above  this  one  because  I'm
saying  the  these  two  end  points  map  to
the  same  point  down  here  so  I  wanted  to
stack  them  vertically  like  that  maybe
that's  the  maybe  that's  the  reason  does
that  make  sense
yeah  yeah
okay  so  now  we  get  to  a  topic  that  I
think  is  fun  which  Peter  introduced  last
time  this  kind  of  norms  on  analytic
rings  so  let's  say  that  R  is  an  analytic
ring  so  definition  uh  is  that  a  norm  on
R  is  a
map  of  analytics
Stacks  um  from  the  algebraic  P1  over  R
which  is  something  you  can  build  over
any  analytic  ring  by  just  base  change
from
the  trivial  case  with  algebraic  geometry
um  to  the  closed  interval  from  0er  to
Infinity  which  is  a  condensed  set  and
thereby  an  analytic  stack  um  so  let's
call  this  map  n  and  then  such  that  and
then  I'm  going  to  give  some
conditions
um  which  are  going  to  be  different  from
the  ones  that  Peter  gave  last  lecture  so
a  PRI  I'm  adding  an  extra  condition  and
maybe  slightly
rephrasing  um  another  of  his  conditions
and  we  don't  know  whether  they're
equivalent  but  certainly  uh  we  want  all
of  the  properties  I'm  about  to  list  so
let's  take  this  as  the  official
definition  now
um
uh  right
um  so  the  first  condition  is
that  okay  maybe  the  first  thing  to
remark  is  that  such  a  norm  function  on
P1  of  r  well  certainly  you  can  restrict
it  to  A1  of  R  you  get  a  norm  function  on
A1  of
R  um  but  then  what  does  it  mean  when  you
have  an  element  of  R  then  do  you  get  a  a
real  number  or  an  element  in  zero
Infinity  no  you  do  not  get  an  element  in
Zer  Infinity  you  get  a  map  so  so  so  well
let  me  so  note  if  you  if  you're  given  a
if  you're  given  an  F  in
R  then  that  induces  a  section  so  from  P1
of  R  to  Spec  R  uh  then  you  get  the
section  corresponding  to  f  um  and  then
you  can  compose  that  with  a  norm  map  0
infinity  and  what  you  get  is  a  map  from
Spec  R  uh  to  0
Infinity  your  yeah  yeah  exactly  now
suppose  just  to  to  I  don't  know  if  this
is  a  big  to  but  suppose  you  start  from
an  obstruct  we  said  that  there  are
several  ways  to  view  it  as  an  analy
Okay  and  like  you  can  use  the  the  or  no
I  mean  anyway  there  at  least  two  or
three  ways  I  remember  and  then  for  each
of  those  you  have  a  notion  of  G  on  the
other  hand  you  have  a  gome  like  in  Bel
or  like  this  so  can  you  say  what  are  the
relations  so  you  have  and  what  are  yes
we  will  we  will  discuss  uh  these  things
later  but  I  want  to  get  the  basic  uh  the
basic  definitions  in  place  first  basic
results  and
definitions  so  yes  so  a  norm  for  an
element  of  the  ring  you  don't  get  a  real
number  you  get  a  map  from  Spec  R  to  to
the  real  numbers  plus  infinity  so  if  you
you  can  think  of  this  as  consisting  of  a
family  of  residue  fields  and  kind  of  for
each  residue  field  you  get  a  real  number
but  they  could  be  varying  with  the
residue  Fields  so  to  speak  um  relatedly
if  you  have  a  norm  on  R  and  you  have  a
map  from  R  to  RP  Prime  you  get  a  norm  on
RP  Prime
so  just  by  composition  um  so  it's  really
um  it's  not  like  giving  a  assigning  a
norm  to  each  element  in  R  it's  a
geometric  thing  so
it's  yeah  it's  it's  it's  a  geometric
thing  it's  you  can  pull  it  back  and  it
still  persists  so  uh  that's  important  to
important  to
realize  uh  okay  right  so  that  that's  a  a
Prelude  and  then  the  first  condition  uh
is  that  Norm  of
zero  which  I  am  which  I'm  which  is  a  map
from  Spec  R  to  01  or  0  Infinity  uh  this
should  Factor  through
so  Factor  through  the  terminal  map  to
this  is  the  this  is
the  this  is  the  terminal  analytic  St
terminal  analytic  stack  but  it's  also
the  analytic  stack  Associated  to  the
condensed  anama  which  is  the  Singleton
point  which  is  a  subset  of  here  um  so
this  is  a  condition  that  the  norm  is  the
constant  function  zero  the  norm  of  zero
is  the  constant  function  Zero  by  the  way
in  this  geometric  perspective  over
there's  something  you  might  appreciate
there's  sometimes  this  question  of
whether  normal  one  should  be  equal  to
one  or  zero  for  the  zero
ring  nor  what
is  yeah  but  this  is  avoided  here  because
when  you  have  the  zero  ring  then  the
norm  is  both  one  and  zero  so  to  speak
because  then  then  this  is  the  empty  set
and  then  uh  uh  that  map  factors  both
through  zero  and  through  one  and  so  in
the  if  you  take  this  geometrical
perspective  on  Norms  there's  no  you  can
never  get  messed
up  it's  not  the  same  notion  but  I  mean
just  to
show  um
okay
um  okay  wait  sorry
over  here  there's  no  like  we're  not
saying  anything  about  what  happens  to  A1
of
R  just  let  me  finish
um  so  the  second  condition  is  that  the
following  diagram  commutes  so  P1  VAR
maps  to  zero  in  the  second  and  third
conditions  I'm  going  to  try  to  say  the
norm  is
multiplicative  um  the  first  thing  I'm
going  to  say  is  that  if  you  it  commutes
with  inversion  so  so  here  we  have  Lambda
goes  to  Lambda  inverse  which  exchanges
zero  and  infinity
um  but  we  also  have  let's  call  the
coordinate  in  P1  T  and  we  also  have  t
goes  to  T
inverse  um  exchanging  zero  and  infinity
and  I  want  this  diagram  to  commute  so  by
the  way  the  maps  from  anything  to  Z
Infinity  of  a  space  like  k  or  a  is  a  set
in  your  it's  a  set  yes  just  a  set  yeah
yes  and  even  with  sem  compact  okay  this
was  said  before  in  yeah  any  any
condensed  set  will  go  to  a  an  analytic
stack  whose  funter  of  points  is  a  set
yeah
mhm
okay  uh  so  this  one  and  two  so  now  note
that  one  and  two  uh  implies  that  uh  if
you  take  the  infinity
section  so  oh  let's  say  Norm  of  Infinity
uh
uh  so  Spec
R  Norm  of  infinity  0  Infinity  this
factors  through
Infinity
um
um  okay  now  uh  before  I  yeah  and  now
yeah  okay  maybe  three  so  Set  uh  I  don't
uh  it's  one
right
ah  does  this  does  this  already  imply
that  yeah  okay  that's  good  I  was  going
to  make  that  remark  after  the  next  one
but  yeah  I  guess  it's  already  implied
here  okay  cool
um  and  Al  the  norm  of  minus  one  is  well
by  the  same
Al
okay  yeah  thanks
guys
okay  so  three
um  um  so  I'll  Define  given  a  norm  on  an
analytic  ring  I'll  Define  the  associated
analytic  Aline  line  to  be  uh  the  subset
of  P1  where  the  Norms  are  a  real  number
and  not
Infinity
um  so
um  uh  then  we  have  uh  then
require  uh  that  if  you  take
A1  R  analytic  cross  A1  are  analytic  and
then  you  have  Norm
Norm  uh  to
oops  oh  I  guess  this  is  just  R  but  okay
I  I  I'll  continue  to  write  it  as  0
Infinity  um  uh  then  here  we  can  take  the
product
so  and  we  still  get  something  in  the
region  from  zero  to  Infinity  which  is
contained  in  uh  Zer  Infinity
closed
um  uh  then  on  the  other  hand  this  we  can
map  to  p1r  cross  P1  oh  sorry  p1r
period  um  via  multiplication  so
TS  goes  to  St  or
TS  this  is  the
five  ah  the  multiplication  M  on  P1  is
not  defined  in
general
TS  ah  the  norm
inverse
why  I'm  going  to  I  I'll  make  the  yeah  so
Peter  is  making  the  pertinent  remark
which  I'll  justify  uh  afterwards  I  mean
so  yeah  so  actually
yeah
um  well
yeah  that  I  I'll  justify  why  this  map  is
well  defined  at  the  end
um  right  so  this  this  map  should
commute  so  that's  saying  that  the  norm
is  multiplicative  but  as  ofer  is
pointing  out  the  one  needs  to  justify
that  such  a  map  indeed  exists  um  let  me
do  that  now  so
uh
so  so  y
TS  goes  to  TS  is  well
defined  on
a1n  um  so  the  claim
uh  uh  is  that  uh  so  A1  RN  is  a  subset  uh
well  sub  monomorphism  admits  a
monomorphism  to  P1  by  definition  another
thing  that  by  definition  admits  a
monomorphism  to  P1  is  the  algebraic
apine  line  over
P1  and  the  claim  is  that  this  one's
contained  in  this
one  ALB  line
is  is  the  you  take  the  polom  bring  one
variable  yes  over  over  R  yeah  over  R  in
which  so  in  other  words  you  take  for  the
condensed  you  take  just  head  po  long  in
in  a  stupid  way  I  mean  without  any
special  and  the  category  is  the  the  s  i
okay  I  would  say  yeah  so  it's  the  thing
it's  an  it's  an  apine  analytic  stack
this  a1r  and  it's  um  just  given  by  again
keeping  the  same  class  of  complete
modules  you  already  had  in  R  and  just
adding  the  polinomial  variable  as
operators  H  why  is  this  true  yeah
so  um
right
so
uh  so  the  point  is  that
um  so  we  want  to  produce  uh  so  you  have
to  check  it  on  the  two  charts  of  P1  one
of  them  is  already  A1  R  and  the  other
one  is  the  other  A1  R  yeah  to  some  Inver
the  operator  when  you  know  that  you
right  so  so  here's  what  I'm  going  to  say
so  we  know  that  a  norm  of  infinity  is
equal  to
Infinity
um  and  uh  so  what  does  this  imply  so  for
Infinity  we  have
um  this  implies  that  the  uh  infinity
section  of
P1  uh  is  uh  a  module  over  or  an  algebra
over  uh  is  an  alge  albra  over  um  so  Norm
upper  star  of  the  structure  chief  of
infinity  yeah  one  way  to  think  about  imp
anal  section  and  then  there's  General
that  many  and  Serv  close  at  Z  and
analytic  St
that
in  the  set  of
sches  where  scheme  is  endowed  with  the
the  trivial
analytic  okay  so  let  me  try  doing  it  the
way  I  think  it's  basically  equivalent
but  let  me  let  me  try  doing  it  the  way
Peter  was  suggesting  so  let's  let's  uh
let's  abstract  a  bit  let's  move  Infinity
to  zero  and  Abstract  a  bit  do  it  for
like  it's  P1  Z  so  it  means  you  can  this
P1  Z
and  so  suppose  given  an  analytic  stack
over  A1  um  such  that  if  you  pull
back  uh  to  zero
section  uh  you  get  uh  the  empty  set  so
it  misses  the  zero  section  then  the
claim  is  that  this  uh  F  factor  is
through  uh
GM  uh  r
and  it's  part  of  a  more  General  claim
that  Peter  was  making  about  a  a  scheme
and  a  closed  sub  scheme  but  okay  scheme
with  the  with  the  kind  of  trivial
analytic  ring  analytic  kind  of  yeah
let's  say  usual  a  fine  scheme  is  the
kind  of  the  most  trivial  analytic  yes
yes
um  so
um
right
um  right  so  fact  factors  oh  so  and  the
reason  for  this  is  you  can  think  of  maps
like  this  um  well  suffice  it  to  treat
the  C  case  where  X  is  is  itself  apine  um
and  then  Maps  like  this  you  can  think  of
in  terms  of  symmetric  monoidal  functors
so  so  if  x  is  D  of  uh  I  don't  know  a
then  you  can  think  in  terms  of  the
corresponding  pullback  functor  from  D  of
a1r  uh  to  D  of
a  um  and  what  do  we  know  about  this
pullback  functor  we  know  that  it  kills
uh  uh  kills  the  structure
sheath  uh  of  the
origin  but  then  it's  just  a  purely
algebraic  fact  that  um  if  you  kill  the
structure  chief  of  the  origin  then  you
factor  through  inverting  the  the
parameter
so
so  the  structure  sheath  of  the  origin  is
just  structure  sheath  on  A1  modulo  T  the
parameter  T  so  if  you  kill  that  thing
then  you  kill  everything  that's  built
from  that  via  colimits  and  so  you  don't
see  the  difference  between  an  object  and
the  result  of  inverting  t  on  that  object
and  then  that  exactly  gives  the  factors
through  yeah
yeah  so  thanks  Peter  I  think  that's  a
much  nicer  way
of  of  saying
it
um  everyone
good
okay  uh  so  right  so  that's  the  claim  and
that  implies  that  this  map  is  well
defined  because  certainly  uh  the
multiplication  is  well  defined  on
A1  but  um
so  maybe  then  I  could  put  I  put  the  A1
here  a  priori  uh  and  a
priori  have  only  the  N  going  here  but
then  a  posteriori  if  I  require  this
diagram  to  commute  then  it  follows  that
actually  this  uh  this  lands  inside  A1
analytic  because  by  definition  that  was
the  pre-image  because  because  this  ma
factors  through  zero
Infinity
okay  all
right
uh  so  oh
man  all  right  uh  I  want  to  get  to
something  fun  yes  please  no  no  please
please
uh  so  the  fact  that  we  denoted  is  and  so
the  Norms  here  should  somehow  correspond
to  like  correspond  to  Norms  and  analytic
the  notion  of  analy  ification  um  what
yeah  so  I'm  going  to  give  some  of  the
motivation  at  the  end  so  but  let  me
finish  with  the  axiomatics  like  building
on  that  like  so  like  I  remember  last
time  like  I  think  R  SCH  said  like  Gaga
really  was  like  at  the  like  he  he  noted
down  Gaga  as  like  like  as  an  isomorphism
of  stacks  yeah  and  uh  and  someh  that  was
some  sort  of  analy  ification  so  does
that  also  like  correspond  to  a  norm  then
so  this  question  I  suggest  you  uh  keep
it  for  later  yeah  yeah  um  so  uh  yeah  um
let  me  finish  with  the  axioms  uh  1  2  3
ah  so
four  okay  so  axium  4  now  so  maybe  now's
the  a  time  to  say  a  bit  about  motivation
so  the  um  so  what  if  you  have  an
analytic  ring  what  we're  going  to  try  to
do  is  we're  going  to  try  to  say  if  you
have  an  analytic  ring  you  want  to  try  to
build  some  geometry  over  that  ring  um
but  it's  hard  if  you're  just  if  you're
just  given  an  analytic  ring  and  you
don't  know  anything  more  about  it  or  you
don't  have  any  extra  structure  on  it
it's  kind  of  hard  to  build  analytic
geometry  over  it  I  mean  basically  all
you  can  do  is  you  can  do  this  trick  of
importing  algebraic  geometry  for  an
arbitrary  analytic  ring  that's  more  or
less  all  you  know  how  to
do
um  uh  so  what  we're  going  to  be
doing  um  and  what  well  one  what's  one
measure  that  you  have  some  good  uh
analytic  geometry  it's  that  you  have
some  nice  subsets  of  the  Aline  line
um  and  nice  in  the  context  that  we're
discussing  here  means  for  example  shable
so  that  the  six  functor  formulism  works
and  then  it  kind  of  really  feels  like
you're  doing  geometry  um  in  some  more  or
less  traditional  sense
um  so  but  uh  again  on  a  general  analytic
ring  you  you  don't  know  how  to  write
down  any  uh  interesting  shable  subsets
of  the  apine  line  so  you  have  to  give
the  you  have  to  give  yourself  some  of
them  and  that's  going  of  that's  the
point  of  this  notion  of  normed  analytic
ring  we're  giving  ourselves  basically  uh
discs  of  certain  of  of  some  arbitrary
radius  inside  the  Aline  line  um  and  they
will  turn  out  to  Define  shable  subsets
of  the  apine  line  and  then  we  can  um  we
can  get  started  on  doing  geometry  that
resembles  some  sort  of  traditional
analytic  geometry  based  on  open  discs  or
closed  discs  or  what  have  you  but  you
have  to  have  this  extra  structure  on
your  base  before  you  can  get  started  on
that  game  if  you  don't  have  a  notion  of
a  norm  on  your  ring  you  can't  start
talking  about  closed  discs  and  open
discs  of  certain  radius  so  that's  what
we're  doing  right  now  but  um  we  already
have  some  sort  of  things  that  kind  of
seem  to  function  as
a  we  had  we've  already  seen  certain
versions  of  the  unit  disk  so  for  example
we  spent  a  lot  of  time  discussing  this
basic  module
p  uh  which  recall  was  this  uh  free  free
module  on  N  Union  infinity  modulo
infinity  um  and  and  of  course  you  can
base  change  it  to  any
ring  uh  any  analytic  ring  even
um  um  and  so  well  what  what  do  we  know
about  this  guy  well  we  know  it  does  have
a  map  from  R  bracket
T  so
geometrically  uh  so  let  me  just  this  is
going  to  be  not  not  good  notation  but
let's  let's  set  uh  D  equals  spec
P  so  it's  some  conversion  of  a  dis  um
and  then  Dr  will  be  the  base  change  to
R  again  your  was  a  usual  algebra
structure  right  that's  right  yeah  and
we've  we've  also  discussed  the  algebra
structure  on  P  which  makes  this  an
algebra
map
um  so  um  so  this  uh  so  D  does  map  to  the
Aline  line  by  this  uh  by  this  map  here
um  but  as  we've  also  maybe  discussed  uh
so  it's  a  but  but  it's  not  a
monomorphism  so  it's  not  really  a  subset
of  the  aine  line  so  it's  a  it's  a  proper
map  in  our  setting  so  it's  shakable  but
um  it's  not  um  it's  not  a  monomorphism
so  it's  not  really  a  subset  of  the  Aline
line  so  what  does  it  look  like  in
examples
um  uh
so  in  general
so  we  have  well  this  this  map  from  RT  to
PR  uh  this  induces  an
isomorphism  on  mod  t  to  the  N  for  all
n  so  uh  this  this  uh  this  map  which  is
not  a  monomorphism  it's  it  is  an
isomorphism  on  the  formal  neighborhood
of  the  origin  so  at  least  in  some  sense
you  should  expect  close  to  the  origin
this  is  a  monomorphism  and  then  what
happens  is  it  kind  of  blows  up  as  you
move  away  from  or  it  can  potentially
blow  up  as  you  move  away  from  the
origin
what  uh  T  goes  to  like  the  The  Rock
measure  concentrated  at  at
one
um  okay  you  use  the  multiplication  on  N
or  you  use  the  addition  on  N  to  yeah  to
give  the  multiplication  on
PR  yeah  um
so  uh  right  so  in  particular  you  get  a
factoring  like  this  so  this
um  and  so  that's  that's  generally  true
that  you  have  a  diagram  like  this  um
where  this  is  the  canonical  map  um  and
in  in  most
examples  so  this  p  r  mapping  to  r  t  is  a
monomorphism  is  an
injective  so  in  most  examples  this  lives
in  degree  zero  and  this  map  is  an
injection  so  this  is  some  kind  of
sequence
space  and
um  what  is  the
condition  that  the  sequence
satisfies
well
uh  so  in  the  aelon  category  the  which  is
the  heart  of
Dr  so  in  in  in  most  examples  this  lives
in  the  heart  this  also  in  most  examples
lives  in  the  heart  I  mean  I  guess  maybe
I  should  be  using  this  notation  but  I'm
being  a  little  bit  sloppy
here
um  yeah  and  then  this  is  just  an
injection  so  to
speak  um  so  so  in  most  examples  PR  R  is
like  the  set  of  sequences  r0  R1  um
satisfying  some  summability  condition
so  such  that  if  you  termwise  multiply  by
a  null
sequence  uh  you  get  a  summable
sequence  uh  where  the  notion  of
summability  depends  on  the  analytic  ring
structure
but  at  the  level  of  this  discussion  you
could  imagine  for  example  are  being  the
real  numbers  and  summable  means  usual
absolute  some  the  absolute  values  you
get  a  finite  number  that's  um  that's  not
actually  a  special  case  but  it's  it's
close  enough  and  it  it  serves  the
purposes  for  this  this  discussion  um  and
this  is  kind  of  just  by  the  universal
property  of  PR  that  this  is  the  the
correct  interpretation  because  PR  is
Maps  out  of  PR  to  an  to  an  m  in  Dr  these
are  supposed  to  correspond  to  null
sequences  in  M  by
construction  um  so
um  you  know  it's  the  kind  of  thing  which
when  paired  with  a  null  sequence  uh  you
get  an  element  in  m  and  so  the  idea  is
this
procedure  yeah  so  we  null  sequence  and
then  yeah
well  if  you  well  I'll  let  you  maybe
maybe  me  trying  to  explain  it  is  not  as
helpful  as
um  all
that  is  any  analytic  ring  I  but  but  this
is  not  precise  mathematics  here  yeah  but
in  this  Norm  business  r  r  is  just  a
discret  ring  in  this  Norm  well  no  no  in
the  norm  business  R  is  an  arbitrary
analytic  ring  am  m  is  then  p  is
a
okay  so  you  you  say  that  usually  it's
say  in
object  in  the  category  of  our  yeah
usually  that's  right  yeah  yeah  and  this
is  the  what  you  call  M  uh  M  oh  sure  well
I  mean  yeah  I  could  yeah  I  mean  so  so  I
was  I  was  saying  that  this  is  the
interpretation  you  get  in  practice  where
the  notion  of  summability  depends  on  the
analytic  ring  structure  and  the  way  you
see  that  this  is  the  correct
interpretation  is  by  thinking  about  what
it  means  to  map  PR  to  m
and
um  so  like  for  example  the  most  basic
thing  was  it  would  be  if  you  map  PR  to  R
triangle  um  then  that's  the  same  thing
as  giving  a  null  sequence
um  but  then  um  if  you  think  in  terms  of
what  happens  when  you  restrict  to  here
you  have  some  Co  if  you  have  some
coefficients  in  a  polinomial  so  if  it
terminates  if  you  have  zeros  after  while
then  what  you're  doing  is  you're  just
summing  the  null  sequence  times  those
things  to  get  an  element  in  our  triangle
and  then  you  imagine  that  that  Su  this
is  saying  that  that  summing  should  make
sense  for  something  which  is  not
necessarily  eventually  zero  and  so  this
is  kind  of  the  interpretation  you  should
give  that
um  right
uh  uh  and  what  so  so  and  so  if  for
example  R  is  C  with  the  gases  or  liquid
liquid  analytic  rank  structures  what  you
see  is  that
um  you  have  if  you  look  at  holomorphic
functions  um  on  the  usual  usual  ring  of
holomorphic  functions  on  the
uh  closed  unit  dis  meaning  they  overon
Converge  on  the  closed  unit  dis
um
then
um  then  every  one  of  those  satisfies
this  if  you  look  at  the  coefficients  of
the  power  series  it  will  satisfy  this
summability  property  because  your  power
Series  has  a  value  at  one  so  the  you
know  the  thing  the  sequence  of
coefficients  better  be  summable  so
you'll  get  a  map  like  this  um  uh  so  this
summability  condition
is  weaker  than  the  condition  which
defines  these  things  but  it's  a  stronger
than  the  condition  which  defines  a
holomorphic  function  on  the  usual  open
dis  um
so  uh  if  you  think  in  terms  of  your
usual
um  your  usual  complex  analysis  then
that's  where  your  p  is  sitting  it's
sitting  somewhere  between  the  Open  disc
of  radius  one  and  the  closed  disc  of
radius
one
um  and  that's  the  kind  of  behavior  we're
going  to  be  aati  in  condition  number
four  yeah  so  this  is  something  like  a
what
uh  X  exponential  decay  and  this  is
something  like  exponential  growth  and
this  is  something  like  Su  ability  so
kind  of  makes  sense  that  it's  it's  in
between  the
two  um
okay  uh  was  a  bit  of  a  long-  winded
explanation
um  what's  I  haven't  written  it  yet  I've
just  given
motivation
um  but  yeah  so  what  what  you  want  to
think  is  in  these  kinds  of  normed
analytic  ring  settings  this  well  okay
so  then  condition  four
is
uh  so  condition
four  uh  is  split  into  two  parts  well  the
the  first  part  is  that  um  if  you  have
you  have  D  mapping  to  P1  or  Dr  mapping
to  P1  R  uh  mapping  to  0  Infinity  um  via
the  norm  uh  you  want  this  to  factor
through  uh  the  map  up  to
01  that  uh  corresponds  to
this  um  saying  that  this  notion  of  unit
dis  is  sandwiched  between  closed  and
open
um  um  but  you  also  want  that  this  if  you
take  Dr  U  mapping  via  the  norm  map  to  0
infinity  and  if  if  you  pull  back  to  uh
01  uh  the
open  well  a  half  open  segment  from  0  to
one  um
then  uh  you  should  get  an  isomorphism
over  here  so  uh
so  uh  sorry
uh  yeah  so  Norm  inverse  of  01  uh  uh
sorry  ah  shoot  um  I  need  I  need  P1  to  be
in  there  thanks
yeah  oh  sorry
Norm  um  so  then  we
have
um  so  I  want  that
um  yeah  wrot  an  inclusion  there  sorry
sorry  oh  I  wrote  I  didn't  mean  to  write
an  inclusion  thank  you  yeah  thank  you
what  I  want  to  say  is  that  uh  so  this  uh
yeah
well  let's  say  this  map  is  an
isomorphism  over  the  locus  uh  uh  given
by  the  open  unit
dis  so  so  this  is  like  some  sort  of
proper  map  which  blows  up  along  the
boundary  of  the  open  unit  dis  um  but  on
the  interior  of  the  open  unit  disc  it's
an
isomorphism  so  uh  this  is  some
something  stronger  than  saying  that  you
have  a  map  like  this  um  and  the  nice
thing  about  this  stronger  condition  is
first  of  all  you  can  check  it  in
practice  and  second  of  all  it's  just  a
condition  whereas  giving  a  map  like  this
is  a  prior
structure  um  so  can  I  think  of  this  nor
edian  nor  you  can  think  of  it  as  non
archimedian  or  archimedian  at  will  we
haven't  enforced  any  compatibility  of
the  norm  with
addition
so  it's
yeah  last  last  time  yeah  so  this
is  it  is  if  you  look  for  real  numbers
for  example  right  so  the  um  so  this  was
the  so  the  the  cosmetically  I  changed
the  discussion  of  multiplicativity  from
Peter's  lecture  that's  that's  just  a
cosmetic  change  this  is  um  potentially
more  serious  Peter  did  not  mention  this
axium  last  time  and  we're  not  sure
whether  it's  um  a  consequence  of  the
other
aums  but  um  yeah  as  Peter  mentions  if  if
you're  over  a  base  which  solidifies  to
zero  so  for  example  if  you're  over  the
real  numbers  then  then  you  can  prove
this  from  the  other  axioms  but  we're  not
sure  whether  which  solidifies  to
zero  it  means  when  you  base  change  to
solid  if  you  base  change  Spec  R  to  spec
of  solid  z  uh  then  you  should  get  the
empty  analytic  stack  if  if  that
condition  is
satisfied  then  then  this  this  this
condition  follows  from  the  other
conditions  but  so  can  you  give  example
of  saying  that  solidify  to  zero  yes  the
real  number  is  solidify  to  zero  so  any
analytic  any  analytic  ring  where  the
underlying  ring  is  an  algebra  over  the
condensed  ring
R  uh  will'll  have  that
property  so  so  again  so  you  have  the
something  going  to  spec  Z  spec  the  Sol
but  all  the  the  nonar  median  seem  for
are
not  so  over  there  potentially  there  so
as  far  as  we  know  so  far  there  could  be
some  weirdo  Norms  which  don't  satisfy
this  condition  but  yeah  they're  not  any
of  the  usual  ones  the
usual  the  the  ones  that  don't  satisfy
this  unless  it's  some  really  extremely
interesting  new  mathematical  phenomenon
which  seems  unlikely  they're  probably
just  very  pathological  we're  taking  this
as  an  axium  okay  and  then  you  could  as
the  question  whether  it  follows  from  the
other  AXS  and
um  okay
so
um  so  what  is  a  so  what  is  a  norm  again
so  what  what  is  a  norm  giving  you  so  if
you  have  a  norm  on  an  analy  so  these  are
the
axioms
um  so  this  gives  you  so  again  in  terms
of  the  discussion  of  maps  to  uh  compact
house  doorf  spaces  um  this  gives  you  for
every  for  example  uh
for  you  get  Norm  upper  star  of  the
constant  sheath  on  z
r
um  so  for  if  you  take
R  non-  negative  real  number  uh  then  you
get  this  item  potent
algebra  uh  in  uh  D  of
p1r
um  uh  and  this  course  this  uh  the
interpretation  of  this  should  be  overon
convergent  uh
functions  on  the  closed  unit
disk  uh  of  radius
r
and  the  overon  convergence  it  it  has  to
be  overon  convergent  there's  no  other
possibility  because  note  that  the
constant  shei  z0
R  uh  is  the  filtered  Co  limit  of  the
constant  shief  Zer  uh  R  Prime  for  all  R
Prime  bigger  than  R  um  under  the
Restriction  Maps  so  the  overon
convergence  has  to  be  built  in  like  if
you  imagined  that  you  were  sending  this
to  some  some  version  of  functions  on
some  version  of  the  unit  disc  or  unit
disc  sorry  uh  I  mean  disc  of  radius  R
some  version  of  the  disc  of  radius  R
then  that  version  would  be  forced  to  be
the  over  convergent  one  um  because  of
this  property
here
uh  and  because  Co  and  because  yes
because  it's  a  pullback  functor  so  it
commutes  with  co-
limits
um  all  right  um  and  okay  so  and  actually
the  data  of  these  guys  uh  given  the
axioms  this  determines
n  because  it's  very  easy  to  classify  the
closed  subsets  of  the  of  the  of  this
interval  and  um  you  only  need  to  know
about  things  less  than  or  equal  to
something  and  things  bigger  than  or
equal  to  something  but  the  axum  about
inversion  gives  you  one  in  terms  of  the
other  so  you  just  have  to  give  these
things  uh  subject  to  some  some  simple
conditions  in  order  to  specify  a
norm  Okay  so
the  next
topic  is  classifying
norms  and  uh  I  mean  the  the  uh  I  don't
mean  like  as  in  classifying  space  or
whatever  I  mean  how  to  classify  Norms  on
a  given  analytic
ring  uh  yeah  so  let  me  start  with  one
aemma
here
question  that  the
usual  yes  and  yes  there  is
yeah
um  right  so  uh  LMA  so  so  given  a  norm  on
an  analytic  ring  uh  so  a  norm
maybe  just  say  just  said  what  data  you
have  to  give  to  Def  find  nor  and  you  can
just  check  that  the  usual  algebra  either
geometry
orry  solid  the  complex  numbers  with  they
do  satisfy  SE  so  this  way  you  can
produce  by  hand  now  such  Norms  over
QP
right  sorry  but  what  does  over
convergent  mean  uh  it  means  so  an  overon
convergent  function  on  a  disc  of  radius
R  means  a  function  which  converges  on
some  dis  centered  at  the  same  point  with
larger
radius  okay  yeah  so  it  extends  to  a
function  on  an  open  neighborhood  of  the
the  closed
dis
um
right  um  yeah  don't  be  shy  about  asking
such  questions  about  terminology  because
I  know
well  anyway  um  so  right  so  given  a  norm
then  you  can  look  at  the  the  locus  of
um  of  um  the  locus  where  the  norm  is
strictly  between  zero  and  one  uh  and
this  projects  down  to  Spec  R  uh  and  I'll
say  that  this  is  a
cover  in  the  in  the  gro  topology  we've
been
considering
so
IE  if  we're  willing  to  work
locally  uh  we  can
assume  uh  we  have  an  element  you  can
call  Q  uh  in  the  underlying  ring  such
that  uh  Norm  of  Q  uh  is  in  so  to
speak  lies  strictly  between  zero  and
one  so  is  it  the  case  that  when  you  take
any  real
number  yes  just  yes  yes  yeah  yeah  so
actually  so  I'll  make  a  stronger  claim
in  the  proof  so  in
fact  uh  if  you  take  Norm  inverse  of
say2  uh
specr  this  is  a
cover  and  that's  a  stronger  claim  um  is
a  cover  in  the  sh  in  this  gr  top  the
only  gr  de  topology  we've  put  on  on  our
category  okay  but  this  is  the  strong  of
the
is
no  no  it's  a  cover  in  the  gr  de  topology
that  it  can  be  refined  to  yeah  in  fact
this  is  not  necessarily
apine
okay  cover  the  can  be  refined  to  yes  can
be
ref
yeah
um  so  but  that  said  for  let  me  so  um  let
me  for  Simplicity  assume  that  this  is
apine  uh  so  there's  an  argument  for  uh
getting  the  conclusion  anyway  um
based  on  you  again  again  resolving  this
by  a  profinite  set  um  and  using  that
descend  ability  that  was  proved  earlier
so  but  um  let  me  I'm  going  to  skip  over
that  part  of  the  argument  so  for
Simplicity  uh
assume  uh  so  that  this  Norm  upper  star
of  the  constant  sheath  on  1/2  is
connective  um  so  that  this  Norm  inverse
of
12  uh  is
apine  and  it  is  uh  and  it's  actually
proper  over
A1  it's  just  given  by  this  this  algebra
um  and  hence
over  hence  over  Spec
R  so  in  that  situation  this  is  a  both
apine  and  sorry  this  is  apine  and  it's
proper  over  Spec  R  so  by  this  descend
ability  Criterion  we  need  that  uh  our
triangle  lies  in  is  generated  by  this
algebra
so
um  but  what  I
claim  uh  our  triangle  is  actually  a
retract  directly  a  retract  without  any
cones
and  so  implicitly  I'm  viewing  this  over
the  apine  line  instead  of  the  projective
line  and  then  I  mean  I  take  the
forgetful  functor  from  D  of  A1  to  D  of  R
um  when  I  when  I  write  this
here  or  in  other  words  this  is  this  is
descendible  of  index  zero  so  it's  like
The  Descent  is  quite  direct  you  just
take  a
retract
um  then  the  intuition  behind  this  should
be  clear  uh  this  is  some  version  of  a
lant  Series  ring  it's  a  lant  Series  ring
and  variable  t  with  some  convergence
condition  um  and  you  can  just  pick  out
say  the  zeroth  coefficient  of  your
laurant  series  that  gives  a  linear  map
not  an  algebra  map  but  a  linear  map
which  splits  the  unit  um  that's  the  idea
um  but  we  have  to  make  sure  it's  it
holds  in  this  completely  abstract
setting  here  so  what  you  do  is  you  take
uh  you  can  take  this  structure  chief  of
P1
um  uh  and  then  you  can  take  the  uh  Norm
upper  star  of  Z  well  yeah  let
me  well
okay
um  so  this  is  a
pullback  in  D  of
P1
um  just  because  the  constant  sheath  on
this  interval  is  the  the  pullback  of  the
constant  chief  on  zero  one2  uh  constant
chief  on  1/2  one  glued  along  constant
Chief  at  the  point
there  um  and  then  we  apply  uh
uh  push  forward  to
d  uh  to  D  of
R  then  the  chology  of  the  structure  chif
on  P1  is  just  it's  just  it's  just  R
concentrated  in  degree  Z  so  what  we  get
we're  R
triangle  um  and  then  I'm  going  to  use
the  same  notation  or  so  the  same  just
the  these  ones  live  over  A1  or  this  one
doesn't  but  well  whatever  the  just  just
push  it  well  it  lives  on  the  other  A1
but  we  can  just  push  it  forward
um  copy  the  same  guys
over  what's
that  first  do  a  twist  byus  to  get  degree
one  why  do  I  want  chology  in  degree
one  because  you  want  a  map
from  that's  that's  not  the  way  I'm  going
to  do  the  argument  so  um  yeah  so  what
I'm  going  to  say  is  that  um  there's  so
this  is  then  a  pullback  in  Dr  uh  but
it's  also  then  well  pullbacks  are  the
same  thing  as  pushouts  so  it's  a  push
out  in  Dr  um  so  if  I  want  to  make  a  map
from  here  to  to  a  certain  place  for
example  to  our  triangle  I  make  a  map
here  and  I  make  a  map  here  and  I  make
sure  they  agree  there  um  so  what  I  do  is
here  I  take  evaluation  at
Zero  from  our  Tri  uh  which  goes  to  R
triangle  here  I  take  evaluation  at
Infinity  which  also  goes  to  our
triangle  and  then  they  clearly  uh  just
both  both  give  the  identity  map  when  you
restrict  to  R  triangle  so  this  gives  the
map  here  which  when  you  restrict  back  to
here  is  the  identity  um  so  that  uh  that
produces  the  desired
retraction  kind
of
yeah  and  what  about  the  this
connectivity  a  restriction  oh  yeah  so
then  you  what  you  have  to  do  is  you  have
to  see  that  so  there's  then  you  you  have
some  it's  not  if  it's  not  necessarily
apine  you  still  have  some  proper  cover
by  something
apine
um  and  then  um  and  not  not  just  proper
but  descendible  so  so  then  the  unit  here
is  generated  by  the  image  of  this  um  and
it  follows  then  that  the  UN  by  combining
the  two  generation  claims  the  unit  here
is  generated  by  the  image  here  and  that
provides  the  um  that  provides  the
desired  refinement  by  a  street  cover  of
a  finds  so  you  take  any  any  proper  map
to  here  which  makes  the  thing  connective
which  I  I  I  explained  why  exists  then
then  this  map  will  be  a  street  cover
because  because  the  push  forward  will
generate  the  unit  because  the  analogous
claim  is  true  both  here  and  here  ah  and
the  in  the  first  Arrow  it  is  true
because  of  of  the  argument  gave  the
first  half  of  the  class  Yeah  by  this
this  descend  ability  of  Canter  set
mapping  to  the  unit
interval
which  was  the
the  uh  but  n  inverse  12
is  where  is  the  inter  no  ah  it's
a  you  apply  the  you  cover  zero  Infinity
like  you  did  and  you  look  at  what  it
gives  in  this
uh
no  sorry  you  want  to  get  a  fine  things
you  said  that  n  inverse  one  up  is  not  a
fine  in  your  terminology  in  general  yeah
so  how  do  you  make  it  find  by  it's  not  a
pullback  on  the  base  it's  not  it's  not
by  working  locally  on  the  base  it's  by
working  locally  on  on  P1  or  A1  it's  by
working  locally  on  A1  using  the  previous
argument  so  A1  maps  to  zero  Infinity  so
then  there  exists  a  proper  cover  of  A1
by  an  apine  where  the  that  map  factors
for  the  caner
set  and  then  that  will  give  you  the
desired  connectivity  and  also  descend
ability  there
youil  yeah  and  you  have  descend  ability
both  here  and  then  by  that  argument  here
and  then  this  is  aine  n  proper
and
yeah  okay
um  right  so  so  now  we  now  now  let's  give
ourselves  this  extra  data  which  we've
assured  can  be  gotten  locally  so  given  a
norm  uh  p1r  Norm  0  infinity  and  a  q  uh
in  uh  such
that
so  the
claim  is  that
um  so  the  well  we  can  think  of  Q  as  as
as  being  given  by  a  um  um  a  map
from
uh  Z  bracket  t  uh  to  our
triangle  or  to  R  why  not  uh  so  so  T  goes
to
Q  uh  claim  this  factors
uniquely  through  this  gaseous  bace  ring
uh  I  let  yeah  so
well  let  me  not  let  me  not  name  the  the
map  let  me  call  the  variable  Q  here  and
um  so  a
a  variable  which  the  norm  makes  look
like  it's  topologically  nil
potent  um  actually  has  to
um  come  from  this  uh  this  gaseous  base
here  so  the  proof  is  so  let  me  start
with  the  uniqueness  which  has  nothing  to
do  with  Norms  it's  actually  a  claim  that
so  spec  the  claim  is  that  spec  z  q  hat
plus  or  minus  one
gas  uh  which  maps  to  the  aine  line  uh
over  Z
say  uh  this  is  a
monomorphism
um  yeah  so  so  that's  the  that's  the
uniqueness  claim  or  a  strong  form  of  the
uniqueness  claim  there's
a  it's  just  there's  a  contractable  space
of  factorings  if  one
exists  um
uh  this  is  not  obvious  from  the
definition  of  the  gaseous  base  ring
because  by  definition  of  the  gashes  base
ring  we  took  this  uh  zq  hat  which  was
just  n  uh  which  was  just  P  um  and  then
we  inverted  q  and  then  we  enforced  this
Q  being
gaseous  um  and  this  this  is  not  item
potent
so  not  item
potent  so  enforcing  the  condition  of
being  gas  what  defines  a  monomorphism  on
the  level  of  analytic  Stacks  because
it's  just  some  quotient  of  categories
but  because  this  is  not  item  potent  it's
really  not  clear  from  that  description
that  that  the  composite  map  all  the  way
to  A1  should  be  item  potent  because  it
seems  at  first  you  have  to  choose
something  which  is  extra  structure
namely  a  a  proof  that  Q  is  topologically
nil  potent  so  to
speak  um  but  you  can  do  it  in  the  other
way
instead  but
uh
uh  you  can  also  think  of  this  as  modules
over  P  or  P  modules  in  you  just  take  a
polinomial  ring  in  one
generator  uh  and  make  that  generator
gaseous  uh  oh  and  then  maybe  I  have  to
invert  Q  too  well
okay
just
um
so
and  then  the  claim  is
that  is  that  P  is  item
potent  uh  in  in  D  of  Z  Q  Plus  orus  One
gas  um  and  this  actually  so  we  want  to
check  that  P1  or  P  equals  P  that's
actually  a  claim  that  happens  after  base
change  to  P  so  to  speak  so  it  actually
reduces  to  a  calculation  here
uh  that  if  you  take  this  p  over  z  q  hat
Plus  or  Min  -1
gas  um  now  we  have  both  a  q  variable  and
then  we  have  a  t  variable  say  coming
from  the  coming  from  the  P  here  um
so  uh  uh  that  if  you  take  this  and  you
mod  out  by  T  minus  Q  um  you  just  get  the
underlying  ring
so  yes
yes  you  should  still  write  guas  because
completion  changes
the  uh  oh  okay  sure  yeah  yeah  I  mean
yeah  the  underlying  ring  of  the  analytic
ring  structure  and  Peter  described  this
uh
this  free  module  and  if  you  use  that
description  you  will  find  yourself  is
you're  just  you  have  to  check  a  short
exact  sequence  and  you  can  do
it
um  so  that's  the
uniqueness
um  as  for  the
Existence
um
uh  well  let  me  just  say  it  in  words  it's
it's  um  fairly  Elementary  from  the
definition  of  the  norm  so  because  our
Norm  is  contained  in  here  uh  this  means
that  we're  away  from  so
existence  so  Norm  of  Q  contain  in  01
means  we're  away
from  Norm  inverse  of  1
infinity  um  and  that  in  particular  means
that  we're  away
from  uh  the
opposite  the  the  well  let's  say  the  the
D  sitting  at
Infinity  so  if  you  take  this  D  the  thing
that  was  spec  of  p  and  you  translate  it
to  Infinity  on  P1  then  that's  going  to
be  uh  contained  in  this  Locus
um  but  then  saying  that  you're  away  from
the  D  setting  at  Infinity  is  exactly  the
same  thing  as  saying  that  Q  is
gaseous  um  but  then  on  the  other  hand
Norm  of  Q  uh  contained  in  uh  Clos  inter
from  01  because  of  the  axium  I'm  I'm
referring  now  to  the  axium  4  about  the
placement  of  D  in  this  uh  in  under  this
Norm  uh  this  implies  that  uh  Q  uh  comes
from  d
so  uh  Q  is  topologically  nil
potent  so  it  comes  from
D  so  D  was  this  spec  of  P  mapping  to  the
apine  line  it  therefore  maps  to
P1  and  then  you  can  pull  back  that  map
uh  under  the  uh  automorphism  of  P1  which
is  the  inversion  map  and  you  get
something  abstractly  isomorphic  to  D  but
the  structure  map  to  P1  is
different  um
yeah  yeah  yeah
yeah
okay  I'm  almost  done  I  apologize  for
going  a  bit  over
um
so  so  now  let  me  State  the  main  result
about
what  is  comes  from  D
um  comes  from  d  means  that  um  it
promotes  to  a  p
module  so  I  the  I  mean  the  or  the
well  what  I  mean  is  that  Q  so  Q  is  a  no
no  what  I  mean  is  a  so  Q  is  a  section  of
this
projection  um  well  it's  a  section  of
this
projection  and  the  claim  is  that  it  fact
through
D  and  the  reason  is  is  that  it  factors
through  the  norm  inverse  of  this  and  one
of  the  axioms  guarantees  that  that
implies  that  it  factors  through  D  yeah
yeah  have  you  stated  what  the
classification  is  of  I'm  doing  it  right
now  so
theorem  uh  so  for  given  an  analytic  ring
R  um  there  is  a  bje  natural  bje  or  the
the
map  um  from  the  set  of  pairs  well  not
not  really  set  but  okay
n  uh  this  is  a  norm  on
R  and  then  a  a  q  uh  and  in
R  uh  such  that  Norm  of  Q  is  contain  in
01  so  up  to  a  a  cover  of  the  of  Spec  R
um  this  is  you  know  we  can  we  can  assume
we  have  this  data  so  up  to  some  cover
we're  classifying  all
Norms  um  so  but  then  from  each  of  these
I've  just  uh  I've  just
said  uh  you  get  a  map  from  Spec
R  to  to  well  first  of  all  uh  you  get  a
map  to  spec  of  Z  Q  Plus  orus  One
liquid  uh  but  then  you  also  get  a  map  to
the  open  interval  open  interval  from  0
to
one
um  yes  it's  not  oh  thank  you  gasas  yeah
thank  you
uh  and  this  is  by  recording  q  and  the
norm  of
Q
um
uh  this  map  is  an
isomorphism
I.E  giving  a  norm  on  an  analytic  ring
plus  an  element  of  Norm  strictly  between
zero  and  one  is  the  same  thing  as
specifying  what  the  norm  of  that  element
should  be  which  is
which  is  this  function  here  and
requiring  that  that  Q  actually  come  from
this  this  gases  theory  that  we  discussed
earlier  and  both  sides  are  sets  no
almost  well  except  for  the  annoying  fact
that  A1  is  not  a  set
so
a  yeah  so  so  the  the
norms  and  Q  the  first  the  first  thing  we
said  is  a  is  the  set  the  nors  really  is
a  set  because  there  is  no  and  the  Q  of
course  is  something  in  some  set  no  no  no
the  Q  is  not  in  a  set  because  this  could
be
derived  so  so  both  of  these  map  to
A1  and  like  sort  of  the  fibers  of  these
maps  are  sets  so  there  there're  as  much
sets  as  each  other  um  but  they're  not
each  individually
sets  ah  so  Q  is  a
derived
ah  okay  is  but  you  still  think  of  Q  as
as  choosing  something  in  the  Zero  part
of  simpli  uh  Zero  part  yeah  yeah  yeah
yeah  implicitly  the  higher  s  the  higher
maps  are  there  also  yeah  so  it's  not  you
have  to  construct  them  as  well  they  come
automatically  from  the  for  I  don't
know  so  let  me  give  the  proof  um  so  uh
how  non  acate
L  not  now  not  now  uh  so  the  proof  uh  so
um  so  after  a  descendible  cover  after  a
cover  uh  we  can
assume  uh  Q  admits  all  power
admits  and
compatible  nth  roots
uh  for  all  n  so  IE  so  this  is  because
the  uh  this  is  descendible
so  it's  countable  in
fpqc  um  so  we're  free  to  because  both
things  are  sheaves  uh  in  our  topology
both  sides  are  sheaves  in  our  topology
uh  we're  free  to  work  locally  so  we're
free  to  assume  U  that  that  the  Q  has  all
all  nth  Roots
um  then
um  then
um  uh  then  the  claim  is  that  so  then  so
then  so
write  uh
DQ  uh  for  the
pullback  so  we  have  D  going  to  P1  or
let's  say  A1
uh  and  then  we  have  multiplication  by  Q
on  A1  um  and  then  we  have  DQ
here
um  so  for  this  would  this  makes  sense
for  any  q  and
GM
uh
um  note  that
if  if  um  Q  equals
Q  maybe  I  maybe  sorry  I  apologize  I
apologize  maybe  I  say  d  alpha  for  Alpha
and  GM  but  note  that  if  Alpha  equals  Q  *
beta  or  Q  is  uh  top  ically  nil
potent  then  we  get  a
map  uh  from  d
alpha  to  and  I  have  to  get  it  going  the
right  direction  so  d  alpha  to  D  beta  is
one  of  the  two  which  is  also  induced  by
multiplication  by  Q  so  here  use  that
this  p  is  a  hop
algebra  encoding  multiplication
so  um
um  right  uh  I  may  have  gotten  the
ordering
wrong  so  this  is  uh  this  is  some  disc
now
um  this  is  some  disc  of  radius  Q  inverse
or  some  version  of  a  disc  of  radius  Q
inverse  it's  not  the  correct  one  because
it  kind  of  it's  not  a  monomorphism  and
it  blows  up  along  the  boundary  but  um
but  then  we  can  fix  that
um  so  then  so  DQ  so  or  d  alpha  sorry
gives  uh  a  disc  of  radius  a  kind  of  disc
of  radius
Alpha  of  radius  Alpha  inverse
maybe  um  and  the  the  the  maps
above  uh  give  give  the  inclusions
between  such
discs
what  about  this
discuss  I'm
sorry  or  the  identity  so  say  what  do  you
mean  the
identity  you
have  to  A1  right
yeah  a  bu  of
want
consider  yeah  I  think  I  I  think  I'm
saying  the  correct  thing  but  um
yeah  um  okay  but  now  we  apply  this  apply
this  uh  to  like  Alpha  equals  Q  to  the  uh
you  know  m/  n  so  m/  n  uh  in  the  rational
numbers  and  we  get  dis
get  discs  uh  of  R  of
radius  uh  nor  of  Q  to  the  m  ah  sorry  so
sorry  we  can  we  can  assume  sorry  I
should  the  first  thing  I  should  have
said  is  we  can  freely  assume  that
the  that  the  norm  of  Q  is  constant  equal
to  12  by  rescaling  the  norm
by  I  I  feel  like  I'm  not  I  Peter  what
just  happened  this  is  crazy  I  I  I
literally  just  repeated  your  lecture
with  more  detail  and  I  didn't  get  any
farther  I  thought  I  was  going  to  get
farther  um  I'm  sorry  for  that  um  so  we
have  we  have  to  do  a  better  job  with  the
organization  I  don't  know  um  but  okay  so
you  get  discs  of  this  radius  and  a
inclusions  between
them
uh
um  but  then  you  uh  you  do  this  over
convergence  so  uh  then  you  use  these  to
make  the  over  convergent
versions  um  for  an  arbitrary  real  number
you  can  look  at  all  the  rational  numbers
bigger  than  it  you  have  these  kind  of
fake  discs  of  that  rational  radius  and
then  in  the  co-limit  all  these  problems
about  blowing  up  along  the  boundary
disappear  and  it  it  turns  into  the  thing
which  has  to  be
specified  uh  if  you're  given  a  norm  on  R
so  when  you  when  you  make  the  make  is
over  convergent  it  will  be  forced  to  be
equal  to  the  thing  that  comes  from  a
norm  if  you  have  a  norm  so  that's  more
or  less  an  argument  why  this  is  uh  a
monomorphism  and  then  if  you  want  to
check  that  it's  a  bje  you  just  have  to
produce  uh  a  norm  on  here  and  you  do  it
by  following  this  this  procedure  or  Norm
on  here  and  you  can  even  adjoin  all  the
roots  that  you  want  um  and  then  you  do
it  by  following  this  procedure  so  uh
yeah  so  you  take  this  fake  disc  you
translate  it  around  by  multiple  by
powers  of  Q  and  then  you  build  the
inclusion  Maps  between  those  discs  uh  I
don't  mean  translate  I  mean  expand  by
multiplication  you  build  the  inclusions
between  those  discs  and  then  you  make
them  over  convergent  and  then  you  check
that  you  have  all  the  axioms  so  uh  yes
sorry  for  going  over  time  thanks  for
paying  attention  so  in  this  inclusion  of
this
using  we  also  add  an  analog  and  discuss
ruber  rings  of  to  power
wounded  oh  yeah  yes  yes  if  Q  is  power
bounded  yes  does  it  still  follow  that
you  have  an  inclusion  or  not  maybe  you
need  the  to  work  in
more  solid  cont  there  but  no  it's  yeah
it's  a  good  point  um  yeah  it's  not  I
mean  it's  not  entirely  clear  in  this  in
this  general  setting  uh  what  the  class
of  Q  for  which  you  get  um  such  a  map  is
yeah  let  me
see
because  I  mean  for  example  this  argument
here  doesn't  prove  that  you  have  an
identity  map  when  when  when  like  I
wasn't  allowed  to  take  Q  equals  1  here
but  but  obviously  you  can  take  Q  equals
1  so  I  obviously  haven't  made  the  um  the
best  possible  claim  but  um  yeah  but  for
example  if  Q  is  a  root  of
unity  okay  for  roots  of  unity  we  know
that  the  norm  is  one  I  guess  we  we  said
it  from  plusus  one  I  did  do  do  we  know
it  for  did  skip  a  step  that  you're
exactly  pointing  at  which  is  that  let's
say  we've  said  that  Norm  of  Q  which  is  a
map  from  Spec  R  uh  to  0  Infinity  let's
say  that  we've  arranged  that  this  is  1
12  then  the  claim  is  that  if  you  have  a
q1  Over  N  then  its  Norm  is  has  to  be2  to
the  N
yeah  so  one  inclusion  is  obvious  uh  by
just  writing  down  the  diagram  with  the
nth  powers
and  then  you  can  prove  the  compliment
you  can  prove  the  other  inclusion  by
passing  to  the  complimentary  opens  and
running  the  same  argument  there
um  what's
that  oh  yeah  whatever  the  right  thing  to
write  down  is  yeah  thanks  yeah
yeah  I  mean  in  both  real  numbers  you
canally  right  yeah  but  you  have
to  but  I  do  think  you  need  to  invoke
this  fact  that  they're  closed  and  the
open  are  the  complement  of  each  other
to  like  with  with  just  with  writing  down
diagrams  you  only  prove  one  inclusion
but  you  have  the  nend  SP  up  on  P1  I
guess  and  then  you  prove  that  the
diagram  commute  because  you  use  the  both
ala  pces  yeah  and  then  you  you  consider
the  pullbacks  and  get  it  directly  but
I'm  not  sure  this  would
work  does  it  is  it
oh  because  this  is  a  terminal  object  or
I
don't  well  anyway  it's
true  well  okay  you're  probably  right  but
let's  let's
try
okay  Norm
Norm  maybe  you're  right  yeah  I  mean  I
was  I  was  thinking  about  yeah  I  don't
know
yeah  it  keeps  the  one  intersection
upstairs  yeah  it  goes  to
yeah  so  this  over  a
half  but  then  it  take  a  on  the  right  of
a  half  since  is  just  yeah  that's  true
that's  true  yeah  yeah  yeah  yeah  yeah
okay  um  yeah  okay  good  so  there's  no  no
subtlety  at  all  but  you  just  directly
check  that  the  N  through  of  Q  has  to
have  the
norm  uh  given  by  the  nth  root  of  the
norm  of  Q
okay  yeah  that  that  implies  in
particular  that  roots  of  unity  go  to  one
yeah
okay  yeah  see  you  on
Friday
so  what's  the  intuition
for
um  like  not  being  able  to  do  geometry
unless  you  choose  a  n  no  I  I  don't  want
to  make  such  a  strong  claim  but  the  I
mean  if  you  want  to  do  some  geometry
what  would  happen  uh  like  if  you  if  you
didn't  fix  wait  but  can  I  try  to  answer
the  question  cuz  yeah  so  cuz  what  I
meant  was  CU  I  have  to  clarify  what  I
meant  so  that  if  you  want  to  define  a
notion  of  geometry  that  somewhat
resembles  usual  complex  analytic
geometry  which  is  in  the  end  based  on
open  discs  closed  discs  and  so  on  then
you  need  to  have  something  that  measures
the  size  of  a  radius  of  a  disc  and
that's  that's  what  the  norm  does  but  but
isn't  it  kind  of  like  packed  already
when  you  fix  the  base  field  or  like  it
will  change  or  are  there  like  is  there
more  than  one  complex  analytic  geometry
based  on  the  norm  like  basically  I  think
this  is  related  to  the  question  about
the  equivalences  like  equivalent  is  up
to  like  the  geometry  that  will  arise
or  I  mean  you  could  CH  I
mean  you  could  you  could  choose
different  Norms  I  mean  obviously  there's
a  conventional  Norm  on  the  complex
numbers  for  example
but  but  the  fact  that  there's  more  than
one  at  least  in  principle  would  mean
that  it's  it's  not  the  usual  complex
analytic  I  mean  you  get  something
basically  equivalent  so  I  mean  you  could
scale  the  norm  is  something  you  can  do
like  scale  the  norm  by  some  Alpha  it's
you  you  get  basically  the  same  geometry
so  actually  one  thing  we're  going  to  do
is  we're  going  to  mod  out  by  at  some
point  we're  going  to  mod  out  by  these
sort  of  exponential  rescalings  of  the
Norms  um  when  you  do  that  you  can't  talk
about  a  disc  of  a  fixed  radius
but  but  still  many  other  things  work
okay  yeah  we  disc  this  many  times  so  if
you  have  AUD  uni  you  choose  let's  say
absolute  Val  Z  and  one  then  you  get  a
norm  on  this  and  in  particular  in  each
residue  field  you  get  a  norm  a  non
archimedian  Norm  yes  and  this  this  is
like  choosing  this  collection  ofan  yes
yes  so  it  is  not  at  all  the  same  as  like
so  so  let  say  for  another  feel  is
related  to  the  previous  National  Norm
but  for  a  more  global  object  it  is  very
far  from  like  the  bovich  space  on  just
Cho  one  of  quite  different  a  way  to
uniform  think  of  nor  this  is  the  inition
yes  and  so  you  and  and  presumably  what
they  said  is  an  equivalence  to  take  Uber
Rings  the  national  norms  and  your  sense
is  the  same  as  a  way  to  of  course  you
can  should  it  continuous  yeah  yeah
exactly  continu  on  the  back  of  rescale  a
given  one  so  it  is  a  torso  under  scaling
of  the  fix  one  that's  precisely  correct
and  it  it  sort  of  uh  sort  of  follows
from  this  uh  ah  okay  this  okay  so  in
this  example  this  is  what  you  get
exactly  and  of  course  then  you  can  uh  uh
and  of  course  for  other  analytic  Uber
just  locally  you  can  do  the  I  mean  you
can  and  for  nonanalytic  Uber
FS  this  is  something  different  well  it's
kind  of  it's  ruled  out  actually  I  mean
you  have  to  be  Tate  you  have  to  be
analytic  I  mean  we  we  were  saying  Tate
instead  of  analytic  in  this  class  but
you  have  to  be  Tate  by  this  result  if
you  have  a  norm  I  mean  then
locally  for  for  so  for  discreet  guys  you
are  not  having  any  non-rival  any  of
exactly  that's  right  but  what  we're
going  to  see  is  that  if  you  if  you  mod
up  by  rescalings  then  there's  a  way  to
there's  a  way  to  extend  this  so  when  you
mod  up  by  rescalings  on  so  Norms  form  an
analytic  stack  by  the  way
so  like  it's  a  it's  a  it's  a  it  fits  the
definition  of  an  analytic  stack  so
sending  r  or  Spec  R  to  the  set  of  norms
on  R  that's  an  analytic  stack  uh  and  you
have  to  put  covered  by  S  yeah  I  just  did
yeah  okay  okay  okay
yeah  okay  so
yes
and  but  and  uh  it  is  such  that  like
discreet  Huber  things  can't  can't  map  to
that  stack  but
um  but  there's  an  enlargement  of  the
stack  which  also  uh  accommodates  solid  Z
and  anything  living  over  solid  Z  and  so
on  which  we  we'll  probably  discuss  um  so
there  was  this  thing  like  in  the  theory
of  diamonds  and  so  that
that  you  want
to  get  from  an  analy  like  in  the  pic
setting  you  get  a  diamond  from
analytic  but  if  you  have  a  non  analytic
still  you  can  look  at  maps  of  of
antic  a  and  get  and  then  get  that  kind
of  there  was  this
uh  non  quas  I  forgot  now  how  maybe  it's
not  related  to  this  but  somehow  looks  I
I  I  I'm  not  sure  what  you're  referring
to  actually  because  there  was
a  diamond  only  allow  taste  test  object
so  even  if  you  start  with  theen  one  I
still  only  remember  how  tast
things  than  okay  it's
a
not  just  Gally  using  this  of  restricting
nor  could  you  repeat  that  comment  Peter
the  sound  wasn't  so
great
um  yeah  okay  I'm  not  sure  if  you  hear  me
better
absolutely  uh  I  was  saying  that  uh  in  my
work  on  diamonds
I  um  like  I  only  look  at  how  t  ring  so
per  pict  take  ring  snap  to
something  and  even  if  I  start  with  a
discret  ring  I  only  like  remember  these
kinds
of  things  and
so  using  this  notion  of  nor  analytic
ring  uh  one  could  now  similarly  restrict
to  non  analytic  rings  and  look  at  things
only  from  their  perspective  uh  which  may
or  may  not  be  something  want
do  but
it's  kind  of  analogous
right  yeah  so  like  yeah  so  I  guess  in
this  diamond  setting  Peter  proved  some
results  saying  that  even  for  like  a
discrete  ring  or  something  like  the
knowledge  of  how  it  maps  to  perfectoid
Rings  gives  you  maybe  quite  a  bit  of
knowledge  about  the  discrete  ring
there's  a  similar  phenomenon  here  where
actually  yeah  you  can  actually  use
normed  analytic  rings  to  you  can  even
yeah  you  can  basically  recover
everything  just  from  the  funter  of
points  on  normed  analytic  rings  but
yeah
\end{unfinished}
% !TeX root = ../AnalyticStacks.tex

\section{\ufs 6 functors for analytic stacks (Scholze)}

\url{https://www.youtube.com/watch?v=R5JNomeHjtI&list=PLx5f8IelFRgGmu6gmL-Kf_Rl_6Mm7juZO}
\renewcommand{\yt}[2]{\href{https://www.youtube.com/watch?v=R5JNomeHjtI&list=PLx5f8IelFRgGmu6gmL-Kf_Rl_6Mm7juZO&t=#1}{#2}}
\vspace{1em}

\begin{unfinished}{0:00}
The main thing I want to talk about today is the form of six funs for step. But before I get there, let me try to say something that Dustin kind of said last time and that I kind of also said last time, but we both didn't quite really do - to do the construction.

First, the construction of the norm on the SK fa. Conveniently, Dustin kind of already explained that we can assume we actually have all the roots of Q as well. And so then we want to define the algebra that corresponds to the locus where the absolute value of T is at most R and, or like the over convergent functions on this disc. As Dustin explained, the ex of Norm some tell you that this has to be like there's an if guess for what this might be. So this might be like this free algebra on the top element t or scale suitably so that the norm is equal to R, but that's not quite right, but it's almost right. But when you pass the over convergence thing, then it actually becomes exactly correct.

I will construct the norm for which the absolute value of Q will be a half. I want to take this here A and so I want to take a p a module on a top l element where I rescale T by some power of cube. And so this should be such that this definitely converges here. We want that we should secretly think that maybe the norm of t r, and then we want the norm of Q to the T * T this should be bigger than one. But okay, so the nor of Q is half, so a half T * R should be bigger than one. And now you can do the algebra. This means that 2 to T must be less than R, so you take the co liit of all t for which two to T is less than R.

This is a condition that if R is very large, say, then we should multiply by a large power of Q to make it somewhat small. Okay, was there a disagreement or just commiseration? Okay, thanks. So this has is the only possible choice is what you explained last time.

And so what we have to see is that these unimportant have all the required properties to Define well, first of all, you have to see this Al item and they plus the anog for G inverse and inin Define map. Okay, now you can replace T by T inverse and then the T supp Infinity.

Okay, so here are some ways to and I mean all of these are some kind of simp computations. Let me just check that the locus where it's at least R and the locus where it's bigger than equal to to S where R is less than S this would be empty, and for this you have to compute something like you take a and then join to T * T head and say noers. And then you can draw in T and the S time T head inverse. You want to show that such a thing is zero for if say, and one situation says R is less than one after rescaling, you can assume that some one is squeezed in between them. And so then you can assume that t is negative and S is also negative.

But in particular here, you have a map from two toping elements which are X goes to Q to the T3 and Y goes to Q F and say satisfies relation that if you multiply the two you get something the negative power of s, so Yus Q to the some axy is equal to zero where a is some positive number. But and in turn if you just remember the product, so then you also have this is an algebra over a Jo C Aus a * C, but this is precisely I mean a join Z head, this is precisely this projective generator that we always use, and then here you're doing oneus Q.
To the a * shift, and if a was equal to one, then we precisely declared this to be an isomorphism on the projective generator in the gas is ring structure. I also explained somewhere that the power doesn't really matter when you do this, so this implies the pr fractional power. So, this is actually zero, and this is really by the definition of the right. So, the minimal thing we really need to get this map is that we know this, which shouldn't intersect, and this really exactly comes down to the thing we enforce and the guesses. Then, there are some other things you have to check, and none of the computations is hard.

So, those, and then you have to check they cover that the guys like the they need to check the cover, but again for this, you just have to show that. Actually, I will probably come to this later in the course again, but I mean basically, you just have to, if you say we SCT to the Fine Line, then you just have to show that. Then, you expect that there is some they form a cover, and so they expect that there is some St like sequence like functions on the fine line just pooms going to functions on a dis and functions on functions on the opposite dis and then functions of the overlap, and for all of these things, you can just write down what they are and then just check that you get an exact sequence, and this implies the covering condition because it implies that the polinomial are generated by in a finary way by modules over over the ah, okay. So, you just have to check a cyclicity like the analog of data cyclicity, you just have to check theity.

I mean, this is enough to check the covering conditions. By the way, the reason that we really want to use this growth topology that we chose, I think there's another perspective here that makes some of these things easier, which is like instead of using this free thing on a topologically nil poent, it's like the free think where you make t topologically nil potent, which is not really like a subset. You could use the subset where te becomes gaseous like an analytic ring that's not induced, but they're all sandwiched between each other, so for the over convergent thing, it doesn't matter. Then, at least you have honest subsets, and then it's like kind of easier to check which subset. Oh, like I'm saying, there's a instead of using this thing I was calling D as your basic dis, you could instead use an analytic ring which is the analytic ring gotten by forcing the coordinate T to be gaseous in the sense of like one minus t * shift. What was D? Oh, D was just the spec of the pr this array which is this U, let me so I keep my notation keeps this notation, which to the line and then projective line. So, that's one thing you can consider. On the other hand, you can also Define the following funny analytic ring. So, you can take the polinomial and then make t guessers in the sense that you enforce 1 minus t * shift as usual on Project generator to be Aus. What is sh here? I mean, on this t-basis change to this ring, and so this is also an analytic R, and you can also take the aspect of this, and this is actually a Subspace of of P1 over a, and it's again a Subspace that will behave like like dis behave, so it will be training Norm, it will be contained in the locus, so it is contained in Long inverse of 01 and it contains Norm inverse of the op of open inter relation don't each other. Well, what one you can't say contain, and you could even consider a third version which would be the intersection, so you can also intersect these two. Then, something that Dustin mentioned last time is that actually, if you intersect these two, this map becomes an embedding, and also, this one becomes an embedding. You can also consider a Jo T head guess, and this is also true here. So, all of these three things, there are some slightly different versions of what a dis might be, and there also sandwich between like the over converging close unit disc and the open unit disc.

In particular, one way to verify the item potency without really sort of without doing any calculations is to see that the this da and this npec with the T gases are are sandwiched in between each other in this co-limit, and then it formally, since for one of the terms, you always have a monomorphism, it follows for the other term, you mean, there one thing you really have to check which is this that these two do not intersect. I think
Clever way to arrange the argument. You can avoid them, but yeah. Yeah, so if the radii are the same, they always intersect. Any disc around zero and one around infinity will intersect. Okay, for all versions of the story. Okay, yeah. I mean, it well, it depends on which base you're working over, okay, but over the universal base. Yeah. Okay, so all right. Um, yeah, so let me start by talking about six functors.

Let me first recall that on an analytic space, we had a functor that was taking an object $a$ to $d(a)$ and we also considered a different functor which was to look at the representable stable infinity line over $d$. This was a functor from the category of rings towards symmetric monoidal infinity categories. In particular, the commutative algebra structure here has insisted on some $T$ functor, and actually, there's also always in both cases they're closed, so there's an internal Hom, and it's also functorial in the ring. This gives you, in particular, for any map of rings, a base change functor, which geometrically is always $f^*$ and $f_*$.

These are four of the six functors, and then we defined this class of free maps, and for these, we had also a functor of lower shriek and upper shriek. Why does he move discretely? Because of the, the, yeah, something, a one-picture instead, yeah, another high quality. Camera usually, like, you don't see, right, exactly. So, we, right, um, so one convenient way to package all this data is in the notion of an abstract six-functor formalism, which I call the following.

Whenever you see any category $\mathcal{C}$ and $\mathcal{E}$ some class of morphisms, let me always assume that my infinity category is all finite limits and colimits, and here is a class of morphisms stable under composition and base change, where composition includes the empty composition. Then, we can define a new infinity category $\mathcal{C}_\mathcal{E}$, where the objects are these spans, and the composition is given by taking two such objects and following the fiber product. It's a symmetric monoidal structure, and there's an obvious way of taking a product of two morphisms.

We have this definition that, well, first of all, a three-functor formalism, where the three functors here that I'm talking about are $f_!$, $f_*$, and $f^!$ when $f$ is an $\mathcal{E}$ morphism. There's a correspondence between this category with the end of product and infinity categories with certain properties. In particular, I would like to discuss this example where these categories themselves are not presentable.

There's also a six-functor formalism, what is the just-Stein product, and the means in which direction there are two ways, like $d(x \times y)$, right? So, yeah, so "lights" here means that you only roughly speaking, you only have maps from $d(x) \times d(y)$ to $T(x)$, but they need not be equal; they just have maps. And if you apply this to the empty product, this also means that $d$ of the final object has a distinguished object.

Like, from the empty product here, so from the final category, so the point, there's a map to the drive category of the empty product, so $P$ of the empty set, and from there, from the empty set, you can go further to anything else by a pullback, so some particular, like all the, all the you have access here are pointed, which is giving you the unit objects, really.

All for point, I think it's of the empty, it's sorry, it's, yes, think it's than you. All right, and so, a this is, very difficult amount of data that's, it's not so clear how to construct it, but conveniently, like Le Jen proved some very difficult Comorbi recipe. Well, the recipe is extremely easy, the proof that it works is rather difficult, um, but that makes it extremely easy in practice to construct these things.

In particular, like for $C$ brings up and $E$ the streetable $NS$, um, sending a either to $D$ of $a$ or to $crl$ of $a$, um, which I'll probably discuss in a second, because, okay, let me just dove this one for now, um, uh, is an example.

Um, and so, uh, but now this and then some to the category of analytics text, and we would like to extend, uh, this, maybe to also some larger class of fre maps, well, not just I mean the fre maps of aine, certainly you want also base changes, but maybe much more than that, um.

And so, uh, yeah, so here's the, here, um, there is a minimal cost of morphism, um, unique, um, morms, the following property. You mean there is the smallest, because any class satisfying the condition contains this, yeah, any other class that satisfies the following properties will contain this class, yeah. So unique minimal is slightly different in some conventions, because it means that the pos it as a, as a unique minimal thing, it doesn't mean that it is the smallest, because so it's, but this depends on your notions, but it's slightly, according to some definition, better to say smallest than unique minimal, but because minimal doesn't imply that it is smallest in for parti other sets, unique minimal doesn't imply it smallest.

Um, uh, properties, so, uh, first of all, the the class, uh, a class of morphisms of maps of analytic rings define maps, that's something we want. Um, also we want that if we have just a disint union of fre m recable, so um, stable under unions in the following sense, so if, uh, you have some $f$ from some $XI$ to some common $y$, uh, all of them, then $F$ the dist Union of all the $F$ from the over $y$, so from the of all $x$ yint this able $i$, it's clear how you would define a l stre function in this case, I mean, the drive car of the dist a product, and then you just take the direct sum of all the streaks.

Um, it's this class is local on the target, uh, in the following sense, so if you have a morphism of analytics text, such that the exist some cover, sorry, I wrote $X$ before, so stick to this, um, that exist some $y$ Prime, subjecting onto $y$, for which, uh, $F$ Prime, which is a base change of of $f$, uh, which fime isn't class, uh, the class, say, uh, I'm not quite sure whether I need to explicitly say it, so I definitely want this class in the end to be stable under compeition and base change, and I'm not sure whether I should, whether this follows at some point, but always assumption.

Um, all right, so, uh, right, so this says that to check whether something is triable, you can, uh, localize on the Target, and by the way, I mean, if you wanted to find the lower streak funter for such a guy, it's actually quite clear you would do it, because you already have this ler streak fun for this one, and you know that streak fun they always come with base change, so to define lower streak on $y$ Prim on $y$, you can do it after this base change to $Y$ Prime, but then all the further
X. G here. The following properties that G already lies in each other and is of universal strength in the sense that I will explain in a second.

And the composite and the composite. MTH, so you should think that this is rather stacky, but now you've covered it by something more reasonable for which you already know that the composite map is is is needer then the original map. Maybe I should make some remark now.

Did you say what Universal strength descent here? I mean, strength descent after pullback to any fine, so because this map already has strength maps, you can always ask whether the like the r of X meaning strength maps to the r ex and all fiber products an equivalence, and I ask this not only over X but after pullback to any.

But Peter, doesn't it follow from the assumption that it's a cover in the gr topology? How do you know is very very true. Thanks, thank you Dustin.

Yeah, what is this? So, how do we know that this, how do we know that there is a, for those kind of, oh, it's the assumption that G is in this, well, G is an. No, if you mean no, no, no, but EA, we don't know that there is, the theorem say there is a minimal class, and wait, wait, maybe maybe I need the next condition to to justify it.

If you have an $m: X \to Y$ is an eer, and why is that fine, then I actually ask that you can always find a cover of $X$ by things which are over $Y$ which I, and fin, then all the fiber products are also. Because and all the, also, $X_i$ is covering over $X$, and then you need the full simplification. So, if I if I if my base is fine, then I make a strong restriction on anything that's possibly allowed, then I always ask that I can always further find such a shable, I mean such a yeah, shable $C$, where all the $X_i$s themselves, they lie in this original class of coverable maps.

Peter, don't you also need a variant of four with disjoint unions, like, or would this be strictly more general? Like, doesn't two say that I could always, no, but two doesn't say that $X_i$ to the disjoint union is in $E$. In other words, if you want to localize by using a family instead of one object, then you need some argument to reduce to the one object, and it seems convenient instead of two to have also the fact, in addition, I guess I suppose in some places that $X_i \to X \sqcup Y$ probably it should be in your class because it's just, yeah, but this is just because it's locally in the class $X \to X \sqcup Y$. Why is it in the class? Because it's locally in the class. Okay, okay, okay, okay, okay, okay. Just sorry, and then the empty to the something is always in the class because it's why, empty to anything is in the class, may you have to add it. No, this might be the empty collection. Ah, okay, okay. No, the empty to something $X$ is empty and then going to $Y$, this would be in the class, yeah, you can reduce to $Y$ being $\emptyset$ by, by, yes, and then it is in $E$. Okay, okay, okay, okay, okay.

Okay, all right, I hope I'm not screwing up, but, right, so to, you see this Universal strength descent after pullback to fine for such a map, then, after pullback to fine, you're in the situation of this one, but then you always know that you can refine by cover by strength, find for which you know strength. And there's one last condition which is really one we're interested in, that the six fun or the three fun formalism than you, Neely, from CE, I mean, from analytic rings and one we have to analytic STs uniquely in the space. This you have to put it in ER to the prev, yeah, so for some of these Surs I'm implicitly using that, I mean, okay, let say that. Yeah, so this follows from results, from exra, I mean, this has nothing to do with a specific six dealm, this just an is all about six formalisms, and I've given some account of this in my notes on six funes using Min different setup, but I think the same argument works here. All right, so that's an extremely

Because it gives us this extremely structured formalism of six operations. Whether I say three or six, it doesn't really matter. Here, let me say six, because for this, these are all presentable categories, and the fun $f$ is all co-limit. Then, as Dustin explained, we always have the right joints anyway.

So, you have these six funs. Now, for some classes of stacks, and if you want to check whether - and it's in practice extremely easy to check whether any map - um, has or, in practice, basically all morphisms have three funs. And using some of these criteria that you can localize on $S$ and $Target$, it's also usually rather easy to check that this is the case. So, you can first simplify the target as much as you want by localizing, and then you just have to find some kind of presentation of your guy which makes it simpler.

All right, let me say a few words about how it's done. It's kind of hard to explicitly describe, but it's easy in practice. You can check that the given map belongs to this class. So, let me just give an example to illustrate this a little bit. What happens? This must be covered by, but then is it enough? No, then you iterate, you do that again and again and again. You iterate, working locally on the source and the target, and you have to iterate and transform this, because once you get new shakable maps, then working locally, you get new things that can be covered by those.

Let's assume you're in a situation where you have some $EX, Y$, where maybe the $Y$ is something simple, just a point, and the $X$ is something spey. But then you can find the cover where, for this, you already have street maps. Then, the $D$ of $X$ because of the cover, it's a limit of the street maps of $EX$ and $Er$, but this is also the same thing as again the co-limit along the lower street funs. Now, taking in $PRL$, these are the upper street fun, but you have lower street fun in the opposite direction, and for all of these things, you already have lower street funs, and they are compatible with the transition maps. Thus, because it's a co-limit and this guy is presentable, all the $M$ preserving, you get the unique $L$ street fun.

Let me actually give an even more specific example. Let's consider the interval mapping to a point. This is not something which happens in condensed sets, but as we said, condensed sets can be embedded into analytic stacks. As $s$ explained last time, there is this descendible cover of this by a contour set, and this map here is stable, because when you face change to some cover here, so you can check it locally. And if I base change this to the contour set, then the fiber product here becomes itself some live profile contour, and this map here is just a flat map, I mean, it corresponds to a faithfully flat map of fine things, and so this here is an $EA$. And also, this map here is already in $E$. And so, up prior in our world, we've only defined street maps here, and we didn't define what the street map would be for the general compact Hausdorff space. But then, it turns out that in fact, at $L^{\star}$, one can observe that actually $F^{\infty}$ happens to also be an $L^{\star}$, as you would actually expect, because this looks like a proper map.

Let me do yet another example. This way, you can automatically extend from profile sets to compact Hausdorff spaces, at least the finite dimensional ones. It also gives an extension to all compact Hausdorff, no, let me not say that, to the finite dimensional ones. But then, you also have locally compact Hausdorff guys, so, for example, you have the reals, and now maybe this is a point where Dustin said I should have taken this $J$ somewhere. Oh, there it's okay.

This is covered by the disjoint union over all $n$ of the interval from $-n$ to $n$, and because, why is it subjective? Because, from the fun points, as condensed said, this is a union of these things. So, this is definitely subjective, it's also, and one

This argument will generally work for any kind of locally compact, supportable, and five-dimensional thing. We get some FL streak, and you can actually check that it is a usual streak. You can observe the comp, and why? I mean, you can look at this definition here. The drive category of R can be written as a Coit of the graph carries on these minus n NS along the lower street maps. To describe the low streak, it just means the col presing fun, so it's enough to describe what it does on all these things, but here it is just chology which is compact for chology. This very abstract procedure actually recovers a lot of like upior geometric intuition about what a low-frequency function should be of.

When I discuss the t a look curve, I was mentioning that some should be proper and that it should be smooth. Let me say a few words about smoothness. There is an inductive definition, a morphism, and why? I actually wanted to be in the, and then let me say proper. Maybe weekly proper, it's not, it doesn't precisely match one something I call proper in the context of an abstract. Let me just say proper, and maybe Dustin will complain that he wants this for something more specific.

There's a question in the chat from Peter asking if we have any explicit description of f uper shriek of the unit object for FX to a point in the category analytics tax or at least in some special cases. This is a very good question. Let me refer to the discussion of Smooths coming up in just a second. If you look at like f low streak in these examples here, it was just usual f low streak, and so then the f uper streak is also usual dualizing complex. In virtually any situation where you already have a dualizing complex, our up streak will agree with what you say is dualizing complexes. Except, one thing you have to be careful about is that if you embed schemes into analytic stats in this way, in this kind of with a tri analytic ring structure and all maps become proper as we said, the f l streak is always f lower star, and the upper streak is then this funny r joint of a lower star that is sometimes considered but is not so well behaved in some ways. If you want the usual dualizing complex on a scheme of finite type or something like this, you should rather use this other embedding using relative s with this caveat.

Let me try to describe find prop morphisms. One case that I definitely want to be proper is that it's relatively representable on f lines, so meaning that after you pull back to any fine becomes fine, uh, repres FES. I'm sorry, the following and proper you want toor for all Prim y, y um X Prime, which is the final product F and X Prime to Y Prime is proper in the sense that we used before for F, so that it has structure. Sorry, Peter, we were just discussing a remark that in the setting of aine analytics, Stacks the prop, you know, this the proper Maps satisfied descent for the gro and de topology. Someone pointed out that whether I say, well, they don't, right, because they have some connectivity issue. That's true, but I think I pass once more to diag or something I can avoid this. So, let me make this, so therefore, it's not the same, not the same as as so as being locally no. Yeah, we should, so we really shouldn't ask for all why Prime mapping to. Yeah, yeah, they will be a second part of this definition. I would hope that, was the second part, it wouldn't matter what I said here, but okay, I didn't carefully check it, so let me maybe not. However, one thing that this implies is that if this so happens in case one, then you already know that f streak has a canonical equivalence with f star, that's given to you because in for these Maps, we Define the f streak to be the f star, then because you can Constructors ISM locally d canonically, get it. So basically, the class of program that's be the one for which a floor speak is a floor.

But, but, say it like this: it's not a condition, but I want to say something which is really just a condition. And one way to do this is to, oh yeah, like for these guys here have such an identification. And in general, you can ask a diagonal $\Delta F$ from $X$ to $X^*$. This is why I'm saying it's an inductive definition.

So, this might be proper in the sense of one, or it might also be proper in the sense of two, in which case it passes the diagonal again, then hope those are proper. And once the diagonal is proper, you get an identification between $\Delta FL^{str}$ and $\Delta FL^{star}$. And this allows you to construct a canonical comparison map from a floor $F^{str}$ to a floor $F^{star}$. Maybe recalling the second where it comes from and induced, does it follow after? And actually, you might think you want to also ask this after any base change, it actually follows. And actually, something even weaker of $F$ is $F$, speak of the unit, and then solve it after the new. So, does this lead to a transfinite process? Namely, suppose you have, you can use $\mathcal{F}$ many times, and then you can say, no, I want us to determinately find many things. Yeah, there's no way to get it to go, let us say, suppose that you have a cover $Y'$ to $Y$ by half. Ah, in this case, one is a terminating condition, and the other is just you can apply $\mathcal{F}$ many times, but at one point, you should run into the one condition.

Okay, so the joint union of countably many guys $X_i$ to $Y_i$, which are proper in the $\mathcal{Y}$ sense, will not be proper if when the sense goes to infinity. No, no, sorry again. When I said there is a cover by such, I was allowing that it's a cover by a collection of such. No, no, no, but I'm asking, suppose $X_i$ to $Y_i$ is proper, then the disjoint union of $X_i$ to the disjoint union of $Y_i$ is not proper. And I think it follows, no, because if you have a sense one because of, ah, because. Okay, I see what you're saying, but maybe you could just take this as the terminating thing as a definition, but, okay, so maybe it's not proper in my sense right now.

Yes, okay, what's the problem? It's not the problem. There's a variant where you, because this is just simple induction, yeah, without going to Omega, yeah. So, suppose you have $X$. So, yeah, so maybe I should really make this, I should say the proper is first defined proper, both of these things are like $\mathcal{QQS}$, because I think this should be the case. And then, in general, if it's true after pullback $\mathcal{J}Y F$ or something. No, no, because suppose you have $\mathcal{QCQ}$ things $X_i$ to $Y_i$ which are proper in the, but in higher and higher senses, you need higher and higher, this will not. All right, but I'm not exactly sure in this theory how much you can reduce to $\mathcal{QCQ}$, because anything is a cover, but this. Yeah, the the equivalence $\mathcal{R}$ itself is not $\mathcal{QQS}$ in general, so it doesn't, you don't have good reduction, because I, the proper case, these things I think should be $\mathcal{QQS}$ because here the basic ones are $\mathcal{I CPS}$, and then the diagonal is. And I think from this you can deduce that $F$ must be $\mathcal{QU}$ compa, at least because it implies that the floor is talking to $\mathcal{SCHORE}$ limits. So, these things should be $\mathcal{QQS}$ in a very strong sense, these maps, all the maps of 

Itself, then it's Asis funs and it's true of Base change, and again this is something you can find in my notes on six fun EX for. And so, in particular with this definition of what proper is, then, yes, so then this map here is actually proper in this non-generalized sense to any careful. However, that if you have some kind of H Cube mapping to a point, then I'm not sure whether it's actually in. DUS, do you know? In any case, it's not proper because there is some the lower star fun doesn't actually con with col because of some infinite ises.

Right, so you can Define this things, and you can also Define smooth. So, for smooth, for smooth system, I want that the upper streak is basically the same as the paper star, but in this case, you don't actually usually expect isomorph on the nose. You rather expect that there's a same some twist, and also for smooth Maps, you would never assume that this is a condition that's stable under passing to diagonals.

So, the definition of SMU is, for this reason, is a bit different. And here, I us call the things loic, and this now really matches the abstract thing defined for any six fun form is if.

And now, I really want to ask, after there any base, sorry, this CL maps on Bas change? So, there's the natural transformation from pullback Stander with f upper streak of the unit, what upper streak, and of unit which will be some object of X, is actually an inversal object tender and commute to space change. And so, I mean, you can form this F upper spe to dualizing complex for x or y, but you can also do it for a base change, but it should also be an invertible object, and I ask that the formation actually commute to space change. Again, you can somewhat reduce the amount of data you actually have to check.

Let me rather just briefly say where the center comes from. And so, again, I'm mapping to write a joint fun. So, what you try should try to do is in produce a m from the corresponding L stre, but then this is precisely a projection form your situation here, pull back of something. And so, then this thing, but then, it's just by a junction. So, this class is St on the base change, uh, composition. I mean, also problem St in the base change, uh, composition. It's in some sense local on the source if you smooth cover which smooth, and okay.

So, there are some ways like that to be able to talk about some Notions of proper, smooth maps, and I mean, they some have the expected consequences on the rough categories, and so you can, yeah, execute many arguments as usual. So, I just want to briefly now come back to ttic curve.

All right, so, a is now again this guest displacer, and so now we have the norm. And so, we can def find the analytic over that's a of. So, we have multiplication by two, all right. Now, we can make some assertions.

First of all, this St here, this is proper. Well, it's not representable in Fes, clearly, but the diagonal is, because it's just a diagonal P1 f, and you can check that p stre and p star the unit just agree, and this is actually just the I don't know, you can check that in algebraic geometry, proper, and it's also smooth.

And for this one, should I mean, I you can just really just check that directly, it's not hard. You can also do the following argument that, whenever you have like the morphism of six fun formalisms, so, for example, like schemes mapping to analytic Stacks, then any such morm of six fun fors will always preserve comically smooth Maps, because there's some kind of diagrammatic way of encoding it in the six fun formalism. And so, the smoothness that you somehow know in algebraic geometry automatically implies it, the algebraic version implies anal question.

This J here, it's actually also, I mean, it's also shable, and it's actually somehow, I mean, I didn't Define it, but I could have defined also in an open imersion. So, this just means that, see identity, and again, this is something you can basically reduce to just zero infinity and like the open guy.
The close guy, very some everything produces the usual six funs on locally compact $\mathcal{A}$ spaces. These are, in particular, also like $\mathcal{J}$ lower streak, so $\mathcal{J}$ upper streak in this case. $\mathcal{A}$ jerar particular is also, so this guy is not proper, just like this, I mean it's not $\mathcal{F}$ compa. Okay, but now again, we can take since $\mathcal{T}$ the curve, which is quotient. Ah, but one thing that is proper right, so the proper Maps they satisfy the two out of spe property because they were Som defined by passing, and this map here is actually proper. And so it also follows that if you pass to this and $\mathcal{G}M$, then it's still proper over zero.

Infinity, there's a question in the chat, Peter. Yeah, are open immersions of analytics Stacks representable? Absolutely not representable by $\mathcal{F}$. I mean if they were represent by $\mathcal{F}$, in particular, would have to be qua complex, but this $\mathcal{M}$ is very much not quic complex, so we get much more General open Emer.

Right, but now the $\mathcal{S}$ tatic curve, it's a of analytic $\mathcal{Z}$, and so this to $\mathcal{Z}$ Infinity mod rescaling by po of two, which is a circle. And so like the base change to $\mathcal{Z}$ infin you see that this map is actually proper, but if you project as one to the point, this is also proper. And so you see that the $\mathcal{T}$tic curve is proper or the point, the point is now like everything.

So you see that this constru gives you an $\mathcal{A}$ntic $\mathcal{S}$t that's proper, but also, I mean this map here is locally split, and so this means also this is here Lo local is the same as this $\mathcal{A}$nalytic $\mathcal{G}M$ which was smoo, so also the sky here isical smoo.

I find it a bit remarkable how you can like combine intuition of about the properness of compa spaces with some really algebraic geometry kind of properness and so on, and it all works really well.

Right, so, yeah, also it comes with a section as a unit section $\mathcal{I}$ given by one and $\mathcal{G}M$.

So, there some of the claim that this $\mathcal{E}$qa curve, which is $\mathcal{A}$nalytic $\mathcal{S}$t over a limit $\mathcal{A}$lun from $\mathcal{S}$ches schemes over the underlying $\mathcal{N}$ice $\mathcal{N}$ essential image of the fully faceful embeding from schemes over the underlying ring and ring ration.

So, to execute this argument in a really nice way, I should prove a little bit more about this ring $\mathcal{A}$ here. In particular, classify like the dualizable objects in $\mathcal{D}$ of $\mathcal{A}$ that they are just perfect complex, which I think is true. Let me do it a little bit more by hand because it's fun.

Modle and I asked that it SL on the essential image of H. Me what, sorry, I because it might be derived. In my situation, I'm in everything, there's no derived structure, so it doesn't really matter.

I definitely want that there's noology and, um, assum have this situation, um, and it's not clear the difference between Poli. I mean, this, it's just, I'm just trying to say you have an an line bundle in some approximate sense. So, for example, I want that it's Global, it or repl by some tens power, it's globally generated, uh, and there's no higher chology of tens powers which you can arrange. And I also want that the global sections are not something, some just to the SC module, because I want an algebraic schem in the end. So then we can define something algebraic to be just the project of the. Is this AR GMA? Is it just H zero, or it have a positive? Yeah, I mean, if some X is actually derived, it could have positive stuff, let's not worry so much about it. So you can Define this thing here is a direct scheme over, and you automatically get a comparison that, uh, from this algebraic guy, analytic fied, and, because, uh, globally generate, you can actually check that this is really a qu comp guy.

Okay, this is the stand, but it's not enough to, it's not L, to I hope. Um, so something one can say in the situation is that, uh, so anything comes from algebraic geometry, as we said, it's automatically improper, so M actually automatically improper. Meod, but also, uh, a simple observation is that in this abstract setting, automatically, it will be the case that aest of the unit is the unit structure, she is a structure, she, um, why, because you can check this, uh, we can check after the pullback, uh, to f, f, x of x, uh, but there on global sections, because I'm the fine guy, like of CH is just of the global section, because it's e, um, uh, but, uh, but this inclusion here is really just given by algebraically inverting something, and so then this a star here is then just some some filter po Lim of inverting some function f, so this actually just reduces to, uh, to to the isor on sections that we know. I mean, so this is like the plus of some f, where f is some fun, some Global section of, and then, uh, you're really just getting Co liit over multiplication by F of global sections, sorry, the Al, I mean, either on the algebraic guy or on the on X, but some the AL guy was ripped, so that like the go section of the line bundle are the same on X and the albra what K, the co liit over K transition modication by some section of l, f section of L, the K or n, maybe I don't know. Uh, F was an unfortunate Choice, actually, because that was your map from X, oh, sorry, yeah, thanks, thanks, sorry, G, did I use G? Let, all right.

So, uh, right, so you have a proper map of analytics F, where FL St unit is unit, uh, this actually implies that it's a s active net, because, like, yeah, this this is a SE true after any pullback, because proper St stre any pullback, and so this means that after pullback to any Fon, you can get the algebra, uh, in terms of the image, and so this actually means that f is subjective. All right, and so what does it now take to show that F isomorphism, uh, see, I mean, this generality it won't automatic the case that it's an isomorphism, but, uh, to check that isomorphism, uh, it now suffices to show that the diagonal of f is an isomorphism, right, because if you have any subject M of sheets, it's Isis only diagonal Isis, uh, but this, the diagonal of X here, he want just this a pull back here, uh, actually this also implies something else I should have said, also implies that the pullback map from the garage category of this guy is actually inly Faithfully into X, because, um, because if I look at low star upper star, then by the proection formula, this is just

So, if it so happens that there is a map $\mathcal{D}$ of $X$ is $\mathcal{F}$ on, and $\Delta x$ of one is in the image of $f * \mathcal{F}$, we discuss to show that to give an example of a proper map, explain the $\mathcal{C}ur$ algebra.

Okay, so I claim that I only need to check that this diagram gives a diagonal commutes. But in practice, like the diagonal, I don't know, some kind of complex or something, this, this, some risky $\mathcal{C}los$ immersion. In particular, some kind of $f$ math, and the $\mathcal{X}$ $\mathcal{S}$ of one is really just some coherent $\mathcal{S}he$.

I claim that it's actually surprising to see that this lies on the image $\mathcal{F}$ time. Then we win because of it $\mathcal{L}IC$ image of times upper star, and it's actually the same thing as applying $f$ $f$ lower star and then upper star again. But if you apply the low stars and you run through this diagram, you see that it's actually the same thing as a pullback of, which actually implies that it's equal to $\mathcal{T}imes$ of $\mathcal{D}el$ algebraic.

But then this means that, yeah, but then this means the some of $\mathcal{P}$ diagram. All right, so the upshot is, if you want to prove that some such analytic space, proper analytic space is algebraic, you have to produce an ample line bundle in that sense over there. Then you automatically get this kind of comparison $\mathcal{M}$ to some algebraic something algebraic.

The only thing that remains to check is to see that the structure sheet of the diagonal lies in the subcategory that comes from this algebraic.

My time is up, but these are things that are simple to check in the case of static $\mathcal{C}$. So, you can, for this $\mathcal{L}$, you just take the $\mathcal{A}$ mod given by the zero section, and then you can actually just compute what all the global sections are just inductively. The global sections of the structure sheets you can just compute the $\mathcal{A}R$ easily, and then the rest you just acquire copies of just the zero section in addition, which are just your base ring. So, this is okay, and well, and for the diagonal, you can actually use that this is a group, so you can move the diagonal to the $\mathcal{Z}$ zero section, and the zero section is easy.

All right, my time's up, so let me stop here. I want to clarify just from the Viewpoint of General scheme Theory some point here. You consider $\mathcal{P}ro$ of rings which are not assumed to be finally generated, that is, suppose that we don't have a derived ring, we just have the usual ring. For example, the graded ring of $\mathcal{H}$ zero on some scheme of powers of a line bundle. So, we just interpret what you said for normal schemes for Simplicity, because I'm not sure about.

Now, when you look at the notion of proper, like $\mathcal{R}otend$ defines finite type, separated, universally close, but of course, you people thought about the case of non-finitely generated. So, you can have a universally closed, maybe separated, but then this turns out to be a projective limit in the $\mathcal{Q}C$ $\mathcal{Q}S$ case, a projective limit of usual proper things. So, my question is, suppose you have a $\mathcal{P}$ of the type that you consider. What we know is that it's covered by Finly many opens, a standard basic principle opens, whatever. But is it the case that it is proper well in this universally closed sense?

The problem is that, of course, in the $\mathcal{T}$ litic $\mathcal{C}$, I think the algebra will be finitely generated, but I allow the case that the graded

Let's specialize the notion of properness I defined here. In that setting, you get the usual notion of properness. Okay, because it implies COMP. Okay, all right.

\end{unfinished}
% !TeX root = AnalyticStacks.tex

\section{\ufs Berkovich spaces (Clausen)}

\url{https://www.youtube.com/watch?v=fnEPiDIF9_k&list=PLx5f8IelFRgGmu6gmL-Kf_Rl_6Mm7juZO}
\renewcommand{\yt}[2]{\href{https://www.youtube.com/watch?v=fnEPiDIF9_k&list=PLx5f8IelFRgGmu6gmL-Kf_Rl_6Mm7juZO&t=#1}{#2}}
\vspace{1em}

\begin{unfinished}{0:00}
  e
so  we're  we're  nearing  the  end  of  the
course  actually  and  um  at  the  very
beginning  in  the  first  lecture  I  I  kind
of  gave  an  introduction  to  what  we
wanted  to  have  out  of  this  theory  of
analytic  stacks  and  in  particular  I  was
putting  some  emphasis  on  the  fact  that
we  want  to  know  that  traditional
Frameworks  for  analytic  geometry  can  fit
into  this  perspective  uh  of  analytic
stacks  and  we've  sort  of  already
discussed  um  attic
spaces  and  little  bit  of  geometry  over
this  um  the  little  bit  of  things  like
like  tape
curves  um  but  what  I  want  to  adct  spaces
there  was  this  issue  of  shiftin  and  then
the  idea  was  that  you  have  to  Define
some  kind  of  derived  adct  spaces  this
will  not  explain  in  full  what  is  Uber
Uber  set  up  and  in  your  formul  yeah  yeah
yeah  maybe  there  are  different  ways  to
I'm  not  sure  it's  possible  there  are
different  ways  to  do  it  yeah  so  this  is
yet  in  in  progress  already  yeah  you  yeah
let's  say
that  or  something
the  it's  in  Juan  estaban's  paper  on  uh
the  analytic  the
ramstack  that's  okay  so  there  you  go  um
all  right  um  okay  so  but  what  I  want  to
start  moving  towards  today  is  uh
relation  with  burkovich
spaces
um
um  but  before  discussing  that  um  I  want
to  just  do  a  little  bit  of  setup  um  so
but
first  I  want  to  talk  about  different
Notions  of  how  to  localize
uh  uh  an  analytic
stack  uh  sort  of  over  at  topological
space  um  so  well  we've  already  actually
seen  one  way  so  if  um  so  let's  say  well
let's  give  ourselves  an  analytic  stack
um  and  if  we  have  uh  s  which  is  maybe
let's  say  a  yeah
metable  finite  dimensional  compact  house
door  space  or  it  could  be  a  maybe  a
locally  compact  house  door  space  which
is
locally  one  of  these  um  then  we've  seen
that  that  can  also  be  viewed  as  a  as  an
analytic  stack  so  we  can  also  view  it  as
an  analytic  stack  um  and  then  we  could
just  ask  for  a  map  uh  F  from  X  to  s  in
the  category  of  analytic
Stacks  um  and  we  discussed  uh  kind  of
what  what  structure  you  get  on  X
basically  you  get  a  bunch  of  uh  item
potent  algebras  in
and  D  of  X  corresponding  to  the  closed
subsets  of  s  um  or  you  can  think  in
terms  of  the  complimentary  opens  as  well
and  you  get  sort  of  some  item  potent
coalgebras  which  would  be  like  the  J
lower  shrieks  of  the  constant  sheath  on
the  open  subset  as  opposed  to  the  like  I
lower  stars  of  the  constant  Chiefs
giving  you  these  item  potent  algebras
but  in  any  case  you  get  a  whole  bunch  of
algebra  objects  in  here  which  kind  of
let  you  localize  this  category  um  over
the  topological  space  s
um  but  um  we're  going  to  want  to  when
discussing  the  relation  with  burkovich
spaces  um  it's  going  to  be  too  annoying
to  require  uh  these  kinds  of  conditions
even  though  they're  satisfied  in
practice  so  I  just  want  to  discuss
something  uh  slightly  more  General  um
some  more  General  notion  of  how  to
localize  an  analytic  stack  along  a
topological  space  and  and  the  relation
with  this  one  in  the  case  where  you  do
have  a  metrizable  finite  dimensional
compact  house  DWF  space
um  so  so  what's  another  what's  another
thing  you  can  do  well  if  you  have  an
analytic  stack  so  we  have  X  um  well  to  X
we  can  assign  a  a
local  I  me  just  call  it  LO
X  which  is  just  uh  so  given  by  the
um  so
monomorphisms
uh  in  analytic
Stacks  so  this  is  kind  of  a  general
thing  you  can  do  whenever  you  have  a
toose  or  or  an  Infinity  topos  it  always
has  an  underlying  local  um  where  you
only  look  at  the
monomorphisms  and  if  you  look  at  the
topos  axioms  then  you  know  the  yeah  you
can  see  the
unions  here  the  POS  unions  and
intersections  exist  yes  but  uh  yeah  so  I
so  I  I  I  I  should  say  that  potentially
there's  some  set  theoretic  difficulty  so
that
this  yeah  potentially  this  is  not  a  set
I  didn't  I  didn't  investigate  seriously
but  it  doesn't  matter  you'll  see  that  it
doesn't  matter  in  a  second  when  I  get  to
the  definition  mean  when  you  allow
larger  and  larger  sizes  you  could  have
more  and  more  sub  objects  in  principle  I
didn't  yeah  I
didn't  I  didn't  I  didn't  spend  much  time
trying  to  prove  that  it's  a  set  because
um  about  the  local  axom  like  of  course
you  have  finite  you  said  that  there  is  a
way  to  you  have  to  produce  finite  binary
intersection  unions  but  also  you  have  to
produce  infin  I  mean  no  but  I  mean  this
is  this  category  has  all  limits  and  Co
limits  this  category  has  all  limits  and
co-  limits  so  I  mean  we  don't  have  to  we
don't  have  to  do  anything  special  here
to  make  this
definition  but  why  the  co  limit  is  a  sub
monomorph  no  Sor  the  co  limit  ah  okay
you  take  the  co  limit  and  then  you  quo
by  the
is  it  any  Z  tox  as  an  image
here  uh  yes  yes  that's  right  yeah  yeah
you  take  the  you  could  that's  a  general
Infinity  topus  thing  for  example  you
could  take  the  check  nerve  of  the  map
and  take  the  co  liit  and  and  this  is
what  Infinity  to  the  analytic  St  yeah
modulo  set  theoretic  technicality  so  it
satisfies  all  the  same  exactness
properties  as  an  Infinity  topos  but  it's
not  presentable  so  it's  okay  so  the
still  could  be
some  okay  up  to  okay  yeah  so  but  we're
going  to  so  so  maybe  it's  not  a  set  I
don't  I  don't  I  don't  care  and  you'll
see  when  I  make  the  definition  that  I'm
interested  in  that  it  doesn't  matter
whether  it's  a  set  or  not  um  sorry  yes
question  the  monom  if  I  only  consider  an
that's  like  taking  connector  component
yes  yes  yes  yes  yes
exactly  yeah  so  yeah  it  just  means  that
if  you  do  the  fiber  product  then  the
diagonal  I  mean  the  diagonal  map
Associated  to  this  inclusion  is  an
isomorphism  um  okay  right  okay  so  and
then  of  course  a  topological  space  also
gives  rise  to  a  local  but  let  me
actually  change  perspectives  and  just
say  s  is  a  local  instead  of  saying  s  is
a  topological  space  um  then  we  can  ask
for  a
map  of
locals  uh  Lo  X  to
S  and  um  whether  or  not  this  is  a  set  uh
this  is  some  data  that  makes  honest
mathematical  sense  because  you're  just
saying  that  for  every  so  a  local  is  is
by  definition  the  collection  of  open
subsets  which  is  a  set  and  you're  just
saying  for  every  open  subset  you  have  to
give  such  a
monomorphism  and  you  know  unions  have  to
go  to  unions  and  finite  intersections
have  to  go  to  finite  intersections  in
the  literature  about  loal  I  don't
remember  I  didn't  don't  use  much  but  so
this  is  like  the  direction  of  M  of  too
but  I  don't  know  when  they  Define  m  of
local  is  it  in  the  other  direction  or  in
this  yeah  yeah  I  don't  know  either  so
I'm  I'm  I'm  doing  the  the  like  geometric
morphism  let's  say  a  geometric  morphism
of  localis  yeah  morph  of  SES  well
morphism  sites  goes  the  other  direction
you  the  morphism  of  SES  goes  the  other
direction  then  I  mean  yes  so  I  mean  I'm
talking  about  a  geometric
morphism  but  also  of  goes  in  the
direction  oh  maybe  oh  I  forgot  yeah  yeah
yeah  okay  yeah  okay  go  in  the  same
direction  okay  not  I
don't  it's  not  used  much
by  okay  well
yeah  uh  anyway  um  okay  um  but  we  can
also  ask  for  a  stronger
property
uh  so  that
each  uh  inclusion  of  open
subsets  so  V  inside  s
um  uh  so  that  this  maps  to  so  it's
supposed  to  give  some  monomorphisms  so
maybe  I  should  give  some  name  for  this
like  f  or  Pi  maybe  so
so  this  goes  to  like  Pi  inverse  U  subset
Pi  inverse  V  subset  X  we  could  ask  that
uh  these  inclusion  maps  are  actually
open
immersions  from  the  perspective  of  the
six  funter
formalism  that  uh  that  Peter  discussed
last  time  the  extended  six  funter
formalism  on  analytic  Stacks  so  we  could
ask  that  each  of  these  inclusions  be
shable  um  and  that  they  be  well
chromologic
smooth  but  in  the  case  of  a  monomorphism
then  it's  quite  easy  to  see  that  the
dualizing  object  is  canonically  just  the
unit  and  it's  the  same  thing  as  an  open
immersion  um
yeah  uh
so  right  so  a  general  monomorphism  I
mean  it  could  look  like  could  look  like
anything  really  it  could  like  look
closed  it  could  look  open  it  could  look
like  some  mix  um  but  um  it's  it's
reasonable  for  to  ask  for  this  stronger
property  that  what  looks  like  an  open
immersion  on  the  level  of  the
topological  space  is  also  is  looks  like
an  open  immersion  from  the  perspective
of  the  six  funter
formalism  um  and  then  uh  in  the
case  where  X  is  a  metri
uh  finite
dimensional  compact  hous  door  space  can
ask  for  as  before  s  s  thank  you  thank
you  can  ask  for  a
map  uh  an  analytics
tax
um  so  um  yeah  sorry  just  a  clarification
So  when  you  say  s  is  a  local  like  what's
the  definition  sorry  oh  the  definition
of  a  local  yeah  so  like  I'm  just  like
what  does  it  mean  for  u  v  to  be  an  open
inside  right  so  right  I'm  using  a  bit  of
loose  terminology  here  but  um  so  about
the  definition  of  a  local  is  it's  a  a
certain  postet  okay  and  the  axioms  that
you  impose  on  this  postet  are  the  axioms
that  are  satisfied  by  the  postet  of  open
subsets  of  a  given  topological
space  but  by  abusive  notation  I'm  kind
of  using  this  symbol  U  subset  s  to  mean
that  U  is  an  element  of  the  postet  which
s  is  secretly  but  it's  because  I'm
thinking  of  it  as  a  topological
equivalent  loal  can  be  sort  of  the
category  of  sub  object  or  the  final
object  in  some  topos  yeah  yeah  and  then
this  defines  a  kind  of  localic
reflection  topos  is  the  about  yeah
there's  another  perspective  is  it's  like
it's  like  a  it's  like  you  take  the
definition  of  a  topos  but  instead  of
sheaves  of  sets  you  look  at  like  sheaves
of  truth  values  I  don't  know  if  this  is
helpful  or  not  but  it's  kind  of  yeah  um
so  it's  it  would  be  so  if  the  usual
topos  is  a  one  topos  then  maybe  a  local
is  a  zero  topos  I  don't  know  or
something
anyway  uh  um  okay  uh  oh  yeah  right  so  um
if  you  have  this  kind  of  structure  then
you  get  this  kind  of  structure  and  if
you  have  this  kind  of  structure  well
kind  of  topologically  you  get  this  kind
of
structure  note  also  that  if  you  fix  X
and  S  then  the  possible  such  this  guy
this  guy  or  this  guy  they're  all  just
sets  um  they're  not  Ana  or  whatever
because  here  well  we're  just  asking  for
a  map  of  locals  it's  just  a  map  of  post
sets  in  the  other  direction  no  but  when
you  say  set  yeah  it  also  means  in  this
sense  of  the  CTIC  size  difficulty  oh
that's  true  that's  true  so  up  to  size
difficult  I  mean  I  mean  there  are  no
automorphisms  of  any  fixed  one  of  these
let  me  say  I  mean  there  no  non-trivial
automorphisms  yeah  yeah  that's  a  that's
a  good  point  okay  so  yeah  so  so  this  is
tological  um  so  yeah  why  so  why  is  this
what  does  this  imply  this
well  basically  you  can  just  you  can  just
check  on  the  level  of  the  analytic  stack
Associated  to  such  a  guy  that  every  open
inclusion  um  actually  is  an  open
inclusion  from  the  perspective  of  the
six  funter  formalism  Peter  more  or  less
discussed  this  last  time  that  our  six
funter  formalism  on  analytic  Stacks  when
we  restricted  to  this  case  recovers  the
usual  six  funter  formalism  on  locally
compact  hous  dorf  spaces  and  then  this
property  of  being  an  open  immersion  is
is  stable  under  base  change  so  that's
um  uh  and  uh  you  get  the  yeah  so  can  you
say  so  you  know  that  open  immersion
holds  for
ah  it  is  stable  under  M  of  analytic
Stacks  or  well  uh  pullback  yeah  it's
stable  under  pullback
yeah
yeah  um  so  yeah  so  this  is  the  strongest
condition  so  if  you  have  this  you  can
ask  whether  this  hold  you  just  check
whether  certain  inclusions  or  open
immersions  if  you  have  this  you  can  ask
whether  this  holds  that's  as  we
discussed  uh  the  last  couple  of  times
that  corresponds  to  some  connectivity
condition  on  the  on  the  itm  potent
algebra  as  you  see  um  okay  um  Let  me
give  a  small
example  the  second  one  is  just  given  off
by  Lo  is  just  one  of  open  opener  yeah
that's  that's  true  yeah  so  yeah  so  right
so  you  could  think  of  this  in  terms  of
there's  Lo  local  X  and  then  there's  some
quotient  local  which  is  like  the  local
of  just  monomorphisms  that  are  opener
verions  with  respect  to  the  six  funter
formalism  and  then  um  the  second  bit  of
data  is  a  map  like  this  um  yeah  thanks
for  that  remark
Peter
uh  and  is  it  the  case  that  the  union
maybe  I  go  confus  Union  of  open
immersions  is  an  open  imersion  or  to
make  it  an  open  yeah  no  no  that  that
that's  true  that  it's  an  open  immersion
yeah  which  is  needed
to  well  it's  needed  to  yeah  it's  needed
to  have  this  loal  I  mean  yeah  so  or  well
remote  I  mean  depends  on  how  your  whe
right  right  right  no  but  it's  true  so
that  if  you  have  a  yeah  a  union  of  open
immersions  then  it's  still  an  open
immersion  and  that's  an  important  point
to  check  when  you're  discussing  these
things  um  and  it  follows  from
this  extension  procedure  for  six  funter
formalisms  that  Peter  discussed
so  if  you  have  a  union  along  open
immersions  then  those  are  shable  maps
and  but  it's  also  a  cover  and  so  you  can
you  can  get  a  a  lower  shriek  map  defined
on  the  union  and  then  you  can  actually
check  locally  that  it's  comically
smooth  um
okay  uh  so  let's  give  an  example
um  so  well  remember  way  back  when  when
we  had  Hub  Pairs  and  stuff  like  that  so
let's
see  um
then  uh  we  assign  to  this  an  analytic
ring  r  r  plus
solid  um  then  can  take  its
Spectrum  um  and  then  we  can  look  at  the
local  Associated  to
this  and  what  we  essentially  already  saw
was  that  this  this  so  This  analytic
stack  localizes  along  the  usual
um  usual  Huber  topological  space  of
continuous  valuations
no  you  you  ah  now  you  consider  the  spec
in  your  in  your  yeah  this  is  this  is
what  this  is  spec  n  so  Peter  yeah  I  I'm
lazy  so
I  yeah  I'm  lazy  so  I  just  say  spec  and
and  then  uh  spies  in  the  old  sense  yes
it's  a  topological  space  yes  and
then  okay  the  local  okay  and  then  you
have  a
map
of  oh
yeah  this  lock  this  means  for  every  open
in  the  add  space  you  give  a  a
monomorphism  of  analytic
stocks  and  uh  this
of  course  this  involves  passing  to  some
derived  pairs  in  some  sense  yeah  so
because  you  are
not  but  your  original  pair  is  is
not  right  so  if  on  the  level  of  rational
opens  this  is  just  going  to  give  another
apine  uh  analytic  stack  which  is  the  one
where  you  enforce  uh  that  F
invertible  just  in  the  world  of  a
analytic  Rings  you  enforce  that  f  is
invertible  and  that  G1  over  F  Etc  GN
over  f  is
solid
solid  um  and  that  defines
some  uh  that  defines  some  analytic  ring
uh  under  this  analytic  ring  here  and
passing  to  spec  it's  actually  a
monomorphism  so  on  the  level  of  uh  the
derived  categories  for  example  it's  a
localization  um  and  that  yeah  that  gives
some  inclusion  of  analytic  Stacks  here
and  um  what  we  argued
um  when  we  were  proving  that  so  we
proved  a  a  less  um  a  less  precise
version  of  this  claim  early  on  when  we
argued  that  the  derived  category  so  this
um  so  this  um  refines  the  statement
that
uh  that  the  drr  plus
solid  uh
localizes  on  Spa
rr+  so  recall  that  we  argue  that  you  you
have  a  sheath  of  infinity  categories  on
this  topological  space  whose  Global
sections  is  equal  to  this  and  whose
sections  on  a  rational  open  is  the
analogous  category  where  you  impose
these  conditions
um  and  we  had  several  discussions  about
when  this  is  the  same  as  when  this  is
also  of  this  form  for  some  non-derived
Huber  pair  and  so  on  so  if  you're  if
you're  a  tate  hu  pair  and  shifi  then
then  in  particular  this
um  this  again  is  just  of  the  same  form
as
before
um  right  and  then  so  it  refines  this
statement  and  the  proof  of  the  proof  uh
of  this  St  statement  that  we  gave  is
actually  shows  this  stronger  claim  so
what  we  did  was  we  argued  that  um  you
can  always  refine  any  open  cover  here  so
that  uh  so  in  fact  you  get  a  an  open
cover  so  in  fact  we
showed  without  having  the  L  without
having  introduced  the  language  for  it  we
showed  that  any  open
cover  or  cover  of  a  rational
open
uh  in  Spa
rr+  uh  pulls
back  to  an  open
cover  uh  after  can  be  refined  to
uh  uh  pulls
back  to  an  open
cover  of  uh  you  know
the
pulls  back
under
Pi  uh  to  an  open
cover  uh  in  the  sense  of  the  six  funter
formalism
no  so  let  me  make  a  a  remark  so  let  me
make
a  open
cover  to
cover  cover  by  monomorphisms  yeah  so  let
me  let  me  make  a  remark  um  so  in
general  uh
this  uh  yeah  it's  not
true  that  a  rational
open  uh  pulls
back  to  an  open
immersion  R  is  there  a  proof  of  this  of
this  oh  yeah  I'll  give  an  I'll  give  an
example  right  now  so  um  so  yeah  I  mean
you  can  do  something  like  I  don't  know
see  yeah
so  I  don't
know  have  two  formal  variables  maybe  p
and  X
um  oh  sorry  I  should
say  just  zp  Q
uh  yeah  that's  yeah  I  don't  need  two
variables  thanks
uh  so  just  one  formal  variable  and  then
invert
it  um  so  what  is  this  correspond  to  uh
so  on  the  level  of  analytic  Rings  we  get
this  and  on  the  level  of  derived
categories
uh  so  this  is  just  solid  zp  modules  and
then  uh  this  is  just  a  um
well  this  is  just  algebraically
inverting  P  so  so  the  left  adjoint  here
is  just  um  inclusion  of  full  subcategory
or  the  right  adjoint  here  is  the
inclusion  of  the  full
subcategory  where  uh  p  is  invertible
so  this  passing  from  zp  to  QP  is  just
algebraically  inverting
P  yeah  p  is  a  prime  yeah  yeah  yeah  um
and  that  actually  uh  from  the
perspective  of  the  six  funter  formalism
is  a  closed  immersion  not  an  open
immersion  that's  a  proper  map  because
the  uh  this  hasn't  the  second  variable
hasn't  changed  so  we're  just  changing
the  underlying  ring  and  not  the  analytic
ring  structure
um  but  it's  also  a  yeah  proper
monomorphism  it's  a  um  it's  a  closed
inclusion  it's
closed  not  immersion  not
open  H  so  so  I  think  we  treated  the  case
of  some  uh  covering  given  by  a  function
like  f  let  F  like  two  Lan  like  the
lauran
cover  yeah  the  one  where  F  or  one  minus
f  is  less  than  equal  to  mean  this  kind
of  basic  cover  and  this  for  those  it  was
they  were  is  it  the  case  for  those  show
that  if  it's  anal  what's  that  I  mean  if
it's  anal  yes  so  then  let  me  make
another  so  however  yeah  uh  if  if  R  is
Tate  um  then  then  indeed  then  this
doesn't
arise  so  in  general  the  problem  is
exactly  these  o  Open  covers  in  hubber
sense  which  are  just  given  by
algebraically  inverting  a  function
without  enforcing  any  any  inequalities
but  in  the  setting  of  a  a  tate  Huber
pair  then  you  know  if  you  want  to  invert
something  it's  you  can  always  write  it
the  the  the  subset  where  you  invert
something  can  always  be  written  as  a
union  of  subsets  obtained  by  forcing
inequalities  so  like  like  the  set
were  some  you  Union  of  like  well  I  I
don't  I  don't  know  yeah  but  it  is  not
quasi  compact  in  general  right  right
right  it's  not  going  to  be  quasi  compact
dealing  with  something  which  is  in  the
sense
of  of  the  adct  space  is  qu  compact
because  it  is  a  rational  domain  I  think
in  the  yes  but  I  mean  it  is  Rush  so  it
is  qu  compact  in  the  yes  in
the  so  this  also  has  to  do  with  the  fact
that  we  didn't  directly
I  was  SL  sliding  one  little  thing  under
the  rug  here  which  is  we  didn't  really
directly  Define  this  as  a  map  we  didn't
really  directly  Define  it  like  this
remember  we  actually  had  this  valuative
spectrum  of  all  of  these  guys  um  and  we
actually  defined  this  map  and  then  we
used  Huber's  retraction  here  to  so  this
was  this  was  not  quite  accurate  um
because  that  was  that  that  would  be
accurate  on  the  level  of  spav  but  that's
not  exactly  how  we  get  it  for  spa  a  so
instead  we
what  just  assume  that
theal  sub  of  SP  ah  yeah  okay  right  right
right  yeah  okay  yeah  theidea  is  open  and
then  it's  okay  yeah  so  then  the  point  is
that  then  um  right
yeah  well  yeah  the  point  is  that  the
rational  opens  in
the  the  rational  opens  in  the  Tate  case
can  always  be  described  by  forcing
inequality
uh  among  the
valuations  um  okay  uh  yeah  let  me  just
try  to  finish  what  I  was  trying  to  say
it's  like  like  I  don't  know  this  is  very
um  I  just  just  to  give  an  idea  of  what's
going  on  um  I  don't  want  to  get  too
precise  about  it  but  it's  this  kind  of
phenomenon  and  where  where  these  guys
correspond  to  um  this  kind  of  thing
which  does  correspond  to  an  open
immersion  so  this  is  open  in  general
this  is  closed  and  in  complete
generality  your  subsets  are  going  to  be
a  mix  of  open  and  closed  but  if  you  only
need  to  refer  to  these  things  and
they're  always  going  to  be
open
so  sheets  on  this  F  is  not  there  in  fact
you  not  algebraically  invert  F  but  but
using  the  right  hand  side  that's  right
that's  right
yep  um
okay  and  also  so
also  uh  there  is  another
topology  on  the  same
space  uh  with  the  same  constructible
subsets  so  where  um  where  uh  open  pulls
back  to
open
so  another  another  fix  even  outside  the
Tate  case  is  to  um  to  choose  a  slightly
different  topology  from  the  one  Huber
described  uh  where  or  like  kind  of  both
of  these  things  are
open  um  that's  that's  not  the
constructible  topology  or  no  it's  not
the  constructible  topology  no  yeah  so  is
it  a  topology  Which  is  less
it's  incomparable  to  the  Huber
topology  so  like  for  this  one  so  like
for  this  subset  here  like  P  not  equal  to
zero  it's  going  to  be  closed  from  one
perspective  and  open  from  the  other
perspective  so  it's  open  from  Huber's
perspective  but  will  be  closed  from  the
perspective  of
this  this  other
topology  is  it  the  spectral  space  it's  a
spectral  space
yeah
to  be  open
now  so  this  is  some  what  he  say  sorry
he's  saying  basically  you  redeclare
zariski  open  subsets  sorry  yeah  you
redeclare  zariski  open  subsets  to  be
closed  which  the  risk  open  they  not
they're  not
open  so  so  for  example  inverting  P
here  inverting
P
okay  is  now  closed  in  the  new
yes  we  haven't  discussed  so  far
that  element  now  thetics  actually  have
compl  that's  true  we  haven't  talked
about  that  yeah  but  let's  leave  that
aside  I  think  there's  already  enough
information  being
um  is  it  the  the  opposite  spectral  space
there  is  a  con  spectral  space  no  it's
not  the  opposite  spectral  space  because
these  ones  are  still  the  ones  where
you're  enforcing  inequalities  like  f
less  than  or  equal  to  one  that's  still
open  that's  open  in  both
cases  it  is  offet  of  of  I  mean  of  of
some  something  a  paper  about  he  def  a
lot  of  topology  and  one  of  them  op  it's
opposite  of  one  of  the  ones  Huber
describes  in  one  of  his  papers  yeah  okay
yeah
yes  can  make  this  more
precise  I'm  going  to  hesitantly  say  yes
it's  obvious  if  you  remember  the
definitions  of  everything  um  maybe  we
need  compact  or  not  because  because
because  I  mean  open  subset  I  mean  I  mean
you  want  you  want  an  item  put  in  Al
correspond  to  it  but  I  I'm  not  sure
whether  if  it's  not  quasy  compact  you
you  have  yeah  let  me  yeah  let  me  say
quasi  compact  uh  yeah
yeah  no  I  mean  I  I  mean  it's  something
we've  actually  discussed  before  in  the
in  the  course  I  drew  some  pictures
trying  to  explain  why  it  makes  sense
that  Z  risky  opens  should  be  closed  um  I
don't  know
yeah  so
um  oh  okay  um  everything  is  defend  all
rational  Subs  are  defend  by  inequality
is  f  i  less  equal  to  g  n  g  not  equal  to
zero  yes  and  this  one  you  replace  by  we
say  that  that  is  a  closed
condition  for  any  G  or  any  yeah  yeah
well  I  mean  you  I  mean  well  let's  say
you  have  a  rational  open  so  the
conditions  are  satisfied  like  the  ideal
generated  by  all  of  them  is
open  um  so  I  mean  so  you  you  can't  just
invert  G  if  that  doesn't  correspond  to  a
a  rational  open  in  Huber  sense  but  in
situations  where  you  can  it's  uh  it
corresponds  to  a  closed  conclusion  I
mean  we  we  actually  discussed  this
topology  in  the  notes  on  um  comp
in  so  I  think  it  was  Gaga  Redux  the
chapter  called  Gaga
Redux  um  we  discussed  this  this
modification  of  Huber's
topology  um  and  yeah  and  apparently  it's
also  in  one  of  Huber's  papers  just
passing  to  the
opposite  uh  okay
uh  so  one  ah  yes
so  one  last
remark
um  about  this  General
setup
uh  so
if  so  if  um  SI  is  some
uh  is  some  inverse
system  of  compact  hous  door
spaces  then  giving  a  map  from  local  X  so
then  giving  compatible
Maps  uh  from  Lal  of  x  to  SI  for  all  I  is
equivalent  to  giving  a  map  um  from  Lal  x
to  the  inverse
limit  so  the  only  claim  I'm  making  here
is  that  like  inverse  limits  in
topological  spaces  are  the  same  as  in
locals  in  this  specific  situation  which
is  something  that  you  can  uh  quite
easily  see  um  and  also  if  if  if  uh  and
it's  the  same  for  Lo
up  so  if  all  of  those  ones  Factor
through  L  up  then  this  one  will  also
Factor  through  l
up
um
look  the  one  we  still  the  one
yeah
um  and  then  um  so  then  there's  a  remark
that  every  compact  house  door
space  is  a
limit  of  metrizable  finite
dimensional
uh  uh  so  potentially  potentially  in  many
different
ways  but  uh  in  many  situations  there's
kind  of  a  natural  way  of  doing  it  we'll
see  this  in  the  setting  of  burkovich
spaces  and  then  this  gives  you  a  way  of
uh  going  from  the  case  of  metrizable
finite  dimensional  compact  house  store
spaces  where  you  can
sometimes  go  from  this  perspective  and
you  can  go  to  this  perspective  and  then
you're  closed  you  know  the  situation
kind  of  passes  to  inverse  limits  in  a
nice  way  and  um  but  now  you  no  longer
have  to  worry  about  things  being
metrizable  or  finite
dimensional  okay  that  was  my  little  um
just
uh  is  a  map  to  s  the  same  thing  as  a  map
from  loal  no  because  you  have  this
connectivity
condition  so  it's
the  you  still  have  to  impose  this
locally  that  locally  all  the  item  poent
algebras  are
connective  so
yeah  I  mean  as  far  I  it's  not  like  I
produced  a  counter  example  but  I  I  I
kind  of  believe  that  it's  not  the
same  uh  okay
okay
so  always  confused  is  there  really  a  t
structure  def  over  D  of  that  object  and
no  no  there's  no  T  structure  so  what  I
didn't  I  don't  have  to  say  what  I  mean
by  connective  I  have  to  say  what  I  mean
by  connective  locally  and  what  I  mean  by
connective  locally  is  that  there  exists
a  cover  by  apine  things  such  that  it's
your  your  object  when  you  pull  back  to
each  of  those  apine  things  is
connective  now  once  you're  in  the  apine
world  connectivity  is  a  pullback  of
something  connective  is  you  have  a  t
structure  and  the  pullback  of  something
connective  is  connective  so  it's  kind  of
and  also  if  the  pullback  is  connective
then  it's  connective  or  not  no  that  that
is  not  necessary  neily  true  okay  okay
okay  so  it's  not  like  so  it's  not
sufficient  condition  for  having
a  so  this  is  like  the  for  schemes  that
you  have  the  the  non  aine
orens  but  still  if  you  use  as  the  flat
or  pqc  still  if  it's  up  upstairs  it's  up
downstairs  so  that's  but  in  your
topology  of  course  you  have  much  more
stuff  yes  yes  so  you  can  have  these
wheel  we  things  where  yeah  you  have  a
map  which  is  apine  locally  but  with  an
with  a  map  with  apine  Target  which  is
apine  locally  but  not
globally  and  this  kind  of  yeah
um  yeah  that's  kind  of  unavoidable  when
you're  doing  analytic  geometry
so  uh  okay  um  where  are  we  okay  so  yeah
so  um  so  burkov
spaces  so  let  me  just  start  with  a
reminder  on  the  definition  so  we  know
where  we're  going  so  so  so  a  this  is  a
bon
ring  so  what  does  that  mean  it  means
that  a  is  a  commutative
ring  um  and  this  Norm  map  from  a  to  the
non-  negative  reals  um
uh
satisfies  so  first  of  all  uh  Norm  of  0
equals  0  second  of  all  uh  the  triangle
inequality  Norm  of  x  +  y  is  less  than  or
equal  to  the  norm  of  X  plus  the  norm  of
Y  and  third  of  all  um  it's  sub
multiplicative
so
um  and  uh  probably  have  to  put  the  norm
of  -  one  is  equal  to  the  norm  of  one
which  is  e  z  yeah  I  should  probably  put
that  yeah  so  uh
either  a  equal  Z  or  Norm  of  1  equal  1
and  Norm  ofus  one  also  do  you  need  that
it  does  not
follow  uh  I'm  not  sure  how  much  it
follows  but  it
it's  usually  they  want  the  triangle
inequality  Al  with  xus  Y  okay  I  think
you  can  you  can  get  it  you  can  give  a
crazy  thing  like  the  negative  multiply
by  two  you  can  do  something  that  still
satisfy  the  a  without  but  of  course  it
would  be  equivalent  something  saf  the  a
this  is  probably  not  how  to  show  oh
okay
uhhuh  yeah
um  yeah  and  then  so  let's  say  that  and
then  a  is  a
complete  uh  with  respect
to  the  nric
uh  um  so  an  example  yeah
there's  so  example  for  say  so  say  s  is  a
compact  house  door
space  uh  you  could  take  for  a  to  be  ous
functions  say  I  don't  know  complex  ah  uh
complex  valued  continuous  functions  and
take  the  norm  to  be  the  soup
Norm  where  you  have  the  usual  absolute
value
um  and  this
maybe  this  is  kind
of  um
uh  right  another  example  would  be  if  uh
if  R  is  a  a  tate  Huber
ring  um  and  if  you  choose  a  pseudo  UT
forer
um  topologically  nilp  poent
unit  uh  then  you  can
Define  uh  Define  a  norm  uh  Peter
actually  wrote  down  the  formula  before
but  uh  basically  it's  like  the  ptic
topology  um
so  yeah  you  can  find
Norm  uh  on
R  uh
with  say  Norm  of  pi  equal  12  and  you
want  and  it  satisfies  also  that  the  norm
of  Pi  *  f  is  always  exactly  the  same  as
the  norm  of  Pi  *  Norm  of
f
um  and  you  need  a  ring  of  definition  for
the
norm  you  have  some  ring  of  definition
where  the  you  declare  the  norm  to  be
less  than  or  equal  to  one
yeah  you  can  take  one  plus
P  1  plus  P  1  plus  P  what's
P
well  uh  I  don't  I  don't
understand  in  zp  you  could  take  P  for
for
yeah  and  also  with  this  there  was  this
again  any  or  no  not  zp  sorry  QP  yeah  QP
take  P  yeah  I  mean  zp  is  not  a  is  not  a
tate  hubber  ring  what  Q  QP  in  QP  bar  you
can  also  take  you  can  also  take  P
actually  so  first  of  you  must  have
complete  because  you  assume  it's
complete  oh  thank  you  yes  yes  I  mean
yeah  thank  you  and  also  if  it  is  zero
then  certainly  Norm  of  Pi  is  not  1  half
so  except  when  it  is  zero  the  norm  of  Pi
is  one  up  and  and  otherwise  you  have  to
to  to  uh  to
say  thank  you  thank
you  Yep  this  have  is  this  nor
multiplicative  it's  say  it's  it's
multiplicative  when  when  one  of  the  guys
is  pi  it's  not  not  multiplicative  in
general
um  okay
uh  so  uh
right
um  so  uh  right  and  then  so  to  this  data
so  so  this  to  a  and  the  norm  Huber
assigned  uh  the  space
um  which  has  a
set  burkovich  thank  you  yeah  we're
switching  over  now  thought  burkovich
um  which  as  a  set  is  given  by  the
um  ah  ah  yeah  okay  I  guess  completeness
includes  that  um  Norm  of  f  equals  z  if
and  only  if  f  equals  z  um  right
so
uh  so  I'm  just  using  this  notation  X  so
kind  of  it's  just  a  decoration  to  uh  so
and  this  is  now  a  multiplicative
seminorm  um
with  which  is  bounded  by  the  given  Norm
you  have  on  the  the  bonac  algebra  a
um  so  um  yeah
so
well  um  yeah  so  for  example
in  and  you  can
take
uh
and  um  if
you  require  that  the  in  this  example  if
you  requir  that  the  norm  restricts  to
the  usual  Norm  on  the  complex  numbers
then  in  fact  um  these  are  all  the
examples  so  that's  maybe  gon's  theorem
so  Bas  so  those  are  basically  the  only
examples  in  this  situation  and
multiplicative  includes  the  of  Zer
element  that  is  the  norm  of  one  is
one  yes  yes  yes
yeah  this  is  otherwise  we  will  get
trivial  thing  out
except  but  this  is
maybe
um  sorry  zero  mul  multiplication  of  zero
numbers  yeah  we  don't  want  that  so  we
want  these  to  be  like  points  so  we  want
them  to  be  non-empty  uh  so  we  want  one
to  be  different  from  zero  I  don't  know  I
mean  yeah  so
yeah  um  okay
but  do  you  allow  a  z  r  no  well  wait  I
allow  okay  I  allow  a  is  the  zero  ring
when  this  will  be  the  empty  set
yeah  um  okay  so  uh  yeah  so  there's  the
so  this  is  actually  a  compact  house  door
space  so  I  didn't  describe  the  topology
but  here  it
is  you  can  view  it  as  a  subset  of  the
product  over  all  F  and  a
uh  um  of  the  interval  from  zero  to  the
norm  of
F  and  it's  actually  a  closed
subset  and  this  map  just  takes  a  a  norm
and  Records  its  value  uh  on
F  uh  and  the  topology  is  just  the
Subspace  topology  so  it's  a  a  compact
house  DWF
space
Okay
so
so  what  we're  going  to  do  is  we're  going
to
um  move  this  definition  we're  going  to
try  try  to  uh  we're  going  to  try  to  make
the  same
definition  to  make  the  same
definition  uh  but  in  the  world  of
analytic  stacks
uh  instead  of  topological
spaces  so  we  want  to  take  the  same  idea
which  is  that  we  want  to  look  at  the  set
of  all  Norms  multiplicative  seminorms  on
a  um  bounded  by  the  given  Norm  on  a  but
we  want  to  say  that  in  the  language  of
analytic  stacks  and  that  we've  discussed
um  so  using  using  the  notion  of
norm
uh  on  an  analytic
ring  uh  we  discussed
earlier  yes  yes  yes  yes  yes  that's
true
um  okay  so  let  me  let  me  remind  you
about  this  notion  of  a  norm  so  so  the
definition  of  so  so  um  well  we  could  say
even  more  generally  in  an  IC
stack  so
uh  um  a  norm  on  X  is  a
map  from  the  algebraic  P1
/x  uh  to  the  closed  interval  from  zero
to  Infinity  including  Infinity
um  uh  which
is  U
multiplicative
uh  so  away  from
Infinity  so  let's  say  on  Norm  inverse  uh
0
Infinity  uh
and  let  me  just  write  it
uh  so  sort  of  suggestively  like  this  you
really  should  write  down  the  commutative
diagram  with  inversion  on  P1  inversion
of  the  coordinate  on  P1  and  inversion  of
the
this  extended  real  positive  non-
negative  real  axis  here
um  um  and  some
condition  on  how  P
sits  p  p  p  uh  yeah  so  this  is  the
P  oh  t  t  is  the  variable  on  P1  yeah  yeah
right
um  um  yes  what  do  you  mean  by  how  P  sits
yeah  so  so  remember  P  was  this
um  uh  this  this  this  ring  you  have  over
any  this  ring  object  you  have  over  any
analytic  ring  which  is  the  free  guy  on  a
topologically  nil  potent  element  so  and
it's  some  version  of  a  unit  dis  um  and
what  we  ask  is  that  it  but  this  Norm
function  gives  you  another  version  of
the  unit  dis  which  is  like  the  inverse
image  of  closed  interval  from  0  to  one
or  you  also  have  the  inverse  image  of
the  open  interval  from  well  half  open
interval  from  0  to  one  and  you  want  to
say  that  P  sits  in  between  the  two  of
those
yeah
uh  okay  uh  where  am  I  yeah  a  nor  okay
and  then  recall  so  so
write  uh  for  the  set  of
norms  on
X  um  so
recall  um  n  is  is  an  analytic
stack  that's  correct  yes  yeah  it's  a
set  because  it's  a  we  I  mean  we  it's
just  given  by  a  map  in  our  category  and
our  category  is  locally
small  a  map  in  the  category  of  analytic
Stacks
yeah  so  these  objects  are  fixed  right
when  you  fix  X  this  is  fixed  and  this  is
fixed  yes  that  you  know  that  there  is
a
uh  why  was  it  uh  well  we  basically  by
definition  every  analytic  stack  was  a
small  co-limit  of  representable  analytic
Stacks  okay  you  have
a
because  there  was  a  notion  of  analytic
ring  where  there  is  the  category  yeah
okay  there  there  you  you  you  handle  the
problem  of  large  sizes  I  mean  because
it's  enough  to  check  the  condition  some
smoke  but  then  you  have  the  analytic
stock  where  you  cover  but  you  don't  know
which  covering  you  need  to  give  aism  so
you  have  need  all  possible  covers  could
be  covers  by  bigger  and  bigger  see  but
you  say  the
category  is  accessible  yeah  so  the  the
the  main  technical  result  you  need  to
prove  is  that  the  the  sheif  ification  of
an  accessible  prief  is  still  an
accessible  prief  that's  that's  the  main
technical  result  you  need  to  prove  we
did  not  discuss  this  at  all  but  that's
what's  what's  underlying  the  the
resolution  to  these
issues  it's  like  in  Waterhouse
uh
yeah
um  okay  so  so  this
um  um  in
fact  so  there's  a
cover  so  this  uh  this  gaseous
base
uh  maps  to
n  um  in  other  words  you  can  write  down  a
norm  on  this  gous  space  Tack  and  it's  uh
and  this  uh  so  this  uh  and  this  is
actually  the  universal
Norm  on  an  analytic
ring  uh  uh  with  a  variable  an  analytic
ring
now  with  a  variable  Q
in  R  such  that  uh  Norm  of  Q  is  precisely
equal  to  1/  12
um  what  did  you  write  NX  is  the  analytic
Stack  n  n  is  an  analytic  stack  yeah  so
yeah  so  the  so  NX  the  set  of  norms  is
actually  just  the  set  of  maps  from  X  to
some  Stack  n
um  right
so  yeah  so  if  you  ask  for  a  norm  on  an
analytic  ring  and  an  element  whose  Norm
is  exactly  equal  to  1/2  which  is  kind  of
somewhat  stringent  condition  because  a
prioria  again  the  norm  map  is  Norm  of  Q
is  a  map  from  Spec  R  uh  to  0  infinity
and  you're  asking  that  it  Factor  through
this  a  prior  its  image  could  be  some
interval  or  something  but  you're  asking
that  its  image  be  exactly  this
Singleton
um  uh  the  universal  example  of  that  is
this  guy  and  moreover  the  map  to  the
base  stack  is  actually  a  cover  in  the
sense  of  our  Gro  de  topology  on  analytic
Stacks  because  every  Norm  on  an  analytic
ring  locally  uh  you  can  find  such  an
element  we  had  this  argument  we  we
discussed  this  two  lectures
ago  um  right  everything  here  was  like  X
was  a  spec  of  an  analytic  rate  um  yeah  I
mean  I  I  made  this  definition  for  a
general  X  but  it  I  mean  it  doesn't
matter  because  the  the  condition  I  mean
the  Norms  on  uh  an  analytic  ring  they
satisfy  descent  basically  I  mean  it's
kind  of  follows  from  General  nonsense
and  so  it  automatically  glues  to  say
what  a  norm  is  on  an  arbitrary  analytic
stack  and  it  just  unwinds  to  the  same
thing  um  so  the  cover  also  exist  because
it  exists  locally  like  you  GRE  it  well
the  map  the  map  no  the  map  exists
because  you  can  write  down  this
Norm  um  and  then  it's  a  cover  because
given  any  Norm  on  an  analytic  ring  uh
after  a  cover  you  can  find  a  q  that
satisfies  this
property  sure  do  you  need  to  do  some
base  tank  I'm  sorry  do  you  need  to  do  a
base  change  falls  back  to  Q  gu  a  base
change  what  do  you
mean
to
is  it  okay  or  you  still  have  a  question
okay
um  so
um  uh  I  want  to  before  before  finishing
the  discussion  of  or  introducing  the
definition  of  this  uh  enhanced  burkovich
Spectrum  as  an  analytic  stack  um  I  want
to
um  I  want  to  explore  a  little  bit  about
uh  so  so  what  does  n  look
like
um  so
uh
so  well  what  do  we  have  on  N  if  you  have
a  norm  on  an  arbitrary  analytic  ring  um
so  if  you  have  a  a
norm  p1r  goes  to  Zer
Infinity  um  so  as  mentioned  for  any
section  here  uh  you  get  a  function  from
Spec  R  to  to  Z  Infinity  but  over  an
arbitrary
analytic  ring  the  only  thing  we  know
exist  are  the
integers  um  so  given  n  in
z  uh  we  get  a
map  which  records  the  value  of  your  Norm
on  uh  on  the  integer  a  little
n
um  so
what  does  this  give  this  gives  a  map
from  n  to  right  so  now  I  should  maybe
pass  to  the  local  perspective  uh  to
product  over  all  integers  of  this
extended  half  real
axis  so  this  um  this  stack  of  norms  it
localizes  on  this  this  product
here  um  and
uh  we  can  ask
for  so  we  want  to  know  what  the  what  the
image  is  so  to  speak
so  and  let  me  say  what  I'm  what  I  mean
by  image  I  mean  the
complement  of  the  uh  largest  open
subset  uh
so  let's  see  with  uh  Pi  inverse  of  U
equals  mty
set
by
definition
yeah  um  and  but  but  we'll  also  see  that
we'll  also  see  that  the  fiber  over  any
point  in  the  image  is  non-empty  so  so
that  at  least  in  this  case  maybe  it's  a
general  fact  I  don't  know  at  least  in
this  case  it's  kind  of  a  theorem  that
the  image  is  closed  so  to  speak  um
okay  uh  right  so  now  note  that  uh  this
so
so  uh  the  more  traditional  thing  is  this
burkovich  spectrum  of  the  integers  so
that  was  also  by  definition  a  subset  of
uh  this
product  uh  going  from  zero  to  infinity
and  it  was  given  by  those  uh
Norms
yeah  um  so
multiplicative  uh  avoiding  Infinity  um
satisfying  triangle
inequality
uh  let  me  remind  you  what  uh  what  this
thing  looks  like  in  in  case  people
haven't  seen  this  before  so
recall  uh  MZ  looks  as
follows
um  you  have  a  point  at  the  center  so  to
speak  um  which  corresponds  to  the
trivial  Norm
um  meaning  uh  zero  Norm  of  Z  is  z  norm
and  Norm  of  everything  else  is  equal  to
one  and  then  you  have  several  branches
so  you  have  um  an  aredian  Branch  so  at
the  end  of  the  archimedian  branch  you
have  the  usual  Norm  uh  the  usual
archimedian
Norm  usual  absolute
value  um  but  then  for  each  prime  P  you
have  uh  and  maybe  uh  yeah  so  for  H  Prime
P  you  have  another  branch  which  also  uh
ends  at  some  point  and  to  get  the
correct  topology  on  embedding  it  into  R2
you  should  probably  make  the  branches
get  shorter  and  shorter  and  shorter  but
okay  that's
um  yeah  but  it's  a  the  situation  with
these  ptic  branches  is  a  little  bit
different  so  what's  going  on  here  here
you  have  the  usual  absolute  value  and
here  at  the  halfway  point  you  have  the
uh  square  root  of  the  usual  absolute
value  and  here  you  know  then  you  know
you  can  put  any  Alpha  between  zero  and
one  and  you  can  kind  of  see  from  the
intuitive  perspective  that  this
interpolates  between  the  usual  absolute
value  and  the  um  the  trivial  absolute
value  here  what  goes  at  the  top  is  not
the  usual  ptic  absolute  value  the  usual
ptic  absolute  value  sits  somewhere  here
so  normalized  to  say  so  that  P  equals  1
over  p  and  now  you  can  actually  scale  it
to  any
uh  any  uh  positive
real
um  um  and  then  there's  also  a  limit  as
the  scaling  goes  to  infinity  and  what
that  gives  is  the  trivial  Norm  or  the
pullback  of  the  trivial  Norm  on  the
residue  field
FP  so  in  other  words  the  norm  of  any
multiple  of  p  is  equal  to  zero  and  the
norm  of  everything  else  is  is
one
um
okay
what  of
course  no  I
yeah  this  is  great  way  I  think  this  was
something  like  this  is  the  first  talk
there  was  a  talk  in  this
course  it  was  in  this  course  or  another
probably  in  this  course  there  was  some
discussion  of  but  maybe  I  could  do  this
another  thing  yeah  so  I'm  I'm  I'm  yeah
I'm  I'm  trying  to  recall  something  which
is  well  known  indeed  um  in  order  to  set
up  the  discussion  of  uh  of  What's
following  here  so  but  but  in  particular
I  want  to  emphasize  this  is  a  really  big
space  but  the  Subspace  uh  MZ  is  quite
small  you  know  onedimensional
um
yeah
um  so  now  what  we're  going  to
see  so  there  there  you  allow  allow  value
to  be  in  but  here  you  don't  that's
correct  so  um  so  here's  going  to  be  the
the
claim  so  the  image  of  Norm  is  a  is  a
larger
subset
uh  so  it  looks  like  this  um  so  you  have
all  the  same  ones  as
before  uh  sorry  I'm  I'm  trying  to  say
that  it  stops  there  yeah  trying  to  draw
like  a  closed  interval
sign  but  then  also  at  the  archimedian
place  um  it  gets  extended  so  here  now
the  usual  archimedian  absolute  value  is
also  in  the  middle  of  of  the
interval  and  you  can  take  arbitrary
powers  of  it  so  for  any  Alpha  in  R
greater  than  zero  you  can  take  powers  of
it
um  so  it'll  go  to  there  in  one  direction
and  to  the  other  direction  you  get  some
really  strange  Point  uh  so  strange
point  where  uh  which  corresponds  to
uh  so  it's  a  subset  of  there  so  it's  a
it's  given  by  some  maps  from  Z  to  the
extended  real  line  there  um  and  it's
given  by  Norm  of  n  equals  infinity  if  uh
if  n  is  not  equal  to  minus  one  0  or
1  z
z  yeah
yeah
so
um  curly  end  here  like  Curly  end  like  in
this  so  it's  like  solid  Z  no  it's  a
different  kind  of
Base  because  I  mean  solid  Z  if  in  over
solid  Z  for  example  the  real  numbers  are
equal  to
zero  so  you
can't
um  I  ask  question  uh  you  can  ask  a
question
yeah  is  the  picture
accurate  in  what  sense  I  mean  I  don't
know  the  price  should  be  closed  or
not  you  mean  this  picture  or  this
picture
this  one
here  I  don't  understand
uh
uh  I  mean  I  I  want  to  say  that  this  is
the  same  as  this  on  all  the  non-
archimedian  branches  and  on  the
archimedian  branch  it  just  gets  extended
past  yeah  so  I  sorry  if  that  wasn't
clear  yeah  it  gets  extended  all  the  way
to  infinity  and  then  compactified  at  the
end  yeah  I  guess  I  guess  what  I  you  can
try  to  write  a  formula  for  it  it's  like
image  Norm  is  like  you  take  the
burkovich  space  of  the  integers  ah  no
let  me  make  a  before  I  say  this  so  in
particular
so  uh  the  triangle  inequality  can
fail  well  that's  quite  clear  here  you
have  Norm  of  one  equal  one  but  Norm  of
two  equals  infinity  okay  that's  pretty
drastic  failure  of  the  triangle
inequality  but  also  like  for  the  square
of  the  usual  absolute  value  which  is  a
new  thing  you  have  then  the  triangle
inequality  fails  as
well  um  so  yeah  I  guess  another  way  of
saying  this  is  like  image  of  Norm  you
can  get  it  from  the  burage  spectrum  of  z
um  on  the  burkovich  spectrum  of  Z  you
have  an  action  of
um  uh  the  real  number  is  bigger  than  or
equal  to  one  wait  did  I  yeah  or
well  yeah
yeah
um  or  maybe  yeah  I  don't  know  but  but  on
the  on  this  other  thing  you  have  an
action  of  the  positive  real  numbers  you
can  kind  of  do  this  and  that  doesn't  do
anything  on  the  non-  archimedian
branches  and  then  on  the  archimedian
branch  it  extends  it  all  the  way  and
then  you  compactify  it  one  point
compactification  so  I  don't  know
um  okay
uh  right  so  let's  explore  this  and  let's
see  what  kind
of  what  what  is  kind  of  going  on  so  uh
let's  so  to  to  explore
this  uh
consider  just  the
map  what's  that  is  the
one  one  point  yeah  the  one  point  yeah
yeah  one  point  Yeah
Yeah
question  picture  you  seem  to  see  this
image  gen  topological  space  but  this  is
supposed  to  be  something  as  a
sub  associated  with  Z  wrong  uh  that's  a
yeah  so  that's  a  good  question  so  by
definition  I  made  it  a  uh  I  was  saying
it's  a  closed  subset  of  uh  of  the
topological
space  um
now  you  can  view  this  as  an  analytic
stack  we  we  saw  how  to  view  this  is  an
analytic  stack  and  you  can  take  the
product  in  the  category  of  analytic
Stacks  that's  perfectly  legitimate  it's
no  longer  metrizable  so  you
know  why  is  it  no  longer  I  mean  no
longer  finite  dimensional  sorry  it  is
metrizable  it's  no  longer  fin  I  said  the
wrong  word  it's  it's  metrizable  it's  no
longer  finite  dimensional  um  so  we
didn't  quite  talk  about  this  thing  but
but  you  can  I  mean  you  can  still  view
this  as  an  analytic  stack  and  you  still
do  get  a  map  of  analytic  Stacks  from  end
to  this  product  and  this  closed  subset
does  correspond  to
a  uh  a  subanalytic  a  monomorphism  of
analytic  Stacks  here  and  the  map  from  n
there  does  Factor  through  that  closed
subset  and  so  you  can  view  it  you  can
view  it  in  several  different  ways  um  as
usual  okay  um  to  explore  this  so  cons
fix  a  prime
p  uh  and  consider  just
uh  uh  as  a  start
um  Norm  of
P
um  so  so  here's  the  first
claim  so  the  claim  is  that  we  can
understand  uh
so
um  so  if  we  look  at  the  um  the  locus  in
this  Universal  space  of  norms  where  the
norm  of  P  lives  between  zero  and  one  we
can  understand  this
um  so  this  is  just  the  same  thing  as  so
maybe  maybe  uh  just  as  notation  I  can
call  this  n  and  then  Z  less  than
absolute  value  of  P  less  than  one  so
it's  the  stack  parametrizing  Norms  on
which  the  variable  P  lives  between  0  and
one  um  this  is  equal  to  uh
spec  QP  the  gashes  version  of  the  pic
numbers  across
this  uh  stack  Associated  to  the  open
interval  from  0  to
one
um
so
yeah  well  in  fact  this  is  quite  easy  to
see  because  um  we  know  that  the
universal  analytic  stack  equipped  with
some  variable  whose  Norm  is  between  zero
and  one  is  uh  so  is  spec  of  zq  hat  plus
or  minus  one  gases  uh  cross  cross
01  um  and  then  you  just  have  to  impose
that  that  variable  becomes  P  so  you  just
set  qals  P  or  you  mod  out  by  Q  minus  P
um  and  then  that  as  Peter  discussed  when
discussing  this  gous  base  stack  gives
you  some  analytic  ring  structure  on  the
pic  numbers  and  then  the  second  variable
doesn't  really
change  uh
so
um
and  so  yeah  and  so  and  then  if  you  look
so  so  I'm  claiming  in  particular  that  if
you  look  at  the  the  the
universal  uh  so  let's  say  we  take  the  so
let's  say  we  take  the  fiber  over  a  point
Lambda  and
01
uh
um  then  the  the  in  particular  you  get  a
normed  analytic  ring  structure
here  um  and  it
is  so  what  does  that  mean  it  means  that
for  every  radius  R  you  have  some  notion
of  overon  convergent  functions  on  a  disc
of  radius  r
uh  and
then  notion
of  on  a  disc  of  radius
R  uh  is  the  usual
one  uh  from  non-  archimedian
geometry  uh
where  uh  you  take  the  normalization  of
the  absolute  value  on  the  peic  numbers
for  which  the  absolute  value  of  p  is
equal  to  this
Lambda
so  um  what  we're  seeing  here  is  the  the
kind  of  the  the  uh  interior  of  the  P
Branch  here  is  kind  of  uh  fairly
straightforward  to
understand
um
so  next  uh  let's  look  at  the  locus  where
another  Locus  that's  fairly  easy  to
understand  um  is  the  locus
where  p  is  between  one  and
infinity  strictly  between  one  and
infinity  because  that's  the  same  thing
as  saying  uh  that  uh  well  first  of  all  P
has  to  be
invertible  um  because  p  is  away  from
zero  the  absolute  value  of  p  is  away
from  zero  um  and  then  it's  the  same
thing  as  saying  that  the  absolute  value
of  1  over  p  is  between  0  and  1
um  and  then  we  again  we  can  use  the
exact  same  argument  to  understand  what
this  is  and  what  you  get  is  you  get  spec
of  uh  the  gaseous  real  numbers  cross
01  and  the  argument  is  the  same  and  that
as  Peter  explained  if  you  take  this  uh
this  gaseous  this  this  ring  here  and  you
mod  out  if  you  set  Q  equal  Al  to  1/  P
then  you  actually  get  the  real  numbers
you  get  a  certain  analytic  ring
structure  on  the  real  numbers  which  is
uh  yeah  this  one  here  um  and  again  if
you  look  at  the  universal  Norm  with  a
fixed  value  of
Lambda
um  uh  fixed  value  of  P  yeah
sorry  sorry  what  for  example  two  P  could
be  two  yeah  yeah
um  the  universal  Norm
here  is  is  the
usual  uh  given  by  usual  overon
convergent
functions  uh  in  archimedian  Geometry
yeah  say  complex
geometry  but  with  respect  to  the
norm  uh  the  which  is  a  power  of  the
usual  um  a  power  of  the  usual  absolute
value  where  Alpha  is  such  that  uh  if  you
take  the  norm  of  1  over  p  uh  uh  you
exactly  get
Lambda
so  so  I'm  not  quite
sure  usual  over  converion  where  does  it
show  in
the
uh  where  where  the  those  uh  over
convergent  so  you  had  some  some  formal
sering  with  Q  yeah  but  and  there  you  can
speak
about  Loy  I  mean  the  the  part  where  the
over  converion  part  I  mean  at  least
there  are  some  algeb  corresponding  to
but  then  you  specify  that  it
is  okay  you  can  walk  over  no  I'm  not  I'm
not  because  the  formal  variable  was
already  replaced  by  P  so  it's  yeah  B
lost  in  this  yeah
usual  yeah  you  have  to  do  a  calculation
so  for  example  yeah  you  could  take  so
the  first  thing  to  understand  of  course
is  why  when  you  take  this  ring  and  you
specialize  to  qal  1  over  P  why  you  get
the  real  numbers  um  so  okay  that  has  to
do  with  some  kind  of  Base  P
expansions  um  and  then  you  have  to
understand  and  say  if  you  take  this
module  P  this  basic  module  P  you  want
you  want  to  know  what  that  base  changes
to  it  should  be  some  module  over  the
real
numbers  so  it  should  be  some  sequence
space  with  some  summability  condition
and  you  can  see  that  the  summability
condition
is  plus  plus  or  minus  Epsilon  maybe  some
little  fuzz  it's  basically  just
um  well  I  mean  Peter  actually  described
what  it  was  it  was  some  kind  of
exponential  decay  thing  thing  but  the
point  is  that  it  that  thing  does  sit
between  the  usual  ring  of  overc
convergent  functions  on  the  unit  dis  uh
and  the  Ring  of  holomorphic  functions  on
the  interior  of  the  unit
dis
um  so  when  you  do  this  over  convergent
business  uh  remember  when  we  did  the
yeah  when  you  do  this  over  convergent
business  it  it  it  it  that  that  subtlety
of  exactly  what  ring  that  is  exactly
what  some  ability  ability  property  you
have  it  doesn't  matter  anymore  and  it
just  Returns  the  usual  ring  of  over
convergent  holomorphic
functions  uh  on  the  dis  and  there's  some
scaling  of  the  usual  absolute  value
which  comes  in  because  when  we  were
building  the  universal  Norm  uh  we  had  to
take  a  fixed  value  of  the  norm  of  Q  and
and  kind
of
yeah  use  that  to  write  down  the  answer
but  it's  such  that  when  you  specialize
to  normal  situations  you  get  the  usual
thing
okay
um  so  in
particular  uh  this  this  Locus
um  this  is  independent  of
P  um
because  in  any  I  just  I  just  told  you
what  the  what  the  what  the  universal  guy
was  and  I  didn't  have  to  make  any  well  I
mean  there's
some  rescaling  property  of  the  of  the
norm  but  um  that  that  kind  of
uh  yeah  it  it  doesn't  affect  this
Subspace  of  the  of  the  of  the  universal
space  of
norms  so  this  is  the  kind  of  thing  that
you  need  to  see  in  order  to  to  see  that
you're  getting  the  Burk  ofage  space  you
have  some  infinite  dimensional  space  but
actually  the  conditions  on  the  various
prime  numbers  are  not  at  all  they're
very  tightly  related  to  each  other  this
is  like  os's  classification  so  if  you
have  a  if  you  have  a  norm  on  the
integers  for  which  the  norm  of  p  is
equal  to  1/2  then  it  you  know  it  kind  of
it  has  to  be  the  pic  norm  and  in
particular  its  values  on  all  the  other
integers  have  to  be  are  determined  by
that  um  and  you  can  see  that  also  in  in
our  situation  as  well  um  that  when  you
force  yourself  to  live  in  this  Locus
which  is  only  a  condition  on  P  that
automatically  tells  you  what  the  norm  of
everything  else  is  because  you  can  just
do  calculations  with  these  usual  uh
rings  of  over  convergent  holomorphic
functions  in  usual  non-  archimedian
geometry  so  on  here  uh  so  the  norm  of
all
other  uh  and  is
determined  each  is  set  of  norms  on  z  uh
yes  uh  not  up  to  equivalence
no  yeah
question  all  the
point  right  yeah
yeah  why  that  would  be  is  there  some
explanation  why  that  would  be  for  me
it's  a  bit  of  a  it's  just  a  calculation
I  mean  um  in  our  in  our  yeah  so  that's  a
good  point  so  in  our  axium  for  normed
analytic  ring  we  had  no  version  of  the
triangle  inequality  whatsoever  we  just
had  this  that  the  norm  should  be
multiplicative  but  then  you  can  look  and
see  well  if  you  believe  what  I'm
claiming  then  kind  of  almost  all  of  it
does  almost  all  of  the  space  does
satisfy  the  triangle  inequality  because
you  can  just  check  on  in  terms  of  the
the  rings  of  functions  that  are  being
assigned  and  you  can  just  verify  that
the  triangle  inequality  holds  so
whenever  you're  in  the  ordinary
burkovich  space  of  Z  you  actually
satisfy  the  the  the  triangle  inequal
your  Norm  actually  satisfies  the
triangle  inequality  and  then  okay  it's  a
you  get  some  sort  of  quasy  Norm  if  you
move  out  in  this  direction  and  this  this
part  is  a  little  funny  but  if  you  throw
that  away  so  there's  always  some  version
of  the  triangle  inequality  that  is
automatically  satisfied  just  as  a
consequence  of  multiplicativity  and
that's  that's  kind  of  funny  yeah  in  the
usual  the  like  in  fact  was  consider
earlier  then  I  mean  it's  I'm  speaking
now  about  Theos  classification  the  or
maybe  so  you  can  put  the  following
condition  I  don't  remember  which  refence
instead  of  inequality  you  can  put  like  X
plusus  Y  Val  less  equal  to  some  constant
time  x  y  and  then  one  can  prove  that
after  a  normalization  by  some  power  you
have  the  triangle  inequality  or  even  the
yeah  and  so  is  it  the  case  that  here  you
can  is  related  somehow  where  yeah  in  the
locus  where  so  if  you  know  that  the
priority  that  the  nor  some  integer  like
or  three  not  Infinity  then  you  can  get
the  triangle  in  quality  by  scaling  it
yes  exactly  exactly  yeah  so  if  um  well  I
have  on  everything  now  I'm  speaking
about  Bigg
like  in  the  big  let  me  make  some  further
claims  so  claim
um  uh  so  yeah  if  um  so  if  you  have  an
normed  analytic
ring  uh  with  the  norm  of  two  say  doesn't
matter  less  than  or  equal  to  one  um
automatically
satisfies  uh  the  non-  archimedian
triangle
inequality
um  a  normed  analytic
ring  uh
with  Norm  of  two  uh  less  than  or  equal
to  two  satisfies  the  usual  triangle
inequality  uh  oh  I  sorry  let  me
move  this  is  the  that  no  of  two  two  is
equivalent  to  no  of  three  three  yes  yes
yes  two  is  an  arbitrary  prime  number
here  two  is  an  arbitrary  prime  number
here  about  to  six  is  it  that's  all
that's  also  equivalent  yeah  yeah  so  I
guess  Prime  is  not  so  important
yeah  two  yeah  two  is  an  arbitrary
integer  bigger  than  one  yeah
um  okay
so
uh  and  then
um  where  am  I  now  oh  yeah  so  a  normed
analytic
ring  uh  with  Norm  of  two  less  than
infinity
always  uh  there  there  always  exists  a
constant  C  greater  than  zero  such  that
the  norm  of  x  +  y  is  less  than  or  equal
to  C  *  the  norm  of
X  the  norm  of
Y  but  this  is  interpreted  not  in  the
sense  of  normal  functions  but  it's  not
yeah  it's  some  Universal  thing  like  uh
so  you  know  you  write  down  you
have  whatever  you
have  yeah  you  have  P1  like  yeah  the
locus  where  the  norm  of  the  T  variable
is  less  than  or  equal  to  a  cross  P1  you
know  Locus  for  the  S  variable  is  less
than  or  equal  to  B  um  yeah  that  this
this  maps
to  so  it's  a  and  b  are  less  than
infinity  here  so
it's  this  maps  to  P1  R  and  then  this
maps  to  Z  Infinity  but  this  should
Factor  through  zero  and  then  C  *  a  plus
b
so  yeah
so
and  this  is
addition
A1  yeah  they  it's  well  defined  because
this  happens  to  live  inside  A1  which  we
already  already  argued  earlier
yeah  yeah  so  there's  always  some  um  some
version  of  the  triangle
inequality  um
yeah
uh  so  how  oh  yeah  so  how  do  you  prove
these  claims  by  the  way  these  claims
imply  the  claim  about  what  the  uh  image
of  uh  of  n  is  because
um  so  yeah  so  and
then
this  uh  this  Extended  burkavage
space  um  the  one  I  mean  the  specific  one
that  I  that  I  wrote  down  uh  the  reason
is  that  you  can  um  you  can  look  at  you
know  the  norm  of
two  uh  which  goes  to  zero  Infinity  ah
yeah  if  you  want  to  if  you  want  to  know
the  the  image  of  something  something  um
you  can  actually  work  uh  you  can
actually  work  on  stratifications  of  your
topologic  you  don't  have  to  work  in
closed  covers  or  open  covers  you  don't
have  to  work  locally  in  the  sense  of
closed  covers  or  open  covers  it's  enough
to  work  locally  in  the  sense  of
stratifications  I  mean  you  can  stratify
the  base  and  look  over  the  different
strata  because  you  you  have  to  see  when
the  pullback  to  an  open  subset  is  empty
and  while  like  a  closed  and  an  open
Don't  form  a  cover  they  still  uh  they
still  detect  ISO  morphisms  in  particular
they  still  detect  whether  an  analytic
stack  is  empty  or  not  so  so  we  can  can
look
over  uh  the  locus  where  two  is  less  than
or  equal  to  one  the  locus  where  one  is
less  than  two  is  less  than  infinity  and
then  the  locus  where  2  is  equal  to
Infinity  um  and  then  we  just  have  to
take  the  union  of  the  the  images  we  see
over
there  and  here  uh  if  you  believe  this
claim  then
uh  where  am  I  uh  yeah  if  you  believe
this  claim  then  there  you  have  the  non-
archimedian  triangle  inequality  so  here
it's  automatic  uh  automatically  a  subset
of  just  by  some  Universal  argument  the
the  non-  archimedian  burkovich  spectrum
of
z  um  so  we're  contained  in  the  in  the
claimed  Locus
there  um  this  thing  we  already
classified  uh  this  is  contained  in  the
archimedian  locus  the  uh  archimedian
Locus  um  and  so  the  last  thing  we  need
to  do  is  to  see  that  uh  this  Locus
consists  of  just  one
point  um  so
need  um  for  all  n  different  from  0  1  and
minus
one  um  and  that  actually  doesn't  follow
from  the  claims  well
yeah  well  I
I'll  what's
that  it  does  follow  from  was
arbitrated  two  was  arated  yeah  that's
right  yeah  yeah  the  claim  that  the  these
conditions  are  independent  of  two  um
does  give  um  does  give  this  as  well  so
um  last  thing  you  say  plus  be  yeah
that's  true  I  mean  you  could  also  yeah
you  can  directly  argue  like  if  three  for
example  wasn't  sent  to  Infinity  then
you'd  be  in  one  of  these  locuses  and
then  two  wouldn't  be  equal  to  Infinity
so  I  mean  it's  kind  of
yeah  um  okay  so  how  do  you  I'm  going  to
finish  up  so  how  do  you  prove  these
kinds  of  claims  well  you  you  just
calculate  so  um  so  to  prove
them
wait  but  for  the  first  part  we  only  know
it's  not  lgers  than  once  and  it's  not
impli  directly  from  yeah  then  you  also
need  to  see  that  uh  if  you  take  any
point  say  here  and  then  take  that  there
exists  some  normed  analytic  ring
which  has  those  values  um  and  we  already
saw  it  for  the  points  in  the  interior  of
the  the  Rays  in  the  Burk  ofage  space  and
it's  also  quite  easy  to
like  hit  the  center  because  you  just  do
some  non-  archimedian  geometry  over  some
lant  Series  ring  with  Q  with  the  trivial
Norm  or  you  can  hit  the  points  at  the
end  for  example  by  doing  some  non-
archimedian  geometry  with  with  FP  so  you
can  you  can  see  that  so  this  gives  a
containment  on  the  image  the  image  is
contained  in  this  this  argument  gives
the  containment  but  you  can  actually  see
the  other  inclusion  by  just  exhibiting
uh  I  just
yeah  um  right  okay
so  uh  where  am  I  oh  yeah  so  to  prove  um
we  um  it's  enough  to  we  we  can
assume  uh  we  have  this  Q  in  addition  we
have  our  arbitrary  Norm  dialytic  ring
satisfying  this  property  but  we  can  add
in  addition  assume  we  have  this  Q  with  a
norm  of  Q  between  0  and
one  um  and  then  we're  working  over
so  uh  so  then  we're  over  this  Gast  to
base
um  and  then  for  the  first  claim  here
this  non-  archimedian
claim  um  what  do  we  need  to  do  so  well
then  the  UN  Universal
case  I  mean  given  that  we  fix  this  data
and  then  then  we  we  we  just  asking  that
Norm  of  two  is  less  than  or  equal  to  one
so  it  just  lives  over  uh  an  item  potent
algebra
so
why  said  pass
the  this  to  it  was  because  of  this  claim
that  you  have  a
cover  okay
yeah  and  you  can  check  things  like  non-
archimedian  triangle  inequality  after
passing  to  a  total  space  of  a  cover  okay
um  and  what  is  this  item  potent
algebra  um  this  is  just  you  take  the  you
know  you  take  the  over  over  this  Q  Plus
orus  One  gas  uh  you  take  the  Ring  of  um
holomorphic  functions  so  to  speak  you
know  this  overon  convergent  version  of
the  unit  dis  um  and  then  you  just  mod
out  by  uh  T  minus
2
and  what  this  gives  um  and  you
calculate  um  and  I  won't  get  into  the
details  um  what  this  gives  is  like  a  on
the  level  of  underlying  Rings  uh  so  you
have
um
um  so  kind  of  yeah  so  you  have  usual  so
you  have  a  power  Ser  or  lant  series  with
integer  coefficients  which  uh
um  over  converge  to  radius  zero  so  which
are  which  are  converge  on  some
unspecified  open  dis  uh  around  the
origin
um  and  yeah  in  this  way  you  see  the  that
this  already  shows  you  that  the
condition  is  independent  of
two  um  because  this  this  description
here  is  independent  of  two  and  then  you
can  just  look  at  the  universal  then  over
this  you  can  also  calculate  these  what
with  the  base  change  of  the  you  know
these  kinds  of  overon  convergent  rings
are  and  you  can  just  check  by  hand  very
brute  force  that  the  non-  archimedian
triangle  inequality  is  satisfied  that
these  it's  just  about  about  a  certain
there  existing  a  certain  map  of  item
potent  algebras  uh  lying  over  the
addition  map  on  on  the  polinomial  ring
um  so  in  particular  this  is  the
formal  the  convergent  formal  Ser
not
well  we  have  we  have  we  have  we  have
negative  powers  of  Q  here  what  does  it
mean
oc0  oh  I'm  sorry  sorry  sorry  sorry  yeah
Q  inverse  yeah  thank  you  thank  you  yeah
miror  morphic  at  the  origin  yeah  thank
you  um  okay  so  uh  that's  that's  how  you
prove  this
claim  um  this  claim  actually  follows
because  uh  the  only  other  Locus  you  need
to  worry  about  is  the  locus  where  you're
between  one  and  two  and  then  you're  in
this  non-  archimedian  branch  and  we  saw
you  get  the  usual  thing  you  have  the
triangle  inequality
there
um  and  um  and  then  um  yeah  so  to  well
actually  yeah  so  then  to  prove  this
claim
um  again  it's  it's  actually  enough  to
it's  I  think  it's  easier  to  prove  well  I
think  the  the  root  is  it's  easiest  to
prove  this  claim  um  and  then  that
implies  that  claim  because  this  means
that  um  again  that  the  only  other
possible  point  to  consider  we're  living
in  the  archimedian  locus  so  we're  just
some  rescaling  of  the  usual  absolute
value  and  then  this  uh  weak  triangle
inequality  this  quasi  Norm  triangle
inequality  is  actually
satisfied  um
so  um  and  to  prove  this  you  again  just
do  a  calculation  it's  sort  of  similar
except  you  um  instead  of  setting  t  equal
to  2  you're  setting  t  equal  to2  um  or
well  or  you're  setting  yeah  oh  there
there's  a  similar  or  t  inverse  equal  to
yeah  what  what  you  end  up  getting  is
some  weird  version  of  functions
convergent  uh  uh  on  the  open  dis  like
this  um  and  again  you  just  observe  that
it's  independent  of
two
um  yeah  so  in  this  Locus  your
convergence  property  is  shrinking  down
to  zero
uh  in  this  Locus  the  the  the  locus
relevant  to  this  claim  um  it's  kind  of
it's  going  in  the  it's  the  laurant  tals
that  are  getting  um  you  know  forcing  the
convergence  out  to  the  boundary  of  the
unit
dis  um  yeah  yes  please  yes  yes  yes  sure
go  ahead  yes  okay  I  don't
know  does  the
a  can  you  uh  uh  explain  a  little  bit
more  by  what  you  mean  by  that  maybe  by
giving  an
example  okay  okay
yeah  um  yeah  okay  so  as  mentioned  uh
this  Friday  there's  no  course  but  please
come  here  and  Peter  will  be  here  in
person  and  so  will  three  other  people
giving  talks  um  some  of  which  are
relevant  to  the  material  here  um  and
then  next  week  I'll  continue  this
discussion  of  Burk  ofit  geometry  next
next  Wednesday  and  then  that  next  Friday
is  actually  going  to  be  the  last  class
so  we're  really  almost  done
here
um  okay  thanks
everyone  over  it's  over
so  yeah  can  I  just  ask  I  mean  what  I
think  I'm  what  what  what  it  means  if  you
have  ontic  stack  which  sits  over  like
any  of  these  classical  regions  of  but
but  what  does  it  mean  if  it  sits  over
infinitely  really  extra  point  I  mean  do
you  have
some  it's
weird  I  don't  know  it's  it's  a  bit
weird  the  norm  is  quite  different  the
norm  there  was  this  Norm  on  a  ring  in
the  sense  but  here  you  have  an  analytic
one  so  a  norm  is  is  something  quite
different  I  mean  it's  as  we  said  that
for  for  like  the  T  ring  it's  somehow  you
choose  one  nor  and  then  you  can  rescate
by  continuous  exponent  exactly  on  the  to
around  continuous  functions  from  the  b
space  to  R  yes  I  mean  this  is  actually
true  uh  in  the  the  set  of  norms  is  it
also  like  this  was  so  it's
not  actually  statement  is  that  related
to  this  SP  that  just  mentioned  that  um  I
mean  on  this  stack  of  norms  mention
of  by  scaning  the  norm  and  you  can  with
both
and  then
any  space  has  a  unique  map  to  the
sky  any  analytic  space  yeah  we  renamed
it  Tate  to  not  conflict  with  uh  yeah
so  an  stack  ofs  Z  by  yeah  exactly  which
is  rescaling  the
norms  and  govern
mentions  to
any  I'm
sorry  no  no  this  is  a  quotient  of  the  I
mean  well  except  except  for  this  AR  it's
a  qu  quo  of  the  burkovich  line  by
rescaling  but  but  there's  more  structure
on  this  because  it's  an  analytic  stack
and  not  just  a  topological  space  so  umst
action  is  free  which  is
not  uh  the  action  of  our  positive  is
because  it's  as  a  set  say  is  a  set  is
free  so  it  yeah  so  the  yeah  so  I  I  so
relatedly  like  when  you  in  some  sense  uh
so  I  I  discussed  it  what  happens  in  the
open  parts  of  these  things  but  actually
if  you  try  to  move  towards  the  the
boundary  points  what  about  the  infinity
point  in  this
question  yeah  it  just  uh  it's  Trigg
well
no  because  I  mean  you  so  from  if  you
only  look  at  the  perspective  of  norms  of
integers  then  it  might  look  like  it's  a
stabilizer  but  um  well
I  yeah
um  so  uh  right  where  oh  yeah  so  yeah  so
I  said  that  so  on  the  interior  it's  kind
of  this  this  this  uh  analytic  stack  and
it  feels  kind  of  like  a  space  like  on
the  interior  you're  just  seeing  that
unit  interval  but  base  change  to  some
gaseous  version  of  say  QP  here  or  the
real  numbers  here  but  as  you  move
towards  the  outer  points  or  towards  the
inner  point  it's  kind  of  it's  it's  more
it  really  is  more  like  a  stack  so  it's
not  that  you  know  the  the  fiber  over
this  point  is  not  uh  not  apine  for
example  neither  is  the  fiber  over  this
point
and  um  that  the  boundary  is
fixed  on  the  component
um
well  from  a  geometric  perspective  no
because  over  this  point  you  can  have
something  like
FP  luron  series  q  and  then  you  have  the
Notions  of  overon  convergent  functions
on  a  dis  there  and  then  when  you  rescale
it's  really  changing  which  dis  you're
labeling  by  a  given  positive  real  number
so  it's  not  universally  true  that
like  yes  yes  yes  exactly
yeah
oh  according  to  the  website  of  I  we  have
two  more  weeks  yes  uh  we  may  have  to
change
that  because  there  was  some  uh
yeah  apparently  on  the  IHS  website  it
says  we  have  two  more  weeks  so  but  I'll
I'll  get  I'll  get  it  fixed  because  the
number  of  talks  was  originally  more  than
the  actual  yeah  yeah  so  we
yeah  I  mean  we  did  decide  that  next  week
is  the  last  week  right  Peter
yes
okay  what's
that  iass
here  yeah  I  see  yeah  yeah  in  in  bond
classes  officially  end  that's  why  we're
stopping
\end{unfinished}
% !TeX root = AnalyticStacks.tex

\section{\ufs Berkovich spaces II (Clausen)}

\url{https://www.youtube.com/watch?v=vXZC3WzKZgo&list=PLx5f8IelFRgGmu6gmL-Kf_Rl_6Mm7juZO}
\renewcommand{\yt}[2]{\href{https://www.youtube.com/watch?v=vXZC3WzKZgo&list=PLx5f8IelFRgGmu6gmL-Kf_Rl_6Mm7juZO&t=#1}{#2}}
\vspace{1em}

\begin{unfinished}{0:00}
e  so  this  uh  this  lecture  is  about
burkovich  spaces  which
I  um  topic  which
I  kind  of  began  discussing  uh  last  time
and
um
so  well  let  me  remind  you  kind  of  the
classical  setup  you  have
a  a  banok
ring  so  it's  a  ring  equipped  with  a  norm
which
is  sub  multiplicative  satisfies  a
triangle
inequality  uh  maybe  I'll  I'll  do  it  do
it  better  this  time  Norm  of  one  equal  to
zero  or
one  unless  R  equals  z  um  and  R  is
complete  and  also
of
okay  yes
uh
yeah
um  okay  uh  and  then  to  this  uh  burkovich
assigns  the  compact  house  door
Space  Mr  Norm  which  is  a  subset  over  the
product  over  all  elements  in  R  of  the
closed  interval  from  uh  Z  to  the  norm  of
F  and  it's  the  set  of
all  uh  the  notation  will
be  uh  that  a  point  in  here  will  be
denoted  X  but  what  it  literally  is  is  a
evaluation  uh  a  map  from  R  to  the  non-
negative  real  numbers  which  is  now
multiplicative
so
um  and  again  satisfies  triangle  in
equality  and  now  one  is  one  yeah  now
it's  it's  kind  of  very  it's  strictly
multiplicative  yeah  so  um  yeah  um  and  we
and  uh  you  know  R  is  not  complete  with
respect  to  this  Norm  X  in  fact  there
could  be  many  elements  of  Norm  zero  but
uh  so
remark  if
you
complete  uh  complete  r  with  respect
to  uh
any  uh  Norm  X
sorry  you  get
a  a  complete  valued
field  KX  hat  I  don't  know  which  is  sort
of  the  residue  field  in  the  sense  of  you
know  since  of  burkovich  theory  and
recall  that  there's  basically  kind  of
three  cases
uh  there's  one  is  a  archimedian
and  that's  the  same  thing  bosski  is
saying  it's  either  r  or  C  so  in  the
archimedian  case  there's  no  variety
really  in  complete  armed  Fields  I  will
see  but  the  norm  could  be  a  power  to
alha  Alpha  One  ex  bigger  than  zero  yes
uh
yes
uh  um  the  second  is  a  it's  non-
archimedian
uh  but
discret  um  and  then  you  have  the  trivial
Norm
so
uh
so  and  then  the  third  case  is  a  it's  an
honest  non-  archimedian
field  non-discrete  I  mean  I  mean  the  the
induced  topology  on  the  field  I  mean
it's  non-  discrete  um  the  Valu  could
come  from  a  discrete
valuation  um  and  then  again  so  then  uh
so  so  you  could  have  some  normal  ized
version  if  you  choose  a  on  discreet
there  is  an  ambiguity  yes  it's  not  yeah
it's  not  about  discret  non-isr  exactly
that  that  was  the  parenthetical  comment
I  was  making  a  non-discrete  topology  it
depends  if  you  call  a  discrete  field  a
non  archimedian  field  I  think  not  but
you  you  wouldn't  you  wouldn't  even  call
this  a  non-  archimedian  field  yeah  okay
so  the  the  norm  is  not  maybe  the
yeah  I  it's  not  the  most  people  call
archimedian  yeah  okay  okay  the  norm  is
non  archimedian
but  but  but  the  but  the  field  is  discret
so  yeah  it's  just  a  discret  just  a
discret  field  um  so  anyway  these  are  the
the  three  kind  of  cases  of  of  possible
residue  field  behaviors  as  you
see
um  uh  right  and  then  so  here  are  the
alpha
alpha
okay  um  there  is  no  normalization  no  I
mean  yeah  you  could  choose  a
normalization  if  you  if  you  fix  a  pseudo
uniformize  you  can  CH  and  a  and  a  real
number  you  can  choose  a
normalization
yeah
yeah  all  right
okay  uh
yes
um  yeah  so  these  are  kinds  of  kind  of
points  and
uh  so  what  we're  going  to  do  now  is
we're  going  to  promote  this  so  the  goal
is
to
promote  uh  Mr  Norm  to  an  analytic
stack  uh  using  the  stack  of
norms  n  so  weall
so  a  map
from  a  map  from  an  analytic  stack  X  to  n
was  the  same  thing  as  a  certain  map  from
P1  of  X  P1  X  to  uh  0  plus
infinity  uh  so  and  uh  satisfying  nor
maxium  uh  the  most  important  one  being
multiplicativity
which  you  have  to  phrase  a  little
carefully  but  in  the  end  it's  just  a
it's  just
multiplicativity
um  okay
so  so  and  we  also  we  kind  of
investigated  the  geometry  of  n  last  time
so  we  saw  that  um  n  Lies  over  this
extended  burkovich  spectrum  of  the
integers
so  um  so  that  was  this  this  space  for  um
so  I  mean  we  here  we  have  a  a  triangle
inequality  you  know  the  usual  triangle
inequality  and  that  that's  what  made
that  the  difference  between  like  the
reason  you  couldn't  take  an  arbitrary
power  Alpha  here  but
I  mean  if  you  take  an  arbitrary  power
Alpha  you  get  something  that  basically
basically  behaves  like  a  norm  but  it's
really  only  a  quasi  Norm  so  you  have  to
put  some  constant  in  front  here
um  plus  the  limit  you  have  also  the  kind
of  the  limit  at  in  yes  there's  also  the
limit  at  Infinity  yeah  so  so  that  that
so  that  makes  uh  yeah  so  and  in  this
stack  of  norms  uh  in  this  stack  of  norms
there's  no  triangle  inequality  imposed
and  it  turns  out  what  you  get  is  this
non-strict  triangle  inequality  where  all
powers  of  the  usual  archimedian  normal
are  allowed  and  then  also  there's  a  some
kind  of  very  strange  limit  point  um
where  your  Norm  takes  infinite  values  on
natural
numbers  uh  and  then  you  have  the  uh  the
archimedian  ones  so  two  Tic  absolute
values  which  end  and  here  in  a  point
which  has  both  characteristic  zero  and
characteristic  P  Behavior  but  where  the
norm  of  two  equals  zero
um  uh  and  then  then  for  the  other  primes
P  um
and  we  also  kind  of  saw  what  the  on  the
interior  of  these  line  segments  you  you
were  getting  the
some  uh  the  gaseous  R  theory  on  the
interior  here  you're  getting  the  gaseous
two  EIC  numbers  and  then  as  you  move
along  the  normed  the  norm  is  changing
but  the  kind  of  so  to  speak  the  analytic
ring  is  not  um  and  Q3  and  so  on  and  then
here  you  have  things  living  you  have  F2
living  for  example  um  you  have  things
living  in  characteristic  two  there
living  in  characteristic  three  here  and
here  you  have  things  uh  living  in
characteristic  zero  so  in  some  sense
uh
um  you  can  imagine  that  the  points  of
this  stack  correspond  to  something  like
these  complete  valued  fields  or  or  the
minimal  choices  of  complete  valued
Fields  like  you  have  real  numbers  piic
numbers  you  have  discrete  FP  you  have
discret  uh  discrete  Q
um  uh  so  that's  kind  of  a  substitute  for
the  notion  of  multiplicative  valuation
but  then  we  still  have  to  input  our  um
our  Bon  ring  R  into  the  construction  in
order  to  get  something  non-trivial  so
here's  the  definition  um  and  I  don't
know  what  good  notation  is  I  I'll  write
I'll  write  it  like  this
so
um  so  this  will  be  an  analytic  stack
which  will  be  a  substack  sub  subset  of
uh  stack  of  of  norms  cross  uh  and  then
some  apine  analytic  stack  which  is  just
we  take  r  with  a  trivial  analytic  ring
topology  uh  I  mean  sorry  the  trivial
analytic  ring  structure  So  What  by  this
what  I  mean  is  that  you  take  a  r  r  is  a
banak  ring  so  it  has  a  topology  but  you
can  also  view  it  as  a  light  condensed
ring  so  the
um  so  using  the  topology  you  consider
those
condensate
yes  um  and  then  trivial  analytic  ring
structure  that
is  all  modules  are  allowed  yes  even  not
okay  it's  the  full  you  know  condens
derived  category  of  this  condensed  ring
so  that  means  that  uh  if  you  want  to
give  a  map  from  X  to  this  spec
R  Tri  that's  exactly  the  same  thing  as
giving  a  map  of  condensed  rings  from  R
to  the  the  value  of  the  structure  sheath
on  on
X  uh  right  yeah  so  uh  right  so  this  U
this  is  going  to  be  a  subset  of  this
product  consisting  of  those  so  such  that
M  well  let's  say  maps  from  X  to  here
these  are  going  to  be  by  definition  in
injection
with  um  so  well  it's  going  to  be  a  par
consist  ing  of  a
norm  uh  and  then  a  map  from  R  to  the
Ring  of  functions  on
X  uh  such
that  and  then  we  impose  a  condition
oh  forgot  to  say  the  the  thing  that  ties
this  this  to  that  so
uh  you  require
the  the  uh  multiplicative  valuation  to
be  bounded  by  the  given  norm  and  that's
exactly  what  we're  going  to  do  here  so
such  that  for  all  F  and
R  um  so  when  you  have  the  norm  here  and
you  have  an  element  in  O  of  x  uh  you  get
a  section  of  A1  and  you  have  a  section
from  an  element  of  ox  you  get  a  section
of  A1  and  in  particular  a  section  of  P1
and  then  you  can  compose  with  the  norm
and  you  get  a  map  Norm  F  from  from
X  X  to  0  plus  infinity  uh  such  that  this
map  lands
in  uh  Z  and  then  Norm  of
f
okay
um  so  um  well  I  guess  the  theorem  first
theorem  would  be  that
um  uh  so  this  is  an  analytic
stack  uh  which
localizes
uh
along  uh  the  burkovich  Spectrum
Mr  so  in  the  sense  that
I  discussed  in  the  last  lecture  so  that
if  you  look  at  this  local  of  all  open
substacks  of  this  spec
Burke  R  Norm  uh  this  Maps  naturally
to
um  this  compact  house  door  space  which
is  the  Burk  of  a  Spectrum  so  uh  you  have
the  local  of  open  in  the  usual  topology
and  the  local  wait  usual
topology  oh  oh  on  the  right  yeah  on  the
right  yeah  but  the  on  the  left  open
substock  means  what  it  means
monomorphism  it  means  a  monomorphism
which  is
shable  and  uh  and  chromologic
smooth  and  this  was  disc  discuss  in  one
of  the  talks
okay  and  you  you  could  even  ask  for  a
map  of  antic  step  right  that  well  when
when  this  is  finite  dimensional  and
metrizable  yes  which  is  basically  all
cases  but  I  mean  I
yeah
yeah  yes
yeah  so  again  um  so  in  in  the  case  when
this  is  metable  and  finite
dimensional  to  verify  that  you  have  a  I
mean  it's  really  just  a  condition  to  say
that  You'  get  a  map  of  analytic  Stacks
to  that  thing  and  it's  just  you  have  to
check  some  connectivity  properties  of  of
the  item  potent  algebras  assigned  to
close  subsets  that  are  kind  of  implicit
in  this  in  this  map  so  it's  a  condition
that  you  can  check  in  practice  and  so  is
it  the  case  that  you
have  since  R  is  cool  you  can  think  of  it
as  a  limit  of  C
sub  Rings  which  are  account  many  element
yes  so  you  can  probably  reduce
to  no  but  it  maybe  it's  not  Dimension
well  I'm  not  sure  but  no  no  but  you  can
look  at  as  a  CO  liit  over  finally
generated  sub  rings  and  then  and  that's
always  embedded  in  a  finite  I  mean  okay
okay
okay  uh  yes  yeah  so  that's  one  way  so
there  is  indeed  a  canonical  way  to  write
this  as  an  inverse  limit  of  finite
dimensional  metrizable  spaces  but  um
let's  not  do  it  let's  just  let's  just
work  with  this  uh  this  setup  here  and  is
it  true  that  you  have  inverse  limits  in
analytic  stat  yes  you  have  you  have  all
limits  you  have  all  limits  in  analytic
Stacks  yeah  ah  okay  so  you  can  syn  of  it
as  a  limit  over  the  stack  is  a  limit
over  the  well  this  isn't  a  stack  now
this  is  just  I'm  just  viewing  this  as
yeah  when  you  have  those  nice  sings  you
can
construct
uh  okay  so  it's  not
uh  no  but  then  you  can  take  for  the  nice
subing  you  can  take  the  stock  Associated
to  m  in  those  limit  you  get  something
which  is  yes  you  get  something  yeah  then
then  you  get  something  the  only  problem
I  have  with  that  it's  not  a  real  problem
but  just  is  that  if  you  started  with
something  which  happened  to  be  already
be  finite  dimensional  and  metrizable
then  you'd  be  non-trivially  writing  it
as  an  inverse  limit  of  other  finite
dimensional  metrizable  things  and  I  mean
you  know  so  let  me  just  not  let  me  just
not  get  into  it
okay  yeah  one  other  question  when  when  R
is  State  then  we  we  also  show  that  we
can  localize  over  the  spa  and  then
there's  a  map  from  the  spa  to  the  I  mean
that's  right  that's  right  yeah  there's  a
commutative  di  there's  a  yes  there's  a
commutative  diagram  that  you  can  write
down  where  here  you  have  map  from  you
know  the  Huber  space  mapping  to  this  and
then  you  have  a  some  this  thing  and  then
you  have  the  the  solid  guy  here  mapping
to  that  and  this  this  last  Arrow  I  I
mean  no  no  yeah  when  I  I  I'll  discuss  in
more  detail  what  this  looks  like  in  the
Tate  case  and  then  um  and  then  you'll
see
yeah  uh  okay  so  in  particular  just  I
want  to  uh  highlight
that  uh  we  get  a  structure
chief  on  on  this  topological  space  and
even  a  structure  chief  of  you  know  of
infinity  categories  so  a  theory  of  Quasi
cerent  shes  on  on  the  usual  burkovich
space  um
so
um  I'll  explain  in  basically  the  Tate
case  how  it's  pretty  easy  to  calculate
this  structure  chath  and  see  what  it's
doing  in  the  case  of
uh  Rings  like  the  integers  well  the
integers  you  can  kind  of  do  it  by  hand
but  it  would  be  interesting  to  compare
so
uh  uh  uh  so  it's  we  I'll  explain
basically  how  you  compare  to  the  cases
burkovich  discussed  but  it  would  be
interesting  to
uh  uh  to
compare  uh  to
puu  who  kind
of  more  or  less  by  hand  described
structure  sheaves  in  certain  cases  uh
over  the
integers  um  so  this  is  a  different
different  approach  where  you  define
something  which  is  a  prioria  structure
sheet  and  then  you  have  to  calculate  it
which  can  be  done  in  principle  but  you
know  takes  a  while  um  in  Poo's  case  he
explicitly  assigns  the  value  and  then  he
has  to  maybe  prove  some  yeah  prove  some
descent  results  so  here  we  automatically
get  some  sort  of  infinity  descent  but
then  you  have  to  calculate  the
value  um
okay  to  like  I  know  this  and  should  be
easy  to  say  what  I  on  a  disc  the
structure  sheet
yeah  it  does  reduce  to  seeing  what  goes
on  on  a  dis  but
um  to  see  what  goes  on  on  a  dis  um  yeah
so  I  I  mean  I  I  I  do  believe  that  it
yeah
so  I  mean  I  I  I'll  explain  what  you  need
to  do  to  do  these  calculations  and  I
think  you'll  see  that  it  is  like  but
probably  there  are  like  in  the  case  of
ho  Rings  there  are  probably  some  derived
phenomena  because  when  you  want  to  look
at  the  things  like  the  the  algebra
functions  on  close  or  open  this  anyway
you  you  quau  something  by  certain  I  mean
you  have  non  closed  idea  so  probably  you
need  to  to  work  in  some  derived  sense  to
get  the  right  to  get  what  you  get  from
your  fan  Theory  you  should  probably  have
the  drinks  which  are  complete  we  do  we
do  we  do  okay  yeah  I  mean  that's  what
yeah  um
yeah  so  I  do  believe  that  in  in  cases
like  uh  what  poo  considers  where  it's
you're  starting  with  a  a  discrete  ring
like  say  the  integers  um  that  that  the
calculations  are  quite  feasible  um  but
if  you  started  with  maybe  a  a  more
arbitrary  Bond  offering  it's  it's  maybe
not  so  obvious  how  to  do  the
calculations  um
okay  so  where  are  we  ah
okay  uh  I  want  to  explain  okay  so  why
this  is  true  so  I  guess  proof
um  so  well  that  it's  an  analytic  stack  I
mean  we  defined  it  as  a  a  funter  a
functor  of  points  so  I  the  only  thing  to
make  it  an  analytic  stack  is  some
accessibility  but  let  me  let  me  explain
how  to  give  charts  for  it  so  let's  let's
give
charts  um  well  so  recall  that  we  had
charts  for  the  Norms  the  space  of  norms
so  we  have  uh  we  had  this  uh  spec  of  Z  q
hat  plus  or  minus  one
um  uh  this  gases
Theory  and  this  we  can  view  as  the
universal  um  sort  of
uh  so  it  it  sits  inside  P1  is  kind  of
the  universal  Locus  where  I  say  absolute
value  of  T  is  equal
to2  um
uh  right
so
uh  so  if  we  pull  back  along  this  uh  if
we  pull  back  along
this  uh
then  um
so
so  so  I  mean  we  look  at  uh  so  we  look  at
this  spec  Burke
R  Norm  Norm  and  then
spec  uh  z  q  plus  or  minus  one  gas  and
then  we  get  this  uh  the  UN  some
Universal  disc  over  here  some  Universal
well  I  don't  know  anulus  over  here  uh
let  me  just  call  it  um  let  me  just  call
it
um  uh
y
um
uh
then  so  for  this  y  thing  um  we've
already  got  the  norm  we've  kind  of
artificially  adjoined  an  element  of  Norm
1/2  um  but  the  only  thing  we  need  to
ensure  to  to  go  from  uh  this  to  this  is
we  need  to  give  the  second  map  we  need
to  give  the  map  to  um  to  spec  of  r  with
a  trivial  thing  and  we  need  to  see  that
they  agree  and  we  need  to  enforce  the
condition  that  um  this  Norm  condition
that  NF  lands  in  inside  that  that  part
there  um  so  we  just  so  to  get
y  you  just  take  uh  spec  of  Z  q  hat  plus
or  minus  one  gas  cross  Spec  R
TR  and
then  pass  to  the
closed
subsets  uh  uh  given  by  the  item  potent
algebra
um  obtained  by
uh
uh  taking  the  norm  F  inverse  of  of  these
closed  subsets
here  so  uh  in  other  words  this  Y  is  our
is
aine
so  this  uh  this  burkovich  spectrum  here
has  a  a  fairly  simple  cover
by  by  an  apine  and  to  calculate  this
apine  the  most  difficult  part  for  like  a
completely  General  R  the  most  difficult
part  is  already  in  the  first  step  that
is  making  this  product  and  uh
calculating  what  analytic  ring  this  is
and  in  particular  calculating  what  the
underlying  condensed  ring  is  because
what  this  tensor  product  involves  is  um
you  have  to  calc  calate  the  the  gaseous
localization  of  of  like  R
yeah
so  you  have  to  then  apply  this  gaseous
localization  again  and  recall  from
Peter's  lecture  where  he  described  the
gaseous  localization  that  it's  given  by
some  sequential  Co  limit  there  are  some
coo  complexes  and  it's  a  little  uh  yeah
then
a  um  but  once  you  have  that  I'm  going  to
I'm  going  to  explain  that  calculating
these  item  potent  algebras  is  not  that
hard  so  once  you  know  the  the  the  ring
you  have  here  then  you're  passing  to
certain  item  potent  algebras  over  it  and
that's  that's  not  the  hard  part  of  the
calculation  but  so  in  particular  if
you're  in  a  situation  where  you  can  do
this  calculation  then  it's  it's  quite
feasible  to  to  calculate  Y  which  is
giving  a  presentation  for  this  stack  and
um  and  we  also  saw  that  like  The  Descent
for  this  cover  um  uh  is  quite  simple  it
it  happens  at  the  the  zero  stage  so  like
the  structure  sheath  here  will  just  be  a
retract  of  the  the  structure  sheath  here
so  the  Ring  of  functions  on  this  y  so
what  did  you  say
The
Descent  oh  yeah  when  we  argued  that  uh
that  this  map  was  a
cover  we  did  it  by  saying  that  it  was  uh
proper  and
descendible  and  we  even  showed  that  like
the  unit  object  the  unit  the  the
structure  sheeve  here  is  just  a  retract
of  the  push  forward  of  the  structure  she
there  um  so  that  makes  The  Descent  uh
quite  straightforward  basically
when  you're  passing  from  here  to  Y
you're  adjoining  some  extra  variable  q
and  it  has  some  analytic  properties  but
you  calculate  some  ring  up  there  and  you
just  look  at  the  zeroth  possible  zeroth
coefficients  of  that  ring  and  that  um
that  tells  you  the  value  of  the
structure  Chief  here
um  look  at  ring  is  a  direct  Factor  one
yes  yeah  exactly  not  as  a  ring  but  I
mean  it's  yeah  it's  a  it  it  maps  and
then  there's  a  linear  retraction  yeah
just  like  if  you  had  laurant  series  like
constant  coefficients  inside  lant  Series
yeah  yeah  mod  yes  exactly
yeah
um  right  okay  so  uh  and  the  yeah  so  the
the  the  map  to  the  burkovich  Spectrum  is
it's  kind  of  uh
well
so
um  so  on  this  spec  Burke  so  the  map
to
uh  so
on  spec
Burke  you  have  the  universal
Norm  um  let's  call  it
n  and  and  then  we  also  have  a  for  every
element  in  R  we  have  by  by
Fiat  a  map  to  the  structure  sheath  here
so  uh  then  we  get  a  map  from  spec  uh
spec  Burke  R  to  product  overall  F  and
R  zero  Norm  of
f
um
but  since  we  enforce  that  every  element
of  R  for  every  element  of  R  the  norm  of
that  element  is  bounded  by  the
prescribed  Norm  on  our  boning  r  that  in
particular  implies  that  the  norm  of  two
is  uh  less  than  or  equal  to  two  which
means  that  by  last
time  the  triangle  inequality
holds  for  n  and  that  means  that  this
this  map  lands  in
uh  inside  the  burkovich  Spectrum
indeed  I'm  sorry  the  image  the  image
mind  yeah  Mr  is  by  definition  a  sub
space  of  the  product
yeah
this  construction  we  could  also  keep
some  of  the  which  does  not  satisfy  the
triangle
inity  right  yeah  I  mean  I  um  yeah  it's
it's  natural  from  the  perspective  of
this  stack  of  norms  that  we've  been
discussing  as  we'  seen  to  relax  the
triangle  inequality  and  it's  and  you  can
do  that  I  mean  you  could  I  mean  the  the
formalism  is  quite  General  there's  I  the
reason  I  I  stuck  to  the  classical  thing
is  just  because  it's  the  classical  thing
you  have  for  the  triang  in
quality  you  assume  that  yeah  you  could
require  that  there  exists  a  constant
such  that  uh  you  know  this  that's  that's
one  thing  you  could  do  okay  but  then  if
you  want
to  uh  ah  so  this  still  defines  a  uniform
structure  so  you  can  say  it's
complete  and  uh  and  it  is  not  equivalent
by  fix  changing  things  slightly  to  a
actual  Norm  if  you  have  this  so  well  I
don't  know  because  for
Fields  just  by
power  H  is  that  true  this  wouldn't  be  I
think  this
is  something  z  uhhuh  may  maybe  it's  true
yeah  yeah  but  you  could  also  conceivably
allow  the  norm  to  take  infinite  values
and  try
to  um  tried  to  build  that
into  into  things  as  well  yeah  I  just  I
wanted  just  wanted  to  stick  with  the
classical  thing
um  so  there  is  a  a  a  theorem
about  the
scaling  no  if  you  have  a  norm  with  the
kind  of  I  don't  know  how  it's  called
with  the  constant  yeah  so  so  so  the
theorem  that  for  Fields  I
think  you  only  get  the  Pu  classification
with  an  alpha  I  think  yeah  yeah  I  think
you're  right  yeah  yeah  yeah  but  for
Rings  you  don't  know  that
because  it's  it's  delicate  because  if  Y
is  a  small  this  doesn't  imply  this
condition  doesn't  imply  that  the  nor  of
X  and  of  X  Plus  Clos  yes  yes  yes  I  I
agree  it  could  be  subtle  for  a  general
ring  I  I  I  don't  want  to  make  any  claims
I  actually  want  to  uh  stick  to  the
classical
setting  um  okay  so  this  was  me
explaining  the  general  case  um  but  there
is  um  and  in  the  general  case  uh  why  oh
sorry  this  this  this  thing  despite  the
notation  with  the  spec  this  um  is  not
going  to  be  apine  so  for
General  um  this
spec
Burke  R  is  not  apine  so  it's  not  the
spec  of  a  of  an  analytic  ring  um  so  for
example  well  if  we  look  at  spec  Bur
uh  Z  and  then  the  usual  archimedian
absolute  value  which  is  the  um  kind  of
the
maximal
the  every  every  Norm  you  could  put  on  Z
will  have  to  be  less  than  or  equal  to
this  one  so  this  is  kind  of  the  the
choice  that  gives  you  the  biggest
possible  burkovich
Spectrum  um  uh  this  is  a  this  is  just
this  uh  Locus  where  yeah  two  absolute
value  of  two  is  less  than  or  equal  to
two  inside  this  stack  of  norms  and  it
really  is  a  stack  as  you  can  kind  of  see
at  the  at  the  points  that  live  at  the
boundaries
um  uh  okay  so
however  uh
suppose  so  let  me  make  a  assumption  star
um  that  uh  so  there  exists  a  let's  say
say  a  pi  in
R  uh  such  that  Norm  of  Pi  is  less  than
one  uh  Pi  is  a
unit  in  the  ring  and  I  want  it  to  be
that  it  strictly  multi  like  the  norm
strictly  multiplies  when  you  multiply  by
the  norm  is  multiplicative  with  respect
to  multiplying  by
pi  um
so
uh  so  this  uh  well  this  condition  is
obviously  not  satisfied  here  but  it's
satisfied  quite  broadly  so  so  example  so
any  non  any  non-discrete
yeah  I  apologize  again  non-discrete  I'll
put  it  way  over  here  so  you  don't  think
I'm  saying  non-  discreetly  valued  field
but
non-isr  valued  field
uh  uh  yeah  admit  such  a
norm  sometimes  they  say  nontrivially
valued  field  okay  yeah  yeah  um  so  you
there  you  have  multiplicativity  for  all
elements  and  then  uh  if  it's  not
discreetly  valued  then  there's  something
with  Norm  between  zero  and  one  and  it
that'll  be  a  unit  and  yeah  of  course
there  is  a  slight  ambiguity  in  valued
field  because  sometimes  it  refers  to
absolute  value  sometimes  to  cruel
valuations  it  could  be  higher  rank  and
then  you  have  to  say  I  mean  it's  not
it's  here  is  valued  in  the  sense  of  of
uh  of  yes  real  value  yeah  exactly
exactly  yeah
um  and  then  so  if  if  um  if
R
satisfies  star  and  uh  Fe  from  R  to  R
Prime  with  which  is  a  map  in  the
appropriate  sense
so
so
um  so  contractive
homomorphism  uh  then  RP  Prime  also
satisfies  uh  so  with  respect  to  F  of
Pi  um
so  uh  well  the  reason  is
uh  yeah  the
uh  yeah  you  take  the  same  Pi  its  image
will  be  a  unit  and  it'll  be  the  norm
will
be  uh  less  than  one  but  also  the  norm
will  have  to  be  the  same  because  uh  yeah
this  condition  is  equivalent  to  uh
unless  unless  maybe  R  is  zero  I  don't
know  um  is  equivalent  to  saying  that  the
norm  of  Pi  inverse  is  actually  the
inverse  of  the  norm  of  Pi
uh  as  you  can  see  by  applying  the
multiplicativity  prop  the  sub
multiplicativity  property  of  the
norm  um  so  then  uh  and  that  condition  is
kind  of  more  obviously  preserved  by  okay
so  actually  in  the  belovich  the  I  think
some  is  natural  to  consider  morphisms
which  are  not  the  norm  is  on  Fe  effect
is  less  say  a  constant  nor  effect  this
defin  still  upap  on  ver  spaces  I  suppose
that  your  definition  does  not  depend
that  is  if  let  say  you  have  two  noes
which  are  equivalent  in  this  sense  by
constants  then  all  you  all  of  this  will
be  the  same  for  the  two  NOS  I  suppose
but  well  is  to  check  something  not  quite
so  I  mean  let  let  me  let  me  say  what  I
can  say  and  then  yeah  sorry  I  just  ah
okay  yeah  but  in  in  under  this
assumption  you  mean  or  in
general
I
think  yeah  yeah  yeah  I  I  agree  with  that
so  that's  why  I  want  to  postpone  the
discussion  yeah  uh  a
bit  okay  so  in  particular  I  mean  uh  the
classical  settings  in  burkovich  Geometry
you  work  over  some  fixed  field  which  is
not  well  often  non-discrete  um  and  uh
and  you're  working  with  Bono  algebras
over  that  field  and  they  will  certainly
uh  satisfy  this  condition  star
but  if  you're  working  over  a  discret
ring  you  know  you  won't  have  this  uh
condition
star  um  so  so  then  I  then  I  claim  uh
so  ah  well  let  me  give  yeah  we're  also
uh  yeah  so  also  any  any  uh  any  Kate
Huber
ring  uh  has  a
norm  defining  the  topology  uh  satisfying
star  with  pi  a  pseudo
uniformer  so  also  some  mixed
characteristic  examples  um
exist
um  uh  right  so
um  where  am  I  oh  yeah  so  then
claim  uh  so  if  star
holds  uh  then
spec
burkovich  uh  R  Norm  is
apine  and  corresponds  to  an  analytic
ring
structure  on  the  condensed  string
r
um  so
moreover  uh  this  analytic  ring
structure  only
depends  on  the  condensed  ring
R  uh  not  on  the
norm  satisfying
star
yeah  so  I  hope  in  some  sense  I'll  be
addressing  your  question  I  mean  no  but  I
see  I  think  that  just  for  your
definition  if  if  you  have  two  equivalent
Norms  then  you  you  by  constants  that
simplest  case  then  you  just  look  at  the
condition  you  impose  apply  to  oh  yes  yes
yeah  I  think  you're  right  I  think  you're
right  I  think  you're  right  yes  yes  yes
so  similarly  for  a  morphism  brings  if  if
it  is  only  with  a  constant  you  could
still  apply  the  same  thing  getm  of  St  I
think  you're  right  ofer  yeah  so  this
because  in  some  I  remember  that  in  some
text  I  don't  know  if  the  book  of  ver  in
some  place  they  they  consider  such
things  which  is  more  yeah  apparently
more  natural  because  I  don't  I  don't
know  that  the  why  but  it's  it's  it's
probably  you  can't  I  think  you're  right
that  because  our  Norms  are  by  definition
multiplicative  I  mean  these  geometric
Norms  that  you  can  uh  you  can  argue
exactly  as  you  suggested  yeah
um  yes  that's  a  good  point
thanks  um
okay  right  um  so  so  so  what  doesn't
depend
on  uh  what  what  depends  on  more  than  R
well  on  this  thing  uh  you  have  this
Universal  Norm  n  um  so  the  the  universal
norm  and  on  spec
b  r  uh  does
depend  on  the  norm  you  choose  on  R  um
but  any  two
choices  are  uh
equivalent  under  uh  so  Norm  passes  to
Norm  to  the  alpha  so  for
some  map  Alpha  from  spec  oh
boy  sorry  let  me  continue  over
here  and  I  could  I  could  remove  the  this
the  second  factor  from  the  notation  here
because  as  I  said  that  the  stack  itself
the  stack  itself  only  depends  on  r  but
okay  I'll  keep  it
um  so  there's  a  what  I'm  referring  to
here  is  there's  a  scaling  action  uh  so
I'm  referring  to  the  fact  that  there's
a
um  there's  a  scaling  action  which  sends
a  norm  uh  and  a  continuous  function
Alpha  to  uh  the  norm  you  take  the  norm
and  you  compose  with  the  alpha  map  on
exponentiation  map  on  on  zero  Infinity
exponentiation  by  Alpha  on  Zer
Infinity  so  what  the  here  you
wrote  uh  spec  no  no  I  I'm  sry  the  what
was  the  the  the  notation  for  the
classical  B
spectrum  is  M  of  r  r  m  yeah  M  okay  so
the  alpha  is  a  map  to  positive  realse
but  does  it  factorize  through  M  of  r  or
is  what  is  it  is
it
no  just
right  um
okay  no  that  doesn't  wouldn't  surprise
me  if  it  did  it's  just  the  argument  I
doesn't  doesn't  quite  show  that  but  U
probably  if  you  look  more  closely  it
will
I  let's  give  the  argument  and  then  uh
and  then  maybe  try  to  address  the
question  I'm  going  to  give  the  proof  of
this  and  then  maybe  it'll  be  easier  for
us  to  answer  the
question  after  having  seen  the
proof  I'm  not  claiming  it  right  now  but
uh  I'll  give  the  proof  for  this  claim
and  then  maybe  at  that  point  we'll  be
able  of  course  you  can  just  change  the
the  norm  on  Itself  by  taking  a  power  and
then  this  will  be  the  alpha  if  you  just
change  the  nor  by  a  power  yeah  if  you
change  the  N  by  a  constant  then  what  I
said  before  it  does  this  doesn't  change
but  if  you  change  it  by  raising  to  a
power  then  you  you  probably  prob  be
constant  yes  but  I'm  not  it  doesn't  no  I
mean  so  the  thing  is  like  uh  there's  two
choices  we  need  to  take  account  for  one
is  like  a  choice  well  let  me  let  me  give
the  let  me  give  the  proof  and  then  and
then  we'll  be  better  equipped  to  try  to
answer
questions  such  as
these
um
okay  okay
so  uh  so  the  proof
um  um
so  I'm  going  to  explain  how  to
produce  uh  this  analytic  ring  structure
on  the  condens  ring
R  um  so  take  take
Pi  uh  as  in
Star
um  then  we  get  a  map
from  uh
zq  to  the  condens  spring  R  which  sends  Q
to  Pi
um  but  uh  Pi  is  of  Norm  less  than  one
which  implies  that  it's  topologically
nil  potent  it's  sequence  of  powers  tend
to  zero  that  implies  that  it  factors
through  this  this  ring  here  and  it's
also  a
unit  um  by  assumption  so  we  get  a  a
factoring  through  this  ring  here  um  and
then  uh
so  uh  we  need  to  check  or  we  want  to
check  um  that  uh  R  is
gaseous  Q
gaseous  as
uh  so  recall  that  this  gases  theory  was
a  non-trivial  analytic  ring
structure  which  was  produced  by  taking
uh  by  realizing  that  that  the  the
category  as  a  full  subcategory  of  over
this  ring  um  and  then  there's  this
completion  procedure  which  changes  the
underlying  ring  to  this  gaseous  thing
but  the  the  category  of  modules  was  just
described  at  this
level
um  so  uh  so  and  and  this  is  something
very  uh  this  is  something  very
straightforward  because  the  definition
of  gases  was  that  some  some  map  from  p  p
to  p  p  being  the  universal  null  sequence
so  to  speak  namely  1
minus  t  *  *  Q  uh  should  be  an
isomorphism  on  on  on  map  uh  on  maps  to
R  but  um  when  you  map  out  to  R  from  this
null  sequence  space  to  your  Bono  space
um  you're  just  getting  the  space  of  null
sequences  in  the  Bono  space  so  that's
equivalent  to  saying  that  if  you  look  at
the  space  of  null  sequences  in
R
uh  and  then  you  have  some  1  -  Q  *  shift
this  should  be  an  isomorphism  of
condensed  uh  of
Bono  Bono  ailan  groups  say  um  but  this
is  uh  but  it's  easy  to  see  what  the
inverse  is  supposed  to  be  and  to  write
it  down  you  need  to  just  you  need  that
that  if  uh  sort
of  FN  is  a  null
sequence  uh  then  then  you  can
sum
uh  uh  FN  pi  to  the  n  and  still  get  an
element  in
R
um  and  uh  the  condition  on  Pi  is  and  the
usual  triangle  inequality  stuff
uh  lets  you  write  this  down  it's  just
the  the  the  limit  of  the  you  Koshi
sequence  um  so  it's  a  it's  quite
straightforward  to  check  that  R
is  R  is  liquid  um  and  then  we  can  just
take  the  induced  analytic  ring  structure
so  what's  that
what  oh  I  said  I  meant  to  I  keep  saying
liquid  instead  of  gases  yeah  it  is
liquid  well  anyway  this  Al  it's  gash
this  uh  take  induced  uh  analytic  ring
structure  uh
from  uh  z  q  plus  or  minus  one  gas  just
um
so  now  recall  that  on  so  we're  call  that
on  on  spec  of  uh  Z  q  hat  plus  or  minus
one  gases  we  have  a  universal
Norm  uh  with  the  norm  of  Pi  strictly
equal  to  uh  this  okay  now  maybe  let  me
say  R  is  not  the  zero  ring  it's  the  zero
ring  I  leave  the  claim  as  an  exercise  so
then  if  it's  not  the  zero  ring  then  this
Pi  will  have  to  have  Norm  bigger  than
zero  and  and  less  and  less  than  one
um  and  then  on  uh  this  we  have  the
universal  Norm  where  the  the  norm  of  Pi
lands  inside  this  Singleton  Subspace
maybe  I'll  write  it  like  that  to  remind
you  that  this  is  kind  of  a  subset  and
not  a
value
um  uh  right  and  then  we  can  then  then  by
by  functoriality  of  norms  but  because
Norms  pull  back  we  get  a  get  a  norm
on  uh  on  what  is  Pi  what  ah  Q  uh  Pi  Pi
is  Q  Pi  is  our  we  we're  fixing  one  of
these  guys  which  exists  by
hypothesis
the  Q  Q  oh  thank  you  thank  you  thank  you
I'm  sorry  yes  yes  thank  you
yeah  uh  right  so  we  get  Norm  on
um  r  with
induced  uh  analytic  ring
structure
um  and  the
claim  so  this  normed  analytic
ring  uh  is  uh  is  is
spec
um  spec  Burke
R  respect  to  this
normes  I'm  sorry  represents  yeah  uh
represents  is
yeah
yeah  it's  the  same  thing  right
um  well  from  geometry  okay  uh  spec  of
this
um  okay
so  what  does  one  need  to  do  to
um  to  prove  this
claim  so  so  what  have  we  done  so  well
maybe  I  need  a  name  for  this  normed
analytic  ring  let  me  call  it  r  liquid
our  gaseous  although  in  the  so  so  far  I
haven't  shown  that  it's  independent  of
the  norm  on  R  but  we'll  see  that  in  just
a  sec
uh  but  but  let  me  let  me  hide  that  in
the
notation
um
right
um  uh  where  am  I  oh  yeah  so  so  so  far  so
what  have  we  so
over  spec  uh  our  gas
we  have  the
map  uh  to  Spec  R
triv  and  we  have  the
norm  uh  n  with  uh  Norm  of  Pi  exactly
equal  to  uh
so
um  the  Singleton  value  pi  and  uh  and
it's  easy  to  see  just  by  tracing  through
the  construction  that  this  is  universal
with  respect  to
that
or  with  respect  to  those  that
structure  um  so  what's  the  what's  the
difference  with  the  thing  we're  trying
to  compare  to  uh  it's  that  in  in  instead
of  having  a  condition  on  just  just  the
norm  of
Pi  uh  which  we  have  here  we  have  a
condition  on  the  norm  of  every  element
so
need  so  we
need  that  uh  uh  this  condition  on  a  norm
is  equivalent  to  the  condition  that  Norm
of  f  uh  is  contained  in  zero
f  uh  for  all  F  and
R  so  we  need  this
equivalence  okay  okay  so  One  Direction
is  quite  easy  so  uh  if  you  have  so  so
this  is
easy  so  if  you  have  this  then  you  apply
it  to  Pi  and  to  Pi  inverse  and  you
deduce  that  uh  so
so  so  the  key  is  to  see  that  just
telling  you  what  the  norm  of  Pi  is  then
I  know  I've  constrained  the  norm  of
every  element  in  our  in  our  Bond
offering  um  so  for
uh  uh  and  also  for  for  the  the  other  oh
well  sorry  that's  ambiguous  right  so
um  yeah  so  for  that  direction  and  and
because  it's  good  to  know  uh  we  will
calculate
uh  so  we  calculate  this  Universal
Norm  so  more
precisely  um
so  so  what  is  a  normed  analytic  ring
structure  recall  that  it  amounts  to
specifying  some  item  potent  algebras
over  P1  um  so  so  we'll
calculate  the
algebras  um
so
uh
so  so  for
all  less  than
c  um  so  what  so  a  norm  in  our  Sense  on
an  analytic  ring  is  implicitly  uh  just
telling  you  what  the  overon  convergent
functions  are  on  a  a  disc  of  arbitrary
radius  centered  around  the  origin
um  um  and  uh  yeah  so  we  just  have  to
specify  this  and  my  claim  is  it's  going
to  be  the  usual  thing  from  burkovich
Theory  so  so
claim  so  this  is  equal  to  filtered  Co
limit  of  let's  say  yeah  radius  bigger
than  C  of  um  you  take  the  so  I  I'll
explain  what  this  is  afterwards  but  kind
of  you  can  make  a  universal  Bono  ring
where  the  norm  is  is  less  than  or  equal
to
R
um  and  uh
so  uh  is  this  or  now  I'm  sorry  is  this
loal  or
now  could  you  make  your  question  more
precise  uh  I  don't  know  if  I  have  fin
time  how  much  time  you
need  well  you  take  the  time  you  need  and
when  it's  precise  ask  again
yeah
okay  so  this  is
like  the  formal  series  that  when  you
replace  each  coefficient  by  the  absolute
value  and  replace  T  by  the  ab  by  R  you
it  converges  the  sum  is  is  finite  yeah
so  this  is  the  set  of  uh  set  of  well
it's  just  the  coefficients  but  let's  say
RN  T  TN  such  that  some  absolute  value  RN
oh  boy  I  shouldn't  use  r  let  me  use  F  I
was  using  f  for  elements  in  my  bonering
so
um  and  then  this  is  the  norm  so  this  is
the
on  um  oh  wait  wait  sorry  sorry  yeah  no
yeah  yeah  that's  right  that's
right  uh
okay  so  um  I  was  also  making  claims  last
lecture  about  calculations  of  the  what
these  overon  convergent  functions  were
in  various  cases  so  I'd  like  to  explain
how  to  make  these  calculations  and  it
turns  out  there's  a  trick  where  you
really  don't  have  to  do  anything  it's
just  kind  of  purely  formal  uh  so
uh  so  there's  a  trick  to
calculating  course  this  is  a  in  the
condensed  now  you  have  to  view  this  as  a
condensed
yeah  this  is  bonak  and  then  it's  uh  yeah
so  it's  condensed  yeah  cond  yeah  and
then  this  is  this  filtered  colum  it  is
taking  place  in  the  condensed  category
yeah  um  oh  yeah  so  well  what  what  is  it
we  need  to  calc  what  is  it  that  we're
calculating  here  actually  so  we're
taking  what  we're  doing  is  we're  well
we're  trying  to  calculate  we  have  the
universal  thing  over  this  gaseous
base
uh  which
we  uh  more  or  less  wrote  down  uh
and  then  we  have  to  tensor  it  with  the
gaseous  tensor  product  so  over  this
analytic
ring
um  uh  with  r  where  here  Q  goes  to
Pi  and  we  have  to  I  mean  actually  you
know  a  priori  it's  a  derived  tensor
product  but  this  is  the  kind  of  thing  we
need  to  do  and  if  we're  being  too  naive
about  it  it  it  can  look  kind  of  tricky
because  um  naively  what  You'  do  is  you'd
write  this  as  as  we've  explained  is  some
filtered  Co  limit  over  copies  of  P  so
that's  kind  of  over
convergence  uh  and  then  you'd  take  uh
you'd  first  calculate  P  tensor  R  and
then  you'd  pass  to  the  filtered  Co  limit
but  actually  it's  not  so  easy  to  unwind
what  P  tensor  R  is  in  particular  it's
not  so  easy  to  see  that  it  would  be
concentrated  in  degree  zero  so  let's  use
a  trick  so  so  let's  not  use  this
approach
that  was  how  we  produced  this  thing  in
the  Universal  case  so  recall  the  idea
was  that  the  this  P  was  some  version  of
functions  on  the  open  unit  disc  and  then
when  you  have  this  q  and  maybe  all  of
its  fractional  Powers  you  could  scale
that  open  unit  disc  and  get  some  version
of  uh  functions  on  an  arbitrary  disc  and
it  wasn't  the  correct  one  but  when  you
make  it  over  convergent  it  doesn't
matter  it'll  be  by  some  kind  of
sandwiching
argument  okay  all  right  so  trick  to
calculate  is  some  general  category
Theory  fact  so  so  LMA
is  so
if  so  so  C  symmetric
monoidal  and's  say  infinity  category
it's  not  too  relevant  um  but  then  uh  so
if  you  have  a
tower  so  X1
X2
X3  uh  in  C  where  each
map  is  Trace
class  so  so  X  to  Y  is  Trace
class  means  it  comes
from  a
map  X  so  from  the  unit  to  x  dual  tensor
y  or  X  dual  I'm  not  assum  assuming  X  is
dualizable  this  is  just  the  internal  H
uh  from  X  to
one
uh  sorry  yes  us  closed  thank
you  m  there's  probably  a  way  to  well
never
mind
um
uh  where  are
we  um  ah  then
then  for  all  Y  in  C  uh  we  can  calculate
the  co  limit  Over  N  of  X  and  dual  uh
tensor
y  so  we  pass  to  the  we  have  a  tower  here
we  pass  to  the  Dual  thing  which  gives  a
a  sequence  and  we  take  the  co  limit  over
that  sequence  um  then  this  uh  is  the
same  thing  as  Co  limit  Over  N  of  the
internal  H  from  xn  to
y
so
uh  this  is
Elementary  um  I'll  leave  it  I'll  leave
it  just  like  that  without  giving  the
proof  you  just  have  a  two  systems  and
you  make  two  in  systems  and  you  make
maps  backwards  that  go  up  a  step  using
the  trace  class  hypothesis  okay  so  again
you  how  do  you  know  there  is  a  the  limit
makes
sense
okay  well  this  this  is  actually  an
equality  of  IND  objects  so  uh  I  mean
it's  an  equality  of  end  objects
does  it  doesn't  it  doesn't  matter  okay
as  a  in  object  and  you  have  to  know  what
is  the  end  for  for
uh  this  is  end  for  for  C  in  in  C  which
makes  sense  is  an  Infinity
okay
um  yeah  so  in  particular  if  C  has  co-
limits  you  you  can  just  you  can  remove
the  quotation  marks  um  if  C  has  Co
limits  in  tensor  product  commutes  with
co-  limits  which  is  the  case  in  our
examples  then  you  can  remove  the  I  mean
you
can  yeah
okay  uh  right  um  okay  so  yeah
so  so  what  we're  going  to  do  so  yeah  in
particular  well  yeah  so
well  so  what  we're  going  to  do  is  we're
going  to  recognize  uh  so  if  you  take  y
equals  the  unit  then  um  you  know  and
yeah  that  this  object  here
um  coim  of  the  internal  H  ask  a
technical  question  yes  about  the
definition  it's  nice  what  do  you  mean  by
it  comes  from  a  map  it's  that  the  is
given  by  this  ah  right  so  if  you're
given  a  map  like  this  then  you  can
tensor  it  with  X  you  get  a  map  from  X  to
X  tensor  x  dual  tensor  Y  and  X  tensor  x
dual  has  an  evaluation  map  to  the  unit
so  you  then  get  a  map  from  X  to  y  yeah
yeah
um  right  um  so  so  we're  going  to  we'll
so  we  uh  will
recognize
uh  c  as
a  CO  limit  over  now  P
du
um  and  apply
this  with  Trace  class  transition
Maps  so  once  we  do  that  then  we  reduce
to
uh  reduce  to  um  uh  some  to  looking  at
null  sequences  in  R  again  and  then  just
some
so  some  filtered  co-limit  of  some  space
of  null  sequences  in  R  and
um  you  can  actually  modify  this  to  the
thing  where  you  require  these  to  form  a
null  sequence  and  they  wouldn't  be  the
same  at  each  term  but  it's  quite  easy  to
be  the  they're  the  same  when  you  take
the  filtered  colum  again  any  two
versions  of  the  unit  dis  are  kind  of  the
same  after  you  make  them  overon
convergent  um  so  uh  yeah  and  then  the
calc  the  calculation  is  very  easy  uh
once  you  once  you  do
this  and  for  this
uh  uh  for  that  we  can  use  S
Duality  on
P1
uh  so  over  well  it  happens  to  be  over
the  liquid  base  but  it  I  mean  the
gaseous  base  but  it  doesn't  much  matter
um  so  St  Duality  on  P1  this  if  you  so
and  the  and  the  six  funter
formalism  so  um  sidity  on  P1  implies
that  the  well  so  we  have  this  proper  map
to  the
base  and  algebraic  serid  Duality  uh
implies  that  the  this  map  is  smooth
smooth  and  proper  um  and  that  the
dualizing  object  is  just  this  Omega  1
shifted  by
one
um  and  then  you  can  calculate  the  then
it's  easy  yeah  it's  easy  to  do
calculations  and  I'll  just  tell  you  what
the  conclusion  is  uh  the  conclusion  is
that  if  you  take  the  Dual  of  the
um
uh  this  identifies  with  uh  the  Ring
of  functions  on  the  open  complement
so
um  so  the
DU
of  in  the  sense  of  uh  uh  a  linear  dual
in  the  category  so  this  is  dual  in
uh  in
there  I'm
sorry  so  this  is  an  object  in  this
category  which  implicitly  there's  an
underlying  condensed  ailan  group  and  so
on  so  yeah  yeah  do  do  you  know  that  you
deal  these  object  which  are  sufficiently
f  so  like  the  double  du  right  so  you
need  to  right  so  we  would  want  to  put
the  Dual  on  the  other  side  but  we  can  do
it  with  a  trick  more  or  less  because
um  so  this  gives  also  so  this  this  over
con  this  by  the  over  convergence  um  you
get  that  the
the  the  single  Duel  of  the  end  object
sorry  sorry  we  can  then  we  can  write
this  uh  this  is  a  pro  object  we  can  we
get  so  yeah  we  get  we  can  get  this  we
can  view  this  over  convergent  thing  as
an  end  object  and  it's  dual  will  be  a
pro  object  and  it  will  be  the  pro  object
given  by  this  thing  where  you  increase
the  radius  as  well  um  but  then  now  we
can  view  that  Pro  object  as  an  inverse
limit  of  the  overon  convergent  things
with  the  the  non-strict  inequalities
over  there  and  then  um  and  then  use  The
Duality  result  in  this  direction  on  each
of  those  and
uh  um  so  there's  some  trick  trick  with
overon  convergence
uh  uh  this  implies  that  if  you  if  you
take  the  the  Dual  of  uh  this
so  if  you  view  this  as  a
proobject  um  because  it's  an  inverse
limit  of  the  things  where  you  um  we  have
a  yeah  greater  than  uh  then  the  Dual  of
that  Pro  object  is  the  end  object  uh
that  we're  interested
in
and  that  gives  a
um  that  gives  an  expression  exactly  like
this  that  this  guy  is  a  CO  limit  of  uh
of  du  of
P's  um  we  still  need  the  trace  class
claim  but  I  claim  that  that  also  holds
for  soft  reasons
um
so  the  trace  class
so  there  there's  a  general  topology  fact
that  if  x  is  a  topological  space  and  z
and  uh  Z  Prime  are  closed
subsets  such  that  there
exists  an  open
U  uh  which  lies  in  between  them  so  two
closed  subsets  which  are  separated  by  an
open  subset  uh  then  uh  the  map  from  uh
the  push  forward  of  the  constant  chath
uh  the  Restriction
map
um  is  Trace
class  in  the  drive  category  of
shes  okay  or  really  I  should  say  maybe  I
should  say  she's  on  X
Val
um  Shi  on  X  and  then  oh  with  values  in  D
of  Z  yeah
um  which  is  not  the  category  of  X  in
general  because  it's
right
um  yeah  uh
so  so  then  uh  on  the  level  of  just  this
closed  interval  from  0  to  plus  infinity
then  any  of  these  transition  mths  will
be  Trace  class  for  this  reason  and  then
you  can  pull  back  that's  a  symmetric
monoidal  funter  you  get  that  Trace  class
you  get  a  trace  class  map  in  the  derived
category  of  P1  but  it  lives  on  A1  and
then  there's  another  trick  to  see  that
it's  image  under  the  forgetful
funter  um  to  the  base  is  also  Trace
class  so  there's  so  so  pull
back  to  A1  use  another
trick  and  the  conclusion  is  that  the
Restriction  map  say  from  uh  any  of  these
overon  convergent  guys
uh  is  Trace
class  um  and  then  also  for  pres  yeah
then  again  by  sand  sandwiching  different
versions  of  the  discs  and  changing  the
Radia  you  also  get  the  and  using  the
trace  class  Maps  as  a  two-sided  ideal  in
all  maps  then  you  get  the  presentation
like  this  which  let  you  calculate  do  it
by  hand  do  what  by
hand  you  want  to  write  over  functions  on
this  disc  I  mean  the  de  clance  is  to  use
pH  piece  yeah  you  can  just  system  and
see  that  transition  has  really
dis  yeah  that  there's  um  yeah  I  I
believe  you  can  do  that  so  certainly  I
remember  doing  that  in  the  complex  case
and  I  assume  it  works  over  the  gashes
base  too  but  I  thought  it  was  nice  to  be
able  to  do  it  without  doing  any
calculations  um
yeah
okay  St  you  still  need  to  know  that  the
way  you  presented  is  the  way  you  scene
you
presented  well  yes  that's  true  that's
true
yes  way  what  was  the  remark  the  the
remark  was  I  I  wasn't  being  very  careful
here  about  writing  what  the  filtered
systems  are  and  all  this  okay  you  have
to  show  that  it  is  there
yeah  um  okay  so  what  that  was  a  that  was
a  a  bit  of  a  uh  um  a  bit  of  digression
so  what  were  we  doing  we  were  um  we  were
we  I  said  we  were  calculating  the  normed
analytic  ring  structure  and  the
conclusion  was  that  the  so  so  the  norm
on  uh
Spec  R  gas  is  given  by
usual  over  convergent
functions  on
diss  over
r  so  that  was  the  the
conclusion
um  so  then
um  then  you  need  to  so  to  see  what  did
what  were  we  trying  to  show  we  were
trying  to  show  that  uh  for  this  Norm
here  this  Universal  Norm  that  we
produced  by  fixing  the  norm  of  Pi  that
automatically  the  norm  of  every  other
element  is  is  correctly  bounded  um  so  to
show  Norm  of  f  uh  contained  in  zero  f
for  all  f  it's  it  suffices  to
show  it  trans  it  algebraically
translates  into
um  oh  no  oh  no  now  we  have  bad
notation
uh  it's  not  clear  a  priority  that  it  is
finite  sorry  you  have  to  know  even  the
finess  is  not  is  a  statement  uh  that's
correct  yeah  yeah  um  so  we  have  to  show
that  if  you  take  this  and  you  mod  out  by
T  minus  F  uh  you  just  get  uh  just  get  R
this  okay  this  this  is  elementary  and
indeed  this  is  Elementary  so  the  map  the
map  giving  this  is  of  course  setting  T
equals  to
F  and  um  it's  quite  Elementary  as  ofer
says  to  see  that  you  get  the  correct
short  exact  sequence  of  uh  of  banak
spaces  so  that  the  kernel  of  this  map  of
T  minus
f  is  of  the  what's
that  mere  existence  of  the  m  is  mere
existence  of  the  map
enough  oh  that's  a  really  good  point
that's  a  really  good  point  Peter  yeah
thanks  yeah  so  what  Peter  was  saying
yeah  so  what  Peter  was  saying  is  that
uh  right  we  know  a  priori  that  this
thing  is  an  IDM  poent
algebra  uh  over  A1  so  over  the
polinomial  ring  on  one  generator  um
therefore  when  you  base  change  it  um
along  here  then  you  get  an  itm  potent
algebra  over  R  bracket  T  mod  T  minus  F
which  is  R  so  this  thing  is  an  item
potent  algebra  over  R  and  so  is  R  itself
and  if  you  want  to  show  that  two  item
poent  algebras  over  R  are  equal  it's
enough  to  Just  Produce  an  algebra  map
between  them  thanks  a  lot  that  that
indeed  makes  it  very  me  very  no  just  uh
we  already  have  the  unit  map  yeah  we  a
map  in  both  directions  indeed  but  we
already  have  the  unit  map  uh  so
so  so  uh  let's  see  uh  because  when  you
do  it  analytically  with  this  uh  instead
of  over  converion  just  converion  on  the
Clos  disc  like  before  and  of  course  you
have  a  a  a  map
when  f  is  less  than  or  equal  to  C  you
get  them  up  but  it  is  but  to  prove  V
division  you  still  get  to  some  conver  is
not  okay  in  the  aredian  case  and  so  it
is  but  it's  not  item  potent  it's  not
item  potent  in  that  case  it's  not  item
potent  in  that  case  if  you  don't  do  the
overc  convergent  one  if  you  just  do  the
the  one  without  the  overon  convergence
it's  not  going  to  be  item  potent  as
argument  would  not  work  exactly  so
somehow  the  proof  of  item  potent  is  you
must  have  already  done  work  similar  to
showing  that  you
know  but  in  the  non  archimedian  case  the
the  thing  with  theid  algebra  does  work
and  in  this  case  aoid  algebras  are
important  or  not  they  are  they  are  yeah
yes  yeah  when  you  do  it  with  the  with
the  the  non  archimedian  I  mean  the
everything  yeah  everything  non-
archimedian  yeah  yeah  then  then  why  is
it  it  ah  because  okay  I  mean  we
basically  we  basically  proved  it  when  we
discussed  the  solid  Theory  but  maybe
okay  if  everything  is  not  so  there  you
can  use  the  yeah  and  solid
then  then  it
is  uh  right  so  okay  but  yeah  so  in  any
case  um  yeah  the  map  exists  because  you
can  evaluate  at  tals  F  and  as  Peter
points  out  that's  enough  thanks  Peter  um
so  uh  right  okay  so  that  was  the  that
was  the  first  part  of  the
claim
um  so  that  was  the  claim  that  this  thing
is  um  this  B  burkovich  Spectrum  this
analytic  stack  which  I'm  calling  the
burkovich  spectrum  is  um  is  apine  when
you  have  that  assumption
star  the  next  claim  was  that  the
analytic  ring  structure  is  independent
of  the  the  choice  of  the  norm  um  so  so
so  proof
continued  uh  need  that  uh  it's  spec  our
gas  is
independent  of  norm  and  Pi
say
um  no  and  that  is  if  you  have  two
topologically
isomorphic  does  depend  on  the  on  the  on
the  condens  the  topological  exactly
exactly  and
uh  without  so  you  could  have  one  nor
with  this  one  Pi  which  is  for  one  Pi
another  Norm  is  pi  Prim  yeah  yeah
exactly  exactly  yeah  so  let  but  let  me
just  give  the  descript  the  independent
description  so  uh  so
claim  so  so  our  gas  uh  is  the  initial
analytic
ring  uh  with  a  map
from  R
triv  to  R
gas  such  that  and  then  we  just  have  a
condition  such  that  for  all  uh
topologically  nil  potent
units  pi  and  R  it's  a  condition  that
only  depends  on  the  topological  ring  R
um  uh  so
the  our  gas  is  uh  is  pi
gases
I.E  that  if  you  the  map  from  z  q  hat
plus  or  minus  one  to  R  uh  Q  goes  to  Pi
um  and  R  lies  in
the  gases
modules  what's
that  oh  I'm  sorry  thank  you  thank  you
thank  you  thank  you  thank  you  yes  uh
yeah  yeah
yeah  thank  thank  you  Peter  yeah  not  just
R
but  is  it  enough  to  check  for  R  the
trivial  module  or  no  no  because  the  I
mean  that  then  you  you  wouldn't  have  to
change  the  analytic  ring  structure  at
all
then  yeah  D  of  r  d  of  R  is  meant  is  d  as
a
condensed  D  of  the  of  the  condensed
uh  D  of  R  means  box  no  it's  the  analytic
ring  it's  the  analytic  ring  the  R  so
yeah  oh  this  should  be  yeah  this  should
be  um  yeah  so  I  want  this  to  yeah  so
this  so  this  map  uh  gives  a  map  of
analytic  Rings  the  condition  is  that
this  map  gives  a  map  of  analytic  Rings
uh  to  R  and  what  that  condition  means
instru  on  R  was  what  was  I  oh  sorry  uh
so  sorry  our  I  mean  I  was  I'm  sorry  let
me  say  R  gas  is  the  initial  analytic
ring  a  with  a  map  from  R  Tri  to  a  such
that  for  all  topologically  nil  potent
units  a  is  pi  gases
IE  uh  this  map
here  uh  yields  a  map  of  analytic  Rings
like  that
so  there  is  an  easier  probably  much
easier  question  in  this  context  which  if
you  have  two
Norms  the  same  then  the  bovich  spaces
are  the  same  is  it  true  or  not  exactly
because  if  you  have  two  Norms  so  since
the  the  condition
is  is  less  or  equal  to
some  some  you  don't  if  you're  not  a
topology  you  don't  control  things  far
away  from  zero  but  because  of  the  unit
you  can  scale  so  but  still  I  wonder
about  at  Le  in  the  another  comedian  case
one  can  compare  to  Uber  and  get  that
it's  the  same  do  fication  it  is  theic  of
some  you  can  but  in  the  aredian  case  of
course  there  is  this  in  the  bovich
Spectrum  there  is  this  condition  that  I
mean  I  think  if  you  take  the  um  you
could  also  like  try  to  make  a
modification  of  the  burkovich  spectrum
only  thinking  of  our  is  a  topological
ring  where  you  ask  for  these  seminorms
to  be  continuous  and  then  I  think  if  you
take  that  space  and  then  you  mod  out  by
this  exponentiation  action  by  positive
real
numbers  um  then  that  will  be  the  same  as
the  burkovich  Spectrum  for  any  fixed
Norm  uh  satisfying  condition
star  no  no  no  because  when  aredian  the
the  non  aredian  things  yes  uh  you  don't
because  in  the  b  space  you  don't
identify  no  to  its  power  to  its  powers
so  I  don't  see  how  you  no  but  yeah  the
identification  is  maybe  not  so  obvious
it's  kind  of  well  I  I  mean  I'm  not  I'm
not  sure  I'm  not  sure  but  so  you  because
you  you  you  want  to  claim  that  your  your
your  belov  spectrum  of  course  it  maps  to
the  space  the  Bel  space  as  we  said  yes
and  but  you  don't  claim  here  that  the
map  is  because  if  you  want  to  CL  the  map
is  the  same  you  have  to  compare  the
verage  spaces  and  and  this  looks  like  a
little  bit  tricky  at  least  in  the  away
from  the  non  archimedian  Cas  we  can
understand
it  anywhere  I'm  not  sure  well  yeah  I
don't  I  don't  know  um  right
um  so
um  yeah  so  let  me  uh  okay  let  me  let  me
give  the  the  proof  of  this  claim  which
is  kind  of  giving  a  intrinsic
description  of  this  gases  analytic  ring
structure  so  um  so  note  that  if  Pi  is
topologically  nil
potent  then
there  exists  a  n  such  that
uh  uh  so  we
um  um  and  after  passing  to  some  power
you  have  small  Norm  in  particular  you
have  Norm  less  than  one  and
um  let  me  note  that  this  condition  here
uh  this  is  invariant
under  uh  replacing  Pi  by  any  power  power
this  is  actually  a  remark  that  that
Peter  made  at  some
point  um  some  point  early
on  uh  so  so  we  can  assume  uh  that
Pi  uh  is  Norm  less  than  one
um  but  then  uh  but  then  uh  for  the
universal  Norm  we  built  over  our  gaseous
uh  sorry  well  sorry  I  need  to  fix  Maybe
okay
so  so  my  claim  is  going  to  be  so  so
certainly  uh  this  R  gas  that  we  built  we
built  it  so  that  it  uh  satisfies  a
weaker  version  of  this  property  where
you  only  demand  it  for  a  fixed  Pi  uh
satisfying  this  condition  here  and  what
we  need  to  show  is  that  so  let's  say  so
we  so  let  me  say  so  our  gas  was
built  to
satisfy  star  just  for  uh  some  fixed  uh
Pi
0  uh  and  now  we  have  to  show  that  it's
satisfied  for  all  choices  of  Pi  um  but
uh  so  yeah
so  um  but  then  uh  so  yeah  so  so  can
assume
uh  0  less  than  Pi  less  than  1  and  then
that  implies  uh  that  the  norm  of  Pi  uh
is  in  this  interval  from  0  to  1  um  which
we  already  showed  implies  that  Pi  is
gaseous  just  from  the  axioms  of  a  a
normed  analytic
ring
okay
uh  what  is  pi  z  r  for  Ral  Z  what  is
it  doesn't  make  sense  because  I've  only
defined  this  when  R  is  a  bonering
satisfying  this  condition  star  well
sorry  the  the  other  the  other  condition
star  about  the  existence  of  a  pi  Norm
between  zero  and  one  Etc  Pi  Z  no  fixed
Pi  Z  is  a  is
a  it's  a  fixed  I  the  way  I  built  the  way
I  built  this  was  I  I  took  my  norm  and  I
took  a
fixed  uh  uh  pi  zero  satisfying  condition
star  and  then  I  built  my  analytic  ring
and  I  built  my  Norm  over  it  and  now  I
want  to  check  that  that  thing  satisfies
this  Universal  Property  which  means  that
so  it  was  universally  built  to  satisfy
that  for  just  for  a  fixed  one  um  but
then  um  and  to  have  the  correct  Norm  on
there  but  and  then  I  want  to  argue  that
it
um  or
well  I  want  to  argue  that  automatically
all  of  the  other  possible  pies  are  also
gases  and  we  can  use  the  norm  to  prove
that
because  um  the  norm  is  such  that  it  you
know  yeah  well  such  that  we  have  this
chain  of
implications
okay
um  so  uh  right  and  then  the  last  part
h
so  the  last  part  is
um  uh  right  that  the  norm  if  you  have  so
that  that  was  the  analytic  ring
structure  being  independent  of  the
choice  of  Norm  on  R  but  then  the  um  so
the  last
part  um  so  this  Norm  on  uh
spec  our  gas  which  this  a  priori  depends
on  um  is
independent  of  the  norm  up  to
exponentiation  by  a
map  uh  spec  our
gas  are  greater  than  zero
um  so
um  yeah
right
so  so  let's  say  we  have  two  different
Norms  uh  we  have  n  given  by  this  here
and  we  have  n  Prime  given  by  this
here
um  um  so  and  both  Norms  have  to  satisfy
conditions  star  but  possibly  for
different  choices  of
Pi  and  so  this  one  uh
satisfies  star  for  say  Pi
Prime
um  then  um  okay  what  can  we  look  at
um
so  again  uh  we  can  assume  that  the  norm
of  Pi  Prime  is  less  than  one  um  so  now
Pi  Prime  had  some  nice  fixed  properties
with  respect  to  this  but  now  with  this
one  it's  just  some  arbitrary  map  to  01
but  that  implies  that  with  respect  to
the  norm  n  uh  that  if  you  take  n  of  Pi
Prime  this  is  a  map  from  Spec  R  gasius
uh  to
01  um  and  then  uh  it  follows  that  but
there  exists  an  alpha  such  that  n  of  Pi
Prime  raised  to  the  alpha  is  equal  to
just  a  constant  uh  value  the  norm  of
Pi  because  uh  exponentiation  acts
transitively  so  or  acts  simply
transitively
so  uniquely  transitively
on
zero
um  so
um  yeah  so  this  is  the  this  is  the  thing
we  have  to
um  yeah  let  me  finish  the  argument  then
we'll  I  think  you're  probably  right  um
so  we'll  address  o  first  point  at  the
end  of  the  the  end  of  the  argument  um  so
then  um  but  then  n  and  n  Prime  are  two
Norms  Norms  on
spec  are
gaseous  uh  uh  both  with
uh
uh  um  wait  sorry  uh  normal  denoted  by
double  double  bar  double  bar  Prime  I
think  n  Prime  should  be  double  bar  Prime
n  Prime  should  be  double
bar  I  don't  know
um  sorry  uh  so  n  to  the  alpha  n  Prime
yeah  are  two  Norms
um  with  uh  sorry  sorry  I  sorry  I  want  to
actually  make  this  equal  to  um  uh  wait
sorry
yeah  right  okay  okay  are  two  two  Norms
with
um  uh  same
value  uh  on  Pi  Prime  so  then  by  the
classification  of
norms  uh  they  must  be
equal  on  P  some  value  so  same  fixed
value  the  the  value  is  a  fixed  real
number  between  zero  and  one  and  we  we
show  that  such  a  norm  is  uniquely
determined  uh  when  we  proved  this
classification  of
norms  okay  uh  and  maybe  I  wanted  to  do
this  or  something  I  don't  know  we  can  do
whatever  we
want  I  I  forget  to
yeah  yeah  I  don't  know  I  think  that's
right  I  think  that's  what  I  want  to
write  okay  um  so  now  to  address  ofer's
question  about  whether  this  map  Alpha  um
is  pulled  back  from  the  burkovich  space
so  first
question  is
Alpha  Spec  R  gas  uh  to  R  greater  than
zero  pulled
back
from
uh  so  this  is  the  the  canonical  map  and
and  this  question  is  whether  there's
some  map  there  um  to  answer  this
question  we  need  to  answer  whether  this
map  is  pulled  back  from  the  burkovich
Spectrum  so  true  if  Norm  of  Pi  Prime  is
but  that's  true  by
construction  because  the  mapping  the
Burk  of  dis  Spectrum  was  exactly
recording  the  Norms  of  all  of  the
elements  in  R  and  in  particular  we're
recording  the  norm  of  Pi  Prime  so  so  the
answer  to  your  question  is
yes  it's  okay  okay  now  I  see  I
understand
the
uh  I
understand  and  I  think  it  should  be  also
true  that  any  two  choices  of  NOS  on  your
ring  are  differ  by  exponentiation  to  and
up  to  con  equivalent  in  the  S  of  to
constant  by  any  two  any  one  is
equivalent  to
some  constant  I  mean  equivalent  in  the
sense  of  bounded  ratio  from  two  side  to
the  power  of  the  other  in  the  in  the
under  conditioned  star  you  mean  under  if
you  have  a  r  topologically  important
unit  first  of  all  if  you  have  a
topological  ring  complete  let  me  say  you
don't  need  complete  this  topologically
important  unit  then  you  can  see  the
topology  is  given  by  a  norm  SA  in  what
you  what  you  want  oh  is  that  true  this
is  something  in  general  topology  where
they  for  a  uniform  space  they  construct
metrics  that  give  it  but  if  you  have  an
ailan  group  it's  enough  if  you  have  got
a  a  system  of  neighborhoods  of  zero
fundamental  systems  that  un  plus  one
plus  un+  one  containing  un  and  let  us
say  symmetric  I'm  not  sure  then  you  can
decide  to  make  a  metric  by  deciding
elements  of  un  of  Norm  it's  most  say  one
two  10  you  check  the  metric  condition
here  you  can  do  something  similar  you
because  if  you  have  topologically
important  unit  yeah  and  if  you  have  a
fundamental  I'm  not  sure  about  the
accountability  okay  the  idea  my  idea  is
to  well  I  need  to  know  that  it's  not
only  topologically  important  but  there
is  a  neighborhood  of  zero  such  that  Q  to
the  pi  to  the  N  time  it  is  Convergence
to  zero  so  some  boundedness  some
conditional  terminology  yeah  okay  then
you  take  U  and  Omega  you  rep  Omega  by
power  so  that  well  okay  I  I  don't  do  you
work  with  two  equivalence  or  no  no  he's
not  asking  a  question  he's  telling  me
something  and  I  want  to  hear  so  I  I  am  I
am  afraid  that  I  don't  know  the
conditions  that  I  need  but  maybe  there
are  some  bound  but  some  natural
condition  in  the  sense  of  topological
ring  that  I  have  to  add  I  have  to  take  a
small  neighborhood  open  neighborhood
such  that  the  power  of  the  that  it  times
small  neighborhood  converges  to  zero  so
pi  to  the  end  it  goes  to  zero  then  like
for  some  end  pi  to  the  end  of  it  plus
end  of  it  is  containing  in  and  then  you
can  construct  the  metric  okay  uh  using  a
replacing  p  and  then  you  you  will
construct  at  least
uh  by  construction  I  think  somehow  I
force  the  P  to  to  have  no  one  enough
roughly  I  can  probably  make  it  Global
okay  one  is  to  work  it  out  the  details
and  then  it  will  turn  out  that  if  you
have  a
continuous  map  to  a  valed  field  which  is
continuous  and  the  norm  of  my  Pi  yes  is
less  than  or  equal
to  is  one  half  anyway  then  it  will  be
also  less  than  or  equal  to  this  no
probably  well  it  will  be  equal  to  1  half
yeah  okay  then  it  will  be  less  than  or
equal  to  this  yeah  and
then  okay  this  is  one  idea  but  then  if
you  have  got  a  but  by  the  way  for  with
this  construction  you'd  get  only  the
triangle  inequality  maybe  up  to  some
constant  again  no  no  no  no  no  if  I  have
what  I  claim  is  this  this  is  something
that  I  check  so  if  you  have  a
non-commutative  forgot  now  if  you  have
got  a  in  general  for  uniform  spes  they
they  have  three  three  uh  you  need  to
they  work  with  three  but  if  you  have  an
Community  grou  you  can  do  it  with  two  so
I  just  claim  first  that  if  you  have  if
you  have  a  topologic  cian  group  with
topology  is  defined  by  sequence  of  a
symmetric  neighborhoods  of  the  origin  un
un  plus  one  plus  un  plus  contain  un  then
you  you  just  get  a  metric  by  this  by
imposing  that  the  guys  in  un  know  at
most  one  over  two  to  the  end  so  you  have
a  metric  in  a  generalized  sense  it  could
be  plus  infinity  for  something  then  I
will  change  it  using  the  unit
my  my  uh  I  will  correct  it  using  my  sud
uniformer  but  at  least  I  will
get  uh  okay  maybe  what  I'm  saying  is  a
bit  uh  maybe  I'm  thinking  too  fast  with
some  mistakes  but  in  any  case  I  think  it
will  also  come  out  that  any  two
Norms
are  uh  in  some  sense
uh  yeah  but  this  I  am  not  sure  any
to  okay  maybe  we  can  discuss  later  let
me  let  me  just  finish  I  have  only  one
more  thing  to  say  and  it's  quite  short
so  yeah
um  so  uh  right  um  the  last  thing  is  when
I  talk  a  bit  about  just  say  Global
globalization  only  to  say  that  it's
trivial
um  so  we  could  make  a  definition  I  don't
know  I  mean  a  definition  of  a
burkovich  analytic  space  I  don't
know  is  a  pair
um  X  or
triple  x  s  uh  PI  from  uh  local  of  opens
in  X  to  S  where  X  is  an  analytic
stack  uh  s  is  a  locally  compact  house
door
space  um  and  Pi  is  a  map  of
locals  uh
such
that
um  such  that  uh  locally  on
S  for  the
topology  the  open  section  topology  or
the  the  local  section
topology
um  uh  it  is
isomorphic
to  spec
Burke  are  uh  uh  sorry  R
Norm  Mr  Norm  uh  and  then  this  canonical
map  Pi  for
some
Bing
R
um  okay  so  this
is  completely  trivial  now  uh  to
globalize  and  yeah  so  the  only  point  to
note  is  that  uh  working  locally  on  S
you're  also  automatically  working
locally  on  X  and  that's  because  these  by
by  definition  of  a  map  of  local  is  a  a
cover  and  the  open  cover  topology  gives
a  cover  in  the  sense  of  open  covers  of
analytics  stxs  here  and  those  are  covers
in  our  Gro  and  deque  topology  that  we
use  to  Define  analytic  Stacks  they're
even  open  covers  in  the  sense  of  the  six
functor  formalism  for
the  what  do  mean  local  section  to  I  mean
the  gro  topology  on  locally  compact
house  door  spaces  which  is  generated  by
open  covers  okay  yeah  it  is  isomorphic
to  isomorphic
to  this  basic
object  space  so  you  locally  yeah  so
recall  that  these  these  aine  ones  are
always  compact  house  DWF  so  you're  not
going  to  kind  of  if  you  if  you  say
locally  in  too  naive  a  sense  you're  not
going  to  get  any  examples  because  you
know  open  covers  open  subsets  of  say  R
are  usually  not  compact  right  so  but  if
you  do  this  usual  thing  of  having  a
compact  neighborhood  of  every  point  then
it's  fine  but  I  I  this  looks  it  I  wonder
about  the  derived  nature  of  the  of  the  r
when  you  localize  because  it  seems  to  me
that
of  course  you  can  make  this  definition
but  then  you  can  ask  whether  for  example
what  does  mean  locally  oness  if  it  is
true  if  you  take  D  do  you  have  like  for
sufficiently  fine  for  small  open  it  is
let  us  say  that  is  if  it  is  true  for
some  cover  it  sufficiently  fine  one  then
probably  you  have  to  to  pass  to  to  to
the  ring  Associated  to  some  small
distance  those  are  derived  in  some
that's  it's  we  have  the  same  problem  we
have  in  general  with  the  you  know  the  H
discussion  of  the  Huber  theory  in  the
solid  case  it's  the  exact  same  issue  and
it's  the  exact  same  situation  that  yeah
you  can  fix  it  by  starting  with  some
more  General  class  of  derived  things
here  probably  I  didn't  think  of  the
details  and  but  also  in  practice  in  most
examples  it  doesn't  show  I  mean  in  in
Practical  examples  it  tends  not  to  tends
not  to  matter  I  mean  that  in  the  sense
that  these  local  the  you'll  have  a
neighborhood  basis  that  you  know  when
you  calculate  these  neighborhood  bases
they  will  just  be  in  degree  zero  and  the
Rings  will  just  be  in  degree  zero
and  and  they  will  fit  back  in  the
framework  so  you're  right  that  there's  a
bit  of  a  fly  in  the  ointment  in  terms  of
the  way  we're  presenting  things  and  that
we're  starting  always  with  classical
objects  like  when  we  talked  about
Huber's
Theory  and  when  we're  talking  about  this
Theory  here  we're  starting  with
classical  objects  instead  of  derived
objects  and  a  priori  when  you  work
locally  on  our  class  on  the  Spectrum
attached  to  our  classical  object  you  see
some  derived  objects  instead  but  in
practice  it  doesn't  tend  to  happen  and
in  any  way  you  can  modify  the
definitions  of  it  to  accommodate  them  so
it's  not  such  a  big  deal  it's  slightly
more  sub  here  that  the  basic
building  rings  but  then  she  of  over
converion  so  it's  actually  never  B  she
of  kind  dis  between  the  global  Al  start
with  and  no  but  still  that  doesn't  I
mean  that  doesn't  obstruct  the  claim
that  that  there's  a  neighborhood  base
you
know  yes  I  mean  ofer's  question  was
about  a  neighborhood  base
yeah  yeah  that's  that  is  a  good  point
and  I  mean  you  know  you  can  modify  these
you  could  also  you  know  from  instead  of
instead  of  these  guys  you  could  also
pass  to  inverse  limits  so  you  could
starting  with  these  apine  guys  you  could
pass  to  arbitrary  inverse
limits  for  example  like  so  inverse
limits  in  the  compact  house  door  space
and  just  filtered  Co  I  mean  inverse
limits  in  the  category  of  analytic
Stacks  which  in  the  Aline  case  is  just
you  know  Co  filtered  Co  limits  of
analytic  rings  and  then  then  these
overon  convergent  things  would  also
count  is  apine  and  then  maybe  that's  a
little
nicer  uh  to  work  with  and  there's  no
harm  in  in  doing  that  um  but  say  will
not  be  B  rings  but  they  will  be
condensed  rings  with  certain  yeah
they'll  still  they'll  still  correspond
to  analytics  an  analytic  stack  with  a
structure  map  to  a  compact  house  door
space  and  so  on  okay  so  you  probably
instead  of  B  ring  you  can  have  a
condensed  ring  with  s  proper  yeah  yeah
with  with  some  Norm  satisfying  some
properties  and  so  on  yeah  I  mean  we
didn't  we  didn't  try  to  give  the  best
possible  formulation  was  just  a  just
wanted  to  connect  to  the  classical  thing
yeah  okay  so  that's  all  thank
you  sry  can  you  again  with  it  local
section  oh  yes  so  on  the  on  the  category
of  locally  compact  house  door  spaces  you
can  define  a  Gro
topology  where  um  a  you  know  a  set  of
maps  I  mean
it's  a  set  of  maps  like  x  i  to  S  forms  a
covering  if  uh  for  every  point  of  s
there's  an  open  neighborhood  of  that
point  and  an  index
I  and  a  section  of  the  pullback  uh  uh
you  know  you  pull  back  x  i  to  that  open
neighborhood  you  should  have  a  a  section
there  um  and  the  map  can  be  ar  yeah  the
map  can  be  arbitrary  the  map  doesn't
have  to  be  an  open  inclusion  but  it's
also  the  same  thing  as  the  gro  dig
topology  generated  by  the  the  covering
families  which  are  just  the  usual  open
covers  so  if  you  look  at  just  the  usual
open  covers  and  say  that  you  want  sheath
condition  for  that  you  automatically  get
sheath  condition  for  anything  any  any
map  that  has  local  sections  so  that's  a
so  if  you  want  if  you  yeah  so  the  seeve
will  always  be  the  same  as  the  seeve
generated  by  some  open  cover  of  s  and  so
yeah  but  it's  convenient  when  you  want
to  talk  about  the  sense  in  which  a
locally  compact  housef  space  is  locally
compact  house  DWF  because  it's  not  true
in  some  naive  sense  but  it's  true  in
this
sense  okay  other
questions  another  question  it  seems  to
me  if  you  take  at  least  naively  you  take
another  p  as  the  norm  you're  supposed  to
be  a  constant  one  uh  n  Prime  of  Pi  Prime
was  supposed  to  be  a  constant  but  n  of
Pi  Prime  can  vary  over  the  burkovich
Spectrum  but  uh  I  think
with  supposed  to  be  a  constant  because
is  small  than  the  normal  Prim  Prim
inverse  the  same  was  it  to  be
actually  no  see  the  norm  Pi  Pi  Prime  was
adapted
to  P  Prime  was  adapted  to
um  uh  absolute  value  prime  it  wasn't
adapted  to  absolute  value  so  it's  not
adapted  so  this  this  n  here  wasn't  such
that  it  satisfies  that  property  with
respect
to
I  so  how  did  we  built  this  n  from  this
absolute  value
here  for  which  Pi  had  this  property  that
Norm  so  we  had  in  other  words  we  had  the
norm  of  Pi  inverse  equals  Norm  of  Pi
inverse
here  right  and  then  we  built  this
n  using  this  so  that  the  the  norm  of
every  element  would  be  bounded  by  the
norm  prescribed  here  that  implies  that  n
of  Pi  has  to  be  this  fixed  value  but  it
doesn't  imply  anything  about  n  of  Pi
Prime  because  Pi  Prime  doesn't  NE  Pi
Prime  doesn't  necessarily  satisfy  this
property  for  this  Norm  it  only  satisfies
it  for  this
Norm  always  you
okay  thanks
guys  
\end{unfinished}
% !TeX root = ../AnalyticStacks.tex

\section{\ufs Outlook (Scholze)}

\url{https://www.youtube.com/watch?v=YKw1XaueLJY&list=PLx5f8IelFRgGmu6gmL-Kf_Rl_6Mm7juZO}
\renewcommand{\yt}[2]{\href{https://www.youtube.com/watch?v=YKw1XaueLJY&list=PLx5f8IelFRgGmu6gmL-Kf_Rl_6Mm7juZO&t=#1}{#2}}
\vspace{1em}

\begin{unfinished}{0:00}
  Okay, so welcome to the last lecture. Today, I want to give some kind of outlook. With Dustin's lecture on Wednesday, we kind of finished what we promised in the first lecture. So today, I want to talk about some directions one could go in with the kind of machinery we developed.

Some years ago, I did a lot of pic geometry, and I always wanted to have a way to do this not just periodically but also with real numbers and over spy. But it was always clear to me that I really needed a completely new language to talk about these things. As I said already in my first lecture, this is the reason that I was really putting a lot of effort into this project. Finally, I have the feeling that we basically have now the language that we always wanted, and then now, it is a sensible question to just try to really use it to do a lot of things.

It appeared to me that there was this original goal that we maybe had in mind for what the series should do, but on the other, it's also good to look in other areas of mathematics to see how the theory might be useful. I'm not really competent, but I still want to give some vague ideas that I think might be worth looking at.

Okay, so here are some possible directions, and I will start from the most well-developed to the most speculative. First, we do have now a general theory of analytic sheaves, not just conditions imposed on the modules in analytic geometry in all flavors of analytic geometry. This unified theory is not just a formal thing but a full six functor formalism of six functor Street, which lets you play a lot. In particular, some things we kind of looked at using this formalism is that it's actually a non-trivial application of this general theory of sheaves without any Mysteron or otherwise hypothesis. For example, even for B spaces of finite type over the integers, we had to really work a lot to define things, but in our formalism, they just come with the structure, period.

There are also all sorts of Gaga theorems that you can reprove, but you can also prove various new sorts of Gaga theorems. There are various results about the sands of vector bundles, and for example, there's this famous paper of Greenfeld about infinite-dimensional vector bundles, where he proves some nice results, and I think our techniques could be useful for proving yet another variant of that kind of result.

We also know complex geometry, and we discussed things like off for complex manifolds, and such things are kind of one kind of approach using this formalism. One thing which we in some sense still haven't quite figured out but are quite optimistic that in principle could be done is to prove the Sing index theorem using our technology.

Related to these last points, they are, of course, very closely related to the notion of CAS theory. And one thing that was kind of missing for a while is the notion of the CAS of analytic spaces, which is defined first by Toal and then for general schemes by Thomason, and it really uses that you have a well-behaved category of coherent sheaves on.
It and then maybe actually stable. Infinity category of per complexes and then you can define the case theory of that.

But going to analytic geometry, there was the issue that there is not a good enough category of modules of which you could then take, apply these categorical techniques and get some kind of case theory.

You can actually do it; you actually don't use exactly those CL shields as there, but some variant of nuclear modules. But this has not been analyzed to some extent, so this definitely uses the work of Sasha Eimo to define Cas of dualizable categories. And then if your analytic spaces are actually Ed space, this was worked out in the Ph.D. thesis of Andf.

We had some problems at first to define the complex numbers, but also some ideas how to do that. The Cas here might actually also help for this. There are actually also some other relations to the work of Sasim, so he has these very strong results about the category of localizing motives, proving that it's a rigid category, particularly dualizable itself. Using this, you can actually define certain refined variants of Clic hology, topological cyclology, and so on, that are actually not taking values in some kind of complete category as usually when you take as one6 points you get modules over some power series ring which are complete, but instead you can go to nuclear modules in our sense again.

Let me mention maybe that if it's okay, Dustin has a joint project with Brund, where they use some of this nuclear Cas theory to settle some like questions from homotopy theory. Oh yeah, but we figured out how to avoid it actually. Ah, okay, yeah, too bad.

I was a bit disappointed too, but all right, so that maybe where I'm kind of coming from, and where there had a lot of applications, and where really a lot of work has already been done is in the area of like thetical modation and so on.

There was a spetic hology, which was defined from former schemes, and one question that people had was how to really define it for rigid generic fibers. This then very much is related to analytic stats in our sense, so in particular, there is no what's called the analytic R st, defined by the work of Gu. It gives you a way to talk about six fun formalism on what's classically known as like dcat modules, so there's a certain completion of the ring of differential operators, giving some kind of differential operators of infinite order, defined by Aov and One SL. They suggested that when you work in analytic geometry, you should really look at these modes over these dcat modules and try to get a six formalism for that, but as usual, you run into something of function analysis issues doing so.

Using our analytic geometry, Ro G was able to write down, by specializing the six fun formul to specific St, a very general six fun formul for these dK modules.

This is related to D modules; there's also in the pic world the analog of like hi hi bundles, and some kind of Simpson correspondence, the first incarnation of which goes back to Dinger and Fings. These can also be interpreted as in terms of a stack, and so that's there so-called Analytics St.

This you can find in particular, there like, okay, maybe not yet using our technology, just essentially work of Aner and H, who recently obtained really strong results on the P correspondence.

This work on these hotate staks is also very closely related to what's known as geometric 10, which originally arose from Sensi, which is something about P go representations, and which has been used to very great effect by Lou Pun and also Ralo for applications to P points. Also, last Friday, Vin Pon gave a talk about his joint work with boxer, Kary, and G, where they proved modularity of Genus 2 curves of a Q in many cases, and some of the key technical parts of this proof is really using this kind of technology.

What this also is pointing to is really that there should be some version of theal for loc representations, in terms of some kind of geometric Lance of the center, so some version of what I did in my paper was long thought. I mentioned here something that's in some sense combining or just different the and St St, so is also from antic citation.

Thing. This is a proposal for something in this direction. I propose this by Helman.

By and by, in this direction. So, I mean this is of course maybe what my original interest is, that there was this work with SPK on consideration of local Langlands, and I would like to formulate all incarnations of local Langlands in such terms and eventually then also Global Langlands.

This is just to a large extent, I mean, at least work in progress, and this is probably already very speculative, but this is something that's very actively investigated.

Actually, a lot of progress on this was made during this house trimester that happened last summer here in Bonn, and in particular, we discussed a lot about these things. Then, at some point, we realized that now that we understand the Brau story really well and that we understand the kind of correct geometric language to phrase these things in, we can just basically one-on-one translate all the ingredients in the Brau world to the real world. So, there's a real analog of virtually everything that's happening in the Brau world.

So, there's also an analytic stack, and that's actually a funny version that is actually isomorphic to the Brau stack, as it so happens in this case. Where this is some incarnation of a real Hodge correspondence, you can also define an analytic P-space, which in this case actually maybe there's some kind of non-trivial JP. Okay, and again, there's some analytics stack that encodes the vector BCE, and it encodes some kind of periodic variations of F-structures, and so there's also an analytic stack including variations of twist structures.

Of course, this ties in very well with all the work of Simpson in this world, and then Maukie developed and really developed the Ser variations of F-structures, which are generalizations of variations of Hodge structures. There's a certain action of new1 on these things, and you want objects, and then it seems to be possible to synthesize everything and get also a formulation of real local Langlands correspondence for real Lie groups, for some kind of locally analytic representations.

Good. Can you produce the weight filtration too in the analytics stack? It's a very good question. Let me comment on this in a second. Not yet, but there's some kind of geometric Langlands on twist1, which is a kind of real analog of Brau.

I gave a talk in Muenster three months ago, where I was outlining the general form that we should take. I will give some three lectures actually in Princeton a month from now, and I'll say a bit more about how this is supposed to work.

Actually, part of this, it's maybe a small thing, but maybe not. Usually, when you talk about representations of G of R, you run into all sorts of functional analysis issues, and usually, you replace them by more algebraic notions of Banach modules by passing to the finite vectors and some compact subgroup. But then the theory somewhat less invariant because you need to choose this K, and it becomes more algebraic.

But in the real world, there's really not much of an issue of really encoding representations of a real group. We don't need those, but you have many representations which have the same Harish-Chandra modules because you can use different functions.

What we actually do is we will look at the real group. It's a real analytic, it's a group object in real analytic manifolds, and so you can do the kind of real analytic incarnation in the analytic stack. And so, this is a group object in the analytic stack, and you can take the classifying space of this. Let's say the stack to guess this complex number is not that, real. I mean, as usual, like the classifying space, there are something like representations of the group, but to realize them representations, you need to go back to the point where, and there's a functor. So, there's a projection from the point to here, and you can take either P or P upper star, and for any of these usual G-K modules, there's a canonical object in here, and then the P upper star, should I mean, this is something we...

The P should produce a minimal globalization. The P Street should produce a maximal globalization. So, what C you take, you take C with the gas. Yeah, you can take the gas as it's enough. And so, you have the.

Of course, this is closely related to some results on existence of analytic vectors in representations. It should; otherwise, you would not do, do you use such? I mean, the fact there are enough analytic vectors. Let me not try to say anything precise about this relation here, because it's something I still need to think much more about.

So, there was this question about weight structures. And I believe it's related to the following.

Now, we run into the speculative realm. Really, so for these things, I'm pretty confident that some version of this will work out. This is at this point more of a speculation, but I have a very strong belief in it.

Classically, in all sorts of questions about function analysis and complex geometry and everything, you often really need to put metrics on something. If you really want to prove the H decomposition, at some point you need to do some L2 stuff, put metrics on stuff, and so on. This is something that we cannot yet incorporate or what we cannot yet translate into our world everything that's related to metrics at this point. But I believe there is a clear way to look, namely, you should look at this extended version F that we had. I will connect this in a second to this question about weight structures.

So, we have this Berkovich space, and maybe the way to do it is WR. So, you have some kind of central point related to the real numbers, and then there was a rate for Q2, and then at the end of this, you have kind of F2, and then there was a rate for G3 at the end of the G, F3, all the other primes, and then there was also the Prim corresponding to the reals.

Usually, like in the Berkovich space, this kind of ends in the middle because you ask for a triangle inequality. But we don't have to do that, and so there is no point at infinity here. And there's this point at infinity, and our geometry kind of tells you that this point must be related to metric. And to some extent, you can already see that when you work with some fragments of it, but also by analogy here, like if you work topologically, then extending over this point precisely means that you put some kind of vector bundle over this thing, like a Z2 mod over the integral topological numbers. And so, extending over bundle places is definitely putting some metric on things, and it's very sensible to think that whatever exactly happens here, it must have something to do with putting metrics on some kind of real stuff.

Some question we have in our mind is whether there is a way to prove the Hodge decomposition, like for complex K-manifolds or something like this, using some geometry that will involve this extra point here. In this abstract language, this is kind of difficult for me to think about, but we actually have a very good analogy. Again, I mean, we can use this analogy between three and four. Periodically, if you work over this kind of part of this picture, then over this part of the Berkovich space, like over the open part, you can define a complex space in the sense called QP, locally analytic. This corresponds to the union over P, where Z is really just the analytic spectrum of the locally Euclidean function from DP to GP, so ones that are locally developable into power series expansion. And this kind of comes up very naturally in all these investigations there, and we know that this analytic space, which lives over the open part because it naturally has K coefficients, this has a canonical extension. What was the F2 point? And so, this is very much related to a theory of locally analytic power series representations, I mean, these are the coefficients of the groups of some PP, at least some fragments of which you can find somewhere in the literature.

Periodically, there is some kind of, yeah, so this canonical is actually a little bit subtle to write down, it has some divided powers. But it exists and is very important for this P story. And one thing this suggests, and which I don't yet know how to do, is that if you look at the real part of the picture, real, and then close on infinity, then...
Over here again, you also have the real numbers, like as a real analytic space. So, the thing that's covered by, yeah, so there are as a real analytic manifold, and again, you locally, the thing with the functions are the real analytic functions. So, the ones that are locally developable and locally be developed to power series expansion, something that match to the base, and I mean, it match to the to the open part of the base, but like each, each $\phi$ over a point is this one, and it suggests that this should this should extend canonically over the function, yeah, canonically over close Point Infinity.

In a way, I don't yet know how to really think about, but this also suggests that if you have some, like I mean, if this could be done, and this would be some kind of ring object, then also this real analytic group would flips over like, oh yes, this should also should also expend.

So, here, you put, you use the $\mathcal{G}$ at all points, you use the $\mathcal{G}$ structure or use $\mathcal{L}$ for different, no, we don't. I will come to $\mathcal{L}$ structure later today for now, it's not needed for anything. So, you just use the same, the same at all points of the, yeah, so acting on this again, you have the rescaling action. So, there's a lot of different copies of the real numbers now, sorry for that. You, a raling action of the, every single $\mathcal{V}$, but but, and it maps to the $\mathcal{B}$ in in your sense, to the $\mathcal{B}$ space, I mean, to the analytic stack of the. So, this, yeah, so this, this maps to the space of norms, and it's an open Subspace of the space of norms, and over there, you have this $\mathcal{S}^\mathcal{T}$ which is like real of locally analytic, analytic thing, which lives up over the open part over the open Ray, but then there should be a way to canonically extent.

So, but properly speaking, you mean $\mathcal{R}\mathcal{L}\mathcal{A}$ cross $\mathcal{O}(1)$ right, or $\mathcal{Z}$. And so, I expect that whatever kind of group that is, in maybe representations of this have some kind of metric structure attached to them. I don't really know like that, if you want extend representation over the open part of the punct, I would expect this something to putting a metric on it, but I don't know. These are some objects that also exist, that I don't yet know.

All right, maybe let me mention, ah, I mean, also, I mean, this some like, and you can also just just try to understand like, we can look at analytics text over the space of Longs that really map to the close Point infinity and try to understand what the kind of geometry is there, and this is some very peculiar geometry where you, it's still about some kind of real complex manifolds or something like that, but you are able to localize much much more, you're able to really zoom in finely into your space, and so I think it's very interesting to try to investigate what geometry this point looks like. I can all agree, so we can zoom in and some, it's a very different.

So, I mean, you have to, for example, you have to use $\mathcal{R}\mathcal{E}$ sphere, and then the bounded part of the $\mathcal{R}\mathcal{E}$ sphere is actually just what seems to be just the Close unit disc, the bounded part is some kind of weird overconvergent, minimally over convergent neighborhood of the of the close unit disc. It's essentially just the $\mathcal{C}\mathcal{L}$, so any real number that's bigger than one is an unbounded function on this thing.

So, I think it will take a while to figure out how, why do you picture the sphere? Sorry, Peter, why do you picture the sphere? I mean, actually, I mean, we were always looking at this, this normal like $\mathcal{T}^1$ right, $\mathcal{T}^1_\infty$, yeah, and and the 
Antincs

So, like any fine guy, I mean, we first associate just the drive category, but then we could also associate module categories over it. But then, as I was explaining last Friday, a series of present infinity n SP due to Stanage, um, and so you can just so you can do two PRL just continue forever. And I mean, so you can define n PRL on X for any X, and you cannot just define it, but again, there's some you have all the kind of six fun on it, and so you can just play with it.

The existence of this is not a speculation, but what comes next is definitely a speculation.

Okay, so I have a strange project with stars and Z where they're computing some Quantum CH Sim Varian, and I don't know what they are, but they seem to get certain power that seem to be very related to the kind of periodic structures I'm seeing. And for a long time, I'm trying to make sense of whatever they're doing.

But now, recently, I realized that I'm able to send about higher categories, and that well, the C hypothesis tells you that this kind of to fi series I should really just be certain higher categories and then try to understand which one they should be. And you can essentially say it, but it doesn't quite work.

So, here's what's called Quantum transus or something. This, I must be aware that I'm saying words that I don't understand.

Okay, so, so, so, to the extent that I understand anything, um, you have maybe start with a complete group and maybe Forlani simple or something like this, relevant. And then you fit what's called level. I mean, so there's a funny computation that if you look at maybe I don't know, let's see simply connected, so then the first interesting formology group of G is a third formology group, which I mean a simple assumption z, um, and yeah, so let's fix the level, which is just an element.

And so, here G is just considered a topological space, but then you also have kind specifying space of G. Um, I mean, g map to be Cub, so this, if you want the net from G cling space of z, um, this actually d loops it's map of groups automatically obstructing, so m from BG to B4 to the z z. Um, and then, but you can also restrict that to G as a real analytic thing.

Right, because in general, like I had to think that whenever you have a real analytic manifold, it maps to the incarnation of m, like is a condensed set kind of incarnation, and this basically what I'm using here. Uh, and so, but then there, like, okay, so then there's also like the exponential sequence analytic.

Yeah, thanks. Um, have exponential sequence, so everything's living here over, let's say, guess is complex numbers, um, and then a composite map to be to the four of GA venes because for here chology has trivial and so this actually lifts to be cubed of the analytic GM and of course the analytic GM fus.

So now, this sounds really like the enact. All right, but so, so, so, what does it mean to give a map to be cubed here? Well, this is too hard for me to think about, but um, so map to BGM, well, that's just the line model, right? So, BGM, that's just a classifying space for line bundles. Then, b square GM, uh, this is something giving you some algebra or something, so this giving you a Twist of the category of modules. And then the Cub GM, this means that you give a Twist of the category of categories is over over it.

Yeah, so this map called Alpha from BG a real GRP QM specifies a Twist of like PRLG, I some, let's call it just L or invertible, it's invertible object is a 2PR.

This, all right, okay. So, have this, uh, and you have the projection just to the point, or the point is like the guess complex numbers. And I think what people do is like, okay, so they, they want to look at like some family of G tsers like bundles with a flat connection or non-flat connection, whatever, um, and but sometimes Twisted by this Alpha. And so, I mean, this is somehow more or less governed by pulling back this L under Alpha, but then they want to integrate over the space of all G ches. So
Somewhere in a three-category, so now we have one. So the question is, is this streetable, and as any expert would immediately tell you, this, there's no chance of working.

Because the space of conformal blocks, I believe it's called, you must put some holomorphicity constraint, and I'm kind of not doing that here. But I mean, in some sense, it's not so far, so, but it is true, duable, I think that's, that's, you don't need much, okay. So I didn't carefully check it, but I believe it is to dualizable, and to check that it would be S-realizable, and for S-realizability, you would need the following things:

You would need that $\pi$ is kind of from proper and smooth, so the diagonal of $\pi$ is proper, smooth, and the diagonal diag of P is proper, smooth. So there are six conditions to check, and only one of them fails. Okay, so more concretely, this is like the point, this one reduces to the group being proper and smooth, and this means that the inclusion of the point into this, so for all of these, you would need that they're proper, that they're smooth, any guesses for which of the six fails? Actually, the one you probably would expect the least is the smoothness of this map, the smoothness of the second, yeah, only.

Just a digression, the map to be, you said that it lifts to a map from B4Z to B3GM analytic, but you want to BGR locally analytic instead of BG, so the obstruction is something having to do with topology with coefficients in something like continuous topology, but it is in your language, so I'm not sure what it corresponds to, but it's just the étale topology of the stack, basically, the topology of the stack, but it is not true for BG itself, it lifts, or I mean, here would be the topology of kind of condensed set ST, which would actually be the étale topology of this funny ST is singular topology. So if you would go here to be for those complex numbers, you would just T the, this is a singular commod with a complex number, so this one wouldn't lift. You need to go to, the obstruction is has to do with maps to, to right? Yeah, so it's some kind of, I mean, up to this analytic, it's basically coherent, but on this classifying ST as a condensed set, the coherent topology is kind of singular topology. Ah, okay, okay, and it's because it's a compact group that it vanishes on the go to the lotic, yeah, but the compact is much more crucial for these kind of problem sols we can hear.

Okay, so this doesn't word, but in some sense, I feel like it was so close to working, so maybe you just need to tweak this, this here a little bit, and as I said in five, there's actually a canonical candidate where you kind of go to the Point Infinity, maybe that one works, very naive question, uh, doesn't word, how replacing by fiber, fiber of the extension first, and it seems weird that this should put the kind of correct holomorphicity constraint into the picture, but I think the formalism might just work out to do that, and there's also some structure you're not using with the like, like, um, like I think this map to B3GM that you build, it should really have a connection, I mean, like, I think that there, you know, could kind of be going to deLine topology and weight too.

Uh, yeah, so, so maybe this is not yet completely correct. I just want to point out that I mean, you can just play with these things, and you can hope that using if you would actually understand what you're really trying supposed to do, uh, you could write down something that would actually produce something special. So I mean, usually what you would try to do this, you would run to all sorts of issues that you always want to do functional analysis and high categories, and I mean, people manage to do a lot of things, but here you can just very naively try.

Alright, so this brings me to the last thing I want to talk about a little, and this one I actually want to go into a little bit more details, and actually, I mean, this was very fancy hyro, and now it becomes a bit more concrete again. 

So this is about a theory of where fundamental proces now, it is we don't see very well, it's condensed, yes, cond, it'll resolve.

So, yeah, so
The kind of theory where our situation where our series should be useful, and so I just try to see to what extent it is useful. Okay, so let's consider a fiber-like model situation that we may be interested in the following: Consider a fiber product of many M1, M2, next of let's say compact, oriented compact M, M1, M2. Feel free to assume that these Ms are closed immersions. I don't think that's relevant, but I mean, so yeah, consider, but which is of expected Dimension zero, in other words, like d M1 plus d M2. So then, if the transverse intersection turned out to be transverse, this X would just be a finite set of points, oriented points actually, and so you could count the number of intersection points.

This intersection is transverse and finite, and actually also oriented, so any point knows whether it should be counted as plus one or minus one. So you get a well-defined counted sign count of the elements of X. This is a variant under so long as you stay as intersection stays transverse. Okay, so then that's the setup. For, always a question, the question is, can you Define this sign count of X in general, I mean, without transverse intersection, purely intrinsically on X? Of course, every connected component will have a well-defined number, which is the part of the intersection number coming from this connected component.

Sure, but even these local ones you need to Define, right? The good situation, I know, mean you have two circles meeting transversely, and then this plus one, this minus one, intersection zero. But now, this might be generated, I, you might have two circles that I don't know, some situation like this where this might be, I know, this, this, might situation, it can be infinitely conic and Stu, I don't know, I mean, this might be tangent to infinite order somewhere, might be the same for, then cross, I don't know, all sorts of really funny behavior.

Okay, so if instead of compact manifolds, you have some kind of smooth projective varieties, then in this case, it's known how to do this, but also like the intersection cannot be as bad as for smooth manifolds, where the intersection, basically, as a topological space, has basically no structure whatsoever.

Okay, so let me first State Z, yes, compute the F product in our C, so in particular, X is some kind of deriv, and this time, it's actually somewhat critical to do this not over the gases real numbers, but over the liquid real numbers for some choice of id structure. This depends on a parameter that doesn't actually matter for this application. Yeah, so for everything else we did in our course, this kind of liquid structure that we once produced was very much effort, not so relevant, but here, I think it actually really is relevant for reasons I will explain in a second.

So what's the, so yeah, so we can Define kind of a notion of we can look at analytics texts in our setups that just have the property that locally, it is possible to write them such an intersection, and then, one can can only produce a virtual fundamental clause on. Okay, so but, there are some symplectic experts present at NP, in particular, Nate Botman and K Barheim, and I kind of discussed this a little bit with them already, and I want to continue those discussions. But in particular, they made me aware that, like, even in this kind of simple space, it's kind of surprising that this should work.

So, let me say why this is surprising. Let's consider the case where those M1 and M2 are Cur, there's my picture over there. And then, just assume you're in a local situation where you kind of have two things meeting at just one point locally. In the best possible scenario, this intersection is transverse, you just get a point count, I don't know, plus or minus one, depending on your orientations. I don't know, then the next worse scenario is if it's a square function, then okay, this to be zero, and if it's like a cubic function, there should be again, first one, and so on. And so, this tells you that if it's Vanishing to finite order, then basically, it's clear that we're okay, right, because we just need to remember the vanishing order of that function, and this will tell you what the this function crosses the line or it doesn't.

No, but now let's consider a situation. As you can have $\mathcal{C}^\infty$ manifolds, but there are more. Tangent to infinite order, something like $x - 1, x^2$, but then you can have a very similar function which is $s(x) \cdot x$. And so what I'm claiming is that the locus where this function is zero, whatever that means, determines whether this function crosses a line or not. The claim is: the vanishing locus of these functions, just the vanishing locus, determines, I mean distinguishes, these two cases in particular, determines whether the function crosses a line or not in which category in our category.

Okay, so let me make it slightly more explicit. That, like, in the classical geometric picture, it's quite unclear what kind of structure you need to give the anything. I mean, the intersection point which knows it, it must, in some sense, must know more than all the derivatives of this function, because it will never be able to double from the derivatives, but it knows much, much less than the germ of this function. It really only knows eventually. So, classically, we would maybe try to use the whole germ of this function in order to remember whether it crossed or not.

Right, so what is the locus of say some $\mathcal{C}^\infty$ function from $\R$ to $\R$? Let's assume it really just has an isolated zero, $Z$. So this vanishing locus, in general, for this, is just the analytic spectrum of the $\mathcal{C}^\infty$ functions from the module generated by $f$, where $f$ is this thing here. I mean, it is a liquid, special type of condensed $\R$, with the analytic ring structure. Analytic ring structure doesn't matter. It is a very funny one, though. I mean, if $f$ would only manage to find out to order then, this would just be the usual polynomial algebra. Finish in, and so this is some nice algebra, but, if $f$ vanishes to infinite order, then this is some funny non-separated thing, and so technically, you would have trouble and this with a topology, but in the condensed world, I mean, it has a natural condensed structure.

So, here's a proposition that tells you that the vanishing locus somewhat knows about this. If you have two functions, $f$ and $g$, and there exists an isomorphism of liquid algebras between their vanishing loci, you mean, analytic rings with the liquid... Well, I mean, I don't need to put the analytic structure on them, because it's just induced one, so I can really just look at them as liquid algebras. Also, I mean, $f$ is still a non-zero divisor here under this assumption I made, so this really still concentrating degree zero, just not Hausdorff. So, I don't have to say anything.

So, assume that these are isomorphic. The condensed set of $\mathcal{C}^\infty(\R, \R)$ means that you have to define smooth functions from $\R \times$ a profinite set to $\R$. You do it in the usual way, yes, the $\varprojlim$ construction, just completely internally in the condensed world, and it will produce the right condensed structure. Or, just remember that it's a $\mathfrak{C}$ space, topological thing, just pass to condensed after verfying those, give you the right condensed liquid thing. It doesn't change the condensed, liquid is just a condition. I mean, just say condensed.

Okay, so assume this, then, actually, $f$ is $g$ times $u$ for some is an invertible function. And so, in particular, being an invertible function from $\R$ to $\R$ must either be everywhere positive or everywhere negative. And so, multiplying by such a unit cannot change whether the function crosses the line or not.

I believe like on nuclear nuclear fet spasis the liquid tensor product is doing the expected thing and so it actually is producing the C functions on the product, but it's not true for the overall G there would be some smaller thing. But concretely, what does it mean? A liquid is a set, I mean R is just the usual condensed ring, yeah, real numbers, but then you put different ring structure there which is much closer to asking that like the building blocks are some kind locally convex things. They're not quite locally convex, they're just $\alpha$ locally convex versus $\alpha$ might be slightly less than one, but you're very close to locally convex setting. And so when you form this T products, it basically allowing all locally convex combinations, and at least if you kind of have nuclear transition maps, then the precise kind of complexity you need actually it doesn't matter, one limit works okay.

So this is the deriv T of product, it would be works. So this is a non-computation that comes out right in the liquid World, and so from this you can deduce that if you take $C^\infty$ functions from $\R$ to $\R$, that this is actually isomorphic over $\R$. 

Uh, is a usual coordinate, yeah, so maybe isomorphism should be condensed our algebra with the coordinate in the proposition, otherwise I don't believe if you don't have the coordinate, so I mean I said that both of them have the isolated zero at zero, that's what I use, but I don't use anything more than that. Let's see what he's doing, but he's not.

All right, so let's analyzation. So let's assume we just have an abstract ASM. So for both of these, they admit unique map to the real numbers, because you can classify all the maps from $C^\infty(\R)$ to $\R$, they just given by real numbers, a variation, that's some real number and only one of them, one of them f is equal to zero. So what is definitely true is that they have this unique evaluation to the real numbers, and this must commute, because in those cases.

All right, and now I have $\R$ joint key equal to zero, nothing to here, why is the usual coordinate. Right, I mean maybe I should $C^\infty(\R, \R)$ here. And so I get there, and so where does this T go? Well, I mean, let's assume G is not just also Vanishing first order, then like I mean you have this unique map, and you also have to Unique first integral neighborhood, so this must be a function that it's not managing to first order here. Do you actually care about this? Maybe I don't care.

All right, so I have this map, but okay, I also have this compos map, and of course it has a quotient here, and if I want, I can also lift to here, right, because I just need to lift my element T from here to here and I can do that. Yeah, so you get it f is equal to G * unit up to a change of coordinate, so it's G times the change of coordinate compos a coordinate change times the unit change.

Okay, let me to this algebra here, just I mean let call this here A and this here B. So one I want to claim that this map from $\R$ joint key to B is a usual, there exists the unique extension which is just a condition. Okay, and uh, so is a module same, and there is a unique extension in which sense, in the sense abstract Rings or I mean this sense here, right, so it can be at most one extension, okay, saying. And I claim it exists, and for this I can, it's enough to check that it exists here, but I can classify all M from joint T to here, and they all do extend, right, okay, they do extend by just comp by comp. All right, and you really just need the extension is necessary, you need to just need to prove existence, but this map is given by some $C^\infty$ function from $\R$ to $\R$, but then any other $C^\infty$ function from $\R$ to $\R$ can just be evaluated
At least, this property about F crossing the line is only deep crossing the line because so there is n's presentation, not saying it right, but let me make a point here. So, there are some existing theorems, a series of derived manifolds, but they proceed very differently, and in particular, they consider some kind of algebra of functions where, by definition, whenever you have a function, you can kind of compose with any $C^\infty$ function from $\R$ to $\R$. This is some kind of data I put on the Rings, but this observation is telling you that these kind of funny derived potions that we're getting, they just have the property that whenever you have a function, you can always canonically extend to $C^\infty$ function. 

So, they acquire the structure that for any function, you can compose with a smooth function, but you don't need to put it in the beginning. Try to see if can Solage was trying to save. This is the canonical put that you use that probably, and I mean those things can be classified; these are just given by usual new function from $\R$ to $\R$. And so then, okay, so has a ization, and so maybe you have to put a reparameterization in, then just makes a similar, sorry, I want to say the same thing, it's faithful embedding. 

H, so this just pullback by right, so yeah, okay, maybe up, okay, so right, but so some of the thing that makes this work is precisely this thing that you do get these unique extensions to $C^\infty$ function, because otherwise, you cannot control the kind of $C^\infty$ from here to the $C^\infty$ section from here, you wouldn't be able to compare.

Alright, so let me now state some more objects. Let's say, as locally compact Hausdorff space, and let's assume also it's a five-dimensional, which is always satisfied when it's intersection-like that, plus a cheve of animated all together. So, that is locally isomorphic, but not with a given data, so this is just a condition, intersection. 

Then, one can define a family, depending on S, of smooth manifolds, or just constant intersection of smooth manifolds. The intersection of smooth manifolds is constant locally or is it varying continuously? No, I mean locally, you can write open subsets of S can be written as the fin where the that's so coming from the point, yeah, ah okay, S itself is okay, S, S, this is okay. Then, one can define an expected Dimension, some C, it's a locally constant function, this integer coefficient, and an orientation, $\Omega_s$, so this is a zero persistent on, right?

And, okay, so let me denote by $\pi$ from S to $\star$ just the projection of locally contact po spaces and virtual fundamental T of like S equ structure sheet Al, also this depends on S, all of these depend on S, which is a global section on S of the dualizing complex. This makes sense, yes, this makes a lot of sense, and is because in particular, if the expected dimension is zero, as is compact, and choose an orientation, oh my God, fix, I mean, assume exist and fixed one, then you can define the Fundamental class, Global fundamental class, as some of the integral of the virtual fundamental class, Global, I know the count, signed count of SOS as the integral of the virtual fundamental class over S.

Okay, so I mean under G equal to zero in this orientation, this goes away, and then you just, and as compact, you have a trace map that goes back to the integers. Alright, so let me sketch the construction, and so basically, this is just the direct analog of the algebraic geometric construction. I think this was pioneered in the work of, and, then many people worked on this, and I'm not an expert on this at all. The paper where I learned it from, some kind of algebra, I mean, lination I know by a, but the ideas, I think, will maybe. 

So, what they define first of all is this, what they, I think, called the intrinsic normal cone, and so this realizes.

That describes some of the fibers. Some of this corresponds to the cotangent complex. Of like over R, in our world, I mean, first of all, you can somehow define such a derived intersection of like smooth things. But also, if you compute what this kind of cotangent complex does, that can like abstractly be defined for $C^\infty$ functions, basically for the same reason that the T-products came out right. The Čech complex also comes out right for $C^\infty$ functions on a manifold. And then, for such a derived T-part, of course, you get an S. So this is actually a perfect complex.

What's the superscript on the L? Here, some kind, so actually I just want to define this here as a locally compact space. Let's say it's a condensed, some kind of stacky version, where some tangent directions give you stack directions. I mean, it's not really, you don't need to go all the way only to group, you don't need high, I don't need high.

For the L itself, it is concentrated in the okay, okay, yeah, sorry, perfect amplitude, amplitude 01. So, the already did it. Okay, for a compact space or a pro-finite set, if you want, T. Well, first of all, from T to S, you can just regard this as a map from the space of continuous functions from T to R to some, let me call it, unexpected SOS. So that you can, by locally the spectrum of this algebra, you can build an analytic stack in all sense. And then, if you evaluate this on just the continuous functions, you just get maps from T to the underlying set. But now, I want to define what are the maps from T to this intrinsic pH.

This can actually be defined as a map from a funny algebra. You take this guy, and you model it by zero. In other words, this is a tensor product of the continuous functions from T to R, over R, join Epsilon, where the degree of Epsilon is equal to 1 and some particular square to zero. But where the structure shift comes in, okay, it's so, here it doesn't really matter because you always factor over the continuous functions, but here's the R-structure that is focusing that to something. There is, okay, so in the first thing, you can replace it by the kind zero, and even by the homotopy class, because here, not, and just by the, you can analyze what does it take to lift the map from here to here, and it's just something in terms of the cotangent complex. And so, you can realize that the fibers here have some kind of geometric incarnation. You also have the cotangent complex, so all the tangent directions, they give you kind of stacky directions. And, yeah, and if there's some kind of actual obstruction, this is the non-smooth part, and there's actually also some vector bundle directions.

Yeah, some particular kind of smooth map, or can understand it. And so, and yeah, right, so there's a Čech complex, and like the perfect complex, so it has a kind of dimension, which is the difference between the vector bundle dimensions. And so, this already gives you the expected dimension. Just look at the rank of this guy. And this kind of situation here, this geometry here, will also produce the orientation data for you. Think about it.

So, actually, what you really need to produce is the global section, an element in this intrinsic normal cor, with values in the data you really need to produce. And now, there's this really funny thing that let's take R as a condensed set here down here, and then one can write down something which everywhere except at zero, it's just a point, but in this F at zero, you get this funny fullness. There's some kind of condensed form which lives over the real numbers and which everywhere except at zero is just a point, but then as you degenerate to the point, suddenly, there's very sticky thing, and you see this condensed thing here.

Corresponds to, but first of all, let me use G for safety because I'm not sure if I've used F before. Corresponds to a map from, like this should map to R. So, first of all, correspond to a continuous map from T to R.

But once you have that, you can look at maps from the continuous function from T2 R modul G, where, of course, if G is the zero maps, then this is just recovering what I told you before. If G is non-zero, well, then if equ equ by an invertible function, this is just zero, so you're not getting this is no data. So, you guess just get a point, but then if G is a function that's somewhere invertible, somewhere zero, then suddenly you have this kind of thing where is producing some kind of doing these things together.

And now, it's just some simple game with six fun just to produce a class because, basically, generically, you just have a canonical section here, it's just a point, and then you just degenerate that section.

Let me try to do it. So, consider the sheath F which is form. So, this guy here is streetable, and that's the only way I currently know how to check it is to use it. I assume that a der section of smth manifolds, and for manifolds, it's true, and then it's disable on fiber products. In principle, you could probably produce this canonic virtual fundamental class with much small assumptions. The only thing you really need to that this map here isable.

But then, you can Define this thing which is a sheet on followe, and then, okay, so you have an open subset J which is the cone where from zero, and then you have I which is exclusion from the cone, and then you have some excision sequence for F. Here, you have which is just I the of the p stre, and then, in particular, you get a boundary map here, and the boundary map just gives you the CLA you want. And that's always a degree shift, I'm getting confused about, but it did work out.

Because the ididentification the I F when you, yeah's a shift by one appearing here because I'm pulled back from, uh, yeah, so there's a shift by plus or minus one here, minus one because I take Z here on the real line and not pulled back from the point. And so, then you get a m from the X zero of R Z Z towards the H, and okay, this, and then you take a class here which is zero on one connected component and one on the other, and the image is what you're looking for.

By the way, is it true that this pie cone def is not just triable but smooth? No, I don't think so. I would have guessed okay, I don't think so. I so this cone of s, this is some crazy thing right? I mean, it's some kind of fiberwise, it's some kind of vector bule like thingy, but over this crazy s, I mean, this s is completely nasty in general. So, if this map was smooth and also like the F was smooth, but this has just no way of being smooth, I don't think.

Okay, so I don't think this thing that appears here is in vertical at all, but you can still can still degenerate the topological fundamental class on a point towards this cone using a little bit of six fun, and then, yeah, this Con difference from the original space just by some kind of fiber BS, which you can analyze and which give you both the dimension shift and the orientation I'm over time.

I want, I don't know if it obstruction. The well, the TR obstruction series isn't that just that there this quention complex that behaves right? Okay, that's just completely infil this six. Any further questions?
Right. So, it stays a triangle on $St$. Okay, okay, okay. Okay, thanks, Peter. Yeah, all right.

\end{unfinished}

\bibliographystyle{amsalpha}
\bibliography{./bibliography}

\end{document}
