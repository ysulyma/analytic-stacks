% !TeX root = AnalyticStacks.tex

\section{\ufs Gaseous modules (Scholze)}

\url{https://www.youtube.com/watch?v=krq6jCy-dhE&list=PLx5f8IelFRgGmu6gmL-Kf_Rl_6Mm7juZO}
\renewcommand{\yt}[2]{\href{https://www.youtube.com/watch?v=krq6jCy-dhE&list=PLx5f8IelFRgGmu6gmL-Kf_Rl_6Mm7juZO&t=#1}{#2}}
\vspace{1em}

\begin{unfinished}{0:00}
e
so  last  Friday  I  was  uh  discussing  the
the  paper  look  the  curve  and  uh  I  was  um
I  was  in  some  sense  already  introducing
a  new  kind  of  analytic  rink  structure
that  we  called  the  guess  analytic  R
structure  and  so  today  I  wanted  to  pick
up  there  um  and  uh  but  some  to  analiz
this  guess  and  on  the  GRE  structure  uh  I
will  need  to  use  some  of  the  results  at
Dustin  um  explained  on
Wednesday
okay
mod  and  okay  so  so  the  plan  roughly  for
today  is  uh  to  First  say  something  about
um
guess  I  call  the  guest  dis
spacement  which  is  a  certain  certain
version  of  arithmetic  long
series  some  in  some  sense  some  kind  of
suffering  of  the  wrong  Series  in  some
variable  that  I  was  s  called  Q  last  time
um
then  actually  want  to  take  a  little  bit
of  time  to  talk  about  the  corresponding
Ser  of  the  real  numbers  guess
real  and  finally  I  want  to  like  the
motivation  for  this  was  uh  coming  from
ptic  curve  and  last  time  I  someone
claimed  that  this
G  structure  would  be  good  for  that  and  I
want  to  start  a  little  bit  explanation
for  that
so  I  want  to  say  just  a  few  words  uh
already  now  about  the
construction  of  the  table  of  the
C
um  but  at  that  point  we'll  have  to  do
some  more  serious  analytic  geometry  and
I  think  that  will  be  a  point  where  we
then  we  start  to  develop  more  a  general
formalism  to  really  carry  that
out  all  right  um  okay  so  let's  start
with  gu  space
l
over  here
um  so  we  call  it  the
motivation  uh  was  to  find
the
minimal
a  triangle
a  um  over  which  can  find  bit
AER  necessarily  bit  imprecise  up  the  r
take  Li  the  curve  there's  an  example  of
Li  the  curve  and  the  universal  one  would
be  the  the  Mod  Space  will  so  this  is  not
what  we  want  to  do  we  really  want
something  that  is  some  of  the  form  DM  Q
to  the
Z  in  some
antic  depending  on
some  um  and  so
the  or
following
that  there  should  be  this  element  Q  This
should  topologic
unit  in
the
one  and  the  second  the  r  that  I
mentioned  and  I  think  is  quite  natural
but  where  I  didn't  so  far  didn't  really
explain  and  maybe  I  will  get  to  the  to
the  what  the  end  of  today  why  it's
really  the  good  condition  is
that  uh  if  p  is  a
free  a
module  on  a  n
sequence  let  me
it
let  um  so  p  is  a  free  uh  light  condenser
being  D  on  a  n  sequence  um
then  1us  Q  *  the
operator  acting  on  P
Tender
Is  So  This  is  similar  to  how  we  Define
the  solid  analytic  R  structure  where  we
so  head  Q  to  one  instead  and  of  course
then  we  didn't  ask  it
one
um  uh  I  mean  this  is  some  weaker
condition
because  saying
that  for  no
sequence  a  n  and
a  we  stand
for  the  sum  of  a
n  right  so  if  you  think  about  what  the
inverse  of  this  would  be  then  be  one
plus  two  time  shift  plus  sare  time  shift
and  so  on  particular  Zer  coordinate
evaluate  such  we  can  form  some  kind  of
geometric  series  uh  in  Cube  uh  with  the
coefficients  from  a  no
sequence  and  we  certainly  I  mean  we
certainly  want  some  condition  here  right
because  if  you  didn't  ask  any  kind  of
completen  this
condition  then  we  would  just  work  with
all  a  modes  it  wouldn't  be  completed  in
any  sense  but  we  do  want  to  to  able  to
form  some  kind  of  infinite
sering  all  right  and  so  something  that  I
said  last  time  was  just  clear  but  maybe
I  should  really  say  a  few  more  words
about  this  um  is  that  there  is  certainly
an  initial  analytic  ring  uh
prop
and
um  you  may  or  may  not  actually  work  with
animators  or  infinity  or  whatever  Rings
uh  in  the  setup
um  the  proof  also  shows  that  there's
initial  animated  such  uh  later  on  we
will  I  will  actually  compute  the  initial
animated  one  and  show  that  it's  actually
con  Zer  and  so  then  agre  with  initial  I
can  reallying  the  previous  sense
was
let  me  introduce  an  notation  so  let  me
denote  the  free  free  condens  strings  on
a  topologically  ne  topologic  ne  poent
element  by  Z  ad  joint  Q  head  so  this  is
defined  to  be  Z  joint  Q
infinity  infinity  equal  to  zero
um  so  actually  this  is  just  uh  P  right
because  I  mean  it's  just  a  condensing
being  group  I'm  just  thinking  Z  join  any
Infinity  mod  Infinity  which  is  precisely
my  definition  of  p  but  then  I'm  ending
it  with  a  spring  structure  where  I'm
interpreting  this  n  here  is  small  the
powers  of  some  some
variable  but  I  want  to  keep  T  sort  of  a
module  separate  from  from  this  kind  of
ring
so  so  then  we
certainly  one  corresponds  to  a
map  from  this  Z  joint  you  head  and
then  and  let  me  call  this  thing
here  a  triangle
zero
so  we  want  one  then  we
want  a  triangle  to  to  come  with  such  an
M  right  because  it's  a  free  free
condensed  ring  uh  equipped  with  a
topolog  NE  poent  unit  this  a  fre  guy
with  topolog  NE  poent  element  and
this  okay  so  then  we  can
form  uh  stre  all  sequence  and  sender  it
up
to  this  guy
and  we  have  already  here  we  have  oneus
2  and  then  we  can  Define  certain  for
subcategory  um  of  all  the  modules
over  the
string  uh  like  we  want  this  to  become  a
nice  LM  so  in  other  words  whenever  we
map  this  to  something  over  a  module  over
analytic  ring  this  should  become  anism
so  we  can  simply  declare  this  to  be  all
those
modules  over  a  z
triangles
such  um  1-  2
*  acting  on  the  internal
H  um
from
p
uh  this  is  extremely  analogous  to  how  we
defined  solid  modules
right  the  ring  that  was  just  the  iners
and  then  we  ask  a  similar  condition  for
qal  to
one
and
uh  because  thisal  home  from  P  such
excellent  property
um  it's  immed  to  check  and  this  is  what
I  did  for  solid  modules  um
that  uh  the  subcategory  has  all  the
stability  properties  that  we  ask  for  for
antic  gr  so
then  um  the  fair  triangle  mod
a  is  some  what  we  call  nothing  called
last  time  with  preanalytic  bra  so  kind
long  analytic  bra  so  it  doesn't  satisfy
I  me  satisfy  all  the  properties  of
analytic  ring  except  for  the  property
that  the  ring  itself  is
complete  the  base  ring  is  really  just
this  uncompleted  uh  thing  so  far  um  and
we  can  see  something
about
sh  but  then  properties  one  and  two  they
p  pris  mean  that
analytic  over
analytic
and  so  now  we  just  use  uh  what  do  pro  on
Wednesday  that  uh  the  inclusion  from  of
analytic  ring  to  panal  rings  has  left
joint  so  you  can  always  complete  such  a
panal
and
so  yeah  s  also  explained  last  time  it's
actually  better  to  uh  when  you  have  a
general  suching  to  complete  it  in  the
sense  of  animat  antic  ranks  um  because  I
prior  the
completion  uh  of  of  the  unit  in  this  me
the  drive  completion  might  not  SRE  zero
and  then  the  better  object  to  consider
is  the  drive  completion  in  this  case  it
will  turn  out  to  be  the  case  that  this
drive  completion  actually  sits  in  degre
Z  so  there  will  not  be  a
difference  uh  so  that's  one  reason  that
uh  we  someh  switch  in  this  discussion
about  completion  of  analytic  structures
um  the  reason  for  the  final  stretch  of
the  talk  last  time  was  that  we  now  want
to
compute  uh  this  analytic  ring  structure
and  for  this  DUS  gave  some  general
recipe  for  computing  uh  this  comption
factor  in  some  some  cases
so  want  to
compute
um
um
okay  so  let  me  first  give  give  um  give
some  formula  uh  for  what  this  comp  looks
like  I  already  stated  it
uh  very  briefly  last  time  but  let  me
deliberate  okay
so  uh  the
following
grab  the  initial  analytic
animat
uming  is
um  but  it  turns  out
to  look  can  get  form
us
yeah  so  first  of  all  a  triangle  but  also
all  the  pre
modules  all  the  three  a  modules  on  live
Prof  sets  uh  they  sit  in  degre  zero
so  they  have  no  hom
Z  and  there  are  also  CA
separation  so  we're  not  describing
some  uh  secondly  this  a  TR
uh  it  is  actually  a  sub  ring  of  the  long
Ser
R
um  let  me
first  gives  a  formula  for  the  underlying
gr  even  a  point  so  these  are  those
sums  2  to  the
n
such  the  absolute  value  of  n  i  mean
certainly  there  are  zero  for  for  very
negative
n  very  long  series  um  but  then  when  then
go  large  they  should  have  most
po
and  so  this  is  tying  things  back  to
discuss  T  of  the
curve  has  the  property  that  when  you
write  down  the  corresponding
y  property  um  and  then  he's  completing
it  to  get  this  anal  but  that
a  a
question  oh  I'm  sorry  I  didn't  know  we
were  miked  up  yeah  I  was  I  answered  it  I
think  so  um  happy  to  answer  it  for
everyone  it's  yeah  well  the  question  was
was  whether  the  um  whether  you're
discussing  a  specific  analytic  ring  here
or  some  general  situation  I'm  I'm  just
I'm  just  here  I'm  describing  of  course
the
initial  one  coming  from  the  position  the
inial
animator
okay  so  the  uh  right  so  I'm  taking  the
underlying  ring  here  in  two  levels  right
if  you  have  an  analytic  ring  that  first
of  all  has  an  underlying  condensed  ring
it  this  has  an  underlying  ring  so  this
is  this  one  um  and  now  if  I  also  want  to
describe  the  condensed  structure  then
here's  a  way  to  do  that  uh  let  me  try  to
write  big  so  that  there  are  no  questions
about  indices  um
uh  take  a  union  over  K  and
NS
um  so  basically  this  is  K  is  giving  me
the  the  polinomial  grow  and
like  Pol  of  degree  K  and  is  giving  me
the  order  of  the  pole  of  of  my  the  wrong
suit  broadly  speaking  and  so  then  I'm
taking  a  product  over  all  n  which  are  at
least  minus
n  uh  of  those  in  which  lie  in  the
interval
from  n  plus  n  to  the  k  n  plus  n  to  the
k  um  right  integer  in  these
bounds
times  so  so  each  of  I  mean  this  is
always  a  finite  set  so  this  product  here
is  is  a  like  Prof  finite  set  and  then
taking  the  Union
this  describes  this  as  a  light  or  as  a
light  set  first  of  all  but
then  so  subset  of  here  and  stable  under
the  ring  scure
here
the  way
what  I'm  not  sure  why  start  call  me
switch  to  my  usual  name  s
um  so  this  describes  this  thing  and  then
he  also  describes  the  free
modules  uh  so
for  and  live  profile  said  written  as  an
limit  of  fin  s
i  um
Jo  s  has  a  sar
form  uh  to  to  write  it  um  right  so  first
of  all  again  you  can  think  of  this  as
uh  like  one  way  to  say  it  if  it's  Liv  in
like  also  on  Z  wrong  series  Q  you  can
take  to  guide  on  S  and  you  can  solidify
that  so  something  we  understand  let  me
recall  this  is
just  limit
over
power  or  in  other  words  these  are  some
kind  of  Z  long  series  two  value  measures
on
this
proficient
um  so  it  sits  inside  there  um  so  in
particular  I  only  just  describe  it  as
a  cond
set  because  the  group  structure  will  be
the  one  on  the  target
um  and  so
uh  One
Way  think  about  this  is
that  well  he  have  an  element  here  some
kind  of  measure  which  is  a  sum  overall  M
and  V  of  some  measures  m  m  *  Q  to
M  um  and  then  again  I  asked  I  put  a
gross  condition
on  the
mm  um  so  first  of  all  actually  say  the
mm  they  are  actually  just  some  of  D
measures  so  for  any  fixed  power  of  Q  you
cannot  really  take  infinite  sum  here
this  really  must  just  be  sum  of  D
measures
this  uh  but  even  more  than  that  you
asked  uh  that  some  these  are  not  too
large  nor
sub  poal
grow  and  I  guess  I  mean  this  is  probably
rather  describing  on  line  like  not  the
condense  structure  so  let  mebe  the  cond
structure
now  uh  it's  very  similar  to  the  above  so
it's  again  Union  over
case  and
ends  uh  play  the  same  row  as
above  uh  but  now  we  also  need  to  squeeze
in  this  inverse  limit
overall  I  to  get  something  for  for
S  and  then  now  we  just  do  the  obvious
things  we  take  M  greater  equal  to  minus
n  of  uh  the  measures  on  Z  join  s  i  which
are  of  Norm  at  most  n  plus  m  to  the
K
time  right  so  here  you're  taking  allow
to  take  some  some  differences  of  at  most
M  plus  m  to  the  K  basis
elements  so  if  he  did  this  when  s  was  I
was  just  in  one  element  would  be  down  to
do  that  thing  there  um  so  this  is  just
some  finite  set  uh  this  was  some  like
Prof  finite  set  if  it  takd  still  set
take
you  uh  Peter  I  wonder  if  might  be  a  bit
clearer  to  just  like  drop  the  n  greater
than  zero  only  like  only  take  the  term  n
equals  z  and  then  just  invert  Q  on  the
outside  yeah  but  then  I  would  have  to  do
this  funny  thing  where  I  we  have  to
replace  n  plus  M  here  by  something  like
n  plus  two  in  order  for  the  zero
coefficients  be  allow  oh  yeah  to  be
allowed  to  goow  yeah  okay  yeah  all  right
and  then  it  looks  artificial  again  so
yeah  or  you  could  you  could  have  a
constant  C  in  front  like  polinomial
growth  yeah  well
anyway  yeah  I
was  okay  this  is  correct  and  you  can
certainly  find  many  other  ways  to  rise
this
in  and  I  thought  if  I'm  already  taking  a
union  then  inverting  Q  at  the
end  seems  like  an
operation  many  I  don't
know
a  matter  of  taste  I
guess  all
right  is  it  okay
um  so  I  don't  want  to  give  the  the  full
argument  here  but  I  do  want  to  provide  a
sketch  I  want  to  at  least  explain  where
the  Pol  normal  growth  condition  suddenly
appears  because  this  maybe  AOS
surprising  thing  because  when  we  Define
this  G  Gus  analytic  R
structure  we  didn't  think  of  any
specific  grow  conditions  at  all  right  we
were  just  asking  that  one  two  time  soon
is  convertible  and  then  we  wonder  what
what  we're  getting
um  and  this  is  certainly  also  how  we
what  happened  to  us  when  we  thought
about  this  business  so  we  just  well
let's  write  down  this  condition  it  seems
natural  and  see  see  what  we  get  uh  this
was  was  what  came
out  uh  so  so  the  way  to  compute  is  to
use  the
uh  formula  from  the  end  of  St
leure  so  there  there  was  a  formula  how
you  compute  some  kind  of  completion  uh
in  cases  where  the  competance  condition
is  given  by  uh  asking  that  the  home
inter  oral  derive  come  from  some  algebra
is  Zer  so
note  that
um
um  this  condition  that  this
interal  uh  is  equal  to  the  point  than
um  so
that's  uh  let  me  call  it  c  um  so  this  is
T  the  a
triangle  mod
oneus  but  now  I  actually  want  to  I  do
want  to  think  about  p  as  a  ring  suddenly
icept  I  didn't  want  to  do  that  and  I
want  to  keep  my  notation
separate  but  at  this  point  I  do  want  to
remember  that  P  has  a  as  a  admits  a  ring
structure
um  and  so  then  this  is
actually
um  it's  a  free  guy  on  the  top  logic  new
element  that  X  head  before  and  then  aad
was  also  a  three  new  variable
and  then  okay  so  I  had  tover
q  and  then  one  minus  I  mean  shift
operator  is  multiplication  by
X  by
one  so  this  is  actually  a  commut
r  uh  so
keep  uh  so  it's  equiv  to  the  F  condition
uh  using  this  P  so  the  condition  is  that
the
r  over  a
triangle  uh  from
B  right  because  this  R  home  is  literally
just  the  cone  of  the  homs  and  and  and
the  homes  are  in  degree  zero  because  of
these  excellent  properties  of
P
and  also  uh  this  B  has  a  property  that
internal  home  from  B  commes  with  all
limits  uh  I  mean  same  property  for
p
so  uh  this  was  one  of  the  conditions
under  which  uh  Dustin  explain  the
formula  for
comption  I  mean  the  r
completion  the  three  analytic
RS
a  z  triang
a  is  given
by  um  taking  a  module  M  which  is  in  the
drive  category  of  a
triangle  gra  your  condens  a  z  triangle
modules  um  and  taking
this  theal
colum  of  the  following
diagram
so  m  m  to  the  internal
AR
from  I
willot  the
fiber
Co
um  can  take  the  r  home  from  I
into  and  because  oops  this  fiber  this
naturally  maps  to  one  so  Dually  on  our
Homestead  met  from  M  this  and  then  there
was  some  last  time  about  how  to  per  the
next  NS  but  I  think  doesn't  really
matter  all  that  one  um
go  to  find  the  Alum  twice  which  is  of
course  just  the  Alum  from  the  tender
product  all  right  so  that's  some  kind  of
formula  and  uh  as  D  mentioned  last  time
I  mean  there  are  some  situations  where
it's  quite  a  useful  formula  for  example
if  you're  in  a  situation  where  you're
just  algebraically  inverting  some
element  F  so  then  this  would  be  some
bring  a  mode  F  then  the  is  usual  way
inverting  F  where  this  Lally  just  always
the  same  module  and  multiplication  by  F
and  so  in  the  column  get
M1
um  but  you  could  also  try  to  use  this
formula  to  compute  solid  confusion  uh
but  this  turns  out  to  be  extremely
confusing
and  yeah  maybe  one  thing  that  will  what
happens  that  really  only  the  co  liit
that's  well  behave  both  in  the  Sol  case
and  this  case  here
um  and  in  the  col  I  claimed  that  you  get
something  nice  and  degre  zero  qu
separator  and  so  on  but  each  of  the
individual  terms  will  actually  be
spread  far  out  and  have  some  non  some
higher  homology  which  probably  not
separate  and  so  so  the  individual  terms
are  a  bit  hard  to
control  um  right
so  uh
uh  anyways
so  so  how  do  we  actually  go  about
Computing  this  so  This
I  uh  I  mean  it's  built  from  one  and  B
one  is  not  so  hard  of  course  and  B  was
just  built  from  P  basically  by  this
conone  so  to  to  understand  what's
Happening  Here  the  key  is  to  understand
what
uh
um  uh  one  key  is  certainly  to
compute  the  internal
home  uh  from
pnal  home  from  B  CH  from  a
triangle  a  triangle  but
also  they
trying
right  so  basically  want  to  evaluate  this
formula  on  the  M  which  are  some  three
guys  on  this  and  then  see  what  we
get
and  so  here's  the
LI
um  maybe  make  let  me  make  a  remark  that
uh  you  don't  actually  have  to  invert  Q
in  any  of  this
um  we  can  proceed
without  inverting
F  and  then  you  you  will  have  similar  the
completion  will  basically  be  the  same  as
have  to  drop  all  the  terms  which  have
negative
positive  um
uh  and  because  yeah  slightly  easier  to
to  write  down  the  formul  if  I  don't  have
to  invert  it  would  just  extra  operation
have  to  do  it  every
um  so
then  if  I  don't  bre  you  then  this  is
really  just  the  free  di  to  new  element
which  is  just  P  so  we  one  thing  we  have
to  understand  is  some  what  is  internal
home  from  T  to  P  where  we  think  of  this
here
say  Z  joint  X  and  just
joint  and  uh  there's  a  formula  for
this  so
uh  so  this  these  three  condensed  to  be
groups  on  some  Pro  Set  they  always  given
as  this  Union  over  all  ends  of  like
things  which  have  bounded  Norm  end  and
so  this  will  some  pull  through  this
internal  home  and  so  this  will  also  be  a
union  Over  N  bigger  than
zero  where  now  I'm  somewh  taking  the
part  of  this  algebra  here  where  some  of
Norm  at  most  n  so  at  most
n  uh  terms  are  allowed  to
appear  but  then  when  I  do  that  uh  the
new  so  now  I'm  allowed  to  do  a
overall  n  which  is  like  the  power  of  Q
dividing
by  and  so  then  that's  so  you  take  V
joint
Q  not  Q  to  the
N  but  then  inside  this  you  take  sums  of
at  most  n  basis
elements  you  take  sums  *  Q  to  the  AI  *  Q
to  the  I  where  the  sum  of  the  absolute
values  of  the  AI  said  most
then  and  then  you  take  polinomial  in
X
um
which  not  sure  I  should  call  x  x  dual
but  never  mind
call
um  uh  but  it's  also  just  a  form  a
variable  so  this  is  the  part  of  poal
algebra  where  each  coefficient  satisfies
expound
okay  and  so  there's  a  similar
formula  uh
for  internal  home  from  P  which  again
think  of  joint
except
towards
uh  the  ad  joint  you
have
adjin
some  light  profile
sets  and  and  uh  it's  just  the  same  sort
of  we  take  Union  of  all  bigger  than
zero  and  take  an  invers  limit  over  all
NS  and  then  you  take  V  joint
Q  to  the
N  Jo  as
I  but  inside  there  again  this  part  that
some  of  at  most  and  some  are  difference
of  at  most  end  basis
elements
maybe  I  should  put  some
operates
um  and  then  I  take  a  pols  in  in  the  Dual
variable
um  uh  right  for  each  term  settis  is
done  all  right  so  Peter  just  a  point  of
information  about  uh  the  way  things  look
on  the  screen  here  so  if  you're  close  to
the  the  boundary  of  the  board  uh  like
far  right  of  the  right  hand  board  or  the
far  left  of  the  left  hand  board  it's
quite  a  shadow  and  it's  we  basically
don't  see
anything  it  said  still  okay  also  to  far
in  um  that's  better  that's  readable  with
some  effort
yeah  anything  limit  over  the  uh  uh  yeah
sure  yes  of
course  what  is  the  Stop  and  the
exponents  of  the
X  sorry  X  maybe  I  should  do  the  EXP  y
it's  it's  just  a  different
okay  I  didn't  want  call  X  because  it's
some  of  the  Dual  of  X
or
now  but  course  maybe  I  use  the  head  to
have  some  specific  meaning  here  the  star
was  didn't  have  any  meaning  that
s
uh  so  in
particular  uh  one  way  to  uh  a
consequence  of  this  set  this  internal
home  from  P
top  is  actually  just  the  Subspace
of
uh  Z  joint
Y  and  it's  a  Subspace  with  a  certain
kind  of  uh
funny  uh  funny  condition
that
um  s  this  um  so  some  kind  of  let  me
notes  as  far  as
why  power  series  and  cubes  bounded
because  in  some  sense  you're  asking
that  yeah  there's  some  B  boundedness  uh
in  terms  of  the  coefficients
of  wise  and  similarly
the
from  from  te  to  this
um  there  also  a  subring  of  U  you  take  Z
joint  uh  Y  and  then  you  take  s  and  then
you
take  and  again  it's  a  substate  of  this
where  each  coefficient  of  Y  uh  satisfies
a  certain  boundness  property  so  again  I
will  denote  this  here  by  the
join  y  join  s  power  series  and
Q  um  Peter  there's  been  a  request  to
explain  again  what  the  less  than  or
equal  to  N  means  in  the  description  of  h
p
top
uh  for
finite
that's
I  I  didn't  use  I  foring  here  right
um  the  join  I  less  equal  to  n  a  certain
sub  space  of
fre  three  free  on  I  and  set  of  those  Su
over  I  of  some  a  itimes
I  so  that  is  sum  of  the  absolute  vales
of
the  is  it
most
all  right  so  if  you  take  this  formula
for  the  internal  home  from  p  and  just  do
these  formal  manipulations  that  you
express  I  in  terms  of  the  other  thing
sorry  sped  way  too
high  um  if  you  just  do  this
manipulations
uh  then  you  for  example  see  that  the
Rome  uh  from  sorry  uh  I  said  I  didn't
invert  didn't  invert  Q  so  let  me
explicitly  say  just  for  Z  Jo  Q  head  and
okay  so  I  can  also  be  defined
corresponding  version  inverting  you  have
um  uh  so  if  you  compute
that  then  this  is  a  two-term
complex  um  but  those  terms  are  see
computer  up  there  it  should  take  this  uh
this  power
series  over  the  one  which  has  some  bound
coefficients
and  previously  there  was  on  transition
that  was  1  -  x  *  Q  uh  but  now  Y  is  some
kind  of  dual  object  X  so  now  actually
transition  turns  out  to  be  just  Q  minus
or  y  q  minus
one  and  so  this  sits  in  degree  zero  and
this  sits  in  homological  degree
one
and
uh  if  you  do  the  similar  computation
with  more
variables  hey  uh  s  effector
here
then  you  get  a  similar  picture  but  just
the  K  of  the  variables  y  so  this  is
computed  um  by  complex  in  uh  y  variable
so  inste  what  take  is  you  take  Z  and
then  you  join  y1  up  to
YK  and  then  again  you  take  such  b  power
Series  in
Q  and  then  I  mean  some  of  this  was  the
derived  potion  by  just  one  variable  tus
Y  and  now  you  take  a  derived  potion  by  Q
minus  y1  up  to
Q  corresponds  to  Tak
fil
and  I  mean  you  can  do  the  same  with  with
a  with  a  sets  in  right  so  if  you  have  a
profile  sets  somewhere  then  you  just
plug  the  profile  sets  here  in  the  middle
and  then  get  the  same
for
all  right  s  so  so  now  we  have  a  somewhat
concrete  formula  for  these  things  right
so  if  you  want  to  compute  the  underlying
ring  we  literally  just  have  to
understand  the  co  liit  of  all  K  of  these
things
so
thus  the  ring  we're  interested  in  is
just  the  co  liit  over
all  of  this
same  take  y  want  to  YK  if  you  found
it  then  you  take  a  d  quent  and  here
turns  out  to  be  a  case  where  it's
really  very  important  that  each
individual  level  to  Tak  the
D  full  CL  on  because  there  will  be  high
homology
then  here  plug
s  what  are  the  transition  Maps  a
transition  sorry  yeah  I  should  have  said
so  this  is  similar  question  to  what  the
transition  mat  in  call  that  D  wrote  down
last  time  um  you  just  I  mean  each  of
these  is  a  subing  of  the  next  one  where
you  just  don't  have  the  new  variable
it's  not  a  sorry  all  these  things  are
actually  not  range  because  if  you
multiply  two  things
uh
what  confusion  but  um  let  me  not  try  to
say  whether  they  r  or  not  but  uh
certainly  there  sub
modules
all  right  um  so  so  far  there  was  no
apparent  polom  grow  condition  at  all
there  was  just  some  kind  of  boundness
that  appeared  for  more  transparent
reasons  but
now  uh  let's
analyze  uh  like  the  of  this  or  rather
separate
Co
so  I  mean  this  for  Simplicity  you  can  do
the  full  analysis  but
uh  it  takes  a  lot  of  concentration  so  I
think  there's  already  enough  interesting
stuff  going  on  without  these  addictional
technicalities
um
so  right  so  so  what  is  the  H  zero  so  you
really  just  take  this  ring  and  then  what
out  by  Q  minus  y1  of  the  Q  minus  y  k  or
if  it  takes  a  separate  equion  really  by
the  the  closure  of  all  these  things  yeah
um  so  in  other  words  here  I'm  are
formally  allowed  to  uh  replace  any
recurrence  of  any  of  these  variables  by
I  that  it's  really  just  set  into  a  q
right
um  so  this
uh
there  a  question
com  what  is  the
image
of
uh  this
thing  uh  mapping  towards  power
Sol  where  all  the
Yi  so  let  me  again  recall  what  this  was
so  this  was  the  union  over  all  end
of  of  the
limit  over
M
of  the  part  of  this  thing  which  was  of
Norm  at  most
n  so  this  wasn't  literally  written  as  a
finite  free  module  but  I  mean  you  take
the  basis  by  giving  my  1  Q  of  to  Q  to
one  uh  and  then  you  pend  the  variables
by
one  but  in  particular  it  will  actually
be  sufficient  to  analyze  the  part  where
all  of  these  are  say  just  equal  to  one
uh  Z  or  one  this  one  um  so  here  we  just
allowed  to  take  some  kind  of  bounded  sum
but  then  we  have  all  these  y  ones
through  y  k  here  at  the  posal  but  for
each  of  them  individually  I  can
take  I  can  take
a  I  mean  just  each  coefficient
individually  is  bounded  so  for  example
if  I  like  for  each  end  we  have  to
analyze  what  the  image
is  but  then  which  which  multiples  of  Q
which  like  integer  times  Q  can  I
achieve  well  this  integer  time  Q  like
for  each  occurrence  of  Q  I  can  decide
whether  I  want  to  keep  it  a  q  or  whether
I  make  it  a  y12
YK  but  for  each  y1  y  k  i  can  Som  use  it
most  end
times  but  then  in  total  like  The  K  uh  I
can  I  can  get  this  variable  q  k  n  k  *  n
times  because  yeah
there
um
uh  and  then  actually  maybe  the  nend  is
rather  red  pairing  so  let's  ignore  the  N
so  if  n  was  equal  to  one  uh  then  we  can
some  get  a  coefficient  up  to  K  here  for
this  Q  because  I  can  split  each
occurrence  of  Q  between  the  different
variables  and  then  if  I  look  at  q
squ  then  I  have  all  the  monomers  of
degree  2  hereos  in  the  y1  YK  so  there
like  k/2  mon  have  at  disposal  each  one  I
can  use  like  at  most  n
times  so  I  get  n  *  K  two  and  then  in
degree  some  big  m  i  sorry  it's  K  plus
one  I  guess  I  have  K  plus  M  over  choices
for  for  these  monomials  here
uh
so
the
coefficient  of  Q  to  the
m  can  lie
in  likeus  n
*  uh  n  plus  k
k  actually
same  and  maybe  I  I  didn't  uh
okay  I  would  have  to  be  slightly  more
careful  here  the  precise  noise  but  I
mean  basically  something  like  that  um
and  so  the  question  is  how  does  this
grow  how  does  this  grow  in  m  and  in  M
this  is  precisely  polom  of  degree  K
right  and  then  this  just  the  polom
function  of
the
yeah  so  whenever  I  have  something  that
grows  in  where  the  coefficients  grow  in
m  just  like  poom  then  because  yeah  I
have  all  these  monomials  here  at
disposal  I  can  somewh  split  up  all  the
individual  summons
uh  to  make  them  bounded  in
here  and  and  there's  c  a  lot  of  choices
about  how  you  go  about  doing  this  and
when  you  want  to  prove  that  actually
this  uh  this  complex  end  up  ends  up
being  in  degree  zero  you  Som  have  to
show  that  a  different  way  of  arranging
the  terms  here  uh  is
actually  uh  like  differs  from  the  other
one  uh  by  uh  something  bounded  some
bounded  some  in  the  next  C  of  the
complex  and  this  will  actually  not  be
true  for  giv  K  but  when  you  allow  your
to  increase  your  k  a  little  bit  then  you
can  do
it
Peter  there's  a  question  in  the  chat
there's  a  question  in  the  chat  yeah
which  is  uh  or  you  want  to  read  it  or  uh
oh  yeah  so  there  was  a  question  about
this  Z  join  y  join  s  whether  I  can
interchange  the  order  and  indeed  you  can
interchange  the
order
so  the  kind  of  argument  you  have  to  do
here  to  see  that  this  is  well  behave
this  a  little  reminiscent  of  the
arguments  you  have  to  prove  that  liquid
stres  well  behav  but  it's  several  orders
of  magnitude  easier
right
so  it's  again  a  situation  where  you  have
uh  some  kind  of  system  of  complexes
where  each  term  has  some  kind  of  norms
and  then  you  have  to  like  see  that  if
you  have  something  with  differential
zero  you  can  bounded  pre  image  but  here
it's  really  you  can  really  just  do
it
I  takes  some  concentration  but  it's  not
all
all  right  all  right  so  um  this  is  all  I
wanted  to  to  say  about  uh  the  proof  of
this  box
ination
um  uh  now  they  I  want  to  talk  just
briefly  about  Gaz  G  real  Vector
spaces
so  let
me  condens  or
light  we  started  emitting  the  light  at  a
couple  of  times  so  throughout  this
course  we're  always  working  light
filling  and  it  might  not  always  be
mentioned
yes  if  I  take
the  internal
home  from  like  P  was  always  ining  the
free  with
integers
here
and  then  I  choose  one  top  poent  unit  in
the  real
say  so  now  you  might  actually  Wonder  uh
doesn't  matter  that  it  shows  a  half  here
and  it  doesn't  so  I  mean  so  in  this  Ser
of  liquid  analytic  of  this  liquid  drink
structures  on  the  reals  there  was  this
extra  parameter  that  you  have  to  choose
your  how  non  convex  you  allow  your
vector  spaces  to  be  um  here  it  seems  you
also  have  to  make  a  choice  but  actually
it  doesn't  matter  the  condition  is
independent
of  number  like  a
half  you  could  Al  take  a
negative  any
top
um  actually
why  so  it's  easy  to  see  and  uh
intuitively  believable  that  if  he  makes
this  number  smaller  then  the
conditioning  becomes
weaker  uh  because  here  some  of  the
series  you're  trying  to  to  to  sum  there
like  even  faster  Decay  but  on  the  other
hand  if  you're  trying  to  sum  a  geometric
Series  where  the  coefficients  are  like
one  over  two  to  the  end  then  you  can
always  um  the  finite  sum  where  you  Su
the  first  few  terms  and
then  have  telescoped  uh  not  really
telescope  but
um
there's  another  question  in  the  chat  I  I
actually  missed  it  uh  so  I  can't  read  it
um  let  me  see  if  I  can  read  by  the  way
general  question  nonl  condens  sets  will
not  be  used  anymore  in  the  Ser
originally  right  so  there  was  a  question
uh  like  originally  extreme  disconnected
set  play  quite  of  a  crucial  role  how  set
up  the  Ser  because  gave  us  enough
projective  objects  and  it  is  really
quite  convenient  to  have
um  so  for  much  of  what  we're  doing  we're
kind  of  forgetting  about  the  full
category  of  condensed  sets  but  on  the
other  hand  for  some  arguments  actually
is  still  quite  convenient  to  have  the
ambient  framework  of  all  condensed  sets
because  I  mean  the  light  ones  they  still
embed  into  all  condensed  sets  and  for
some  so  if  you  want  to  check  that
something  is  an  isomorphism  you  can
still  do  that  by  checking  the  values  and
extremely  disconnected  and  this  is  still
convenient  for  a  few  arguments  um
so
yeah  yeah  for  and  the  other  setting  is
also  nice  but  like  for  for  the  specific
application  analytic  geometry  we  found
it  more  convenient  to  make  this
restricture  because  there  are  a  couple
of  things  in  particular  this  internal
projectivity  of  P  that  we're  using  all
over  the  place  that's  only  from  this
uh  maybe  related  questions  like  can  you
Al  like  the  Ser  of  liquid  real  Vector
space  that  also  worked  not  just  in  liing
but  the  full  setting  you  can  ask  the
same  question  about  the  gasius  anding
structure  whether  you  can  also  Define
that  on  all  condens  modules  I  think  you
can  but  the  argument  we  gave  here  does
not  work  because  we  use  internal
projectivity  of  P  so  you  would  have  to
argue  slightly  more  carefully  to  see
that  there  is  a  gr  structure  of  G  real
Vector  spaces  in  the  full  condens
setting
um
uh  generally  is
onlyus  yeah  and  this  is  basically  a
telescopic
argument
all  right  I  mean  if  if  she  wanted  some
the
you  can  always  rewrite  like  at  a  finite
sum  and  then  new  sum  where  you  still
have  a  no  sequence  of  coefficients  and
now  powers  of  Q  to  the  end
here  uh  all  right  so  you  you
have  let's  say  um
and  so  anything  that's  P  liquid  for  any
P  or  Alpha  liquid  for  any  Alpha  uh  is
always  cases  so  this  is  extremely
General  class  of  condensed  Vector  spaces
uh  but  it's  still  a  workable  class  for
many
purposes  um  and
so  uh  you  can  now  wonder  what  are  the
three  uh  guesses  real  Vector
spaces  and
here
so  first  of  all  so  here  here's  a
proposition  what  does  it  have  to  do  uh
with  uh  what  we  had  previously
so  our  guess  which  notes  the  analytic  R
uh  corresponding  to  gu  is  Vector
spaces  so  again  it's  clear  that  this
defines  structure  because  of  this  good
properties  of  P  um  this  just  gotten  by
taking  this  one  we  had
is
makes  a  completed
one  and  the  g  mod  over
this  and  then  in  some  moding  out  by
oneus  2
*  setting  Q  equal  to  a
half  you  take  this  one  you  complete  that
antic  as  we  did  and  then  you  set  Q
so  it's  easy  to  see  that  when  you  do
this  to  this  uh
ring
uh  Point  number  gross  condition  that
you're  getting  you  actually  get  the  real
numb  H  you  don't
all  you  don't
all  yeah  no  I  mean  like  if  if  you  take
this  completed  ring  and  one  by  oneus  PQ
you  really  get  the
real
I  so  these  are  like  functions  on  like
like  the  real  part  of  this  thing  is  like
a  function  open  unit  the  real  numbers
and  then  if  you  set  P  equal  to  a  half
there  like  this  one  dimensional  thing
you  just  get  real
numbers  in  particular  um  this  tells  you
what  the  the  free  gas  miners  are  you  can
compute  them  the  free  G  your  vector
Spaces  by  taking  the  fre  guys  here  then
taking  the
poent  uh  but  now  the  description  becomes
a  bit  confusing  because  when  we  wrote
down  the  gross  conditions  that  we  allow
here  it  was  some  kind  of  Pol  normal  grow
condition  in  q  but  now  Q  has  become  a
constant  so  it's  not  immediately  C  clear
how  you  phrase  these  Rose  conditions  uh
directly  on  the  real  numbers  uh  but  you
can  do  it  and  here's  the
description  so  the  free  guess  is  uh  real
vectors  space  on  some  light
profs  uh  has  the  following  crazy
description  we  take  a  union  over  again
some  case  which  corresponding  to  this
poal  draws  that  we  had  previously  and
now  there  is  a
constant  then  I  take  a  limit  over  I
and  then  I  take  a  certain  bounded
subset  um  in  the  three  real  Vector  space
on  the  fin  set  s
i  uh  which  satisfy  some  condition  on  how
large  I  all  to  be  and  the  condition  is
that  if  you
sum  um  Su  the  size  and  measure  it  or
their  size  measure  it  in  terms  of
certain  functions  FK  that  I  will
introduce  in  a  second  there  at  most
C
so  formally  this  is  very  similar  to  what
the  three  liquid  Vector  space  look  like
and  there  the  the  Norms  you  were  taking
there  were  like  you  were  raising  to  some
power  also  or  maybe
Al
um  generally  the  functions  that  you
would  like  to  put  there  um  they  should
be  increasing  of  course  because  larger
larger  exes  should  be  calized  more  um  on
the  other  hand  you  want  the  transition
apps  to  be  well  defined  and  for  this  you
need  the  functions  to  be  uh  concave
turns  out  um
so  well  this  could  be
reasonable  um  these  to  FK  they  should
yeah  they  should  measure  how  lot  to  real
number  is  in  terms  of
some  let  it  some  real  number  greater
equal  to  zero
um  uh  I  mean  they  should  be
increasing  and
conct
so  basically  you  need  that  FK  of  X  Plus
y  should  be  at  most  FK  of
X  Plus  FK  of
Y  she  wants  a  transition  Ms  there  to  be
well  defined  because  the  transition  maps
are  somehow  summing  some  of  these  X's  um
and  this  is  Ed  by
theity  uh
but  on  the  other  hand  you  don't  really
care  what  your  function  is  for  large
values  of  of  of  X  because  uh  really  what
you're  trying  to  think  out  is  just  which
infinite  sums  are  allowed  and  of  course
in  these  infinite  sums  almost  all  the
terms  should  be  very  very  small  so  when
I  choose  such  a  function  FK  really  only
the  only  thing  that  matters  is  the
behavior  near
zero
okay  this  is  a  Prelude  to  saying  that  uh
so
FK  is
any  uh  increase  in  comp
functions
such  that  for  small
X  um  it's  given  by  the  following
it's  uh  the  absolute  value  of  log  X  to
Theus
K  so  I'm  saying  it's  this  way  because  I
mean  in  principle  I  just  like  to  take
this  function  and  in  the  neighborhood  of
zero  that  is  a  nice  increasing  concave
function  but  that  at  some  point  uh  it's
becomes  convex  goes  to  infinity  and  then
does  some  nonsense  because  like  if  you
said  x  equal  to  one  log  of  one  is  zero
one  to  the  minus  K  is  some  nonsense  so
um  so  I  can  write  down  this  function  for
all  X  but  I  can  do  it  for  small  X  and
then  at  some  point  I  just  aritar  extend
and  it  doesn't  do
me
what
all  right  so  this  is  maybe  a  little  bit
hard  to  just  so  let  me  let  me  give  a
sort  of
diagram  uh  so  in  the  liquid  in  the
liquid
story  uh  there  this
function  have  like  some  the  of  x  to  the
beta  where  beta  is  some  number  between  Z
and
one  so  there  some  function  so  this  is
always  increasing  in  concave  so  I  can
just  always  use  it  so  some  function  like
that
um  and
then  the  locus  where  the  sum  of  the  X  to
the
beta  um  in  the  two  variable  cases  will
be  some  some  kind  of  lus  like
that  some  sling  on
convexing  uh  in  the  G
case
um  it's  some  function  that's  extremely
steeply  ascending
near  near  uh  zero  so  even  if  it's  just  a
tiny  number  it's  often  treated  as  rather
large  by  this  by  the  function  FK  then
then  the  actual
function  that  I  wrote  there  would  make  a
turn  and  then  go  up  and  okay  instead  I
just
cut  I  can  make  it  linear  if  I  want  to
right  need  not  be
strictly
um  and  then  if  I  look  at  the  set  where
the  see  this  will  similarly  be  something
that  extremely  closely
tied  uh  to  to  the  coordinate  X
so  you're  allowed  to  take  certain
infinite  sums  but  basically  you  have  to
ensure  that  the  coefficients  have
exponentially
P  but  not  quite  that  actually
so
from  the  liquid
the  we  can
form
some  of  some  and
and  there  exist  some  detail  some
prespecified
Al  such  that  the
sum  of  the  X  the
BET  finite  so
there  in  principle  we  like  to  just  ask
usual  stability  but  as  I  said  this
doesn't  really  work  but  when
you  slightly  restrict  the  class  of  sums
for  which  they  ask  us  that  they  are
always  like  part  of  the  structure  ofid
re  Vector  space  to  be  the  speed  want  and
this
works  uh  in  the  G  the  you  have  much  more
spring
what's  uh  so  yeah  the  alha  liquid  Series
so  the  liquid  Ser  depends  on  exra
parameter  uh  Cas
Theory  can  only  form  those
sums  or  theic  say  but  the
some  ni  um  and  this  actually  two  one  to
say
xn  up  to
reordering  if  you  sum  in  the  order
starting  with  the  largest  ones  and
descending  uh  then  the
xn  F
uh  one/
xn  sorry  the  should  have  prob  the
exponential  growth  by  which  I  mean  the
following
so  you  can  form  such  a  su  only  if  exist
someon  greater  than  equals  to  zero  and
maybe  some  constant  C  greater  than  zero
such  that  the  abute  value  of  xn  is  at
most  c
times  a  half  to  the  end  to  the
absolute  right  so  if  I  didn't  put  the
EPS  on  here  then  this  would  be  some  kind
of  exponential  decay
uh  uh  it's  enough  to  have  some
fractional  power  of  n  here  in  the
exponent  um  and  if  they  satisfy  some
Decay  condition  like  this  then  then
you're  allow  to  S
guesses  so  why  does  this  expression
appear  here  well  if  you  take  a  half  to
the  end  to  the  Epsilon  and  then  apply
this  kind  of  procedure  here  then  the  log
turns  this  into  to  the  Epsilon  and  then
to  the  minus  K  you  get  one  over  n  to  the
Epsilon  *  K  uh  and  this  becomes  summable
when  I  don't  know  Epsilon  *  K  becomes
very  all  right  so  some  rather  crazy  type
of  grow  conditions  from  get  was  real
numbers  but  it's  still  a  workable  thing
because
um  like  do  agation  CA  on  complex
geometry
and  uh  at  least  all  the  all  the  stuff  in
complex  geometry  would  work  as  well  as
Gus  Vector  spaces  because  the  only  thing
that  we  really  needed  that  this  algebras
of  conver  phoric
functions
um  where  some  kind  of  thing  in  the
theory  and  some  nuclear  uh  and  this  is
still  true  uh  in  this
story  because  the  transition  Maps  when
you
form  like  when  you  restrict  all
functions  on  one  this  all  function  on
small  this  then  you're  precisely  if  you
think  in  terms  of  their  power  series
expansions  you're  rescaling  all  the
power  series  coefficients  by  but  fixed
call  of  the  number  less  than  one  so
actually  the  transition  map  they  have
exponential
decay
so  yeah  so  I  think  the
guest  uh  story  would  be  just  as  good  for
this  discussion  of  complex
geometry  all  right
um  yeah
so  yeah  so  the  claim  is  that  you  can  do
this  computation  of  the  three  gas
elector  spaces  uh  forly  from  the  one  I
did  over  this
uhing  in  the  variable
Q  takes  a  little  bit  of  unraveling  but
it's  not  it's  not  all  that
hard  all  right  so  uh  finally  let  me  try
to  do  some
geometry
yes  so  the  goal  is  to
define
the  GU  is
uh  okay  let  me  again
recorded
and  so  the  idea  is  something  that  we
already  used  a  couple  of
times  we  have  this
fixed  to
and  like  an  add  B  some  use  that  when  a
fixed  to  unit  you  can  use  this  to
normalize  absolute  values  uh  so  that
some  of  when  you  want  to  understand  how
large  some  other  function  is  uh  you  can
see  some  how  large  this  compared  to  to  Q
um  uh  so  we  can  use
Q
measure  the  size  of  any  other
function  so  basically  you're  you're
wondering
like  how  do  powers  of  the  absolute  value
of  uh  F  from  comp  to  PO
of
um  and  uh  one  way  to  organize  this
information  um  is  in  the  following
place
and  now  I  will  start  to  use  this  formal
of  analytics  STX  uh  that  this  whole
course  like  the  main  aim  of  introducing
um  and  this  language  is  really
convenient  to  phrase  before  play  because
I  could  give  a  more  downt  version  of
this  but  it  would  be  much  more  confusing
um
and  I  think  it's  a  really  neque  way  of
packaging  the  information  I  will  maybe
explain  a  little  bit  about  those  so  the
exist  the
morm
T
um
from  let's  say  first  from  the  f  l  over  a
um  and  on  the  fine  line  you  have  the
coordinate
function  in  particular  like  when  you
want  to  do  something  for  any  function
you  can  something  do  for  the  universal
function  which  is  the  function  t  on  an
A1  right  so  when  you  have  any  function
on  the  a  then  this  gives  you  M
from  the  spectrum  of  A2  with  A1  let  your
function  T  to  your  given  one  so  to
define  something  for  any  function  is
enough  to  do  it  for  an  A1  Anders  Cas  U
and  so  then  there  is  some  kind  of
absolute  value  of  T
function
measuring  measuring  against  the  ABS
value  of  Q  uh  that  goes  to  well  in  this
case  it  can  actually  be
infinite
uh
because  some  could  be  some  kind  of
unrounded  function  maybe  a  better  way  to
think  about  that  you  have  one  on  the
projected
line  and  if  you  have  the
point  this  homogeneous  coordinates  he
want  to2  then  this  goes
abute
implicit  here  is  that  something  is  down
to  earth  as  this  closed  interval  uh  will
have  a  certain  very  crazy  Incarnation
and  an  antic  St  in  all  setup  and  uh  this
is  required  to  make  S  of  this  mths
um  so  simpit  here  again  is  again  so  if
you  really  want  to  go  to  zero  Infinity
here
uh
um  I  again  need  to  normalize  the
absolute  value  of  Q  and  so  I  what  I
always  do  is  to
let  but  I'm  me  in  my  last  lecture  it
kind  of  suddenly  mattered  at  one  point
uh  what  I  show  there  but  here  it  doesn't
matter  again  and  always  just
resale  but  let's  assume  we're  in  a
formalism  where  something  like  this
makes  sense  where  we  have  some  kind  of
way  measuring  the  absolute  value  on  this
same  something  there  so  then  we
can
Define  what  I  call  the  analytic
Chay  has  a
pre-image  of  the  open  part  so  this  is
intuitively
speaking
the  union  over  all  n  of  the  locus  where
the  absolute  value  of  of  T  is  bounded
between  some
Powers  some  powers  of  the  absolute  value
of
G
uh
right  and  this  was  something
that  already  entered  at  the  beginning  of
the  last  my  last  lecture  um  that  this  is
what  you  just  start  with  if  you  want  to
define  the  T  with  the
curveent  and  then  find
that
um  and  then  you  still  have  acting  on
this  multiplication  by
two
and  uh  this  map  here  is  actually  some
kind  of  multiplicative  map  so  if  you
mute  value  of  product  the  prodct  the
absolute
values  where  this
if  it  so  happens  that  one  is  zero  and
the  other  is  infinity  then  there's  no
claim  being
um  but  in  particular  on  this  Locus  I  me
you
have  this  analytic
here  to  Z
Infinity  you  have  multiplication  by  Q
here  this  corresponds  to  by
multiplication  by  a  half
here  and  this  map  here  is
actually  this  is  actually  some  kind  of
proper
uh
that
because  P1  is  proper  zero  Infinity  is
proper  so  the  map  is
proper  um  proper  map  to  here  um  and  then
if  you  pass  the
equion  um  then  you
have  GM
Z  I  mean  this  must  be  totally
discontinuous  I  mean  free  and  totally
discontinuous  is  actually  because  it's
it  is  here  right  so  this  then  is  just
F  over  Z
INF  half
Z
which  copy  of
circle
um  so  so  the  m
is  it's  locally  as  is  proper  but  now  S1
is  again
proper
um  and  so  this  means  that  this  one  is
proper  but  of  course  it's  also  locally
asmic  to  this  one  which  is  an  open
subset  of  P1  and  P1  ought  to  be  smooth
uh  so  it's  and  a  curve  so  this  must  be  a
proper  smooth
curve  and  this  is  the  D
right  and  so  uh  this  is  how  we  want  to
argue  how  we  think
construct  so  take  a  look  to
curve  all
right
questions  this  Z  Infinity  is  also  def
over  this
a  you  can  Bas  change  it  to  a  you  don't
have  to  write
okay  and  this  always
that's  you
can
from
what  is  the
usual  there  will  be  a  fun  from  like
condens  s  towards  analytic
steps  and  so  zero  Infinity  I  consider  as
a  life  set  and  this  will  have  an
Incarnation  as  analytic
St  this  will  actually  mean  that  the
theory  of  analytic  stes  will  be  related
to  the  condens  story  in  two
ways  one  because  like  the  analytic  Rings
themselves  are  found  some  on  on  condens
things  uh  which  is  the  role  that  condens
have  played  so  far  but  suddenly  there
will  be  another  role  that  the  condens
Stu  plays  because  yeah  condens  will  also
have  an  ination  in  the  worlds  actually
two  kind  somewhat  AAL  condens
Direction  because  the  way  this  this
thing  will  be  realize  is  antic
St  it  actually  just  use  the  sced  rins  it
will  not  use
any  interesting  condensed  structure  on
the  ring
level
any  thanks  Peter
yeah
\end{unfinished}